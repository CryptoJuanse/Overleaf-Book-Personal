\part*{Book Nine: 1812}

% % % % % % % % % % % % % % % % % % % % % % % % % % % % % % % % %
% % % % % % % % % % % % % % % % % % % % % % % % % % % % % % % % %
% % % % % % % % % % % % % % % % % % % % % % % % % % % % % % % % %
% % % % % % % % % % % % % % % % % % % % % % % % % % % % % % % % %
% % % % % % % % % % % % % % % % % % % % % % % % % % % % % % % % %
% % % % % % % % % % % % % % % % % % % % % % % % % % % % % % % % %
% % % % % % % % % % % % % % % % % % % % % % % % % % % % % % % % %
% % % % % % % % % % % % % % % % % % % % % % % % % % % % % % % % %
% % % % % % % % % % % % % % % % % % % % % % % % % % % % % % % % %
% % % % % % % % % % % % % % % % % % % % % % % % % % % % % % % % %
% % % % % % % % % % % % % % % % % % % % % % % % % % % % % % % % %
% % % % % % % % % % % % % % % % % % % % % % % % % % % % % %

\chapter*{Chapter I}
\ifaudio
\marginpar{
\href{http://ia802702.us.archive.org/23/items/war_and_peace_09_0811_librivox/war_and_peace_09_01_tolstoy_64kb.mp3}{Audio}} 
\fi

\initial{F}{rom} the close of the year 1811 intensified arming and
concentrating of the forces of Western Europe began, and in 1812
these forces---millions of men, reckoning those transporting and
feeding the army---moved from the west eastwards to the Russian
frontier, toward which since 1811 Russian forces had been
similarly drawn. On the twelfth of June, 1812, the forces of
Western Europe crossed the Russian frontier and war began, that
is, an event took place opposed to human reason and to human
nature. Millions of men perpetrated against one another such
innumerable crimes, frauds, treacheries, thefts, forgeries,
issues of false money, burglaries, incendiarisms, and murders as
in whole centuries are not recorded in the annals of all the law
courts of the world, but which those who committed them did not
at the time regard as being crimes.

What produced this extraordinary occurrence? What were its
causes? The historians tell us with naive assurance that its
causes were the wrongs inflicted on the Duke of Oldenburg, the
nonobservance of the Continental System, the ambition of
Napoleon, the firmness of Alexander, the mistakes of the
diplomatists, and so on.

Consequently, it would only have been necessary for Metternich,
Rumyantsev, or Talleyrand, between a levee and an evening party,
to have taken proper pains and written a more adroit note, or for
Napoleon to have written to Alexander: ``My respected Brother, I
consent to restore the duchy to the Duke of Oldenburg''---and
there would have been no war.

We can understand that the matter seemed like that to
contemporaries. It naturally seemed to Napoleon that the war was
caused by England's intrigues (as in fact he said on the island
of St. Helena). It naturally seemed to members of the English
Parliament that the cause of the war was Napoleon's ambition; to
the Duke of Oldenburg, that the cause of the war was the violence
done to him; to businessmen that the cause of the war was the
Continental System which was ruining Europe; to the generals and
old soldiers that the chief reason for the war was the necessity
of giving them employment; to the legitimists of that day that it
was the need of re-establishing les bons principes, and to the
diplomatists of that time that it all resulted from the fact that
the alliance between Russia and Austria in 1809 had not been
sufficiently well concealed from Napoleon, and from the awkward
wording of Memorandum No. 178. It is natural that these and a
countless and infinite quantity of other reasons, the number
depending on the endless diversity of points of view, presented
themselves to the men of that day; but to us, to posterity who
view the thing that happened in all its magnitude and perceive
its plain and terrible meaning, these causes seem insufficient.
To us it is incomprehensible that millions of Christian men
killed and tortured each other either because Napoleon was
ambitious or Alexander was firm, or because England's policy was
astute or the Duke of Oldenburg wronged. We cannot grasp what
connection such circumstances have with the actual fact of
slaughter and violence: why because the Duke was wronged,
thousands of men from the other side of Europe killed and ruined
the people of Smolensk and Moscow and were killed by them.

To us, their descendants, who are not historians and are not
carried away by the process of research and can therefore regard
the event with unclouded common sense, an incalculable number of
causes present themselves. The deeper we delve in search of these
causes the more of them we find; and each separate cause or whole
series of causes appears to us equally valid in itself and
equally false by its insignificance compared to the magnitude of
the events, and by its impotence---apart from the cooperation of
all the other coincident causes---to occasion the event. To us,
the wish or objection of this or that French corporal to serve a
second term appears as much a cause as Napoleon's refusal to
withdraw his troops beyond the Vistula and to restore the duchy
of Oldenburg; for had he not wished to serve, and had a second, a
third, and a thousandth corporal and private also refused, there
would have been so many less men in Napoleon's army and the war
could not have occurred.

Had Napoleon not taken offense at the demand that he should
withdraw beyond the Vistula, and not ordered his troops to
advance, there would have been no war; but had all his sergeants
objected to serving a second term then also there could have been
no war. Nor could there have been a war had there been no English
intrigues and no Duke of Oldenburg, and had Alexander not felt
insulted, and had there not been an autocratic government in
Russia, or a Revolution in France and a subsequent dictatorship
and Empire, or all the things that produced the French
Revolution, and so on. Without each of these causes nothing could
have happened. So all these causes---myriads of
causes---coincided to bring it about. And so there was no one
cause for that occurrence, but it had to occur because it had
to. Millions of men, renouncing their human feelings and reason,
had to go from west to east to slay their fellows, just as some
centuries previously hordes of men had come from the east to the
west, slaying their fellows.

The actions of Napoleon and Alexander, on whose words the event
seemed to hang, were as little voluntary as the actions of any
soldier who was drawn into the campaign by lot or by
conscription. This could not be otherwise, for in order that the
will of Napoleon and Alexander (on whom the event seemed to
depend) should be carried out, the concurrence of innumerable
circumstances was needed without any one of which the event could
not have taken place. It was necessary that millions of men in
whose hands lay the real power---the soldiers who fired, or
transported provisions and guns---should consent to carry out the
will of these weak individuals, and should have been induced to
do so by an infinite number of diverse and complex causes.

We are forced to fall back on fatalism as an explanation of
irrational events (that is to say, events the reasonableness of
which we do not understand). The more we try to explain such
events in history reasonably, the more unreasonable and
incomprehensible do they become to us.

Each man lives for himself, using his freedom to attain his
personal aims, and feels with his whole being that he can now do
or abstain from doing this or that action; but as soon as he has
done it, that action performed at a certain moment in time
becomes irrevocable and belongs to history, in which it has not a
free but a predestined significance.

There are two sides to the life of every man, his individual
life, which is the more free the more abstract its interests, and
his elemental hive life in which he inevitably obeys laws laid
down for him.

Man lives consciously for himself, but is an unconscious
instrument in the attainment of the historic, universal, aims of
humanity. A deed done is irrevocable, and its result coinciding
in time with the actions of millions of other men assumes an
historic significance. The higher a man stands on the social
ladder, the more people he is connected with and the more power
he has over others, the more evident is the predestination and
inevitability of his every action.

``The king's heart is in the hands of the Lord.''

A king is history's slave.

History, that is, the unconscious, general, hive life of mankind,
uses every moment of the life of kings as a tool for its own
purposes.

Though Napoleon at that time, in 1812, was more convinced than
ever that it depended on him, verser (ou ne pas verser) le sang
de ses peuples\footnote{``To shed (or not to shed) the blood of
his peoples.”}---as Alexander expressed it in the last letter he
wrote him---he had never been so much in the grip of inevitable
laws, which compelled him, while thinking that he was acting on
his own volition, to perform for the hive life---that is to say,
for history---whatever had to be performed.

The people of the west moved eastwards to slay their fellow men,
and by the law of coincidence thousands of minute causes fitted
in and co-ordinated to produce that movement and war: reproaches
for the nonobservance of the Continental System, the Duke of
Oldenburg's wrongs, the movement of troops into
Prussia---undertaken (as it seemed to Napoleon) only for the
purpose of securing an armed peace, the French Emperor's love and
habit of war coinciding with his people's inclinations,
allurement by the grandeur of the preparations, and the
expenditure on those preparations and the need of obtaining
advantages to compensate for that expenditure, the intoxicating
honors he received in Dresden, the diplomatic negotiations which,
in the opinion of contemporaries, were carried on with a sincere
desire to attain peace, but which only wounded the self-love of
both sides, and millions of other causes that adapted themselves
to the event that was happening or coincided with it.

When an apple has ripened and falls, why does it fall? Because of
its attraction to the earth, because its stalk withers, because
it is dried by the sun, because it grows heavier, because the
wind shakes it, or because the boy standing below wants to eat
it?

Nothing is the cause. All this is only the coincidence of
conditions in which all vital organic and elemental events
occur. And the botanist who finds that the apple falls because
the cellular tissue decays and so forth is equally right with the
child who stands under the tree and says the apple fell because
he wanted to eat it and prayed for it. Equally right or wrong is
he who says that Napoleon went to Moscow because he wanted to,
and perished because Alexander desired his destruction, and he
who says that an undermined hill weighing a million tons fell
because the last navvy struck it for the last time with his
mattock. In historic events the so-called great men are labels
giving names to events, and like labels they have but the
smallest connection with the event itself.

Every act of theirs, which appears to them an act of their own
will, is in an historical sense involuntary and is related to the
whole course of history and predestined from eternity.

% % % % % % % % % % % % % % % % % % % % % % % % % % % % % % % % %
% % % % % % % % % % % % % % % % % % % % % % % % % % % % % % % % %
% % % % % % % % % % % % % % % % % % % % % % % % % % % % % % % % %
% % % % % % % % % % % % % % % % % % % % % % % % % % % % % % % % %
% % % % % % % % % % % % % % % % % % % % % % % % % % % % % % % % %
% % % % % % % % % % % % % % % % % % % % % % % % % % % % % % % % %
% % % % % % % % % % % % % % % % % % % % % % % % % % % % % % % % %
% % % % % % % % % % % % % % % % % % % % % % % % % % % % % % % % %
% % % % % % % % % % % % % % % % % % % % % % % % % % % % % % % % %
% % % % % % % % % % % % % % % % % % % % % % % % % % % % % % % % %
% % % % % % % % % % % % % % % % % % % % % % % % % % % % % % % % %
% % % % % % % % % % % % % % % % % % % % % % % % % % % % % %

\chapter*{Chapter II}
\ifaudio     
\marginpar{
\href{http://ia802702.us.archive.org/23/items/war_and_peace_09_0811_librivox/war_and_peace_09_02_tolstoy_64kb.mp3}{Audio}} 
\fi

\initial{O}{n} the twenty-ninth of May Napoleon left Dresden, where he had
spent three weeks surrounded by a court that included princes,
dukes, kings, and even an emperor. Before leaving, Napoleon
showed favor to the emperor, kings, and princes who had deserved
it, reprimanded the kings and princes with whom he was
dissatisfied, presented pearls and diamonds of his own---that is,
which he had taken from other kings---to the Empress of Austria,
and having, as his historian tells us, tenderly embraced the
Empress Marie Louise---who regarded him as her husband, though he
had left another wife in Paris---left her grieved by the parting
which she seemed hardly able to bear. Though the diplomatists
still firmly believed in the possibility of peace and worked
zealously to that end, and though the Emperor Napoleon himself
wrote a letter to Alexander, calling him Monsieur mon frere, and
sincerely assured him that he did not want war and would always
love and honor him---yet he set off to join his army, and at
every station gave fresh orders to accelerate the movement of his
troops from west to east. He went in a traveling coach with six
horses, surrounded by pages, aides-de-camp, and an escort, along
the road to Posen, Thorn, Danzig, and Konigsberg. At each of
these towns thousands of people met him with excitement and
enthusiasm.

The army was moving from west to east, and relays of six horses
carried him in the same direction. On the tenth of
June,\footnote{Old style.}  coming up with the army, he spent the
night in apartments prepared for him on the estate of a Polish
count in the Vilkavisski forest.

Next day, overtaking the army, he went in a carriage to the
Niemen, and, changing into a Polish uniform, he drove to the
riverbank in order to select a place for the crossing.

Seeing, on the other side, some Cossacks (les Cosaques) and the
wide-spreading steppes in the midst of which lay the holy city of
Moscow (Moscou, la ville sainte), the capital of a realm such as
the Scythia into which Alexander the Great had marched---Napoleon
unexpectedly, and contrary alike to strategic and diplomatic
considerations, ordered an advance, and the next day his army
began to cross the Niemen.

Early in the morning of the twelfth of June he came out of his
tent, which was pitched that day on the steep left bank of the
Niemen, and looked through a spyglass at the streams of his
troops pouring out of the Vilkavisski forest and flowing over the
three bridges thrown across the river. The troops, knowing of the
Emperor's presence, were on the lookout for him, and when they
caught sight of a figure in an overcoat and a cocked hat standing
apart from his suite in front of his tent on the hill, they threw
up their caps and shouted: ``Vive l'Empereur!'' and one after
another poured in a ceaseless stream out of the vast forest that
had concealed them and, separating, flowed on and on by the three
bridges to the other side.

``Now we'll go into action. Oh, when he takes it in hand himself,
things get hot... by heaven!... There he is!... Vive l'Empereur!
So these are the steppes of Asia! It's a nasty country all the
same. Au revoir, Beauche; I'll keep the best palace in Moscow for
you! Au revoir. Good luck!... Did you see the Emperor? Vive
l'Empereur!... preur!---If they make me Governor of India,
Gerard, I'll make you Minister of Kashmir---that's settled. Vive
l'Empereur! Hurrah! hurrah! hurrah!  The Cossacks---those
rascals---see how they run! Vive l'Empereur!  There he is, do you
see him? I've seen him twice, as I see you now. The little
corporal... I saw him give the cross to one of the
veterans... Vive l'Empereur!'' came the voices of men, old and
young, of most diverse characters and social positions. On the
faces of all was one common expression of joy at the commencement
of the long-expected campaign and of rapture and devotion to the
man in the gray coat who was standing on the hill.

On the thirteenth of June a rather small, thoroughbred Arab horse
was brought to Napoleon. He mounted it and rode at a gallop to
one of the bridges over the Niemen, deafened continually by
incessant and rapturous acclamations which he evidently endured
only because it was impossible to forbid the soldiers to express
their love of him by such shouting, but the shouting which
accompanied him everywhere disturbed him and distracted him from
the military cares that had occupied him from the time he joined
the army. He rode across one of the swaying pontoon bridges to
the farther side, turned sharply to the left, and galloped in the
direction of Kovno, preceded by enraptured, mounted chasseurs of
the Guard who, breathless with delight, galloped ahead to clear a
path for him through the troops. On reaching the broad river
Viliya, he stopped near a regiment of Polish uhlans stationed by
the river.

``Vivat!'' shouted the Poles, ecstatically, breaking their ranks
and pressing against one another to see him.

Napoleon looked up and down the river, dismounted, and sat down
on a log that lay on the bank. At a mute sign from him, a
telescope was handed him which he rested on the back of a happy
page who had run up to him, and he gazed at the opposite
bank. Then he became absorbed in a map laid out on the
logs. Without lifting his head he said something, and two of his
aides-de-camp galloped off to the Polish uhlans.

``What? What did he say?'' was heard in the ranks of the Polish
uhlans when one of the aides-de-camp rode up to them.

The order was to find a ford and to cross the river. The colonel
of the Polish uhlans, a handsome old man, flushed and, fumbling
in his speech from excitement, asked the aide-de-camp whether he
would be permitted to swim the river with his uhlans instead of
seeking a ford. In evident fear of refusal, like a boy asking for
permission to get on a horse, he begged to be allowed to swim
across the river before the Emperor's eyes.  The aide-de-camp
replied that probably the Emperor would not be displeased at this
excess of zeal.

As soon as the aide-de-camp had said this, the old mustached
officer, with happy face and sparkling eyes, raised his saber,
shouted ``Vivat!''  and, commanding the uhlans to follow him,
spurred his horse and galloped into the river. He gave an angry
thrust to his horse, which had grown restive under him, and
plunged into the water, heading for the deepest part where the
current was swift. Hundreds of uhlans galloped in after him. It
was cold and uncanny in the rapid current in the middle of the
stream, and the uhlans caught hold of one another as they fell
off their horses. Some of the horses were drowned and some of the
men; the others tried to swim on, some in the saddle and some
clinging to their horses' manes. They tried to make their way
forward to the opposite bank and, though there was a ford one
third of a mile away, were proud that they were swimming and
drowning in this river under the eyes of the man who sat on the
log and was not even looking at what they were doing. When the
aide-de-camp, having returned and choosing an opportune moment,
ventured to draw the Emperor's attention to the devotion of the
Poles to his person, the little man in the gray overcoat got up
and, having summoned Berthier, began pacing up and down the bank
with him, giving him instructions and occasionally glancing
disapprovingly at the drowning uhlans who distracted his
attention.

For him it was no new conviction that his presence in any part of
the world, from Africa to the steppes of Muscovy alike, was
enough to dumfound people and impel them to insane
self-oblivion. He called for his horse and rode to his quarters.

Some forty uhlans were drowned in the river, though boats were
sent to their assistance. The majority struggled back to the bank
from which they had started. The colonel and some of his men got
across and with difficulty clambered out on the further bank. And
as soon as they had got out, in their soaked and streaming
clothes, they shouted ``Vivat!''  and looked ecstatically at the
spot where Napoleon had been but where he no longer was and at
that moment considered themselves happy.

That evening, between issuing one order that the forged Russian
paper money prepared for use in Russia should be delivered as
quickly as possible and another that a Saxon should be shot, on
whom a letter containing information about the orders to the
French army had been found, Napoleon also gave instructions that
the Polish colonel who had needlessly plunged into the river
should be enrolled in the Legion d'honneur of which Napoleon was
himself the head.

Quos vult perdere dementat.\footnote{Those whom (God) wishes to
destroy he drives mad.}

% % % % % % % % % % % % % % % % % % % % % % % % % % % % % % % % %
% % % % % % % % % % % % % % % % % % % % % % % % % % % % % % % % %
% % % % % % % % % % % % % % % % % % % % % % % % % % % % % % % % %
% % % % % % % % % % % % % % % % % % % % % % % % % % % % % % % % %
% % % % % % % % % % % % % % % % % % % % % % % % % % % % % % % % %
% % % % % % % % % % % % % % % % % % % % % % % % % % % % % % % % %
% % % % % % % % % % % % % % % % % % % % % % % % % % % % % % % % %
% % % % % % % % % % % % % % % % % % % % % % % % % % % % % % % % %
% % % % % % % % % % % % % % % % % % % % % % % % % % % % % % % % %
% % % % % % % % % % % % % % % % % % % % % % % % % % % % % % % % %
% % % % % % % % % % % % % % % % % % % % % % % % % % % % % % % % %
% % % % % % % % % % % % % % % % % % % % % % % % % % % % % %

\chapter*{Chapter III} 
\ifaudio     
\marginpar{
\href{http://ia802702.us.archive.org/23/items/war_and_peace_09_0811_librivox/war_and_peace_09_03_tolstoy_64kb.mp3}{Audio}} 
\fi

\initial{T}{he} Emperor of Russia had, meanwhile, been in Vilna for more than
a month, reviewing troops and holding maneuvers. Nothing was
ready for the war that everyone expected and to prepare for which
the Emperor had come from Petersburg. There was no general plan
of action. The vacillation between the various plans that were
proposed had even increased after the Emperor had been at
headquarters for a month. Each of the three armies had its own
commander-in-chief, but there was no supreme commander of all the
forces, and the Emperor did not assume that responsibility
himself.

The longer the Emperor remained in Vilna the less did
everybody---tired of waiting---prepare for the war. All the
efforts of those who surrounded the sovereign seemed directed
merely to making him spend his time pleasantly and forget that
war was impending.

In June, after many balls and fetes given by the Polish magnates,
by the courtiers, and by the Emperor himself, it occurred to one
of the Polish aides-de-camp in attendance that a dinner and ball
should be given for the Emperor by his aides-de-camp. This idea
was eagerly received. The Emperor gave his consent. The
aides-de-camp collected money by subscription. The lady who was
thought to be most pleasing to the Emperor was invited to act as
hostess. Count Bennigsen, being a landowner in the Vilna
province, offered his country house for the fete, and the
thirteenth of June was fixed for a ball, dinner, regatta, and
fireworks at Zakret, Count Bennigsen's country seat.

The very day that Napoleon issued the order to cross the Niemen,
and his vanguard, driving off the Cossacks, crossed the Russian
frontier, Alexander spent the evening at the entertainment given
by his aides-de-camp at Bennigsen's country house.

It was a gay and brilliant fete. Connoisseurs of such matters
declared that rarely had so many beautiful women been assembled
in one place.  Countess Bezukhova was present among other Russian
ladies who had followed the sovereign from Petersburg to Vilna
and eclipsed the refined Polish ladies by her massive, so-called
Russian type of beauty. The Emperor noticed her and honored her
with a dance.

Boris Drubetskoy, having left his wife in Moscow and being for
the present en garcon (as he phrased it), was also there and,
though not an aide-de-camp, had subscribed a large sum toward the
expenses. Boris was now a rich man who had risen to high honors
and no longer sought patronage but stood on an equal footing with
the highest of those of his own age. He was meeting Helene in
Vilna after not having seen her for a long time and did not
recall the past, but as Helene was enjoying the favors of a very
important personage and Boris had only recently married, they met
as good friends of long standing.

At midnight dancing was still going on. Helene, not having a
suitable partner, herself offered to dance the mazurka with
Boris. They were the third couple. Boris, coolly looking at
Helene's dazzling bare shoulders which emerged from a dark,
gold-embroidered, gauze gown, talked to her of old acquaintances
and at the same time, unaware of it himself and unnoticed by
others, never for an instant ceased to observe the Emperor who
was in the same room. The Emperor was not dancing, he stood in
the doorway, stopping now one pair and now another with gracious
words which he alone knew how to utter.

As the mazurka began, Boris saw that Adjutant General Balashev,
one of those in closest attendance on the Emperor, went up to him
and contrary to court etiquette stood near him while he was
talking to a Polish lady.  Having finished speaking to her, the
Emperor looked inquiringly at Balashev and, evidently
understanding that he only acted thus because there were
important reasons for so doing, nodded slightly to the lady and
turned to him. Hardly had Balashev begun to speak before a look
of amazement appeared on the Emperor's face. He took Balashev by
the arm and crossed the room with him, unconsciously clearing a
path seven yards wide as the people on both sides made way for
him. Boris noticed Arakcheev's excited face when the sovereign
went out with Balashev.  Arakcheev looked at the Emperor from
under his brow and, sniffing with his red nose, stepped forward
from the crowd as if expecting the Emperor to address him. (Boris
understood that Arakcheev envied Balashev and was displeased that
evidently important news had reached the Emperor otherwise than
through himself.)

But the Emperor and Balashev passed out into the illuminated
garden without noticing Arakcheev who, holding his sword and
glancing wrathfully around, followed some twenty paces behind
them.

All the time Boris was going through the figures of the mazurka,
he was worried by the question of what news Balashev had brought
and how he could find it out before others. In the figure in
which he had to choose two ladies, he whispered to Helene that he
meant to choose Countess Potocka who, he thought, had gone out
onto the veranda, and glided over the parquet to the door opening
into the garden, where, seeing Balashev and the Emperor returning
to the veranda, he stood still. They were moving toward the
door. Boris, fluttering as if he had not had time to withdraw,
respectfully pressed close to the doorpost with bowed head.

The Emperor, with the agitation of one who has been personally
affronted, was finishing with these words:

``To enter Russia without declaring war! I will not make peace as
long as a single armed enemy remains in my country!'' It seemed
to Boris that it gave the Emperor pleasure to utter these
words. He was satisfied with the form in which he had expressed
his thoughts, but displeased that Boris had overheard it.

``Let no one know of it!'' the Emperor added with a frown.

Boris understood that this was meant for him and, closing his
eyes, slightly bowed his head. The Emperor re-entered the
ballroom and remained there about another half-hour.

Boris was thus the first to learn the news that the French army
had crossed the Niemen and, thanks to this, was able to show
certain important personages that much that was concealed from
others was usually known to him, and by this means he rose higher
in their estimation.

The unexpected news of the French having crossed the Niemen was
particularly startling after a month of unfulfilled expectations,
and at a ball. On first receiving the news, under the influence
of indignation and resentment the Emperor had found a phrase that
pleased him, fully expressed his feelings, and has since become
famous. On returning home at two o'clock that night he sent for
his secretary, Shishkov, and told him to write an order to the
troops and a rescript to Field Marshal Prince Saltykov, in which
he insisted on the words being inserted that he would not make
peace so long as a single armed Frenchman remained on Russian
soil.

Next day the following letter was sent to Napoleon:
\begin{quote} \calli
  Monsieur mon frere,

  Yesterday I learned that, despite the loyalty with which I have
  kept my engagements with Your Majesty, your troops have crossed
  the Russian frontier, and I have this moment received from
  Petersburg a note, in which Count Lauriston informs me, as a
  reason for this aggression, that Your Majesty has considered
  yourself to be in a state of war with me from the time Prince
  Kuragin asked for his passports. The reasons on which the Duc
  de Bassano based his refusal to deliver them to him would never
  have led me to suppose that that could serve as a pretext for
  aggression. In fact, the ambassador, as he himself has
  declared, was never authorized to make that demand, and as soon
  as I was informed of it I let him know how much I disapproved
  of it and ordered him to remain at his post. If Your Majesty
  does not intend to shed the blood of our peoples for such a
  misunderstanding, and consents to withdraw your troops from
  Russian territory, I will regard what has passed as not having
  occurred and an understanding between us will be possible. In
  the contrary case, Your Majesty, I shall see myself forced to
  repel an attack that nothing on my part has provoked. It still
  depends on Your Majesty to preserve humanity from the calamity
  of another war. 

  I am, etc.,

  \textsc{Alexander.}
\end{quote}

% % % % % % % % % % % % % % % % % % % % % % % % % % % % % % % % %
% % % % % % % % % % % % % % % % % % % % % % % % % % % % % % % % %
% % % % % % % % % % % % % % % % % % % % % % % % % % % % % % % % %
% % % % % % % % % % % % % % % % % % % % % % % % % % % % % % % % %
% % % % % % % % % % % % % % % % % % % % % % % % % % % % % % % % %
% % % % % % % % % % % % % % % % % % % % % % % % % % % % % % % % %
% % % % % % % % % % % % % % % % % % % % % % % % % % % % % % % % %
% % % % % % % % % % % % % % % % % % % % % % % % % % % % % % % % %
% % % % % % % % % % % % % % % % % % % % % % % % % % % % % % % % %
% % % % % % % % % % % % % % % % % % % % % % % % % % % % % % % % %
% % % % % % % % % % % % % % % % % % % % % % % % % % % % % % % % %
% % % % % % % % % % % % % % % % % % % % % % % % % % % % % %

\chapter*{Chapter IV}
\ifaudio     
\marginpar{
\href{http://ia802702.us.archive.org/23/items/war_and_peace_09_0811_librivox/war_and_peace_09_04_tolstoy_64kb.mp3}{Audio}} 
\fi

\initial{A}{t} two in the morning of the fourteenth of June, the Emperor,
having sent for Balashev and read him his letter to Napoleon,
ordered him to take it and hand it personally to the French
Emperor. When dispatching Balashev, the Emperor repeated to him
the words that he would not make peace so long as a single armed
enemy remained on Russian soil and told him to transmit those
words to Napoleon. Alexander did not insert them in his letter to
Napoleon, because with his characteristic tact he felt it would
be injudicious to use them at a moment when a last attempt at
reconciliation was being made, but he definitely instructed
Balashev to repeat them personally to Napoleon.

Having set off in the small hours of the fourteenth, accompanied
by a bugler and two Cossacks, Balashev reached the French
outposts at the village of Rykonty, on the Russian side of the
Niemen, by dawn. There he was stopped by French cavalry
sentinels.

A French noncommissioned officer of hussars, in crimson uniform
and a shaggy cap, shouted to the approaching Balashev to
halt. Balashev did not do so at once, but continued to advance
along the road at a walking pace.

The noncommissioned officer frowned and, muttering words of
abuse, advanced his horse's chest against Balashev, put his hand
to his saber, and shouted rudely at the Russian general, asking:
was he deaf that he did not do as he was told? Balashev mentioned
who he was. The noncommissioned officer began talking with his
comrades about regimental matters without looking at the Russian
general.

After living at the seat of the highest authority and power,
after conversing with the Emperor less than three hours before,
and in general being accustomed to the respect due to his rank in
the service, Balashev found it very strange here on Russian soil
to encounter this hostile, and still more this disrespectful,
application of brute force to himself.

The sun was only just appearing from behind the clouds, the air
was fresh and dewy. A herd of cattle was being driven along the
road from the village, and over the fields the larks rose
trilling, one after another, like bubbles rising in water.

Balashev looked around him, awaiting the arrival of an officer
from the village. The Russian Cossacks and bugler and the French
hussars looked silently at one another from time to time.

A French colonel of hussars, who had evidently just left his bed,
came riding from the village on a handsome sleek gray horse,
accompanied by two hussars. The officer, the soldiers, and their
horses all looked smart and well kept.

It was that first period of a campaign when troops are still in
full trim, almost like that of peacetime maneuvers, but with a
shade of martial swagger in their clothes, and a touch of the
gaiety and spirit of enterprise which always accompany the
opening of a campaign.

The French colonel with difficulty repressed a yawn, but was
polite and evidently understood Balashev's importance. He led him
past his soldiers and behind the outposts and told him that his
wish to be presented to the Emperor would most likely be
satisfied immediately, as the Emperor's quarters were, he
believed, not far off.

They rode through the village of Rykonty, past tethered French
hussar horses, past sentinels and men who saluted their colonel
and stared with curiosity at a Russian uniform, and came out at
the other end of the village. The colonel said that the commander
of the division was a mile and a quarter away and would receive
Balashev and conduct him to his destination.

The sun had by now risen and shone gaily on the bright verdure.

They had hardly ridden up a hill, past a tavern, before they saw
a group of horsemen coming toward them. In front of the group, on
a black horse with trappings that glittered in the sun, rode a
tall man with plumes in his hat and black hair curling down to
his shoulders. He wore a red mantle, and stretched his long legs
forward in French fashion. This man rode toward Balashev at a
gallop, his plumes flowing and his gems and gold lace glittering
in the bright June sunshine.

Balashev was only two horses' length from the equestrian with the
bracelets, plumes, necklaces, and gold embroidery, who was
galloping toward him with a theatrically solemn countenance, when
Julner, the French colonel, whispered respectfully: ``The King of
Naples!'' It was, in fact, Murat, now called ``King of Naples.''
Though it was quite incomprehensible why he should be King of
Naples, he was called so, and was himself convinced that he was
so, and therefore assumed a more solemn and important air than
formerly. He was so sure that he really was the King of Naples
that when, on the eve of his departure from that city, while
walking through the streets with his wife, some Italians called
out to him: ``Viva il re!''\footnote{``Long live the king.''} he
turned to his wife with a pensive smile and said: ``Poor fellows,
they don't know that I am leaving them tomorrow!''

But though he firmly believed himself to be King of Naples and
pitied the grief felt by the subjects he was abandoning,
latterly, after he had been ordered to return to military
service---and especially since his last interview with Napoleon
in Danzig, when his august brother-in-law had told him: ``I made
you King that you should reign in my way, but not in
yours!''---he had cheerfully taken up his familiar business,
and---like a well-fed but not overfat horse that feels himself in
harness and grows skittish between the shafts---he dressed up in
clothes as variegated and expensive as possible, and gaily and
contentedly galloped along the roads of Poland, without himself
knowing why or whither.

On seeing the Russian general he threw back his head, with its
long hair curling to his shoulders, in a majestically royal
manner, and looked inquiringly at the French colonel. The colonel
respectfully informed His Majesty of Balashev's mission, whose
name he could not pronounce.

``De Bal-macheve!'' said the King (overcoming by his assurance
the difficulty that had presented itself to the
colonel). ``Charmed to make your acquaintance, General!'' he
added, with a gesture of kingly condescension.

As soon as the King began to speak loud and fast his royal
dignity instantly forsook him, and without noticing it he passed
into his natural tone of good-natured familiarity. He laid his
hand on the withers of Balashev's horse and said:

``Well, General, it all looks like war,'' as if regretting a
circumstance of which he was unable to judge.

``Your Majesty,'' replied Balashev, ``my master, the Emperor,
does not desire war and as Your Majesty sees...'' said Balashev,
using the words Your Majesty at every opportunity, with the
affectation unavoidable in frequently addressing one to whom the
title was still a novelty.

Murat's face beamed with stupid satisfaction as he listened to
``Monsieur de Bal-macheve.'' But royaute
oblige!\footnote{``Royalty has its obligations.''} and he felt it
incumbent on him, as a king and an ally, to confer on state
affairs with Alexander's envoy. He dismounted, took Balashev's
arm, and moving a few steps away from his suite, which waited
respectfully, began to pace up and down with him, trying to speak
significantly. He referred to the fact that the Emperor Napoleon
had resented the demand that he should withdraw his troops from
Prussia, especially when that demand became generally known and
the dignity of France was thereby offended.

Balashev replied that there was ``nothing offensive in the
demand, because...'' but Murat interrupted him.

``Then you don't consider the Emperor Alexander the aggressor?''
he asked unexpectedly, with a kindly and foolish smile.

Balashev told him why he considered Napoleon to be the originator
of the war.

``Oh, my dear general!'' Murat again interrupted him, ``with all
my heart I wish the Emperors may arrange the affair between them,
and that the war begun by no wish of mine may finish as quickly
as possible!'' said he, in the tone of a servant who wants to
remain good friends with another despite a quarrel between their
masters.

And he went on to inquiries about the Grand Duke and the state of
his health, and to reminiscences of the gay and amusing times he
had spent with him in Naples. Then suddenly, as if remembering
his royal dignity, Murat solemnly drew himself up, assumed the
pose in which he had stood at his coronation, and, waving his
right arm, said:

``I won't detain you longer, General. I wish success to your
mission,'' and with his embroidered red mantle, his flowing
feathers, and his glittering ornaments, he rejoined his suite who
were respectfully awaiting him.

Balashev rode on, supposing from Murat's words that he would very
soon be brought before Napoleon himself. But instead of that, at
the next village the sentinels of Davout's infantry corps
detained him as the pickets of the vanguard had done, and an
adjutant of the corps commander, who was fetched, conducted him
into the village to Marshal Davout.

% % % % % % % % % % % % % % % % % % % % % % % % % % % % % % % % %
% % % % % % % % % % % % % % % % % % % % % % % % % % % % % % % % %
% % % % % % % % % % % % % % % % % % % % % % % % % % % % % % % % %
% % % % % % % % % % % % % % % % % % % % % % % % % % % % % % % % %
% % % % % % % % % % % % % % % % % % % % % % % % % % % % % % % % %
% % % % % % % % % % % % % % % % % % % % % % % % % % % % % % % % %
% % % % % % % % % % % % % % % % % % % % % % % % % % % % % % % % %
% % % % % % % % % % % % % % % % % % % % % % % % % % % % % % % % %
% % % % % % % % % % % % % % % % % % % % % % % % % % % % % % % % %
% % % % % % % % % % % % % % % % % % % % % % % % % % % % % % % % %
% % % % % % % % % % % % % % % % % % % % % % % % % % % % % % % % %
% % % % % % % % % % % % % % % % % % % % % % % % % % % % % %

\chapter*{Chapter V}
\ifaudio     
\marginpar{
\href{http://ia802702.us.archive.org/23/items/war_and_peace_09_0811_librivox/war_and_peace_09_05_tolstoy_64kb.mp3}{Audio}} 
\fi

\initial{D}{avout} was to Napoleon what Arakcheev was to Alexander---though
not a coward like Arakcheev, he was as precise, as cruel, and as
unable to express his devotion to his monarch except by cruelty.

In the organism of states such men are necessary, as wolves are
necessary in the organism of nature, and they always exist,
always appear and hold their own, however incongruous their
presence and their proximity to the head of the government may
be. This inevitability alone can explain how the cruel Arakcheev,
who tore out a grenadier's mustache with his own hands, whose
weak nerves rendered him unable to face danger, and who was
neither an educated man nor a courtier, was able to maintain his
powerful position with Alexander, whose own character was
chivalrous, noble, and gentle.

Balashev found Davout seated on a barrel in the shed of a
peasant's hut, writing---he was auditing accounts. Better
quarters could have been found him, but Marshal Davout was one of
those men who purposely put themselves in most depressing
conditions to have a justification for being gloomy. For the same
reason they are always hard at work and in a hurry. ``How can I
think of the bright side of life when, as you see, I am sitting
on a barrel and working in a dirty shed?'' the expression of his
face seemed to say. The chief pleasure and necessity of such men,
when they encounter anyone who shows animation, is to flaunt
their own dreary, persistent activity. Davout allowed himself
that pleasure when Balashev was brought in. He became still more
absorbed in his task when the Russian general entered, and after
glancing over his spectacles at Balashev's face, which was
animated by the beauty of the morning and by his talk with Murat,
he did not rise or even stir, but scowled still more and sneered
malevolently.

When he noticed in Balashev's face the disagreeable impression
this reception produced, Davout raised his head and coldly asked
what he wanted.

Thinking he could have been received in such a manner only
because Davout did not know that he was adjutant general to the
Emperor Alexander and even his envoy to Napoleon, Balashev
hastened to inform him of his rank and mission. Contrary to his
expectation, Davout, after hearing him, became still surlier and
ruder.

``Where is your dispatch?'' he inquired. ``Give it to me. I will
send it to the Emperor.''

Balashev replied that he had been ordered to hand it personally
to the Emperor.

``Your Emperor's orders are obeyed in your army, but here,'' said
Davout, ``you must do as you're told.''

And, as if to make the Russian general still more conscious of
his dependence on brute force, Davout sent an adjutant to call
the officer on duty.

Balashev took out the packet containing the Emperor's letter and
laid it on the table (made of a door with its hinges still
hanging on it, laid across two barrels). Davout took the packet
and read the inscription.

``You are perfectly at liberty to treat me with respect or not,''
protested Balashev, ``but permit me to observe that I have the
honor to be adjutant general to His Majesty...''

Davout glanced at him silently and plainly derived pleasure from
the signs of agitation and confusion which appeared on Balashev's
face.

``You will be treated as is fitting,'' said he and, putting the
packet in his pocket, left the shed.

A minute later the marshal's adjutant, de Castres, came in and
conducted Balashev to the quarters assigned him.

That day he dined with the marshal, at the same board on the
barrels.

Next day Davout rode out early and, after asking Balashev to come
to him, peremptorily requested him to remain there, to move on
with the baggage train should orders come for it to move, and to
talk to no one except Monsieur de Castres.

After four days of solitude, ennui, and consciousness of his
impotence and insignificance---particularly acute by contrast
with the sphere of power in which he had so lately moved---and
after several marches with the marshal's baggage and the French
army, which occupied the whole district, Balashev was brought to
Vilna---now occupied by the French---through the very gate by
which he had left it four days previously.

Next day the imperial gentleman-in-waiting, the Comte de Turenne,
came to Balashev and informed him of the Emperor Napoleon's wish
to honor him with an audience.

Four days before, sentinels of the Preobrazhensk regiment had
stood in front of the house to which Balashev was conducted, and
now two French grenadiers stood there in blue uniforms unfastened
in front and with shaggy caps on their heads, and an escort of
hussars and uhlans and a brilliant suite of aides-de-camp, pages,
and generals, who were waiting for Napoleon to come out, were
standing at the porch, round his saddle horse and his Mameluke,
Rustan. Napoleon received Balashev in the very house in Vilna
from which Alexander had dispatched him on his mission.

% % % % % % % % % % % % % % % % % % % % % % % % % % % % % % % % %
% % % % % % % % % % % % % % % % % % % % % % % % % % % % % % % % %
% % % % % % % % % % % % % % % % % % % % % % % % % % % % % % % % %
% % % % % % % % % % % % % % % % % % % % % % % % % % % % % % % % %
% % % % % % % % % % % % % % % % % % % % % % % % % % % % % % % % %
% % % % % % % % % % % % % % % % % % % % % % % % % % % % % % % % %
% % % % % % % % % % % % % % % % % % % % % % % % % % % % % % % % %
% % % % % % % % % % % % % % % % % % % % % % % % % % % % % % % % %
% % % % % % % % % % % % % % % % % % % % % % % % % % % % % % % % %
% % % % % % % % % % % % % % % % % % % % % % % % % % % % % % % % %
% % % % % % % % % % % % % % % % % % % % % % % % % % % % % % % % %
% % % % % % % % % % % % % % % % % % % % % % % % % % % % % %

\chapter*{Chapter VI}
\ifaudio     
\marginpar{
\href{http://ia802702.us.archive.org/23/items/war_and_peace_09_0811_librivox/war_and_peace_09_06_tolstoy_64kb.mp3}{Audio}} 
\fi

\initial{T}{hough} Balashev was used to imperial pomp, he was amazed at the
luxury and magnificence of Napoleon's court.

The Comte de Turenne showed him into a big reception room where
many generals, gentlemen-in-waiting, and Polish
magnates---several of whom Balashev had seen at the court of the
Emperor of Russia---were waiting.  Duroc said that Napoleon would
receive the Russian general before going for his ride.

After some minutes, the gentleman-in-waiting who was on duty came
into the great reception room and, bowing politely, asked
Balashev to follow him.

Balashev went into a small reception room, one door of which led
into a study, the very one from which the Russian Emperor had
dispatched him on his mission. He stood a minute or two,
waiting. He heard hurried footsteps beyond the door, both halves
of it were opened rapidly; all was silent and then from the study
the sound was heard of other steps, firm and resolute---they were
those of Napoleon. He had just finished dressing for his ride,
and wore a blue uniform, opening in front over a white waistcoat
so long that it covered his rotund stomach, white leather
breeches tightly fitting the fat thighs of his short legs, and
Hessian boots. His short hair had evidently just been brushed,
but one lock hung down in the middle of his broad forehead. His
plump white neck stood out sharply above the black collar of his
uniform, and he smelled of Eau de Cologne. His full face, rather
young-looking, with its prominent chin, wore a gracious and
majestic expression of imperial welcome.

He entered briskly, with a jerk at every step and his head
slightly thrown back. His whole short corpulent figure with broad
thick shoulders, and chest and stomach involuntarily protruding,
had that imposing and stately appearance one sees in men of forty
who live in comfort. It was evident, too, that he was in the best
of spirits that day.

He nodded in answer to Balashav's low and respectful bow, and
coming up to him at once began speaking like a man who values
every moment of his time and does not condescend to prepare what
he has to say but is sure he will always say the right thing and
say it well.

``Good day, General!'' said he. ``I have received the letter you
brought from the Emperor Alexander and am very glad to see you.''
He glanced with his large eyes into Balashav's face and
immediately looked past him.

It was plain that Balashev's personality did not interest him at
all.  Evidently only what took place within his own mind
interested him.  Nothing outside himself had any significance for
him, because everything in the world, it seemed to him, depended
entirely on his will.

``I do not, and did not, desire war,'' he continued, ``but it has
been forced on me. Even now'' (he emphasized the word) ``I am
ready to receive any explanations you can give me.''

And he began clearly and concisely to explain his reasons for
dissatisfaction with the Russian government. Judging by the
calmly moderate and amicable tone in which the French Emperor
spoke, Balashev was firmly persuaded that he wished for peace and
intended to enter into negotiations.

When Napoleon, having finished speaking, looked inquiringly at
the Russian envoy, Balashev began a speech he had prepared long
before: ``Sire! The Emperor, my master...'' but the sight of the
Emperor's eyes bent on him confused him. ``You are
flurried---compose yourself!'' Napoleon seemed to say, as with a
scarcely perceptible smile he looked at Balashev's uniform and
sword.

Balashev recovered himself and began to speak. He said that the
Emperor Alexander did not consider Kurakin's demand for his
passports a sufficient cause for war; that Kurakin had acted on
his own initiative and without his sovereign's assent, that the
Emperor Alexander did not desire war, and had no relations with
England.

``Not yet!'' interposed Napoleon, and, as if fearing to give vent
to his feelings, he frowned and nodded slightly as a sign that
Balashev might proceed.

After saying all he had been instructed to say, Balashev added
that the Emperor Alexander wished for peace, but would not enter
into negotiations except on condition that... Here Balashev
hesitated: he remembered the words the Emperor Alexander had not
written in his letter, but had specially inserted in the rescript
to Saltykov and had told Balashev to repeat to Napoleon. Balashev
remembered these words, ``So long as a single armed foe remains
on Russian soil,'' but some complex feeling restrained him. He
could not utter them, though he wished to do so. He grew confused
and said: ``On condition that the French army retires beyond the
Niemen.''

Napoleon noticed Balashev's embarrassment when uttering these
last words; his face twitched and the calf of his left leg began
to quiver rhythmically. Without moving from where he stood he
began speaking in a louder tone and more hurriedly than
before. During the speech that followed, Balashev, who more than
once lowered his eyes, involuntarily noticed the quivering of
Napoleon's left leg which increased the more Napoleon raised his
voice.

``I desire peace, no less than the Emperor Alexander,'' he
began. ``Have I not for eighteen months been doing everything to
obtain it? I have waited eighteen months for explanations. But in
order to begin negotiations, what is demanded of me?'' he said,
frowning and making an energetic gesture of inquiry with his
small white plump hand.

``The withdrawal of your army beyond the Niemen, sire,'' replied
Balashev.

``The Niemen?'' repeated Napoleon. ``So now you want me to retire
beyond the Niemen---only the Niemen?'' repeated Napoleon, looking
straight at Balashev.

The latter bowed his head respectfully.

Instead of the demand of four months earlier to withdraw from
Pomerania, only a withdrawal beyond the Niemen was now
demanded. Napoleon turned quickly and began to pace the room.

``You say the demand now is that I am to withdraw beyond the
Niemen before commencing negotiations, but in just the same way
two months ago the demand was that I should withdraw beyond the
Vistula and the Oder, and yet you are willing to negotiate.''

He went in silence from one corner of the room to the other and
again stopped in front of Balashev. Balashev noticed that his
left leg was quivering faster than before and his face seemed
petrified in its stern expression. This quivering of his left leg
was a thing Napoleon was conscious of. ``The vibration of my left
calf is a great sign with me,'' he remarked at a later date.

``Such demands as to retreat beyond the Vistula and Oder may be
made to a Prince of Baden, but not to me!'' Napoleon almost
screamed, quite to his own surprise. ``If you gave me Petersburg
and Moscow I could not accept such conditions. You say I have
begun this war! But who first joined his army? The Emperor
Alexander, not I! And you offer me negotiations when I have
expended millions, when you are in alliance with England, and
when your position is a bad one. You offer me negotiations! But
what is the aim of your alliance with England? What has she given
you?'' he continued hurriedly, evidently no longer trying to show
the advantages of peace and discuss its possibility, but only to
prove his own rectitude and power and Alexander's errors and
duplicity.

The commencement of his speech had obviously been made with the
intention of demonstrating the advantages of his position and
showing that he was nevertheless willing to negotiate. But he had
begun talking, and the more he talked the less could he control
his words.

The whole purport of his remarks now was evidently to exalt
himself and insult Alexander---just what he had least desired at
the commencement of the interview.

``I hear you have made peace with Turkey?''

Balashev bowed his head affirmatively.

``Peace has been concluded...'' he began.

But Napoleon did not let him speak. He evidently wanted to do all
the talking himself, and continued to talk with the sort of
eloquence and unrestrained irritability to which spoiled people
are so prone.

``Yes, I know you have made peace with the Turks without
obtaining Moldavia and Wallachia; I would have given your
sovereign those provinces as I gave him Finland. Yes,'' he went
on, ``I promised and would have given the Emperor Alexander
Moldavia and Wallachia, and now he won't have those splendid
provinces. Yet he might have united them to his empire and in a
single reign would have extended Russia from the Gulf of Bothnia
to the mouths of the Danube. Catherine the Great could not have
done more,'' said Napoleon, growing more and more excited as he
paced up and down the room, repeating to Balashev almost the very
words he had used to Alexander himself at Tilsit. ``All that, he
would have owed to my friendship. Oh, what a splendid reign!'' he
repeated several times, then paused, drew from his pocket a gold
snuffbox, lifted it to his nose, and greedily sniffed at it.

``What a splendid reign the Emperor Alexander's might have
been!''

He looked compassionately at Balashev, and as soon as the latter
tried to make some rejoinder hastily interrupted him.

``What could he wish or look for that he would not have obtained
through my friendship?'' demanded Napoleon, shrugging his
shoulders in perplexity. ``But no, he has preferred to surround
himself with my enemies, and with whom? With Steins, Armfeldts,
Bennigsens, and Wintzingerodes! Stein, a traitor expelled from
his own country; Armfeldt, a rake and an intriguer;
Wintzingerode, a fugitive French subject; Bennigsen, rather more
of a soldier than the others, but all the same an incompetent who
was unable to do anything in 1807 and who should awaken terrible
memories in the Emperor Alexander's mind...  Granted that were
they competent they might be made use of,'' continued
Napoleon---hardly able to keep pace in words with the rush of
thoughts that incessantly sprang up, proving how right and strong
he was (in his perception the two were one and the same)---``but
they are not even that!  They are neither fit for war nor peace!
Barclay is said to be the most capable of them all, but I cannot
say so, judging by his first movements. And what are they doing,
all these courtiers? Pfuel proposes, Armfeldt disputes, Bennigsen
considers, and Barclay, called on to act, does not know what to
decide on, and time passes bringing no result.  Bagration alone
is a military man. He's stupid, but he has experience, a quick
eye, and resolution... And what role is your young monarch
playing in that monstrous crowd? They compromise him and throw on
him the responsibility for all that happens. A sovereign should
not be with the army unless he is a general!'' said Napoleon,
evidently uttering these words as a direct challenge to the
Emperor. He knew how Alexander desired to be a military
commander.

``The campaign began only a week ago, and you haven't even been
able to defend Vilna. You are cut in two and have been driven out
of the Polish provinces. Your army is grumbling.''

``On the contrary, Your Majesty,'' said Balashev, hardly able to
remember what had been said to him and following these verbal
fireworks with difficulty, ``the troops are burning with
eagerness...''

``I know everything!'' Napoleon interrupted him. ``I know
everything. I know the number of your battalions as exactly as I
know my own. You have not two hundred thousand men, and I have
three times that number. I give you my word of honor,'' said
Napoleon, forgetting that his word of honor could carry no
weight---``I give you my word of honor that I have five hundred
and thirty thousand men this side of the Vistula. The Turks will
be of no use to you; they are worth nothing and have shown it by
making peace with you. As for the Swedes---it is their fate to be
governed by mad kings. Their king was insane and they changed him
for another---Bernadotte, who promptly went mad---for no Swede
would ally himself with Russia unless he were mad.''

Napoleon grinned maliciously and again raised his snuffbox to his
nose.

Balashev knew how to reply to each of Napoleon's remarks, and
would have done so; he continually made the gesture of a man
wishing to say something, but Napoleon always interrupted him. To
the alleged insanity of the Swedes, Balashev wished to reply that
when Russia is on her side Sweden is practically an island: but
Napoleon gave an angry exclamation to drown his voice. Napoleon
was in that state of irritability in which a man has to talk,
talk, and talk, merely to convince himself that he is in the
right. Balashev began to feel uncomfortable: as envoy he feared
to demean his dignity and felt the necessity of replying; but, as
a man, he shrank before the transport of groundless wrath that
had evidently seized Napoleon. He knew that none of the words now
uttered by Napoleon had any significance, and that Napoleon
himself would be ashamed of them when he came to his
senses. Balashev stood with downcast eyes, looking at the
movements of Napoleon's stout legs and trying to avoid meeting
his eyes.

``But what do I care about your allies?'' said Napoleon. ``I have
allies---the Poles. There are eighty thousand of them and they
fight like lions. And there will be two hundred thousand of
them.''

And probably still more perturbed by the fact that he had uttered
this obvious falsehood, and that Balashev still stood silently
before him in the same attitude of submission to fate, Napoleon
abruptly turned round, drew close to Balashev's face, and,
gesticulating rapidly and energetically with his white hands,
almost shouted:

``Know that if you stir up Prussia against me, I'll wipe it off
the map of Europe!'' he declared, his face pale and distorted by
anger, and he struck one of his small hands energetically with
the other. ``Yes, I will throw you back beyond the Dvina and
beyond the Dnieper, and will re-erect against you that barrier
which it was criminal and blind of Europe to allow to be
destroyed. Yes, that is what will happen to you. That is what you
have gained by alienating me!'' And he walked silently several
times up and down the room, his fat shoulders twitching.

He put his snuffbox into his waistcoat pocket, took it out again,
lifted it several times to his nose, and stopped in front of
Balashev. He paused, looked ironically straight into Balashev's
eyes, and said in a quiet voice:

``And yet what a splendid reign your master might have had!''

Balashev, feeling it incumbent on him to reply, said that from
the Russian side things did not appear in so gloomy a
light. Napoleon was silent, still looking derisively at him and
evidently not listening to him. Balashev said that in Russia the
best results were expected from the war. Napoleon nodded
condescendingly, as if to say, ``I know it's your duty to say
that, but you don't believe it yourself. I have convinced you.''

When Balashev had ended, Napoleon again took out his snuffbox,
sniffed at it, and stamped his foot twice on the floor as a
signal. The door opened, a gentleman-in-waiting, bending
respectfully, handed the Emperor his hat and gloves; another
brought him a pocket handkerchief. Napoleon, without giving them
a glance, turned to Balashev:

``Assure the Emperor Alexander from me,'' said he, taking his
hat, ``that I am as devoted to him as before: I know him
thoroughly and very highly esteem his lofty qualities. I will
detain you no longer, General; you shall receive my letter to the
Emperor.''

And Napoleon went quickly to the door. Everyone in the reception
room rushed forward and descended the staircase.

% % % % % % % % % % % % % % % % % % % % % % % % % % % % % % % % %
% % % % % % % % % % % % % % % % % % % % % % % % % % % % % % % % %
% % % % % % % % % % % % % % % % % % % % % % % % % % % % % % % % %
% % % % % % % % % % % % % % % % % % % % % % % % % % % % % % % % %
% % % % % % % % % % % % % % % % % % % % % % % % % % % % % % % % %
% % % % % % % % % % % % % % % % % % % % % % % % % % % % % % % % %
% % % % % % % % % % % % % % % % % % % % % % % % % % % % % % % % %
% % % % % % % % % % % % % % % % % % % % % % % % % % % % % % % % %
% % % % % % % % % % % % % % % % % % % % % % % % % % % % % % % % %
% % % % % % % % % % % % % % % % % % % % % % % % % % % % % % % % %
% % % % % % % % % % % % % % % % % % % % % % % % % % % % % % % % %
% % % % % % % % % % % % % % % % % % % % % % % % % % % % % %

\chapter*{Chapter VII}
\ifaudio     
\marginpar{
\href{http://ia802702.us.archive.org/23/items/war_and_peace_09_0811_librivox/war_and_peace_09_07_tolstoy_64kb.mp3}{Audio}} 
\fi

\initial{A}{fter} all that Napoleon had said to him---those bursts of anger
and the last dryly spoken words: ``I will detain you no longer,
General; you shall receive my letter,'' Balashev felt convinced
that Napoleon would not wish to see him, and would even avoid
another meeting with him---an insulted envoy---especially as he
had witnessed his unseemly anger. But, to his surprise, Balashev
received, through Duroc, an invitation to dine with the Emperor
that day.

Bessieres, Caulaincourt, and Berthier were present at that
dinner.

Napoleon met Balashev cheerfully and amiably. He not only showed
no sign of constraint or self-reproach on account of his outburst
that morning, but, on the contrary, tried to reassure
Balashev. It was evident that he had long been convinced that it
was impossible for him to make a mistake, and that in his
perception whatever he did was right, not because it harmonized
with any idea of right and wrong, but because he did it.

The Emperor was in very good spirits after his ride through
Vilna, where crowds of people had rapturously greeted and
followed him. From all the windows of the streets through which
he rode, rugs, flags, and his monogram were displayed, and the
Polish ladies, welcoming him, waved their handkerchiefs to him.

At dinner, having placed Balashev beside him, Napoleon not only
treated him amiably but behaved as if Balashev were one of his
own courtiers, one of those who sympathized with his plans and
ought to rejoice at his success. In the course of conversation he
mentioned Moscow and questioned Balashev about the Russian
capital, not merely as an interested traveler asks about a new
city he intends to visit, but as if convinced that Balashev, as a
Russian, must be flattered by his curiosity.

``How many inhabitants are there in Moscow? How many houses? Is
it true that Moscow is called 'Holy Moscow'? How many churches
are there in Moscow?'' he asked.

And receiving the reply that there were more than two hundred
churches, he remarked:

``Why such a quantity of churches?''

``The Russians are very devout,'' replied Balashev.

``But a large number of monasteries and churches is always a sign
of the backwardness of a people,'' said Napoleon, turning to
Caulaincourt for appreciation of this remark.

Balashev respectfully ventured to disagree with the French
Emperor.

``Every country has its own character,'' said he.

``But nowhere in Europe is there anything like that,'' said
Napoleon.

``I beg your Majesty's pardon,'' returned Balashev, ``besides
Russia there is Spain, where there are also many churches and
monasteries.''

This reply of Balashev's, which hinted at the recent defeats of
the French in Spain, was much appreciated when he related it at
Alexander's court, but it was not much appreciated at Napoleon's
dinner, where it passed unnoticed.

The uninterested and perplexed faces of the marshals showed that
they were puzzled as to what Balashev's tone suggested. ``If
there is a point we don't see it, or it is not at all witty,''
their expressions seemed to say. So little was his rejoinder
appreciated that Napoleon did not notice it at all and naively
asked Balashev through what towns the direct road from there to
Moscow passed. Balashev, who was on the alert all through the
dinner, replied that just as ``all roads lead to Rome,'' so all
roads lead to Moscow: there were many roads, and ``among them the
road through Poltava, which Charles XII chose.'' Balashev
involuntarily flushed with pleasure at the aptitude of this
reply, but hardly had he uttered the word Poltava before
Caulaincourt began speaking of the badness of the road from
Petersburg to Moscow and of his Petersburg reminiscences.

After dinner they went to drink coffee in Napoleon's study, which
four days previously had been that of the Emperor
Alexander. Napoleon sat down, toying with his Sevres coffee cup,
and motioned Balashev to a chair beside him.

Napoleon was in that well-known after-dinner mood which, more
than any reasoned cause, makes a man contented with himself and
disposed to consider everyone his friend. It seemed to him that
he was surrounded by men who adored him: and he felt convinced
that, after his dinner, Balashev too was his friend and
worshiper. Napoleon turned to him with a pleasant, though
slightly ironic, smile.

``They tell me this is the room the Emperor Alexander occupied?
Strange, isn't it, General?'' he said, evidently not doubting
that this remark would be agreeable to his hearer since it went
to prove his, Napoleon's, superiority to Alexander.

Balashev made no reply and bowed his head in silence.

``Yes. Four days ago in this room, Wintzingerode and Stein were
deliberating,'' continued Napoleon with the same derisive and
self-confident smile. ``What I can't understand,'' he went on,
``is that the Emperor Alexander has surrounded himself with my
personal enemies. That I do not... understand. Has he not thought
that I may do the same?'' and he turned inquiringly to Balashev,
and evidently this thought turned him back on to the track of his
morning's anger, which was still fresh in him.

``And let him know that I will do so!'' said Napoleon, rising and
pushing his cup away with his hand. ``I'll drive all his
Wurttemberg, Baden, and Weimar relations out of
Germany... Yes. I'll drive them out. Let him prepare an asylum
for them in Russia!''

Balashev bowed his head with an air indicating that he would like
to make his bow and leave, and only listened because he could not
help hearing what was said to him. Napoleon did not notice this
expression; he treated Balashev not as an envoy from his enemy,
but as a man now fully devoted to him and who must rejoice at his
former master's humiliation.

``And why has the Emperor Alexander taken command of the armies?
What is the good of that? War is my profession, but his business
is to reign and not to command armies! Why has he taken on
himself such a responsibility?''

Again Napoleon brought out his snuffbox, paced several times up
and down the room in silence, and then, suddenly and
unexpectedly, went up to Balashev and with a slight smile, as
confidently, quickly, and simply as if he were doing something
not merely important but pleasing to Balashev, he raised his hand
to the forty-year-old Russian general's face and, taking him by
the ear, pulled it gently, smiling with his lips only.

To have one's ear pulled by the Emperor was considered the
greatest honor and mark of favor at the French court.

``Well, adorer and courtier of the Emperor Alexander, why don't
you say anything?'' said he, as if it was ridiculous, in his
presence, to be the adorer and courtier of anyone but himself,
Napoleon. ``Are the horses ready for the general?'' he added,
with a slight inclination of his head in reply to Balashev's
bow. ``Let him have mine, he has a long way to go!''

The letter taken by Balashev was the last Napoleon sent to
Alexander.  Every detail of the interview was communicated to the
Russian monarch, and the war began...

% % % % % % % % % % % % % % % % % % % % % % % % % % % % % % % % %
% % % % % % % % % % % % % % % % % % % % % % % % % % % % % % % % %
% % % % % % % % % % % % % % % % % % % % % % % % % % % % % % % % %
% % % % % % % % % % % % % % % % % % % % % % % % % % % % % % % % %
% % % % % % % % % % % % % % % % % % % % % % % % % % % % % % % % %
% % % % % % % % % % % % % % % % % % % % % % % % % % % % % % % % %
% % % % % % % % % % % % % % % % % % % % % % % % % % % % % % % % %
% % % % % % % % % % % % % % % % % % % % % % % % % % % % % % % % %
% % % % % % % % % % % % % % % % % % % % % % % % % % % % % % % % %
% % % % % % % % % % % % % % % % % % % % % % % % % % % % % % % % %
% % % % % % % % % % % % % % % % % % % % % % % % % % % % % % % % %
% % % % % % % % % % % % % % % % % % % % % % % % % % % % % %

\chapter*{Chapter VIII}
\ifaudio     
\marginpar{
\href{http://ia802702.us.archive.org/23/items/war_and_peace_09_0811_librivox/war_and_peace_09_08_tolstoy_64kb.mp3}{Audio}} 
\fi

\initial{A}{fter} his interview with Pierre in Moscow, Prince Andrew went to
Petersburg, on business as he told his family, but really to meet
Anatole Kuragin whom he felt it necessary to encounter. On
reaching Petersburg he inquired for Kuragin but the latter had
already left the city. Pierre had warned his brother-in-law that
Prince Andrew was on his track. Anatole Kuragin promptly obtained
an appointment from the Minister of War and went to join the army
in Moldavia. While in Petersburg Prince Andrew met Kutuzov, his
former commander who was always well disposed toward him, and
Kutuzov suggested that he should accompany him to the army in
Moldavia, to which the old general had been appointed
commander-in-chief. So Prince Andrew, having received an
appointment on the headquarters staff, left for Turkey.

Prince Andrew did not think it proper to write and challenge
Kuragin. He thought that if he challenged him without some fresh
cause it might compromise the young Countess Rostova and so he
wanted to meet Kuragin personally in order to find a fresh
pretext for a duel. But he again failed to meet Kuragin in
Turkey, for soon after Prince Andrew arrived, the latter returned
to Russia. In a new country, amid new conditions, Prince Andrew
found life easier to bear. After his betrothed had broken faith
with him---which he felt the more acutely the more he tried to
conceal its effects---the surroundings in which he had been happy
became trying to him, and the freedom and independence he had
once prized so highly were still more so. Not only could he no
longer think the thoughts that had first come to him as he lay
gazing at the sky on the field of Austerlitz and had later
enlarged upon with Pierre, and which had filled his solitude at
Bogucharovo and then in Switzerland and Rome, but he even dreaded
to recall them and the bright and boundless horizons they had
revealed. He was now concerned only with the nearest practical
matters unrelated to his past interests, and he seized on these
the more eagerly the more those past interests were closed to
him. It was as if that lofty, infinite canopy of heaven that had
once towered above him had suddenly turned into a low, solid
vault that weighed him down, in which all was clear, but nothing
eternal or mysterious.

Of the activities that presented themselves to him, army service
was the simplest and most familiar. As a general on duty on
Kutuzov's staff, he applied himself to business with zeal and
perseverance and surprised Kutuzov by his willingness and
accuracy in work. Not having found Kuragin in Turkey, Prince
Andrew did not think it necessary to rush back to Russia after
him, but all the same he knew that however long it might be
before he met Kuragin, despite his contempt for him and despite
all the proofs he deduced to convince himself that it was not
worth stooping to a conflict with him---he knew that when he did
meet him he would not be able to resist calling him out, any more
than a ravenous man can help snatching at food. And the
consciousness that the insult was not yet avenged, that his
rancor was still unspent, weighed on his heart and poisoned the
artificial tranquillity which he managed to obtain in Turkey by
means of restless, plodding, and rather vainglorious and
ambitious activity.

In the year 1812, when news of the war with Napoleon reached
Bucharest---where Kutuzov had been living for two months, passing
his days and nights with a Wallachian woman---Prince Andrew asked
Kutuzov to transfer him to the Western Army. Kutuzov, who was
already weary of Bolkonski's activity which seemed to reproach
his own idleness, very readily let him go and gave him a mission
to Barclay de Tolly.

Before joining the Western Army which was then, in May, encamped
at Drissa, Prince Andrew visited Bald Hills which was directly on
his way, being only two miles off the Smolensk highroad. During
the last three years there had been so many changes in his life,
he had thought, felt, and seen so much (having traveled both in
the east and the west), that on reaching Bald Hills it struck him
as strange and unexpected to find the way of life there unchanged
and still the same in every detail. He entered through the gates
with their stone pillars and drove up the avenue leading to the
house as if he were entering an enchanted, sleeping castle. The
same old stateliness, the same cleanliness, the same stillness
reigned there, and inside there was the same furniture, the same
walls, sounds, and smell, and the same timid faces, only somewhat
older. Princess Mary was still the same timid, plain maiden
getting on in years, uselessly and joylessly passing the best
years of her life in fear and constant suffering. Mademoiselle
Bourienne was the same coquettish, self-satisfied girl, enjoying
every moment of her existence and full of joyous hopes for the
future. She had merely become more self-confident, Prince Andrew
thought. Dessalles, the tutor he had brought from Switzerland,
was wearing a coat of Russian cut and talking broken Russian to
the servants, but was still the same narrowly intelligent,
conscientious, and pedantic preceptor. The old prince had changed
in appearance only by the loss of a tooth, which left a
noticeable gap on one side of his mouth; in character he was the
same as ever, only showing still more irritability and skepticism
as to what was happening in the world. Little Nicholas alone had
changed. He had grown, become rosier, had curly dark hair, and,
when merry and laughing, quite unconsciously lifted the upper lip
of his pretty little mouth just as the little princess used to
do. He alone did not obey the law of immutability in the
enchanted, sleeping castle. But though externally all remained as
of old, the inner relations of all these people had changed since
Prince Andrew had seen them last. The household was divided into
two alien and hostile camps, who changed their habits for his
sake and only met because he was there. To the one camp belonged
the old prince, Mademoiselle Bourienne, and the architect; to the
other Princess Mary, Dessalles, little Nicholas, and all the old
nurses and maids.

During his stay at Bald Hills all the family dined together, but
they were ill at ease and Prince Andrew felt that he was a
visitor for whose sake an exception was being made and that his
presence made them all feel awkward. Involuntarily feeling this
at dinner on the first day, he was taciturn, and the old prince
noticing this also became morosely dumb and retired to his
apartments directly after dinner. In the evening, when Prince
Andrew went to him and, trying to rouse him, began to tell him of
the young Count Kamensky's campaign, the old prince began
unexpectedly to talk about Princess Mary, blaming her for her
superstitions and her dislike of Mademoiselle Bourienne, who, he
said, was the only person really attached to him.

The old prince said that if he was ill it was only because of
Princess Mary: that she purposely worried and irritated him, and
that by indulgence and silly talk she was spoiling little Prince
Nicholas. The old prince knew very well that he tormented his
daughter and that her life was very hard, but he also knew that
he could not help tormenting her and that she deserved it. ``Why
does Prince Andrew, who sees this, say nothing to me about his
sister? Does he think me a scoundrel, or an old fool who, without
any reason, keeps his own daughter at a distance and attaches
this Frenchwoman to himself? He doesn't understand, so I must
explain it, and he must hear me out,'' thought the old
prince. And he began explaining why he could not put up with his
daughter's unreasonable character.

``If you ask me,'' said Prince Andrew, without looking up (he was
censuring his father for the first time in his life), ``I did not
wish to speak about it, but as you ask me I will give you my
frank opinion. If there is any misunderstanding and discord
between you and Mary, I can't blame her for it at all. I know how
she loves and respects you. Since you ask me,'' continued Prince
Andrew, becoming irritable---as he was always liable to do of
late---``I can only say that if there are any misunderstandings
they are caused by that worthless woman, who is not fit to be my
sister's companion.''

The old man at first stared fixedly at his son, and an unnatural
smile disclosed the fresh gap between his teeth to which Prince
Andrew could not get accustomed.

``What companion, my dear boy? Eh? You've already been talking it
over!  Eh?''

``Father, I did not want to judge,'' said Prince Andrew, in a
hard and bitter tone, ``but you challenged me, and I have said,
and always shall say, that Mary is not to blame, but those to
blame---the one to blame---is that Frenchwoman.''

``Ah, he has passed judgment... passed judgement!'' said the old
man in a low voice and, as it seemed to Prince Andrew, with some
embarrassment, but then he suddenly jumped up and cried: ``Be
off, be off! Let not a trace of you remain here!...''

Prince Andrew wished to leave at once, but Princess Mary
persuaded him to stay another day. That day he did not see his
father, who did not leave his room and admitted no one but
Mademoiselle Bourienne and Tikhon, but asked several times
whether his son had gone. Next day, before leaving, Prince Andrew
went to his son's rooms. The boy, curly-headed like his mother
and glowing with health, sat on his knee, and Prince Andrew began
telling him the story of Bluebeard, but fell into a reverie
without finishing the story. He thought not of this pretty child,
his son whom he held on his knee, but of himself. He sought in
himself either remorse for having angered his father or regret at
leaving home for the first time in his life on bad terms with
him, and was horrified to find neither. What meant still more to
him was that he sought and did not find in himself the former
tenderness for his son which he had hoped to reawaken by
caressing the boy and taking him on his knee.

``Well, go on!'' said his son.

Prince Andrew, without replying, put him down from his knee and
went out of the room.

As soon as Prince Andrew had given up his daily occupations, and
especially on returning to the old conditions of life amid which
he had been happy, weariness of life overcame him with its former
intensity, and he hastened to escape from these memories and to
find some work as soon as possible.

``So you've decided to go, Andrew?'' asked his sister.

``Thank God that I can,'' replied Prince Andrew. ``I am very
sorry you can't.''

``Why do you say that?'' replied Princess Mary. ``Why do you say
that, when you are going to this terrible war, and he is so old?
Mademoiselle Bourienne says he has been asking about you...''

As soon as she began to speak of that, her lips trembled and her
tears began to fall. Prince Andrew turned away and began pacing
the room.

``Ah, my God! my God! When one thinks who and what---what
trash---can cause people misery!'' he said with a malignity that
alarmed Princess Mary.

She understood that when speaking of ``trash'' he referred not
only to Mademoiselle Bourienne, the cause of her misery, but also
to the man who had ruined his own happiness.

``Andrew! One thing I beg, I entreat of you!'' she said, touching
his elbow and looking at him with eyes that shone through her
tears. ``I understand you'' (she looked down). ``Don't imagine
that sorrow is the work of men. Men are His tools.'' She looked a
little above Prince Andrew's head with the confident, accustomed
look with which one looks at the place where a familiar portrait
hangs. ``Sorrow is sent by Him, not by men. Men are His
instruments, they are not to blame. If you think someone has
wronged you, forget it and forgive! We have no right to
punish. And then you will know the happiness of forgiving.''

``If I were a woman I would do so, Mary. That is a woman's
virtue. But a man should not and cannot forgive and forget,'' he
replied, and though till that moment he had not been thinking of
Kuragin, all his unexpended anger suddenly swelled up in his
heart.

``If Mary is already persuading me to forgive, it means that I
ought long ago to have punished him,'' he thought. And giving her
no further reply, he began thinking of the glad vindictive moment
when he would meet Kuragin who he knew was now in the army.

Princess Mary begged him to stay one day more, saying that she
knew how unhappy her father would be if Andrew left without being
reconciled to him, but Prince Andrew replied that he would
probably soon be back again from the army and would certainly
write to his father, but that the longer he stayed now the more
embittered their differences would become.

``Good-bye, Andrew! Remember that misfortunes come from God, and
men are never to blame,'' were the last words he heard from his
sister when he took leave of her.

``Then it must be so!'' thought Prince Andrew as he drove out of
the avenue from the house at Bald Hills. ``She, poor innocent
creature, is left to be victimized by an old man who has outlived
his wits. The old man feels he is guilty, but cannot change
himself. My boy is growing up and rejoices in life, in which like
everybody else he will deceive or be deceived. And I am off to
the army. Why? I myself don't know. I want to meet that man whom
I despise, so as to give him a chance to kill and laugh at me!''

These conditions of life had been the same before, but then they
were all connected, while now they had all tumbled to
pieces. Only senseless things, lacking coherence, presented
themselves one after another to Prince Andrew's mind.

% % % % % % % % % % % % % % % % % % % % % % % % % % % % % % % % %
% % % % % % % % % % % % % % % % % % % % % % % % % % % % % % % % %
% % % % % % % % % % % % % % % % % % % % % % % % % % % % % % % % %
% % % % % % % % % % % % % % % % % % % % % % % % % % % % % % % % %
% % % % % % % % % % % % % % % % % % % % % % % % % % % % % % % % %
% % % % % % % % % % % % % % % % % % % % % % % % % % % % % % % % %
% % % % % % % % % % % % % % % % % % % % % % % % % % % % % % % % %
% % % % % % % % % % % % % % % % % % % % % % % % % % % % % % % % %
% % % % % % % % % % % % % % % % % % % % % % % % % % % % % % % % %
% % % % % % % % % % % % % % % % % % % % % % % % % % % % % % % % %
% % % % % % % % % % % % % % % % % % % % % % % % % % % % % % % % %
% % % % % % % % % % % % % % % % % % % % % % % % % % % % % %

\chapter*{Chapter IX}
\ifaudio     
\marginpar{
\href{http://ia802702.us.archive.org/23/items/war_and_peace_09_0811_librivox/war_and_peace_09_09_tolstoy_64kb.mp3}{Audio}} 
\fi

\initial{P}{rince} Andrew reached the general headquarters of the army at the
end of June. The first army, with which was the Emperor, occupied
the fortified camp at Drissa; the second army was retreating,
trying to effect a junction with the first one from which it was
said to be cut off by large French forces. Everyone was
dissatisfied with the general course of affairs in the Russian
army, but no one anticipated any danger of invasion of the
Russian provinces, and no one thought the war would extend
farther than the western, the Polish, provinces.

Prince Andrew found Barclay de Tolly, to whom he had been
assigned, on the bank of the Drissa. As there was not a single
town or large village in the vicinity of the camp, the immense
number of generals and courtiers accompanying the army were
living in the best houses of the villages on both sides of the
river, over a radius of six miles. Barclay de Tolly was quartered
nearly three miles from the Emperor. He received Bolkonski
stiffly and coldly and told him in his foreign accent that he
would mention him to the Emperor for a decision as to his
employment, but asked him meanwhile to remain on his
staff. Anatole Kuragin, whom Prince Andrew had hoped to find with
the army, was not there. He had gone to Petersburg, but Prince
Andrew was glad to hear this. His mind was occupied by the
interests of the center that was conducting a gigantic war, and
he was glad to be free for a while from the distraction caused by
the thought of Kuragin. During the first four days, while no
duties were required of him, Prince Andrew rode round the whole
fortified camp and, by the aid of his own knowledge and by talks
with experts, tried to form a definite opinion about it. But the
question whether the camp was advantageous or disadvantageous
remained for him undecided. Already from his military experience
and what he had seen in the Austrian campaign, he had come to the
conclusion that in war the most deeply considered plans have no
significance and that all depends on the way unexpected movements
of the enemy---that cannot be foreseen---are met, and on how and
by whom the whole matter is handled.  To clear up this last point
for himself, Prince Andrew, utilizing his position and
acquaintances, tried to fathom the character of the control of
the army and of the men and parties engaged in it, and he deduced
for himself the following of the state of affairs.

While the Emperor had still been at Vilna, the forces had been
divided into three armies. First, the army under Barclay de
Tolly, secondly, the army under Bagration, and thirdly, the one
commanded by Tormasov. The Emperor was with the first army, but
not as commander-in-chief. In the orders issued it was stated,
not that the Emperor would take command, but only that he would
be with the army. The Emperor, moreover, had with him not a
commander-in-chief's staff but the imperial headquarters
staff. In attendance on him was the head of the imperial staff,
Quartermaster General Prince Volkonski, as well as generals,
imperial aides-de-camp, diplomatic officials, and a large number
of foreigners, but not the army staff. Besides these, there were
in attendance on the Emperor without any definite appointments:
Arakcheev, the ex-Minister of War; Count Bennigsen, the senior
general in rank; the Grand Duke Tsarevich Constantine Pavlovich;
Count Rumyantsev, the Chancellor; Stein, a former Prussian
minister; Armfeldt, a Swedish general; Pfuel, the chief author of
the plan of campaign; Paulucci, an adjutant general and Sardinian
emigre; Wolzogen---and many others. Though these men had no
military appointment in the army, their position gave them
influence, and often a corps commander, or even the
commander-in-chief, did not know in what capacity he was
questioned by Bennigsen, the Grand Duke, Arakcheev, or Prince
Volkonski, or was given this or that advice and did not know
whether a certain order received in the form of advice emanated
from the man who gave it or from the Emperor and whether it had
to be executed or not. But this was only the external condition;
the essential significance of the presence of the Emperor and of
all these people, from a courtier's point of view (and in an
Emperor's vicinity all became courtiers), was clear to
everyone. It was this: the Emperor did not assume the title of
commander-in-chief, but disposed of all the armies; the men
around him were his assistants. Arakcheev was a faithful
custodian to enforce order and acted as the sovereign's
bodyguard.  Bennigsen was a landlord in the Vilna province who
appeared to be doing the honors of the district, but was in
reality a good general, useful as an adviser and ready at hand to
replace Barclay. The Grand Duke was there because it suited him
to be. The ex-Minister Stein was there because his advice was
useful and the Emperor Alexander held him in high esteem
personally. Armfeldt virulently hated Napoleon and was a general
full of self-confidence, a quality that always influenced
Alexander.  Paulucci was there because he was bold and decided in
speech. The adjutants general were there because they always
accompanied the Emperor, and lastly and chiefly Pfuel was there
because he had drawn up the plan of campaign against Napoleon
and, having induced Alexander to believe in the efficacy of that
plan, was directing the whole business of the war. With Pfuel was
Wolzogen, who expressed Pfuel's thoughts in a more comprehensible
way than Pfuel himself (who was a harsh, bookish theorist,
self-confident to the point of despising everyone else) was able
to do.

Besides these Russians and foreigners who propounded new and
unexpected ideas every day---especially the foreigners, who did
so with a boldness characteristic of people employed in a country
not their own---there were many secondary personages accompanying
the army because their principals were there.

Among the opinions and voices in this immense, restless,
brilliant, and proud sphere, Prince Andrew noticed the following
sharply defined subdivisions of tendencies and parties:

The first party consisted of Pfuel and his adherents---military
theorists who believed in a science of war with immutable
laws---laws of oblique movements, outflankings, and so
forth. Pfuel and his adherents demanded a retirement into the
depths of the country in accordance with precise laws defined by
a pseudo-theory of war, and they saw only barbarism, ignorance,
or evil intention in every deviation from that theory. To this
party belonged the foreign nobles, Wolzogen, Wintzingerode, and
others, chiefly Germans.

The second party was directly opposed to the first; one extreme,
as always happens, was met by representatives of the other. The
members of this party were those who had demanded an advance from
Vilna into Poland and freedom from all prearranged plans. Besides
being advocates of bold action, this section also represented
nationalism, which made them still more one-sided in the
dispute. They were Russians: Bagration, Ermolov (who was
beginning to come to the front), and others. At that time a
famous joke of Ermolov's was being circulated, that as a great
favor he had petitioned the Emperor to make him a German. The men
of that party, remembering Suvorov, said that what one had to do
was not to reason, or stick pins into maps, but to fight, beat
the enemy, keep him out of Russia, and not let the army get
discouraged.

To the third party---in which the Emperor had most
confidence---belonged the courtiers who tried to arrange
compromises between the other two.  The members of this party,
chiefly civilians and to whom Arakcheev belonged, thought and
said what men who have no convictions but wish to seem to have
some generally say. They said that undoubtedly war, particularly
against such a genius as Bonaparte (they called him Bonaparte
now), needs most deeply devised plans and profound scientific
knowledge and in that respect Pfuel was a genius, but at the same
time it had to be acknowledged that the theorists are often
one-sided, and therefore one should not trust them absolutely,
but should also listen to what Pfuel's opponents and practical
men of experience in warfare had to say, and then choose a middle
course. They insisted on the retention of the camp at Drissa,
according to Pfuel's plan, but on changing the movements of the
other armies. Though, by this course, neither one aim nor the
other could be attained, yet it seemed best to the adherents of
this third party.

Of a fourth opinion the most conspicuous representative was the
Tsarevich, who could not forget his disillusionment at
Austerlitz, where he had ridden out at the head of the Guards, in
his casque and cavalry uniform as to a review, expecting to crush
the French gallantly; but unexpectedly finding himself in the
front line had narrowly escaped amid the general confusion. The
men of this party had both the quality and the defect of
frankness in their opinions. They feared Napoleon, recognized his
strength and their own weakness, and frankly said so.  They said:
``Nothing but sorrow, shame, and ruin will come of all this!  We
have abandoned Vilna and Vitebsk and shall abandon Drissa. The
only reasonable thing left to do is to conclude peace as soon as
possible, before we are turned out of Petersburg.''

This view was very general in the upper army circles and found
support also in Petersburg and from the chancellor, Rumyantsev,
who, for other reasons of state, was in favor of peace.

The fifth party consisted of those who were adherents of Barclay
de Tolly, not so much as a man but as minister of war and
commander-in-chief. ``Be he what he may'' (they always began like
that), ``he is an honest, practical man and we have nobody
better. Give him real power, for war cannot be conducted
successfully without unity of command, and he will show what he
can do, as he did in Finland. If our army is well organized and
strong and has withdrawn to Drissa without suffering any defeats,
we owe this entirely to Barclay. If Barclay is now to be
superseded by Bennigsen all will be lost, for Bennigsen showed
his incapacity already in 1807.''

The sixth party, the Bennigsenites, said, on the contrary, that
at any rate there was no one more active and experienced than
Bennigsen: ``and twist about as you may, you will have to come to
Bennigsen eventually.  Let the others make mistakes now!'' said
they, arguing that our retirement to Drissa was a most shameful
reverse and an unbroken series of blunders. ``The more mistakes
that are made the better. It will at any rate be understood all
the sooner that things cannot go on like this.  What is wanted is
not some Barclay or other, but a man like Bennigsen, who made his
mark in 1807, and to whom Napoleon himself did justice---a man
whose authority would be willingly recognized, and Bennigsen is
the only such man.''

The seventh party consisted of the sort of people who are always
to be found, especially around young sovereigns, and of whom
there were particularly many round Alexander---generals and
imperial aides-de-camp passionately devoted to the Emperor, not
merely as a monarch but as a man, adoring him sincerely and
disinterestedly, as Rostov had done in 1805, and who saw in him
not only all the virtues but all human capabilities as
well. These men, though enchanted with the sovereign for refusing
the command of the army, yet blamed him for such excessive
modesty, and only desired and insisted that their adored
sovereign should abandon his diffidence and openly announce that
he would place himself at the head of the army, gather round him
a commander-in-chief's staff, and, consulting experienced
theoreticians and practical men where necessary, would himself
lead the troops, whose spirits would thereby be raised to the
highest pitch.

The eighth and largest group, which in its enormous numbers was
to the others as ninety-nine to one, consisted of men who desired
neither peace nor war, neither an advance nor a defensive camp at
the Drissa or anywhere else, neither Barclay nor the Emperor,
neither Pfuel nor Bennigsen, but only the one most essential
thing---as much advantage and pleasure for themselves as
possible. In the troubled waters of conflicting and intersecting
intrigues that eddied about the Emperor's headquarters, it was
possible to succeed in many ways unthinkable at other times. A
man who simply wished to retain his lucrative post would today
agree with Pfuel, tomorrow with his opponent, and the day after,
merely to avoid responsibility or to please the Emperor, would
declare that he had no opinion at all on the matter. Another who
wished to gain some advantage would attract the Emperor's
attention by loudly advocating the very thing the Emperor had
hinted at the day before, and would dispute and shout at the
council, beating his breast and challenging those who did not
agree with him to duels, thereby proving that he was prepared to
sacrifice himself for the common good. A third, in the absence of
opponents, between two councils would simply solicit a special
gratuity for his faithful services, well knowing that at that
moment people would be too busy to refuse him. A fourth while
seemingly overwhelmed with work would often come accidentally
under the Emperor's eye. A fifth, to achieve his long-cherished
aim of dining with the Emperor, would stubbornly insist on the
correctness or falsity of some newly emerging opinion and for
this object would produce arguments more or less forcible and
correct.

All the men of this party were fishing for rubles, decorations,
and promotions, and in this pursuit watched only the weathercock
of imperial favor, and directly they noticed it turning in any
direction, this whole drone population of the army began blowing
hard that way, so that it was all the harder for the Emperor to
turn it elsewhere. Amid the uncertainties of the position, with
the menace of serious danger giving a peculiarly threatening
character to everything, amid this vortex of intrigue, egotism,
conflict of views and feelings, and the diversity of race among
these people---this eighth and largest party of those preoccupied
with personal interests imparted great confusion and obscurity to
the common task. Whatever question arose, a swarm of these
drones, without having finished their buzzing on a previous
theme, flew over to the new one and by their hum drowned and
obscured the voices of those who were disputing honestly.

From among all these parties, just at the time Prince Andrew
reached the army, another, a ninth party, was being formed and
was beginning to raise its voice. This was the party of the
elders, reasonable men experienced and capable in state affairs,
who, without sharing any of those conflicting opinions, were able
to take a detached view of what was going on at the staff at
headquarters and to consider means of escape from this muddle,
indecision, intricacy, and weakness.

The men of this party said and thought that what was wrong
resulted chiefly from the Emperor's presence in the army with his
military court and from the consequent presence there of an
indefinite, conditional, and unsteady fluctuation of relations,
which is in place at court but harmful in an army; that a
sovereign should reign but not command the army, and that the
only way out of the position would be for the Emperor and his
court to leave the army; that the mere presence of the Emperor
paralyzed the action of fifty thousand men required to secure his
personal safety, and that the worst commander-in-chief, if
independent, would be better than the very best one trammeled by
the presence and authority of the monarch.

Just at the time Prince Andrew was living unoccupied at Drissa,
Shishkov, the Secretary of State and one of the chief
representatives of this party, wrote a letter to the Emperor
which Arakcheev and Balashev agreed to sign. In this letter,
availing himself of permission given him by the Emperor to
discuss the general course of affairs, he respectfully
suggested---on the plea that it was necessary for the sovereign
to arouse a warlike spirit in the people of the capital---that
the Emperor should leave the army.

That arousing of the people by their sovereign and his call to
them to defend their country---the very incitement which was the
chief cause of Russia's triumph in so far as it was produced by
the Tsar's personal presence in Moscow---was suggested to the
Emperor, and accepted by him, as a pretext for quitting the army.

% % % % % % % % % % % % % % % % % % % % % % % % % % % % % % % % %
% % % % % % % % % % % % % % % % % % % % % % % % % % % % % % % % %
% % % % % % % % % % % % % % % % % % % % % % % % % % % % % % % % %
% % % % % % % % % % % % % % % % % % % % % % % % % % % % % % % % %
% % % % % % % % % % % % % % % % % % % % % % % % % % % % % % % % %
% % % % % % % % % % % % % % % % % % % % % % % % % % % % % % % % %
% % % % % % % % % % % % % % % % % % % % % % % % % % % % % % % % %
% % % % % % % % % % % % % % % % % % % % % % % % % % % % % % % % %
% % % % % % % % % % % % % % % % % % % % % % % % % % % % % % % % %
% % % % % % % % % % % % % % % % % % % % % % % % % % % % % % % % %
% % % % % % % % % % % % % % % % % % % % % % % % % % % % % % % % %
% % % % % % % % % % % % % % % % % % % % % % % % % % % % % %

\chapter*{Chapter X}
\ifaudio     
\marginpar{
\href{http://ia802702.us.archive.org/23/items/war_and_peace_09_0811_librivox/war_and_peace_09_10_tolstoy_64kb.mp3}{Audio}} 
\fi

\initial{T}{his} letter had not yet been presented to the Emperor when
Barclay, one day at dinner, informed Bolkonski that the sovereign
wished to see him personally, to question him about Turkey, and
that Prince Andrew was to present himself at Bennigsen's quarters
at six that evening.

News was received at the Emperor's quarters that very day of a
fresh movement by Napoleon which might endanger the army---news
subsequently found to be false. And that morning Colonel Michaud
had ridden round the Drissa fortifications with the Emperor and
had pointed out to him that this fortified camp constructed by
Pfuel, and till then considered a chef-d'oeuvre of tactical
science which would ensure Napoleon's destruction, was an
absurdity, threatening the destruction of the Russian army.

Prince Andrew arrived at Bennigsen's quarters---a country
gentleman's house of moderate size, situated on the very banks of
the river. Neither Bennigsen nor the Emperor was there, but
Chernyshev, the Emperor's aide-de-camp, received Bolkonski and
informed him that the Emperor, accompanied by General Bennigsen
and Marquis Paulucci, had gone a second time that day to inspect
the fortifications of the Drissa camp, of the suitability of
which serious doubts were beginning to be felt.

Chernyshev was sitting at a window in the first room with a
French novel in his hand. This room had probably been a music
room; there was still an organ in it on which some rugs were
piled, and in one corner stood the folding bedstead of
Bennigsen's adjutant. This adjutant was also there and sat dozing
on the rolled-up bedding, evidently exhausted by work or by
feasting. Two doors led from the room, one straight on into what
had been the drawing room, and another, on the right, to the
study.  Through the first door came the sound of voices
conversing in German and occasionally in French. In that drawing
room were gathered, by the Emperor's wish, not a military council
(the Emperor preferred indefiniteness), but certain persons whose
opinions he wished to know in view of the impending
difficulties. It was not a council of war, but, as it were, a
council to elucidate certain questions for the Emperor
personally. To this semicouncil had been invited the Swedish
General Armfeldt, Adjutant General Wolzogen, Wintzingerode (whom
Napoleon had referred to as a renegade French subject), Michaud,
Toll, Count Stein who was not a military man at all, and Pfuel
himself, who, as Prince Andrew had heard, was the mainspring of
the whole affair. Prince Andrew had an opportunity of getting a
good look at him, for Pfuel arrived soon after himself and, in
passing through to the drawing room, stopped a minute to speak to
Chernyshev.

At first sight, Pfuel, in his ill-made uniform of a Russian
general, which fitted him badly like a fancy costume, seemed
familiar to Prince Andrew, though he saw him now for the first
time. There was about him something of Weyrother, Mack, and
Schmidt, and many other German theorist-generals whom Prince
Andrew had seen in 1805, but he was more typical than any of
them. Prince Andrew had never yet seen a German theorist in whom
all the characteristics of those others were united to such an
extent.

Pfuel was short and very thin but broad-boned, of coarse, robust
build, broad in the hips, and with prominent shoulder blades. His
face was much wrinkled and his eyes deep set. His hair had
evidently been hastily brushed smooth in front of the temples,
but stuck up behind in quaint little tufts. He entered the room,
looking restlessly and angrily around, as if afraid of everything
in that large apartment. Awkwardly holding up his sword, he
addressed Chernyshev and asked in German where the Emperor
was. One could see that he wished to pass through the rooms as
quickly as possible, finish with the bows and greetings, and sit
down to business in front of a map, where he would feel at
home. He nodded hurriedly in reply to Chernyshev, and smiled
ironically on hearing that the sovereign was inspecting the
fortifications that he, Pfuel, had planned in accord with his
theory. He muttered something to himself abruptly and in a bass
voice, as self-assured Germans do---it might have been ``stupid
fellow''... or ``the whole affair will be ruined,'' or
``something absurd will come of it.''... Prince Andrew did not
catch what he said and would have passed on, but Chernyshev
introduced him to Pfuel, remarking that Prince Andrew was just
back from Turkey where the war had terminated so
fortunately. Pfuel barely glanced---not so much at Prince Andrew
as past him---and said, with a laugh: ``That must have been a
fine tactical war''; and, laughing contemptuously, went on into
the room from which the sound of voices was heard.

Pfuel, always inclined to be irritably sarcastic, was
particularly disturbed that day, evidently by the fact that they
had dared to inspect and criticize his camp in his absence. From
this short interview with Pfuel, Prince Andrew, thanks to his
Austerlitz experiences, was able to form a clear conception of
the man. Pfuel was one of those hopelessly and immutably
self-confident men, self-confident to the point of martyrdom as
only Germans are, because only Germans are self-confident on the
basis of an abstract notion---science, that is, the supposed
knowledge of absolute truth. A Frenchman is self-assured because
he regards himself personally, both in mind and body, as
irresistibly attractive to men and women. An Englishman is
self-assured, as being a citizen of the best-organized state in
the world, and therefore as an Englishman always knows what he
should do and knows that all he does as an Englishman is
undoubtedly correct. An Italian is self-assured because he is
excitable and easily forgets himself and other people. A Russian
is self-assured just because he knows nothing and does not want
to know anything, since he does not believe that anything can be
known. The German's self-assurance is worst of all, stronger and
more repulsive than any other, because he imagines that he knows
the truth---science---which he himself has invented but which is
for him the absolute truth.

Pfuel was evidently of that sort. He had a science---the theory
of oblique movements deduced by him from the history of Frederick
the Great's wars, and all he came across in the history of more
recent warfare seemed to him absurd and barbarous---monstrous
collisions in which so many blunders were committed by both sides
that these wars could not be called wars, they did not accord
with the theory, and therefore could not serve as material for
science.

In 1806 Pfuel had been one of those responsible, for the plan of
campaign that ended in Jena and Auerstadt, but he did not see the
least proof of the fallibility of his theory in the disasters of
that war. On the contrary, the deviations made from his theory
were, in his opinion, the sole cause of the whole disaster, and
with characteristically gleeful sarcasm he would remark, ``There,
I said the whole affair would go to the devil!'' Pfuel was one of
those theoreticians who so love their theory that they lose sight
of the theory's object---its practical application. His love of
theory made him hate everything practical, and he would not
listen to it. He was even pleased by failures, for failures
resulting from deviations in practice from the theory only proved
to him the accuracy of his theory.

He said a few words to Prince Andrew and Chernyshev about the
present war, with the air of a man who knows beforehand that all
will go wrong, and who is not displeased that it should be
so. The unbrushed tufts of hair sticking up behind and the
hastily brushed hair on his temples expressed this most
eloquently.

He passed into the next room, and the deep, querulous sounds of
his voice were at once heard from there.

% % % % % % % % % % % % % % % % % % % % % % % % % % % % % % % % %
% % % % % % % % % % % % % % % % % % % % % % % % % % % % % % % % %
% % % % % % % % % % % % % % % % % % % % % % % % % % % % % % % % %
% % % % % % % % % % % % % % % % % % % % % % % % % % % % % % % % %
% % % % % % % % % % % % % % % % % % % % % % % % % % % % % % % % %
% % % % % % % % % % % % % % % % % % % % % % % % % % % % % % % % %
% % % % % % % % % % % % % % % % % % % % % % % % % % % % % % % % %
% % % % % % % % % % % % % % % % % % % % % % % % % % % % % % % % %
% % % % % % % % % % % % % % % % % % % % % % % % % % % % % % % % %
% % % % % % % % % % % % % % % % % % % % % % % % % % % % % % % % %
% % % % % % % % % % % % % % % % % % % % % % % % % % % % % % % % %
% % % % % % % % % % % % % % % % % % % % % % % % % % % % % %

\chapter*{Chapter XI}
\ifaudio     
\marginpar{
\href{http://ia802702.us.archive.org/23/items/war_and_peace_09_0811_librivox/war_and_peace_09_11_tolstoy_64kb.mp3}{Audio}} 
\fi

\initial{P}{rince} Andrew's eyes were still following Pfuel out of the room
when Count Bennigsen entered hurriedly, and nodding to Bolkonski,
but not pausing, went into the study, giving instructions to his
adjutant as he went. The Emperor was following him, and Bennigsen
had hastened on to make some preparations and to be ready to
receive the sovereign.  Chernyshev and Prince Andrew went out
into the porch, where the Emperor, who looked fatigued, was
dismounting. Marquis Paulucci was talking to him with particular
warmth and the Emperor, with his head bent to the left, was
listening with a dissatisfied air. The Emperor moved forward
evidently wishing to end the conversation, but the flushed and
excited Italian, oblivious of decorum, followed him and continued
to speak.

``And as for the man who advised forming this camp---the Drissa
camp,'' said Paulucci, as the Emperor mounted the steps and
noticing Prince Andrew scanned his unfamiliar face, ``as to that
person, sire...''  continued Paulucci, desperately, apparently
unable to restrain himself, ``the man who advised the Drissa
camp---I see no alternative but the lunatic asylum or the
gallows!''

Without heeding the end of the Italian's remarks, and as though
not hearing them, the Emperor, recognizing Bolkonski, addressed
him graciously.

``I am very glad to see you! Go in there where they are meeting,
and wait for me.''

The Emperor went into the study. He was followed by Prince Peter
Mikhaylovich Volkonski and Baron Stein, and the door closed
behind them.  Prince Andrew, taking advantage of the Emperor's
permission, accompanied Paulucci, whom he had known in Turkey,
into the drawing room where the council was assembled.

Prince Peter Mikhaylovich Volkonski occupied the position, as it
were, of chief of the Emperor's staff. He came out of the study
into the drawing room with some maps which he spread on a table,
and put questions on which he wished to hear the opinion of the
gentlemen present. What had happened was that news (which
afterwards proved to be false) had been received during the night
of a movement by the French to outflank the Drissa camp.

The first to speak was General Armfeldt who, to meet the
difficulty that presented itself, unexpectedly proposed a
perfectly new position away from the Petersburg and Moscow
roads. The reason for this was inexplicable (unless he wished to
show that he, too, could have an opinion), but he urged that at
this point the army should unite and there await the enemy. It
was plain that Armfeldt had thought out that plan long ago and
now expounded it not so much to answer the questions put---which,
in fact, his plan did not answer---as to avail himself of the
opportunity to air it. It was one of the millions of proposals,
one as good as another, that could be made as long as it was
quite unknown what character the war would take. Some disputed
his arguments, others defended them. Young Count Toll objected to
the Swedish general's views more warmly than anyone else, and in
the course of the dispute drew from his side pocket a well-filled
notebook, which he asked permission to read to them. In these
voluminous notes Toll suggested another scheme, totally different
from Armfeldt's or Pfuel's plan of campaign. In answer to Toll,
Paulucci suggested an advance and an attack, which, he urged,
could alone extricate us from the present uncertainty and from
the trap (as he called the Drissa camp) in which we were
situated.

During all these discussions Pfuel and his interpreter, Wolzogen
(his ``bridge'' in court relations), were silent. Pfuel only
snorted contemptuously and turned away, to show that he would
never demean himself by replying to such nonsense as he was now
hearing. So when Prince Volkonski, who was in the chair, called
on him to give his opinion, he merely said:

``Why ask me? General Armfeldt has proposed a splendid position
with an exposed rear, or why not this Italian gentleman's
attack---very fine, or a retreat, also good! Why ask me?'' said
he. ``Why, you yourselves know everything better than I do.''

But when Volkonski said, with a frown, that it was in the
Emperor's name that he asked his opinion, Pfuel rose and,
suddenly growing animated, began to speak:

``Everything has been spoiled, everything muddled, everybody
thought they knew better than I did, and now you come to me! How
mend matters? There is nothing to mend! The principles laid down
by me must be strictly adhered to,'' said he, drumming on the
table with his bony fingers. ``What is the difficulty? Nonsense,
childishness!''

He went up to the map and speaking rapidly began proving that no
eventuality could alter the efficiency of the Drissa camp, that
everything had been foreseen, and that if the enemy were really
going to outflank it, the enemy would inevitably be destroyed.

Paulucci, who did not know German, began questioning him in
French.  Wolzogen came to the assistance of his chief, who spoke
French badly, and began translating for him, hardly able to keep
pace with Pfuel, who was rapidly demonstrating that not only all
that had happened, but all that could happen, had been foreseen
in his scheme, and that if there were now any difficulties the
whole fault lay in the fact that his plan had not been precisely
executed. He kept laughing sarcastically, he demonstrated, and at
last contemptuously ceased to demonstrate, like a mathematician
who ceases to prove in various ways the accuracy of a problem
that has already been proved. Wolzogen took his place and
continued to explain his views in French, every now and then
turning to Pfuel and saying, ``Is it not so, your excellency?''
But Pfuel, like a man heated in a fight who strikes those on his
own side, shouted angrily at his own supporter, Wolzogen:

``Well, of course, what more is there to explain?''

Paulucci and Michaud both attacked Wolzogen simultaneously in
French.  Armfeldt addressed Pfuel in German. Toll explained to
Volkonski in Russian. Prince Andrew listened and observed in
silence.

Of all these men Prince Andrew sympathized most with Pfuel,
angry, determined, and absurdly self-confident as he was. Of all
those present, evidently he alone was not seeking anything for
himself, nursed no hatred against anyone, and only desired that
the plan, formed on a theory arrived at by years of toil, should
be carried out. He was ridiculous, and unpleasantly sarcastic,
but yet he inspired involuntary respect by his boundless devotion
to an idea. Besides this, the remarks of all except Pfuel had one
common trait that had not been noticeable at the council of war
in 1805: there was now a panic fear of Napoleon's genius, which,
though concealed, was noticeable in every rejoinder.  Everything
was assumed to be possible for Napoleon, they expected him from
every side, and invoked his terrible name to shatter each other's
proposals. Pfuel alone seemed to consider Napoleon a barbarian
like everyone else who opposed his theory. But besides this
feeling of respect, Pfuel evoked pity in Prince Andrew. From the
tone in which the courtiers addressed him and the way Paulucci
had allowed himself to speak of him to the Emperor, but above all
from a certain desperation in Pfuel's own expressions, it was
clear that the others knew, and Pfuel himself felt, that his fall
was at hand. And despite his self-confidence and grumpy German
sarcasm he was pitiable, with his hair smoothly brushed on the
temples and sticking up in tufts behind. Though he concealed the
fact under a show of irritation and contempt, he was evidently in
despair that the sole remaining chance of verifying his theory by
a huge experiment and proving its soundness to the whole world
was slipping away from him.

The discussions continued a long time, and the longer they lasted
the more heated became the disputes, culminating in shouts and
personalities, and the less was it possible to arrive at any
general conclusion from all that had been said. Prince Andrew,
listening to this polyglot talk and to these surmises, plans,
refutations, and shouts, felt nothing but amazement at what they
were saying. A thought that had long since and often occurred to
him during his military activities---the idea that there is not
and cannot be any science of war, and that therefore there can be
no such thing as a military genius---now appeared to him an
obvious truth. ``What theory and science is possible about a
matter the conditions and circumstances of which are unknown and
cannot be defined, especially when the strength of the acting
forces cannot be ascertained? No one was or is able to foresee in
what condition our or the enemy's armies will be in a day's time,
and no one can gauge the force of this or that
detachment. Sometimes---when there is not a coward at the front
to shout, 'We are cut off!' and start running, but a brave and
jolly lad who shouts, 'Hurrah!'---a detachment of five thousand
is worth thirty thousand, as at Schon Grabern, while at times
fifty thousand run from eight thousand, as at Austerlitz. What
science can there be in a matter in which, as in all practical
matters, nothing can be defined and everything depends on
innumerable conditions, the significance of which is determined
at a particular moment which arrives no one knows when? Armfeldt
says our army is cut in half, and Paulucci says we have got the
French army between two fires; Michaud says that the
worthlessness of the Drissa camp lies in having the river behind
it, and Pfuel says that is what constitutes its strength; Toll
proposes one plan, Armfeldt another, and they are all good and
all bad, and the advantages of any suggestions can be seen only
at the moment of trial.  And why do they all speak of a 'military
genius'? Is a man a genius who can order bread to be brought up
at the right time and say who is to go to the right and who to
the left? It is only because military men are invested with pomp
and power and crowds of sychophants flatter power, attributing to
it qualities of genius it does not possess. The best generals I
have known were, on the contrary, stupid or absent-minded
men. Bagration was the best, Napoleon himself admitted that. And
of Bonaparte himself! I remember his limited, self-satisfied face
on the field of Austerlitz. Not only does a good army commander
not need any special qualities, on the contrary he needs the
absence of the highest and best human attributes---love, poetry,
tenderness, and philosophic inquiring doubt. He should be
limited, firmly convinced that what he is doing is very important
(otherwise he will not have sufficient patience), and only then
will he be a brave leader. God forbid that he should be humane,
should love, or pity, or think of what is just and unjust. It is
understandable that a theory of their 'genius' was invented for
them long ago because they have power! The success of a military
action depends not on them, but on the man in the ranks who
shouts, 'We are lost!' or who shouts, 'Hurrah!' And only in the
ranks can one serve with assurance of being useful.''

So thought Prince Andrew as he listened to the talking, and he
roused himself only when Paulucci called him and everyone was
leaving.

At the review next day the Emperor asked Prince Andrew where he
would like to serve, and Prince Andrew lost his standing in court
circles forever by not asking to remain attached to the
sovereign's person, but for permission to serve in the army.

% % % % % % % % % % % % % % % % % % % % % % % % % % % % % % % % %
% % % % % % % % % % % % % % % % % % % % % % % % % % % % % % % % %
% % % % % % % % % % % % % % % % % % % % % % % % % % % % % % % % %
% % % % % % % % % % % % % % % % % % % % % % % % % % % % % % % % %
% % % % % % % % % % % % % % % % % % % % % % % % % % % % % % % % %
% % % % % % % % % % % % % % % % % % % % % % % % % % % % % % % % %
% % % % % % % % % % % % % % % % % % % % % % % % % % % % % % % % %
% % % % % % % % % % % % % % % % % % % % % % % % % % % % % % % % %
% % % % % % % % % % % % % % % % % % % % % % % % % % % % % % % % %
% % % % % % % % % % % % % % % % % % % % % % % % % % % % % % % % %
% % % % % % % % % % % % % % % % % % % % % % % % % % % % % % % % %
% % % % % % % % % % % % % % % % % % % % % % % % % % % % % %

\chapter*{Chapter XII}
\ifaudio     
\marginpar{
\href{http://ia802702.us.archive.org/23/items/war_and_peace_09_0811_librivox/war_and_peace_09_12_tolstoy_64kb.mp3}{Audio}} 
\fi

\initial{B}{efore} the beginning of the campaign, Rostov had received a
letter from his parents in which they told him briefly of
Natasha's illness and the breaking off of her engagement to
Prince Andrew (which they explained by Natasha's having rejected
him) and again asked Nicholas to retire from the army and return
home. On receiving this letter, Nicholas did not even make any
attempt to get leave of absence or to retire from the army, but
wrote to his parents that he was sorry Natasha was ill and her
engagement broken off, and that he would do all he could to meet
their wishes. To Sonya he wrote separately.

``Adored friend of my soul!'' he wrote. ``Nothing but honor could
keep me from returning to the country. But now, at the
commencement of the campaign, I should feel dishonored, not only
in my comrades' eyes but in my own, if I preferred my own
happiness to my love and duty to the Fatherland. But this shall
be our last separation. Believe me, directly the war is over, if
I am still alive and still loved by you, I will throw up
everything and fly to you, to press you forever to my ardent
breast.''

It was, in fact, only the commencement of the campaign that
prevented Rostov from returning home as he had promised and
marrying Sonya. The autumn in Otradnoe with the hunting, and the
winter with the Christmas holidays and Sonya's love, had opened
out to him a vista of tranquil rural joys and peace such as he
had never known before, and which now allured him. ``A splendid
wife, children, a good pack of hounds, a dozen leashes of smart
borzois, agriculture, neighbors, service by election...'' thought
he. But now the campaign was beginning, and he had to remain with
his regiment. And since it had to be so, Nicholas Rostov, as was
natural to him, felt contented with the life he led in the
regiment and was able to find pleasure in that life.

On his return from his furlough Nicholas, having been joyfully
welcomed by his comrades, was sent to obtain remounts and brought
back from the Ukraine excellent horses which pleased him and
earned him commendation from his commanders. During his absence
he had been promoted captain, and when the regiment was put on
war footing with an increase in numbers, he was again allotted
his old squadron.

The campaign began, the regiment was moved into Poland on double
pay, new officers arrived, new men and horses, and above all
everybody was infected with the merrily excited mood that goes
with the commencement of a war, and Rostov, conscious of his
advantageous position in the regiment, devoted himself entirely
to the pleasures and interests of military service, though he
knew that sooner or later he would have to relinquish them.

The troops retired from Vilna for various complicated reasons of
state, political and strategic. Each step of the retreat was
accompanied by a complicated interplay of interests, arguments,
and passions at headquarters. For the Pavlograd hussars, however,
the whole of this retreat during the finest period of summer and
with sufficient supplies was a very simple and agreeable
business.

It was only at headquarters that there was depression,
uneasiness, and intriguing; in the body of the army they did not
ask themselves where they were going or why. If they regretted
having to retreat, it was only because they had to leave billets
they had grown accustomed to, or some pretty young Polish
lady. If the thought that things looked bad chanced to enter
anyone's head, he tried to be as cheerful as befits a good
soldier and not to think of the general trend of affairs, but
only of the task nearest to hand. First they camped gaily before
Vilna, making acquaintance with the Polish landowners, preparing
for reviews and being reviewed by the Emperor and other high
commanders. Then came an order to retreat to Sventsyani and
destroy any provisions they could not carry away with
them. Sventsyani was remembered by the hussars only as the
drunken camp, a name the whole army gave to their encampment
there, and because many complaints were made against the troops,
who, taking advantage of the order to collect provisions, took
also horses, carriages, and carpets from the Polish
proprietors. Rostov remembered Sventsyani, because on the first
day of their arrival at that small town he changed his sergeant
major and was unable to manage all the drunken men of his
squadron who, unknown to him, had appropriated five barrels of
old beer. From Sventsyani they retired farther and farther to
Drissa, and thence again beyond Drissa, drawing near to the
frontier of Russia proper.

On the thirteenth of July the Pavlograds took part in a serious
action for the first time.

On the twelfth of July, on the eve of that action, there was a
heavy storm of rain and hail. In general, the summer of 1812 was
remarkable for its storms.

The two Pavlograd squadrons were bivouacking on a field of rye,
which was already in ear but had been completely trodden down by
cattle and horses. The rain was descending in torrents, and
Rostov, with a young officer named Ilyin, his protege, was
sitting in a hastily constructed shelter. An officer of their
regiment, with long mustaches extending onto his cheeks, who
after riding to the staff had been overtaken by the rain, entered
Rostov's shelter.

``I have come from the staff, Count. Have you heard of Raevski's
exploit?''

And the officer gave them details of the Saltanov battle, which
he had heard at the staff.

Rostov, smoking his pipe and turning his head about as the water
trickled down his neck, listened inattentively, with an
occasional glance at Ilyin, who was pressing close to him. This
officer, a lad of sixteen who had recently joined the regiment,
was now in the same relation to Nicholas that Nicholas had been
to Denisov seven years before. Ilyin tried to imitate Rostov in
everything and adored him as a girl might have done.

Zdrzhinski, the officer with the long mustache, spoke
grandiloquently of the Saltanov dam being ``a Russian
Thermopylae,'' and of how a deed worthy of antiquity had been
performed by General Raevski. He recounted how Raevski had led
his two sons onto the dam under terrific fire and had charged
with them beside him. Rostov heard the story and not only said
nothing to encourage Zdrzhinski's enthusiasm but, on the
contrary, looked like a man ashamed of what he was hearing,
though with no intention of contradicting it. Since the campaigns
of Austerlitz and of 1807 Rostov knew by experience that men
always lie when describing military exploits, as he himself had
done when recounting them; besides that, he had experience enough
to know that nothing happens in war at all as we can imagine or
relate it. And so he did not like Zdrzhinski's tale, nor did he
like Zdrzhinski himself who, with his mustaches extending over
his cheeks, bent low over the face of his hearer, as was his
habit, and crowded Rostov in the narrow shanty. Rostov looked at
him in silence. ``In the first place, there must have been such a
confusion and crowding on the dam that was being attacked that if
Raevski did lead his sons there, it could have had no effect
except perhaps on some dozen men nearest to him,'' thought he,
``the rest could not have seen how or with whom Raevski came onto
the dam. And even those who did see it would not have been much
stimulated by it, for what had they to do with Raevski's tender
paternal feelings when their own skins were in danger?  And
besides, the fate of the Fatherland did not depend on whether
they took the Saltanov dam or not, as we are told was the case at
Thermopylae. So why should he have made such a sacrifice? And why
expose his own children in the battle? I would not have taken my
brother Petya there, or even Ilyin, who's a stranger to me but a
nice lad, but would have tried to put them somewhere under
cover,'' Nicholas continued to think, as he listened to
Zdrzhinski. But he did not express his thoughts, for in such
matters, too, he had gained experience. He knew that this tale
redounded to the glory of our arms and so one had to pretend not
to doubt it. And he acted accordingly.

``I can't stand this any more,'' said Ilyin, noticing that Rostov
did not relish Zdrzhinski's conversation. ``My stockings and
shirt... and the water is running on my seat! I'll go and look
for shelter. The rain seems less heavy.''

Ilyin went out and Zdrzhinski rode away.

Five minutes later Ilyin, splashing through the mud, came running
back to the shanty.

``Hurrah! Rostov, come quick! I've found it! About two hundred
yards away there's a tavern where ours have already gathered. We
can at least get dry there, and Mary Hendrikhovna's there.''

Mary Hendrikhovna was the wife of the regimental doctor, a pretty
young German woman he had married in Poland. The doctor, whether
from lack of means or because he did not like to part from his
young wife in the early days of their marriage, took her about
with him wherever the hussar regiment went and his jealousy had
become a standing joke among the hussar officers.

Rostov threw his cloak over his shoulders, shouted to Lavrushka
to follow with the things, and---now slipping in the mud, now
splashing right through it---set off with Ilyin in the lessening
rain and the darkness that was occasionally rent by distant
lightning.

``Rostov, where are you?''

``Here. What lightning!'' they called to one another.

% % % % % % % % % % % % % % % % % % % % % % % % % % % % % % % % %
% % % % % % % % % % % % % % % % % % % % % % % % % % % % % % % % %
% % % % % % % % % % % % % % % % % % % % % % % % % % % % % % % % %
% % % % % % % % % % % % % % % % % % % % % % % % % % % % % % % % %
% % % % % % % % % % % % % % % % % % % % % % % % % % % % % % % % %
% % % % % % % % % % % % % % % % % % % % % % % % % % % % % % % % %
% % % % % % % % % % % % % % % % % % % % % % % % % % % % % % % % %
% % % % % % % % % % % % % % % % % % % % % % % % % % % % % % % % %
% % % % % % % % % % % % % % % % % % % % % % % % % % % % % % % % %
% % % % % % % % % % % % % % % % % % % % % % % % % % % % % % % % %
% % % % % % % % % % % % % % % % % % % % % % % % % % % % % % % % %
% % % % % % % % % % % % % % % % % % % % % % % % % % % % % %

\chapter*{Chapter XIII}
\ifaudio     
\marginpar{
\href{http://ia802702.us.archive.org/23/items/war_and_peace_09_0811_librivox/war_and_peace_09_13_tolstoy_64kb.mp3}{Audio}} 
\fi

\initial{I}{n} the tavern, before which stood the doctor's covered cart,
there were already some five officers. Mary Hendrikhovna, a plump
little blonde German, in a dressing jacket and nightcap, was
sitting on a broad bench in the front corner. Her husband, the
doctor, lay asleep behind her.  Rostov and Ilyin, on entering the
room, were welcomed with merry shouts and laughter.

``Dear me, how jolly we are!'' said Rostov laughing.

``And why do you stand there gaping?''

``What swells they are! Why, the water streams from them! Don't
make our drawing room so wet.''

``Don't mess Mary Hendrikhovna's dress!'' cried other voices.

Rostov and Ilyin hastened to find a corner where they could
change into dry clothes without offending Mary Hendrikhovna's
modesty. They were going into a tiny recess behind a partition to
change, but found it completely filled by three officers who sat
playing cards by the light of a solitary candle on an empty box,
and these officers would on no account yield their position. Mary
Hendrikhovna obliged them with the loan of a petticoat to be used
as a curtain, and behind that screen Rostov and Ilyin, helped by
Lavrushka who had brought their kits, changed their wet things
for dry ones.

A fire was made up in the dilapidated brick stove. A board was
found, fixed on two saddles and covered with a horsecloth, a
small samovar was produced and a cellaret and half a bottle of
rum, and having asked Mary Hendrikhovna to preside, they all
crowded round her. One offered her a clean handkerchief to wipe
her charming hands, another spread a jacket under her little feet
to keep them from the damp, another hung his coat over the window
to keep out the draft, and yet another waved the flies off her
husband's face, lest he should wake up.

``Leave him alone,'' said Mary Hendrikhovna, smiling timidly and
happily.  ``He is sleeping well as it is, after a sleepless
night.''

``Oh, no, Mary Hendrikhovna,'' replied the officer, ``one must
look after the doctor. Perhaps he'll take pity on me someday,
when it comes to cutting off a leg or an arm for me.''

There were only three tumblers, the water was so muddy that one
could not make out whether the tea was strong or weak, and the
samovar held only six tumblers of water, but this made it all the
pleasanter to take turns in order of seniority to receive one's
tumbler from Mary Hendrikhovna's plump little hands with their
short and not overclean nails. All the officers appeared to be,
and really were, in love with her that evening. Even those
playing cards behind the partition soon left their game and came
over to the samovar, yielding to the general mood of courting
Mary Hendrikhovna. She, seeing herself surrounded by such
brilliant and polite young men, beamed with satisfaction, try as
she might to hide it, and perturbed as she evidently was each
time her husband moved in his sleep behind her.

There was only one spoon, sugar was more plentiful than anything
else, but it took too long to dissolve, so it was decided that
Mary Hendrikhovna should stir the sugar for everyone in
turn. Rostov received his tumbler, and adding some rum to it
asked Mary Hendrikhovna to stir it.

``But you take it without sugar?'' she said, smiling all the
time, as if everything she said and everything the others said
was very amusing and had a double meaning.

``It is not the sugar I want, but only that your little hand
should stir my tea.''

Mary Hendrikhovna assented and began looking for the spoon which
someone meanwhile had pounced on.

``Use your finger, Mary Hendrikhovna, it will be still nicer,''
said Rostov.

``Too hot!'' she replied, blushing with pleasure.

Ilyin put a few drops of rum into the bucket of water and brought
it to Mary Hendrikhovna, asking her to stir it with her finger.

``This is my cup,'' said he. ``Only dip your finger in it and
I'll drink it all up.''

When they had emptied the samovar, Rostov took a pack of cards
and proposed that they should play \emph{Kings} with Mary
Hendrikhovna. They drew lots to settle who should make up her
set. At Rostov's suggestion it was agreed that whoever became
\emph{King} should have the right to kiss Mary Hendrikhovna's hand,
and that the \emph{Booby} should go to refill and reheat the samovar
for the doctor when the latter awoke.

``Well, but supposing Mary Hendrikhovna is \emph{King}?'' asked Ilyin.

``As it is, she is Queen, and her word is law!''

They had hardly begun to play before the doctor's disheveled head
suddenly appeared from behind Mary Hendrikhovna. He had been
awake for some time, listening to what was being said, and
evidently found nothing entertaining or amusing in what was going
on. His face was sad and depressed. Without greeting the
officers, he scratched himself and asked to be allowed to pass as
they were blocking the way. As soon as he had left the room all
the officers burst into loud laughter and Mary Hendrikhovna
blushed till her eyes filled with tears and thereby became still
more attractive to them. Returning from the yard, the doctor told
his wife (who had ceased to smile so happily, and looked at him
in alarm, awaiting her sentence) that the rain had ceased and
they must go to sleep in their covered cart, or everything in it
would be stolen.

``But I'll send an orderly... Two of them!'' said Rostov. ``What
an idea, doctor!''

``I'll stand guard on it myself!'' said Ilyin.

``No, gentlemen, you have had your sleep, but I have not slept
for two nights,'' replied the doctor, and he sat down morosely
beside his wife, waiting for the game to end.

Seeing his gloomy face as he frowned at his wife, the officers
grew still merrier, and some of them could not refrain from
laughter, for which they hurriedly sought plausible
pretexts. When he had gone, taking his wife with him, and had
settled down with her in their covered cart, the officers lay
down in the tavern, covering themselves with their wet cloaks,
but they did not sleep for a long time; now they exchanged
remarks, recalling the doctor's uneasiness and his wife's
delight, now they ran out into the porch and reported what was
taking place in the covered trap. Several times Rostov, covering
his head, tried to go to sleep, but some remark would arouse him
and conversation would be resumed, to the accompaniment of
unreasoning, merry, childlike laughter.

% % % % % % % % % % % % % % % % % % % % % % % % % % % % % % % % %
% % % % % % % % % % % % % % % % % % % % % % % % % % % % % % % % %
% % % % % % % % % % % % % % % % % % % % % % % % % % % % % % % % %
% % % % % % % % % % % % % % % % % % % % % % % % % % % % % % % % %
% % % % % % % % % % % % % % % % % % % % % % % % % % % % % % % % %
% % % % % % % % % % % % % % % % % % % % % % % % % % % % % % % % %
% % % % % % % % % % % % % % % % % % % % % % % % % % % % % % % % %
% % % % % % % % % % % % % % % % % % % % % % % % % % % % % % % % %
% % % % % % % % % % % % % % % % % % % % % % % % % % % % % % % % %
% % % % % % % % % % % % % % % % % % % % % % % % % % % % % % % % %
% % % % % % % % % % % % % % % % % % % % % % % % % % % % % % % % %
% % % % % % % % % % % % % % % % % % % % % % % % % % % % % %

\chapter*{Chapter XIV}
\ifaudio     
\marginpar{
\href{http://ia802702.us.archive.org/23/items/war_and_peace_09_0811_librivox/war_and_peace_09_14_tolstoy_64kb.mp3}{Audio}} 
\fi

\initial{I}{t} was nearly three o'clock but no one was yet asleep, when the
quartermaster appeared with an order to move on to the little
town of Ostrovna. Still laughing and talking, the officers began
hurriedly getting ready and again boiled some muddy water in the
samovar. But Rostov went off to his squadron without waiting for
tea. Day was breaking, the rain had ceased, and the clouds were
dispersing. It felt damp and cold, especially in clothes that
were still moist. As they left the tavern in the twilight of the
dawn, Rostov and Ilyin both glanced under the wet and glistening
leather hood of the doctor's cart, from under the apron of which
his feet were sticking out, and in the middle of which his wife's
nightcap was visible and her sleepy breathing audible.

``She really is a dear little thing,'' said Rostov to Ilyin, who
was following him.

``A charming woman!'' said Ilyin, with all the gravity of a boy
of sixteen.

Half an hour later the squadron was lined up on the road. The
command was heard to \emph{mount} and the soldiers crossed
themselves and mounted.  Rostov riding in front gave the order
``Forward!'' and the hussars, with clanking sabers and subdued
talk, their horses' hoofs splashing in the mud, defiled in fours
and moved along the broad road planted with birch trees on each
side, following the infantry and a battery that had gone on in
front.

Tattered, blue-purple clouds, reddening in the east, were
scudding before the wind. It was growing lighter and
lighter. That curly grass which always grows by country roadsides
became clearly visible, still wet with the night's rain; the
drooping branches of the birches, also wet, swayed in the wind
and flung down bright drops of water to one side. The soldiers'
faces were more and more clearly visible. Rostov, always closely
followed by Ilyin, rode along the side of the road between two
rows of birch trees.

When campaigning, Rostov allowed himself the indulgence of riding
not a regimental but a Cossack horse. A judge of horses and a
sportsman, he had lately procured himself a large, fine,
mettlesome, Donets horse, dun-colored, with light mane and tail,
and when he rode it no one could outgallop him. To ride this
horse was a pleasure to him, and he thought of the horse, of the
morning, of the doctor's wife, but not once of the impending
danger.

Formerly, when going into action, Rostov had felt afraid; now he
had not the least feeling of fear. He was fearless, not because
he had grown used to being under fire (one cannot grow used to
danger), but because he had learned how to manage his thoughts
when in danger. He had grown accustomed when going into action to
think about anything but what would seem most likely to interest
him---the impending danger. During the first period of his
service, hard as he tried and much as he reproached himself with
cowardice, he had not been able to do this, but with time it had
come of itself. Now he rode beside Ilyin under the birch trees,
occasionally plucking leaves from a branch that met his hand,
sometimes touching his horse's side with his foot, or, without
turning round, handing a pipe he had finished to an hussar riding
behind him, with as calm and careless an air as though he were
merely out for a ride. He glanced with pity at the excited face
of Ilyin, who talked much and in great agitation. He knew from
experience the tormenting expectation of terror and death the
cornet was suffering and knew that only time could help him.

As soon as the sun appeared in a clear strip of sky beneath the
clouds, the wind fell, as if it dared not spoil the beauty of the
summer morning after the storm; drops still continued to fall,
but vertically now, and all was still. The whole sun appeared on
the horizon and disappeared behind a long narrow cloud that hung
above it. A few minutes later it reappeared brighter still from
behind the top of the cloud, tearing its edge. Everything grew
bright and glittered. And with that light, and as if in reply to
it, came the sound of guns ahead of them.

Before Rostov had had time to consider and determine the distance
of that firing, Count Ostermann-Tolstoy's adjutant came galloping
from Vitebsk with orders to advance at a trot along the road.

The squadron overtook and passed the infantry and the
battery---which had also quickened their pace---rode down a hill,
and passing through an empty and deserted village again
ascended. The horses began to lather and the men to flush.

``Halt! Dress your ranks!'' the order of the regimental commander
was heard ahead. ``Forward by the left. Walk, march!'' came the
order from in front.

And the hussars, passing along the line of troops on the left
flank of our position, halted behind our uhlans who were in the
front line. To the right stood our infantry in a dense column:
they were the reserve.  Higher up the hill, on the very horizon,
our guns were visible through the wonderfully clear air, brightly
illuminated by slanting morning sunbeams. In front, beyond a
hollow dale, could be seen the enemy's columns and guns. Our
advanced line, already in action, could be heard briskly
exchanging shots with the enemy in the dale.

At these sounds, long unheard, Rostov's spirits rose, as at the
strains of the merriest music. Trap-ta-ta-tap! cracked the shots,
now together, now several quickly one after another. Again all
was silent and then again it sounded as if someone were walking
on detonators and exploding them.

The hussars remained in the same place for about an hour. A
cannonade began. Count Ostermann with his suite rode up behind
the squadron, halted, spoke to the commander of the regiment, and
rode up the hill to the guns.

After Ostermann had gone, a command rang out to the uhlans.

``Form column! Prepare to charge!''

The infantry in front of them parted into platoons to allow the
cavalry to pass. The uhlans started, the streamers on their
spears fluttering, and trotted downhill toward the French cavalry
which was seen below to the left.

As soon as the uhlans descended the hill, the hussars were
ordered up the hill to support the battery. As they took the
places vacated by the uhlans, bullets came from the front,
whining and whistling, but fell spent without taking effect.

The sounds, which he had not heard for so long, had an even more
pleasurable and exhilarating effect on Rostov than the previous
sounds of firing. Drawing himself up, he viewed the field of
battle opening out before him from the hill, and with his whole
soul followed the movement of the uhlans. They swooped down close
to the French dragoons, something confused happened there amid
the smoke, and five minutes later our uhlans were galloping back,
not to the place they had occupied but more to the left, and
among the orange-colored uhlans on chestnut horses and behind
them, in a large group, blue French dragoons on gray horses could
be seen.

% % % % % % % % % % % % % % % % % % % % % % % % % % % % % % % % %
% % % % % % % % % % % % % % % % % % % % % % % % % % % % % % % % %
% % % % % % % % % % % % % % % % % % % % % % % % % % % % % % % % %
% % % % % % % % % % % % % % % % % % % % % % % % % % % % % % % % %
% % % % % % % % % % % % % % % % % % % % % % % % % % % % % % % % %
% % % % % % % % % % % % % % % % % % % % % % % % % % % % % % % % %
% % % % % % % % % % % % % % % % % % % % % % % % % % % % % % % % %
% % % % % % % % % % % % % % % % % % % % % % % % % % % % % % % % %
% % % % % % % % % % % % % % % % % % % % % % % % % % % % % % % % %
% % % % % % % % % % % % % % % % % % % % % % % % % % % % % % % % %
% % % % % % % % % % % % % % % % % % % % % % % % % % % % % % % % %
% % % % % % % % % % % % % % % % % % % % % % % % % % % % % %

\chapter*{Chapter XV}
\ifaudio     
\marginpar{
\href{http://ia802702.us.archive.org/23/items/war_and_peace_09_0811_librivox/war_and_peace_09_15_tolstoy_64kb.mp3}{Audio}} 
\fi

\initial{R}{ostov}, with his keen sportsman's eye, was one of the first to
catch sight of these blue French dragoons pursuing our
uhlans. Nearer and nearer in disorderly crowds came the uhlans
and the French dragoons pursuing them. He could already see how
these men, who looked so small at the foot of the hill, jostled
and overtook one another, waving their arms and their sabers in
the air.

Rostov gazed at what was happening before him as at a hunt. He
felt instinctively that if the hussars struck at the French
dragoons now, the latter could not withstand them, but if a
charge was to be made it must be done now, at that very moment,
or it would be too late. He looked around. A captain, standing
beside him, was gazing like himself with eyes fixed on the
cavalry below them.

``Andrew Sevastyanych!'' said Rostov. ``You know, we could crush
them...''

``A fine thing too!'' replied the captain, ``and really...''

Rostov, without waiting to hear him out, touched his horse,
galloped to the front of his squadron, and before he had time to
finish giving the word of command, the whole squadron, sharing
his feeling, was following him. Rostov himself did not know how
or why he did it. He acted as he did when hunting, without
reflecting or considering. He saw the dragoons near and that they
were galloping in disorder; he knew they could not withstand an
attack---knew there was only that moment and that if he let it
slip it would not return. The bullets were whining and whistling
so stimulatingly around him and his horse was so eager to go that
he could not restrain himself. He touched his horse, gave the
word of command, and immediately, hearing behind him the tramp of
the horses of his deployed squadron, rode at full trot downhill
toward the dragoons.  Hardly had they reached the bottom of the
hill before their pace instinctively changed to a gallop, which
grew faster and faster as they drew nearer to our uhlans and the
French dragoons who galloped after them. The dragoons were now
close at hand. On seeing the hussars, the foremost began to turn,
while those behind began to halt. With the same feeling with
which he had galloped across the path of a wolf, Rostov gave rein
to his Donets horse and galloped to intersect the path of the
dragoons' disordered lines. One Uhlan stopped, another who was on
foot flung himself to the ground to avoid being knocked over, and
a riderless horse fell in among the hussars. Nearly all the
French dragoons were galloping back. Rostov, picking out one on a
gray horse, dashed after him. On the way he came upon a bush, his
gallant horse cleared it, and almost before he had righted
himself in his saddle he saw that he would immediately overtake
the enemy he had selected. That Frenchman, by his uniform an
officer, was going at a gallop, crouching on his gray horse and
urging it on with his saber. In another moment Rostov's horse
dashed its breast against the hindquarters of the officer's
horse, almost knocking it over, and at the same instant Rostov,
without knowing why, raised his saber and struck the Frenchman
with it.

The instant he had done this, all Rostov's animation
vanished. The officer fell, not so much from the blow---which had
but slightly cut his arm above the elbow---as from the shock to
his horse and from fright.  Rostov reined in his horse, and his
eyes sought his foe to see whom he had vanquished. The French
dragoon officer was hopping with one foot on the ground, the
other being caught in the stirrup. His eyes, screwed up with fear
as if he every moment expected another blow, gazed up at Rostov
with shrinking terror. His pale and mud-stained face---fair and
young, with a dimple in the chin and light-blue eyes---was not an
enemy's face at all suited to a battlefield, but a most ordinary,
homelike face.  Before Rostov had decided what to do with him,
the officer cried, ``I surrender!'' He hurriedly but vainly tried
to get his foot out of the stirrup and did not remove his
frightened blue eyes from Rostov's face.  Some hussars who
galloped up disengaged his foot and helped him into the
saddle. On all sides, the hussars were busy with the dragoons;
one was wounded, but though his face was bleeding, he would not
give up his horse; another was perched up behind an hussar with
his arms round him; a third was being helped by an hussar to
mount his horse. In front, the French infantry were firing as
they ran. The hussars galloped hastily back with their
prisoners. Rostov galloped back with the rest, aware of an
unpleasant feeling of depression in his heart. Something vague
and confused, which he could not at all account for, had come
over him with the capture of that officer and the blow he had
dealt him.

Count Ostermann-Tolstoy met the returning hussars, sent for
Rostov, thanked him, and said he would report his gallant deed to
the Emperor and would recommend him for a St. George's
Cross. When sent for by Count Ostermann, Rostov, remembering that
he had charged without orders, felt sure his commander was
sending for him to punish him for breach of
discipline. Ostermann's flattering words and promise of a reward
should therefore have struck him all the more pleasantly, but he
still felt that same vaguely disagreeable feeling of moral
nausea. ``But what on earth is worrying me?'' he asked himself as
he rode back from the general. ``Ilyin? No, he's safe. Have I
disgraced myself in any way? No, that's not it.'' Something else,
resembling remorse, tormented him. ``Yes, oh yes, that French
officer with the dimple. And I remember how my arm paused when I
raised it.''

Rostov saw the prisoners being led away and galloped after them
to have a look at his Frenchman with the dimple on his chin. He
was sitting in his foreign uniform on an hussar packhorse and
looked anxiously about him; The sword cut on his arm could
scarcely be called a wound. He glanced at Rostov with a feigned
smile and waved his hand in greeting.  Rostov still had the same
indefinite feeling, as of shame.

All that day and the next his friends and comrades noticed that
Rostov, without being dull or angry, was silent, thoughtful, and
preoccupied. He drank reluctantly, tried to remain alone, and
kept turning something over in his mind.

Rostov was always thinking about that brilliant exploit of his,
which to his amazement had gained him the St. George's Cross and
even given him a reputation for bravery, and there was something
he could not at all understand. ``So others are even more afraid
than I am!'' he thought. ``So that's all there is in what is
called heroism! And did I do it for my country's sake? And how
was he to blame, with his dimple and blue eyes?  And how
frightened he was! He thought that I should kill him. Why should
I kill him? My hand trembled. And they have given me a
St. George's Cross... I can't make it out at all.''

But while Nicholas was considering these questions and still
could reach no clear solution of what puzzled him so, the wheel
of fortune in the service, as often happens, turned in his
favor. After the affair at Ostrovna he was brought into notice,
received command of an hussar battalion, and when a brave officer
was needed he was chosen.

% % % % % % % % % % % % % % % % % % % % % % % % % % % % % % % % %
% % % % % % % % % % % % % % % % % % % % % % % % % % % % % % % % %
% % % % % % % % % % % % % % % % % % % % % % % % % % % % % % % % %
% % % % % % % % % % % % % % % % % % % % % % % % % % % % % % % % %
% % % % % % % % % % % % % % % % % % % % % % % % % % % % % % % % %
% % % % % % % % % % % % % % % % % % % % % % % % % % % % % % % % %
% % % % % % % % % % % % % % % % % % % % % % % % % % % % % % % % %
% % % % % % % % % % % % % % % % % % % % % % % % % % % % % % % % %
% % % % % % % % % % % % % % % % % % % % % % % % % % % % % % % % %
% % % % % % % % % % % % % % % % % % % % % % % % % % % % % % % % %
% % % % % % % % % % % % % % % % % % % % % % % % % % % % % % % % %
% % % % % % % % % % % % % % % % % % % % % % % % % % % % % %

\chapter*{Chapter XVI}
\ifaudio     
\marginpar{
\href{http://ia802702.us.archive.org/23/items/war_and_peace_09_0811_librivox/war_and_peace_09_16_tolstoy_64kb.mp3}{Audio}} 
\fi

\initial{O}{n} receiving news of Natasha's illness, the countess, though not
quite well yet and still weak, went to Moscow with Petya and the
rest of the household, and the whole family moved from Marya
Dmitrievna's house to their own and settled down in town.

Natasha's illness was so serious that, fortunately for her and
for her parents, the consideration of all that had caused the
illness, her conduct and the breaking off of her engagement,
receded into the background. She was so ill that it was
impossible for them to consider in how far she was to blame for
what had happened. She could not eat or sleep, grew visibly
thinner, coughed, and, as the doctors made them feel, was in
danger. They could not think of anything but how to help
her. Doctors came to see her singly and in consultation, talked
much in French, German, and Latin, blamed one another, and
prescribed a great variety of medicines for all the diseases
known to them, but the simple idea never occurred to any of them
that they could not know the disease Natasha was suffering from,
as no disease suffered by a live man can be known, for every
living person has his own peculiarities and always has his own
peculiar, personal, novel, complicated disease, unknown to
medicine---not a disease of the lungs, liver, skin, heart,
nerves, and so on mentioned in medical books, but a disease
consisting of one of the innumerable combinations of the maladies
of those organs. This simple thought could not occur to the
doctors (as it cannot occur to a wizard that he is unable to work
his charms) because the business of their lives was to cure, and
they received money for it and had spent the best years of their
lives on that business. But, above all, that thought was kept out
of their minds by the fact that they saw they were really useful,
as in fact they were to the whole Rostov family. Their usefulness
did not depend on making the patient swallow substances for the
most part harmful (the harm was scarcely perceptible, as they
were given in small doses), but they were useful, necessary, and
indispensable because they satisfied a mental need of the invalid
and of those who loved her---and that is why there are, and
always will be, pseudo-healers, wise women, homeopaths, and
allopaths. They satisfied that eternal human need for hope of
relief, for sympathy, and that something should be done, which is
felt by those who are suffering. They satisfied the need seen in
its most elementary form in a child, when it wants to have a
place rubbed that has been hurt. A child knocks itself and runs
at once to the arms of its mother or nurse to have the aching
spot rubbed or kissed, and it feels better when this is done. The
child cannot believe that the strongest and wisest of its people
have no remedy for its pain, and the hope of relief and the
expression of its mother's sympathy while she rubs the bump
comforts it. The doctors were of use to Natasha because they
kissed and rubbed her bump, assuring her that it would soon pass
if only the coachman went to the chemist's in the Arbat and got a
powder and some pills in a pretty box for a ruble and seventy
kopeks, and if she took those powders in boiled water at
intervals of precisely two hours, neither more nor less.

What would Sonya and the count and countess have done, how would
they have looked, if nothing had been done, if there had not been
those pills to give by the clock, the warm drinks, the chicken
cutlets, and all the other details of life ordered by the
doctors, the carrying out of which supplied an occupation and
consolation to the family circle? How would the count have borne
his dearly loved daughter's illness had he not known that it was
costing him a thousand rubles, and that he would not grudge
thousands more to benefit her, or had he not known that if her
illness continued he would not grudge yet other thousands and
would take her abroad for consultations there, and had he not
been able to explain the details of how Metivier and Feller had
not understood the symptoms, but Frise had, and Mudrov had
diagnosed them even better? What would the countess have done had
she not been able sometimes to scold the invalid for not strictly
obeying the doctor's orders?

``You'll never get well like that,'' she would say, forgetting
her grief in her vexation, ``if you won't obey the doctor and
take your medicine at the right time! You mustn't trifle with it,
you know, or it may turn to pneumonia,'' she would go on,
deriving much comfort from the utterance of that foreign word,
incomprehensible to others as well as to herself.

What would Sonya have done without the glad consciousness that
she had not undressed during the first three nights, in order to
be ready to carry out all the doctor's injunctions with
precision, and that she still kept awake at night so as not to
miss the proper time when the slightly harmful pills in the
little gilt box had to be administered?  Even to Natasha herself
it was pleasant to see that so many sacrifices were being made
for her sake, and to know that she had to take medicine at
certain hours, though she declared that no medicine would cure
her and that it was all nonsense. And it was even pleasant to be
able to show, by disregarding the orders, that she did not
believe in medical treatment and did not value her life.

The doctor came every day, felt her pulse, looked at her tongue,
and regardless of her grief-stricken face joked with her. But
when he had gone into another room, to which the countess
hurriedly followed him, he assumed a grave air and thoughtfully
shaking his head said that though there was danger, he had hopes
of the effect of this last medicine and one must wait and see,
that the malady was chiefly mental, but... And the countess,
trying to conceal the action from herself and from him, slipped a
gold coin into his hand and always returned to the patient with a
more tranquil mind.

The symptoms of Natasha's illness were that she ate little, slept
little, coughed, and was always low-spirited. The doctors said
that she could not get on without medical treatment, so they kept
her in the stifling atmosphere of the town, and the Rostovs did
not move to the country that summer of 1812.

In spite of the many pills she swallowed and the drops and
powders out of the little bottles and boxes of which Madame
Schoss who was fond of such things made a large collection, and
in spite of being deprived of the country life to which she was
accustomed, youth prevailed. Natasha's grief began to be overlaid
by the impressions of daily life, it ceased to press so painfully
on her heart, it gradually faded into the past, and she began to
recover physically.

% % % % % % % % % % % % % % % % % % % % % % % % % % % % % % % % %
% % % % % % % % % % % % % % % % % % % % % % % % % % % % % % % % %
% % % % % % % % % % % % % % % % % % % % % % % % % % % % % % % % %
% % % % % % % % % % % % % % % % % % % % % % % % % % % % % % % % %
% % % % % % % % % % % % % % % % % % % % % % % % % % % % % % % % %
% % % % % % % % % % % % % % % % % % % % % % % % % % % % % % % % %
% % % % % % % % % % % % % % % % % % % % % % % % % % % % % % % % %
% % % % % % % % % % % % % % % % % % % % % % % % % % % % % % % % %
% % % % % % % % % % % % % % % % % % % % % % % % % % % % % % % % %
% % % % % % % % % % % % % % % % % % % % % % % % % % % % % % % % %
% % % % % % % % % % % % % % % % % % % % % % % % % % % % % % % % %
% % % % % % % % % % % % % % % % % % % % % % % % % % % % % %

\chapter*{Chapter XVII}
\ifaudio     
\marginpar{
\href{http://ia802702.us.archive.org/23/items/war_and_peace_09_0811_librivox/war_and_peace_09_17_tolstoy_64kb.mp3}{Audio}} 
\fi

\initial{N}{atasha} was calmer but no happier. She not merely avoided all
external forms of pleasure---balls, promenades, concerts, and
theaters---but she never laughed without a sound of tears in her
laughter. She could not sing. As soon as she began to laugh, or
tried to sing by herself, tears choked her: tears of remorse,
tears at the recollection of those pure times which could never
return, tears of vexation that she should so uselessly have
ruined her young life which might have been so happy.  Laughter
and singing in particular seemed to her like a blasphemy, in face
of her sorrow. Without any need of self-restraint, no wish to
coquet ever entered her head. She said and felt at that time that
no man was more to her than Nastasya Ivanovna, the
buffoon. Something stood sentinel within her and forbade her
every joy. Besides, she had lost all the old interests of her
carefree girlish life that had been so full of hope. The previous
autumn, the hunting, \emph{Uncle}, and the Christmas holidays spent
with Nicholas at Otradnoe were what she recalled oftenest and
most painfully. What would she not have given to bring back even
a single day of that time! But it was gone forever. Her
presentiment at the time had not deceived her---that that state
of freedom and readiness for any enjoyment would not return
again. Yet it was necessary to live on.

It comforted her to reflect that she was not better as she had
formerly imagined, but worse, much worse, than anybody else in
the world. But this was not enough. She knew that, and asked
herself, ``What next?'' But there was nothing to come. There was
no joy in life, yet life was passing. Natasha apparently tried
not to be a burden or a hindrance to anyone, but wanted nothing
for herself. She kept away from everyone in the house and felt at
ease only with her brother Petya. She liked to be with him better
than with the others, and when alone with him she sometimes
laughed. She hardly ever left the house and of those who came to
see them was glad to see only one person, Pierre. It would have
been impossible to treat her with more delicacy, greater care,
and at the same time more seriously than did Count
Bezukhov. Natasha unconsciously felt this delicacy and so found
great pleasure in his society. But she was not even grateful to
him for it; nothing good on Pierre's part seemed to her to be an
effort, it seemed so natural for him to be kind to everyone that
there was no merit in his kindness. Sometimes Natasha noticed
embarrassment and awkwardness on his part in her presence,
especially when he wanted to do something to please her, or
feared that something they spoke of would awaken memories
distressing to her. She noticed this and attributed it to his
general kindness and shyness, which she imagined must be the same
toward everyone as it was to her.  After those involuntary
words---that if he were free he would have asked on his knees for
her hand and her love---uttered at a moment when she was so
strongly agitated, Pierre never spoke to Natasha of his feelings;
and it seemed plain to her that those words, which had then so
comforted her, were spoken as all sorts of meaningless words are
spoken to comfort a crying child. It was not because Pierre was a
married man, but because Natasha felt very strongly with him that
moral barrier the absence of which she had experienced with
Kuragin that it never entered her head that the relations between
him and herself could lead to love on her part, still less on
his, or even to the kind of tender, self-conscious, romantic
friendship between a man and a woman of which she had known
several instances.

Before the end of the fast of St. Peter, Agrafena Ivanovna
Belova, a country neighbor of the Rostovs, came to Moscow to pay
her devotions at the shrines of the Moscow saints. She suggested
that Natasha should fast and prepare for Holy Communion, and
Natasha gladly welcomed the idea.  Despite the doctor's orders
that she should not go out early in the morning, Natasha insisted
on fasting and preparing for the sacrament, not as they generally
prepared for it in the Rostov family by attending three services
in their own house, but as Agrafena Ivanovna did, by going to
church every day for a week and not once missing Vespers, Matins,
or Mass.

The countess was pleased with Natasha's zeal; after the poor
results of the medical treatment, in the depths of her heart she
hoped that prayer might help her daughter more than medicines
and, though not without fear and concealing it from the doctor,
she agreed to Natasha's wish and entrusted her to
Belova. Agrafena Ivanovna used to come to wake Natasha at three
in the morning, but generally found her already awake. She was
afraid of being late for Matins. Hastily washing, and meekly
putting on her shabbiest dress and an old mantilla, Natasha,
shivering in the fresh air, went out into the deserted streets
lit by the clear light of dawn.  By Agrafena Ivanovna's advice
Natasha prepared herself not in their own parish, but at a church
where, according to the devout Agrafena Ivanovna, the priest was
a man of very severe and lofty life. There were never many people
in the church; Natasha always stood beside Belova in the
customary place before an icon of the Blessed Virgin, let into
the screen before the choir on the left side, and a feeling, new
to her, of humility before something great and incomprehensible,
seized her when at that unusual morning hour, gazing at the dark
face of the Virgin illuminated by the candles burning before it
and by the morning light falling from the window, she listened to
the words of the service which she tried to follow with
understanding. When she understood them her personal feeling
became interwoven in the prayers with shades of its own. When she
did not understand, it was sweeter still to think that the wish
to understand everything is pride, that it is impossible to
understand all, that it is only necessary to believe and to
commit oneself to God, whom she felt guiding her soul at those
moments. She crossed herself, bowed low, and when she did not
understand, in horror at her own vileness, simply asked God to
forgive her everything, everything, to have mercy upon her. The
prayers to which she surrendered herself most of all were those
of repentance. On her way home at an early hour when she met no
one but bricklayers going to work or men sweeping the street, and
everybody within the houses was still asleep, Natasha experienced
a feeling new to her, a sense of the possibility of correcting
her faults, the possibility of a new, clean life, and of
happiness.

During the whole week she spent in this way, that feeling grew
every day. And the happiness of taking communion, or
\emph{communing} as Agrafena Ivanovna, joyously playing with the
word, called it, seemed to Natasha so great that she felt she
should never live till that blessed Sunday.

But the happy day came, and on that memorable Sunday, when,
dressed in white muslin, she returned home after communion, for
the first time for many months she felt calm and not oppressed by
the thought of the life that lay before her.

The doctor who came to see her that day ordered her to continue
the powders he had prescribed a fortnight previously.

``She must certainly go on taking them morning and evening,''
said he, evidently sincerely satisfied with his success. ``Only,
please be particular about it.''

``Be quite easy,'' he continued playfully, as he adroitly took
the gold coin in his palm. ``She will soon be singing and
frolicking about. The last medicine has done her a very great
deal of good. She has freshened up very much.''

The countess, with a cheerful expression on her face, looked down
at her nails and spat a little for luck as she returned to the
drawing room.

% % % % % % % % % % % % % % % % % % % % % % % % % % % % % % % % %
% % % % % % % % % % % % % % % % % % % % % % % % % % % % % % % % %
% % % % % % % % % % % % % % % % % % % % % % % % % % % % % % % % %
% % % % % % % % % % % % % % % % % % % % % % % % % % % % % % % % %
% % % % % % % % % % % % % % % % % % % % % % % % % % % % % % % % %
% % % % % % % % % % % % % % % % % % % % % % % % % % % % % % % % %
% % % % % % % % % % % % % % % % % % % % % % % % % % % % % % % % %
% % % % % % % % % % % % % % % % % % % % % % % % % % % % % % % % %
% % % % % % % % % % % % % % % % % % % % % % % % % % % % % % % % %
% % % % % % % % % % % % % % % % % % % % % % % % % % % % % % % % %
% % % % % % % % % % % % % % % % % % % % % % % % % % % % % % % % %
% % % % % % % % % % % % % % % % % % % % % % % % % % % % % %

\chapter*{Chapter XVIII}
\ifaudio     
\marginpar{
\href{http://ia802702.us.archive.org/23/items/war_and_peace_09_0811_librivox/war_and_peace_09_18_tolstoy_64kb.mp3}{Audio}} 
\fi

\initial{A}{t} the beginning of July more and more disquieting reports about
the war began to spread in Moscow; people spoke of an appeal by
the Emperor to the people, and of his coming himself from the
army to Moscow. And as up to the eleventh of July no manifesto or
appeal had been received, exaggerated reports became current
about them and about the position of Russia. It was said that the
Emperor was leaving the army because it was in danger, it was
said that Smolensk had surrendered, that Napoleon had an army of
a million and only a miracle could save Russia.

On the eleventh of July, which was Saturday, the manifesto was
received but was not yet in print, and Pierre, who was at the
Rostovs', promised to come to dinner next day, Sunday, and bring
a copy of the manifesto and appeal, which he would obtain from
Count Rostopchin.

That Sunday, the Rostovs went to Mass at the Razumovskis' private
chapel as usual. It was a hot July day. Even at ten o'clock, when
the Rostovs got out of their carriage at the chapel, the sultry
air, the shouts of hawkers, the light and gay summer clothes of
the crowd, the dusty leaves of the trees on the boulevard, the
sounds of the band and the white trousers of a battalion marching
to parade, the rattling of wheels on the cobblestones, and the
brilliant, hot sunshine were all full of that summer languor,
that content and discontent with the present, which is most
strongly felt on a bright, hot day in town. All the Moscow
notabilities, all the Rostovs' acquaintances, were at the
Razumovskis' chapel, for, as if expecting something to happen,
many wealthy families who usually left town for their country
estates had not gone away that summer. As Natasha, at her
mother's side, passed through the crowd behind a liveried footman
who cleared the way for them, she heard a young man speaking
about her in too loud a whisper.

``That's Rostova, the one who...''

``She's much thinner, but all the same she's pretty!''

She heard, or thought she heard, the names of Kuragin and
Bolkonski. But she was always imagining that. It always seemed to
her that everyone who looked at her was thinking only of what had
happened to her. With a sinking heart, wretched as she always was
now when she found herself in a crowd, Natasha in her lilac silk
dress trimmed with black lace walked---as women can walk---with
the more repose and stateliness the greater the pain and shame in
her soul. She knew for certain that she was pretty, but this no
longer gave her satisfaction as it used to. On the contrary it
tormented her more than anything else of late, and particularly
so on this bright, hot summer day in town. ``It's Sunday
again---another week past,'' she thought, recalling that she had
been here the Sunday before, ``and always the same life that is
no life, and the same surroundings in which it used to be so easy
to live. I'm pretty, I'm young, and I know that now I am good. I
used to be bad, but now I know I am good,'' she thought, ``but
yet my best years are slipping by and are no good to anyone.''
She stood by her mother's side and exchanged nods with
acquaintances near her. From habit she scrutinized the ladies'
dresses, condemned the bearing of a lady standing close by who
was not crossing herself properly but in a cramped manner, and
again she thought with vexation that she was herself being judged
and was judging others, and suddenly, at the sound of the
service, she felt horrified at her own vileness, horrified that
the former purity of her soul was again lost to her.

A comely, fresh-looking old man was conducting the service with
that mild solemnity which has so elevating and soothing an effect
on the souls of the worshipers. The gates of the sanctuary screen
were closed, the curtain was slowly drawn, and from behind it a
soft mysterious voice pronounced some words. Tears, the cause of
which she herself did not understand, made Natasha's breast
heave, and a joyous but oppressive feeling agitated her.

``Teach me what I should do, how to live my life, how I may grow
good forever, forever!'' she pleaded.

The deacon came out onto the raised space before the altar screen
and, holding his thumb extended, drew his long hair from under
his dalmatic and, making the sign of the cross on his breast,
began in a loud and solemn voice to recite the words of the
prayer...

``In peace let us pray unto the Lord.''

``As one community, without distinction of class, without enmity,
united by brotherly love---let us pray!'' thought Natasha.

``For the peace that is from above, and for the salvation of our
souls.''

``For the world of angels and all the spirits who dwell above
us,'' prayed Natasha.

When they prayed for the warriors, she thought of her brother and
Denisov. When they prayed for all traveling by land and sea, she
remembered Prince Andrew, prayed for him, and asked God to
forgive her all the wrongs she had done him. When they prayed for
those who love us, she prayed for the members of her own family,
her father and mother and Sonya, realizing for the first time how
wrongly she had acted toward them, and feeling all the strength
of her love for them. When they prayed for those who hate us, she
tried to think of her enemies and people who hated her, in order
to pray for them. She included among her enemies the creditors
and all who had business dealings with her father, and always at
the thought of enemies and those who hated her she remembered
Anatole who had done her so much harm---and though he did not
hate her she gladly prayed for him as for an enemy. Only at
prayer did she feel able to think clearly and calmly of Prince
Andrew and Anatole, as men for whom her feelings were as nothing
compared with her awe and devotion to God. When they prayed for
the Imperial family and the Synod, she bowed very low and made
the sign of the cross, saying to herself that even if she did not
understand, still she could not doubt, and at any rate loved the
governing Synod and prayed for it.

When he had finished the Litany the deacon crossed the stole over
his breast and said, ``Let us commit ourselves and our whole
lives to Christ the Lord!''

``Commit ourselves to God,'' Natasha inwardly repeated. ``Lord
God, I submit myself to Thy will!'' she thought. ``I want
nothing, wish for nothing; teach me what to do and how to use my
will! Take me, take me!''  prayed Natasha, with impatient emotion
in her heart, not crossing herself but letting her slender arms
hang down as if expecting some invisible power at any moment to
take her and deliver her from herself, from her regrets, desires,
remorse, hopes, and sins.

The countess looked round several times at her daughter's
softened face and shining eyes and prayed God to help her.

Unexpectedly, in the middle of the service, and not in the usual
order Natasha knew so well, the deacon brought out a small stool,
the one he knelt on when praying on Trinity Sunday, and placed it
before the doors of the sanctuary screen. The priest came out
with his purple velvet biretta on his head, adjusted his hair,
and knelt down with an effort.  Everybody followed his example
and they looked at one another in surprise. Then came the prayer
just received from the Synod---a prayer for the deliverance of
Russia from hostile invasion.

``Lord God of might, God of our salvation!'' began the priest in
that voice, clear, not grandiloquent but mild, in which only the
Slav clergy read and which acts so irresistibly on a Russian
heart.

``Lord God of might, God of our salvation! Look this day in mercy
and blessing on Thy humble people, and graciously hear us, spare
us, and have mercy upon us! This foe confounding Thy land,
desiring to lay waste the whole world, rises against us; these
lawless men are gathered together to overthrow Thy kingdom, to
destroy Thy dear Jerusalem, Thy beloved Russia; to defile Thy
temples, to overthrow Thine altars, and to desecrate our holy
shrines. How long, O Lord, how long shall the wicked triumph? How
long shall they wield unlawful power?''

``Lord God! Hear us when we pray to Thee; strengthen with Thy
might our most gracious sovereign lord, the Emperor Alexander
Pavlovich; be mindful of his uprightness and meekness, reward him
according to his righteousness, and let it preserve us, Thy
chosen Israel! Bless his counsels, his undertakings, and his
work; strengthen his kingdom by Thine almighty hand, and give him
victory over his enemy, even as Thou gavest Moses the victory
over Amalek, Gideon over Midian, and David over Goliath. Preserve
his army, put a bow of brass in the hands of those who have armed
themselves in Thy Name, and gird their loins with strength for
the fight. Take up the spear and shield and arise to help us;
confound and put to shame those who have devised evil against us,
may they be before the faces of Thy faithful warriors as dust
before the wind, and may Thy mighty Angel confound them and put
them to flight; may they be ensnared when they know it not, and
may the plots they have laid in secret be turned against them;
let them fall before Thy servants' feet and be laid low by our
hosts! Lord, Thou art able to save both great and small; Thou art
God, and man cannot prevail against Thee!''

``God of our fathers! Remember Thy bounteous mercy and
loving-kindness which are from of old; turn not Thy face from us,
but be gracious to our unworthiness, and in Thy great goodness
and Thy many mercies regard not our transgressions and
iniquities! Create in us a clean heart and renew a right spirit
within us, strengthen us all in Thy faith, fortify our hope,
inspire us with true love one for another, arm us with unity of
spirit in the righteous defense of the heritage Thou gavest to us
and to our fathers, and let not the scepter of the wicked be
exalted against the destiny of those Thou hast sanctified.''

``O Lord our God, in whom we believe and in whom we put our
trust, let us not be confounded in our hope of Thy mercy, and
give us a token of Thy blessing, that those who hate us and our
Orthodox faith may see it and be put to shame and perish, and may
all the nations know that Thou art the Lord and we are Thy
people. Show Thy mercy upon us this day, O Lord, and grant us Thy
salvation; make the hearts of Thy servants to rejoice in Thy
mercy; smite down our enemies and destroy them swiftly beneath
the feet of Thy faithful servants! For Thou art the defense, the
succor, and the victory of them that put their trust in Thee, and
to Thee be all glory, to Father, Son, and Holy Ghost, now and
forever, world without end. Amen.''

In Natasha's receptive condition of soul this prayer affected her
strongly. She listened to every word about the victory of Moses
over Amalek, of Gideon over Midian, and of David over Goliath,
and about the destruction of \emph{Thy Jerusalem}, and she prayed
to God with the tenderness and emotion with which her heart was
overflowing, but without fully understanding what she was asking
of God in that prayer. She shared with all her heart in the
prayer for the spirit of righteousness, for the strengthening of
the heart by faith and hope, and its animation by love. But she
could not pray that her enemies might be trampled under foot when
but a few minutes before she had been wishing she had more of
them that she might pray for them. But neither could she doubt
the righteousness of the prayer that was being read on bended
knees. She felt in her heart a devout and tremulous awe at the
thought of the punishment that overtakes men for their sins, and
especially of her own sins, and she prayed to God to forgive them
all, and her too, and to give them all, and her too, peace and
happiness. And it seemed to her that God heard her prayer.

% % % % % % % % % % % % % % % % % % % % % % % % % % % % % % % % %
% % % % % % % % % % % % % % % % % % % % % % % % % % % % % % % % %
% % % % % % % % % % % % % % % % % % % % % % % % % % % % % % % % %
% % % % % % % % % % % % % % % % % % % % % % % % % % % % % % % % %
% % % % % % % % % % % % % % % % % % % % % % % % % % % % % % % % %
% % % % % % % % % % % % % % % % % % % % % % % % % % % % % % % % %
% % % % % % % % % % % % % % % % % % % % % % % % % % % % % % % % %
% % % % % % % % % % % % % % % % % % % % % % % % % % % % % % % % %
% % % % % % % % % % % % % % % % % % % % % % % % % % % % % % % % %
% % % % % % % % % % % % % % % % % % % % % % % % % % % % % % % % %
% % % % % % % % % % % % % % % % % % % % % % % % % % % % % % % % %
% % % % % % % % % % % % % % % % % % % % % % % % % % % % % %

\chapter*{Chapter XIX}
\ifaudio     
\marginpar{
\href{http://ia802702.us.archive.org/23/items/war_and_peace_09_0811_librivox/war_and_peace_09_19_tolstoy_64kb.mp3}{Audio}} 
\fi

\initial{F}{rom} the day when Pierre, after leaving the Rostovs' with
Natasha's grateful look fresh in his mind, had gazed at the comet
that seemed to be fixed in the sky and felt that something new
was appearing on his own horizon---from that day the problem of
the vanity and uselessness of all earthly things, that had
incessantly tormented him, no longer presented itself. That
terrible question ``Why?'' ``Wherefore?'' which had come to him
amid every occupation, was now replaced, not by another question
or by a reply to the former question, but by her image. When he
listened to, or himself took part in, trivial conversations, when
he read or heard of human baseness or folly, he was not horrified
as formerly, and did not ask himself why men struggled so about
these things when all is so transient and incomprehensible---but
he remembered her as he had last seen her, and all his doubts
vanished---not because she had answered the questions that had
haunted him, but because his conception of her transferred him
instantly to another, a brighter, realm of spiritual activity in
which no one could be justified or guilty---a realm of beauty and
love which it was worth living for. Whatever worldly baseness
presented itself to him, he said to himself:

``Well, supposing N. N. swindled the country and the Tsar, and
the country and the Tsar confer honors upon him, what does that
matter? She smiled at me yesterday and asked me to come again,
and I love her, and no one will ever know it.'' And his soul felt
calm and peaceful.

Pierre still went into society, drank as much and led the same
idle and dissipated life, because besides the hours he spent at
the Rostovs' there were other hours he had to spend somehow, and
the habits and acquaintances he had made in Moscow formed a
current that bore him along irresistibly. But latterly, when more
and more disquieting reports came from the seat of war and
Natasha's health began to improve and she no longer aroused in
him the former feeling of careful pity, an ever-increasing
restlessness, which he could not explain, took possession of
him. He felt that the condition he was in could not continue
long, that a catastrophe was coming which would change his whole
life, and he impatiently sought everywhere for signs of that
approaching catastrophe.  One of his brother Masons had revealed
to Pierre the following prophecy concerning Napoleon, drawn from
the Revelation of St. John.

In chapter 13, verse 18, of the Apocalypse, it is said:

\begin{quote}
Here is wisdom. Let him that hath understanding count the number
of the beast: for it is the number of a man; and his number is
Six hundred threescore and six.
\end{quote}

And in the fifth verse of the same chapter:

\begin{quote}
And there was given unto him a mouth speaking great things and
blasphemies; and power was given unto him to continue forty and
two months.
\end{quote}

The French alphabet, written out with the same numerical values
as the Hebrew, in which the first nine letters denote units and
the others tens, will have the following significance:

\begin{quote} \small
a b c d e f g h i k \\ 
1 2 3 4 5 6 7 8 9 10 \\ 
l m n o p q r s \\ 
20 30 40 50 60 70 80 90 \\
t u v w x y \\ 
100 110 120 130 140 150 z 160
\end{quote}

Writing the words L'Empereur Napoleon in numbers, it appears that
the sum of them is 666, and that Napoleon was therefore the beast
foretold in the Apocalypse. Moreover, by applying the same system
to the words quarante-deux, \footnote{Forty-two.} which was the
term allowed to the beast that ``spoke great things and
blasphemies,'' the same number 666 was obtained; from which it
followed that the limit fixed for Napoleon's power had come in
the year 1812 when the French emperor was forty-two. This
prophecy pleased Pierre very much and he often asked himself what
would put an end to the power of the beast, that is, of Napoleon,
and tried by the same system of using letters as numbers and
adding them up, to find an answer to the question that engrossed
him. He wrote the words L'Empereur Alexandre, La nation russe and
added up their numbers, but the sums were either more or less
than 666. Once when making such calculations he wrote down his
own name in French, Comte Pierre Besouhoff, but the sum of the
numbers did not come right. Then he changed the spelling,
substituting a z for the s and adding de and the article le,
still without obtaining the desired result. Then it occurred to
him: if the answer to the question were contained in his name,
his nationality would also be given in the answer. So he wrote Le
russe Besuhof and adding up the numbers got 671. This was only
five too much, and five was represented by e, the very letter
elided from the article le before the word Empereur. By omitting
the e, though incorrectly, Pierre got the answer he
sought. L'russe Besuhof made 666. This discovery excited him.
How, or by what means, he was connected with the great event
foretold in the Apocalypse he did not know, but he did not doubt
that connection for a moment. His love for Natasha, Antichrist,
Napoleon, the invasion, the comet, 666, L'Empereur Napoleon, and
L'russe Besuhof---all this had to mature and culminate, to lift
him out of that spellbound, petty sphere of Moscow habits in
which he felt himself held captive and lead him to a great
achievement and great happiness.

On the eve of the Sunday when the special prayer was read, Pierre
had promised the Rostovs to bring them, from Count Rostopchin
whom he knew well, both the appeal to the people and the news
from the army. In the morning, when he went to call at
Rostopchin's he met there a courier fresh from the army, an
acquaintance of his own, who often danced at Moscow balls.

``Do, please, for heaven's sake, relieve me of something!'' said
the courier. ``I have a sackful of letters to parents.''

Among these letters was one from Nicholas Rostov to his
father. Pierre took that letter, and Rostopchin also gave him the
Emperor's appeal to Moscow, which had just been printed, the last
army orders, and his own most recent bulletin. Glancing through
the army orders, Pierre found in one of them, in the lists of
killed, wounded, and rewarded, the name of Nicholas Rostov,
awarded a St. George's Cross of the Fourth Class for courage
shown in the Ostrovna affair, and in the same order the name of
Prince Andrew Bolkonski, appointed to the command of a regiment
of Chasseurs. Though he did not want to remind the Rostovs of
Bolkonski, Pierre could not refrain from making them happy by the
news of their son's having received a decoration, so he sent that
printed army order and Nicholas' letter to the Rostovs, keeping
the appeal, the bulletin, and the other orders to take with him
when he went to dinner.

His conversation with Count Rostopchin and the latter's tone of
anxious hurry, the meeting with the courier who talked casually
of how badly things were going in the army, the rumors of the
discovery of spies in Moscow and of a leaflet in circulation
stating that Napoleon promised to be in both the Russian capitals
by the autumn, and the talk of the Emperor's being expected to
arrive next day---all aroused with fresh force that feeling of
agitation and expectation in Pierre which he had been conscious
of ever since the appearance of the comet, and especially since
the beginning of the war.

He had long been thinking of entering the army and would have
done so had he not been hindered, first, by his membership of the
Society of Freemasons to which he was bound by oath and which
preached perpetual peace and the abolition of war, and secondly,
by the fact that when he saw the great mass of Muscovites who had
donned uniform and were talking patriotism, he somehow felt
ashamed to take the step. But the chief reason for not carrying
out his intention to enter the army lay in the vague idea that he
was L'russe Besuhof who had the number of the beast, 666; that
his part in the great affair of setting a limit to the power of
the beast that spoke great and blasphemous things had been
predestined from eternity, and that therefore he ought not to
undertake anything, but wait for what was bound to come to pass.

% % % % % % % % % % % % % % % % % % % % % % % % % % % % % % % % %
% % % % % % % % % % % % % % % % % % % % % % % % % % % % % % % % %
% % % % % % % % % % % % % % % % % % % % % % % % % % % % % % % % %
% % % % % % % % % % % % % % % % % % % % % % % % % % % % % % % % %
% % % % % % % % % % % % % % % % % % % % % % % % % % % % % % % % %
% % % % % % % % % % % % % % % % % % % % % % % % % % % % % % % % %
% % % % % % % % % % % % % % % % % % % % % % % % % % % % % % % % %
% % % % % % % % % % % % % % % % % % % % % % % % % % % % % % % % %
% % % % % % % % % % % % % % % % % % % % % % % % % % % % % % % % %
% % % % % % % % % % % % % % % % % % % % % % % % % % % % % % % % %
% % % % % % % % % % % % % % % % % % % % % % % % % % % % % % % % %
% % % % % % % % % % % % % % % % % % % % % % % % % % % % % %

\chapter*{Chapter XX}
\ifaudio     
\marginpar{
\href{http://ia802702.us.archive.org/23/items/war_and_peace_09_0811_librivox/war_and_peace_09_20_tolstoy_64kb.mp3}{Audio}} 
\fi

\initial{A}{} few intimate friends were dining with the Rostovs that day, as
usual on Sundays.

Pierre came early so as to find them alone.

He had grown so stout this year that he would have been abnormal
had he not been so tall, so broad of limb, and so strong that he
carried his bulk with evident ease.

He went up the stairs, puffing and muttering something. His
coachman did not even ask whether he was to wait. He knew that
when his master was at the Rostovs' he stayed till midnight. The
Rostovs' footman rushed eagerly forward to help him off with his
cloak and take his hat and stick. Pierre, from club habit, always
left both hat and stick in the anteroom.

The first person he saw in the house was Natasha. Even before he
saw her, while taking off his cloak, he heard her. She was
practicing solfa exercises in the music room. He knew that she
had not sung since her illness, and so the sound of her voice
surprised and delighted him. He opened the door softly and saw
her, in the lilac dress she had worn at church, walking about the
room singing. She had her back to him when he opened the door,
but when, turning quickly, she saw his broad, surprised face, she
blushed and came rapidly up to him.

``I want to try to sing again,'' she said, adding as if by way of
excuse, ``it is, at least, something to do.''

``That's capital!''

``How glad I am you've come! I am so happy today,'' she said,
with the old animation Pierre had not seen in her for a long
time. ``You know Nicholas has received a St. George's Cross? I am
so proud of him.''

``Oh yes, I sent that announcement. But I don't want to interrupt
you,'' he added, and was about to go to the drawing room.

Natasha stopped him.

``Count, is it wrong of me to sing?'' she said blushing, and
fixing her eyes inquiringly on him.

``No... Why should it be? On the contrary... But why do you ask
me?''

``I don't know myself,'' Natasha answered quickly, ``but I should
not like to do anything you disapproved of. I believe in you
completely. You don't know how important you are to me, how much
you've done for me...''  She spoke rapidly and did not notice how
Pierre flushed at her words. ``I saw in that same army order that
he, Bolkonski'' (she whispered the name hastily), ``is in Russia,
and in the army again. What do you think?''---she was speaking
hurriedly, evidently afraid her strength might fail her---``Will
he ever forgive me? Will he not always have a bitter feeling
toward me? What do you think? What do you think?''

``I think...'' Pierre replied, ``that he has nothing to
forgive... If I were in his place...''

By association of ideas, Pierre was at once carried back to the
day when, trying to comfort her, he had said that if he were not
himself but the best man in the world and free, he would ask on
his knees for her hand; and the same feeling of pity, tenderness,
and love took possession of him and the same words rose to his
lips. But she did not give him time to say them.

``Yes, you... you...'' she said, uttering the word you
rapturously---``that's a different thing. I know no one kinder,
more generous, or better than you; nobody could be! Had you not
been there then, and now too, I don't know what would have become
of me, because...''

Tears suddenly rose in her eyes, she turned away, lifted her
music before her eyes, began singing again, and again began
walking up and down the room.

Just then Petya came running in from the drawing room.

Petya was now a handsome rosy lad of fifteen with full red lips
and resembled Natasha. He was preparing to enter the university,
but he and his friend Obolenski had lately, in secret, agreed to
join the hussars.

Petya had come rushing out to talk to his namesake about this
affair. He had asked Pierre to find out whether he would be
accepted in the hussars.

Pierre walked up and down the drawing room, not listening to what
Petya was saying.

Petya pulled him by the arm to attract his attention.

``Well, what about my plan? Peter Kirilych, for heaven's sake!
You are my only hope,'' said Petya.

``Oh yes, your plan. To join the hussars? I'll mention it, I'll
bring it all up today.''

``Well, mon cher, have you got the manifesto?'' asked the old
count. ``The countess has been to Mass at the Razumovskis' and
heard the new prayer.  She says it's very fine.''

``Yes, I've got it,'' said Pierre. ``The Emperor is to be here
tomorrow...  there's to be an Extraordinary Meeting of the
nobility, and they are talking of a levy of ten men per
thousand. Oh yes, let me congratulate you!''

``Yes, yes, thank God! Well, and what news from the army?''

``We are again retreating. They say we're already near
Smolensk,'' replied Pierre.

``O Lord, O Lord!'' exclaimed the count. ``Where is the
manifesto?''

``The Emperor's appeal? Oh yes!''

Pierre began feeling in his pockets for the papers, but could not
find them. Still slapping his pockets, he kissed the hand of the
countess who entered the room and glanced uneasily around,
evidently expecting Natasha, who had left off singing but had not
yet come into the drawing room.

``On my word, I don't know what I've done with it,'' he said.

``There he is, always losing everything!'' remarked the countess.

Natasha entered with a softened and agitated expression of face
and sat down looking silently at Pierre. As soon as she entered,
Pierre's features, which had been gloomy, suddenly lighted up,
and while still searching for the papers he glanced at her
several times.

``No, really! I'll drive home, I must have left them there. I'll
certainly...''

``But you'll be late for dinner.''

``Oh! And my coachman has gone.''

But Sonya, who had gone to look for the papers in the anteroom,
had found them in Pierre's hat, where he had carefully tucked
them under the lining. Pierre was about to begin reading.

``No, after dinner,'' said the old count, evidently expecting
much enjoyment from that reading.

At dinner, at which champagne was drunk to the health of the new
chevalier of St. George, Shinshin told them the town news, of the
illness of the old Georgian princess, of Metivier's disappearance
from Moscow, and of how some German fellow had been brought to
Rostopchin and accused of being a French \emph{spyer} (so Count
Rostopchin had told the story), and how Rostopchin let him go and
assured the people that he was ``not a spire at all, but only an
old German ruin.''

``People are being arrested...'' said the count. ``I've told the
countess she should not speak French so much. It's not the time
for it now.''

``And have you heard?'' Shinshin asked. ``Prince Golitsyn has
engaged a master to teach him Russian. It is becoming dangerous
to speak French in the streets.''

``And how about you, Count Peter Kirilych? If they call up the
militia, you too will have to mount a horse,'' remarked the old
count, addressing Pierre.

Pierre had been silent and preoccupied all through dinner,
seeming not to grasp what was said. He looked at the count.

``Oh yes, the war,'' he said. ``No! What sort of warrior should I
make? And yet everything is so strange, so strange! I can't make
it out. I don't know, I am very far from having military tastes,
but in these times no one can answer for himself.''

After dinner the count settled himself comfortably in an easy
chair and with a serious face asked Sonya, who was considered an
excellent reader, to read the appeal.

``To Moscow, our ancient Capital!''

``The enemy has entered the borders of Russia with immense
forces. He comes to despoil our beloved country.''

Sonya read painstakingly in her high-pitched voice. The count
listened with closed eyes, heaving abrupt sighs at certain
passages.

Natasha sat erect, gazing with a searching look now at her father
and now at Pierre.

Pierre felt her eyes on him and tried not to look round. The
countess shook her head disapprovingly and angrily at every
solemn expression in the manifesto. In all these words she saw
only that the danger threatening her son would not soon be
over. Shinshin, with a sarcastic smile on his lips, was evidently
preparing to make fun of anything that gave him the opportunity:
Sonya's reading, any remark of the count's, or even the manifesto
itself should no better pretext present itself.

After reading about the dangers that threatened Russia, the hopes
the Emperor placed on Moscow and especially on its illustrious
nobility, Sonya, with a quiver in her voice due chiefly to the
attention that was being paid to her, read the last words:

``We ourselves will not delay to appear among our people in that
Capital and in other parts of our realm for consultation, and for
the direction of all our levies, both those now barring the
enemy's path and those freshly formed to defeat him wherever he
may appear. May the ruin he hopes to bring upon us recoil on his
own head, and may Europe delivered from bondage glorify the name
of Russia!''

``Yes, that's it!'' cried the count, opening his moist eyes and
sniffing repeatedly, as if a strong vinaigrette had been held to
his nose; and he added, ``Let the Emperor but say the word and
we'll sacrifice everything and begrudge nothing.''

Before Shinshin had time to utter the joke he was ready to make
on the count's patriotism, Natasha jumped up from her place and
ran to her father.

``What a darling our Papa is!'' she cried, kissing him, and she
again looked at Pierre with the unconscious coquetry that had
returned to her with her better spirits.

``There! Here's a patriot for you!'' said Shinshin.

``Not a patriot at all, but simply...'' Natasha replied in an
injured tone. ``Everything seems funny to you, but this isn't at
all a joke...''

``A joke indeed!'' put in the count. ``Let him but say the word
and we'll all go... We're not Germans!''

``But did you notice, it says, 'for consultation'?'' said Pierre.

``Never mind what it's for...''

At this moment, Petya, to whom nobody was paying any attention,
came up to his father with a very flushed face and said in his
breaking voice that was now deep and now shrill:

``Well, Papa, I tell you definitely, and Mamma too, it's as you
please, but I say definitely that you must let me enter the army,
because I can't... that's all...''

The countess, in dismay, looked up to heaven, clasped her hands,
and turned angrily to her husband.

``That comes of your talking!'' said she.

But the count had already recovered from his excitement.

``Come, come!'' said he. ``Here's a fine warrior! No! Nonsense!
You must study.''

``It's not nonsense, Papa. Fedya Obolenski is younger than I, and
he's going too. Besides, all the same I can't study now when...''
Petya stopped short, flushed till he perspired, but still got out
the words, ``when our Fatherland is in danger.''

``That'll do, that'll do---nonsense...''

``But you said yourself that we would sacrifice everything.''

``Petya! Be quiet, I tell you!'' cried the count, with a glance
at his wife, who had turned pale and was staring fixedly at her
son.

``And I tell you---Peter Kirilych here will also tell you...''

``Nonsense, I tell you. Your mother's milk has hardly dried on
your lips and you want to go into the army! There, there, I tell
you,'' and the count moved to go out of the room, taking the
papers, probably to reread them in his study before having a nap.

``Well, Peter Kirilych, let's go and have a smoke,'' he said.

Pierre was agitated and undecided. Natasha's unwontedly brilliant
eyes, continually glancing at him with a more than cordial look,
had reduced him to this condition.

``No, I think I'll go home.''

``Home? Why, you meant to spend the evening with us... You don't
often come nowadays as it is, and this girl of mine,'' said the
count good-naturedly, pointing to Natasha, ``only brightens up
when you're here.''

``Yes, I had forgotten... I really must go home... business...''
said Pierre hurriedly.

``Well, then, au revoir!'' said the count, and went out of the
room.

``Why are you going? Why are you upset?'' asked Natasha, and she
looked challengingly into Pierre's eyes.

``Because I love you!'' was what he wanted to say, but he did not
say it, and only blushed till the tears came, and lowered his
eyes.

``Because it is better for me to come less
often... because... No, simply I have business...''

``Why? No, tell me!'' Natasha began resolutely and suddenly
stopped.

They looked at each other with dismayed and embarrassed faces. He
tried to smile but could not: his smile expressed suffering, and
he silently kissed her hand and went out.

Pierre made up his mind not to go to the Rostovs' any more.

% % % % % % % % % % % % % % % % % % % % % % % % % % % % % % % % %
% % % % % % % % % % % % % % % % % % % % % % % % % % % % % % % % %
% % % % % % % % % % % % % % % % % % % % % % % % % % % % % % % % %
% % % % % % % % % % % % % % % % % % % % % % % % % % % % % % % % %
% % % % % % % % % % % % % % % % % % % % % % % % % % % % % % % % %
% % % % % % % % % % % % % % % % % % % % % % % % % % % % % % % % %
% % % % % % % % % % % % % % % % % % % % % % % % % % % % % % % % %
% % % % % % % % % % % % % % % % % % % % % % % % % % % % % % % % %
% % % % % % % % % % % % % % % % % % % % % % % % % % % % % % % % %
% % % % % % % % % % % % % % % % % % % % % % % % % % % % % % % % %
% % % % % % % % % % % % % % % % % % % % % % % % % % % % % % % % %
% % % % % % % % % % % % % % % % % % % % % % % % % % % % % %

\chapter*{Chapter XXI}
\ifaudio
\marginpar{
\href{http://ia802702.us.archive.org/23/items/war_and_peace_09_0811_librivox/war_and_peace_09_21_tolstoy_64kb.mp3}{Audio}} 
\fi

\initial{A}{fter} the definite refusal he had received, Petya went to his
room and there locked himself in and wept bitterly. When he came
in to tea, silent, morose, and with tear-stained face, everybody
pretended not to notice anything.

Next day the Emperor arrived in Moscow, and several of the
Rostovs' domestic serfs begged permission to go to have a look at
him. That morning Petya was a long time dressing and arranging
his hair and collar to look like a grown-up man. He frowned
before his looking glass, gesticulated, shrugged his shoulders,
and finally, without saying a word to anyone, took his cap and
left the house by the back door, trying to avoid notice. Petya
decided to go straight to where the Emperor was and to explain
frankly to some gentleman-in-waiting (he imagined the Emperor to
be always surrounded by gentlemen-in-waiting) that he, Count
Rostov, in spite of his youth wished to serve his country; that
youth could be no hindrance to loyalty, and that he was ready
to... While dressing, Petya had prepared many fine things he
meant to say to the gentleman-in-waiting.

It was on the very fact of being so young that Petya counted for
success in reaching the Emperor---he even thought how surprised
everyone would be at his youthfulness---and yet in the
arrangement of his collar and hair and by his sedate deliberate
walk he wished to appear a grown-up man.  But the farther he went
and the more his attention was diverted by the ever-increasing
crowds moving toward the Kremlin, the less he remembered to walk
with the sedateness and deliberation of a man. As he approached
the Kremlin he even began to avoid being crushed and resolutely
stuck out his elbows in a menacing way. But within the Trinity
Gateway he was so pressed to the wall by people who probably were
unaware of the patriotic intentions with which he had come that
in spite of all his determination he had to give in, and stop
while carriages passed in, rumbling beneath the archway. Beside
Petya stood a peasant woman, a footman, two tradesmen, and a
discharged soldier. After standing some time in the gateway,
Petya tried to move forward in front of the others without
waiting for all the carriages to pass, and he began resolutely
working his way with his elbows, but the woman just in front of
him, who was the first against whom he directed his efforts,
angrily shouted at him:

``What are you shoving for, young lordling? Don't you see we're
all standing still? Then why push?''

``Anybody can shove,'' said the footman, and also began working
his elbows to such effect that he pushed Petya into a very filthy
corner of the gateway.

Petya wiped his perspiring face with his hands and pulled up the
damp collar which he had arranged so well at home to seem like a
man's.

He felt that he no longer looked presentable, and feared that if
he were now to approach the gentlemen-in-waiting in that plight
he would not be admitted to the Emperor. But it was impossible to
smarten oneself up or move to another place, because of the
crowd. One of the generals who drove past was an acquaintance of
the Rostovs', and Petya thought of asking his help, but came to
the conclusion that that would not be a manly thing to do. When
the carriages had all passed in, the crowd, carrying Petya with
it, streamed forward into the Kremlin Square which was already
full of people. There were people not only in the square, but
everywhere---on the slopes and on the roofs. As soon as Petya
found himself in the square he clearly heard the sound of bells
and the joyous voices of the crowd that filled the whole Kremlin.

For a while the crowd was less dense, but suddenly all heads were
bared, and everyone rushed forward in one direction. Petya was
being pressed so that he could scarcely breathe, and everybody
shouted, ``Hurrah! hurrah!  hurrah!'' Petya stood on tiptoe and
pushed and pinched, but could see nothing except the people about
him.

All the faces bore the same expression of excitement and
enthusiasm. A tradesman's wife standing beside Petya sobbed, and
the tears ran down her cheeks.

``Father! Angel! Dear one!'' she kept repeating, wiping away her
tears with her fingers.

``Hurrah!'' was heard on all sides.

For a moment the crowd stood still, but then it made another rush
forward.

Quite beside himself, Petya, clinching his teeth and rolling his
eyes ferociously, pushed forward, elbowing his way and shouting
``hurrah!'' as if he were prepared that instant to kill himself
and everyone else, but on both sides of him other people with
similarly ferocious faces pushed forward and everybody shouted
``hurrah!''

``So this is what the Emperor is!'' thought Petya. ``No, I can't
petition him myself---that would be too bold.'' But in spite of
this he continued to struggle desperately forward, and from
between the backs of those in front he caught glimpses of an open
space with a strip of red cloth spread out on it; but just then
the crowd swayed back---the police in front were pushing back
those who had pressed too close to the procession: the Emperor
was passing from the palace to the Cathedral of the
Assumption---and Petya unexpectedly received such a blow on his
side and ribs and was squeezed so hard that suddenly everything
grew dim before his eyes and he lost consciousness. When he came
to himself, a man of clerical appearance with a tuft of gray hair
at the back of his head and wearing a shabby blue
cassock---probably a church clerk and chanter---was holding him
under the arm with one hand while warding off the pressure of the
crowd with the other.

``You've crushed the young gentleman!'' said the clerk. ``What
are you up to? Gently!... They've crushed him, crushed him!''

The Emperor entered the Cathedral of the Assumption. The crowd
spread out again more evenly, and the clerk led Petya---pale and
breathless---to the Tsar-cannon. Several people were sorry for
Petya, and suddenly a crowd turned toward him and pressed round
him. Those who stood nearest him attended to him, unbuttoned his
coat, seated him on the raised platform of the cannon, and
reproached those others (whoever they might be) who had crushed
him.

``One might easily get killed that way! What do they mean by it?
Killing people! Poor dear, he's as white as a sheet!''---various
voices were heard saying.

Petya soon came to himself, the color returned to his face, the
pain had passed, and at the cost of that temporary unpleasantness
he had obtained a place by the cannon from where he hoped to see
the Emperor who would be returning that way. Petya no longer
thought of presenting his petition. If he could only see the
Emperor he would be happy!

While the service was proceeding in the Cathedral of the
Assumption---it was a combined service of prayer on the occasion
of the Emperor's arrival and of thanksgiving for the conclusion
of peace with the Turks---the crowd outside spread out and
hawkers appeared, selling kvas, gingerbread, and poppyseed sweets
(of which Petya was particularly fond), and ordinary conversation
could again be heard. A tradesman's wife was showing a rent in
her shawl and telling how much the shawl had cost; another was
saying that all silk goods had now got dear. The clerk who had
rescued Petya was talking to a functionary about the priests who
were officiating that day with the bishop. The clerk several
times used the word \emph{plenary} (of the service), a word Petya
did not understand.  Two young citizens were joking with some
serf girls who were cracking nuts. All these conversations,
especially the joking with the girls, were such as might have had
a particular charm for Petya at his age, but they did not
interest him now. He sat on his elevation---the pedestal of the
cannon---still agitated as before by the thought of the Emperor
and by his love for him. The feeling of pain and fear he had
experienced when he was being crushed, together with that of
rapture, still further intensified his sense of the importance of
the occasion.

Suddenly the sound of a firing of cannon was heard from the
embankment, to celebrate the signing of peace with the Turks, and
the crowd rushed impetuously toward the embankment to watch the
firing. Petya too would have run there, but the clerk who had
taken the young gentleman under his protection stopped him. The
firing was still proceeding when officers, generals, and
gentlemen-in-waiting came running out of the cathedral, and after
them others in a more leisurely manner: caps were again raised,
and those who had run to look at the cannon ran back again. At
last four men in uniforms and sashes emerged from the cathedral
doors. ``Hurrah! hurrah!'' shouted the crowd again.

``Which is he? Which?'' asked Petya in a tearful voice, of those
around him, but no one answered him, everybody was too excited;
and Petya, fixing on one of those four men, whom he could not
clearly see for the tears of joy that filled his eyes,
concentrated all his enthusiasm on him---though it happened not
to be the Emperor---frantically shouted ``Hurrah!'' and resolved
that tomorrow, come what might, he would join the army.

The crowd ran after the Emperor, followed him to the palace, and
began to disperse. It was already late, and Petya had not eaten
anything and was drenched with perspiration, yet he did not go
home but stood with that diminishing, but still considerable,
crowd before the palace while the Emperor dined---looking in at
the palace windows, expecting he knew not what, and envying alike
the notables he saw arriving at the entrance to dine with the
Emperor and the court footmen who served at table, glimpses of
whom could be seen through the windows.

While the Emperor was dining, Valuev, looking out of the window,
said:

``The people are still hoping to see Your Majesty again.''

The dinner was nearly over, and the Emperor, munching a biscuit,
rose and went out onto the balcony. The people, with Petya among
them, rushed toward the balcony.

``Angel! Dear one! Hurrah! Father!...'' cried the crowd, and
Petya with it, and again the women and men of weaker mold, Petya
among them, wept with joy.

A largish piece of the biscuit the Emperor was holding in his
hand broke off, fell on the balcony parapet, and then to the
ground. A coachman in a jerkin, who stood nearest, sprang forward
and snatched it up. Several people in the crowd rushed at the
coachman. Seeing this the Emperor had a plateful of biscuits
brought him and began throwing them down from the
balcony. Petya's eyes grew bloodshot, and still more excited by
the danger of being crushed, he rushed at the biscuits. He did
not know why, but he had to have a biscuit from the Tsar's hand
and he felt that he must not give way. He sprang forward and
upset an old woman who was catching at a biscuit; the old woman
did not consider herself defeated though she was lying on the
ground---she grabbed at some biscuits but her hand did not reach
them. Petya pushed her hand away with his knee, seized a biscuit,
and as if fearing to be too late, again shouted ``Hurrah!'' with
a voice already hoarse.

The Emperor went in, and after that the greater part of the crowd
began to disperse.

``There! I said if only we waited---and so it was!'' was being
joyfully said by various people.

Happy as Petya was, he felt sad at having to go home knowing that
all the enjoyment of that day was over. He did not go straight
home from the Kremlin, but called on his friend Obolenski, who
was fifteen and was also entering the regiment. On returning home
Petya announced resolutely and firmly that if he was not allowed
to enter the service he would run away. And next day, Count Ilya
Rostov---though he had not yet quite yielded---went to inquire
how he could arrange for Petya to serve where there would be
least danger.

% % % % % % % % % % % % % % % % % % % % % % % % % % % % % % % % %
% % % % % % % % % % % % % % % % % % % % % % % % % % % % % % % % %
% % % % % % % % % % % % % % % % % % % % % % % % % % % % % % % % %
% % % % % % % % % % % % % % % % % % % % % % % % % % % % % % % % %
% % % % % % % % % % % % % % % % % % % % % % % % % % % % % % % % %
% % % % % % % % % % % % % % % % % % % % % % % % % % % % % % % % %
% % % % % % % % % % % % % % % % % % % % % % % % % % % % % % % % %
% % % % % % % % % % % % % % % % % % % % % % % % % % % % % % % % %
% % % % % % % % % % % % % % % % % % % % % % % % % % % % % % % % %
% % % % % % % % % % % % % % % % % % % % % % % % % % % % % % % % %
% % % % % % % % % % % % % % % % % % % % % % % % % % % % % % % % %
% % % % % % % % % % % % % % % % % % % % % % % % % % % % % %

\chapter*{Chapter XXII}
\ifaudio     
\marginpar{
\href{http://ia802702.us.archive.org/23/items/war_and_peace_09_0811_librivox/war_and_peace_09_22_tolstoy_64kb.mp3}{Audio}} 
\fi

\initial{T}{wo} days later, on the fifteenth of July, an immense number of
carriages were standing outside the Sloboda Palace.

The great halls were full. In the first were the nobility and
gentry in their uniforms, in the second bearded merchants in
full-skirted coats of blue cloth and wearing medals. In the
noblemen's hall there was an incessant movement and buzz of
voices. The chief magnates sat on high-backed chairs at a large
table under the portrait of the Emperor, but most of the gentry
were strolling about the room.

All these nobles, whom Pierre met every day at the club or in
their own houses, were in uniform---some in that of Catherine's
day, others in that of Emperor Paul, others again in the new
uniforms of Alexander's time or the ordinary uniform of the
nobility, and the general characteristic of being in uniform
imparted something strange and fantastic to these diverse and
familiar personalities, both old and young. The old men,
dim-eyed, toothless, bald, sallow, and bloated, or gaunt and
wrinkled, were especially striking. For the most part they sat
quietly in their places and were silent, or, if they walked about
and talked, attached themselves to someone younger. On all these
faces, as on the faces of the crowd Petya had seen in the Square,
there was a striking contradiction: the general expectation of a
solemn event, and at the same time the everyday interests in a
boston card party, Peter the cook, Zinaida Dmitrievna's health,
and so on.

Pierre was there too, buttoned up since early morning in a
nobleman's uniform that had become too tight for him. He was
agitated; this extraordinary gathering not only of nobles but
also of the merchant-class---les etats generaux
(States-General)---evoked in him a whole series of ideas he had
long laid aside but which were deeply graven in his soul:
thoughts of the Contrat Social and the French Revolution. The
words that had struck him in the Emperor's appeal---that the
sovereign was coming to the capital for consultation with his
people---strengthened this idea. And imagining that in this
direction something important which he had long awaited was
drawing near, he strolled about watching and listening to
conversations, but nowhere finding any confirmation of the ideas
that occupied him.

The Emperor's manifesto was read, evoking enthusiasm, and then
all moved about discussing it. Besides the ordinary topics of
conversation, Pierre heard questions of where the marshals of the
nobility were to stand when the Emperor entered, when a ball
should be given in the Emperor's honor, whether they should group
themselves by districts or by whole provinces... and so on; but
as soon as the war was touched on, or what the nobility had been
convened for, the talk became undecided and indefinite. Then all
preferred listening to speaking.

A middle-aged man, handsome and virile, in the uniform of a
retired naval officer, was speaking in one of the rooms, and a
small crowd was pressing round him. Pierre went up to the circle
that had formed round the speaker and listened. Count Ilya
Rostov, in a military uniform of Catherine's time, was sauntering
with a pleasant smile among the crowd, with all of whom he was
acquainted. He too approached that group and listened with a
kindly smile and nods of approval, as he always did, to what the
speaker was saying. The retired naval man was speaking very
boldly, as was evident from the expression on the faces of the
listeners and from the fact that some people Pierre knew as the
meekest and quietest of men walked away disapprovingly or
expressed disagreement with him. Pierre pushed his way into the
middle of the group, listened, and convinced himself that the man
was indeed a liberal, but of views quite different from his
own. The naval officer spoke in a particularly sonorous, musical,
and aristocratic baritone voice, pleasantly swallowing his r's
and generally slurring his consonants: the voice of a man calling
out to his servant, ``Heah! Bwing me my pipe!'' It was indicative
of dissipation and the exercise of authority.

``What if the Smolensk people have offahd to waise militia for
the Empewah? Ah we to take Smolensk as our patte'n? If the noble
awistocwacy of the pwovince of Moscow thinks fit, it can show its
loyalty to our sov'weign the Empewah in other ways. Have we
fo'gotten the waising of the militia in the yeah 'seven? All that
did was to enwich the pwiests' sons and thieves and wobbahs...''

Count Ilya Rostov smiled blandly and nodded approval.

``And was our militia of any use to the Empia? Not at all! It
only wuined our farming! Bettah have another conscwiption... o'
ou' men will wetu'n neithah soldiers no' peasants, and we'll get
only depwavity fwom them.  The nobility don't gwudge theah
lives---evewy one of us will go and bwing in more wecwuits, and
the sov'weign'' (that was the way he referred to the Emperor)
``need only say the word and we'll all die fo' him!'' added the
orator with animation.

Count Rostov's mouth watered with pleasure and he nudged Pierre,
but Pierre wanted to speak himself. He pushed forward, feeling
stirred, but not yet sure what stirred him or what he would
say. Scarcely had he opened his mouth when one of the senators, a
man without a tooth in his head, with a shrewd though angry
expression, standing near the first speaker, interrupted
him. Evidently accustomed to managing debates and to maintaining
an argument, he began in low but distinct tones:

``I imagine, sir,'' said he, mumbling with his toothless mouth,
``that we have been summoned here not to discuss whether it's
best for the empire at the present moment to adopt conscription
or to call out the militia.  We have been summoned to reply to
the appeal with which our sovereign the Emperor has honored
us. But to judge what is best---conscription or the militia---we
can leave to the supreme authority...''

Pierre suddenly saw an outlet for his excitement. He hardened his
heart against the senator who was introducing this set and narrow
attitude into the deliberations of the nobility. Pierre stepped
forward and interrupted him. He himself did not yet know what he
would say, but he began to speak eagerly, occasionally lapsing
into French or expressing himself in bookish Russian.

``Excuse me, your excellency,'' he began. (He was well acquainted
with the senator, but thought it necessary on this occasion to
address him formally.) ``Though I don't agree with the
gentleman...'' (he hesitated: he wished to say, ``Mon tres
honorable preopinant''---``My very honorable opponent'') ``with
the gentleman... whom I have not the honor of knowing, I suppose
that the nobility have been summoned not merely to express their
sympathy and enthusiasm but also to consider the means by which
we can assist our Fatherland! I imagine,'' he went on, warming to
his subject, ``that the Emperor himself would not be satisfied to
find in us merely owners of serfs whom we are willing to devote
to his service, and chair a canon\footnote{``Food for cannon.''}
we are ready to make of ourselves---and not to obtain from us any
co-co-counsel.''

Many persons withdrew from the circle, noticing the senator's
sarcastic smile and the freedom of Pierre's remarks. Only Count
Rostov was pleased with them as he had been pleased with those of
the naval officer, the senator, and in general with whatever
speech he had last heard.

``I think that before discussing these questions,'' Pierre
continued, ``we should ask the Emperor---most respectfully ask
His Majesty---to let us know the number of our troops and the
position in which our army and our forces now are, and then...''

But scarcely had Pierre uttered these words before he was
attacked from three sides. The most vigorous attack came from an
old acquaintance, a boston player who had always been well
disposed toward him, Stepan Stepanovich Adraksin. Adraksin was in
uniform, and whether as a result of the uniform or from some
other cause Pierre saw before him quite a different man. With a
sudden expression of malevolence on his aged face, Adraksin
shouted at Pierre:

``In the first place, I tell you we have no right to question the
Emperor about that, and secondly, if the Russian nobility had
that right, the Emperor could not answer such a question. The
troops are moved according to the enemy's movements and the
number of men increases and decreases...''

Another voice, that of a nobleman of medium height and about
forty years of age, whom Pierre had formerly met at the gypsies'
and knew as a bad cardplayer, and who, also transformed by his
uniform, came up to Pierre, interrupted Adraksin.

``Yes, and this is not a time for discussing,'' he continued,
``but for acting: there is war in Russia! The enemy is advancing
to destroy Russia, to desecrate the tombs of our fathers, to
carry off our wives and children.'' The nobleman smote his
breast. ``We will all arise, every one of us will go, for our
father the Tsar!'' he shouted, rolling his bloodshot
eyes. Several approving voices were heard in the crowd. ``We are
Russians and will not grudge our blood in defense of our faith,
the throne, and the Fatherland! We must cease raving if we are
sons of our Fatherland! We will show Europe how Russia rises to
the defense of Russia!''

Pierre wished to reply, but could not get in a word. He felt that
his words, apart from what meaning they conveyed, were less
audible than the sound of his opponent's voice.

Count Rostov at the back of the crowd was expressing approval;
several persons, briskly turning a shoulder to the orator at the
end of a phrase, said:

``That's right, quite right! Just so!''

Pierre wished to say that he was ready to sacrifice his money,
his serfs, or himself, only one ought to know the state of
affairs in order to be able to improve it, but he was unable to
speak. Many voices shouted and talked at the same time, so that
Count Rostov had not time to signify his approval of them all,
and the group increased, dispersed, re-formed, and then moved
with a hum of talk into the largest hall and to the big
table. Not only was Pierre's attempt to speak unsuccessful, but
he was rudely interrupted, pushed aside, and people turned away
from him as from a common enemy. This happened not because they
were displeased by the substance of his speech, which had even
been forgotten after the many subsequent speeches, but to animate
it the crowd needed a tangible object to love and a tangible
object to hate. Pierre became the latter. Many other orators
spoke after the excited nobleman, and all in the same tone. Many
spoke eloquently and with originality.

Glinka, the editor of the Russian Messenger, who was recognized
(cries of ``author! author!'' were heard in the crowd), said that
``hell must be repulsed by hell,'' and that he had seen a child
smiling at lightning flashes and thunderclaps, but ``we will not
be that child.''

``Yes, yes, at thunderclaps!'' was repeated approvingly in the
back rows of the crowd.

The crowd drew up to the large table, at which sat gray-haired or
bald seventy-year-old magnates, uniformed and besashed almost all
of whom Pierre had seen in their own homes with their buffoons,
or playing boston at the clubs. With an incessant hum of voices
the crowd advanced to the table. Pressed by the throng against
the high backs of the chairs, the orators spoke one after another
and sometimes two together.  Those standing behind noticed what a
speaker omitted to say and hastened to supply it. Others in that
heat and crush racked their brains to find some thought and
hastened to utter it. The old magnates, whom Pierre knew, sat and
turned to look first at one and then at another, and their faces
for the most part only expressed the fact that they found it very
hot. Pierre, however, felt excited, and the general desire to
show that they were ready to go to all lengths---which found
expression in the tones and looks more than in the substance of
the speeches---infected him too. He did not renounce his
opinions, but felt himself in some way to blame and wished to
justify himself.

``I only said that it would be more to the purpose to make
sacrifices when we know what is needed!'' said he, trying to be
heard above the other voices.

One of the old men nearest to him looked round, but his attention
was immediately diverted by an exclamation at the other side of
the table.

``Yes, Moscow will be surrendered! She will be our expiation!''
shouted one man.

``He is the enemy of mankind!'' cried another. ``Allow me to
speak...''  ``Gentlemen, you are crushing me!...''

% % % % % % % % % % % % % % % % % % % % % % % % % % % % % % % % %
% % % % % % % % % % % % % % % % % % % % % % % % % % % % % % % % %
% % % % % % % % % % % % % % % % % % % % % % % % % % % % % % % % %
% % % % % % % % % % % % % % % % % % % % % % % % % % % % % % % % %
% % % % % % % % % % % % % % % % % % % % % % % % % % % % % % % % %
% % % % % % % % % % % % % % % % % % % % % % % % % % % % % % % % %
% % % % % % % % % % % % % % % % % % % % % % % % % % % % % % % % %
% % % % % % % % % % % % % % % % % % % % % % % % % % % % % % % % %
% % % % % % % % % % % % % % % % % % % % % % % % % % % % % % % % %
% % % % % % % % % % % % % % % % % % % % % % % % % % % % % % % % %
% % % % % % % % % % % % % % % % % % % % % % % % % % % % % % % % %
% % % % % % % % % % % % % % % % % % % % % % % % % % % % % %

\chapter*{Chapter XXIII}
\ifaudio     
\marginpar{
\href{http://ia802702.us.archive.org/23/items/war_and_peace_09_0811_librivox/war_and_peace_09_23_tolstoy_64kb.mp3}{Audio}} 
\fi

\initial{A}{t} that moment Count Rostopchin with his protruding chin and
alert eyes, wearing the uniform of a general with sash over his
shoulder, entered the room, stepping briskly to the front of the
crowd of gentry.

``Our sovereign the Emperor will be here in a moment,'' said
Rostopchin.  ``I am straight from the palace. Seeing the position
we are in, I think there is little need for discussion. The
Emperor has deigned to summon us and the merchants. Millions will
pour forth from there''---he pointed to the merchants'
hall---``but our business is to supply men and not spare
ourselves... That is the least we can do!''

A conference took place confined to the magnates sitting at the
table.  The whole consultation passed more than quietly. After
all the preceding noise the sound of their old voices saying one
after another, ``I agree,'' or for variety, ``I too am of that
opinion,'' and so on had even a mournful effect.

The secretary was told to write down the resolution of the Moscow
nobility and gentry, that they would furnish ten men, fully
equipped, out of every thousand serfs, as the Smolensk gentry had
done. Their chairs made a scraping noise as the gentlemen who had
conferred rose with apparent relief, and began walking up and
down, arm in arm, to stretch their legs and converse in couples.

``The Emperor! The Emperor!'' a sudden cry resounded through the
halls and the whole throng hurried to the entrance.

The Emperor entered the hall through a broad path between two
lines of nobles. Every face expressed respectful, awe-struck
curiosity. Pierre stood rather far off and could not hear all
that the Emperor said. From what he did hear he understood that
the Emperor spoke of the danger threatening the empire and of the
hopes he placed on the Moscow nobility. He was answered by a
voice which informed him of the resolution just arrived at.

``Gentlemen!'' said the Emperor with a quivering voice.

There was a rustling among the crowd and it again subsided, so
that Pierre distinctly heard the pleasantly human voice of the
Emperor saying with emotion:

``I never doubted the devotion of the Russian nobles, but today
it has surpassed my expectations. I thank you in the name of the
Fatherland!  Gentlemen, let us act! Time is most precious...''

The Emperor ceased speaking, the crowd began pressing round him,
and rapturous exclamations were heard from all sides.

``Yes, most precious... a royal word,'' said Count Rostov, with a
sob. He stood at the back, and, though he had heard hardly
anything, understood everything in his own way.

From the hall of the nobility the Emperor went to that of the
merchants.  There he remained about ten minutes. Pierre was among
those who saw him come out from the merchants' hall with tears of
emotion in his eyes. As became known later, he had scarcely begun
to address the merchants before tears gushed from his eyes and he
concluded in a trembling voice.  When Pierre saw the Emperor he
was coming out accompanied by two merchants, one of whom Pierre
knew, a fat otkupshchik. The other was the mayor, a man with a
thin sallow face and narrow beard. Both were weeping. Tears
filled the thin man's eyes, and the fat otkupshchik sobbed
outright like a child and kept repeating:

``Our lives and property---take them, Your Majesty!''

Pierre's one feeling at the moment was a desire to show that he
was ready to go all lengths and was prepared to sacrifice
everything. He now felt ashamed of his speech with its
constitutional tendency and sought an opportunity of effacing
it. Having heard that Count Mamonov was furnishing a regiment,
Bezukhov at once informed Rostopchin that he would give a
thousand men and their maintenance.

Old Rostov could not tell his wife of what had passed without
tears, and at once consented to Petya's request and went himself
to enter his name.

Next day the Emperor left Moscow. The assembled nobles all took
off their uniforms and settled down again in their homes and
clubs, and not without some groans gave orders to their stewards
about the enrollment, feeling amazed themselves at what they had
done.

