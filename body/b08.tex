\part*{Book Eight: 1811 - 12}

% % % % % % % % % % % % % % % % % % % % % % % % % % % % % % % % %
% % % % % % % % % % % % % % % % % % % % % % % % % % % % % % % % %
% % % % % % % % % % % % % % % % % % % % % % % % % % % % % % % % %
% % % % % % % % % % % % % % % % % % % % % % % % % % % % % % % % %
% % % % % % % % % % % % % % % % % % % % % % % % % % % % % % % % %
% % % % % % % % % % % % % % % % % % % % % % % % % % % % % % % % %
% % % % % % % % % % % % % % % % % % % % % % % % % % % % % % % % %
% % % % % % % % % % % % % % % % % % % % % % % % % % % % % % % % %
% % % % % % % % % % % % % % % % % % % % % % % % % % % % % % % % %
% % % % % % % % % % % % % % % % % % % % % % % % % % % % % % % % %
% % % % % % % % % % % % % % % % % % % % % % % % % % % % % % % % %
% % % % % % % % % % % % % % % % % % % % % % % % % % % % % %

\chapter*{Chapter I}
\ifaudio
\marginpar{
\href{http://ia601406.us.archive.org/23/items/war_and_peace_08_0810_librivox/war_and_peace_08_01_tolstoy_64kb.mp3}{Audio}} 
\fi

\lettrine[lines=2, loversize=0.3, lraise=0]{\initfamily A}{fter}
Prince Andrew's engagement to Natasha, Pierre without any
apparent cause suddenly felt it impossible to go on living as
before. Firmly convinced as he was of the truths revealed to him
by his benefactor, and happy as he had been in perfecting his
inner man, to which he had devoted himself with such ardor---all
the zest of such a life vanished after the engagement of Andrew
and Natasha and the death of Joseph Alexeevich, the news of which
reached him almost at the same time. Only the skeleton of life
remained: his house, a brilliant wife who now enjoyed the favors
of a very important personage, acquaintance with all Petersburg,
and his court service with its dull formalities. And this life
suddenly seemed to Pierre unexpectedly loathsome. He ceased
keeping a diary, avoided the company of the Brothers, began going
to the club again, drank a great deal, and came once more in
touch with the bachelor sets, leading such a life that the
Countess Helene thought it necessary to speak severely to him
about it. Pierre felt that she was right, and to avoid
compromising her went away to Moscow.

In Moscow as soon as he entered his huge house in which the faded
and fading princesses still lived, with its enormous retinue; as
soon as, driving through the town, he saw the Iberian shrine with
innumerable tapers burning before the golden covers of the icons,
the Kremlin Square with its snow undisturbed by vehicles, the
sleigh drivers and hovels of the Sivtsev Vrazhok, those old
Moscovites who desired nothing, hurried nowhere, and were ending
their days leisurely; when he saw those old Moscow ladies, the
Moscow balls, and the English Club, he felt himself at home in a
quiet haven. In Moscow he felt at peace, at home, warm and dirty
as in an old dressing gown.

Moscow society, from the old women down to the children, received
Pierre like a long-expected guest whose place was always ready
awaiting him.  For Moscow society Pierre was the nicest, kindest,
most intellectual, merriest, and most magnanimous of cranks, a
heedless, genial nobleman of the old Russian type. His purse was
always empty because it was open to everyone.

Benefit performances, poor pictures, statues, benevolent
societies, gypsy choirs, schools, subscription dinners, sprees,
Freemasons, churches, and books---no one and nothing met with a
refusal from him, and had it not been for two friends who had
borrowed large sums from him and taken him under their
protection, he would have given everything away.  There was never
a dinner or soiree at the club without him. As soon as he sank
into his place on the sofa after two bottles of Margaux he was
surrounded, and talking, disputing, and joking began. When there
were quarrels, his kindly smile and well-timed jests reconciled
the antagonists. The masonic dinners were dull and dreary when he
was not there.

When after a bachelor supper he rose with his amiable and kindly
smile, yielding to the entreaties of the festive company to drive
off somewhere with them, shouts of delight and triumph arose
among the young men. At balls he danced if a partner was
needed. Young ladies, married and unmarried, liked him because
without making love to any of them, he was equally amiable to
all, especially after supper. ``Il est charmant; il n'a pas de
sexe,''\footnote{``He is charming; he has no sex.''} they said of
him.

Pierre was one of those retired gentlemen-in-waiting of whom
there were hundreds good-humoredly ending their days in Moscow.

How horrified he would have been seven years before, when he
first arrived from abroad, had he been told that there was no
need for him to seek or plan anything, that his rut had long been
shaped, eternally predetermined, and that wriggle as he might, he
would be what all in his position were. He could not have
believed it! Had he not at one time longed with all his heart to
establish a republic in Russia; then himself to be a Napoleon;
then to be a philosopher; and then a strategist and the conqueror
of Napoleon? Had he not seen the possibility of, and passionately
desired, the regeneration of the sinful human race, and his own
progress to the highest degree of perfection?  Had he not
established schools and hospitals and liberated his serfs?

But instead of all that---here he was, the wealthy husband of an
unfaithful wife, a retired gentleman-in-waiting, fond of eating
and drinking and, as he unbuttoned his waistcoat, of abusing the
government a bit, a member of the Moscow English Club, and a
universal favorite in Moscow society. For a long time he could
not reconcile himself to the idea that he was one of those same
retired Moscow gentlemen-in-waiting he had so despised seven
years before.

Sometimes he consoled himself with the thought that he was only
living this life temporarily; but then he was shocked by the
thought of how many, like himself, had entered that life and that
club temporarily, with all their teeth and hair, and had only
left it when not a single tooth or hair remained.

In moments of pride, when he thought of his position it seemed to
him that he was quite different and distinct from those other
retired gentlemen-in-waiting he had formerly despised: they were
empty, stupid, contented fellows, satisfied with their position,
``while I am still discontented and want to do something for
mankind. But perhaps all these comrades of mine struggled just
like me and sought something new, a path in life of their own,
and like me were brought by force of circumstances, society, and
race---by that elemental force against which man is
powerless---to the condition I am in,'' said he to himself in
moments of humility; and after living some time in Moscow he no
longer despised, but began to grow fond of, to respect, and to
pity his comrades in destiny, as he pitied himself.

Pierre no longer suffered moments of despair, hypochondria, and
disgust with life, but the malady that had formerly found
expression in such acute attacks was driven inwards and never
left him for a moment. ``What for? Why? What is going on in the
world?'' he would ask himself in perplexity several times a day,
involuntarily beginning to reflect anew on the meaning of the
phenomena of life; but knowing by experience that there were no
answers to these questions he made haste to turn away from them,
and took up a book, or hurried off to the club or to Apollon
Nikolaevich's, to exchange the gossip of the town.

``Helene, who has never cared for anything but her own body and
is one of the stupidest women in the world,'' thought Pierre,
``is regarded by people as the acme of intelligence and
refinement, and they pay homage to her. Napoleon Bonaparte was
despised by all as long as he was great, but now that he has
become a wretched comedian the Emperor Francis wants to offer him
his daughter in an illegal marriage. The Spaniards, through the
Catholic clergy, offer praise to God for their victory over the
French on the fourteenth of June, and the French, also through
the Catholic clergy, offer praise because on that same fourteenth
of June they defeated the Spaniards. My brother Masons swear by
the blood that they are ready to sacrifice everything for their
neighbor, but they do not give a ruble each to the collections
for the poor, and they intrigue, the Astraea Lodge against the
Manna Seekers, and fuss about an authentic Scotch carpet and a
charter that nobody needs, and the meaning of which the very man
who wrote it does not understand. We all profess the Christian
law of forgiveness of injuries and love of our neighbors, the law
in honor of which we have built in Moscow forty times forty
churches---but yesterday a deserter was knouted to death and a
minister of that same law of love and forgiveness, a priest, gave
the soldier a cross to kiss before his execution.'' So thought
Pierre, and the whole of this general deception which everyone
accepts, accustomed as he was to it, astonished him each time as
if it were something new. ``I understand the deception and
confusion,'' he thought, ``but how am I to tell them all that I
see? I have tried, and have always found that they too in the
depths of their souls understand it as I do, and only try not to
see it.  So it appears that it must be so! But I---what is to
become of me?''  thought he. He had the unfortunate capacity many
men, especially Russians, have of seeing and believing in the
possibility of goodness and truth, but of seeing the evil and
falsehood of life too clearly to be able to take a serious part
in it. Every sphere of work was connected, in his eyes, with evil
and deception. Whatever he tried to be, whatever he engaged in,
the evil and falsehood of it repulsed him and blocked every path
of activity. Yet he had to live and to find occupation. It was
too dreadful to be under the burden of these insoluble problems,
so he abandoned himself to any distraction in order to forget
them. He frequented every kind of society, drank much, bought
pictures, engaged in building, and above all---read.

He read, and read everything that came to hand. On coming home,
while his valets were still taking off his things, he picked up a
book and began to read. From reading he passed to sleeping, from
sleeping to gossip in drawing rooms of the club, from gossip to
carousals and women; from carousals back to gossip, reading, and
wine. Drinking became more and more a physical and also a moral
necessity. Though the doctors warned him that with his corpulence
wine was dangerous for him, he drank a great deal. He was only
quite at ease when having poured several glasses of wine
mechanically into his large mouth he felt a pleasant warmth in
his body, an amiability toward all his fellows, and a readiness
to respond superficially to every idea without probing it
deeply. Only after emptying a bottle or two did he feel dimly
that the terribly tangled skein of life which previously had
terrified him was not as dreadful as he had thought. He was
always conscious of some aspect of that skein, as with a buzzing
in his head after dinner or supper he chatted or listened to
conversation or read. But under the influence of wine he said to
himself: ``It doesn't matter. I'll get it unraveled. I have a
solution ready, but have no time now---I'll think it all out
later on!'' But the later on never came.

In the morning, on an empty stomach, all the old questions
appeared as insoluble and terrible as ever, and Pierre hastily
picked up a book, and if anyone came to see him he was glad.

Sometimes he remembered how he had heard that soldiers in war
when entrenched under the enemy's fire, if they have nothing to
do, try hard to find some occupation the more easily to bear the
danger. To Pierre all men seemed like those soldiers, seeking
refuge from life: some in ambition, some in cards, some in
framing laws, some in women, some in toys, some in horses, some
in politics, some in sport, some in wine, and some in
governmental affairs. ``Nothing is trivial, and nothing is
important, it's all the same---only to save oneself from it as
best one can,'' thought Pierre. ``Only not to see it, that
dreadful it!''

% % % % % % % % % % % % % % % % % % % % % % % % % % % % % % % % %
% % % % % % % % % % % % % % % % % % % % % % % % % % % % % % % % %
% % % % % % % % % % % % % % % % % % % % % % % % % % % % % % % % %
% % % % % % % % % % % % % % % % % % % % % % % % % % % % % % % % %
% % % % % % % % % % % % % % % % % % % % % % % % % % % % % % % % %
% % % % % % % % % % % % % % % % % % % % % % % % % % % % % % % % %
% % % % % % % % % % % % % % % % % % % % % % % % % % % % % % % % %
% % % % % % % % % % % % % % % % % % % % % % % % % % % % % % % % %
% % % % % % % % % % % % % % % % % % % % % % % % % % % % % % % % %
% % % % % % % % % % % % % % % % % % % % % % % % % % % % % % % % %
% % % % % % % % % % % % % % % % % % % % % % % % % % % % % % % % %
% % % % % % % % % % % % % % % % % % % % % % % % % % % % % %

\chapter*{Chapter II}
\ifaudio     
\marginpar{
\href{http://ia601406.us.archive.org/23/items/war_and_peace_08_0810_librivox/war_and_peace_08_02_tolstoy_64kb.mp3}{Audio}} 
\fi

\lettrine[lines=2, loversize=0.3, lraise=0]{\initfamily A}{t}
the beginning of winter Prince Nicholas Bolkonski and his
daughter moved to Moscow. At that time enthusiasm for the Emperor
Alexander's regime had weakened and a patriotic and anti-French
tendency prevailed there, and this, together with his past and
his intellect and his originality, at once made Prince Nicholas
Bolkonski an object of particular respect to the Moscovites and
the center of the Moscow opposition to the government.

The prince had aged very much that year. He showed marked signs
of senility by a tendency to fall asleep, forgetfulness of quite
recent events, remembrance of remote ones, and the childish
vanity with which he accepted the role of head of the Moscow
opposition. In spite of this the old man inspired in all his
visitors alike a feeling of respectful veneration---especially of
an evening when he came in to tea in his old-fashioned coat and
powdered wig and, aroused by anyone, told his abrupt stories of
the past, or uttered yet more abrupt and scathing criticisms of
the present. For them all, that old-fashioned house with its
gigantic mirrors, pre-Revolution furniture, powdered footmen, and
the stern shrewd old man (himself a relic of the past century)
with his gentle daughter and the pretty Frenchwoman who were
reverently devoted to him presented a majestic and agreeable
spectacle. But the visitors did not reflect that besides the
couple of hours during which they saw their host, there were also
twenty-two hours in the day during which the private and intimate
life of the house continued.

Latterly that private life had become very trying for Princess
Mary.  There in Moscow she was deprived of her greatest
pleasures---talks with the pilgrims and the solitude which
refreshed her at Bald Hills---and she had none of the advantages
and pleasures of city life. She did not go out into society;
everyone knew that her father would not let her go anywhere
without him, and his failing health prevented his going out
himself, so that she was not invited to dinners and evening
parties. She had quite abandoned the hope of getting married. She
saw the coldness and malevolence with which the old prince
received and dismissed the young men, possible suitors, who
sometimes appeared at their house. She had no friends: during
this visit to Moscow she had been disappointed in the two who had
been nearest to her. Mademoiselle Bourienne, with whom she had
never been able to be quite frank, had now become unpleasant to
her, and for various reasons Princess Mary avoided her. Julie,
with whom she had corresponded for the last five years, was in
Moscow, but proved to be quite alien to her when they met. Just
then Julie, who by the death of her brothers had become one of
the richest heiresses in Moscow, was in the full whirl of society
pleasures. She was surrounded by young men who, she fancied, had
suddenly learned to appreciate her worth.  Julie was at that
stage in the life of a society woman when she feels that her last
chance of marrying has come and that her fate must be decided now
or never. On Thursdays Princess Mary remembered with a mournful
smile that she now had no one to write to, since Julie---whose
presence gave her no pleasure was here and they met every
week. Like the old emigre who declined to marry the lady with
whom he had spent his evenings for years, she regretted Julie's
presence and having no one to write to. In Moscow Princess Mary
had no one to talk to, no one to whom to confide her sorrow, and
much sorrow fell to her lot just then. The time for Prince
Andrew's return and marriage was approaching, but his request to
her to prepare his father for it had not been carried out; in
fact, it seemed as if matters were quite hopeless, for at every
mention of the young Countess Rostova the old prince (who apart
from that was usually in a bad temper) lost control of
himself. Another lately added sorrow arose from the lessons she
gave her six year-old nephew. To her consternation she detected
in herself in relation to little Nicholas some symptoms of her
father's irritability. However often she told herself that she
must not get irritable when teaching her nephew, almost every
time that, pointer in hand, she sat down to show him the French
alphabet, she so longed to pour her own knowledge quickly and
easily into the child---who was already afraid that Auntie might
at any moment get angry---that at his slightest inattention she
trembled, became flustered and heated, raised her voice, and
sometimes pulled him by the arm and put him in the corner. Having
put him in the corner she would herself begin to cry over her
cruel, evil nature, and little Nicholas, following her example,
would sob, and without permission would leave his corner, come to
her, pull her wet hands from her face, and comfort her.  But what
distressed the princess most of all was her father's
irritability, which was always directed against her and had of
late amounted to cruelty. Had he forced her to prostrate herself
to the ground all night, had he beaten her or made her fetch wood
or water, it would never have entered her mind to think her
position hard; but this loving despot---the more cruel because he
loved her and for that reason tormented himself and her---knew
how not merely to hurt and humiliate her deliberately, but to
show her that she was always to blame for everything. Of late he
had exhibited a new trait that tormented Princess Mary more than
anything else; this was his ever-increasing intimacy with
Mademoiselle Bourienne. The idea that at the first moment of
receiving the news of his son's intentions had occurred to him in
jest---that if Andrew got married he himself would marry
Bourienne---had evidently pleased him, and latterly he had
persistently, and as it seemed to Princess Mary merely to offend
her, shown special endearments to the companion and expressed his
dissatisfaction with his daughter by demonstrations of love of
Bourienne.

One day in Moscow in Princess Mary's presence (she thought her
father did it purposely when she was there) the old prince kissed
Mademoiselle Bourienne's hand and, drawing her to him, embraced
her affectionately.  Princess Mary flushed and ran out of the
room. A few minutes later Mademoiselle Bourienne came into
Princess Mary's room smiling and making cheerful remarks in her
agreeable voice. Princess Mary hastily wiped away her tears, went
resolutely up to Mademoiselle Bourienne, and evidently
unconscious of what she was doing began shouting in angry haste
at the Frenchwoman, her voice breaking: ``It's horrible, vile,
inhuman, to take advantage of the weakness...'' She did not
finish.  ``Leave my room,'' she exclaimed, and burst into sobs.

Next day the prince did not say a word to his daughter, but she
noticed that at dinner he gave orders that Mademoiselle Bourienne
should be served first. After dinner, when the footman handed
coffee and from habit began with the princess, the prince
suddenly grew furious, threw his stick at Philip, and instantly
gave instructions to have him conscripted for the army.

``He doesn't obey... I said it twice... and he doesn't obey! She
is the first person in this house; she's my best friend,'' cried
the prince.  ``And if you allow yourself,'' he screamed in a
fury, addressing Princess Mary for the first time, ``to forget
yourself again before her as you dared to do yesterday, I will
show you who is master in this house. Go!  Don't let me set eyes
on you; beg her pardon!''

Princess Mary asked Mademoiselle Bourienne's pardon, and also her
father's pardon for herself and for Philip the footman, who had
begged for her intervention.

At such moments something like a pride of sacrifice gathered in
her soul. And suddenly that father whom she had judged would look
for his spectacles in her presence, fumbling near them and not
seeing them, or would forget something that had just occurred, or
take a false step with his failing legs and turn to see if anyone
had noticed his feebleness, or, worst of all, at dinner when
there were no visitors to excite him would suddenly fall asleep,
letting his napkin drop and his shaking head sink over his
plate. ``He is old and feeble, and I dare to condemn him!''  she
thought at such moments, with a feeling of revulsion against
herself.

% % % % % % % % % % % % % % % % % % % % % % % % % % % % % % % % %
% % % % % % % % % % % % % % % % % % % % % % % % % % % % % % % % %
% % % % % % % % % % % % % % % % % % % % % % % % % % % % % % % % %
% % % % % % % % % % % % % % % % % % % % % % % % % % % % % % % % %
% % % % % % % % % % % % % % % % % % % % % % % % % % % % % % % % %
% % % % % % % % % % % % % % % % % % % % % % % % % % % % % % % % %
% % % % % % % % % % % % % % % % % % % % % % % % % % % % % % % % %
% % % % % % % % % % % % % % % % % % % % % % % % % % % % % % % % %
% % % % % % % % % % % % % % % % % % % % % % % % % % % % % % % % %
% % % % % % % % % % % % % % % % % % % % % % % % % % % % % % % % %
% % % % % % % % % % % % % % % % % % % % % % % % % % % % % % % % %
% % % % % % % % % % % % % % % % % % % % % % % % % % % % % %

\chapter*{Chapter III}
\ifaudio     
\marginpar{
\href{http://ia601406.us.archive.org/23/items/war_and_peace_08_0810_librivox/war_and_peace_08_03_tolstoy_64kb.mp3}{Audio}} 
\fi

\lettrine[lines=2, loversize=0.3, lraise=0]{\initfamily I}{n}
1811 there was living in Moscow a French
doctor---Metivier---who had rapidly become the fashion. He was
enormously tall, handsome, amiable as Frenchmen are, and was, as
all Moscow said, an extraordinarily clever doctor. He was
received in the best houses not merely as a doctor, but as an
equal.

Prince Nicholas had always ridiculed medicine, but latterly on
Mademoiselle Bourienne's advice had allowed this doctor to visit
him and had grown accustomed to him. Metivier came to see the
prince about twice a week.

On December 6th---St. Nicholas' Day and the prince's name
day---all Moscow came to the prince's front door but he gave
orders to admit no one and to invite to dinner only a small
number, a list of whom he gave to Princess Mary.

Metivier, who came in the morning with his felicitations,
considered it proper in his quality of doctor de forcer la
consigne,\footnote{To force the guard.} as he told Princess Mary,
and went in to see the prince. It happened that on that morning
of his name day the prince was in one of his worst moods. He had
been going about the house all the morning finding fault with
everyone and pretending not to understand what was said to him
and not to be understood himself. Princess Mary well knew this
mood of quiet absorbed querulousness, which generally culminated
in a burst of rage, and she went about all that morning as though
facing a cocked and loaded gun and awaited the inevitable
explosion. Until the doctor's arrival the morning had passed off
safely. After admitting the doctor, Princess Mary sat down with a
book in the drawing room near the door through which she could
hear all that passed in the study.

At first she heard only Metivier's voice, then her father's, then
both voices began speaking at the same time, the door was flung
open, and on the threshold appeared the handsome figure of the
terrified Metivier with his shock of black hair, and the prince
in his dressing gown and fez, his face distorted with fury and
the pupils of his eyes rolled downwards.

``You don't understand?'' shouted the prince, ``but I do! French
spy, slave of Buonaparte, spy, get out of my house! Be off, I
tell you...''

Metivier, shrugging his shoulders, went up to Mademoiselle
Bourienne who at the sound of shouting had run in from an
adjoining room.

``The prince is not very well: bile and rush of blood to the
head. Keep calm, I will call again tomorrow,'' said Metivier; and
putting his fingers to his lips he hastened away.

Through the study door came the sound of slippered feet and the
cry: ``Spies, traitors, traitors everywhere! Not a moment's peace
in my own house!''

After Metivier's departure the old prince called his daughter in,
and the whole weight of his wrath fell on her. She was to blame
that a spy had been admitted. Had he not told her, yes, told her
to make a list, and not to admit anyone who was not on that list?
Then why was that scoundrel admitted? She was the cause of it
all. With her, he said, he could not have a moment's peace and
could not die quietly.

``No, ma'am! We must part, we must part! Understand that,
understand it!  I cannot endure any more,'' he said, and left the
room. Then, as if afraid she might find some means of
consolation, he returned and trying to appear calm added: ``And
don't imagine I have said this in a moment of anger. I am calm. I
have thought it over, and it will be carried out---we must part;
so find some place for yourself...'' But he could not restrain
himself and with the virulence of which only one who loves is
capable, evidently suffering himself, he shook his fists at her
and screamed:

``If only some fool would marry her!'' Then he slammed the door,
sent for Mademoiselle Bourienne, and subsided into his study.

At two o'clock the six chosen guests assembled for dinner.

These guests---the famous Count Rostopchin, Prince Lopukhin with
his nephew, General Chatrov an old war comrade of the prince's,
and of the younger generation Pierre and Boris
Drubetskoy---awaited the prince in the drawing room.

Boris, who had come to Moscow on leave a few days before, had
been anxious to be presented to Prince Nicholas Bolkonski, and
had contrived to ingratiate himself so well that the old prince
in his case made an exception to the rule of not receiving
bachelors in his house.

The prince's house did not belong to what is known as fashionable
society, but his little circle---though not much talked about in
town---was one it was more flattering to be received in than any
other. Boris had realized this the week before when the
commander-in-chief in his presence invited Rostopchin to dinner
on St. Nicholas' Day, and Rostopchin had replied that he could
not come:

``On that day I always go to pay my devotions to the relics of
Prince Nicholas Bolkonski.''

``Oh, yes, yes!'' replied the commander-in-chief. ``How is
he?...''

The small group that assembled before dinner in the lofty
old-fashioned drawing room with its old furniture resembled the
solemn gathering of a court of justice. All were silent or talked
in low tones. Prince Nicholas came in serious and
taciturn. Princess Mary seemed even quieter and more diffident
than usual. The guests were reluctant to address her, feeling
that she was in no mood for their conversation. Count Rostopchin
alone kept the conversation going, now relating the latest town
news, and now the latest political gossip.

Lopukhin and the old general occasionally took part in the
conversation.  Prince Bolkonski listened as a presiding judge
receives a report, only now and then, silently or by a brief
word, showing that he took heed of what was being reported to
him. The tone of the conversation was such as indicated that no
one approved of what was being done in the political
world. Incidents were related evidently confirming the opinion
that everything was going from bad to worse, but whether telling
a story or giving an opinion the speaker always stopped, or was
stopped, at the point beyond which his criticism might touch the
sovereign himself.

At dinner the talk turned on the latest political news:
Napoleon's seizure of the Duke of Oldenburg's territory, and the
Russian Note, hostile to Napoleon, which had been sent to all the
European courts.

``Bonaparte treats Europe as a pirate does a captured vessel,''
said Count Rostopchin, repeating a phrase he had uttered several
times before. ``One only wonders at the long-suffering or
blindness of the crowned heads.  Now the Pope's turn has come and
Bonaparte doesn't scruple to depose the head of the Catholic
Church---yet all keep silent! Our sovereign alone has protested
against the seizure of the Duke of Oldenburg's territory, and
even...'' Count Rostopchin paused, feeling that he had reached
the limit beyond which censure was impossible.

``Other territories have been offered in exchange for the Duchy
of Oldenburg,'' said Prince Bolkonski. ``He shifts the Dukes
about as I might move my serfs from Bald Hills to Bogucharovo or
my Ryazan estates.''

``The Duke of Oldenburg bears his misfortunes with admirable
strength of character and resignation,'' remarked Boris, joining
in respectfully.

He said this because on his journey from Petersburg he had had
the honor of being presented to the Duke. Prince Bolkonski
glanced at the young man as if about to say something in reply,
but changed his mind, evidently considering him too young.

``I have read our protests about the Oldenburg affair and was
surprised how badly the Note was worded,'' remarked Count
Rostopchin in the casual tone of a man dealing with a subject
quite familiar to him.

Pierre looked at Rostopchin with naive astonishment, not
understanding why he should be disturbed by the bad composition
of the Note.

``Does it matter, Count, how the Note is worded,'' he asked, ``so
long as its substance is forcible?''

``My dear fellow, with our five hundred thousand troops it should
be easy to have a good style,'' returned Count Rostopchin.

Pierre now understood the count's dissatisfaction with the
wording of the Note.

``One would have thought quill drivers enough had sprung up,''
remarked the old prince. ``There in Petersburg they are always
writing---not notes only but even new laws. My Andrew there has
written a whole volume of laws for Russia. Nowadays they are
always writing!'' and he laughed unnaturally.

There was a momentary pause in the conversation; the old general
cleared his throat to draw attention.

``Did you hear of the last event at the review in Petersburg? The
figure cut by the new French ambassador.''

``Eh? Yes, I heard something: he said something awkward in His
Majesty's presence.''

``His Majesty drew attention to the Grenadier division and to the
march past,'' continued the general, ``and it seems the
ambassador took no notice and allowed himself to reply that: 'We
in France pay no attention to such trifles!' The Emperor did not
condescend to reply. At the next review, they say, the Emperor
did not once deign to address him.''

All were silent. On this fact relating to the Emperor personally,
it was impossible to pass any judgment.

``Impudent fellows!'' said the prince. ``You know Metivier? I
turned him out of my house this morning. He was here; they
admitted him in spite of my request that they should let no one
in,'' he went on, glancing angrily at his daughter.

And he narrated his whole conversation with the French doctor and
the reasons that convinced him that Metivier was a spy. Though
these reasons were very insufficient and obscure, no one made any
rejoinder.

After the roast, champagne was served. The guests rose to
congratulate the old prince. Princess Mary, too, went round to
him.

He gave her a cold, angry look and offered her his wrinkled,
clean-shaven cheek to kiss. The whole expression of his face told
her that he had not forgotten the morning's talk, that his
decision remained in force, and only the presence of visitors
hindered his speaking of it to her now.

When they went into the drawing room where coffee was served, the
old men sat together.

Prince Nicholas grew more animated and expressed his views on the
impending war.

He said that our wars with Bonaparte would be disastrous so long
as we sought alliances with the Germans and thrust ourselves into
European affairs, into which we had been drawn by the Peace of
Tilsit. ``We ought not to fight either for or against
Austria. Our political interests are all in the East, and in
regard to Bonaparte the only thing is to have an armed frontier
and a firm policy, and he will never dare to cross the Russian
frontier, as was the case in 1807!''

``How can we fight the French, Prince?'' said Count
Rostopchin. ``Can we arm ourselves against our teachers and
divinities? Look at our youths, look at our ladies! The French
are our Gods: Paris is our Kingdom of Heaven.''

He began speaking louder, evidently to be heard by everyone.

``French dresses, French ideas, French feelings! There now, you
turned Metivier out by the scruff of his neck because he is a
Frenchman and a scoundrel, but our ladies crawl after him on
their knees. I went to a party last night, and there out of five
ladies three were Roman Catholics and had the Pope's indulgence
for doing woolwork on Sundays.  And they themselves sit there
nearly naked, like the signboards at our Public Baths if I may
say so. Ah, when one looks at our young people, Prince, one would
like to take Peter the Great's old cudgel out of the museum and
belabor them in the Russian way till all the nonsense jumps out
of them.''

All were silent. The old prince looked at Rostopchin with a smile
and wagged his head approvingly.

``Well, good-by, your excellency, keep well!'' said Rostopchin,
getting up with characteristic briskness and holding out his hand
to the prince.

``Good-bye, my dear fellow... His words are music, I never tire
of hearing him!'' said the old prince, keeping hold of the hand
and offering his cheek to be kissed.

Following Rostopchin's example the others also rose.

% % % % % % % % % % % % % % % % % % % % % % % % % % % % % % % % %
% % % % % % % % % % % % % % % % % % % % % % % % % % % % % % % % %
% % % % % % % % % % % % % % % % % % % % % % % % % % % % % % % % %
% % % % % % % % % % % % % % % % % % % % % % % % % % % % % % % % %
% % % % % % % % % % % % % % % % % % % % % % % % % % % % % % % % %
% % % % % % % % % % % % % % % % % % % % % % % % % % % % % % % % %
% % % % % % % % % % % % % % % % % % % % % % % % % % % % % % % % %
% % % % % % % % % % % % % % % % % % % % % % % % % % % % % % % % %
% % % % % % % % % % % % % % % % % % % % % % % % % % % % % % % % %
% % % % % % % % % % % % % % % % % % % % % % % % % % % % % % % % %
% % % % % % % % % % % % % % % % % % % % % % % % % % % % % % % % %
% % % % % % % % % % % % % % % % % % % % % % % % % % % % % %

\chapter*{Chapter IV}
\ifaudio     
\marginpar{
\href{http://ia601406.us.archive.org/23/items/war_and_peace_08_0810_librivox/war_and_peace_08_03_tolstoy_64kb.mp3}{Audio}} 
\fi

\lettrine[lines=2, loversize=0.3, lraise=0]{\initfamily P}{rincess}
Mary as she sat listening to the old men's talk and
faultfinding, understood nothing of what she heard; she only
wondered whether the guests had all observed her father's hostile
attitude toward her. She did not even notice the special
attentions and amiabilities shown her during dinner by Boris
Drubetskoy, who was visiting them for the third time already.

Princess Mary turned with absent-minded questioning look to
Pierre, who hat in hand and with a smile on his face was the last
of the guests to approach her after the old prince had gone out
and they were left alone in the drawing room.

``May I stay a little longer?'' he said, letting his stout body
sink into an armchair beside her.

``Oh yes,'' she answered. ``You noticed nothing?'' her look
asked.

Pierre was in an agreeable after-dinner mood. He looked straight
before him and smiled quietly.

``Have you known that young man long, Princess?'' he asked.

``Who?''

``Drubetskoy.''

``No, not long...''

``Do you like him?''

``Yes, he is an agreeable young man... Why do you ask me that?''
said Princess Mary, still thinking of that morning's conversation
with her father.

``Because I have noticed that when a young man comes on leave
from Petersburg to Moscow it is usually with the object of
marrying an heiress.''

``You have observed that?'' said Princess Mary.

``Yes,'' returned Pierre with a smile, ``and this young man now
manages matters so that where there is a wealthy heiress there he
is too. I can read him like a book. At present he is hesitating
whom to lay siege to---you or Mademoiselle Julie Karagina. He is
very attentive to her.''

``He visits them?''

``Yes, very often. And do you know the new way of courting?''
said Pierre with an amused smile, evidently in that cheerful mood
of good humored raillery for which he so often reproached himself
in his diary.

``No,'' replied Princess Mary.

``To please Moscow girls nowadays one has to be melancholy. He is
very melancholy with Mademoiselle Karagina,'' said Pierre.

``Really?'' asked Princess Mary, looking into Pierre's kindly
face and still thinking of her own sorrow. ``It would be a
relief,'' thought she, ``if I ventured to confide what I am
feeling to someone. I should like to tell everything to
Pierre. He is kind and generous. It would be a relief. He would
give me advice.''

``Would you marry him?''

``Oh, my God, Count, there are moments when I would marry
anybody!'' she cried suddenly to her own surprise and with tears
in her voice. ``Ah, how bitter it is to love someone near to you
and to feel that...'' she went on in a trembling voice, ``that
you can do nothing for him but grieve him, and to know that you
cannot alter this. Then there is only one thing left---to go
away, but where could I go?''

``What is wrong? What is it, Princess?''

But without finishing what she was saying, Princess Mary burst
into tears.

``I don't know what is the matter with me today. Don't take any
notice---forget what I have said!''

Pierre's gaiety vanished completely. He anxiously questioned the
princess, asked her to speak out fully and confide her grief to
him; but she only repeated that she begged him to forget what she
had said, that she did not remember what she had said, and that
she had no trouble except the one he knew of---that Prince
Andrew's marriage threatened to cause a rupture between father
and son.

``Have you any news of the Rostovs?'' she asked, to change the
subject. ``I was told they are coming soon. I am also expecting
Andrew any day. I should like them to meet here.''

``And how does he now regard the matter?'' asked Pierre,
referring to the old prince.

Princess Mary shook her head.

``What is to be done? In a few months the year will be up. The
thing is impossible. I only wish I could spare my brother the
first moments. I wish they would come sooner. I hope to be
friends with her. You have known them a long time,'' said
Princess Mary. ``Tell me honestly the whole truth: what sort of
girl is she, and what do you think of her?---The real truth,
because you know Andrew is risking so much doing this against his
father's will that I should like to know...''

An undefined instinct told Pierre that these explanations, and
repeated requests to be told the whole truth, expressed ill-will
on the princess' part toward her future sister-in-law and a wish
that he should disapprove of Andrew's choice; but in reply he
said what he felt rather than what he thought.

``I don't know how to answer your question,'' he said, blushing
without knowing why. ``I really don't know what sort of girl she
is; I can't analyze her at all. She is enchanting, but what makes
her so I don't know. That is all one can say about her.''

Princess Mary sighed, and the expression on her face said: ``Yes,
that's what I expected and feared.''

``Is she clever?'' she asked.

Pierre considered.

``I think not,'' he said, ``and yet---yes. She does not deign to
be clever... Oh no, she is simply enchanting, and that is all.''

Princess Mary again shook her head disapprovingly.

``Ah, I so long to like her! Tell her so if you see her before I
do.''

``I hear they are expected very soon,'' said Pierre.

Princess Mary told Pierre of her plan to become intimate with her
future sister-in-law as soon as the Rostovs arrived and to try to
accustom the old prince to her.

% % % % % % % % % % % % % % % % % % % % % % % % % % % % % % % % %
% % % % % % % % % % % % % % % % % % % % % % % % % % % % % % % % %
% % % % % % % % % % % % % % % % % % % % % % % % % % % % % % % % %
% % % % % % % % % % % % % % % % % % % % % % % % % % % % % % % % %
% % % % % % % % % % % % % % % % % % % % % % % % % % % % % % % % %
% % % % % % % % % % % % % % % % % % % % % % % % % % % % % % % % %
% % % % % % % % % % % % % % % % % % % % % % % % % % % % % % % % %
% % % % % % % % % % % % % % % % % % % % % % % % % % % % % % % % %
% % % % % % % % % % % % % % % % % % % % % % % % % % % % % % % % %
% % % % % % % % % % % % % % % % % % % % % % % % % % % % % % % % %
% % % % % % % % % % % % % % % % % % % % % % % % % % % % % % % % %
% % % % % % % % % % % % % % % % % % % % % % % % % % % % % %

\chapter*{Chapter V}
\ifaudio     
\marginpar{
\href{http://ia601406.us.archive.org/23/items/war_and_peace_08_0810_librivox/war_and_peace_08_05_tolstoy_64kb.mp3}{Audio}} 
\fi

\lettrine[lines=2, loversize=0.3, lraise=0]{\initfamily B}{oris}
had not succeeded in making a wealthy match in Petersburg,
so with the same object in view he came to Moscow. There he
wavered between the two richest heiresses, Julie and Princess
Mary. Though Princess Mary despite her plainness seemed to him
more attractive than Julie, he, without knowing why, felt awkward
about paying court to her. When they had last met on the old
prince's name day, she had answered at random all his attempts to
talk sentimentally, evidently not listening to what he was
saying.

Julie on the contrary accepted his attentions readily, though in
a manner peculiar to herself.

She was twenty-seven. After the death of her brothers she had
become very wealthy. She was by now decidedly plain, but thought
herself not merely as good-looking as before but even far more
attractive. She was confirmed in this delusion by the fact that
she had become a very wealthy heiress and also by the fact that
the older she grew the less dangerous she became to men, and the
more freely they could associate with her and avail themselves of
her suppers, soirees, and the animated company that assembled at
her house, without incurring any obligation. A man who would have
been afraid ten years before of going every day to the house when
there was a girl of seventeen there, for fear of compromising her
and committing himself, would now go boldly every day and treat
her not as a marriageable girl but as a sexless acquaintance.

That winter the Karagins' house was the most agreeable and
hospitable in Moscow. In addition to the formal evening and
dinner parties, a large company, chiefly of men, gathered there
every day, supping at midnight and staying till three in the
morning. Julie never missed a ball, a promenade, or a play. Her
dresses were always of the latest fashion. But in spite of that
she seemed to be disillusioned about everything and told everyone
that she did not believe either in friendship or in love, or any
of the joys of life, and expected peace only \emph{yonder}. She
adopted the tone of one who has suffered a great disappointment,
like a girl who has either lost the man she loved or been cruelly
deceived by him. Though nothing of the kind had happened to her
she was regarded in that light, and had even herself come to
believe that she had suffered much in life. This melancholy,
which did not prevent her amusing herself, did not hinder the
young people who came to her house from passing the time
pleasantly. Every visitor who came to the house paid his tribute
to the melancholy mood of the hostess, and then amused himself
with society gossip, dancing, intellectual games, and bouts
rimes, which were in vogue at the Karagins'. Only a few of these
young men, among them Boris, entered more deeply into Julie's
melancholy, and with these she had prolonged conversations in
private on the vanity of all worldly things, and to them she
showed her albums filled with mournful sketches, maxims, and
verses.

To Boris, Julie was particularly gracious: she regretted his
early disillusionment with life, offered him such consolation of
friendship as she who had herself suffered so much could render,
and showed him her album. Boris sketched two trees in the album
and wrote: ``Rustic trees, your dark branches shed gloom and
melancholy upon me.''

On another page he drew a tomb, and wrote:

\begin{quote} \calli
La mort est secourable et la mort est
tranquille. Ah! contre les douleurs il n'y a pas d'autre
asile.\footnote{Death gives relief and death is peaceful.}

  Ah! from suffering there is no other refuge.
\end{quote}

Julie said this was charming

``There is something so enchanting in the smile of melancholy,''
she said to Boris, repeating word for word a passage she had
copied from a book.  ``It is a ray of light in the darkness, a
shade between sadness and despair, showing the possibility of
consolation.''

In reply Boris wrote these lines: 

\begin{quote} \calli
Aliment de poison d'une ame trop sensible, Toi,
sans qui le bonheur me serait impossible, Tendre melancholie, ah,
viens me consoler, Viens calmer les tourments de ma sombre
retraite, Et mele une douceur secrete A ces pleurs que je sens
couler.\footnote{Poisonous nourishment of a too sensitive soul,
Thou, without whom happiness would for me be impossible, Tender
melancholy, ah, come to console me, Come to calm the torments of
my gloomy retreat, And mingle a secret sweetness With these tears
that I feel to be flowing.}
\end{quote}

For Boris, Julie played most doleful nocturnes on her harp. Boris
read 'Poor Liza' aloud to her, and more than once interrupted the
reading because of the emotions that choked him. Meeting at large
gatherings Julie and Boris looked on one another as the only
souls who understood one another in a world of indifferent
people.

Anna Mikhaylovna, who often visited the Karagins, while playing
cards with the mother made careful inquiries as to Julie's dowry
(she was to have two estates in Penza and the Nizhegorod
forests). Anna Mikhaylovna regarded the refined sadness that
united her son to the wealthy Julie with emotion, and resignation
to the Divine will.

``You are always charming and melancholy, my dear Julie,'' she
said to the daughter. ``Boris says his soul finds repose at your
house. He has suffered so many disappointments and is so
sensitive,'' said she to the mother. ``Ah, my dear, I can't tell
you how fond I have grown of Julie latterly,'' she said to her
son. ``But who could help loving her? She is an angelic being!
Ah, Boris, Boris!''---she paused. ``And how I pity her mother,''
she went on; ``today she showed me her accounts and letters from
Penza (they have enormous estates there), and she, poor thing,
has no one to help her, and they do cheat her so!''

Boris smiled almost imperceptibly while listening to his
mother. He laughed blandly at her naive diplomacy but listened to
what she had to say, and sometimes questioned her carefully about
the Penza and Nizhegorod estates.

Julie had long been expecting a proposal from her melancholy
adorer and was ready to accept it; but some secret feeling of
repulsion for her, for her passionate desire to get married, for
her artificiality, and a feeling of horror at renouncing the
possibility of real love still restrained Boris. His leave was
expiring. He spent every day and whole days at the Karagins', and
every day on thinking the matter over told himself that he would
propose tomorrow. But in Julie's presence, looking at her red
face and chin (nearly always powdered), her moist eyes, and her
expression of continual readiness to pass at once from melancholy
to an unnatural rapture of married bliss, Boris could not utter
the decisive words, though in imagination he had long regarded
himself as the possessor of those Penza and Nizhegorod estates
and had apportioned the use of the income from them. Julie saw
Boris' indecision, and sometimes the thought occurred to her that
she was repulsive to him, but her feminine self-deception
immediately supplied her with consolation, and she told herself
that he was only shy from love. Her melancholy, however, began to
turn to irritability, and not long before Boris' departure she
formed a definite plan of action. Just as Boris' leave of absence
was expiring, Anatole Kuragin made his appearance in Moscow, and
of course in the Karagins' drawing room, and Julie, suddenly
abandoning her melancholy, became cheerful and very attentive to
Kuragin.

``My dear,'' said Anna Mikhaylovna to her son, ``I know from a
reliable source that Prince Vasili has sent his son to Moscow to
get him married to Julie. I am so fond of Julie that I should be
sorry for her. What do you think of it, my dear?''

The idea of being made a fool of and of having thrown away that
whole month of arduous melancholy service to Julie, and of seeing
all the revenue from the Penza estates which he had already
mentally apportioned and put to proper use fall into the hands of
another, and especially into the hands of that idiot Anatole,
pained Boris. He drove to the Karagins' with the firm intention
of proposing. Julie met him in a gay, careless manner, spoke
casually of how she had enjoyed yesterday's ball, and asked when
he was leaving. Though Boris had come intentionally to speak of
his love and therefore meant to be tender, he began speaking
irritably of feminine inconstancy, of how easily women can turn
from sadness to joy, and how their moods depend solely on who
happens to be paying court to them. Julie was offended and
replied that it was true that a woman needs variety, and the same
thing over and over again would weary anyone.

``Then I should advise you...'' Boris began, wishing to sting
her; but at that instant the galling thought occurred to him that
he might have to leave Moscow without having accomplished his
aim, and have vainly wasted his efforts---which was a thing he
never allowed to happen.

He checked himself in the middle of the sentence, lowered his
eyes to avoid seeing her unpleasantly irritated and irresolute
face, and said:

``I did not come here at all to quarrel with you. On the
contrary...''

He glanced at her to make sure that he might go on. Her
irritability had suddenly quite vanished, and her anxious,
imploring eyes were fixed on him with greedy expectation. ``I can
always arrange so as not to see her often,'' thought Boris. ``The
affair has been begun and must be finished!''  He blushed hotly,
raised his eyes to hers, and said:

``You know my feelings for you!''

There was no need to say more: Julie's face shone with triumph
and self-satisfaction; but she forced Boris to say all that is
said on such occasions---that he loved her and had never loved
any other woman more than her. She knew that for the Penza
estates and Nizhegorod forests she could demand this, and she
received what she demanded.

The affianced couple, no longer alluding to trees that shed gloom
and melancholy upon them, planned the arrangements of a splendid
house in Petersburg, paid calls, and prepared everything for a
brilliant wedding.

% % % % % % % % % % % % % % % % % % % % % % % % % % % % % % % % %
% % % % % % % % % % % % % % % % % % % % % % % % % % % % % % % % %
% % % % % % % % % % % % % % % % % % % % % % % % % % % % % % % % %
% % % % % % % % % % % % % % % % % % % % % % % % % % % % % % % % %
% % % % % % % % % % % % % % % % % % % % % % % % % % % % % % % % %
% % % % % % % % % % % % % % % % % % % % % % % % % % % % % % % % %
% % % % % % % % % % % % % % % % % % % % % % % % % % % % % % % % %
% % % % % % % % % % % % % % % % % % % % % % % % % % % % % % % % %
% % % % % % % % % % % % % % % % % % % % % % % % % % % % % % % % %
% % % % % % % % % % % % % % % % % % % % % % % % % % % % % % % % %
% % % % % % % % % % % % % % % % % % % % % % % % % % % % % % % % %
% % % % % % % % % % % % % % % % % % % % % % % % % % % % % %

\chapter*{Chapter VI}
\ifaudio     
\marginpar{
\href{http://ia601406.us.archive.org/23/items/war_and_peace_08_0810_librivox/war_and_peace_08_06_tolstoy_64kb.mp3}{Audio}} 
\fi

\lettrine[lines=2, loversize=0.3, lraise=0]{\initfamily A}{t}
the end of January old Count Rostov went to Moscow with
Natasha and Sonya. The countess was still unwell and unable to
travel but it was impossible to wait for her recovery. Prince
Andrew was expected in Moscow any day, the trousseau had to be
ordered and the estate near Moscow had to be sold, besides which
the opportunity of presenting his future daughter-in-law to old
Prince Bolkonski while he was in Moscow could not be missed. The
Rostovs' Moscow house had not been heated that winter and, as
they had come only for a short time and the countess was not with
them, the count decided to stay with Marya Dmitrievna
Akhrosimova, who had long been pressing her hospitality on them.

Late one evening the Rostovs' four sleighs drove into Marya
Dmitrievna's courtyard in the old Konyusheny street. Marya
Dmitrievna lived alone.  She had already married off her
daughter, and her sons were all in the service.

She held herself as erect, told everyone her opinion as candidly,
loudly, and bluntly as ever, and her whole bearing seemed a
reproach to others for any weakness, passion, or temptation---the
possibility of which she did not admit. From early in the
morning, wearing a dressing jacket, she attended to her household
affairs, and then she drove out: on holy days to church and after
the service to jails and prisons on affairs of which she never
spoke to anyone. On ordinary days, after dressing, she received
petitioners of various classes, of whom there were always
some. Then she had dinner, a substantial and appetizing meal at
which there were always three or four guests; after dinner she
played a game of boston, and at night she had the newspapers or a
new book read to her while she knitted. She rarely made an
exception and went out to pay visits, and then only to the most
important persons in the town.

She had not yet gone to bed when the Rostovs arrived and the
pulley of the hall door squeaked from the cold as it let in the
Rostovs and their servants. Marya Dmitrievna, with her spectacles
hanging down on her nose and her head flung back, stood in the
hall doorway looking with a stern, grim face at the new
arrivals. One might have thought she was angry with the travelers
and would immediately turn them out, had she not at the same time
been giving careful instructions to the servants for the
accommodation of the visitors and their belongings.

``The count's things? Bring them here,'' she said, pointing to
the portmanteaus and not greeting anyone. ``The young ladies'?
There to the left. Now what are you dawdling for?'' she cried to
the maids. ``Get the samovar ready!... You've grown plumper and
prettier,'' she remarked, drawing Natasha (whose cheeks were
glowing from the cold) to her by the hood. ``Foo! You are cold!
Now take off your things, quick!'' she shouted to the count who
was going to kiss her hand. ``You're half frozen, I'm sure! Bring
some rum for tea!... Bonjour, Sonya dear!'' she added, turning to
Sonya and indicating by this French greeting her slightly
contemptuous though affectionate attitude toward her.

When they came in to tea, having taken off their outdoor things
and tidied themselves up after their journey, Marya Dmitrievna
kissed them all in due order.

``I'm heartily glad you have come and are staying with me. It was
high time,'' she said, giving Natasha a significant look. ``The
old man is here and his son's expected any day. You'll have to
make his acquaintance.  But we'll speak of that later on,'' she
added, glancing at Sonya with a look that showed she did not want
to speak of it in her presence. ``Now listen,'' she said to the
count. ``What do you want tomorrow? Whom will you send for?
Shinshin?'' she crooked one of her fingers. ``The sniveling Anna
Mikhaylovna? That's two. She's here with her son. The son is
getting married! Then Bezukhov, eh? He is here too, with his
wife. He ran away from her and she came galloping after him. He
dined with me on Wednesday. As for them''---and she pointed to
the girls---``tomorrow I'll take them first to the Iberian shrine
of the Mother of God, and then we'll drive to the
Super-Rogue's. I suppose you'll have everything new.  Don't judge
by me: sleeves nowadays are this size! The other day young
Princess Irina Vasilevna came to see me; she was an awful
sight---looked as if she had put two barrels on her arms. You
know not a day passes now without some new fashion... And what
have you to do yourself?'' she asked the count sternly.

``One thing has come on top of another: her rags to buy, and now
a purchaser has turned up for the Moscow estate and for the
house. If you will be so kind, I'll fix a time and go down to the
estate just for a day, and leave my lassies with you.''

``All right. All right. They'll be safe with me, as safe as in
Chancery!  I'll take them where they must go, scold them a bit,
and pet them a bit,'' said Marya Dmitrievna, touching her
goddaughter and favorite, Natasha, on the cheek with her large
hand.

Next morning Marya Dmitrievna took the young ladies to the
Iberian shrine of the Mother of God and to Madame Suppert-Roguet,
who was so afraid of Marya Dmitrievna that she always let her
have costumes at a loss merely to get rid of her. Marya
Dmitrievna ordered almost the whole trousseau. When they got home
she turned everybody out of the room except Natasha, and then
called her pet to her armchair.

``Well, now we'll talk. I congratulate you on your
betrothed. You've hooked a fine fellow! I am glad for your sake
and I've known him since he was so high.'' She held her hand a
couple of feet from the ground.  Natasha blushed happily. ``I
like him and all his family. Now listen! You know that old Prince
Nicholas much dislikes his son's marrying. The old fellow's
crotchety! Of course Prince Andrew is not a child and can shift
without him, but it's not nice to enter a family against a
father's will. One wants to do it peacefully and lovingly. You're
a clever girl and you'll know how to manage. Be kind, and use
your wits. Then all will be well.''

Natasha remained silent, from shyness Marya Dmitrievna supposed,
but really because she disliked anyone interfering in what
touched her love of Prince Andrew, which seemed to her so apart
from all human affairs that no one could understand it. She loved
and knew Prince Andrew, he loved her only, and was to come one of
these days and take her. She wanted nothing more.

``You see I have known him a long time and am also fond of Mary,
your future sister-in-law. 'Husbands' sisters bring up blisters,'
but this one wouldn't hurt a fly. She has asked me to bring you
two together.  Tomorrow you'll go with your father to see her. Be
very nice and affectionate to her: you're younger than she. When
he comes, he'll find you already know his sister and father and
are liked by them. Am I right or not? Won't that be best?''

``Yes, it will,'' Natasha answered reluctantly.

% % % % % % % % % % % % % % % % % % % % % % % % % % % % % % % % %
% % % % % % % % % % % % % % % % % % % % % % % % % % % % % % % % %
% % % % % % % % % % % % % % % % % % % % % % % % % % % % % % % % %
% % % % % % % % % % % % % % % % % % % % % % % % % % % % % % % % %
% % % % % % % % % % % % % % % % % % % % % % % % % % % % % % % % %
% % % % % % % % % % % % % % % % % % % % % % % % % % % % % % % % %
% % % % % % % % % % % % % % % % % % % % % % % % % % % % % % % % %
% % % % % % % % % % % % % % % % % % % % % % % % % % % % % % % % %
% % % % % % % % % % % % % % % % % % % % % % % % % % % % % % % % %
% % % % % % % % % % % % % % % % % % % % % % % % % % % % % % % % %
% % % % % % % % % % % % % % % % % % % % % % % % % % % % % % % % %
% % % % % % % % % % % % % % % % % % % % % % % % % % % % % %

\chapter*{Chapter VII}
\ifaudio     
\marginpar{
\href{http://ia601406.us.archive.org/23/items/war_and_peace_08_0810_librivox/war_and_peace_08_07_tolstoy_64kb.mp3}{Audio}} 
\fi

\lettrine[lines=2, loversize=0.3, lraise=0]{\initfamily N}{ext}
day, by Marya Dmitrievna's advice, Count Rostov took Natasha
to call on Prince Nicholas Bolkonski. The count did not set out
cheerfully on this visit, at heart he felt afraid. He well
remembered the last interview he had had with the old prince at
the time of the enrollment, when in reply to an invitation to
dinner he had had to listen to an angry reprimand for not having
provided his full quota of men. Natasha, on the other hand,
having put on her best gown, was in the highest spirits. ``They
can't help liking me,'' she thought. ``Everybody always has liked
me, and I am so willing to do anything they wish, so ready to be
fond of him---for being his father---and of her---for being his
sister---that there is no reason for them not to like me...''

They drove up to the gloomy old house on the Vozdvizhenka and
entered the vestibule.

``Well, the Lord have mercy on us!'' said the count, half in
jest, half in earnest; but Natasha noticed that her father was
flurried on entering the anteroom and inquired timidly and softly
whether the prince and princess were at home.

When they had been announced a perturbation was noticeable among
the servants. The footman who had gone to announce them was
stopped by another in the large hall and they whispered to one
another. Then a maidservant ran into the hall and hurriedly said
something, mentioning the princess. At last an old, cross looking
footman came and announced to the Rostovs that the prince was not
receiving, but that the princess begged them to walk up. The
first person who came to meet the visitors was Mademoiselle
Bourienne. She greeted the father and daughter with special
politeness and showed them to the princess' room. The princess,
looking excited and nervous, her face flushed in patches, ran in
to meet the visitors, treading heavily, and vainly trying to
appear cordial and at ease. From the first glance Princess Mary
did not like Natasha. She thought her too fashionably dressed,
frivolously gay and vain. She did not at all realize that before
having seen her future sister-in-law she was prejudiced against
her by involuntary envy of her beauty, youth, and happiness, as
well as by jealousy of her brother's love for her. Apart from
this insuperable antipathy to her, Princess Mary was agitated
just then because on the Rostovs' being announced, the old prince
had shouted that he did not wish to see them, that Princess Mary
might do so if she chose, but they were not to be admitted to
him. She had decided to receive them, but feared lest the prince
might at any moment indulge in some freak, as he seemed much
upset by the Rostovs' visit.

``There, my dear princess, I've brought you my songstress,'' said
the count, bowing and looking round uneasily as if afraid the old
prince might appear. ``I am so glad you should get to know one
another... very sorry the prince is still ailing,'' and after a
few more commonplace remarks he rose. ``If you'll allow me to
leave my Natasha in your hands for a quarter of an hour,
Princess, I'll drive round to see Anna Semenovna, it's quite near
in the Dogs' Square, and then I'll come back for her.''

The count had devised this diplomatic ruse (as he afterwards told
his daughter) to give the future sisters-in-law an opportunity to
talk to one another freely, but another motive was to avoid the
danger of encountering the old prince, of whom he was afraid. He
did not mention this to his daughter, but Natasha noticed her
father's nervousness and anxiety and felt mortified by it. She
blushed for him, grew still angrier at having blushed, and looked
at the princess with a bold and defiant expression which said
that she was not afraid of anybody. The princess told the count
that she would be delighted, and only begged him to stay longer
at Anna Semenovna's, and he departed.

Despite the uneasy glances thrown at her by Princess Mary---who
wished to have a tête-à-tête with Natasha---Mademoiselle
Bourienne remained in the room and persistently talked about
Moscow amusements and theaters.  Natasha felt offended by the
hesitation she had noticed in the anteroom, by her father's
nervousness, and by the unnatural manner of the princess
who---she thought---was making a favor of receiving her, and so
everything displeased her. She did not like Princess Mary, whom
she thought very plain, affected, and dry. Natasha suddenly
shrank into herself and involuntarily assumed an offhand air
which alienated Princess Mary still more. After five minutes of
irksome, constrained conversation, they heard the sound of
slippered feet rapidly approaching. Princess Mary looked
frightened.

The door opened and the old prince, in a dressing gown and a
white nightcap, came in.

``Ah, madam!'' he began. ``Madam, Countess... Countess Rostova,
if I am not mistaken... I beg you to excuse me, to excuse me... I
did not know, madam. God is my witness, I did not know you had
honored us with a visit, and I came in such a costume only to see
my daughter. I beg you to excuse me... God is my witness, I
didn't know-'' he repeated, stressing the word ``God'' so
unnaturally and so unpleasantly that Princess Mary stood with
downcast eyes not daring to look either at her father or at
Natasha.

Nor did the latter, having risen and curtsied, know what to do.
Mademoiselle Bourienne alone smiled agreeably.

``I beg you to excuse me, excuse me! God is my witness, I did not
know,'' muttered the old man, and after looking Natasha over from
head to foot he went out.

Mademoiselle Bourienne was the first to recover herself after
this apparition and began speaking about the prince's
indisposition. Natasha and Princess Mary looked at one another in
silence, and the longer they did so without saying what they
wanted to say, the greater grew their antipathy to one another.

When the count returned, Natasha was impolitely pleased and
hastened to get away: at that moment she hated the stiff, elderly
princess, who could place her in such an embarrassing position
and had spent half an hour with her without once mentioning
Prince Andrew. ``I couldn't begin talking about him in the
presence of that Frenchwoman,'' thought Natasha.  The same
thought was meanwhile tormenting Princess Mary. She knew what she
ought to have said to Natasha, but she had been unable to say it
because Mademoiselle Bourienne was in the way, and because,
without knowing why, she felt it very difficult to speak of the
marriage. When the count was already leaving the room, Princess
Mary went up hurriedly to Natasha, took her by the hand, and said
with a deep sigh:

``Wait, I must...''

Natasha glanced at her ironically without knowing why.

``Dear Natalie,'' said Princess Mary, ``I want you to know that I
am glad my brother has found happiness...''

She paused, feeling that she was not telling the truth. Natasha
noticed this and guessed its reason.

``I think, Princess, it is not convenient to speak of that now,''
she said with external dignity and coldness, though she felt the
tears choking her.

``What have I said and what have I done?'' thought she, as soon
as she was out of the room.

They waited a long time for Natasha to come to dinner that
day. She sat in her room crying like a child, blowing her nose
and sobbing. Sonya stood beside her, kissing her hair.

``Natasha, what is it about?'' she asked. ``What do they matter
to you? It will all pass, Natasha.''

``But if you only knew how offensive it was... as if I...''

``Don't talk about it, Natasha. It wasn't your fault so why
should you mind? Kiss me,'' said Sonya.

Natasha raised her head and, kissing her friend on the lips,
pressed her wet face against her.

``I can't tell you, I don't know. No one's to blame,'' said
Natasha---``It's my fault. But it all hurts terribly. Oh, why
doesn't he come?...''

She came in to dinner with red eyes. Marya Dmitrievna, who knew
how the prince had received the Rostovs, pretended not to notice
how upset Natasha was and jested resolutely and loudly at table
with the count and the other guests.

% % % % % % % % % % % % % % % % % % % % % % % % % % % % % % % % %
% % % % % % % % % % % % % % % % % % % % % % % % % % % % % % % % %
% % % % % % % % % % % % % % % % % % % % % % % % % % % % % % % % %
% % % % % % % % % % % % % % % % % % % % % % % % % % % % % % % % %
% % % % % % % % % % % % % % % % % % % % % % % % % % % % % % % % %
% % % % % % % % % % % % % % % % % % % % % % % % % % % % % % % % %
% % % % % % % % % % % % % % % % % % % % % % % % % % % % % % % % %
% % % % % % % % % % % % % % % % % % % % % % % % % % % % % % % % %
% % % % % % % % % % % % % % % % % % % % % % % % % % % % % % % % %
% % % % % % % % % % % % % % % % % % % % % % % % % % % % % % % % %
% % % % % % % % % % % % % % % % % % % % % % % % % % % % % % % % %
% % % % % % % % % % % % % % % % % % % % % % % % % % % % % %

\chapter*{Chapter VIII}
\ifaudio     
\marginpar{
\href{http://ia601406.us.archive.org/23/items/war_and_peace_08_0810_librivox/war_and_peace_08_08_tolstoy_64kb.mp3}{Audio}} 
\fi

\lettrine[lines=2, loversize=0.3, lraise=0]{\initfamily T}{hat}
evening the Rostovs went to the Opera, for which Marya
Dmitrievna had taken a box.

Natasha did not want to go, but could not refuse Marya
Dmitrievna's kind offer which was intended expressly for
her. When she came ready dressed into the ballroom to await her
father, and looking in the large mirror there saw that she was
pretty, very pretty, she felt even more sad, but it was a sweet,
tender sadness.

``O God, if he were here now I would not behave as I did then,
but differently. I would not be silly and afraid of things, I
would simply embrace him, cling to him, and make him look at me
with those searching inquiring eyes with which he has so often
looked at me, and then I would make him laugh as he used to
laugh. And his eyes---how I see those eyes!''  thought
Natasha. ``And what do his father and sister matter to me? I love
him alone, him, him, with that face and those eyes, with his
smile, manly and yet childlike... No, I had better not think of
him; not think of him but forget him, quite forget him for the
present. I can't bear this waiting and I shall cry in a minute!''
and she turned away from the glass, making an effort not to
cry. ``And how can Sonya love Nicholas so calmly and quietly and
wait so long and so patiently?'' thought she, looking at Sonya,
who also came in quite ready, with a fan in her hand.  ``No,
she's altogether different. I can't!''

Natasha at that moment felt so softened and tender that it was
not enough for her to love and know she was beloved, she wanted
now, at once, to embrace the man she loved, to speak and hear
from him words of love such as filled her heart. While she sat in
the carriage beside her father, pensively watching the lights of
the street lamps flickering on the frozen window, she felt still
sadder and more in love, and forgot where she was going and with
whom. Having fallen into the line of carriages, the Rostovs'
carriage drove up to the theater, its wheels squeaking over the
snow. Natasha and Sonya, holding up their dresses, jumped out
quickly. The count got out helped by the footmen, and, passing
among men and women who were entering and the program sellers,
they all three went along the corridor to the first row of boxes.
Through the closed doors the music was already audible.

``Natasha, your hair!...'' whispered Sonya.

An attendant deferentially and quickly slipped before the ladies
and opened the door of their box. The music sounded louder and
through the door rows of brightly lit boxes in which ladies sat
with bare arms and shoulders, and noisy stalls brilliant with
uniforms, glittered before their eyes. A lady entering the next
box shot a glance of feminine envy at Natasha. The curtain had
not yet risen and the overture was being played. Natasha,
smoothing her gown, went in with Sonya and sat down, scanning the
brilliant tiers of boxes opposite. A sensation she had not
experienced for a long time---that of hundreds of eyes looking at
her bare arms and neck---suddenly affected her both agreeably and
disagreeably and called up a whole crowd of memories, desires and
emotions associated with that feeling.

The two remarkably pretty girls, Natasha and Sonya, with Count
Rostov who had not been seen in Moscow for a long time, attracted
general attention. Moreover, everybody knew vaguely of Natasha's
engagement to Prince Andrew, and knew that the Rostovs had lived
in the country ever since, and all looked with curiosity at a
fiancee who was making one of the best matches in Russia.

Natasha's looks, as everyone told her, had improved in the
country, and that evening thanks to her agitation she was
particularly pretty. She struck those who saw her by her fullness
of life and beauty, combined with her indifference to everything
about her. Her black eyes looked at the crowd without seeking
anyone, and her delicate arm, bare to above the elbow, lay on the
velvet edge of the box, while, evidently unconsciously, she
opened and closed her hand in time to the music, crumpling her
program. ``Look, there's Alenina,'' said Sonya, ``with her
mother, isn't it?''

``Dear me, Michael Kirilovich has grown still stouter!'' remarked
the count.

``Look at our Anna Mikhaylovna---what a headdress she has on!''

``The Karagins, Julie---and Boris with them. One can see at once
that they're engaged...''

``Drubetskoy has proposed?''

``Oh yes, I heard it today,'' said Shinshin, coming into the
Rostovs' box.

Natasha looked in the direction in which her father's eyes were
turned and saw Julie sitting beside her mother with a happy look
on her face and a string of pearls round her thick red
neck---which Natasha knew was covered with powder. Behind them,
wearing a smile and leaning over with an ear to Julie's mouth,
was Boris' handsome smoothly brushed head. He looked at the
Rostovs from under his brows and said something, smiling, to his
betrothed.

``They are talking about us, about me and him!'' thought
Natasha. ``And he no doubt is calming her jealousy of me. They
needn't trouble themselves!  If only they knew how little I am
concerned about any of them.''

Behind them sat Anna Mikhaylovna wearing a green headdress and
with a happy look of resignation to the will of God on her
face. Their box was pervaded by that atmosphere of an affianced
couple which Natasha knew so well and liked so much. She turned
away and suddenly remembered all that had been so humiliating in
her morning's visit.

``What right has he not to wish to receive me into his family?
Oh, better not think of it---not till he comes back!'' she told
herself, and began looking at the faces, some strange and some
familiar, in the stalls. In the front, in the very center,
leaning back against the orchestra rail, stood Dolokhov in a
Persian dress, his curly hair brushed up into a huge shock. He
stood in full view of the audience, well aware that he was
attracting everyone's attention, yet as much at ease as though he
were in his own room. Around him thronged Moscow's most brilliant
young men, whom he evidently dominated.

The count, laughing, nudged the blushing Sonya and pointed to her
former adorer.

``Do you recognize him?'' said he. ``And where has he sprung
from?'' he asked, turning to Shinshin. ``Didn't he vanish
somewhere?''

``He did,'' replied Shinshin. ``He was in the Caucasus and ran
away from there. They say he has been acting as minister to some
ruling prince in Persia, where he killed the Shah's brother. Now
all the Moscow ladies are mad about him! It's 'Dolokhov the
Persian' that does it! We never hear a word but Dolokhov is
mentioned. They swear by him, they offer him to you as they would
a dish of choice sterlet. Dolokhov and Anatole Kuragin have
turned all our ladies' heads.''

A tall, beautiful woman with a mass of plaited hair and much
exposed plump white shoulders and neck, round which she wore a
double string of large pearls, entered the adjoining box rustling
her heavy silk dress and took a long time settling into her
place.

Natasha involuntarily gazed at that neck, those shoulders, and
pearls and coiffure, and admired the beauty of the shoulders and
the pearls.  While Natasha was fixing her gaze on her for the
second time the lady looked round and, meeting the count's eyes,
nodded to him and smiled.  She was the Countess Bezukhova,
Pierre's wife, and the count, who knew everyone in society,
leaned over and spoke to her.

``Have you been here long, Countess?'' he inquired. ``I'll call,
I'll call to kiss your hand. I'm here on business and have
brought my girls with me. They say Semenova acts
marvelously. Count Pierre never used to forget us. Is he here?''

``Yes, he meant to look in,'' answered Helene, and glanced
attentively at Natasha.

Count Rostov resumed his seat.

``Handsome, isn't she?'' he whispered to Natasha.

``Wonderful!'' answered Natasha. ``She's a woman one could easily
fall in love with.''

Just then the last chords of the overture were heard and the
conductor tapped with his stick. Some latecomers took their seats
in the stalls, and the curtain rose.

As soon as it rose everyone in the boxes and stalls became
silent, and all the men, old and young, in uniform and evening
dress, and all the women with gems on their bare flesh, turned
their whole attention with eager curiosity to the stage. Natasha
too began to look at it.

% % % % % % % % % % % % % % % % % % % % % % % % % % % % % % % % %
% % % % % % % % % % % % % % % % % % % % % % % % % % % % % % % % %
% % % % % % % % % % % % % % % % % % % % % % % % % % % % % % % % %
% % % % % % % % % % % % % % % % % % % % % % % % % % % % % % % % %
% % % % % % % % % % % % % % % % % % % % % % % % % % % % % % % % %
% % % % % % % % % % % % % % % % % % % % % % % % % % % % % % % % %
% % % % % % % % % % % % % % % % % % % % % % % % % % % % % % % % %
% % % % % % % % % % % % % % % % % % % % % % % % % % % % % % % % %
% % % % % % % % % % % % % % % % % % % % % % % % % % % % % % % % %
% % % % % % % % % % % % % % % % % % % % % % % % % % % % % % % % %
% % % % % % % % % % % % % % % % % % % % % % % % % % % % % % % % %
% % % % % % % % % % % % % % % % % % % % % % % % % % % % % %

\chapter*{Chapter IX}
\ifaudio     
\marginpar{
\href{http://ia601406.us.archive.org/23/items/war_and_peace_08_0810_librivox/war_and_peace_08_09_tolstoy_64kb.mp3}{Audio}} 
\fi

\lettrine[lines=2, loversize=0.3, lraise=0]{\initfamily T}{he}
floor of the stage consisted of smooth boards, at the sides
was some painted cardboard representing trees, and at the back
was a cloth stretched over boards. In the center of the stage sat
some girls in red bodices and white skirts. One very fat girl in
a white silk dress sat apart on a low bench, to the back of which
a piece of green cardboard was glued. They all sang
something. When they had finished their song the girl in white
went up to the prompter's box and a man with tight silk trousers
over his stout legs, and holding a plume and a dagger, went up to
her and began singing, waving his arms about.

First the man in the tight trousers sang alone, then she sang,
then they both paused while the orchestra played and the man
fingered the hand of the girl in white, obviously awaiting the
beat to start singing with her. They sang together and everyone
in the theater began clapping and shouting, while the man and
woman on the stage---who represented lovers---began smiling,
spreading out their arms, and bowing.

After her life in the country, and in her present serious mood,
all this seemed grotesque and amazing to Natasha. She could not
follow the opera nor even listen to the music; she saw only the
painted cardboard and the queerly dressed men and women who
moved, spoke, and sang so strangely in that brilliant light. She
knew what it was all meant to represent, but it was so
pretentiously false and unnatural that she first felt ashamed for
the actors and then amused at them. She looked at the faces of
the audience, seeking in them the same sense of ridicule and
perplexity she herself experienced, but they all seemed attentive
to what was happening on the stage, and expressed delight which
to Natasha seemed feigned. ``I suppose it has to be like this!''
she thought. She kept looking round in turn at the rows of
pomaded heads in the stalls and then at the seminude women in the
boxes, especially at Helene in the next box, who---apparently
quite unclothed---sat with a quiet tranquil smile, not taking her
eyes off the stage. And feeling the bright light that flooded the
whole place and the warm air heated by the crowd, Natasha little
by little began to pass into a state of intoxication she had not
experienced for a long while. She did not realize who and where
she was, nor what was going on before her. As she looked and
thought, the strangest fancies unexpectedly and disconnectedly
passed through her mind: the idea occurred to her of jumping onto
the edge of the box and singing the air the actress was singing,
then she wished to touch with her fan an old gentleman sitting
not far from her, then to lean over to Helene and tickle her.

At a moment when all was quiet before the commencement of a song,
a door leading to the stalls on the side nearest the Rostovs' box
creaked, and the steps of a belated arrival were heard. ``There's
Kuragin!'' whispered Shinshin. Countess Bezukhova turned smiling
to the newcomer, and Natasha, following the direction of that
look, saw an exceptionally handsome adjutant approaching their
box with a self-assured yet courteous bearing. This was Anatole
Kuragin whom she had seen and noticed long ago at the ball in
Petersburg. He was now in an adjutant's uniform with one epaulet
and a shoulder knot. He moved with a restrained swagger which
would have been ridiculous had he not been so good-looking and
had his handsome face not worn such an expression of good-humored
complacency and gaiety. Though the performance was proceeding, he
walked deliberately down the carpeted gangway, his sword and
spurs slightly jingling and his handsome perfumed head held
high. Having looked at Natasha he approached his sister, laid his
well gloved hand on the edge of her box, nodded to her, and
leaning forward asked a question, with a motion toward Natasha.

``Mais charmante!'' said he, evidently referring to Natasha, who
did not exactly hear his words but understood them from the
movement of his lips. Then he took his place in the first row of
the stalls and sat down beside Dolokhov, nudging with his elbow
in a friendly and offhand way that Dolokhov whom others treated
so fawningly. He winked at him gaily, smiled, and rested his foot
against the orchestra screen.

``How like the brother is to the sister,'' remarked the
count. ``And how handsome they both are!''

Shinshin, lowering his voice, began to tell the count of some
intrigue of Kuragin's in Moscow, and Natasha tried to overhear it
just because he had said she was ``charmante.''

The first act was over. In the stalls everyone began moving
about, going out and coming in.

Boris came to the Rostovs' box, received their congratulations
very simply, and raising his eyebrows with an absent-minded smile
conveyed to Natasha and Sonya his fiancee's invitation to her
wedding, and went away. Natasha with a gay, coquettish smile
talked to him, and congratulated on his approaching wedding that
same Boris with whom she had formerly been in love. In the state
of intoxication she was in, everything seemed simple and natural.

The scantily clad Helene smiled at everyone in the same way, and
Natasha gave Boris a similar smile.

Helene's box was filled and surrounded from the stalls by the
most distinguished and intellectual men, who seemed to vie with
one another in their wish to let everyone see that they knew her.

During the whole of that entr'acte Kuragin stood with Dolokhov in
front of the orchestra partition, looking at the Rostovs'
box. Natasha knew he was talking about her and this afforded her
pleasure. She even turned so that he should see her profile in
what she thought was its most becoming aspect. Before the
beginning of the second act Pierre appeared in the stalls. The
Rostovs had not seen him since their arrival. His face looked
sad, and he had grown still stouter since Natasha last saw him.
He passed up to the front rows, not noticing anyone. Anatole went
up to him and began speaking to him, looking at and indicating
the Rostovs' box. On seeing Natasha Pierre grew animated and,
hastily passing between the rows, came toward their box. When he
got there he leaned on his elbows and, smiling, talked to her for
a long time. While conversing with Pierre, Natasha heard a man's
voice in Countess Bezukhova's box and something told her it was
Kuragin. She turned and their eyes met. Almost smiling, he gazed
straight into her eyes with such an enraptured caressing look
that it seemed strange to be so near him, to look at him like
that, to be so sure he admired her, and not to be acquainted with
him.

In the second act there was scenery representing tombstones,
there was a round hole in the canvas to represent the moon,
shades were raised over the footlights, and from horns and
contrabass came deep notes while many people appeared from right
and left wearing black cloaks and holding things like daggers in
their hands. They began waving their arms. Then some other people
ran in and began dragging away the maiden who had been in white
and was now in light blue. They did not drag her away at once,
but sang with her for a long time and then at last dragged her
off, and behind the scenes something metallic was struck three
times and everyone knelt down and sang a prayer. All these things
were repeatedly interrupted by the enthusiastic shouts of the
audience.

During this act every time Natasha looked toward the stalls she
saw Anatole Kuragin with an arm thrown across the back of his
chair, staring at her. She was pleased to see that he was
captivated by her and it did not occur to her that there was
anything wrong in it.

When the second act was over Countess Bezukhova rose, turned to
the Rostovs' box---her whole bosom completely exposed---beckoned
the old count with a gloved finger, and paying no attention to
those who had entered her box began talking to him with an
amiable smile.

``Do make me acquainted with your charming daughters,'' said
she. ``The whole town is singing their praises and I don't even
know them!''

Natasha rose and curtsied to the splendid countess. She was so
pleased by praise from this brilliant beauty that she blushed
with pleasure.

``I want to become a Moscovite too, now,'' said Helene. ``How is
it you're not ashamed to bury such pearls in the country?''

Countess Bezukhova quite deserved her reputation of being a
fascinating woman. She could say what she did not
think---especially what was flattering---quite simply and
naturally.

``Dear count, you must let me look after your daughters! Though I
am not staying here long this time---nor are you---I will try to
amuse them. I have already heard much of you in Petersburg and
wanted to get to know you,'' said she to Natasha with her
stereotyped and lovely smile. ``I had heard about you from my
page, Drubetskoy. Have you heard he is getting married? And also
from my husband's friend Bolkonski, Prince Andrew Bolkonski,''
she went on with special emphasis, implying that she knew of his
relation to Natasha. To get better acquainted she asked that one
of the young ladies should come into her box for the rest of the
performance, and Natasha moved over to it.

The scene of the third act represented a palace in which many
candles were burning and pictures of knights with short beards
hung on the walls. In the middle stood what were probably a king
and a queen. The king waved his right arm and, evidently nervous,
sang something badly and sat down on a crimson throne. The maiden
who had been first in white and then in light blue, now wore only
a smock, and stood beside the throne with her hair down. She sang
something mournfully, addressing the queen, but the king waved
his arm severely, and men and women with bare legs came in from
both sides and began dancing all together. Then the violins
played very shrilly and merrily and one of the women with thick
bare legs and thin arms, separating from the others, went behind
the wings, adjusted her bodice, returned to the middle of the
stage, and began jumping and striking one foot rapidly against
the other. In the stalls everyone clapped and shouted ``bravo!''
Then one of the men went into a corner of the stage. The cymbals
and horns in the orchestra struck up more loudly, and this man
with bare legs jumped very high and waved his feet about very
rapidly. (He was Duport, who received sixty thousand rubles a
year for this art.) Everybody in the stalls, boxes, and galleries
began clapping and shouting with all their might, and the man
stopped and began smiling and bowing to all sides. Then other men
and women danced with bare legs. Then the king again shouted to
the sound of music, and they all began singing. But suddenly a
storm came on, chromatic scales and diminished sevenths were
heard in the orchestra, everyone ran off, again dragging one of
their number away, and the curtain dropped. Once more there was a
terrible noise and clatter among the audience, and with rapturous
faces everyone began shouting: ``Duport! Duport! Duport!''
Natasha no longer thought this strange. She looked about with
pleasure, smiling joyfully.

``Isn't Duport delightful?'' Helene asked her.

``Oh, yes,'' replied Natasha.

% % % % % % % % % % % % % % % % % % % % % % % % % % % % % % % % %
% % % % % % % % % % % % % % % % % % % % % % % % % % % % % % % % %
% % % % % % % % % % % % % % % % % % % % % % % % % % % % % % % % %
% % % % % % % % % % % % % % % % % % % % % % % % % % % % % % % % %
% % % % % % % % % % % % % % % % % % % % % % % % % % % % % % % % %
% % % % % % % % % % % % % % % % % % % % % % % % % % % % % % % % %
% % % % % % % % % % % % % % % % % % % % % % % % % % % % % % % % %
% % % % % % % % % % % % % % % % % % % % % % % % % % % % % % % % %
% % % % % % % % % % % % % % % % % % % % % % % % % % % % % % % % %
% % % % % % % % % % % % % % % % % % % % % % % % % % % % % % % % %
% % % % % % % % % % % % % % % % % % % % % % % % % % % % % % % % %
% % % % % % % % % % % % % % % % % % % % % % % % % % % % % %

\chapter*{Chapter X}
\ifaudio     
\marginpar{
\href{http://ia601406.us.archive.org/23/items/war_and_peace_08_0810_librivox/war_and_peace_08_10_tolstoy_64kb.mp3}{Audio}} 
\fi

\lettrine[lines=2, loversize=0.3, lraise=0]{\initfamily D}{uring}
the entr'acte a whiff of cold air came into Helene's box,
the door opened, and Anatole entered, stooping and trying not to
brush against anyone.

``Let me introduce my brother to you,'' said Helene, her eyes
shifting uneasily from Natasha to Anatole.

Natasha turned her pretty little head toward the elegant young
officer and smiled at him over her bare shoulder. Anatole, who
was as handsome at close quarters as at a distance, sat down
beside her and told her he had long wished to have this
happiness---ever since the Naryshkins' ball in fact, at which he
had had the well-remembered pleasure of seeing her.  Kuragin was
much more sensible and simple with women than among men. He
talked boldly and naturally, and Natasha was strangely and
agreeably struck by the fact that there was nothing formidable in
this man about whom there was so much talk, but that on the
contrary his smile was most naive, cheerful, and good-natured.

Kuragin asked her opinion of the performance and told her how at
a previous performance Semenova had fallen down on the stage.

``And do you know, Countess,'' he said, suddenly addressing her
as an old, familiar acquaintance, ``we are getting up a costume
tournament; you ought to take part in it! It will be great
fun. We shall all meet at the Karagins'! Please come! No! Really,
eh?'' said he.

While saying this he never removed his smiling eyes from her
face, her neck, and her bare arms. Natasha knew for certain that
he was enraptured by her. This pleased her, yet his presence made
her feel constrained and oppressed. When she was not looking at
him she felt that he was looking at her shoulders, and she
involuntarily caught his eye so that he should look into hers
rather than this. But looking into his eyes she was frightened,
realizing that there was not that barrier of modesty she had
always felt between herself and other men. She did not know how
it was that within five minutes she had come to feel herself
terribly near to this man. When she turned away she feared he
might seize her from behind by her bare arm and kiss her on the
neck. They spoke of most ordinary things, yet she felt that they
were closer to one another than she had ever been to any
man. Natasha kept turning to Helene and to her father, as if
asking what it all meant, but Helene was engaged in conversation
with a general and did not answer her look, and her father's eyes
said nothing but what they always said: ``Having a good time?
Well, I'm glad of it!''

During one of these moments of awkward silence when Anatole's
prominent eyes were gazing calmly and fixedly at her, Natasha, to
break the silence, asked him how he liked Moscow. She asked the
question and blushed. She felt all the time that by talking to
him she was doing something improper. Anatole smiled as though to
encourage her.

``At first I did not like it much, because what makes a town
pleasant ce sont les jolies femmes,\footnote{Are the pretty
women.} isn't that so? But now I like it very much indeed,'' he
said, looking at her significantly. ``You'll come to the costume
tournament, Countess? Do come!'' and putting out his hand to her
bouquet and dropping his voice, he added, ``You will be the
prettiest there. Do come, dear countess, and give me this flower
as a pledge!''

Natasha did not understand what he was saying any more than he
did himself, but she felt that his incomprehensible words had an
improper intention. She did not know what to say and turned away
as if she had not heard his remark. But as soon as she had turned
away she felt that he was there, behind, so close behind her.

``How is he now? Confused? Angry? Ought I to put it right?'' she
asked herself, and she could not refrain from turning round. She
looked straight into his eyes, and his nearness, self-assurance,
and the good-natured tenderness of his smile vanquished her. She
smiled just as he was doing, gazing straight into his eyes. And
again she felt with horror that no barrier lay between him and
her.

The curtain rose again. Anatole left the box, serene and
gay. Natasha went back to her father in the other box, now quite
submissive to the world she found herself in. All that was going
on before her now seemed quite natural, but on the other hand all
her previous thoughts of her betrothed, of Princess Mary, or of
life in the country did not once recur to her mind and were as if
belonging to a remote past.

In the fourth act there was some sort of devil who sang waving
his arm about, till the boards were withdrawn from under him and
he disappeared down below. That was the only part of the fourth
act that Natasha saw.  She felt agitated and tormented, and the
cause of this was Kuragin whom she could not help watching. As
they were leaving the theater Anatole came up to them, called
their carriage, and helped them in. As he was putting Natasha in
he pressed her arm above the elbow. Agitated and flushed she
turned round. He was looking at her with glittering eyes, smiling
tenderly.

Only after she had reached home was Natasha able clearly to think
over what had happened to her, and suddenly remembering Prince
Andrew she was horrified, and at tea to which all had sat down
after the opera, she gave a loud exclamation, flushed, and ran
out of the room.

``O God! I am lost!'' she said to herself. ``How could I let
him?'' She sat for a long time hiding her flushed face in her
hands trying to realize what had happened to her, but was unable
either to understand what had happened or what she
felt. Everything seemed dark, obscure, and terrible. There in
that enormous, illuminated theater where the bare-legged Duport,
in a tinsel-decorated jacket, jumped about to the music on wet
boards, and young girls and old men, and the nearly naked Helene
with her proud, calm smile, rapturously cried ``bravo!''---there
in the presence of that Helene it had all seemed clear and
simple; but now, alone by herself, it was
incomprehensible. ``What is it? What was that terror I felt of
him? What is this gnawing of conscience I am feeling now?'' she
thought.

Only to the old countess at night in bed could Natasha have told
all she was feeling. She knew that Sonya with her severe and
simple views would either not understand it at all or would be
horrified at such a confession. So Natasha tried to solve what
was torturing her by herself.

``Am I spoiled for Andrew's love or not?'' she asked herself, and
with soothing irony replied: ``What a fool I am to ask that! What
did happen to me? Nothing! I have done nothing, I didn't lead him
on at all. Nobody will know and I shall never see him again,''
she told herself. ``So it is plain that nothing has happened and
there is nothing to repent of, and Andrew can love me still. But
why 'still?' O God, why isn't he here?''  Natasha quieted herself
for a moment, but again some instinct told her that though all
this was true, and though nothing had happened, yet the former
purity of her love for Prince Andrew had perished. And again in
imagination she went over her whole conversation with Kuragin,
and again saw the face, gestures, and tender smile of that bold
handsome man when he pressed her arm.

% % % % % % % % % % % % % % % % % % % % % % % % % % % % % % % % %
% % % % % % % % % % % % % % % % % % % % % % % % % % % % % % % % %
% % % % % % % % % % % % % % % % % % % % % % % % % % % % % % % % %
% % % % % % % % % % % % % % % % % % % % % % % % % % % % % % % % %
% % % % % % % % % % % % % % % % % % % % % % % % % % % % % % % % %
% % % % % % % % % % % % % % % % % % % % % % % % % % % % % % % % %
% % % % % % % % % % % % % % % % % % % % % % % % % % % % % % % % %
% % % % % % % % % % % % % % % % % % % % % % % % % % % % % % % % %
% % % % % % % % % % % % % % % % % % % % % % % % % % % % % % % % %
% % % % % % % % % % % % % % % % % % % % % % % % % % % % % % % % %
% % % % % % % % % % % % % % % % % % % % % % % % % % % % % % % % %
% % % % % % % % % % % % % % % % % % % % % % % % % % % % % %

\chapter*{Chapter XI}
\ifaudio     
\marginpar{
\href{http://ia601406.us.archive.org/23/items/war_and_peace_08_0810_librivox/war_and_peace_08_11_tolstoy_64kb.mp3}{Audio}} 
\fi

\lettrine[lines=2, loversize=0.3, lraise=0]{\initfamily A}{natole}
Kuragin was staying in Moscow because his father had sent
him away from Petersburg, where he had been spending twenty
thousand rubles a year in cash, besides running up debts for as
much more, which his creditors demanded from his father.

His father announced to him that he would now pay half his debts
for the last time, but only on condition that he went to Moscow
as adjutant to the commander-in-chief---a post his father had
procured for him---and would at last try to make a good match
there. He indicated to him Princess Mary and Julie Karagina.

Anatole consented and went to Moscow, where he put up at Pierre's
house.  Pierre received him unwillingly at first, but got used to
him after a while, sometimes even accompanied him on his
carousals, and gave him money under the guise of loans.

As Shinshin had remarked, from the time of his arrival Anatole
had turned the heads of the Moscow ladies, especially by the fact
that he slighted them and plainly preferred the gypsy girls and
French actresses---with the chief of whom, Mademoiselle George,
he was said to be on intimate relations. He had never missed a
carousal at Danilov's or other Moscow revelers', drank whole
nights through, outvying everyone else, and was at all the balls
and parties of the best society. There was talk of his intrigues
with some of the ladies, and he flirted with a few of them at the
balls. But he did not run after the unmarried girls, especially
the rich heiresses who were most of them plain. There was a
special reason for this, as he had got married two years
before---a fact known only to his most intimate friends. At that
time while with his regiment in Poland, a Polish landowner of
small means had forced him to marry his daughter. Anatole had
very soon abandoned his wife and, for a payment which he agreed
to send to his father-in-law, had arranged to be free to pass
himself off as a bachelor.

Anatole was always content with his position, with himself, and
with others. He was instinctively and thoroughly convinced that
it was impossible for him to live otherwise than as he did and
that he had never in his life done anything base. He was
incapable of considering how his actions might affect others or
what the consequences of this or that action of his might be. He
was convinced that, as a duck is so made that it must live in
water, so God had made him such that he must spend thirty
thousand rubles a year and always occupy a prominent position in
society. He believed this so firmly that others, looking at him,
were persuaded of it too and did not refuse him either a leading
place in society or money, which he borrowed from anyone and
everyone and evidently would not repay.

He was not a gambler, at any rate he did not care about
winning. He was not vain. He did not mind what people thought of
him. Still less could he be accused of ambition. More than once
he had vexed his father by spoiling his own career, and he
laughed at distinctions of all kinds. He was not mean, and did
not refuse anyone who asked of him. All he cared about was gaiety
and women, and as according to his ideas there was nothing
dishonorable in these tastes, and he was incapable of considering
what the gratification of his tastes entailed for others, he
honestly considered himself irreproachable, sincerely despised
rogues and bad people, and with a tranquil conscience carried his
head high.

Rakes, those male Magdalenes, have a secret feeling of innocence
similar to that which female Magdalenes have, based on the same
hope of forgiveness. ``All will be forgiven her, for she loved
much; and all will be forgiven him, for he enjoyed much.''

Dolokhov, who had reappeared that year in Moscow after his exile
and his Persian adventures, and was leading a life of luxury,
gambling, and dissipation, associated with his old Petersburg
comrade Kuragin and made use of him for his own ends.

Anatole was sincerely fond of Dolokhov for his cleverness and
audacity.  Dolokhov, who needed Anatole Kuragin's name, position,
and connections as a bait to draw rich young men into his
gambling set, made use of him and amused himself at his expense
without letting the other feel it.  Apart from the advantage he
derived from Anatole, the very process of dominating another's
will was in itself a pleasure, a habit, and a necessity to
Dolokhov.

Natasha had made a strong impression on Kuragin. At supper after
the opera he described to Dolokhov with the air of a connoisseur
the attractions of her arms, shoulders, feet, and hair and
expressed his intention of making love to her. Anatole had no
notion and was incapable of considering what might come of such
love-making, as he never had any notion of the outcome of any of
his actions.

``She's first-rate, my dear fellow, but not for us,'' replied
Dolokhov.

``I will tell my sister to ask her to dinner,'' said
Anatole. ``Eh?''

``You'd better wait till she's married...''

``You know, I adore little girls, they lose their heads at
once,'' pursued Anatole.

``You have been caught once already by a 'little girl,'\ '' said
Dolokhov who knew of Kuragin's marriage. ``Take care!''

``Well, that can't happen twice! Eh?'' said Anatole, with a
good-humored laugh.

% % % % % % % % % % % % % % % % % % % % % % % % % % % % % % % % %
% % % % % % % % % % % % % % % % % % % % % % % % % % % % % % % % %
% % % % % % % % % % % % % % % % % % % % % % % % % % % % % % % % %
% % % % % % % % % % % % % % % % % % % % % % % % % % % % % % % % %
% % % % % % % % % % % % % % % % % % % % % % % % % % % % % % % % %
% % % % % % % % % % % % % % % % % % % % % % % % % % % % % % % % %
% % % % % % % % % % % % % % % % % % % % % % % % % % % % % % % % %
% % % % % % % % % % % % % % % % % % % % % % % % % % % % % % % % %
% % % % % % % % % % % % % % % % % % % % % % % % % % % % % % % % %
% % % % % % % % % % % % % % % % % % % % % % % % % % % % % % % % %
% % % % % % % % % % % % % % % % % % % % % % % % % % % % % % % % %
% % % % % % % % % % % % % % % % % % % % % % % % % % % % % %

\chapter*{Chapter XII}
\ifaudio     
\marginpar{
\href{http://ia601406.us.archive.org/23/items/war_and_peace_08_0810_librivox/war_and_peace_08_12_tolstoy_64kb.mp3}{Audio}} 
\fi

\lettrine[lines=2, loversize=0.3, lraise=0]{\initfamily T}{he}
day after the opera the Rostovs went nowhere and nobody came
to see them. Marya Dmitrievna talked to the count about something
which they concealed from Natasha. Natasha guessed they were
talking about the old prince and planning something, and this
disquieted and offended her. She was expecting Prince Andrew any
moment and twice that day sent a manservant to the Vozdvizhenka
to ascertain whether he had come. He had not arrived. She
suffered more now than during her first days in Moscow.  To her
impatience and pining for him were now added the unpleasant
recollection of her interview with Princess Mary and the old
prince, and a fear and anxiety of which she did not understand
the cause. She continually fancied that either he would never
come or that something would happen to her before he came. She
could no longer think of him by herself calmly and continuously
as she had done before. As soon as she began to think of him, the
recollection of the old prince, of Princess Mary, of the theater,
and of Kuragin mingled with her thoughts. The question again
presented itself whether she was not guilty, whether she had not
already broken faith with Prince Andrew, and again she found
herself recalling to the minutest detail every word, every
gesture, and every shade in the play of expression on the face of
the man who had been able to arouse in her such an
incomprehensible and terrifying feeling. To the family Natasha
seemed livelier than usual, but she was far less tranquil and
happy than before.

On Sunday morning Marya Dmitrievna invited her visitors to Mass
at her parish church---the Church of the Assumption built over
the graves of victims of the plague.

``I don't like those fashionable churches,'' she said, evidently
priding herself on her independence of thought. ``God is the same
everywhere. We have an excellent priest, he conducts the service
decently and with dignity, and the deacon is the same. What
holiness is there in giving concerts in the choir? I don't like
it, it's just self-indulgence!''

Marya Dmitrievna liked Sundays and knew how to keep them. Her
whole house was scrubbed and cleaned on Saturdays; neither she
nor the servants worked, and they all wore holiday dress and went
to church. At her table there were extra dishes at dinner, and
the servants had vodka and roast goose or suckling pig. But in
nothing in the house was the holiday so noticeable as in Marya
Dmitrievna's broad, stern face, which on that day wore an
invariable look of solemn festivity.

After Mass, when they had finished their coffee in the dining
room where the loose covers had been removed from the furniture,
a servant announced that the carriage was ready, and Marya
Dmitrievna rose with a stern air. She wore her holiday shawl, in
which she paid calls, and announced that she was going to see
Prince Nicholas Bolkonski to have an explanation with him about
Natasha.

After she had gone, a dressmaker from Madame Suppert-Roguet
waited on the Rostovs, and Natasha, very glad of this diversion,
having shut herself into a room adjoining the drawing room,
occupied herself trying on the new dresses. Just as she had put
on a bodice without sleeves and only tacked together, and was
turning her head to see in the glass how the back fitted, she
heard in the drawing room the animated sounds of her father's
voice and another's---a woman's---that made her flush. It was
Helene. Natasha had not time to take off the bodice before the
door opened and Countess Bezukhova, dressed in a purple velvet
gown with a high collar, came into the room beaming with
good-humored amiable smiles.

``Oh, my enchantress!'' she cried to the blushing
Natasha. ``Charming! No, this is really beyond anything, my dear
count,'' said she to Count Rostov who had followed her in. ``How
can you live in Moscow and go nowhere? No, I won't let you off!
Mademoiselle George will recite at my house tonight and there'll
be some people, and if you don't bring your lovely girls---who
are prettier than Mademoiselle George---I won't know you!  My
husband is away in Tver or I would send him to fetch you. You
must come. You positively must! Between eight and nine.''

She nodded to the dressmaker, whom she knew and who had curtsied
respectfully to her, and seated herself in an armchair beside the
looking glass, draping the folds of her velvet dress
picturesquely. She did not cease chattering good-naturedly and
gaily, continually praising Natasha's beauty. She looked at
Natasha's dresses and praised them, as well as a new dress of her
own made of \emph{metallic gauze}, which she had received from
Paris, and advised Natasha to have one like it.

``But anything suits you, my charmer!'' she remarked.

A smile of pleasure never left Natasha's face. She felt happy and
as if she were blossoming under the praise of this dear Countess
Bezukhova who had formerly seemed to her so unapproachable and
important and was now so kind to her. Natasha brightened up and
felt almost in love with this woman, who was so beautiful and so
kind. Helene for her part was sincerely delighted with Natasha
and wished to give her a good time.  Anatole had asked her to
bring him and Natasha together, and she was calling on the
Rostovs for that purpose. The idea of throwing her brother and
Natasha together amused her.

Though at one time, in Petersburg, she had been annoyed with
Natasha for drawing Boris away, she did not think of that now,
and in her own way heartily wished Natasha well. As she was
leaving the Rostovs she called her protegee aside.

``My brother dined with me yesterday---we nearly died of
laughter---he ate nothing and kept sighing for you, my charmer!
He is madly, quite madly, in love with you, my dear.''

Natasha blushed scarlet when she heard this.

``How she blushes, how she blushes, my pretty!'' said
Helene. ``You must certainly come. If you love somebody, my
charmer, that is not a reason to shut yourself up. Even if you
are engaged, I am sure your fiance would wish you to go into
society rather than be bored to death.''

``So she knows I am engaged, and she and her husband
Pierre---that good Pierre---have talked and laughed about
this. So it's all right.'' And again, under Helene's influence,
what had seemed terrible now seemed simple and natural. ``And she
is such a grande dame, so kind, and evidently likes me so
much. And why not enjoy myself?'' thought Natasha, gazing at
Helene with wide-open, wondering eyes.

Marya Dmitrievna came back to dinner taciturn and serious, having
evidently suffered a defeat at the old prince's. She was still
too agitated by the encounter to be able to talk of the affair
calmly. In answer to the count's inquiries she replied that
things were all right and that she would tell about it next
day. On hearing of Countess Bezukhova's visit and the invitation
for that evening, Marya Dmitrievna remarked:

``I don't care to have anything to do with Bezukhova and don't
advise you to; however, if you've promised---go. It will divert
your thoughts,'' she added, addressing Natasha.

% % % % % % % % % % % % % % % % % % % % % % % % % % % % % % % % %
% % % % % % % % % % % % % % % % % % % % % % % % % % % % % % % % %
% % % % % % % % % % % % % % % % % % % % % % % % % % % % % % % % %
% % % % % % % % % % % % % % % % % % % % % % % % % % % % % % % % %
% % % % % % % % % % % % % % % % % % % % % % % % % % % % % % % % %
% % % % % % % % % % % % % % % % % % % % % % % % % % % % % % % % %
% % % % % % % % % % % % % % % % % % % % % % % % % % % % % % % % %
% % % % % % % % % % % % % % % % % % % % % % % % % % % % % % % % %
% % % % % % % % % % % % % % % % % % % % % % % % % % % % % % % % %
% % % % % % % % % % % % % % % % % % % % % % % % % % % % % % % % %
% % % % % % % % % % % % % % % % % % % % % % % % % % % % % % % % %
% % % % % % % % % % % % % % % % % % % % % % % % % % % % % %

\chapter*{Chapter XIII}
\ifaudio     
\marginpar{
\href{http://ia601406.us.archive.org/23/items/war_and_peace_08_0810_librivox/war_and_peace_08_13_tolstoy_64kb.mp3}{Audio}} 
\fi

\lettrine[lines=2, loversize=0.3, lraise=0]{\initfamily C}{ount}
Rostov took the girls to Countess Bezukhova's. There were a
good many people there, but nearly all strangers to
Natasha. Count Rostov was displeased to see that the company
consisted almost entirely of men and women known for the freedom
of their conduct. Mademoiselle George was standing in a corner of
the drawing room surrounded by young men. There were several
Frenchmen present, among them Metivier who from the time Helene
reached Moscow had been an intimate in her house. The count
decided not to sit down to cards or let his girls out of his
sight and to get away as soon as Mademoiselle George's
performance was over.

Anatole was at the door, evidently on the lookout for the
Rostovs.  Immediately after greeting the count he went up to
Natasha and followed her. As soon as she saw him she was seized
by the same feeling she had had at the opera---gratified vanity
at his admiration of her and fear at the absence of a moral
barrier between them.

Helene welcomed Natasha delightedly and was loud in admiration of
her beauty and her dress. Soon after their arrival Mademoiselle
George went out of the room to change her costume. In the drawing
room people began arranging the chairs and taking their
seats. Anatole moved a chair for Natasha and was about to sit
down beside her, but the count, who never lost sight of her, took
the seat himself. Anatole sat down behind her.

Mademoiselle George, with her bare, fat, dimpled arms, and a red
shawl draped over one shoulder, came into the space left vacant
for her, and assumed an unnatural pose. Enthusiastic whispering
was audible.

Mademoiselle George looked sternly and gloomily at the audience
and began reciting some French verses describing her guilty love
for her son. In some places she raised her voice, in others she
whispered, lifting her head triumphantly; sometimes she paused
and uttered hoarse sounds, rolling her eyes.

``Adorable! divine! delicious!'' was heard from every side.

Natasha looked at the fat actress, but neither saw nor heard nor
understood anything of what went on before her. She only felt
herself again completely borne away into this strange senseless
world---so remote from her old world---a world in which it was
impossible to know what was good or bad, reasonable or
senseless. Behind her sat Anatole, and conscious of his proximity
she experienced a frightened sense of expectancy.

After the first monologue the whole company rose and surrounded
Mademoiselle George, expressing their enthusiasm.

``How beautiful she is!'' Natasha remarked to her father who had
also risen and was moving through the crowd toward the actress.

``I don't think so when I look at you!'' said Anatole, following
Natasha.  He said this at a moment when she alone could hear
him. ``You are enchanting... from the moment I saw you I have
never ceased...''

``Come, come, Natasha!'' said the count, as he turned back for
his daughter. ``How beautiful she is!'' Natasha without saying
anything stepped up to her father and looked at him with
surprised inquiring eyes.

After giving several recitations, Mademoiselle George left, and
Countess Bezukhova asked her visitors into the ballroom.

The count wished to go home, but Helene entreated him not to
spoil her improvised ball, and the Rostovs stayed on. Anatole
asked Natasha for a valse and as they danced he pressed her waist
and hand and told her she was bewitching and that he loved
her. During the ecossaise, which she also danced with him,
Anatole said nothing when they happened to be by themselves, but
merely gazed at her. Natasha lifted her frightened eyes to him,
but there was such confident tenderness in his affectionate look
and smile that she could not, whilst looking at him, say what she
had to say. She lowered her eyes.

``Don't say such things to me. I am betrothed and love another,''
she said rapidly... She glanced at him.

Anatole was not upset or pained by what she had said.

``Don't speak to me of that! What can I do?'' said he. ``I tell
you I am madly, madly, in love with you! Is it my fault that you
are enchanting?... It's our turn to begin.''

Natasha, animated and excited, looked about her with wide-open
frightened eyes and seemed merrier than usual. She understood
hardly anything that went on that evening. They danced the
ecossaise and the Grossvater. Her father asked her to come home,
but she begged to remain.  Wherever she went and whomever she was
speaking to, she felt his eyes upon her. Later on she recalled
how she had asked her father to let her go to the dressing room
to rearrange her dress, that Helene had followed her and spoken
laughingly of her brother's love, and that she again met Anatole
in the little sitting room. Helene had disappeared leaving them
alone, and Anatole had taken her hand and said in a tender voice:

``I cannot come to visit you but is it possible that I shall
never see you? I love you madly. Can I never...?'' and, blocking
her path, he brought his face close to hers.

His large, glittering, masculine eyes were so close to hers that
she saw nothing but them.

``Natalie?'' he whispered inquiringly while she felt her hands
being painfully pressed. ``Natalie?''

``I don't understand. I have nothing to say,'' her eyes replied.

Burning lips were pressed to hers, and at the same instant she
felt herself released, and Helene's footsteps and the rustle of
her dress were heard in the room. Natasha looked round at her,
and then, red and trembling, threw a frightened look of inquiry
at Anatole and moved toward the door.

``One word, just one, for God's sake!'' cried Anatole.

She paused. She so wanted a word from him that would explain to
her what had happened and to which she could find no answer.

``Natalie, just a word, only one!'' he kept repeating, evidently
not knowing what to say and he repeated it till Helene came up to
them.

Helene returned with Natasha to the drawing room. The Rostovs
went away without staying for supper.

After reaching home Natasha did not sleep all night. She was
tormented by the insoluble question whether she loved Anatole or
Prince Andrew.  She loved Prince Andrew---she remembered
distinctly how deeply she loved him. But she also loved Anatole,
of that there was no doubt. ``Else how could all this have
happened?'' thought she. ``If, after that, I could return his
smile when saying good-by, if I was able to let it come to that,
it means that I loved him from the first. It means that he is
kind, noble, and splendid, and I could not help loving him. What
am I to do if I love him and the other one too?'' she asked
herself, unable to find an answer to these terrible questions.

% % % % % % % % % % % % % % % % % % % % % % % % % % % % % % % % %
% % % % % % % % % % % % % % % % % % % % % % % % % % % % % % % % %
% % % % % % % % % % % % % % % % % % % % % % % % % % % % % % % % %
% % % % % % % % % % % % % % % % % % % % % % % % % % % % % % % % %
% % % % % % % % % % % % % % % % % % % % % % % % % % % % % % % % %
% % % % % % % % % % % % % % % % % % % % % % % % % % % % % % % % %
% % % % % % % % % % % % % % % % % % % % % % % % % % % % % % % % %
% % % % % % % % % % % % % % % % % % % % % % % % % % % % % % % % %
% % % % % % % % % % % % % % % % % % % % % % % % % % % % % % % % %
% % % % % % % % % % % % % % % % % % % % % % % % % % % % % % % % %
% % % % % % % % % % % % % % % % % % % % % % % % % % % % % % % % %
% % % % % % % % % % % % % % % % % % % % % % % % % % % % % %

\chapter*{Chapter XIV}
\ifaudio     
\marginpar{
\href{http://ia601406.us.archive.org/23/items/war_and_peace_08_0810_librivox/war_and_peace_08_14_tolstoy_64kb.mp3}{Audio}} 
\fi

\lettrine[lines=2, loversize=0.3, lraise=0]{\initfamily M}{orning}
came with its cares and bustle. Everyone got up and began
to move about and talk, dressmakers came again. Marya Dmitrievna
appeared, and they were called to breakfast. Natasha kept looking
uneasily at everybody with wide-open eyes, as if wishing to
intercept every glance directed toward her, and tried to appear
the same as usual.

After breakfast, which was her best time, Marya Dmitrievna sat
down in her armchair and called Natasha and the count to her.

``Well, friends, I have now thought the whole matter over and
this is my advice,'' she began. ``Yesterday, as you know, I went
to see Prince Bolkonski. Well, I had a talk with him... He took
it into his head to begin shouting, but I am not one to be
shouted down. I said what I had to say!''

``Well, and he?'' asked the count.

``He? He's crazy... he did not want to listen. But what's the use
of talking? As it is we have worn the poor girl out,'' said Marya
Dmitrievna. ``My advice to you is finish your business and go
back home to Otradnoe... and wait there.''

``Oh, no!'' exclaimed Natasha.

``Yes, go back,'' said Marya Dmitrievna, ``and wait there. If
your betrothed comes here now---there will be no avoiding a
quarrel; but alone with the old man he will talk things over and
then come on to you.''

Count Rostov approved of this suggestion, appreciating its
reasonableness. If the old man came round it would be all the
better to visit him in Moscow or at Bald Hills later on; and if
not, the wedding, against his wishes, could only be arranged at
Otradnoe.

``That is perfectly true. And I am sorry I went to see him and
took her,'' said the old count.

``No, why be sorry? Being here, you had to pay your respects. But
if he won't---that's his affair,'' said Marya Dmitrievna, looking
for something in her reticule. ``Besides, the trousseau is ready,
so there is nothing to wait for; and what is not ready I'll send
after you. Though I don't like letting you go, it is the best
way. So go, with God's blessing!''

Having found what she was looking for in the reticule she handed
it to Natasha. It was a letter from Princess Mary.

``She has written to you. How she torments herself, poor thing!
She's afraid you might think that she does not like you.''

``But she doesn't like me,'' said Natasha.

``Don't talk nonsense!'' cried Marya Dmitrievna.

``I shan't believe anyone, I know she doesn't like me,'' replied
Natasha boldly as she took the letter, and her face expressed a
cold and angry resolution that caused Marya Dmitrievna to look at
her more intently and to frown.

``Don't answer like that, my good girl!'' she said. ``What I say
is true!  Write an answer!'' Natasha did not reply and went to
her own room to read Princess Mary's letter.

Princess Mary wrote that she was in despair at the
misunderstanding that had occurred between them. Whatever her
father's feelings might be, she begged Natasha to believe that
she could not help loving her as the one chosen by her brother,
for whose happiness she was ready to sacrifice everything.

``Do not think, however,'' she wrote, ``that my father is
ill-disposed toward you. He is an invalid and an old man who must
be forgiven; but he is good and magnanimous and will love her who
makes his son happy.''  Princess Mary went on to ask Natasha to
fix a time when she could see her again.

After reading the letter Natasha sat down at the writing table to
answer it. ``Dear Princess,'' she wrote in French quickly and
mechanically, and then paused. What more could she write after
all that had happened the evening before? ``Yes, yes! All that
has happened, and now all is changed,'' she thought as she sat
with the letter she had begun before her. ``Must I break off with
him? Must I really? That's awful...'' and to escape from these
dreadful thoughts she went to Sonya and began sorting patterns
with her.

After dinner Natasha went to her room and again took up Princess
Mary's letter. ``Can it be that it is all over?'' she
thought. ``Can it be that all this has happened so quickly and
has destroyed all that went before?'' She recalled her love for
Prince Andrew in all its former strength, and at the same time
felt that she loved Kuragin. She vividly pictured herself as
Prince Andrew's wife, and the scenes of happiness with him she
had so often repeated in her imagination, and at the same time,
aglow with excitement, recalled every detail of yesterday's
interview with Anatole.

``Why could that not be as well?'' she sometimes asked herself in
complete bewilderment. ``Only so could I be completely happy; but
now I have to choose, and I can't be happy without either of
them. Only,'' she thought, ``to tell Prince Andrew what has
happened or to hide it from him are both equally impossible. But
with that one nothing is spoiled. But am I really to abandon
forever the joy of Prince Andrew's love, in which I have lived so
long?''

``Please, Miss!'' whispered a maid entering the room with a
mysterious air. ``A man told me to give you this-'' and she
handed Natasha a letter.

``Only, for Christ's sake...'' the girl went on, as Natasha,
without thinking, mechanically broke the seal and read a love
letter from Anatole, of which, without taking in a word, she
understood only that it was a letter from him---from the man she
loved. Yes, she loved him, or else how could that have happened
which had happened? And how could she have a love letter from him
in her hand?

With trembling hands Natasha held that passionate love letter
which Dolokhov had composed for Anatole, and as she read it she
found in it an echo of all that she herself imagined she was
feeling.

``Since yesterday evening my fate has been sealed; to be loved by
you or to die. There is no other way for me,'' the letter
began. Then he went on to say that he knew her parents would not
give her to him---for this there were secret reasons he could
reveal only to her---but that if she loved him she need only say
the word yes, and no human power could hinder their bliss. Love
would conquer all. He would steal her away and carry her off to
the ends of the earth.

``Yes, yes! I love him!'' thought Natasha, reading the letter for
the twentieth time and finding some peculiarly deep meaning in
each word of it.

That evening Marya Dmitrievna was going to the Akharovs' and
proposed to take the girls with her. Natasha, pleading a
headache, remained at home.

% % % % % % % % % % % % % % % % % % % % % % % % % % % % % % % % %
% % % % % % % % % % % % % % % % % % % % % % % % % % % % % % % % %
% % % % % % % % % % % % % % % % % % % % % % % % % % % % % % % % %
% % % % % % % % % % % % % % % % % % % % % % % % % % % % % % % % %
% % % % % % % % % % % % % % % % % % % % % % % % % % % % % % % % %
% % % % % % % % % % % % % % % % % % % % % % % % % % % % % % % % %
% % % % % % % % % % % % % % % % % % % % % % % % % % % % % % % % %
% % % % % % % % % % % % % % % % % % % % % % % % % % % % % % % % %
% % % % % % % % % % % % % % % % % % % % % % % % % % % % % % % % %
% % % % % % % % % % % % % % % % % % % % % % % % % % % % % % % % %
% % % % % % % % % % % % % % % % % % % % % % % % % % % % % % % % %
% % % % % % % % % % % % % % % % % % % % % % % % % % % % % %

\chapter*{Chapter XV}
\ifaudio     
\marginpar{
\href{http://ia601406.us.archive.org/23/items/war_and_peace_08_0810_librivox/war_and_peace_08_15_tolstoy_64kb.mp3}{Audio}} 
\fi

\lettrine[lines=2, loversize=0.3, lraise=0]{\initfamily O}{n}
returning late in the evening Sonya went to Natasha's room,
and to her surprise found her still dressed and asleep on the
sofa. Open on the table, beside her lay Anatole's letter. Sonya
picked it up and read it.

As she read she glanced at the sleeping Natasha, trying to find
in her face an explanation of what she was reading, but did not
find it. Her face was calm, gentle, and happy. Clutching her
breast to keep herself from choking, Sonya, pale and trembling
with fear and agitation, sat down in an armchair and burst into
tears.

``How was it I noticed nothing? How could it go so far? Can she
have left off loving Prince Andrew? And how could she let Kuragin
go to such lengths? He is a deceiver and a villain, that's plain!
What will Nicholas, dear noble Nicholas, do when he hears of it?
So this is the meaning of her excited, resolute, unnatural look
the day before yesterday, yesterday, and today,'' thought
Sonya. ``But it can't be that she loves him! She probably opened
the letter without knowing who it was from. Probably she is
offended by it. She could not do such a thing!''

Sonya wiped away her tears and went up to Natasha, again scanning
her face.

``Natasha!'' she said, just audibly.

Natasha awoke and saw Sonya.

``Ah, you're back?''

And with the decision and tenderness that often come at the
moment of awakening, she embraced her friend, but noticing
Sonya's look of embarrassment, her own face expressed confusion
and suspicion.

``Sonya, you've read that letter?'' she demanded.

``Yes,'' answered Sonya softly.

Natasha smiled rapturously.

``No, Sonya, I can't any longer!'' she said. ``I can't hide it
from you any longer. You know, we love one another! Sonya,
darling, he writes...  Sonya...''

Sonya stared open-eyed at Natasha, unable to believe her ears.

``And Bolkonski?'' she asked.

``Ah, Sonya, if you only knew how happy I am!'' cried
Natasha. ``You don't know what love is...''

``But, Natasha, can that be all over?''

Natasha looked at Sonya with wide-open eyes as if she could not
grasp the question.

``Well, then, are you refusing Prince Andrew?'' said Sonya.

``Oh, you don't understand anything! Don't talk nonsense, just
listen!''  said Natasha, with momentary vexation.

``But I can't believe it,'' insisted Sonya. ``I don't
understand. How is it you have loved a man for a whole year and
suddenly... Why, you have only seen him three times! Natasha, I
don't believe you, you're joking! In three days to forget
everything and so...''

``Three days?'' said Natasha. ``It seems to me I've loved him a
hundred years. It seems to me that I have never loved anyone
before. You can't understand it... Sonya, wait a bit, sit here,''
and Natasha embraced and kissed her.

``I had heard that it happens like this, and you must have heard
it too, but it's only now that I feel such love. It's not the
same as before. As soon as I saw him I felt he was my master and
I his slave, and that I could not help loving him. Yes, his
slave! Whatever he orders I shall do. You don't understand
that. What can I do? What can I do, Sonya?''  cried Natasha with
a happy yet frightened expression.

``But think what you are doing,'' cried Sonya. ``I can't leave it
like this. This secret correspondence... How could you let him go
so far?''  she went on, with a horror and disgust she could
hardly conceal.

``I told you that I have no will,'' Natasha replied. ``Why can't
you understand? I love him!''

``Then I won't let it come to that... I shall tell!'' cried
Sonya, bursting into tears.

``What do you mean? For God's sake... If you tell, you are my
enemy!''  declared Natasha. ``You want me to be miserable, you
want us to be separated...''

When she saw Natasha's fright, Sonya shed tears of shame and pity
for her friend.

``But what has happened between you?'' she asked. ``What has he
said to you? Why doesn't he come to the house?''

Natasha did not answer her questions.

``For God's sake, Sonya, don't tell anyone, don't torture me,''
Natasha entreated. ``Remember no one ought to interfere in such
matters! I have confided in you...''

``But why this secrecy? Why doesn't he come to the house?'' asked
Sonya.  ``Why doesn't he openly ask for your hand? You know
Prince Andrew gave you complete freedom---if it is really so; but
I don't believe it!  Natasha, have you considered what these
secret reasons can be?''

Natasha looked at Sonya with astonishment. Evidently this
question presented itself to her mind for the first time and she
did not know how to answer it.

``I don't know what the reasons are. But there must be reasons!''

Sonya sighed and shook her head incredulously.

``If there were reasons...'' she began.

But Natasha, guessing her doubts, interrupted her in alarm.

``Sonya, one can't doubt him! One can't, one can't! Don't you
understand?'' she cried.

``Does he love you?''

``Does he love me?'' Natasha repeated with a smile of pity at her
friend's lack of comprehension. ``Why, you have read his letter
and you have seen him.''

``But if he is dishonorable?''

``He! dishonorable? If you only knew!'' exclaimed Natasha.

``If he is an honorable man he should either declare his
intentions or cease seeing you; and if you won't do this, I
will. I will write to him, and I will tell Papa!'' said Sonya
resolutely.

``But I can't live without him!'' cried Natasha.

``Natasha, I don't understand you. And what are you saying! Think
of your father and of Nicholas.''

``I don't want anyone, I don't love anyone but him. How dare you
say he is dishonorable? Don't you know that I love him?''
screamed Natasha. ``Go away, Sonya! I don't want to quarrel with
you, but go, for God's sake go! You see how I am suffering!''
Natasha cried angrily, in a voice of despair and repressed
irritation. Sonya burst into sobs and ran from the room.

Natasha went to the table and without a moment's reflection wrote
that answer to Princess Mary which she had been unable to write
all the morning. In this letter she said briefly that all their
misunderstandings were at an end; that availing herself of the
magnanimity of Prince Andrew who when he went abroad had given
her her freedom, she begged Princess Mary to forget everything
and forgive her if she had been to blame toward her, but that she
could not be his wife.  At that moment this all seemed quite
easy, simple, and clear to Natasha.

On Friday the Rostovs were to return to the country, but on
Wednesday the count went with the prospective purchaser to his
estate near Moscow.

On the day the count left, Sonya and Natasha were invited to a
big dinner party at the Karagins', and Marya Dmitrievna took them
there. At that party Natasha again met Anatole, and Sonya noticed
that she spoke to him, trying not to be overheard, and that all
through dinner she was more agitated than ever. When they got
home Natasha was the first to begin the explanation Sonya
expected.

``There, Sonya, you were talking all sorts of nonsense about
him,'' Natasha began in a mild voice such as children use when
they wish to be praised. ``We have had an explanation today.''

``Well, what happened? What did he say? Natasha, how glad I am
you're not angry with me! Tell me everything---the whole
truth. What did he say?''

Natasha became thoughtful.

``Oh, Sonya, if you knew him as I do! He said... He asked me what
I had promised Bolkonski. He was glad I was free to refuse him.''

Sonya sighed sorrowfully.

``But you haven't refused Bolkonski?'' said she.

``Perhaps I have. Perhaps all is over between me and
Bolkonski. Why do you think so badly of me?''

``I don't think anything, only I don't understand this...''

``Wait a bit, Sonya, you'll understand everything. You'll see
what a man he is! Now don't think badly of me or of him. I don't
think badly of anyone: I love and pity everybody. But what am I
to do?''

Sonya did not succumb to the tender tone Natasha used toward
her. The more emotional and ingratiating the expression of
Natasha's face became, the more serious and stern grew Sonya's.

``Natasha,'' said she, ``you asked me not to speak to you, and I
haven't spoken, but now you yourself have begun. I don't trust
him, Natasha. Why this secrecy?''

``Again, again!'' interrupted Natasha.

``Natasha, I am afraid for you!''

``Afraid of what?''

``I am afraid you're going to your ruin,'' said Sonya resolutely,
and was herself horrified at what she had said.

Anger again showed in Natasha's face.

``And I'll go to my ruin, I will, as soon as possible! It's not
your business! It won't be you, but I, who'll suffer. Leave me
alone, leave me alone! I hate you!''

``Natasha!'' moaned Sonya, aghast.

``I hate you, I hate you! You're my enemy forever!'' And Natasha
ran out of the room.

Natasha did not speak to Sonya again and avoided her. With the
same expression of agitated surprise and guilt she went about the
house, taking up now one occupation, now another, and at once
abandoning them.

Hard as it was for Sonya, she watched her friend and did not let
her out of her sight.

The day before the count was to return, Sonya noticed that
Natasha sat by the drawing-room window all the morning as if
expecting something and that she made a sign to an officer who
drove past, whom Sonya took to be Anatole.

Sonya began watching her friend still more attentively and
noticed that at dinner and all that evening Natasha was in a
strange and unnatural state. She answered questions at random,
began sentences she did not finish, and laughed at everything.

After tea Sonya noticed a housemaid at Natasha's door timidly
waiting to let her pass. She let the girl go in, and then
listening at the door learned that another letter had been
delivered.

Then suddenly it became clear to Sonya that Natasha had some
dreadful plan for that evening. Sonya knocked at her
door. Natasha did not let her in.

``She will run away with him!'' thought Sonya. ``She is capable
of anything. There was something particularly pathetic and
resolute in her face today. She cried as she said good-by to
Uncle,'' Sonya remembered.  ``Yes, that's it, she means to elope
with him, but what am I to do?''  thought she, recalling all the
signs that clearly indicated that Natasha had some terrible
intention. ``The count is away. What am I to do? Write to Kuragin
demanding an explanation? But what is there to oblige him to
reply? Write to Pierre, as Prince Andrew asked me to in case of
some misfortune?... But perhaps she really has already refused
Bolkonski---she sent a letter to Princess Mary yesterday. And
Uncle is away...'' To tell Marya Dmitrievna who had such faith in
Natasha seemed to Sonya terrible.  ``Well, anyway,'' thought
Sonya as she stood in the dark passage, ``now or never I must
prove that I remember the family's goodness to me and that I love
Nicholas. Yes! If I don't sleep for three nights I'll not leave
this passage and will hold her back by force and will and not let
the family be disgraced,'' thought she.

% % % % % % % % % % % % % % % % % % % % % % % % % % % % % % % % %
% % % % % % % % % % % % % % % % % % % % % % % % % % % % % % % % %
% % % % % % % % % % % % % % % % % % % % % % % % % % % % % % % % %
% % % % % % % % % % % % % % % % % % % % % % % % % % % % % % % % %
% % % % % % % % % % % % % % % % % % % % % % % % % % % % % % % % %
% % % % % % % % % % % % % % % % % % % % % % % % % % % % % % % % %
% % % % % % % % % % % % % % % % % % % % % % % % % % % % % % % % %
% % % % % % % % % % % % % % % % % % % % % % % % % % % % % % % % %
% % % % % % % % % % % % % % % % % % % % % % % % % % % % % % % % %
% % % % % % % % % % % % % % % % % % % % % % % % % % % % % % % % %
% % % % % % % % % % % % % % % % % % % % % % % % % % % % % % % % %
% % % % % % % % % % % % % % % % % % % % % % % % % % % % % %

\chapter*{Chapter XVI}
\ifaudio     
\marginpar{
\href{http://ia601406.us.archive.org/23/items/war_and_peace_08_0810_librivox/war_and_peace_08_16_tolstoy_64kb.mp3}{Audio}} 
\fi

\lettrine[lines=2, loversize=0.3, lraise=0]{\initfamily A}{natole}
had lately moved to Dolokhov's. The plan for Natalie
Rostova's abduction had been arranged and the preparations made
by Dolokhov a few days before, and on the day that Sonya, after
listening at Natasha's door, resolved to safeguard her, it was to
have been put into execution.  Natasha had promised to come out
to Kuragin at the back porch at ten that evening. Kuragin was to
put her into a troyka he would have ready and to drive her forty
miles to the village of Kamenka, where an unfrocked priest was in
readiness to perform a marriage ceremony over them. At Kamenka a
relay of horses was to wait which would take them to the Warsaw
highroad, and from there they would hasten abroad with post
horses.

Anatole had a passport, an order for post horses, ten thousand
rubles he had taken from his sister and another ten thousand
borrowed with Dolokhov's help.

Two witnesses for the mock marriage---Khvostikov, a retired petty
official whom Dolokhov made use of in his gambling transactions,
and Makarin, a retired hussar, a kindly, weak fellow who had an
unbounded affection for Kuragin---were sitting at tea in
Dolokhov's front room.

In his large study, the walls of which were hung to the ceiling
with Persian rugs, bearskins, and weapons, sat Dolokhov in a
traveling cloak and high boots, at an open desk on which lay an
abacus and some bundles of paper money. Anatole, with uniform
unbuttoned, walked to and fro from the room where the witnesses
were sitting, through the study to the room behind, where his
French valet and others were packing the last of his
things. Dolokhov was counting the money and noting something
down.

``Well,'' he said, ``Khvostikov must have two thousand.''

``Give it to him, then,'' said Anatole.

``Makarka'' (their name for Makarin) ``will go through fire and
water for you for nothing. So here are our accounts all
settled,'' said Dolokhov, showing him the memorandum. ``Is that
right?''

``Yes, of course,'' returned Anatole, evidently not listening to
Dolokhov and looking straight before him with a smile that did
not leave his face.

Dolokhov banged down the lid of his desk and turned to Anatole
with an ironic smile:

``Do you know? You'd really better drop it all. There's still
time!''

``Fool,'' retorted Anatole. ``Don't talk nonsense! If you only
knew... it's the devil knows what!''

``No, really, give it up!'' said Dolokhov. ``I am speaking
seriously. It's no joke, this plot you've hatched.''

``What, teasing again? Go to the devil! Eh?'' said Anatole,
making a grimace. ``Really it's no time for your stupid jokes,''
and he left the room.

Dolokhov smiled contemptuously and condescendingly when Anatole
had gone out.

``You wait a bit,'' he called after him. ``I'm not joking, I'm
talking sense. Come here, come here!''

Anatole returned and looked at Dolokhov, trying to give him his
attention and evidently submitting to him involuntarily.

``Now listen to me. I'm telling you this for the last time. Why
should I joke about it? Did I hinder you? Who arranged everything
for you? Who found the priest and got the passport? Who raised
the money? I did it all.''

``Well, thank you for it. Do you think I am not grateful?'' And
Anatole sighed and embraced Dolokhov.

``I helped you, but all the same I must tell you the truth; it is
a dangerous business, and if you think about it---a stupid
business. Well, you'll carry her off---all right! Will they let
it stop at that? It will come out that you're already
married. Why, they'll have you in the criminal court...''

``Oh, nonsense, nonsense!'' Anatole ejaculated and again made a
grimace.  ``Didn't I explain to you? What?'' And Anatole, with
the partiality dull-witted people have for any conclusion they
have reached by their own reasoning, repeated the argument he had
already put to Dolokhov a hundred times. ``Didn't I explain to
you that I have come to this conclusion: if this marriage is
invalid,'' he went on, crooking one finger, ``then I have nothing
to answer for; but if it is valid, no matter! Abroad no one will
know anything about it. Isn't that so? And don't talk to me,
don't, don't.''

``Seriously, you'd better drop it! You'll only get yourself into
a mess!''

``Go to the devil!'' cried Anatole and, clutching his hair, left
the room, but returned at once and dropped into an armchair in
front of Dolokhov with his feet turned under him. ``It's the very
devil! What? Feel how it beats!'' He took Dolokhov's hand and put
it on his heart. ``What a foot, my dear fellow! What a glance! A
goddess!'' he added in French. ``What?''

Dolokhov with a cold smile and a gleam in his handsome insolent
eyes looked at him---evidently wishing to get some more amusement
out of him.

``Well and when the money's gone, what then?''

``What then? Eh?'' repeated Anatole, sincerely perplexed by a
thought of the future. ``What then?... Then, I don't know... But
why talk nonsense!'' He glanced at his watch. ``It's time!''

Anatole went into the back room.

``Now then! Nearly ready? You're dawdling!'' he shouted to the
servants.

Dolokhov put away the money, called a footman whom he ordered to
bring something for them to eat and drink before the journey, and
went into the room where Khvostikov and Makarin were sitting.

Anatole lay on the sofa in the study leaning on his elbow and
smiling pensively, while his handsome lips muttered tenderly to
himself.

``Come and eat something. Have a drink!'' Dolokhov shouted to him
from the other room.

``I don't want to,'' answered Anatole continuing to smile.

``Come! Balaga is here.''

Anatole rose and went into the dining room. Balaga was a famous
troyka driver who had known Dolokhov and Anatole some six years
and had given them good service with his troykas. More than once
when Anatole's regiment was stationed at Tver he had taken him
from Tver in the evening, brought him to Moscow by daybreak, and
driven him back again the next night. More than once he had
enabled Dolokhov to escape when pursued. More than once he had
driven them through the town with gypsies and \emph{ladykins} as
he called the cocottes. More than once in their service he had
run over pedestrians and upset vehicles in the streets of Moscow
and had always been protected from the consequences by \emph{my
gentlemen} as he called them. He had ruined more than one horse
in their service. More than once they had beaten him, and more
than once they had made him drunk on champagne and Madeira, which
he loved; and he knew more than one thing about each of them
which would long ago have sent an ordinary man to Siberia. They
often called Balaga into their orgies and made him drink and
dance at the gypsies', and more than one thousand rubles of their
money had passed through his hands. In their service he risked
his skin and his life twenty times a year, and in their service
had lost more horses than the money he had from them would
buy. But he liked them; liked that mad driving at twelve miles an
hour, liked upsetting a driver or running down a pedestrian, and
flying at full gallop through the Moscow streets. He liked to
hear those wild, tipsy shouts behind him: ``Get on! Get on!''
when it was impossible to go any faster. He liked giving a
painful lash on the neck to some peasant who, more dead than
alive, was already hurrying out of his way. ``Real gentlemen!''
he considered them.

Anatole and Dolokhov liked Balaga too for his masterly driving
and because he liked the things they liked. With others Balaga
bargained, charging twenty-five rubles for a two hours' drive,
and rarely drove himself, generally letting his young men do
so. But with \emph{his gentlemen} he always drove himself and
never demanded anything for his work. Only a couple of times a
year---when he knew from their valets that they had money in
hand---he would turn up of a morning quite sober and with a deep
bow would ask them to help him. The gentlemen always made him sit
down.

``Do help me out, Theodore Ivanych, sir,'' or ``your
excellency,'' he would say. ``I am quite out of horses. Let me
have what you can to go to the fair.''

And Anatole and Dolokhov, when they had money, would give him a
thousand or a couple of thousand rubles.

Balaga was a fair-haired, short, and snub-nosed peasant of about
twenty-seven; red-faced, with a particularly red thick neck,
glittering little eyes, and a small beard. He wore a fine,
dark-blue, silk-lined cloth coat over a sheepskin.

On entering the room now he crossed himself, turning toward the
front corner of the room, and went up to Dolokhov, holding out a
small, black hand.

``Theodore Ivanych!'' he said, bowing.

``How d'you do, friend? Well, here he is!''

``Good day, your excellency!'' he said, again holding out his
hand to Anatole who had just come in.

``I say, Balaga,'' said Anatole, putting his hands on the man's
shoulders, ``do you care for me or not? Eh? Now, do me a
service... What horses have you come with? Eh?''

``As your messenger ordered, your special beasts,'' replied
Balaga.

``Well, listen, Balaga! Drive all three to death but get me there
in three hours. Eh?''

``When they are dead, what shall I drive?'' said Balaga with a
wink.

``Mind, I'll smash your face in! Don't make jokes!'' cried
Anatole, suddenly rolling his eyes.

``Why joke?'' said the driver, laughing. ``As if I'd grudge my
gentlemen anything! As fast as ever the horses can gallop, so
fast we'll go!''

``Ah!'' said Anatole. ``Well, sit down.''

``Yes, sit down!'' said Dolokhov.

``I'll stand, Theodore Ivanych.''

``Sit down; nonsense! Have a drink!'' said Anatole, and filled a
large glass of Madeira for him.

The driver's eyes sparkled at the sight of the wine. After
refusing it for manners' sake, he drank it and wiped his mouth
with a red silk handkerchief he took out of his cap.

``And when are we to start, your excellency?''

``Well...'' Anatole looked at his watch. ``We'll start at
once. Mind, Balaga! You'll get there in time? Eh?''

``That depends on our luck in starting, else why shouldn't we be
there in time?'' replied Balaga. ``Didn't we get you to Tver in
seven hours? I think you remember that, your excellency?''

``Do you know, one Christmas I drove from Tver,'' said Anatole,
smilingly at the recollection and turning to Makarin who gazed
rapturously at him with wide-open eyes. ``Will you believe it,
Makarka, it took one's breath away, the rate we flew. We came
across a train of loaded sleighs and drove right over two of
them. Eh?''

``Those were horses!'' Balaga continued the tale. ``That time I'd
harnessed two young side horses with the bay in the shafts,'' he
went on, turning to Dolokhov. ``Will you believe it, Theodore
Ivanych, those animals flew forty miles? I couldn't hold them in,
my hands grew numb in the sharp frost so that I threw down the
reins---'Catch hold yourself, your excellency!' says I, and I
just tumbled on the bottom of the sleigh and sprawled there. It
wasn't a case of urging them on, there was no holding them in
till we reached the place. The devils took us there in three
hours! Only the near one died of it.''

% % % % % % % % % % % % % % % % % % % % % % % % % % % % % % % % %
% % % % % % % % % % % % % % % % % % % % % % % % % % % % % % % % %
% % % % % % % % % % % % % % % % % % % % % % % % % % % % % % % % %
% % % % % % % % % % % % % % % % % % % % % % % % % % % % % % % % %
% % % % % % % % % % % % % % % % % % % % % % % % % % % % % % % % %
% % % % % % % % % % % % % % % % % % % % % % % % % % % % % % % % %
% % % % % % % % % % % % % % % % % % % % % % % % % % % % % % % % %
% % % % % % % % % % % % % % % % % % % % % % % % % % % % % % % % %
% % % % % % % % % % % % % % % % % % % % % % % % % % % % % % % % %
% % % % % % % % % % % % % % % % % % % % % % % % % % % % % % % % %
% % % % % % % % % % % % % % % % % % % % % % % % % % % % % % % % %
% % % % % % % % % % % % % % % % % % % % % % % % % % % % % %

\chapter*{Chapter XVII}
\ifaudio     
\marginpar{
\href{http://ia601406.us.archive.org/23/items/war_and_peace_08_0810_librivox/war_and_peace_08_17_tolstoy_64kb.mp3}{Audio}} 
\fi

\lettrine[lines=2, loversize=0.3, lraise=0]{\initfamily A}{natole}
went out of the room and returned a few minutes later
wearing a fur coat girt with a silver belt, and a sable cap
jauntily set on one side and very becoming to his handsome
face. Having looked in a mirror, and standing before Dolokhov in
the same pose he had assumed before it, he lifted a glass of
wine.

``Well, good-by, Theodore. Thank you for everything and
farewell!'' said Anatole. ``Well, comrades and friends...'' he
considered for a moment ``...of my youth, farewell!'' he said,
turning to Makarin and the others.

Though they were all going with him, Anatole evidently wished to
make something touching and solemn out of this address to his
comrades. He spoke slowly in a loud voice and throwing out his
chest slightly swayed one leg.

``All take glasses; you too, Balaga. Well, comrades and friends
of my youth, we've had our fling and lived and reveled. Eh? And
now, when shall we meet again? I am going abroad. We have had a
good time---now farewell, lads! To our health! Hurrah!...'' he
cried, and emptying his glass flung it on the floor.

``To your health!'' said Balaga who also emptied his glass, and
wiped his mouth with his handkerchief.

Makarin embraced Anatole with tears in his eyes.

``Ah, Prince, how sorry I am to part from you!''

``Let's go. Let's go! cried Anatole.''

Balaga was about to leave the room.

``No, stop!'' said Anatole. ``Shut the door; we have first to sit
down.  That's the way.''

They shut the door and all sat down.

``Now, quick march, lads!'' said Anatole, rising.

Joseph, his valet, handed him his sabretache and saber, and they
all went out into the vestibule.

``And where's the fur cloak?'' asked Dolokhov. ``Hey, Ignatka! Go
to Matrena Matrevna and ask her for the sable cloak. I have heard
what elopements are like,'' continued Dolokhov with a
wink. ``Why, she'll rush out more dead than alive just in the
things she is wearing; if you delay at all there'll be tears and
'Papa' and 'Mamma,' and she's frozen in a minute and must go
back---but you wrap the fur cloak round her first thing and carry
her to the sleigh.''

The valet brought a woman's fox-lined cloak.

``Fool, I told you the sable one! Hey, Matrena, the sable!'' he
shouted so that his voice rang far through the rooms.

A handsome, slim, and pale-faced gypsy girl with glittering black
eyes and curly blue-black hair, wearing a red shawl, ran out with
a sable mantle on her arm.

``Here, I don't grudge it---take it!'' she said, evidently afraid
of her master and yet regretful of her cloak.

Dolokhov, without answering, took the cloak, threw it over
Matrena, and wrapped her up in it.

``That's the way,'' said Dolokhov, ``and then so!'' and he turned
the collar up round her head, leaving only a little of the face
uncovered. ``And then so, do you see?'' and he pushed Anatole's
head forward to meet the gap left by the collar, through which
Matrena's brilliant smile was seen.

``Well, good-by, Matrena,'' said Anatole, kissing her. ``Ah, my
revels here are over. Remember me to Steshka. There, good-by!
Good-bye, Matrena, wish me luck!''

``Well, Prince, may God give you great luck!'' said Matrena in
her gypsy accent.

Two troykas were standing before the porch and two young drivers
were holding the horses. Balaga took his seat in the front one
and holding his elbows high arranged the reins
deliberately. Anatole and Dolokhov got in with him. Makarin,
Khvostikov, and a valet seated themselves in the other sleigh.

``Well, are you ready?'' asked Balaga.

``Go!'' he cried, twisting the reins round his hands, and the
troyka tore down the Nikitski Boulevard.

``Tproo! Get out of the way! Hi!... Tproo!...'' The shouting of
Balaga and of the sturdy young fellow seated on the box was all
that could be heard. On the Arbat Square the troyka caught
against a carriage; something cracked, shouts were heard, and the
troyka flew along the Arbat Street.

After taking a turn along the Podnovinski Boulevard, Balaga began
to rein in, and turning back drew up at the crossing of the old
Konyusheny Street.

The young fellow on the box jumped down to hold the horses and
Anatole and Dolokhov went along the pavement. When they reached
the gate Dolokhov whistled. The whistle was answered, and a
maidservant ran out.

``Come into the courtyard or you'll be seen; she'll come out
directly,'' said she.

Dolokhov stayed by the gate. Anatole followed the maid into the
courtyard, turned the corner, and ran up into the porch.

He was met by Gabriel, Marya Dmitrievna's gigantic footman.

``Come to the mistress, please,'' said the footman in his deep
bass, intercepting any retreat.

``To what Mistress? Who are you?'' asked Anatole in a breathless
whisper.

``Kindly step in, my orders are to bring you in.''

``Kuragin! Come back!'' shouted Dolokhov. ``Betrayed! Back!''

Dolokhov, after Anatole entered, had remained at the wicket gate
and was struggling with the yard porter who was trying to lock
it. With a last desperate effort Dolokhov pushed the porter
aside, and when Anatole ran back seized him by the arm, pulled
him through the wicket, and ran back with him to the troyka.

% % % % % % % % % % % % % % % % % % % % % % % % % % % % % % % % %
% % % % % % % % % % % % % % % % % % % % % % % % % % % % % % % % %
% % % % % % % % % % % % % % % % % % % % % % % % % % % % % % % % %
% % % % % % % % % % % % % % % % % % % % % % % % % % % % % % % % %
% % % % % % % % % % % % % % % % % % % % % % % % % % % % % % % % %
% % % % % % % % % % % % % % % % % % % % % % % % % % % % % % % % %
% % % % % % % % % % % % % % % % % % % % % % % % % % % % % % % % %
% % % % % % % % % % % % % % % % % % % % % % % % % % % % % % % % %
% % % % % % % % % % % % % % % % % % % % % % % % % % % % % % % % %
% % % % % % % % % % % % % % % % % % % % % % % % % % % % % % % % %
% % % % % % % % % % % % % % % % % % % % % % % % % % % % % % % % %
% % % % % % % % % % % % % % % % % % % % % % % % % % % % % %

\chapter*{Chapter XVIII}
\ifaudio     
\marginpar{
\href{http://ia601406.us.archive.org/23/items/war_and_peace_08_0810_librivox/war_and_peace_08_18_tolstoy_64kb.mp3}{Audio}} 
\fi

\lettrine[lines=2, loversize=0.3, lraise=0]{\initfamily M}{arya}
Dmitrievna, having found Sonya weeping in the corridor,
made her confess everything, and intercepting the note to Natasha
she read it and went into Natasha's room with it in her hand.

``You shameless good-for-nothing!'' said she. ``I won't hear a
word.''

Pushing back Natasha who looked at her with astonished but
tearless eyes, she locked her in; and having given orders to the
yard porter to admit the persons who would be coming that
evening, but not to let them out again, and having told the
footman to bring them up to her, she seated herself in the
drawing room to await the abductors.

When Gabriel came to inform her that the men who had come had run
away again, she rose frowning, and clasping her hands behind her
paced through the rooms a long time considering what she should
do. Toward midnight she went to Natasha's room fingering the key
in her pocket.  Sonya was sitting sobbing in the
corridor. ``Marya Dmitrievna, for God's sake let me in to her!''
she pleaded, but Marya Dmitrievna unlocked the door and went in
without giving her an answer... ``Disgusting, abominable... In my
house... horrid girl, hussy! I'm only sorry for her father!''
thought she, trying to restrain her wrath. ``Hard as it may be,
I'll tell them all to hold their tongues and will hide it from
the count.'' She entered the room with resolute steps. Natasha
lying on the sofa, her head hidden in her hands, and she did not
stir. She was in just the same position in which Marya Dmitrievna
had left her.

``A nice girl! Very nice!'' said Marya Dmitrievna. ``Arranging
meetings with lovers in my house! It's no use pretending: you
listen when I speak to you!'' And Marya Dmitrievna touched her
arm. ``Listen when I speak!  You've disgraced yourself like the
lowest of hussies. I'd treat you differently, but I'm sorry for
your father, so I will conceal it.''

Natasha did not change her position, but her whole body heaved
with noiseless, convulsive sobs which choked her. Marya
Dmitrievna glanced round at Sonya and seated herself on the sofa
beside Natasha.

``It's lucky for him that he escaped me; but I'll find him!'' she
said in her rough voice. ``Do you hear what I am saying or not?''
she added.

She put her large hand under Natasha's face and turned it toward
her.  Both Marya Dmitrievna and Sonya were amazed when they saw
how Natasha looked. Her eyes were dry and glistening, her lips
compressed, her cheeks sunken.

``Let me be!... What is it to me?... I shall die!'' she muttered,
wrenching herself from Marya Dmitrievna's hands with a vicious
effort and sinking down again into her former position.

``Natalie!'' said Marya Dmitrievna. ``I wish for your good. Lie
still, stay like that then, I won't touch you. But listen. I
won't tell you how guilty you are. You know that yourself. But
when your father comes back tomorrow what am I to tell him? Eh?''

Again Natasha's body shook with sobs.

``Suppose he finds out, and your brother, and your betrothed?''

``I have no betrothed: I have refused him!'' cried Natasha.

``That's all the same,'' continued Marya Dmitrievna. ``If they
hear of this, will they let it pass? He, your father, I know
him... if he challenges him to a duel will that be all right?
Eh?''

``Oh, let me be! Why have you interfered at all? Why? Why? Who
asked you to?'' shouted Natasha, raising herself on the sofa and
looking malignantly at Marya Dmitrievna.

``But what did you want?'' cried Marya Dmitrievna, growing angry
again.  ``Were you kept under lock and key? Who hindered his
coming to the house?  Why carry you off as if you were some gypsy
singing girl?... Well, if he had carried you off... do you think
they wouldn't have found him? Your father, or brother, or your
betrothed? And he's a scoundrel, a wretch---that's a fact!''

``He is better than any of you!'' exclaimed Natasha getting
up. ``If you hadn't interfered... Oh, my God! What is it all?
What is it? Sonya, why?... Go away!''

And she burst into sobs with the despairing vehemence with which
people bewail disasters they feel they have themselves
occasioned. Marya Dmitrievna was to speak again but Natasha cried
out:

``Go away! Go away! You all hate and despise me!'' and she threw
herself back on the sofa.

Marya Dmitrievna went on admonishing her for some time, enjoining
on her that it must all be kept from her father and assuring her
that nobody would know anything about it if only Natasha herself
would undertake to forget it all and not let anyone see that
something had happened.  Natasha did not reply, nor did she sob
any longer, but she grew cold and had a shivering fit. Marya
Dmitrievna put a pillow under her head, covered her with two
quilts, and herself brought her some lime-flower water, but
Natasha did not respond to her.

``Well, let her sleep,'' said Marya Dmitrievna as she went out of
the room supposing Natasha to be asleep.

But Natasha was not asleep; with pale face and fixed wide-open
eyes she looked straight before her. All that night she did not
sleep or weep and did not speak to Sonya who got up and went to
her several times.

Next day Count Rostov returned from his estate near Moscow in
time for lunch as he had promised. He was in very good spirits;
the affair with the purchaser was going on satisfactorily, and
there was nothing to keep him any longer in Moscow, away from the
countess whom he missed. Marya Dmitrievna met him and told him
that Natasha had been very unwell the day before and that they
had sent for the doctor, but that she was better now. Natasha had
not left her room that morning. With compressed and parched lips
and dry fixed eyes, she sat at the window, uneasily watching the
people who drove past and hurriedly glancing round at anyone who
entered the room. She was evidently expecting news of him and
that he would come or would write to her.

When the count came to see her she turned anxiously round at the
sound of a man's footstep, and then her face resumed its cold and
malevolent expression. She did not even get up to greet
him. ``What is the matter with you, my angel? Are you ill?''
asked the count.

After a moment's silence Natasha answered: ``Yes, ill.''

In reply to the count's anxious inquiries as to why she was so
dejected and whether anything had happened to her betrothed, she
assured him that nothing had happened and asked him not to
worry. Marya Dmitrievna confirmed Natasha's assurances that
nothing had happened. From the pretense of illness, from his
daughter's distress, and by the embarrassed faces of Sonya and
Marya Dmitrievna, the count saw clearly that something had gone
wrong during his absence, but it was so terrible for him to think
that anything disgraceful had happened to his beloved daughter,
and he so prized his own cheerful tranquillity, that he avoided
inquiries and tried to assure himself that nothing particularly
had happened; and he was only dissatisfied that her indisposition
delayed their return to the country.

% % % % % % % % % % % % % % % % % % % % % % % % % % % % % % % % %
% % % % % % % % % % % % % % % % % % % % % % % % % % % % % % % % %
% % % % % % % % % % % % % % % % % % % % % % % % % % % % % % % % %
% % % % % % % % % % % % % % % % % % % % % % % % % % % % % % % % %
% % % % % % % % % % % % % % % % % % % % % % % % % % % % % % % % %
% % % % % % % % % % % % % % % % % % % % % % % % % % % % % % % % %
% % % % % % % % % % % % % % % % % % % % % % % % % % % % % % % % %
% % % % % % % % % % % % % % % % % % % % % % % % % % % % % % % % %
% % % % % % % % % % % % % % % % % % % % % % % % % % % % % % % % %
% % % % % % % % % % % % % % % % % % % % % % % % % % % % % % % % %
% % % % % % % % % % % % % % % % % % % % % % % % % % % % % % % % %
% % % % % % % % % % % % % % % % % % % % % % % % % % % % % %

\chapter*{Chapter XIX}
\ifaudio     
\marginpar{
\href{http://ia601406.us.archive.org/23/items/war_and_peace_08_0810_librivox/war_and_peace_08_19_tolstoy_64kb.mp3}{Audio}} 
\fi

\lettrine[lines=2, loversize=0.3, lraise=0]{\initfamily F}{rom}
the day his wife arrived in Moscow Pierre had been intending
to go away somewhere, so as not to be near her. Soon after the
Rostovs came to Moscow the effect Natasha had on him made him
hasten to carry out his intention. He went to Tver to see Joseph
Alexeevich's widow, who had long since promised to hand over to
him some papers of her deceased husband's.

When he returned to Moscow Pierre was handed a letter from Marya
Dmitrievna asking him to come and see her on a matter of great
importance relating to Andrew Bolkonski and his betrothed. Pierre
had been avoiding Natasha because it seemed to him that his
feeling for her was stronger than a married man's should be for
his friend's fiancee.  Yet some fate constantly threw them
together.

``What can have happened? And what can they want with me?''
thought he as he dressed to go to Marya Dmitrievna's. ``If only
Prince Andrew would hurry up and come and marry her!'' thought he
on his way to the house.

On the Tverskoy Boulevard a familiar voice called to him.

``Pierre! Been back long?'' someone shouted. Pierre raised his
head. In a sleigh drawn by two gray trotting-horses that were
bespattering the dashboard with snow, Anatole and his constant
companion Makarin dashed past. Anatole was sitting upright in the
classic pose of military dandies, the lower part of his face
hidden by his beaver collar and his head slightly bent. His face
was fresh and rosy, his white-plumed hat, tilted to one side,
disclosed his curled and pomaded hair besprinkled with powdery
snow.

``Yes, indeed, that's a true sage,'' thought Pierre. ``He sees
nothing beyond the pleasure of the moment, nothing troubles him
and so he is always cheerful, satisfied, and serene. What
wouldn't I give to be like him!'' he thought enviously.

In Marya Dmitrievna's anteroom the footman who helped him off
with his fur coat said that the mistress asked him to come to her
bedroom.

When he opened the ballroom door Pierre saw Natasha sitting at
the window, with a thin, pale, and spiteful face. She glanced
round at him, frowned, and left the room with an expression of
cold dignity.

``What has happened?'' asked Pierre, entering Marya Dmitrievna's
room.

``Fine doings!'' answered Dmitrievna. ``For fifty-eight years
have I lived in this world and never known anything so
disgraceful!''

And having put him on his honor not to repeat anything she told
him, Marya Dmitrievna informed him that Natasha had refused
Prince Andrew without her parents' knowledge and that the cause
of this was Anatole Kuragin into whose society Pierre's wife had
thrown her and with whom Natasha had tried to elope during her
father's absence, in order to be married secretly.

Pierre raised his shoulders and listened open-mouthed to what was
told him, scarcely able to believe his own ears. That Prince
Andrew's deeply loved affianced wife---the same Natasha Rostova
who used to be so charming---should give up Bolkonski for that
fool Anatole who was already secretly married (as Pierre knew),
and should be so in love with him as to agree to run away with
him, was something Pierre could not conceive and could not
imagine.

He could not reconcile the charming impression he had of Natasha,
whom he had known from a child, with this new conception of her
baseness, folly, and cruelty. He thought of his wife. ``They are
all alike!'' he said to himself, reflecting that he was not the
only man unfortunate enough to be tied to a bad woman. But still
he pitied Prince Andrew to the point of tears and sympathized
with his wounded pride, and the more he pitied his friend the
more did he think with contempt and even with disgust of that
Natasha who had just passed him in the ballroom with such a look
of cold dignity. He did not know that Natasha's soul was
overflowing with despair, shame, and humiliation, and that it was
not her fault that her face happened to assume an expression of
calm dignity and severity.

``But how get married?'' said Pierre, in answer to Marya
Dmitrievna. ``He could not marry---he is married!''

``Things get worse from hour to hour!'' ejaculated Marya
Dmitrievna. ``A nice youth! What a scoundrel! And she's expecting
him---expecting him since yesterday. She must be told! Then at
least she won't go on expecting him.''

After hearing the details of Anatole's marriage from Pierre, and
giving vent to her anger against Anatole in words of abuse, Marya
Dmitrievna told Pierre why she had sent for him. She was afraid
that the count or Bolkonski, who might arrive at any moment, if
they knew of this affair (which she hoped to hide from them)
might challenge Anatole to a duel, and she therefore asked Pierre
to tell his brother-in-law in her name to leave Moscow and not
dare to let her set eyes on him again. Pierre---only now
realizing the danger to the old count, Nicholas, and Prince
Andrew---promised to do as she wished. Having briefly and exactly
explained her wishes to him, she let him go to the drawing room.

``Mind, the count knows nothing. Behave as if you know nothing
either,'' she said. ``And I will go and tell her it is no use
expecting him! And stay to dinner if you care to!'' she called
after Pierre.

Pierre met the old count, who seemed nervous and upset. That
morning Natasha had told him that she had rejected Bolkonski.

``Troubles, troubles, my dear fellow!'' he said to Pierre. ``What
troubles one has with these girls without their mother! I do so
regret having come here... I will be frank with you. Have you
heard she has broken off her engagement without consulting
anybody? It's true this engagement never was much to my
liking. Of course he is an excellent man, but still, with his
father's disapproval they wouldn't have been happy, and Natasha
won't lack suitors. Still, it has been going on so long, and to
take such a step without father's or mother's consent! And now
she's ill, and God knows what! It's hard, Count, hard to manage
daughters in their mother's absence...''

Pierre saw that the count was much upset and tried to change the
subject, but the count returned to his troubles.

Sonya entered the room with an agitated face.

``Natasha is not quite well; she's in her room and would like to
see you.  Marya Dmitrievna is with her and she too asks you to
come.''

``Yes, you are a great friend of Bolkonski's, no doubt she wants
to send him a message,'' said the count. ``Oh dear! Oh dear! How
happy it all was!''

And clutching the spare gray locks on his temples the count left
the room.

When Marya Dmitrievna told Natasha that Anatole was married,
Natasha did not wish to believe it and insisted on having it
confirmed by Pierre himself. Sonya told Pierre this as she led
him along the corridor to Natasha's room.

Natasha, pale and stern, was sitting beside Marya Dmitrievna, and
her eyes, glittering feverishly, met Pierre with a questioning
look the moment he entered. She did not smile or nod, but only
gazed fixedly at him, and her look asked only one thing: was he a
friend, or like the others an enemy in regard to Anatole? As for
Pierre, he evidently did not exist for her.

``He knows all about it,'' said Marya Dmitrievna pointing to
Pierre and addressing Natasha. ``Let him tell you whether I have
told the truth.''

Natasha looked from one to the other as a hunted and wounded
animal looks at the approaching dogs and sportsmen.

``Natalya Ilynichna,'' Pierre began, dropping his eyes with a
feeling of pity for her and loathing for the thing he had to do,
``whether it is true or not should make no difference to you,
because...''

``Then it is not true that he's married!''

``Yes, it is true.''

``Has he been married long?'' she asked. ``On your honor?...''

Pierre gave his word of honor.

``Is he still here?'' she asked, quickly.

``Yes, I have just seen him.''

She was evidently unable to speak and made a sign with her hands
that they should leave her alone.

% % % % % % % % % % % % % % % % % % % % % % % % % % % % % % % % %
% % % % % % % % % % % % % % % % % % % % % % % % % % % % % % % % %
% % % % % % % % % % % % % % % % % % % % % % % % % % % % % % % % %
% % % % % % % % % % % % % % % % % % % % % % % % % % % % % % % % %
% % % % % % % % % % % % % % % % % % % % % % % % % % % % % % % % %
% % % % % % % % % % % % % % % % % % % % % % % % % % % % % % % % %
% % % % % % % % % % % % % % % % % % % % % % % % % % % % % % % % %
% % % % % % % % % % % % % % % % % % % % % % % % % % % % % % % % %
% % % % % % % % % % % % % % % % % % % % % % % % % % % % % % % % %
% % % % % % % % % % % % % % % % % % % % % % % % % % % % % % % % %
% % % % % % % % % % % % % % % % % % % % % % % % % % % % % % % % %
% % % % % % % % % % % % % % % % % % % % % % % % % % % % % %

\chapter*{Chapter XX}
\ifaudio     
\marginpar{
\href{http://ia601406.us.archive.org/23/items/war_and_peace_08_0810_librivox/war_and_peace_08_20_tolstoy_64kb.mp3}{Audio}} 
\fi

\lettrine[lines=2, loversize=0.3, lraise=0]{\initfamily P}{ierre}
did not stay for dinner, but left the room and went away
at once.  He drove through the town seeking Anatole Kuragin, at
the thought of whom now the blood rushed to his heart and he felt
a difficulty in breathing. He was not at the ice hills, nor at
the gypsies', nor at Komoneno's. Pierre drove to the club. In the
club all was going on as usual. The members who were assembling
for dinner were sitting about in groups; they greeted Pierre and
spoke of the town news. The footman having greeted him, knowing
his habits and his acquaintances, told him there was a place left
for him in the small dining room and that Prince Michael
Zakharych was in the library, but Paul Timofeevich had not yet
arrived. One of Pierre's acquaintances, while they were talking
about the weather, asked if he had heard of Kuragin's abduction
of Rostova which was talked of in the town, and was it true?
Pierre laughed and said it was nonsense for he had just come from
the Rostovs'. He asked everyone about Anatole. One man told him
he had not come yet, and another that he was coming to
dinner. Pierre felt it strange to see this calm, indifferent
crowd of people unaware of what was going on in his soul. He
paced through the ballroom, waited till everyone had come, and as
Anatole had not turned up did not stay for dinner but drove home.

Anatole, for whom Pierre was looking, dined that day with
Dolokhov, consulting him as to how to remedy this unfortunate
affair. It seemed to him essential to see Natasha. In the evening
he drove to his sister's to discuss with her how to arrange a
meeting. When Pierre returned home after vainly hunting all over
Moscow, his valet informed him that Prince Anatole was with the
countess. The countess' drawing room was full of guests.

Pierre without greeting his wife whom he had not seen since his
return---at that moment she was more repulsive to him than
ever---entered the drawing room and seeing Anatole went up to
him.

``Ah, Pierre,'' said the countess going up to her husband. ``You
don't know what a plight our Anatole...''

She stopped, seeing in the forward thrust of her husband's head,
in his glowing eyes and his resolute gait, the terrible
indications of that rage and strength which she knew and had
herself experienced after his duel with Dolokhov.

``Where you are, there is vice and evil!'' said Pierre to his
wife.  ``Anatole, come with me! I must speak to you,'' he added
in French.

Anatole glanced round at his sister and rose submissively, ready
to follow Pierre. Pierre, taking him by the arm, pulled him
toward himself and was leading him from the room.

``If you allow yourself in my drawing room...'' whispered Helene,
but Pierre did not reply and went out of the room.

Anatole followed him with his usual jaunty step but his face
betrayed anxiety.

Having entered his study Pierre closed the door and addressed
Anatole without looking at him.

``You promised Countess Rostova to marry her and were about to
elope with her, is that so?''

``Mon cher,'' answered Anatole (their whole conversation was in
French), ``I don't consider myself bound to answer questions put
to me in that tone.''

Pierre's face, already pale, became distorted by fury. He seized
Anatole by the collar of his uniform with his big hand and shook
him from side to side till Anatole's face showed a sufficient
degree of terror.

``When I tell you that I must talk to you!...'' repeated Pierre.

``Come now, this is stupid. What?'' said Anatole, fingering a
button of his collar that had been wrenched loose with a bit of
the cloth.

``You're a scoundrel and a blackguard, and I don't know what
deprives me from the pleasure of smashing your head with this!''
said Pierre, expressing himself so artificially because he was
talking French.

He took a heavy paperweight and lifted it threateningly, but at
once put it back in its place.

``Did you promise to marry her?''

``I... I didn't think of it. I never promised, because...''

Pierre interrupted him.

``Have you any letters of hers? Any letters?'' he said, moving
toward Anatole.

Anatole glanced at him and immediately thrust his hand into his
pocket and drew out his pocketbook.

Pierre took the letter Anatole handed him and, pushing aside a
table that stood in his way, threw himself on the sofa.

``I shan't be violent, don't be afraid!'' said Pierre in answer
to a frightened gesture of Anatole's. ``First, the letters,''
said he, as if repeating a lesson to himself. ``Secondly,'' he
continued after a short pause, again rising and again pacing the
room, ``tomorrow you must get out of Moscow.''

``But how can I?...''

``Thirdly,'' Pierre continued without listening to him, ``you
must never breathe a word of what has passed between you and
Countess Rostova. I know I can't prevent your doing so, but if
you have a spark of conscience...'' Pierre paced the room several
times in silence.

Anatole sat at a table frowning and biting his lips.

``After all, you must understand that besides your pleasure there
is such a thing as other people's happiness and peace, and that
you are ruining a whole life for the sake of amusing yourself!
Amuse yourself with women like my wife---with them you are within
your rights, for they know what you want of them. They are armed
against you by the same experience of debauchery; but to promise
a maid to marry her... to deceive, to kidnap... Don't you
understand that it is as mean as beating an old man or a
child?...''

Pierre paused and looked at Anatole no longer with an angry but
with a questioning look.

``I don't know about that, eh?'' said Anatole, growing more
confident as Pierre mastered his wrath. ``I don't know that and
don't want to,'' he said, not looking at Pierre and with a slight
tremor of his lower jaw, ``but you have used such words to
me---'mean' and so on---which as a man of honor I can't allow
anyone to use.''

Pierre glanced at him with amazement, unable to understand what
he wanted.

``Though it was tête-à-tête,'' Anatole continued, ``still I
can't...''

``Is it satisfaction you want?'' said Pierre ironically.

``You could at least take back your words. What? If you want me
to do as you wish, eh?''

``I take them back, I take them back!'' said Pierre, ``and I ask
you to forgive me.'' Pierre involuntarily glanced at the loose
button. ``And if you require money for your journey...''

Anatole smiled. The expression of that base and cringing smile,
which Pierre knew so well in his wife, revolted him.

``Oh, vile and heartless brood!'' he exclaimed, and left the
room.

Next day Anatole left for Petersburg.

% % % % % % % % % % % % % % % % % % % % % % % % % % % % % % % % %
% % % % % % % % % % % % % % % % % % % % % % % % % % % % % % % % %
% % % % % % % % % % % % % % % % % % % % % % % % % % % % % % % % %
% % % % % % % % % % % % % % % % % % % % % % % % % % % % % % % % %
% % % % % % % % % % % % % % % % % % % % % % % % % % % % % % % % %
% % % % % % % % % % % % % % % % % % % % % % % % % % % % % % % % %
% % % % % % % % % % % % % % % % % % % % % % % % % % % % % % % % %
% % % % % % % % % % % % % % % % % % % % % % % % % % % % % % % % %
% % % % % % % % % % % % % % % % % % % % % % % % % % % % % % % % %
% % % % % % % % % % % % % % % % % % % % % % % % % % % % % % % % %
% % % % % % % % % % % % % % % % % % % % % % % % % % % % % % % % %
% % % % % % % % % % % % % % % % % % % % % % % % % % % % % %

\chapter*{Chapter XXI}
\ifaudio
\marginpar{
\href{http://ia601406.us.archive.org/23/items/war_and_peace_08_0810_librivox/war_and_peace_08_21_tolstoy_64kb.mp3}{Audio}} 
\fi

\lettrine[lines=2, loversize=0.3, lraise=0]{\initfamily P}{ierre}
drove to Marya Dmitrievna's to tell her of the fulfillment
of her wish that Kuragin should be banished from Moscow. The
whole house was in a state of alarm and commotion. Natasha was
very ill, having, as Marya Dmitrievna told him in secret,
poisoned herself the night after she had been told that Anatole
was married, with some arsenic she had stealthily procured. After
swallowing a little she had been so frightened that she woke
Sonya and told her what she had done. The necessary antidotes had
been administered in time and she was now out of danger, though
still so weak that it was out of the question to move her to the
country, and so the countess had been sent for. Pierre saw the
distracted count, and Sonya, who had a tear-stained face, but he
could not see Natasha.

Pierre dined at the club that day and heard on all sides gossip
about the attempted abduction of Rostova. He resolutely denied
these rumors, assuring everyone that nothing had happened except
that his brother-in-law had proposed to her and been refused. It
seemed to Pierre that it was his duty to conceal the whole affair
and re-establish Natasha's reputation.

He was awaiting Prince Andrew's return with dread and went every
day to the old prince's for news of him.

Old Prince Bolkonski heard all the rumors current in the town
from Mademoiselle Bourienne and had read the note to Princess
Mary in which Natasha had broken off her engagement. He seemed in
better spirits than usual and awaited his son with great
impatience.

Some days after Anatole's departure Pierre received a note from
Prince Andrew, informing him of his arrival and asking him to
come to see him.

As soon as he reached Moscow, Prince Andrew had received from his
father Natasha's note to Princess Mary breaking off her
engagement (Mademoiselle Bourienne had purloined it from Princess
Mary and given it to the old prince), and he heard from him the
story of Natasha's elopement, with additions.

Prince Andrew had arrived in the evening and Pierre came to see
him next morning. Pierre expected to find Prince Andrew in almost
the same state as Natasha and was therefore surprised on entering
the drawing room to hear him in the study talking in a loud
animated voice about some intrigue going on in Petersburg. The
old prince's voice and another now and then interrupted
him. Princess Mary came out to meet Pierre. She sighed, looking
toward the door of the room where Prince Andrew was, evidently
intending to express her sympathy with his sorrow, but Pierre saw
by her face that she was glad both at what had happened and at
the way her brother had taken the news of Natasha's
faithlessness.

``He says he expected it,'' she remarked. ``I know his pride will
not let him express his feelings, but still he has taken it
better, far better, than I expected. Evidently it had to be...''

``But is it possible that all is really ended?'' asked Pierre.

Princess Mary looked at him with astonishment. She did not
understand how he could ask such a question. Pierre went into the
study. Prince Andrew, greatly changed and plainly in better
health, but with a fresh horizontal wrinkle between his brows,
stood in civilian dress facing his father and Prince Meshcherski,
warmly disputing and vigorously gesticulating. The conversation
was about Speranski---the news of whose sudden exile and alleged
treachery had just reached Moscow.

``Now he is censured and accused by all who were enthusiastic
about him a month ago,'' Prince Andrew was saying, ``and by those
who were unable to understand his aims. To judge a man who is in
disfavor and to throw on him all the blame of other men's
mistakes is very easy, but I maintain that if anything good has
been accomplished in this reign it was done by him, by him
alone.''

He paused at the sight of Pierre. His face quivered and
immediately assumed a vindictive expression.

``Posterity will do him justice,'' he concluded, and at once
turned to Pierre.

``Well, how are you? Still getting stouter?'' he said with
animation, but the new wrinkle on his forehead deepened. ``Yes, I
am well,'' he said in answer to Pierre's question, and smiled.

To Pierre that smile said plainly: ``I am well, but my health is
now of no use to anyone.''

After a few words to Pierre about the awful roads from the Polish
frontier, about people he had met in Switzerland who knew Pierre,
and about M. Dessalles, whom he had brought from abroad to be his
son's tutor, Prince Andrew again joined warmly in the
conversation about Speranski which was still going on between the
two old men.

``If there were treason, or proofs of secret relations with
Napoleon, they would have been made public,'' he said with warmth
and haste. ``I do not, and never did, like Speranski personally,
but I like justice!''

Pierre now recognized in his friend a need with which he was only
too familiar, to get excited and to have arguments about
extraneous matters in order to stifle thoughts that were too
oppressive and too intimate.  When Prince Meshcherski had left,
Prince Andrew took Pierre's arm and asked him into the room that
had been assigned him. A bed had been made up there, and some
open portmanteaus and trunks stood about. Prince Andrew went to
one and took out a small casket, from which he drew a packet
wrapped in paper. He did it all silently and very quickly. He
stood up and coughed. His face was gloomy and his lips
compressed.

``Forgive me for troubling you...''

Pierre saw that Prince Andrew was going to speak of Natasha, and
his broad face expressed pity and sympathy. This expression
irritated Prince Andrew, and in a determined, ringing, and
unpleasant tone he continued:

``I have received a refusal from Countess Rostova and have heard
reports of your brother-in-law having sought her hand, or
something of that kind. Is that true?''

``Both true and untrue,'' Pierre began; but Prince Andrew
interrupted him.

``Here are her letters and her portrait,'' said he.

He took the packet from the table and handed it to Pierre.

``Give this to the countess... if you see her.''

``She is very ill,'' said Pierre.

``Then she is here still?'' said Prince Andrew. ``And Prince
Kuragin?'' he added quickly.

``He left long ago. She has been at death's door.''

``I much regret her illness,'' said Prince Andrew; and he smiled
like his father, coldly, maliciously, and unpleasantly.

``So Monsieur Kuragin has not honored Countess Rostova with his
hand?''  said Prince Andrew, and he snorted several times.

``He could not marry, for he was married already,'' said Pierre.

Prince Andrew laughed disagreeably, again reminding one of his
father.

``And where is your brother-in-law now, if I may ask?'' he said.

``He has gone to Peters... But I don't know,'' said Pierre.

``Well, it doesn't matter,'' said Prince Andrew. ``Tell Countess
Rostova that she was and is perfectly free and that I wish her
all that is good.''

Pierre took the packet. Prince Andrew, as if trying to remember
whether he had something more to say, or waiting to see if Pierre
would say anything, looked fixedly at him.

``I say, do you remember our discussion in Petersburg?'' asked
Pierre, ``about...''

``Yes,'' returned Prince Andrew hastily. ``I said that a fallen
woman should be forgiven, but I didn't say I could forgive her. I
can't.''

``But can this be compared...?'' said Pierre.

Prince Andrew interrupted him and cried sharply: ``Yes, ask her
hand again, be magnanimous, and so on?... Yes, that would be very
noble, but I am unable to follow in that gentleman's
footsteps. If you wish to be my friend never speak to me of
that... of all that! Well, good-by. So you'll give her the
packet?''

Pierre left the room and went to the old prince and Princess
Mary.

The old man seemed livelier than usual. Princess Mary was the
same as always, but beneath her sympathy for her brother, Pierre
noticed her satisfaction that the engagement had been broken
off. Looking at them Pierre realized what contempt and animosity
they all felt for the Rostovs, and that it was impossible in
their presence even to mention the name of her who could give up
Prince Andrew for anyone else.

At dinner the talk turned on the war, the approach of which was
becoming evident. Prince Andrew talked incessantly, arguing now
with his father, now with the Swiss tutor Dessalles, and showing
an unnatural animation, the cause of which Pierre so well
understood.

% % % % % % % % % % % % % % % % % % % % % % % % % % % % % % % % %
% % % % % % % % % % % % % % % % % % % % % % % % % % % % % % % % %
% % % % % % % % % % % % % % % % % % % % % % % % % % % % % % % % %
% % % % % % % % % % % % % % % % % % % % % % % % % % % % % % % % %
% % % % % % % % % % % % % % % % % % % % % % % % % % % % % % % % %
% % % % % % % % % % % % % % % % % % % % % % % % % % % % % % % % %
% % % % % % % % % % % % % % % % % % % % % % % % % % % % % % % % %
% % % % % % % % % % % % % % % % % % % % % % % % % % % % % % % % %
% % % % % % % % % % % % % % % % % % % % % % % % % % % % % % % % %
% % % % % % % % % % % % % % % % % % % % % % % % % % % % % % % % %
% % % % % % % % % % % % % % % % % % % % % % % % % % % % % % % % %
% % % % % % % % % % % % % % % % % % % % % % % % % % % % % %

\chapter*{Chapter XXII}
\ifaudio     
\marginpar{
\href{http://ia601406.us.archive.org/23/items/war_and_peace_08_0810_librivox/war_and_peace_08_22_tolstoy_64kb.mp3}{Audio}} 
\fi

\lettrine[lines=2, loversize=0.3, lraise=0]{\initfamily T}{hat}
same evening Pierre went to the Rostovs' to fulfill the
commission entrusted to him. Natasha was in bed, the count at the
club, and Pierre, after giving the letters to Sonya, went to
Marya Dmitrievna who was interested to know how Prince Andrew had
taken the news. Ten minutes later Sonya came to Marya Dmitrievna.

``Natasha insists on seeing Count Peter Kirilovich,'' said she.

``But how? Are we to take him up to her? The room there has not
been tidied up.''

``No, she has dressed and gone into the drawing room,'' said
Sonya.

Marya Dmitrievna only shrugged her shoulders.

``When will her mother come? She has worried me to death! Now
mind, don't tell her everything!'' said she to Pierre. ``One
hasn't the heart to scold her, she is so much to be pitied, so
much to be pitied.''

Natasha was standing in the middle of the drawing room,
emaciated, with a pale set face, but not at all shamefaced as
Pierre expected to find her. When he appeared at the door she
grew flurried, evidently undecided whether to go to meet him or
to wait till he came up.

Pierre hastened to her. He thought she would give him her hand as
usual; but she, stepping up to him, stopped, breathing heavily,
her arms hanging lifelessly just in the pose she used to stand in
when she went to the middle of the ballroom to sing, but with
quite a different expression of face.

``Peter Kirilovich,'' she began rapidly, ``Prince Bolkonski was
your friend---is your friend,'' she corrected herself. (It seemed
to her that everything that had once been must now be different.)
``He told me once to apply to you...''

Pierre sniffed as he looked at her, but did not speak. Till then
he had reproached her in his heart and tried to despise her, but
he now felt so sorry for her that there was no room in his soul
for reproach.

``He is here now: tell him... to for... forgive me!'' She stopped
and breathed still more quickly, but did not shed tears.

``Yes... I will tell him,'' answered Pierre; ``but...''

He did not know what to say.

Natasha was evidently dismayed at the thought of what he might
think she had meant.

``No, I know all is over,'' she said hurriedly. ``No, that can
never be.  I'm only tormented by the wrong I have done him. Tell
him only that I beg him to forgive, forgive, forgive me for
everything...''

She trembled all over and sat down on a chair.

A sense of pity he had never before known overflowed Pierre's
heart.

``I will tell him, I will tell him everything once more,'' said
Pierre.  ``But... I should like to know one thing...''

``Know what?'' Natasha's eyes asked.

``I should like to know, did you love...'' Pierre did not know
how to refer to Anatole and flushed at the thought of him---``did
you love that bad man?''

``Don't call him bad!'' said Natasha. ``But I don't know, don't
know at all...''

She began to cry and a still greater sense of pity, tenderness,
and love welled up in Pierre. He felt the tears trickle under his
spectacles and hoped they would not be noticed.

``We won't speak of it any more, my dear,'' said Pierre, and his
gentle, cordial tone suddenly seemed very strange to Natasha.

``We won't speak of it, my dear---I'll tell him everything; but
one thing I beg of you, consider me your friend and if you want
help, advice, or simply to open your heart to someone---not now,
but when your mind is clearer think of me!'' He took her hand and
kissed it. ``I shall be happy if it's in my power...''

Pierre grew confused.

``Don't speak to me like that. I am not worth it!'' exclaimed
Natasha and turned to leave the room, but Pierre held her hand.

He knew he had something more to say to her. But when he said it
he was amazed at his own words.

``Stop, stop! You have your whole life before you,'' said he to
her.

``Before me? No! All is over for me,'' she replied with shame and
self-abasement.

``All over?'' he repeated. ``If I were not myself, but the
handsomest, cleverest, and best man in the world, and were free,
I would this moment ask on my knees for your hand and your
love!''

For the first time for many days Natasha wept tears of gratitude
and tenderness, and glancing at Pierre she went out of the room.

Pierre too when she had gone almost ran into the anteroom,
restraining tears of tenderness and joy that choked him, and
without finding the sleeves of his fur cloak threw it on and got
into his sleigh.

``Where to now, your excellency?'' asked the coachman.

``Where to?'' Pierre asked himself. ``Where can I go now? Surely
not to the club or to pay calls?'' All men seemed so pitiful, so
poor, in comparison with this feeling of tenderness and love he
experienced: in comparison with that softened, grateful, last
look she had given him through her tears.

``Home!'' said Pierre, and despite twenty-two degrees of frost
Fahrenheit he threw open the bearskin cloak from his broad chest
and inhaled the air with joy.

It was clear and frosty. Above the dirty, ill-lit streets, above
the black roofs, stretched the dark starry sky. Only looking up
at the sky did Pierre cease to feel how sordid and humiliating
were all mundane things compared with the heights to which his
soul had just been raised.  At the entrance to the Arbat Square
an immense expanse of dark starry sky presented itself to his
eyes. Almost in the center of it, above the Prechistenka
Boulevard, surrounded and sprinkled on all sides by stars but
distinguished from them all by its nearness to the earth, its
white light, and its long uplifted tail, shone the enormous and
brilliant comet of 1812---the comet which was said to portend all
kinds of woes and the end of the world. In Pierre, however, that
comet with its long luminous tail aroused no feeling of fear. On
the contrary he gazed joyfully, his eyes moist with tears, at
this bright comet which, having traveled in its orbit with
inconceivable velocity through immeasurable space, seemed
suddenly---like an arrow piercing the earth---to remain fixed in
a chosen spot, vigorously holding its tail erect, shining and
displaying its white light amid countless other scintillating
stars. It seemed to Pierre that this comet fully responded to
what was passing in his own softened and uplifted soul, now
blossoming into a new life.