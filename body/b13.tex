\part*{Book Thirteen: 1812}

% % % % % % % % % % % % % % % % % % % % % % % % % % % % % % % % %
% % % % % % % % % % % % % % % % % % % % % % % % % % % % % % % % %
% % % % % % % % % % % % % % % % % % % % % % % % % % % % % % % % %
% % % % % % % % % % % % % % % % % % % % % % % % % % % % % % % % %
% % % % % % % % % % % % % % % % % % % % % % % % % % % % % % % % %
% % % % % % % % % % % % % % % % % % % % % % % % % % % % % % % % %
% % % % % % % % % % % % % % % % % % % % % % % % % % % % % % % % %
% % % % % % % % % % % % % % % % % % % % % % % % % % % % % % % % %
% % % % % % % % % % % % % % % % % % % % % % % % % % % % % % % % %
% % % % % % % % % % % % % % % % % % % % % % % % % % % % % % % % %
% % % % % % % % % % % % % % % % % % % % % % % % % % % % % % % % %
% % % % % % % % % % % % % % % % % % % % % % % % % % % % % %

\chapter*{Chapter I} \ifaudio \marginpar{
\href{http://ia800204.us.archive.org/15/items/war_and_peace_13_0912_librivox/war_and_peace_13_01_tolstoy_64kb.mp3}{Audio}}
\fi

\initial{M}{an}'s mind cannot grasp the causes of events in their
completeness, but the desire to find those causes is implanted in
man's soul. And without considering the multiplicity and
complexity of the conditions any one of which taken separately
may seem to be the cause, he snatches at the first approximation
to a cause that seems to him intelligible and says: ``This is the
cause!'' In historical events (where the actions of men are the
subject of observation) the first and most primitive
approximation to present itself was the will of the gods and,
after that, the will of those who stood in the most prominent
position---the heroes of history.  But we need only penetrate to
the essence of any historic event---which lies in the activity of
the general mass of men who take part in it---to be convinced
that the will of the historic hero does not control the actions
of the mass but is itself continually controlled. It may seem to
be a matter of indifference whether we understand the meaning of
historical events this way or that; yet there is the same
difference between a man who says that the people of the West
moved on the East because Napoleon wished it and a man who says
that this happened because it had to happen, as there is between
those who declared that the earth was stationary and that the
planets moved round it and those who admitted that they did not
know what upheld the earth, but knew there were laws directing
its movement and that of the other planets. There is, and can be,
no cause of an historical event except the one cause of all
causes. But there are laws directing events, and some of these
laws are known to us while we are conscious of others we cannot
comprehend.  The discovery of these laws is only possible when we
have quite abandoned the attempt to find the cause in the will of
some one man, just as the discovery of the laws of the motion of
the planets was possible only when men abandoned the conception
of the fixity of the earth.

The historians consider that, next to the battle of Borodino and
the occupation of Moscow by the enemy and its destruction by
fire, the most important episode of the war of 1812 was the
movement of the Russian army from the Ryazana to the Kaluga road
and to the Tarutino camp---the so-called flank march across the
Krasnaya Pakhra River. They ascribe the glory of that achievement
of genius to different men and dispute as to whom the honor is
due. Even foreign historians, including the French, acknowledge
the genius of the Russian commanders when they speak of that
flank march. But it is hard to understand why military writers,
and following them others, consider this flank march to be the
profound conception of some one man who saved Russia and
destroyed Napoleon. In the first place it is hard to understand
where the profundity and genius of this movement lay, for not
much mental effort was needed to see that the best position for
an army when it is not being attacked is where there are most
provisions; and even a dull boy of thirteen could have guessed
that the best position for an army after its retreat from Moscow
in 1812 was on the Kaluga road. So it is impossible to understand
by what reasoning the historians reach the conclusion that this
maneuver was a profound one. And it is even more difficult to
understand just why they think that this maneuver was calculated
to save Russia and destroy the French; for this flank march, had
it been preceded, accompanied, or followed by other
circumstances, might have proved ruinous to the Russians and
salutary for the French. If the position of the Russian army
really began to improve from the time of that march, it does not
at all follow that the march was the cause of it.

That flank march might not only have failed to give any advantage
to the Russian army, but might in other circumstances have led to
its destruction. What would have happened had Moscow not burned
down? If Murat had not lost sight of the Russians? If Napoleon
had not remained inactive? If the Russian army at Krasnaya Pakhra
had given battle as Bennigsen and Barclay advised? What would
have happened had the French attacked the Russians while they
were marching beyond the Pakhra? What would have happened if on
approaching Tarutino, Napoleon had attacked the Russians with but
a tenth of the energy he had shown when he attacked them at
Smolensk? What would have happened had the French moved on
Petersburg?... In any of these eventualities the flank march that
brought salvation might have proved disastrous.

The third and most incomprehensible thing is that people studying
history deliberately avoid seeing that this flank march cannot be
attributed to any one man, that no one ever foresaw it, and that
in reality, like the retreat from Fili, it did not suggest itself
to anyone in its entirety, but resulted---moment by moment, step
by step, event by event---from an endless number of most diverse
circumstances and was only seen in its entirety when it had been
accomplished and belonged to the past.

At the council at Fili the prevailing thought in the minds of the
Russian commanders was the one naturally suggesting itself,
namely, a direct retreat by the Nizhni road. In proof of this
there is the fact that the majority of the council voted for such
a retreat, and above all there is the well-known conversation
after the council, between the commander in chief and Lanskoy,
who was in charge of the commissariat department. Lanskoy
informed the commander-in-chief that the army supplies were for
the most part stored along the Oka in the Tula and Ryazan
provinces, and that if they retreated on Nizhni the army would be
separated from its supplies by the broad river Oka, which cannot
be crossed early in winter. This was the first indication of the
necessity of deviating from what had previously seemed the most
natural course---a direct retreat on Nizhni-Novgorod. The army
turned more to the south, along the Ryazan road and nearer to its
supplies. Subsequently the inactivity of the French (who even
lost sight of the Russian army), concern for the safety of the
arsenal at Tula, and especially the advantages of drawing nearer
to its supplies caused the army to turn still further south to
the Tula road. Having crossed over, by a forced march, to the
Tula road beyond the Pakhra, the Russian commanders intended to
remain at Podolsk and had no thought of the Tarutino position;
but innumerable circumstances and the reappearance of French
troops who had for a time lost touch with the Russians, and
projects of giving battle, and above all the abundance of
provisions in Kaluga province, obliged our army to turn still
more to the south and to cross from the Tula to the Kaluga road
and go to Tarutino, which was between the roads along which those
supplies lay. Just as it is impossible to say when it was decided
to abandon Moscow, so it is impossible to say precisely when, or
by whom, it was decided to move to Tarutino. Only when the army
had got there, as the result of innumerable and varying forces,
did people begin to assure themselves that they had desired this
movement and long ago foreseen its result.

% % % % % % % % % % % % % % % % % % % % % % % % % % % % % % % % %
% % % % % % % % % % % % % % % % % % % % % % % % % % % % % % % % %
% % % % % % % % % % % % % % % % % % % % % % % % % % % % % % % % %
% % % % % % % % % % % % % % % % % % % % % % % % % % % % % % % % %
% % % % % % % % % % % % % % % % % % % % % % % % % % % % % % % % %
% % % % % % % % % % % % % % % % % % % % % % % % % % % % % % % % %
% % % % % % % % % % % % % % % % % % % % % % % % % % % % % % % % %
% % % % % % % % % % % % % % % % % % % % % % % % % % % % % % % % %
% % % % % % % % % % % % % % % % % % % % % % % % % % % % % % % % %
% % % % % % % % % % % % % % % % % % % % % % % % % % % % % % % % %
% % % % % % % % % % % % % % % % % % % % % % % % % % % % % % % % %
% % % % % % % % % % % % % % % % % % % % % % % % % % % % % %

\chapter*{Chapter II} \ifaudio \marginpar{
\href{http://ia800204.us.archive.org/15/items/war_and_peace_13_0912_librivox/war_and_peace_13_02_tolstoy_64kb.mp3}{Audio}}
\fi

\initial{T}{he} famous flank movement merely consisted in this: after the
advance of the French had ceased, the Russian army, which had
been continually retreating straight back from the invaders,
deviated from that direct course and, not finding itself pursued,
was naturally drawn toward the district where supplies were
abundant.

If instead of imagining to ourselves commanders of genius leading
the Russian army, we picture that army without any leaders, it
could not have done anything but make a return movement toward
Moscow, describing an arc in the direction where most provisions
were to be found and where the country was richest.

That movement from the Nizhni to the Ryazan, Tula, and Kaluga
roads was so natural that even the Russian marauders moved in
that direction, and demands were sent from Petersburg for Kutuzov
to take his army that way.  At Tarutino Kutuzov received what was
almost a reprimand from the Emperor for having moved his army
along the Ryazan road, and the Emperor's letter indicated to him
the very position he had already occupied near Kaluga.

Having rolled like a ball in the direction of the impetus given
by the whole campaign and by the battle of Borodino, the Russian
army---when the strength of that impetus was exhausted and no
fresh push was received---assumed the position natural to it.

Kutuzov's merit lay, not in any strategic maneuver of genius, as
it is called, but in the fact that he alone understood the
significance of what had happened. He alone then understood the
meaning of the French army's inactivity, he alone continued to
assert that the battle of Borodino had been a victory, he
alone---who as commander-in-chief might have been expected to be
eager to attack---employed his whole strength to restrain the
Russian army from useless engagements.

The beast wounded at Borodino was lying where the fleeing hunter
had left him; but whether he was still alive, whether he was
strong and merely lying low, the hunter did not know. Suddenly
the beast was heard to moan.

The moan of that wounded beast (the French army) which betrayed
its calamitous condition was the sending of Lauriston to
Kutuzov's camp with overtures for peace.

Napoleon, with his usual assurance that whatever entered his head
was right, wrote to Kutuzov the first words that occurred to him,
though they were meaningless.

\begin{quote}\calli Monsieur le Prince Koutouzov: I am sending
one of my adjutants-general to discuss several interesting
questions with you. I beg your Highness to credit what he says to
you, especially when he expresses the sentiment of esteem and
special regard I have long entertained for your person. This
letter having no other object, I pray God, monsieur le Prince
Koutouzov, to keep you in His holy and gracious protection!

  Napoleon Moscow, October 30, 1812 \end{quote}

Kutuzov replied: \begin{quote}\calli I should be cursed by
posterity were I looked on as the initiator of a settlement of
any sort. Such is the present spirit of my nation.  \end{quote}

But he continued to exert all his powers to restrain his troops
from attacking.

During the month that the French troops were pillaging in Moscow
and the Russian troops were quietly encamped at Tarutino, a
change had taken place in the relative strength of the two
armies---both in spirit and in number---as a result of which the
superiority had passed to the Russian side. Though the condition
and numbers of the French army were unknown to the Russians, as
soon as that change occurred the need of attacking at once showed
itself by countless signs. These signs were: Lauriston's mission;
the abundance of provisions at Tarutino; the reports coming in
from all sides of the inactivity and disorder of the French; the
flow of recruits to our regiments; the fine weather; the long
rest the Russian soldiers had enjoyed, and the impatience to do
what they had been assembled for, which usually shows itself in
an army that has been resting; curiosity as to what the French
army, so long lost sight of, was doing; the boldness with which
our outposts now scouted close up to the French stationed at
Tarutino; the news of easy successes gained by peasants and
guerrilla troops over the French, the envy aroused by this; the
desire for revenge that lay in the heart of every Russian as long
as the French were in Moscow, and (above all) a dim consciousness
in every soldier's mind that the relative strength of the armies
had changed and that the advantage was now on our side. There was
a substantial change in the relative strength, and an advance had
become inevitable. And at once, as a clock begins to strike and
chime as soon as the minute hand has completed a full circle,
this change was shown by an increased activity, whirring, and
chiming in the higher spheres.

% % % % % % % % % % % % % % % % % % % % % % % % % % % % % % % % %
% % % % % % % % % % % % % % % % % % % % % % % % % % % % % % % % %
% % % % % % % % % % % % % % % % % % % % % % % % % % % % % % % % %
% % % % % % % % % % % % % % % % % % % % % % % % % % % % % % % % %
% % % % % % % % % % % % % % % % % % % % % % % % % % % % % % % % %
% % % % % % % % % % % % % % % % % % % % % % % % % % % % % % % % %
% % % % % % % % % % % % % % % % % % % % % % % % % % % % % % % % %
% % % % % % % % % % % % % % % % % % % % % % % % % % % % % % % % %
% % % % % % % % % % % % % % % % % % % % % % % % % % % % % % % % %
% % % % % % % % % % % % % % % % % % % % % % % % % % % % % % % % %
% % % % % % % % % % % % % % % % % % % % % % % % % % % % % % % % %
% % % % % % % % % % % % % % % % % % % % % % % % % % % % % %

\chapter*{Chapter III} \ifaudio \marginpar{
\href{http://ia800204.us.archive.org/15/items/war_and_peace_13_0912_librivox/war_and_peace_13_03_tolstoy_64kb.mp3}{Audio}}
\fi

\initial{T}{he} Russian army was commanded by Kutuzov and his staff, and also
by the Emperor from Petersburg. Before the news of the
abandonment of Moscow had been received in Petersburg, a detailed
plan of the whole campaign had been drawn up and sent to Kutuzov
for his guidance. Though this plan had been drawn up on the
supposition that Moscow was still in our hands, it was approved
by the staff and accepted as a basis for action. Kutuzov only
replied that movements arranged from a distance were always
difficult to execute. So fresh instructions were sent for the
solution of difficulties that might be encountered, as well as
fresh people who were to watch Kutuzov's actions and report upon
them.

Besides this, the whole staff of the Russian army was now
reorganized.  The posts left vacant by Bagration, who had been
killed, and by Barclay, who had gone away in dudgeon, had to be
filled. Very serious consideration was given to the question
whether it would be better to put A in B's place and B in D's, or
on the contrary to put D in A's place, and so on---as if anything
more than A's or B's satisfaction depended on this.

As a result of the hostility between Kutuzov and Bennigsen, his
Chief of Staff, the presence of confidential representatives of
the Emperor, and these transfers, a more than usually complicated
play of parties was going on among the staff of the army. A was
undermining B, D was undermining C, and so on in all possible
combinations and permutations.  In all these plottings the
subject of intrigue was generally the conduct of the war, which
all these men believed they were directing; but this affair of
the war went on independently of them, as it had to go: that is,
never in the way people devised, but flowing always from the
essential attitude of the masses. Only in the highest spheres did
all these schemes, crossings, and interminglings appear to be a
true reflection of what had to happen.

Prince Michael Ilarionovich! (wrote the Emperor on the second of
October in a letter that reached Kutuzov after the battle at
Tarutino) Since September 2 Moscow has been in the hands of the
enemy. Your last reports were written on the twentieth, and
during all this time not only has no action been taken against
the enemy or for the relief of the ancient capital, but according
to your last report you have even retreated farther. Serpukhov is
already occupied by an enemy detachment and Tula with its famous
arsenal so indispensable to the army, is in danger. From General
Wintzingerode's reports, I see that an enemy corps of ten
thousand men is moving on the Petersburg road. Another corps of
several thousand men is moving on Dmitrov. A third has advanced
along the Vladimir road, and a fourth, rather considerable
detachment is stationed between Ruza and Mozhaysk. Napoleon
himself was in Moscow as late as the twenty-fifth. In view of all
this information, when the enemy has scattered his forces in
large detachments, and with Napoleon and his Guards in Moscow, is
it possible that the enemy's forces confronting you are so
considerable as not to allow of your taking the offensive? On the
contrary, he is probably pursuing you with detachments, or at
most with an army corps much weaker than the army entrusted to
you. It would seem that, availing yourself of these
circumstances, you might advantageously attack a weaker one and
annihilate him, or at least oblige him to retreat, retaining in
our hands an important part of the provinces now occupied by the
enemy, and thereby averting danger from Tula and other towns in
the interior. You will be responsible if the enemy is able to
direct a force of any size against Petersburg to threaten this
capital in which it has not been possible to retain many troops;
for with the army entrusted to you, and acting with resolution
and energy, you have ample means to avert this fresh
calamity. Remember that you have still to answer to our offended
country for the loss of Moscow. You have experienced my readiness
to reward you. That readiness will not weaken in me, but I and
Russia have a right to expect from you all the zeal, firmness,
and success which your intellect, military talent, and the
courage of the troops you command justify us in expecting.

But by the time this letter, which proved that the real relation
of the forces had already made itself felt in Petersburg, was
dispatched, Kutuzov had found himself unable any longer to
restrain the army he commanded from attacking and a battle had
taken place.

On the second of October a Cossack, Shapovalov, who was out
scouting, killed one hare and wounded another. Following the
wounded hare he made his way far into the forest and came upon
the left flank of Murat's army, encamped there without any
precautions. The Cossack laughingly told his comrades how he had
almost fallen into the hands of the French.  A cornet, hearing
the story, informed his commander.

The Cossack was sent for and questioned. The Cossack officers
wished to take advantage of this chance to capture some horses,
but one of the superior officers, who was acquainted with the
higher authorities, reported the incident to a general on the
staff. The state of things on the staff had of late been
exceedingly strained. Ermolov had been to see Bennigsen a few
days previously and had entreated him to use his influence with
the commander-in-chief to induce him to take the offensive.

``If I did not know you I should think you did not want what you
are asking for. I need only advise anything and his Highness is
sure to do the opposite,'' replied Bennigsen.

The Cossack's report, confirmed by horse patrols who were sent
out, was the final proof that events had matured. The tightly
coiled spring was released, the clock began to whirr and the
chimes to play. Despite all his supposed power, his intellect,
his experience, and his knowledge of men, Kutuzov---having taken
into consideration the Cossack's report, a note from Bennigsen
who sent personal reports to the Emperor, the wishes he supposed
the Emperor to hold, and the fact that all the generals expressed
the same wish---could no longer check the inevitable movement,
and gave the order to do what he regarded as useless and
harmful---gave his approval, that is, to the accomplished fact.

% % % % % % % % % % % % % % % % % % % % % % % % % % % % % % % % %
% % % % % % % % % % % % % % % % % % % % % % % % % % % % % % % % %
% % % % % % % % % % % % % % % % % % % % % % % % % % % % % % % % %
% % % % % % % % % % % % % % % % % % % % % % % % % % % % % % % % %
% % % % % % % % % % % % % % % % % % % % % % % % % % % % % % % % %
% % % % % % % % % % % % % % % % % % % % % % % % % % % % % % % % %
% % % % % % % % % % % % % % % % % % % % % % % % % % % % % % % % %
% % % % % % % % % % % % % % % % % % % % % % % % % % % % % % % % %
% % % % % % % % % % % % % % % % % % % % % % % % % % % % % % % % %
% % % % % % % % % % % % % % % % % % % % % % % % % % % % % % % % %
% % % % % % % % % % % % % % % % % % % % % % % % % % % % % % % % %
% % % % % % % % % % % % % % % % % % % % % % % % % % % % % %

\chapter*{Chapter IV} \ifaudio \marginpar{
\href{http://ia800204.us.archive.org/15/items/war_and_peace_13_0912_librivox/war_and_peace_13_04_tolstoy_64kb.mp3}{Audio}}
\fi

\initial{B}{ennigsen}'s note and the Cossack's information that the left
flank of the French was unguarded were merely final indications
that it was necessary to order an attack, and it was fixed for
the fifth of October.

On the morning of the fourth of October Kutuzov signed the
dispositions.  Toll read them to Ermolov, asking him to attend to
the further arrangements.

``All right---all right. I haven't time just now,'' replied
Ermolov, and left the hut.

The dispositions drawn up by Toll were very good. As in the
Austerlitz dispositions, it was written---though not in German
this time:

``The First Column will march here and here,'' ``the Second
Column will march there and there,'' and so on; and on paper, all
these columns arrived at their places at the appointed time and
destroyed the enemy.  Everything had been admirably thought out
as is usual in dispositions, and as is always the case, not a
single column reached its place at the appointed time.

When the necessary number of copies of the dispositions had been
prepared, an officer was summoned and sent to deliver them to
Ermolov to deal with. A young officer of the Horse Guards,
Kutuzov's orderly, pleased at the importance of the mission
entrusted to him, went to Ermolov's quarters.

``Gone away,'' said Ermolov's orderly.

The officer of the Horse Guards went to a general with whom
Ermolov was often to be found.

``No, and the general's out too.''

The officer, mounting his horse, rode off to someone else.

``No, he's gone out.''

``If only they don't make me responsible for this delay! What a
nuisance it is!'' thought the officer, and he rode round the
whole camp. One man said he had seen Ermolov ride past with some
other generals, others said he must have returned home. The
officer searched till six o'clock in the evening without even
stopping to eat. Ermolov was nowhere to be found and no one knew
where he was. The officer snatched a little food at a comrade's,
and rode again to the vanguard to find Miloradovich.
Miloradovich too was away, but here he was told that he had gone
to a ball at General Kikin's and that Ermolov was probably there
too.

``But where is it?''

``Why, there, over at Echkino,'' said a Cossack officer, pointing
to a country house in the far distance.

``What, outside our line?''

``They've put two regiments as outposts, and they're having such
a spree there, it's awful! Two bands and three sets of singers!''

The officer rode out beyond our lines to Echkino. While still at
a distance he heard as he rode the merry sounds of a soldier's
dance song proceeding from the house.

``In the meadows... in the meadows!'' he heard, accompanied by
whistling and the sound of a torban, drowned every now and then
by shouts. These sounds made his spirits rise, but at the same
time he was afraid that he would be blamed for not having
executed sooner the important order entrusted to him. It was
already past eight o'clock. He dismounted and went up into the
porch of a large country house which had remained intact between
the Russian and French forces. In the refreshment room and the
hall, footmen were bustling about with wine and viands. Groups of
singers stood outside the windows. The officer was admitted and
immediately saw all the chief generals of the army together, and
among them Ermolov's big imposing figure. They all had their
coats unbuttoned and were standing in a semicircle with flushed
and animated faces, laughing loudly. In the middle of the room a
short handsome general with a red face was dancing the trepak
with much spirit and agility.

``Ha, ha, ha! Bravo, Nicholas Ivanych! Ha, ha, ha!''

The officer felt that by arriving with important orders at such a
moment he was doubly to blame, and he would have preferred to
wait; but one of the generals espied him and, hearing what he had
come about, informed Ermolov.

Ermolov came forward with a frown on his face and, hearing what
the officer had to say, took the papers from him without a word.

``You think he went off just by chance?'' said a comrade, who was
on the staff that evening, to the officer of the Horse Guards,
referring to Ermolov. ``It was a trick. It was done on purpose to
get Konovnitsyn into trouble. You'll see what a mess there'll be
tomorrow.''

% % % % % % % % % % % % % % % % % % % % % % % % % % % % % % % % %
% % % % % % % % % % % % % % % % % % % % % % % % % % % % % % % % %
% % % % % % % % % % % % % % % % % % % % % % % % % % % % % % % % %
% % % % % % % % % % % % % % % % % % % % % % % % % % % % % % % % %
% % % % % % % % % % % % % % % % % % % % % % % % % % % % % % % % %
% % % % % % % % % % % % % % % % % % % % % % % % % % % % % % % % %
% % % % % % % % % % % % % % % % % % % % % % % % % % % % % % % % %
% % % % % % % % % % % % % % % % % % % % % % % % % % % % % % % % %
% % % % % % % % % % % % % % % % % % % % % % % % % % % % % % % % %
% % % % % % % % % % % % % % % % % % % % % % % % % % % % % % % % %
% % % % % % % % % % % % % % % % % % % % % % % % % % % % % % % % %
% % % % % % % % % % % % % % % % % % % % % % % % % % % % % %

\chapter*{Chapter V} \ifaudio \marginpar{
\href{http://ia800204.us.archive.org/15/items/war_and_peace_13_0912_librivox/war_and_peace_13_05_tolstoy_64kb.mp3}{Audio}}
\fi

\initial{N}{ext} day the decrepit Kutuzov, having given orders to be called
early, said his prayers, dressed, and, with an unpleasant
consciousness of having to direct a battle he did not approve of,
got into his caleche and drove from Letashovka (a village three
and a half miles from Tarutino) to the place where the attacking
columns were to meet. He sat in the caleche, dozing and waking up
by turns, and listening for any sound of firing on the right as
an indication that the action had begun.  But all was still
quiet. A damp dull autumn morning was just dawning. On
approaching Tarutino Kutuzov noticed cavalrymen leading their
horses to water across the road along which he was
driving. Kutuzov looked at them searchingly, stopped his
carriage, and inquired what regiment they belonged to. They
belonged to a column that should have been far in front and in
ambush long before then. ``It may be a mistake,'' thought the old
commander-in-chief. But a little further on he saw infantry
regiments with their arms piled and the soldiers, only partly
dressed, eating their rye porridge and carrying fuel. He sent for
an officer. The officer reported that no order to advance had
been received.

``How! Not rec...'' Kutuzov began, but checked himself
immediately and sent for a senior officer. Getting out of his
caleche, he waited with drooping head and breathing heavily,
pacing silently up and down. When Eykhen, the officer of the
general staff whom he had summoned, appeared, Kutuzov went purple
in the face, not because that officer was to blame for the
mistake, but because he was an object of sufficient importance
for him to vent his wrath on. Trembling and panting the old man
fell into that state of fury in which he sometimes used to roll
on the ground, and he fell upon Eykhen, threatening him with his
hands, shouting and loading him with gross abuse. Another man,
Captain Brozin, who happened to turn up and who was not at all to
blame, suffered the same fate.

``What sort of another blackguard are you? I'll have you shot!
Scoundrels!'' yelled Kutuzov in a hoarse voice, waving his arms
and reeling.

He was suffering physically. He, the commander-in-chief, a Serene
Highness who everybody said possessed powers such as no man had
ever had in Russia, to be placed in this position---made the
laughingstock of the whole army! ``I needn't have been in such a
hurry to pray about today, or have kept awake thinking everything
over all night,'' thought he to himself. ``When I was a chit of
an officer no one would have dared to mock me so... and now!'' He
was in a state of physical suffering as if from corporal
punishment, and could not avoid expressing it by cries of anger
and distress. But his strength soon began to fail him, and
looking about him, conscious of having said much that was amiss,
he again got into his caleche and drove back in silence.

His wrath, once expended, did not return, and blinking feebly he
listened to excuses and self-justifications (Ermolov did not come
to see him till the next day) and to the insistence of Bennigsen,
Konovnitsyn, and Toll that the movement that had miscarried
should be executed next day. And once more Kutuzov had to
consent.

% % % % % % % % % % % % % % % % % % % % % % % % % % % % % % % % %
% % % % % % % % % % % % % % % % % % % % % % % % % % % % % % % % %
% % % % % % % % % % % % % % % % % % % % % % % % % % % % % % % % %
% % % % % % % % % % % % % % % % % % % % % % % % % % % % % % % % %
% % % % % % % % % % % % % % % % % % % % % % % % % % % % % % % % %
% % % % % % % % % % % % % % % % % % % % % % % % % % % % % % % % %
% % % % % % % % % % % % % % % % % % % % % % % % % % % % % % % % %
% % % % % % % % % % % % % % % % % % % % % % % % % % % % % % % % %
% % % % % % % % % % % % % % % % % % % % % % % % % % % % % % % % %
% % % % % % % % % % % % % % % % % % % % % % % % % % % % % % % % %
% % % % % % % % % % % % % % % % % % % % % % % % % % % % % % % % %
% % % % % % % % % % % % % % % % % % % % % % % % % % % % % %

\chapter*{Chapter VI} \ifaudio \marginpar{
\href{http://ia800204.us.archive.org/15/items/war_and_peace_13_0912_librivox/war_and_peace_13_06_tolstoy_64kb.mp3}{Audio}}
\fi

\initial{N}{ext} day the troops assembled in their appointed places in the
evening and advanced during the night. It was an autumn night
with dark purple clouds, but no rain. The ground was damp but not
muddy, and the troops advanced noiselessly, only occasionally a
jingling of the artillery could be faintly heard. The men were
forbidden to talk out loud, to smoke their pipes, or to strike a
light, and they tried to prevent their horses neighing. The
secrecy of the undertaking heightened its charm and they marched
gaily. Some columns, supposing they had reached their
destination, halted, piled arms, and settled down on the cold
ground, but the majority marched all night and arrived at places
where they evidently should not have been.

Only Count Orlov-Denisov with his Cossacks (the least important
detachment of all) got to his appointed place at the right
time. This detachment halted at the outskirts of a forest, on the
path leading from the village of Stromilova to Dmitrovsk.

Toward dawn, Count Orlov-Denisov, who had dozed off, was awakened
by a deserter from the French army being brought to him. This was
a Polish sergeant of Poniatowski's corps, who explained in Polish
that he had come over because he had been slighted in the
service: that he ought long ago to have been made an officer,
that he was braver than any of them, and so he had left them and
wished to pay them out. He said that Murat was spending the night
less than a mile from where they were, and that if they would let
him have a convoy of a hundred men he would capture him
alive. Count Orlov-Denisov consulted his fellow officers.

The offer was too tempting to be refused. Everyone volunteered to
go and everybody advised making the attempt. After much disputing
and arguing, Major-General Grekov with two Cossack regiments
decided to go with the Polish sergeant.

``Now, remember,'' said Count Orlov-Denisov to the sergeant at
parting, ``if you have been lying I'll have you hanged like a
dog; but if it's true you shall have a hundred gold pieces!''

Without replying, the sergeant, with a resolute air, mounted and
rode away with Grekov whose men had quickly assembled. They
disappeared into the forest, and Count Orlov-Denisov, having seen
Grekov off, returned, shivering from the freshness of the early
dawn and excited by what he had undertaken on his own
responsibility, and began looking at the enemy camp, now just
visible in the deceptive light of dawn and the dying
campfires. Our columns ought to have begun to appear on an open
declivity to his right. He looked in that direction, but though
the columns would have been visible quite far off, they were not
to be seen.  It seemed to the count that things were beginning to
stir in the French camp, and his keen-sighted adjutant confirmed
this.

``Oh, it is really too late,'' said Count Orlov, looking at the
camp.

As often happens when someone we have trusted is no longer before
our eyes, it suddenly seemed quite clear and obvious to him that
the sergeant was an impostor, that he had lied, and that the
whole Russian attack would be ruined by the absence of those two
regiments, which he would lead away heaven only knew where. How
could one capture a commander-in-chief from among such a mass of
troops!

``I am sure that rascal was lying,'' said the count.

``They can still be called back,'' said one of his suite, who
like Count Orlov felt distrustful of the adventure when he looked
at the enemy's camp.

``Eh? Really... what do you think? Should we let them go on or
not?''

``Will you have them fetched back?''

``Fetch them back, fetch them back!'' said Count Orlov with
sudden determination, looking at his watch. ``It will be too
late. It is quite light.''

And the adjutant galloped through the forest after Grekov. When
Grekov returned, Count Orlov-Denisov, excited both by the
abandoned attempt and by vainly awaiting the infantry columns
that still did not appear, as well as by the proximity of the
enemy, resolved to advance. All his men felt the same excitement.

``Mount!'' he commanded in a whisper. The men took their places
and crossed themselves... ``Forward, with God's aid!''

``Hurrah-ah-ah!'' reverberated in the forest, and the Cossack
companies, trailing their lances and advancing one after another
as if poured out of a sack, dashed gaily across the brook toward
the camp.

One desperate, frightened yell from the first French soldier who
saw the Cossacks, and all who were in the camp, undressed and
only just waking up, ran off in all directions, abandoning
cannons, muskets, and horses.

Had the Cossacks pursued the French, without heeding what was
behind and around them, they would have captured Murat and
everything there. That was what the officers desired. But it was
impossible to make the Cossacks budge when once they had got
booty and prisoners. None of them listened to orders. Fifteen
hundred prisoners and thirty-eight guns were taken on the spot,
besides standards and (what seemed most important to the
Cossacks) horses, saddles, horsecloths, and the like. All this
had to be dealt with, the prisoners and guns secured, the booty
divided---not without some shouting and even a little fighting
among themselves---and it was on this that the Cossacks all
busied themselves.

The French, not being farther pursued, began to recover
themselves: they formed into detachments and began
firing. Orlov-Denisov, still waiting for the other columns to
arrive, advanced no further.

Meantime, according to the dispositions which said that ``the
First Column will march'' and so on, the infantry of the belated
columns, commanded by Bennigsen and directed by Toll, had started
in due order and, as always happens, had got somewhere, but not
to their appointed places. As always happens the men, starting
cheerfully, began to halt; murmurs were heard, there was a sense
of confusion, and finally a backward movement. Adjutants and
generals galloped about, shouted, grew angry, quarreled, said
they had come quite wrong and were late, gave vent to a little
abuse, and at last gave it all up and went forward, simply to get
somewhere. ``We shall get somewhere or other!'' And they did
indeed get somewhere, though not to their right places; a few
eventually even got to their right place, but too late to be of
any use and only in time to be fired at. Toll, who in this battle
played the part of Weyrother at Austerlitz, galloped assiduously
from place to place, finding everything upside down
everywhere. Thus he stumbled on Bagovut's corps in a wood when it
was already broad daylight, though the corps should long before
have joined Orlov-Denisov. Excited and vexed by the failure and
supposing that someone must be responsible for it, Toll galloped
up to the commander of the corps and began upbraiding him
severely, saying that he ought to be shot. General Bagovut, a
fighting old soldier of placid temperament, being also upset by
all the delay, confusion, and cross-purposes, fell into a rage to
everybody's surprise and quite contrary to his usual character
and said disagreeable things to Toll.

``I prefer not to take lessons from anyone, but I can die with my
men as well as anybody,'' he said, and advanced with a single
division.

Coming out onto a field under the enemy's fire, this brave
general went straight ahead, leading his men under fire, without
considering in his agitation whether going into action now, with
a single division, would be of any use or no. Danger, cannon
balls, and bullets were just what he needed in his angry
mood. One of the first bullets killed him, and other bullets
killed many of his men. And his division remained under fire for
some time quite uselessly.

% % % % % % % % % % % % % % % % % % % % % % % % % % % % % % % % %
% % % % % % % % % % % % % % % % % % % % % % % % % % % % % % % % %
% % % % % % % % % % % % % % % % % % % % % % % % % % % % % % % % %
% % % % % % % % % % % % % % % % % % % % % % % % % % % % % % % % %
% % % % % % % % % % % % % % % % % % % % % % % % % % % % % % % % %
% % % % % % % % % % % % % % % % % % % % % % % % % % % % % % % % %
% % % % % % % % % % % % % % % % % % % % % % % % % % % % % % % % %
% % % % % % % % % % % % % % % % % % % % % % % % % % % % % % % % %
% % % % % % % % % % % % % % % % % % % % % % % % % % % % % % % % %
% % % % % % % % % % % % % % % % % % % % % % % % % % % % % % % % %
% % % % % % % % % % % % % % % % % % % % % % % % % % % % % % % % %
% % % % % % % % % % % % % % % % % % % % % % % % % % % % % %

\chapter*{Chapter VII} \ifaudio \marginpar{
\href{http://ia800204.us.archive.org/15/items/war_and_peace_13_0912_librivox/war_and_peace_13_07_tolstoy_64kb.mp3}{Audio}}
\fi

\initial{M}{eanwhile} another column was to have attacked the French from the
front, but Kutuzov accompanied that column. He well knew that
nothing but confusion would come of this battle undertaken
against his will, and as far as was in his power held the troops
back. He did not advance.

He rode silently on his small gray horse, indolently answering
suggestions that they should attack.

``The word attack is always on your tongue, but you don't see
that we are unable to execute complicated maneuvers,'' said he to
Miloradovich who asked permission to advance.

``We couldn't take Murat prisoner this morning or get to the
place in time, and nothing can be done now!'' he replied to
someone else.

When Kutuzov was informed that at the French rear---where
according to the reports of the Cossacks there had previously
been nobody---there were now two battalions of Poles, he gave a
sidelong glance at Ermolov who was behind him and to whom he had
not spoken since the previous day.

``You see! They are asking to attack and making plans of all
kinds, but as soon as one gets to business nothing is ready, and
the enemy, forewarned, takes measures accordingly.''

Ermolov screwed up his eyes and smiled faintly on hearing these
words.  He understood that for him the storm had blown over, and
that Kutuzov would content himself with that hint.

``He's having a little fun at my expense,'' said Ermolov softly,
nudging with his knee Raevski who was at his side.

Soon after this, Ermolov moved up to Kutuzov and respectfully
remarked:

``It is not too late yet, your Highness---the enemy has not gone
away---if you were to order an attack! If not, the Guards will
not so much as see a little smoke.''

Kutuzov did not reply, but when they reported to him that Murat's
troops were in retreat he ordered an advance, though at every
hundred paces he halted for three quarters of an hour.

The whole battle consisted in what Orlov-Denisov's Cossacks had
done: the rest of the army merely lost some hundreds of men
uselessly.

In consequence of this battle Kutuzov received a diamond
decoration, and Bennigsen some diamonds and a hundred thousand
rubles, others also received pleasant recognitions corresponding
to their various grades, and following the battle fresh changes
were made in the staff.

``That's how everything is done with us, all topsy-turvy!'' said
the Russian officers and generals after the Tarutino battle,
letting it be understood that some fool there is doing things all
wrong but that we ourselves should not have done so, just as
people speak today. But people who talk like that either do not
know what they are talking about or deliberately deceive
themselves. No battle---Tarutino, Borodino, or Austerlitz---takes
place as those who planned it anticipated. That is an essential
condition.

A countless number of free forces (for nowhere is man freer than
during a battle, where it is a question of life and death)
influence the course taken by the fight, and that course never
can be known in advance and never coincides with the direction of
any one force.

If many simultaneously and variously directed forces act on a
given body, the direction of its motion cannot coincide with any
one of those forces, but will always be a mean---what in
mechanics is represented by the diagonal of a parallelogram of
forces.

If in the descriptions given by historians, especially French
ones, we find their wars and battles carried out in accordance
with previously formed plans, the only conclusion to be drawn is
that those descriptions are false.

The battle of Tarutino obviously did not attain the aim Toll had
in view---to lead the troops into action in the order prescribed
by the dispositions; nor that which Count Orlov-Denisov may have
had in view---to take Murat prisoner; nor the result of
immediately destroying the whole corps, which Bennigsen and
others may have had in view; nor the aim of the officer who
wished to go into action to distinguish himself; nor that of the
Cossack who wanted more booty than he got, and so on.  But if the
aim of the battle was what actually resulted and what all the
Russians of that day desired---to drive the French out of Russia
and destroy their army---it is quite clear that the battle of
Tarutino, just because of its incongruities, was exactly what was
wanted at that stage of the campaign. It would be difficult and
even impossible to imagine any result more opportune than the
actual outcome of this battle. With a minimum of effort and
insignificant losses, despite the greatest confusion, the most
important results of the whole campaign were attained: the
transition from retreat to advance, an exposure of the weakness
of the French, and the administration of that shock which
Napoleon's army had only awaited to begin its flight.

% % % % % % % % % % % % % % % % % % % % % % % % % % % % % % % % %
% % % % % % % % % % % % % % % % % % % % % % % % % % % % % % % % %
% % % % % % % % % % % % % % % % % % % % % % % % % % % % % % % % %
% % % % % % % % % % % % % % % % % % % % % % % % % % % % % % % % %
% % % % % % % % % % % % % % % % % % % % % % % % % % % % % % % % %
% % % % % % % % % % % % % % % % % % % % % % % % % % % % % % % % %
% % % % % % % % % % % % % % % % % % % % % % % % % % % % % % % % %
% % % % % % % % % % % % % % % % % % % % % % % % % % % % % % % % %
% % % % % % % % % % % % % % % % % % % % % % % % % % % % % % % % %
% % % % % % % % % % % % % % % % % % % % % % % % % % % % % % % % %
% % % % % % % % % % % % % % % % % % % % % % % % % % % % % % % % %
% % % % % % % % % % % % % % % % % % % % % % % % % % % % % %

\chapter*{Chapter VIII} \ifaudio \marginpar{
\href{http://ia800204.us.archive.org/15/items/war_and_peace_13_0912_librivox/war_and_peace_13_08_tolstoy_64kb.mp3}{Audio}}
\fi

\initial{N}{apoleon} enters Moscow after the brilliant victory de la Moskowa;
there can be no doubt about the victory for the battlefield
remains in the hands of the French. The Russians retreat and
abandon their ancient capital. Moscow, abounding in provisions,
arms, munitions, and incalculable wealth, is in Napoleon's
hands. The Russian army, only half the strength of the French,
does not make a single attempt to attack for a whole
month. Napoleon's position is most brilliant. He can either fall
on the Russian army with double its strength and destroy it;
negotiate an advantageous peace, or in case of a refusal make a
menacing move on Petersburg, or even, in the case of a reverse,
return to Smolensk or Vilna; or remain in Moscow; in short, no
special genius would seem to be required to retain the brilliant
position the French held at that time.  For that, only very
simple and easy steps were necessary: not to allow the troops to
loot, to prepare winter clothing---of which there was sufficient
in Moscow for the whole army---and methodically to collect the
provisions, of which (according to the French historians) there
were enough in Moscow to supply the whole army for six
months. Yet Napoleon, that greatest of all geniuses, who the
historians declare had control of the army, took none of these
steps.

He not merely did nothing of the kind, but on the contrary he
used his power to select the most foolish and ruinous of all the
courses open to him. Of all that Napoleon might have done:
wintering in Moscow, advancing on Petersburg or on
Nizhni-Novgorod, or retiring by a more northerly or more
southerly route (say by the road Kutuzov afterwards took),
nothing more stupid or disastrous can be imagined than what he
actually did. He remained in Moscow till October, letting the
troops plunder the city; then, hesitating whether to leave a
garrison behind him, he quitted Moscow, approached Kutuzov
without joining battle, turned to the right and reached
Malo-Yaroslavets, again without attempting to break through and
take the road Kutuzov took, but retiring instead to Mozhaysk
along the devastated Smolensk road. Nothing more stupid than that
could have been devised, or more disastrous for the army, as the
sequel showed. Had Napoleon's aim been to destroy his army, the
most skillful strategist could hardly have devised any series of
actions that would so completely have accomplished that purpose,
independently of anything the Russian army might do.

Napoleon, the man of genius, did this! But to say that he
destroyed his army because he wished to, or because he was very
stupid, would be as unjust as to say that he had brought his
troops to Moscow because he wished to and because he was very
clever and a genius.

In both cases his personal activity, having no more force than
the personal activity of any soldier, merely coincided with the
laws that guided the event.

The historians quite falsely represent Napoleon's faculties as
having weakened in Moscow, and do so only because the results did
not justify his actions. He employed all his ability and strength
to do the best he could for himself and his army, as he had done
previously and as he did subsequently in 1813. His activity at
that time was no less astounding than it was in Egypt, in Italy,
in Austria, and in Prussia. We do not know for certain in how far
his genius was genuine in Egypt---where forty centuries looked
down upon his grandeur---for his great exploits there are all
told us by Frenchmen. We cannot accurately estimate his genius in
Austria or Prussia, for we have to draw our information from
French or German sources, and the incomprehensible surrender of
whole corps without fighting and of fortresses without a siege
must incline Germans to recognize his genius as the only
explanation of the war carried on in Germany. But we, thank God,
have no need to recognize his genius in order to hide our
shame. We have paid for the right to look at the matter plainly
and simply, and we will not abandon that right.

His activity in Moscow was as amazing and as full of genius as
elsewhere. Order after order and plan after plan were issued by
him from the time he entered Moscow till the time he left it. The
absence of citizens and of a deputation, and even the burning of
Moscow, did not disconcert him. He did not lose sight either of
the welfare of his army or of the doings of the enemy, or of the
welfare of the people of Russia, or of the direction of affairs
in Paris, or of diplomatic considerations concerning the terms of
the anticipated peace.

% % % % % % % % % % % % % % % % % % % % % % % % % % % % % % % % %
% % % % % % % % % % % % % % % % % % % % % % % % % % % % % % % % %
% % % % % % % % % % % % % % % % % % % % % % % % % % % % % % % % %
% % % % % % % % % % % % % % % % % % % % % % % % % % % % % % % % %
% % % % % % % % % % % % % % % % % % % % % % % % % % % % % % % % %
% % % % % % % % % % % % % % % % % % % % % % % % % % % % % % % % %
% % % % % % % % % % % % % % % % % % % % % % % % % % % % % % % % %
% % % % % % % % % % % % % % % % % % % % % % % % % % % % % % % % %
% % % % % % % % % % % % % % % % % % % % % % % % % % % % % % % % %
% % % % % % % % % % % % % % % % % % % % % % % % % % % % % % % % %
% % % % % % % % % % % % % % % % % % % % % % % % % % % % % % % % %
% % % % % % % % % % % % % % % % % % % % % % % % % % % % % %

\chapter*{Chapter IX} \ifaudio \marginpar{
\href{http://ia800204.us.archive.org/15/items/war_and_peace_13_0912_librivox/war_and_peace_13_09_tolstoy_64kb.mp3}{Audio}}
\fi

\initial{W}{ith} regard to military matters, Napoleon immediately on his
entry into Moscow gave General Sabastiani strict orders to
observe the movements of the Russian army, sent army corps out
along the different roads, and charged Murat to find
Kutuzov. Then he gave careful directions about the fortification
of the Kremlin, and drew up a brilliant plan for a future
campaign over the whole map of Russia.

With regard to diplomatic questions, Napoleon summoned Captain
Yakovlev, who had been robbed and was in rags and did not know
how to get out of Moscow, minutely explained to him his whole
policy and his magnanimity, and having written a letter to the
Emperor Alexander in which he considered it his duty to inform
his Friend and Brother that Rostopchin had managed affairs badly
in Moscow, he dispatched Yakovlev to Petersburg.

Having similarly explained his views and his magnanimity to
Tutolmin, he dispatched that old man also to Petersburg to
negotiate.

With regard to legal matters, immediately after the fires he gave
orders to find and execute the incendiaries. And the scoundrel
Rostopchin was punished by an order to burn down his houses.

With regard to administrative matters, Moscow was granted a
constitution. A municipality was established and the following
announcement issued:

\begin{quote} Inhabitants of Moscow!

Your misfortunes are cruel, but His Majesty the Emperor and King
desires to arrest their course. Terrible examples have taught you
how he punishes disobedience and crime. Strict measures have been
taken to put an end to disorder and to re-establish public
security. A paternal administration, chosen from among
yourselves, will form your municipality or city government. It
will take care of you, of your needs, and of your welfare. Its
members will be distinguished by a red ribbon worn across the
shoulder, and the mayor of the city will wear a white belt as
well. But when not on duty they will only wear a red ribbon round
the left arm.

The city police is established on its former footing, and better
order already prevails in consequence of its activity. The
government has appointed two commissaries general, or chiefs of
police, and twenty commissaries or captains of wards have been
appointed to the different wards of the city. You will recognize
them by the white ribbon they will wear on the left arm. Several
churches of different denominations are open, and divine service
is performed in them unhindered. Your fellow citizens are
returning every day to their homes and orders have been given
that they should find in them the help and protection due to
their misfortunes. These are the measures the government has
adopted to re-establish order and relieve your condition. But to
achieve this aim it is necessary that you should add your efforts
and should, if possible, forget the misfortunes you have
suffered, should entertain the hope of a less cruel fate, should
be certain that inevitable and ignominious death awaits those who
make any attempt on your persons or on what remains of your
property, and finally that you should not doubt that these will
be safeguarded, since such is the will of the greatest and most
just of monarchs. Soldiers and citizens, of whatever nation you
may be, re- establish public confidence, the source of the
welfare of a state, live like brothers, render mutual aid and
protection one to another, unite to defeat the intentions of the
evil-minded, obey the military and civil authorities, and your
tears will soon cease to flow!  \end{quote}

With regard to supplies for the army, Napoleon decreed that all
the troops in turn should enter Moscow \emph{a la
maraude}\footnote{As looters.} to obtain provisions for
themselves, so that the army might have its future provided for.

With regard to religion, Napoleon ordered the priests to be
brought back and services to be again performed in the churches.

With regard to commerce and to provisioning the army, the
following was placarded everywhere:

\begin{quote} Proclamation!

You, peaceful inhabitants of Moscow, artisans and workmen whom
misfortune has driven from the city, and you scattered tillers of
the soil, still kept out in the fields by groundless fear,
listen!  Tranquillity is returning to this capital and order is
being restored in it. Your fellow countrymen are emerging boldly
from their hiding places on finding that they are respected. Any
violence to them or to their property is promptly punished. His
Majesty the Emperor and King protects them, and considers no one
among you his enemy except those who disobey his orders. He
desires to end your misfortunes and restore you to your homes and
families. Respond, therefore, to his benevolent intentions and
come to us without fear. Inhabitants, return with confidence to
your abodes! You will soon find means of satisfying your
needs. Craftsmen and industrious artisans, return to your work,
your houses, your shops, where the protection of guards awaits
you! You shall receive proper pay for your work. And lastly you
too, peasants, come from the forests where you are hiding in
terror, return to your huts without fear, in full assurance that
you will find protection! Markets are established in the city
where peasants can bring their surplus supplies and the products
of the soil. The government has taken the following steps to
ensure freedom of sale for them: \begin{enumerate} \item From
today, peasants, husbandmen, and those living in the neighborhood
of Moscow may without any danger bring their supplies of all
kinds to two appointed markets, of which one is on the Mokhovaya
Street and the other at the Provision Market.  \item Such
supplies will be bought from them at such prices as seller and
buyer may agree on, and if a seller is unable to obtain a fair
price he will be free to take his goods back to his village and
no one may hinder him under any pretense.  \item Sunday and
Wednesday of each week are appointed as the chief market days and
to that end a sufficient number of troops will be stationed along
the highroads on Tuesdays and Saturdays at such distances from
the town as to protect the carts.  \item Similar measures will be
taken that peasants with their carts and horses may meet with no
hindrance on their return journey.  \item Steps will immediately
be taken to re-establish ordinary trading.  \end{enumerate}

Inhabitants of the city and villages, and you, workingmen and
artisans, to whatever nation you belong, you are called on to
carry out the paternal intentions of His Majesty the Emperor and
King and to co-operate with him for the public welfare! Lay your
respect and confidence at his feet and do not delay to unite with
us!  \end{quote}

With the object of raising the spirits of the troops and of the
people, reviews were constantly held and rewards distributed. The
Emperor rode through the streets to comfort the inhabitants, and,
despite his preoccupation with state affairs, himself visited the
theaters that were established by his order.

In regard to philanthropy, the greatest virtue of crowned heads,
Napoleon also did all in his power. He caused the words Maison de
ma Mere to be inscribed on the charitable institutions, thereby
combining tender filial affection with the majestic benevolence
of a monarch. He visited the Foundling Hospital and, allowing the
orphans saved by him to kiss his white hands, graciously
conversed with Tutolmin. Then, as Thiers eloquently recounts, he
ordered his soldiers to be paid in forged Russian money which he
had prepared: \emph{Raising the use of these means by an act
worthy of himself and of the French army, he let relief be
distributed to those who had been burned out. But as food was too
precious to be given to foreigners, who were for the most part
enemies, Napoleon preferred to supply them with money with which
to purchase food from outside, and had paper rubles distributed
to them.}

With reference to army discipline, orders were continually being
issued to inflict severe punishment for the nonperformance of
military duties and to suppress robbery.

% % % % % % % % % % % % % % % % % % % % % % % % % % % % % % % % %
% % % % % % % % % % % % % % % % % % % % % % % % % % % % % % % % %
% % % % % % % % % % % % % % % % % % % % % % % % % % % % % % % % %
% % % % % % % % % % % % % % % % % % % % % % % % % % % % % % % % %
% % % % % % % % % % % % % % % % % % % % % % % % % % % % % % % % %
% % % % % % % % % % % % % % % % % % % % % % % % % % % % % % % % %
% % % % % % % % % % % % % % % % % % % % % % % % % % % % % % % % %
% % % % % % % % % % % % % % % % % % % % % % % % % % % % % % % % %
% % % % % % % % % % % % % % % % % % % % % % % % % % % % % % % % %
% % % % % % % % % % % % % % % % % % % % % % % % % % % % % % % % %
% % % % % % % % % % % % % % % % % % % % % % % % % % % % % % % % %
% % % % % % % % % % % % % % % % % % % % % % % % % % % % % %

\chapter*{Chapter X} \ifaudio \marginpar{
\href{http://ia800204.us.archive.org/15/items/war_and_peace_13_0912_librivox/war_and_peace_13_10_tolstoy_64kb.mp3}{Audio}}
\fi

\initial{B}{ut} strange to say, all these measures, efforts, and
plans---which were not at all worse than others issued in similar
circumstances---did not affect the essence of the matter but,
like the hands of a clock detached from the mechanism, swung
about in an arbitrary and aimless way without engaging the
cogwheels.

With reference to the military side---the plan of campaign---that
work of genius of which Thiers remarks that, \emph{His genius
never devised anything more profound, more skillful, or more
admirable}, and enters into a polemic with M. Fain to prove that
this work of genius must be referred not to the fourth but to the
fifteenth of October---that plan never was or could be executed,
for it was quite out of touch with the facts of the case. The
fortifying of the Kremlin, for which la Mosquee (as Napoleon
termed the church of Basil the Beatified) was to have been razed
to the ground, proved quite useless. The mining of the Kremlin
only helped toward fulfilling Napoleon's wish that it should be
blown up when he left Moscow---as a child wants the floor on
which he has hurt himself to be beaten. The pursuit of the
Russian army, about which Napoleon was so concerned, produced an
unheard-of result. The French generals lost touch with the
Russian army of sixty thousand men, and according to Thiers it
was only eventually found, like a lost pin, by the skill---and
apparently the genius---of Murat.

With reference to diplomacy, all Napoleon's arguments as to his
magnanimity and justice, both to Tutolmin and to Yakovlev (whose
chief concern was to obtain a greatcoat and a conveyance), proved
useless; Alexander did not receive these envoys and did not reply
to their embassage.

With regard to legal matters, after the execution of the supposed
incendiaries the rest of Moscow burned down.

With regard to administrative matters, the establishment of a
municipality did not stop the robberies and was only of use to
certain people who formed part of that municipality and under
pretext of preserving order looted Moscow or saved their own
property from being looted.

With regard to religion, as to which in Egypt matters had so
easily been settled by Napoleon's visit to a mosque, no results
were achieved. Two or three priests who were found in Moscow did
try to carry out Napoleon's wish, but one of them was slapped in
the face by a French soldier while conducting service, and a
French official reported of another that: \emph{The priest whom I
found and invited to say Mass cleaned and locked up the
church. That night the doors were again broken open, the padlocks
smashed, the books mutilated, and other disorders perpetrated.}

With reference to commerce, the proclamation to industrious
workmen and to peasants evoked no response. There were no
industrious workmen, and the peasants caught the commissaries who
ventured too far out of town with the proclamation and killed
them.

As to the theaters for the entertainment of the people and the
troops, these did not meet with success either. The theaters set
up in the Kremlin and in Posnyakov's house were closed again at
once because the actors and actresses were robbed.

Even philanthropy did not have the desired effect. The genuine as
well as the false paper money which flooded Moscow lost its
value. The French, collecting booty, cared only for gold. Not
only was the paper money valueless which Napoleon so graciously
distributed to the unfortunate, but even silver lost its value in
relation to gold.

But the most amazing example of the ineffectiveness of the orders
given by the authorities at that time was Napoleon's attempt to
stop the looting and re-establish discipline.

This is what the army authorities were reporting:

\begin{quote} \calli Looting continues in the city despite the
decrees against it. Order is not yet restored and not a single
merchant is carrying on trade in a lawful manner. The sutlers
alone venture to trade, and they sell stolen goods.

  The neighborhood of my ward continues to be pillaged by
soldiers of the 3rd Corps who, not satisfied with taking from the
unfortunate inhabitants hiding in the cellars the little they
have left, even have the ferocity to wound them with their
sabers, as I have repeatedly witnessed.

  Nothing new, except that the soldiers are robbing and
pillaging---October 9.

  Robbery and pillaging continue. There is a band of thieves in
our district who ought to be arrested by a strong force---October
11.

  The Emperor is extremely displeased that despite the strict
orders to stop pillage, parties of marauding Guards are
continually seen returning to the Kremlin. Among the Old Guard
disorder and pillage were renewed more violently than ever
yesterday evening, last night, and today. The Emperor sees with
regret that the picked soldiers appointed to guard his person,
who should set an example of discipline, carry disobedience to
such a point that they break into the cellars and stores
containing army supplies. Others have disgraced themselves to the
extent of disobeying sentinels and officers, and have abused and
beaten them.  \end{quote}

\emph{The Grand Marshal of the palace,} wrote the
governor, \emph{complains bitterly that in spite of
repeated orders, the soldiers continue to commit nuisances in all
the courtyards and even under the very windows of the Emperor.}

That army, like a herd of cattle run wild and trampling underfoot
the provender which might have saved it from starvation,
disintegrated and perished with each additional day it remained
in Moscow. But it did not go away.

It began to run away only when suddenly seized by a panic caused
by the capture of transport trains on the Smolensk road, and by
the battle of Tarutino. The news of that battle of Tarutino,
unexpectedly received by Napoleon at a review, evoked in him a
desire to punish the Russians (Thiers says), and he issued the
order for departure which the whole army was demanding.

Fleeing from Moscow the soldiers took with them everything they
had stolen. Napoleon, too, carried away his own personal tresor,
but on seeing the baggage trains that impeded the army, he was
(Thiers says) horror-struck. And yet with his experience of war
he did not order all the superfluous vehicles to be burned, as he
had done with those of a certain marshal when approaching
Moscow. He gazed at the caleches and carriages in which soldiers
were riding and remarked that it was a very good thing, as those
vehicles could be used to carry provisions, the sick, and the
wounded.

The plight of the whole army resembled that of a wounded animal
which feels it is perishing and does not know what it is
doing. To study the skillful tactics and aims of Napoleon and his
army from the time it entered Moscow till it was destroyed is
like studying the dying leaps and shudders of a mortally wounded
animal. Very often a wounded animal, hearing a rustle, rushes
straight at the hunter's gun, runs forward and back again, and
hastens its own end. Napoleon, under pressure from his whole
army, did the same thing. The rustle of the battle of Tarutino
frightened the beast, and it rushed forward onto the hunter's
gun, reached him, turned back, and finally---like any wild
beast---ran back along the most disadvantageous and dangerous
path, where the old scent was familiar.

During the whole of that period Napoleon, who seems to us to have
been the leader of all these movements---as the figurehead of a
ship may seem to a savage to guide the vessel---acted like a
child who, holding a couple of strings inside a carriage, thinks
he is driving it.

% % % % % % % % % % % % % % % % % % % % % % % % % % % % % % % % %
% % % % % % % % % % % % % % % % % % % % % % % % % % % % % % % % %
% % % % % % % % % % % % % % % % % % % % % % % % % % % % % % % % %
% % % % % % % % % % % % % % % % % % % % % % % % % % % % % % % % %
% % % % % % % % % % % % % % % % % % % % % % % % % % % % % % % % %
% % % % % % % % % % % % % % % % % % % % % % % % % % % % % % % % %
% % % % % % % % % % % % % % % % % % % % % % % % % % % % % % % % %
% % % % % % % % % % % % % % % % % % % % % % % % % % % % % % % % %
% % % % % % % % % % % % % % % % % % % % % % % % % % % % % % % % %
% % % % % % % % % % % % % % % % % % % % % % % % % % % % % % % % %
% % % % % % % % % % % % % % % % % % % % % % % % % % % % % % % % %
% % % % % % % % % % % % % % % % % % % % % % % % % % % % % %

\chapter*{Chapter XI} \ifaudio \marginpar{
\href{http://ia800204.us.archive.org/15/items/war_and_peace_13_0912_librivox/war_and_peace_13_11_tolstoy_64kb.mp3}{Audio}}
\fi 

\initial{E}{arly} in the morning of the sixth of October Pierre went out of
the shed, and on returning stopped by the door to play with a
little blue-gray dog, with a long body and short bandy legs, that
jumped about him.  This little dog lived in their shed, sleeping
beside Karataev at night; it sometimes made excursions into the
town but always returned again.  Probably it had never had an
owner, and it still belonged to nobody and had no name. The
French called it Azor; the soldier who told stories called it
Femgalka; Karataev and others called it Gray, or sometimes
Flabby. Its lack of a master, a name, or even of a breed or any
definite color did not seem to trouble the blue-gray dog in the
least. Its furry tail stood up firm and round as a plume, its
bandy legs served it so well that it would often gracefully lift
a hind leg and run very easily and quickly on three legs, as if
disdaining to use all four. Everything pleased it. Now it would
roll on its back, yelping with delight, now bask in the sun with
a thoughtful air of importance, and now frolic about playing with
a chip of wood or a straw.

Pierre's attire by now consisted of a dirty torn shirt (the only
remnant of his former clothing), a pair of soldier's trousers
which by Karataev's advice he tied with string round the ankles
for warmth, and a peasant coat and cap. Physically he had changed
much during this time.  He no longer seemed stout, though he
still had the appearance of solidity and strength hereditary in
his family. A beard and mustache covered the lower part of his
face, and a tangle of hair, infested with lice, curled round his
head like a cap. The look of his eyes was resolute, calm, and
animatedly alert, as never before. The former slackness which had
shown itself even in his eyes was now replaced by an energetic
readiness for action and resistance. His feet were bare.

Pierre first looked down the field across which vehicles and
horsemen were passing that morning, then into the distance across
the river, then at the dog who was pretending to be in earnest
about biting him, and then at his bare feet which he placed with
pleasure in various positions, moving his dirty thick big
toes. Every time he looked at his bare feet a smile of animated
self-satisfaction flitted across his face.  The sight of them
reminded him of all he had experienced and learned during these
weeks and this recollection was pleasant to him.

For some days the weather had been calm and clear with slight
frosts in the mornings---what is called an \emph{old wives'
summer}.

In the sunshine the air was warm, and that warmth was
particularly pleasant with the invigorating freshness of the
morning frost still in the air.

On everything---far and near---lay the magic crystal glitter seen
only at that time of autumn. The Sparrow Hills were visible in
the distance, with the village, the church, and the large white
house. The bare trees, the sand, the bricks and roofs of the
houses, the green church spire, and the corners of the white
house in the distance, all stood out in the transparent air in
most delicate outline and with unnatural clearness.  Near by
could be seen the familiar ruins of a half-burned mansion
occupied by the French, with lilac bushes still showing dark
green beside the fence. And even that ruined and befouled
house---which in dull weather was repulsively ugly---seemed
quietly beautiful now, in the clear, motionless brilliance.

A French corporal, with coat unbuttoned in a homely way, a
skullcap on his head, and a short pipe in his mouth, came from
behind a corner of the shed and approached Pierre with a friendly
wink.

``What sunshine, Monsieur Kiril!'' (Their name for Pierre.) ``Eh?
Just like spring!''

And the corporal leaned against the door and offered Pierre his
pipe, though whenever he offered it Pierre always declined it.

``To be on the march in such weather...'' he began.

Pierre inquired what was being said about leaving, and the
corporal told him that nearly all the troops were starting and
there ought to be an order about the prisoners that day. Sokolov,
one of the soldiers in the shed with Pierre, was dying, and
Pierre told the corporal that something should be done about
him. The corporal replied that Pierre need not worry about that
as they had an ambulance and a permanent hospital and
arrangements would be made for the sick, and that in general
everything that could happen had been foreseen by the
authorities.

``Besides, Monsieur Kiril, you have only to say a word to the
captain, you know. He is a man who never forgets anything. Speak
to the captain when he makes his round, he will do anything for
you.''

(The captain of whom the corporal spoke often had long chats with
Pierre and showed him all sorts of favors.)

``'You see, St. Thomas,' he said to me the other day. 'Monsieur
Kiril is a man of education, who speaks French. He is a Russian
seigneur who has had misfortunes, but he is a man. He knows
what's what... If he wants anything and asks me, he won't get a
refusal. When one has studied, you see, one likes education and
well-bred people.' It is for your sake I mention it, Monsieur
Kiril. The other day if it had not been for you that affair would
have ended ill.''

And after chatting a while longer, the corporal went away. (The
affair he had alluded to had happened a few days before---a fight
between the prisoners and the French soldiers, in which Pierre
had succeeded in pacifying his comrades.) Some of the prisoners
who had heard Pierre talking to the corporal immediately asked
what the Frenchman had said.  While Pierre was repeating what he
had been told about the army leaving Moscow, a thin, sallow,
tattered French soldier came up to the door of the shed. Rapidly
and timidly raising his fingers to his forehead by way of
greeting, he asked Pierre whether the soldier Platoche to whom he
had given a shirt to sew was in that shed.

A week before the French had had boot leather and linen issued to
them, which they had given out to the prisoners to make up into
boots and shirts for them.

``Ready, ready, dear fellow!'' said Karataev, coming out with a
neatly folded shirt.

Karataev, on account of the warm weather and for convenience at
work, was wearing only trousers and a tattered shirt as black as
soot. His hair was bound round, workman fashion, with a wisp of
lime-tree bast, and his round face seemed rounder and pleasanter
than ever.

``A promise is own brother to performance! I said Friday and here
it is, ready,'' said Platon, smiling and unfolding the shirt he
had sewn.

The Frenchman glanced around uneasily and then, as if overcoming
his hesitation, rapidly threw off his uniform and put on the
shirt. He had a long, greasy, flowered silk waistcoat next to his
sallow, thin bare body, but no shirt. He was evidently afraid the
prisoners looking on would laugh at him, and thrust his head into
the shirt hurriedly. None of the prisoners said a word.

``See, it fits well!'' Platon kept repeating, pulling the shirt
straight.

The Frenchman, having pushed his head and hands through, without
raising his eyes, looked down at the shirt and examined the
seams.

``You see, dear man, this is not a sewing shop, and I had no
proper tools; and, as they say, one needs a tool even to kill a
louse,'' said Platon with one of his round smiles, obviously
pleased with his work.

``It's good, quite good, thank you,'' said the Frenchman, in
French, ``but there must be some linen left over.''

``It will fit better still when it sets to your body,'' said
Karataev, still admiring his handiwork. ``You'll be nice and
comfortable...''

``Thanks, thanks, old fellow... But the bits left over?'' said
the Frenchman again and smiled. He took out an assignation ruble
note and gave it to Karataev. ``But give me the pieces that are
over.''

Pierre saw that Platon did not want to understand what the
Frenchman was saying, and he looked on without
interfering. Karataev thanked the Frenchman for the money and
went on admiring his own work. The Frenchman insisted on having
the pieces returned that were left over and asked Pierre to
translate what he said.

``What does he want the bits for?'' said Karataev. ``They'd make
fine leg bands for us. Well, never mind.''

And Karataev, with a suddenly changed and saddened expression,
took a small bundle of scraps from inside his shirt and gave it
to the Frenchman without looking at him. ``Oh dear!'' muttered
Karataev and went away. The Frenchman looked at the linen,
considered for a moment, then looked inquiringly at Pierre and,
as if Pierre's look had told him something, suddenly blushed and
shouted in a squeaky voice:

``Platoche! Eh, Platoche! Keep them yourself!'' And handing back
the odd bits he turned and went out.

``There, look at that,'' said Karataev, swaying his
head. ``People said they were not Christians, but they too have
souls. It's what the old folk used to say: 'A sweating hand's an
open hand, a dry hand's close.'  He's naked, but yet he's given
it back.''

Karataev smiled thoughtfully and was silent awhile looking at the
pieces.

``But they'll make grand leg bands, dear friend,'' he said, and
went back into the shed.

% % % % % % % % % % % % % % % % % % % % % % % % % % % % % % % % %
% % % % % % % % % % % % % % % % % % % % % % % % % % % % % % % % %
% % % % % % % % % % % % % % % % % % % % % % % % % % % % % % % % %
% % % % % % % % % % % % % % % % % % % % % % % % % % % % % % % % %
% % % % % % % % % % % % % % % % % % % % % % % % % % % % % % % % %
% % % % % % % % % % % % % % % % % % % % % % % % % % % % % % % % %
% % % % % % % % % % % % % % % % % % % % % % % % % % % % % % % % %
% % % % % % % % % % % % % % % % % % % % % % % % % % % % % % % % %
% % % % % % % % % % % % % % % % % % % % % % % % % % % % % % % % %
% % % % % % % % % % % % % % % % % % % % % % % % % % % % % % % % %
% % % % % % % % % % % % % % % % % % % % % % % % % % % % % % % % %
% % % % % % % % % % % % % % % % % % % % % % % % % % % % % %

\chapter*{Chapter XII} \ifaudio \marginpar{
\href{http://ia800204.us.archive.org/15/items/war_and_peace_13_0912_librivox/war_and_peace_13_12_tolstoy_64kb.mp3}{Audio}}
\fi

\initial{F}{our} weeks had passed since Pierre had been taken prisoner and
though the French had offered to move him from the men's to the
officers' shed, he had stayed in the shed where he was first put.

In burned and devastated Moscow Pierre experienced almost the
extreme limits of privation a man can endure; but thanks to his
physical strength and health, of which he had till then been
unconscious, and thanks especially to the fact that the
privations came so gradually that it was impossible to say when
they began, he endured his position not only lightly but
joyfully. And just at this time he obtained the tranquillity and
ease of mind he had formerly striven in vain to reach.  He had
long sought in different ways that tranquillity of mind, that
inner harmony which had so impressed him in the soldiers at the
battle of Borodino. He had sought it in philanthropy, in
Freemasonry, in the dissipations of town life, in wine, in heroic
feats of self-sacrifice, and in romantic love for Natasha; he had
sought it by reasoning---and all these quests and experiments had
failed him. And now without thinking about it he had found that
peace and inner harmony only through the horror of death, through
privation, and through what he recognized in Karataev.

Those dreadful moments he had lived through at the executions had
as it were forever washed away from his imagination and memory
the agitating thoughts and feelings that had formerly seemed so
important. It did not now occur to him to think of Russia, or the
war, or politics, or Napoleon. It was plain to him that all these
things were no business of his, and that he was not called on to
judge concerning them and therefore could not do so. ``Russia and
summer weather are not bound together,'' he thought, repeating
words of Karataev's which he found strangely consoling. His
intention of killing Napoleon and his calculations of the
cabalistic number of the beast of the Apocalypse now seemed to
him meaningless and even ridiculous. His anger with his wife and
anxiety that his name should not be smirched now seemed not
merely trivial but even amusing. What concern was it of his that
somewhere or other that woman was leading the life she preferred?
What did it matter to anybody, and especially to him, whether or
not they found out that their prisoner's name was Count Bezukhov?

He now often remembered his conversation with Prince Andrew and
quite agreed with him, though he understood Prince Andrew's
thoughts somewhat differently. Prince Andrew had thought and said
that happiness could only be negative, but had said it with a
shade of bitterness and irony as though he was really saying that
all desire for positive happiness is implanted in us merely to
torment us and never be satisfied. But Pierre believed it without
any mental reservation. The absence of suffering, the
satisfaction of one's needs and consequent freedom in the choice
of one's occupation, that is, of one's way of life, now seemed to
Pierre to be indubitably man's highest happiness. Here and now
for the first time he fully appreciated the enjoyment of eating
when he wanted to eat, drinking when he wanted to drink, sleeping
when he wanted to sleep, of warmth when he was cold, of talking
to a fellow man when he wished to talk and to hear a human
voice. The satisfaction of one's needs---good food, cleanliness,
and freedom---now that he was deprived of all this, seemed to
Pierre to constitute perfect happiness; and the choice of
occupation, that is, of his way of life---now that that was so
restricted---seemed to him such an easy matter that he forgot
that a superfluity of the comforts of life destroys all joy in
satisfying one's needs, while great freedom in the choice of
occupation---such freedom as his wealth, his education, and his
social position had given him in his own life---is just what
makes the choice of occupation insolubly difficult and destroys
the desire and possibility of having an occupation.

All Pierre's daydreams now turned on the time when he would be
free. Yet subsequently, and for the rest of his life, he thought
and spoke with enthusiasm of that month of captivity, of those
irrecoverable, strong, joyful sensations, and chiefly of the
complete peace of mind and inner freedom which he experienced
only during those weeks.

When on the first day he got up early, went out of the shed at
dawn, and saw the cupolas and crosses of the New Convent of the
Virgin still dark at first, the hoarfrost on the dusty grass, the
Sparrow Hills, and the wooded banks above the winding river
vanishing in the purple distance, when he felt the contact of the
fresh air and heard the noise of the crows flying from Moscow
across the field, and when afterwards light gleamed from the east
and the sun's rim appeared solemnly from behind a cloud, and the
cupolas and crosses, the hoarfrost, the distance and the river,
all began to sparkle in the glad light---Pierre felt a new joy
and strength in life such as he had never before known. And this
not only stayed with him during the whole of his imprisonment,
but even grew in strength as the hardships of his position
increased.

That feeling of alertness and of readiness for anything was still
further strengthened in him by the high opinion his fellow
prisoners formed of him soon after his arrival at the shed. With
his knowledge of languages, the respect shown him by the French,
his simplicity, his readiness to give anything asked of him (he
received the allowance of three rubles a week made to officers);
with his strength, which he showed to the soldiers by pressing
nails into the walls of the hut; his gentleness to his
companions, and his capacity for sitting still and thinking
without doing anything (which seemed to them incomprehensible),
he appeared to them a rather mysterious and superior being. The
very qualities that had been a hindrance, if not actually
harmful, to him in the world he had lived in---his strength, his
disdain for the comforts of life, his absent-mindedness and
simplicity---here among these people gave him almost the status
of a hero. And Pierre felt that their opinion placed
responsibilities upon him.

% % % % % % % % % % % % % % % % % % % % % % % % % % % % % % % % %
% % % % % % % % % % % % % % % % % % % % % % % % % % % % % % % % %
% % % % % % % % % % % % % % % % % % % % % % % % % % % % % % % % %
% % % % % % % % % % % % % % % % % % % % % % % % % % % % % % % % %
% % % % % % % % % % % % % % % % % % % % % % % % % % % % % % % % %
% % % % % % % % % % % % % % % % % % % % % % % % % % % % % % % % %
% % % % % % % % % % % % % % % % % % % % % % % % % % % % % % % % %
% % % % % % % % % % % % % % % % % % % % % % % % % % % % % % % % %
% % % % % % % % % % % % % % % % % % % % % % % % % % % % % % % % %
% % % % % % % % % % % % % % % % % % % % % % % % % % % % % % % % %
% % % % % % % % % % % % % % % % % % % % % % % % % % % % % % % % %
% % % % % % % % % % % % % % % % % % % % % % % % % % % % % %

\chapter*{Chapter XIII} \ifaudio \marginpar{
\href{http://ia800204.us.archive.org/15/items/war_and_peace_13_0912_librivox/war_and_peace_13_13_tolstoy_64kb.mp3}{Audio}}
\fi

\initial{T}{he} French evacuation began on the night between the sixth and
seventh of October: kitchens and sheds were dismantled, carts
loaded, and troops and baggage trains started.

At seven in the morning a French convoy in marching trim, wearing
shakos and carrying muskets, knapsacks, and enormous sacks, stood
in front of the sheds, and animated French talk mingled with
curses sounded all along the lines.

In the shed everyone was ready, dressed, belted, shod, and only
awaited the order to start. The sick soldier, Sokolov, pale and
thin with dark shadows round his eyes, alone sat in his place
barefoot and not dressed.  His eyes, prominent from the
emaciation of his face, gazed inquiringly at his comrades who
were paying no attention to him, and he moaned regularly and
quietly. It was evidently not so much his sufferings that caused
him to moan (he had dysentery) as his fear and grief at being
left alone.

Pierre, girt with a rope round his waist and wearing shoes
Karataev had made for him from some leather a French soldier had
torn off a tea chest and brought to have his boots mended with,
went up to the sick man and squatted down beside him.

``You know, Sokolov, they are not all going away! They have a
hospital here. You may be better off than we others,'' said
Pierre.

``O Lord! Oh, it will be the death of me! O Lord!'' moaned the
man in a louder voice.

``I'll go and ask them again directly,'' said Pierre, rising and
going to the door of the shed.

Just as Pierre reached the door, the corporal who had offered him
a pipe the day before came up to it with two soldiers. The
corporal and soldiers were in marching kit with knapsacks and
shakos that had metal straps, and these changed their familiar
faces.

The corporal came, according to orders, to shut the door. The
prisoners had to be counted before being let out.

``Corporal, what will they do with the sick man?...'' Pierre
began.

But even as he spoke he began to doubt whether this was the
corporal he knew or a stranger, so unlike himself did the
corporal seem at that moment. Moreover, just as Pierre was
speaking a sharp rattle of drums was suddenly heard from both
sides. The corporal frowned at Pierre's words and, uttering some
meaningless oaths, slammed the door. The shed became semidark,
and the sharp rattle of the drums on two sides drowned the sick
man's groans.

``There it is!... It again!...'' said Pierre to himself, and an
involuntary shudder ran down his spine. In the corporal's changed
face, in the sound of his voice, in the stirring and deafening
noise of the drums, he recognized that mysterious, callous force
which compelled people against their will to kill their fellow
men---that force the effect of which he had witnessed during the
executions. To fear or to try to escape that force, to address
entreaties or exhortations to those who served as its tools, was
useless. Pierre knew this now. One had to wait and endure. He did
not again go to the sick man, nor turn to look at him, but stood
frowning by the door of the hut.

When that door was opened and the prisoners, crowding against one
another like a flock of sheep, squeezed into the exit, Pierre
pushed his way forward and approached that very captain who as
the corporal had assured him was ready to do anything for
him. The captain was also in marching kit, and on his cold face
appeared that same it which Pierre had recognized in the
corporal's words and in the roll of the drums.

``Pass on, pass on!'' the captain reiterated, frowning sternly,
and looking at the prisoners who thronged past him.

Pierre went up to him, though he knew his attempt would be vain.

``What now?'' the officer asked with a cold look as if not
recognizing Pierre.

Pierre told him about the sick man.

``He'll manage to walk, devil take him!'' said the
captain. ``Pass on, pass on!'' he continued without looking at
Pierre.

``But he is dying,'' Pierre again began.

``Be so good...'' shouted the captain, frowning angrily.

``Dram-da-da-dam, dam-dam...'' rattled the drums, and Pierre
understood that this mysterious force completely controlled these
men and that it was now useless to say any more.

The officer prisoners were separated from the soldiers and told
to march in front. There were about thirty officers, with Pierre
among them, and about three hundred men.

The officers, who had come from the other sheds, were all
strangers to Pierre and much better dressed than he. They looked
at him and at his shoes mistrustfully, as at an alien. Not far
from him walked a fat major with a sallow, bloated, angry face,
who was wearing a Kazan dressing gown tied round with a towel,
and who evidently enjoyed the respect of his fellow prisoners. He
kept one hand, in which he clasped his tobacco pouch, inside the
bosom of his dressing gown and held the stem of his pipe firmly
with the other. Panting and puffing, the major grumbled and
growled at everybody because he thought he was being pushed and
that they were all hurrying when they had nowhere to hurry to and
were all surprised at something when there was nothing to be
surprised at.  Another, a thin little officer, was speaking to
everyone, conjecturing where they were now being taken and how
far they would get that day. An official in felt boots and
wearing a commissariat uniform ran round from side to side and
gazed at the ruins of Moscow, loudly announcing his observations
as to what had been burned down and what this or that part of the
city was that they could see. A third officer, who by his accent
was a Pole, disputed with the commissariat officer, arguing that
he was mistaken in his identification of the different wards of
Moscow.

``What are you disputing about?'' said the major angrily. ``What
does it matter whether it is St. Nicholas or St. Blasius? You see
it's burned down, and there's an end of it... What are you
pushing for? Isn't the road wide enough?'' said he, turning to a
man behind him who was not pushing him at all.

``Oh, oh, oh! What have they done?'' the prisoners on one side
and another were heard saying as they gazed on the charred
ruins. ``All beyond the river, and Zubova, and in the
Kremlin... Just look! There's not half of it left. Yes, I told
you---the whole quarter beyond the river, and so it is.''

``Well, you know it's burned, so what's the use of talking?''
said the major.

As they passed near a church in the Khamovniki (one of the few
unburned quarters of Moscow) the whole mass of prisoners suddenly
started to one side and exclamations of horror and disgust were
heard.

``Ah, the villains! What heathens! Yes; dead, dead, so he
is... And smeared with something!''

Pierre too drew near the church where the thing was that evoked
these exclamations, and dimly made out something leaning against
the palings surrounding the church. From the words of his
comrades who saw better than he did, he found that this was the
body of a man, set upright against the palings with its face
smeared with soot.

``Go on! What the devil... Go on! Thirty thousand devils!...''
the convoy guards began cursing and the French soldiers, with
fresh virulence, drove away with their swords the crowd of
prisoners who were gazing at the dead man.

% % % % % % % % % % % % % % % % % % % % % % % % % % % % % % % % %
% % % % % % % % % % % % % % % % % % % % % % % % % % % % % % % % %
% % % % % % % % % % % % % % % % % % % % % % % % % % % % % % % % %
% % % % % % % % % % % % % % % % % % % % % % % % % % % % % % % % %
% % % % % % % % % % % % % % % % % % % % % % % % % % % % % % % % %
% % % % % % % % % % % % % % % % % % % % % % % % % % % % % % % % %
% % % % % % % % % % % % % % % % % % % % % % % % % % % % % % % % %
% % % % % % % % % % % % % % % % % % % % % % % % % % % % % % % % %
% % % % % % % % % % % % % % % % % % % % % % % % % % % % % % % % %
% % % % % % % % % % % % % % % % % % % % % % % % % % % % % % % % %
% % % % % % % % % % % % % % % % % % % % % % % % % % % % % % % % %
% % % % % % % % % % % % % % % % % % % % % % % % % % % % % %

\chapter*{Chapter XIV} \ifaudio \marginpar{
\href{http://ia800204.us.archive.org/15/items/war_and_peace_13_0912_librivox/war_and_peace_13_14_tolstoy_64kb.mp3}{Audio}}
\fi

\initial{T}{hrough} the cross streets of the Khamovniki quarter the prisoners
marched, followed only by their escort and the vehicles and
wagons belonging to that escort, but when they reached the supply
stores they came among a huge and closely packed train of
artillery mingled with private vehicles.

At the bridge they all halted, waiting for those in front to get
across.  From the bridge they had a view of endless lines of
moving baggage trains before and behind them. To the right, where
the Kaluga road turns near Neskuchny, endless rows of troops and
carts stretched away into the distance. These were troops of
Beauharnais' corps which had started before any of the
others. Behind, along the riverside and across the Stone Bridge,
were Ney's troops and transport.

Davout's troops, in whose charge were the prisoners, were
crossing the Crimean bridge and some were already debouching into
the Kaluga road.  But the baggage trains stretched out so that
the last of Beauharnais' train had not yet got out of Moscow and
reached the Kaluga road when the vanguard of Ney's army was
already emerging from the Great Ordynka Street.

When they had crossed the Crimean bridge the prisoners moved a
few steps forward, halted, and again moved on, and from all sides
vehicles and men crowded closer and closer together. They
advanced the few hundred paces that separated the bridge from the
Kaluga road, taking more than an hour to do so, and came out upon
the square where the streets of the Transmoskva ward and the
Kaluga road converge, and the prisoners jammed close together had
to stand for some hours at that crossway. From all sides, like
the roar of the sea, were heard the rattle of wheels, the tramp
of feet, and incessant shouts of anger and abuse. Pierre stood
pressed against the wall of a charred house, listening to that
noise which mingled in his imagination with the roll of the
drums.

To get a better view, several officer prisoners climbed onto the
wall of the half-burned house against which Pierre was leaning.

``What crowds! Just look at the crowds!... They've loaded goods
even on the cannon! Look there, those are furs!'' they
exclaimed. ``Just see what the blackguards have looted... There!
See what that one has behind in the cart... Why, those are
settings taken from some icons, by heaven!... Oh, the
rascals!... See how that fellow has loaded himself up, he can
hardly walk! Good lord, they've even grabbed those
chaises!... See that fellow there sitting on the
trunks... Heavens!  They're fighting.''

``That's right, hit him on the snout---on his snout! Like this,
we shan't get away before evening. Look, look there... Why, that
must be Napoleon's own. See what horses! And the monograms with a
crown! It's like a portable house... That fellow's dropped his
sack and doesn't see it. Fighting again... A woman with a baby,
and not bad-looking either!  Yes, I dare say, that's the way
they'll let you pass... Just look, there's no end to it. Russian
wenches, by heaven, so they are! In carriages---see how
comfortably they've settled themselves!''

Again, as at the church in Khamovniki, a wave of general
curiosity bore all the prisoners forward onto the road, and
Pierre, thanks to his stature, saw over the heads of the others
what so attracted their curiosity. In three carriages involved
among the munition carts, closely squeezed together, sat women
with rouged faces, dressed in glaring colors, who were shouting
something in shrill voices.

From the moment Pierre had recognized the appearance of the
mysterious force nothing had seemed to him strange or dreadful:
neither the corpse smeared with soot for fun nor these women
hurrying away nor the burned ruins of Moscow. All that he now
witnessed scarcely made an impression on him---as if his soul,
making ready for a hard struggle, refused to receive impressions
that might weaken it.

The women's vehicles drove by. Behind them came more carts,
soldiers, wagons, soldiers, gun carriages, carriages, soldiers,
ammunition carts, more soldiers, and now and then women.

Pierre did not see the people as individuals but saw their
movement.

All these people and horses seemed driven forward by some
invisible power. During the hour Pierre watched them they all
came flowing from the different streets with one and the same
desire to get on quickly; they all jostled one another, began to
grow angry and to fight, white teeth gleamed, brows frowned, ever
the same words of abuse flew from side to side, and all the faces
bore the same swaggeringly resolute and coldly cruel expression
that had struck Pierre that morning on the corporal's face when
the drums were beating.

It was not till nearly evening that the officer commanding the
escort collected his men and with shouts and quarrels forced his
way in among the baggage trains, and the prisoners, hemmed in on
all sides, emerged onto the Kaluga road.

They marched very quickly, without resting, and halted only when
the sun began to set. The baggage carts drew up close together
and the men began to prepare for their night's rest. They all
appeared angry and dissatisfied. For a long time, oaths, angry
shouts, and fighting could be heard from all sides. A carriage
that followed the escort ran into one of the carts and knocked a
hole in it with its pole. Several soldiers ran toward the cart
from different sides: some beat the carriage horses on their
heads, turning them aside, others fought among themselves, and
Pierre saw that one German was badly wounded on the head by a
sword.

It seemed that all these men, now that they had stopped amid
fields in the chill dusk of the autumn evening, experienced one
and the same feeling of unpleasant awakening from the hurry and
eagerness to push on that had seized them at the start. Once at a
standstill they all seemed to understand that they did not yet
know where they were going, and that much that was painful and
difficult awaited them on this journey.

During this halt the escort treated the prisoners even worse than
they had done at the start. It was here that the prisoners for
the first time received horseflesh for their meat ration.

From the officer down to the lowest soldier they showed what
seemed like personal spite against each of the prisoners, in
unexpected contrast to their former friendly relations.

This spite increased still more when, on calling over the roll of
prisoners, it was found that in the bustle of leaving Moscow one
Russian soldier, who had pretended to suffer from colic, had
escaped. Pierre saw a Frenchman beat a Russian soldier cruelly
for straying too far from the road, and heard his friend the
captain reprimand and threaten to court-martial a noncommissioned
officer on account of the escape of the Russian. To the
noncommissioned officer's excuse that the prisoner was ill and
could not walk, the officer replied that the order was to shoot
those who lagged behind. Pierre felt that that fatal force which
had crushed him during the executions, but which he had not felt
during his imprisonment, now again controlled his existence. It
was terrible, but he felt that in proportion to the efforts of
that fatal force to crush him, there grew and strengthened in his
soul a power of life independent of it.

He ate his supper of buckwheat soup with horseflesh and chatted
with his comrades.

Neither Pierre nor any of the others spoke of what they had seen
in Moscow, or of the roughness of their treatment by the French,
or of the order to shoot them which had been announced to
them. As if in reaction against the worsening of their position
they were all particularly animated and gay. They spoke of
personal reminiscences, of amusing scenes they had witnessed
during the campaign, and avoided all talk of their present
situation.

The sun had set long since. Bright stars shone out here and there
in the sky. A red glow as of a conflagration spread above the
horizon from the rising full moon, and that vast red ball swayed
strangely in the gray haze. It grew light. The evening was
ending, but the night had not yet come. Pierre got up and left
his new companions, crossing between the campfires to the other
side of the road where he had been told the common soldier
prisoners were stationed. He wanted to talk to them. On the road
he was stopped by a French sentinel who ordered him back.

Pierre turned back, not to his companions by the campfire, but to
an unharnessed cart where there was nobody. Tucking his legs
under him and dropping his head he sat down on the cold ground by
the wheel of the cart and remained motionless a long while sunk
in thought. Suddenly he burst out into a fit of his broad,
good-natured laughter, so loud that men from various sides turned
with surprise to see what this strange and evidently solitary
laughter could mean.

``Ha-ha-ha!'' laughed Pierre. And he said aloud to himself: ``The
soldier did not let me pass. They took me and shut me up. They
hold me captive.  What, me? Me? My immortal soul? Ha-ha-ha!
Ha-ha-ha!...'' and he laughed till tears started to his eyes.

A man got up and came to see what this queer big fellow was
laughing at all by himself. Pierre stopped laughing, got up, went
farther away from the inquisitive man, and looked around him.

The huge, endless bivouac that had previously resounded with the
crackling of campfires and the voices of many men had grown
quiet, the red campfires were growing paler and dying down. High
up in the light sky hung the full moon. Forests and fields beyond
the camp, unseen before, were now visible in the distance. And
farther still, beyond those forests and fields, the bright,
oscillating, limitless distance lured one to itself. Pierre
glanced up at the sky and the twinkling stars in its faraway
depths. ``And all that is me, all that is within me, and it is
all I!'' thought Pierre. ``And they caught all that and put it
into a shed boarded up with planks!'' He smiled, and went and lay
down to sleep beside his companions.

% % % % % % % % % % % % % % % % % % % % % % % % % % % % % % % % %
% % % % % % % % % % % % % % % % % % % % % % % % % % % % % % % % %
% % % % % % % % % % % % % % % % % % % % % % % % % % % % % % % % %
% % % % % % % % % % % % % % % % % % % % % % % % % % % % % % % % %
% % % % % % % % % % % % % % % % % % % % % % % % % % % % % % % % %
% % % % % % % % % % % % % % % % % % % % % % % % % % % % % % % % %
% % % % % % % % % % % % % % % % % % % % % % % % % % % % % % % % %
% % % % % % % % % % % % % % % % % % % % % % % % % % % % % % % % %
% % % % % % % % % % % % % % % % % % % % % % % % % % % % % % % % %
% % % % % % % % % % % % % % % % % % % % % % % % % % % % % % % % %
% % % % % % % % % % % % % % % % % % % % % % % % % % % % % % % % %
% % % % % % % % % % % % % % % % % % % % % % % % % % % % % %

\chapter*{Chapter XV} \ifaudio \marginpar{
\href{http://ia800204.us.archive.org/15/items/war_and_peace_13_0912_librivox/war_and_peace_13_15_tolstoy_64kb.mp3}{Audio}}
\fi

\initial{I}{n} the early days of October another 
envoy came to Kutuzov with a
letter from Napoleon proposing peace and falsely dated from
Moscow, though Napoleon was already not far from Kutuzov on the
old Kaluga road.  Kutuzov replied to this letter as he had done
to the one formerly brought by Lauriston, saying that there could
be no question of peace.

Soon after that a report was received from Dorokhov's guerrilla
detachment operating to the left of Tarutino that troops of
Broussier's division had been seen at Forminsk and that being
separated from the rest of the French army they might easily be
destroyed. The soldiers and officers again demanded
action. Generals on the staff, excited by the memory of the easy
victory at Tarutino, urged Kutuzov to carry out Dorokhov's
suggestion. Kutuzov did not consider any offensive necessary.
The result was a compromise which was inevitable: a small
detachment was sent to Forminsk to attack Broussier.

By a strange coincidence, this task, which turned out to be a
most difficult and important one, was entrusted to
Dokhturov---that same modest little Dokhturov whom no one had
described to us as drawing up plans of battles, dashing about in
front of regiments, showering crosses on batteries, and so on,
and who was thought to be and was spoken of as undecided and
undiscerning---but whom we find commanding wherever the position
was most difficult all through the Russo-French wars from
Austerlitz to the year 1813. At Austerlitz he remained last at
the Augezd dam, rallying the regiments, saving what was possible
when all were flying and perishing and not a single general was
left in the rear guard. Ill with fever he went to Smolensk with
twenty thousand men to defend the town against Napoleon's whole
army. In Smolensk, at the Malakhov Gate, he had hardly dozed off
in a paroxysm of fever before he was awakened by the bombardment
of the town---and Smolensk held out all day long. At the battle
of Borodino, when Bagration was killed and nine tenths of the men
of our left flank had fallen and the full force of the French
artillery fire was directed against it, the man sent there was
this same irresolute and undiscerning Dokhturov---Kutuzov
hastening to rectify a mistake he had made by sending someone
else there first. And the quiet little Dokhturov rode thither,
and Borodino became the greatest glory of the Russian army. Many
heroes have been described to us in verse and prose, but of
Dokhturov scarcely a word has been said.

It was Dokhturov again whom they sent to Forminsk and from there
to Malo-Yaroslavets, the place where the last battle with the
French was fought and where the obvious disintegration of the
French army began; and we are told of many geniuses and heroes of
that period of the campaign, but of Dokhturov nothing or very
little is said and that dubiously. And this silence about
Dokhturov is the clearest testimony to his merit.

It is natural for a man who does not understand the workings of a
machine to imagine that a shaving that has fallen into it by
chance and is interfering with its action and tossing about in it
is its most important part. The man who does not understand the
construction of the machine cannot conceive that the small
connecting cogwheel which revolves quietly is one of the most
essential parts of the machine, and not the shaving which merely
harms and hinders the working.

On the tenth of October when Dokhturov had gone halfway to
Forminsk and stopped at the village of Aristovo, preparing
faithfully to execute the orders he had received, the whole
French army having, in its convulsive movement, reached Murat's
position apparently in order to give battle---suddenly without
any reason turned off to the left onto the new Kaluga road and
began to enter Forminsk, where only Broussier had been till
then. At that time Dokhturov had under his command, besides
Dorokhov's detachment, the two small guerrilla detachments of
Figner and Seslavin.

On the evening of October 11th Seslavin came to the Aristovo
headquarters with a French guardsman he had captured. The
prisoner said that the troops that had entered Forminsk that day
were the vanguard of the whole army, that Napoleon was there and
the whole army had left Moscow four days previously. That same
evening a house serf who had come from Borovsk said he had seen
an immense army entering the town. Some Cossacks of Dokhturov's
detachment reported having sighted the French Guards marching
along the road to Borovsk. From all these reports it was evident
that where they had expected to meet a single division there was
now the whole French army marching from Moscow in an unexpected
direction---along the Kaluga road. Dokhturov was unwilling to
undertake any action, as it was not clear to him now what he
ought to do. He had been ordered to attack Forminsk. But only
Broussier had been there at that time and now the whole French
army was there. Ermolov wished to act on his own judgment, but
Dokhturov insisted that he must have Kutuzov's instructions. So
it was decided to send a dispatch to the staff.

For this purpose a capable officer, Bolkhovitinov, was chosen,
who was to explain the whole affair by word of mouth, besides
delivering a written report. Toward midnight Bolkhovitinov,
having received the dispatch and verbal instructions, galloped
off to the General Staff accompanied by a Cossack with spare
horses.

% % % % % % % % % % % % % % % % % % % % % % % % % % % % % % % % %
% % % % % % % % % % % % % % % % % % % % % % % % % % % % % % % % %
% % % % % % % % % % % % % % % % % % % % % % % % % % % % % % % % %
% % % % % % % % % % % % % % % % % % % % % % % % % % % % % % % % %
% % % % % % % % % % % % % % % % % % % % % % % % % % % % % % % % %
% % % % % % % % % % % % % % % % % % % % % % % % % % % % % % % % %
% % % % % % % % % % % % % % % % % % % % % % % % % % % % % % % % %
% % % % % % % % % % % % % % % % % % % % % % % % % % % % % % % % %
% % % % % % % % % % % % % % % % % % % % % % % % % % % % % % % % %
% % % % % % % % % % % % % % % % % % % % % % % % % % % % % % % % %
% % % % % % % % % % % % % % % % % % % % % % % % % % % % % % % % %
% % % % % % % % % % % % % % % % % % % % % % % % % % % % % %

\chapter*{Chapter XVI} \ifaudio \marginpar{
\href{http://ia800204.us.archive.org/15/items/war_and_peace_13_0912_librivox/war_and_peace_13_16_tolstoy_64kb.mp3}{Audio}}
\fi

\initial{I}{t} was a warm, dark, autumn night. It had been raining for four
days.  Having changed horses twice and galloped twenty miles in
an hour and a half over a sticky, muddy road, Bolkhovitinov
reached Litashevka after one o'clock at night. Dismounting at a
cottage on whose wattle fence hung a signboard, GENERAL STAFF,
and throwing down his reins, he entered a dark passage.

``The general on duty, quick! It's very important!'' said he to
someone who had risen and was sniffing in the dark passage.

``He has been very unwell since the evening and this is the third
night he has not slept,'' said the orderly pleadingly in a
whisper. ``You should wake the captain first.''

``But this is very important, from General Dokhturov,'' said
Bolkhovitinov, entering the open door which he had found by
feeling in the dark.

The orderly had gone in before him and began waking somebody.

``Your honor, your honor! A courier.''

``What? What's that? From whom?'' came a sleepy voice.

``From Dokhturov and from Alexey Petrovich. Napoleon is at
Forminsk,'' said Bolkhovitinov, unable to see in the dark who was
speaking but guessing by the voice that it was not Konovnitsyn.

The man who had wakened yawned and stretched himself.

``I don't like waking him,'' he said, fumbling for
something. ``He is very ill. Perhaps this is only a rumor.''

``Here is the dispatch,'' said Bolkhovitinov. ``My orders are to
give it at once to the general on duty.''

``Wait a moment, I'll light a candle. You damned rascal, where do
you always hide it?'' said the voice of the man who was
stretching himself, to the orderly. (This was Shcherbinin,
Konovnitsyn's adjutant.) ``I've found it, I've found it!'' he
added.

The orderly was striking a light and Shcherbinin was fumbling for
something on the candlestick.

``Oh, the nasty beasts!'' said he with disgust.

By the light of the sparks Bolkhovitinov saw Shcherbinin's
youthful face as he held the candle, and the face of another man
who was still asleep.  This was Konovnitsyn.

When the flame of the sulphur splinters kindled by the tinder
burned up, first blue and then red, Shcherbinin lit the tallow
candle, from the candlestick of which the cockroaches that had
been gnawing it were running away, and looked at the
messenger. Bolkhovitinov was bespattered all over with mud and
had smeared his face by wiping it with his sleeve.

``Who gave the report?'' inquired Shcherbinin, taking the
envelope.

``The news is reliable,'' said Bolkhovitinov. ``Prisoners,
Cossacks, and the scouts all say the same thing.''

``There's nothing to be done, we'll have to wake him,'' said
Shcherbi\-nin, rising and going up to the man in the nightcap who
lay covered by a greatcoat. ``Peter Petrovich!'' said
he. (Konovnitsyn did not stir.) ``To the General Staff!'' he said
with a smile, knowing that those words would be sure to arouse
him.

And in fact the head in the nightcap was lifted at once. On
Konovnitsyn's handsome, resolute face with cheeks flushed by
fever, there still remained for an instant a faraway dreamy
expression remote from present affairs, but then he suddenly
started and his face assumed its habitual calm and firm
appearance.

``Well, what is it? From whom?'' he asked immediately but without
hurry, blinking at the light.

While listening to the officer's report Konovnitsyn broke the
seal and read the dispatch. Hardly had he done so before he
lowered his legs in their woolen stockings to the earthen floor
and began putting on his boots. Then he took off his nightcap,
combed his hair over his temples, and donned his cap.

``Did you get here quickly? Let us go to his Highness.''

Konovnitsyn had understood at once that the news brought was of
great importance and that no time must be lost. He did not
consider or ask himself whether the news was good or bad. That
did not interest him. He regarded the whole business of the war
not with his intelligence or his reason but by something
else. There was within him a deep unexpressed conviction that all
would be well, but that one must not trust to this and still less
speak about it, but must only attend to one's own work.  And he
did his work, giving his whole strength to the task.

Peter Petrovich Konovnitsyn, like Dokhturov, seems to have been
included merely for propriety's sake in the list of the so-called
heroes of 1812---the Barclays, Raevskis, Ermolovs, Platovs, and
Miloradoviches. Like Dokhturov he had the reputation of being a
man of very limited capacity and information, and like Dokhturov
he never made plans of battle but was always found where the
situation was most difficult. Since his appointment as general on
duty he had always slept with his door open, giving orders that
every messenger should be allowed to wake him up. In battle he
was always under fire, so that Kutuzov reproved him for it and
feared to send him to the front, and like Dokhturov he was one of
those unnoticed cogwheels that, without clatter or noise,
constitute the most essential part of the machine.

Coming out of the hut into the damp, dark night Konovnitsyn
frowned---partly from an increased pain in his head and partly at
the unpleasant thought that occurred to him, of how all that nest
of influential men on the staff would be stirred up by this news,
especially Bennigsen, who ever since Tarutino had been at daggers
drawn with Kutuzov; and how they would make suggestions, quarrel,
issue orders, and rescind them. And this premonition was
disagreeable to him though he knew it could not be helped.

And in fact Toll, to whom he went to communicate the news,
immediately began to expound his plans to a general sharing his
quarters, until Konovnitsyn, who listened in weary silence,
reminded him that they must go to see his Highness.

% % % % % % % % % % % % % % % % % % % % % % % % % % % % % % % % %
% % % % % % % % % % % % % % % % % % % % % % % % % % % % % % % % %
% % % % % % % % % % % % % % % % % % % % % % % % % % % % % % % % %
% % % % % % % % % % % % % % % % % % % % % % % % % % % % % % % % %
% % % % % % % % % % % % % % % % % % % % % % % % % % % % % % % % %
% % % % % % % % % % % % % % % % % % % % % % % % % % % % % % % % %
% % % % % % % % % % % % % % % % % % % % % % % % % % % % % % % % %
% % % % % % % % % % % % % % % % % % % % % % % % % % % % % % % % %
% % % % % % % % % % % % % % % % % % % % % % % % % % % % % % % % %
% % % % % % % % % % % % % % % % % % % % % % % % % % % % % % % % %
% % % % % % % % % % % % % % % % % % % % % % % % % % % % % % % % %
% % % % % % % % % % % % % % % % % % % % % % % % % % % % % %

\chapter*{Chapter XVII} \ifaudio \marginpar{
\href{http://ia800204.us.archive.org/15/items/war_and_peace_13_0912_librivox/war_and_peace_13_17_tolstoy_64kb.mp3}{Audio}}
\fi

\initial{K}{utuzov} like all old people did not sleep much at night. He often
fell asleep unexpectedly in the daytime, but at night, lying on
his bed without undressing, he generally remained awake thinking.

So he lay now on his bed, supporting his large, heavy, scarred
head on his plump hand, with his one eye open, meditating and
peering into the darkness.

Since Bennigsen, who corresponded with the Emperor and had more
influence than anyone else on the staff, had begun to avoid him,
Kutuzov was more at ease as to the possibility of himself and his
troops being obliged to take part in useless aggressive
movements. The lesson of the Tarutino battle and of the day
before it, which Kutuzov remembered with pain, must, he thought,
have some effect on others too.

``They must understand that we can only lose by taking the
offensive.  Patience and time are my warriors, my champions,''
thought Kutuzov. He knew that an apple should not be plucked
while it is green. It will fall of itself when ripe, but if
picked unripe the apple is spoiled, the tree is harmed, and your
teeth are set on edge. Like an experienced sportsman he knew that
the beast was wounded, and wounded as only the whole strength of
Russia could have wounded it, but whether it was mortally wounded
or not was still an undecided question. Now by the fact of
Lauriston and Barthelemi having been sent, and by the reports of
the guerrillas, Kutuzov was almost sure that the wound was
mortal. But he needed further proofs and it was necessary to
wait.

``They want to run to see how they have wounded it. Wait and we
shall see! Continual maneuvers, continual advances!'' thought
he. ``What for?  Only to distinguish themselves! As if fighting
were fun. They are like children from whom one can't get any
sensible account of what has happened because they all want to
show how well they can fight. But that's not what is needed
now.''

``And what ingenious maneuvers they all propose to me! It seems
to them that when they have thought of two or three
contingencies'' (he remembered the general plan sent him from
Petersburg) ``they have foreseen everything. But the
contingencies are endless.''

The undecided question as to whether the wound inflicted at
Borodino was mortal or not had hung over Kutuzov's head for a
whole month. On the one hand the French had occupied Moscow. On
the other Kutuzov felt assured with all his being that the
terrible blow into which he and all the Russians had put their
whole strength must have been mortal. But in any case proofs were
needed; he had waited a whole month for them and grew more
impatient the longer he waited. Lying on his bed during those
sleepless nights he did just what he reproached those younger
generals for doing. He imagined all sorts of possible
contingencies, just like the younger men, but with this
difference, that he saw thousands of contingencies instead of two
or three and based nothing on them. The longer he thought the
more contingencies presented themselves. He imagined all sorts of
movements of the Napoleonic army as a whole or in
sections---against Petersburg, or against him, or to outflank
him. He thought too of the possibility (which he feared most of
all) that Napoleon might fight him with his own weapon and remain
in Moscow awaiting him. Kutuzov even imagined that Napoleon's
army might turn back through Medyn and Yukhnov, but the one thing
he could not foresee was what happened---the insane, convulsive
stampede of Napoleon's army during its first eleven days after
leaving Moscow: a stampede which made possible what Kutuzov had
not yet even dared to think of---the complete extermination of
the French. Dorokhov's report about Broussier's division, the
guerrillas' reports of distress in Napoleon's army, rumors of
preparations for leaving Moscow, all confirmed the supposition
that the French army was beaten and preparing for flight. But
these were only suppositions, which seemed important to the
younger men but not to Kutuzov. With his sixty years' experience
he knew what value to attach to rumors, knew how apt people who
desire anything are to group all news so that it appears to
confirm what they desire, and he knew how readily in such cases
they omit all that makes for the contrary. And the more he
desired it the less he allowed himself to believe it. This
question absorbed all his mental powers. All else was to him only
life's customary routine. To such customary routine belonged his
conversations with the staff, the letters he wrote from Tarutino
to Madame de Stael, the reading of novels, the distribution of
awards, his correspondence with Petersburg, and so on. But the
destruction of the French, which he alone foresaw, was his
heart's one desire.

On the night of the eleventh of October he lay leaning on his arm
and thinking of that.

There was a stir in the next room and he heard the steps of Toll,
Konovnitsyn, and Bolkhovitinov.

``Eh, who's there? Come in, come in! What news?'' the field
marshal called out to them.

While a footman was lighting a candle, Toll communicated the
substance of the news.

``Who brought it?'' asked Kutuzov with a look which, when the
candle was lit, struck Toll by its cold severity.

``There can be no doubt about it, your Highness.''

``Call him in, call him here.''

Kutuzov sat up with one leg hanging down from the bed and his big
paunch resting against the other which was doubled under him. He
screwed up his seeing eye to scrutinize the messenger more
carefully, as if wishing to read in his face what preoccupied his
own mind.

``Tell me, tell me, friend,'' said he to Bolkhovitinov in his
low, aged voice, as he pulled together the shirt which gaped open
on his chest, ``come nearer---nearer. What news have you brought
me?  Eh? That Napoleon has left Moscow? Are you sure? Eh?''

Bolkhovitinov gave a detailed account from the beginning of all
he had been told to report.

``Speak quicker, quicker! Don't torture me!'' Kutuzov interrupted
him.

Bolkhovitinov told him everything and was then silent, awaiting
instructions. Toll was beginning to say something but Kutuzov
checked him. He tried to say something, but his face suddenly
puckered and wrinkled; he waved his arm at Toll and turned to the
opposite side of the room, to the corner darkened by the icons
that hung there.

``O Lord, my Creator, Thou has heard our prayer...'' said he in a
tremulous voice with folded hands. ``Russia is saved. I thank
Thee, O Lord!'' and he wept.

% % % % % % % % % % % % % % % % % % % % % % % % % % % % % % % % %
% % % % % % % % % % % % % % % % % % % % % % % % % % % % % % % % %
% % % % % % % % % % % % % % % % % % % % % % % % % % % % % % % % %
% % % % % % % % % % % % % % % % % % % % % % % % % % % % % % % % %
% % % % % % % % % % % % % % % % % % % % % % % % % % % % % % % % %
% % % % % % % % % % % % % % % % % % % % % % % % % % % % % % % % %
% % % % % % % % % % % % % % % % % % % % % % % % % % % % % % % % %
% % % % % % % % % % % % % % % % % % % % % % % % % % % % % % % % %
% % % % % % % % % % % % % % % % % % % % % % % % % % % % % % % % %
% % % % % % % % % % % % % % % % % % % % % % % % % % % % % % % % %
% % % % % % % % % % % % % % % % % % % % % % % % % % % % % % % % %
% % % % % % % % % % % % % % % % % % % % % % % % % % % % % %

\chapter*{Chapter XVIII} \ifaudio \marginpar{
\href{http://ia800204.us.archive.org/15/items/war_and_peace_13_0912_librivox/war_and_peace_13_18_tolstoy_64kb.mp3}{Audio}}
\fi

\initial{F}{rom} the time he received this news to the end of the campaign
all Kutuzov's activity was directed toward restraining his
troops, by authority, by guile, and by entreaty, from useless
attacks, maneuvers, or encounters with the perishing
enemy. Dokhturov went to Malo-Yaroslavets, but Kutuzov lingered
with the main army and gave orders for the evacuation of
Kaluga---a retreat beyond which town seemed to him quite
possible.

Everywhere Kutuzov retreated, but the enemy without waiting for
his retreat fled in the opposite direction.

Napoleon's historians describe to us his skilled maneuvers at
Tarutino and Malo-Yaroslavets, and make conjectures as to what
would have happened had Napoleon been in time to penetrate into
the rich southern provinces.

But not to speak of the fact that nothing prevented him from
advancing into those southern provinces (for the Russian army did
not bar his way), the historians forget that nothing could have
saved his army, for then already it bore within itself the germs
of inevitable ruin. How could that army---which had found
abundant supplies in Moscow and had trampled them underfoot
instead of keeping them, and on arriving at Smolensk had looted
provisions instead of storing them---how could that army
recuperate in Kaluga province, which was inhabited by Russians
such as those who lived in Moscow, and where fire had the same
property of consuming what was set ablaze?

That army could not recover anywhere. Since the battle of
Borodino and the pillage of Moscow it had borne within itself, as
it were, the chemical elements of dissolution.

The members of what had once been an army---Napoleon himself and
all his soldiers fled---without knowing whither, each concerned
only to make his escape as quickly as possible from this
position, of the hopelessness of which they were all more or less
vaguely conscious.

So it came about that at the council at Malo-Yaroslavets, when
the generals pretending to confer together expressed various
opinions, all mouths were closed by the opinion uttered by the
simple-minded soldier Mouton who, speaking last, said what they
all felt: that the one thing needful was to get away as quickly
as possible; and no one, not even Napoleon, could say anything
against that truth which they all recognized.

But though they all realized that it was necessary to get away,
there still remained a feeling of shame at admitting that they
must flee. An external shock was needed to overcome that shame,
and this shock came in due time. It was what the French called
``le hourra de l'Empereur.''

The day after the council at Malo-Yaroslavets Napoleon rode out
early in the morning amid the lines of his army with his suite of
marshals and an escort, on the pretext of inspecting the army and
the scene of the previous and of the impending battle. Some
Cossacks on the prowl for booty fell in with the Emperor and very
nearly captured him. If the Cossacks did not capture Napoleon
then, what saved him was the very thing that was destroying the
French army, the booty on which the Cossacks fell. Here as at
Tarutino they went after plunder, leaving the men. Disregarding
Napoleon they rushed after the plunder and Napoleon managed to
escape.

When les enfants du Don might so easily have taken the Emperor
himself in the midst of his army, it was clear that there was
nothing for it but to fly as fast as possible along the nearest,
familiar road. Napoleon with his forty-year-old stomach
understood that hint, not feeling his former agility and
boldness, and under the influence of the fright the Cossacks had
given him he at once agreed with Mouton and issued orders---as
the historians tell us---to retreat by the Smolensk road.

That Napoleon agreed with Mouton, and that the army retreated,
does not prove that Napoleon caused it to retreat, but that the
forces which influenced the whole army and directed it along the
Mozhaysk (that is, the Smolensk) road acted simultaneously on him
also.

% % % % % % % % % % % % % % % % % % % % % % % % % % % % % % % % %
% % % % % % % % % % % % % % % % % % % % % % % % % % % % % % % % %
% % % % % % % % % % % % % % % % % % % % % % % % % % % % % % % % %
% % % % % % % % % % % % % % % % % % % % % % % % % % % % % % % % %
% % % % % % % % % % % % % % % % % % % % % % % % % % % % % % % % %
% % % % % % % % % % % % % % % % % % % % % % % % % % % % % % % % %
% % % % % % % % % % % % % % % % % % % % % % % % % % % % % % % % %
% % % % % % % % % % % % % % % % % % % % % % % % % % % % % % % % %
% % % % % % % % % % % % % % % % % % % % % % % % % % % % % % % % %
% % % % % % % % % % % % % % % % % % % % % % % % % % % % % % % % %
% % % % % % % % % % % % % % % % % % % % % % % % % % % % % % % % %
% % % % % % % % % % % % % % % % % % % % % % % % % % % % % %

\chapter*{Chapter XIX} \ifaudio \marginpar{
\href{http://ia800204.us.archive.org/15/items/war_and_peace_13_0912_librivox/war_and_peace_13_19_tolstoy_64kb.mp3}{Audio}}
\fi

\initial{A}{}
man in motion always devises an aim for that motion. To be able
to go a thousand miles he must imagine that something good awaits
him at the end of those thousand miles. One must have the
prospect of a promised land to have the strength to move.

The promised land for the French during their advance had been
Moscow, during their retreat it was their native land. But that
native land was too far off, and for a man going a thousand miles
it is absolutely necessary to set aside his final goal and to say
to himself: ``Today I shall get to a place twenty-five miles off
where I shall rest and spend the night,'' and during the first
day's journey that resting place eclipses his ultimate goal and
attracts all his hopes and desires. And the impulses felt by a
single person are always magnified in a crowd.

For the French retreating along the old Smolensk road, the final
goal---their native land---was too remote, and their immediate
goal was Smolensk, toward which all their desires and hopes,
enormously intensified in the mass, urged them on. It was not
that they knew that much food and fresh troops awaited them in
Smolensk, nor that they were told so (on the contrary their
superior officers, and Napoleon himself, knew that provisions
were scarce there), but because this alone could give them
strength to move on and endure their present privations. So both
those who knew and those who did not know deceived themselves,
and pushed on to Smolensk as to a promised land.

Coming out onto the highroad the French fled with surprising
energy and unheard-of rapidity toward the goal they had fixed
on. Besides the common impulse which bound the whole crowd of
French into one mass and supplied them with a certain energy,
there was another cause binding them together---their great
numbers. As with the physical law of gravity, their enormous mass
drew the individual human atoms to itself. In their hundreds of
thousands they moved like a whole nation.

Each of them desired nothing more than to give himself up as a
prisoner to escape from all this horror and misery; but on the
one hand the force of this common attraction to Smolensk, their
goal, drew each of them in the same direction; on the other hand
an army corps could not surrender to a company, and though the
French availed themselves of every convenient opportunity to
detach themselves and to surrender on the slightest decent
pretext, such pretexts did not always occur. Their very numbers
and their crowded and swift movement deprived them of that
possibility and rendered it not only difficult but impossible for
the Russians to stop this movement, to which the French were
directing all their energies. Beyond a certain limit no
mechanical disruption of the body could hasten the process of
decomposition.

A lump of snow cannot be melted instantaneously. There is a
certain limit of time in less than which no amount of heat can
melt the snow. On the contrary the greater the heat the more
solidified the remaining snow becomes.

Of the Russian commanders Kutuzov alone understood this. When the
flight of the French army along the Smolensk road became well
defined, what Konovnitsyn had foreseen on the night of the
eleventh of October began to occur. The superior officers all
wanted to distinguish themselves, to cut off, to seize, to
capture, and to overthrow the French, and all clamored for
action.

Kutuzov alone used all his power (and such power is very limited
in the case of any commander-in-chief) to prevent an attack.

He could not tell them what we say now: ``Why fight, why block
the road, losing our own men and inhumanly slaughtering
unfortunate wretches? What is the use of that, when a third of
their army has melted away on the road from Moscow to Vyazma
without any battle?'' But drawing from his aged wisdom what they
could understand, he told them of the golden bridge, and they
laughed at and slandered him, flinging themselves on, rending and
exulting over the dying beast.

Ermolov, Miloradovich, Platov, and others in proximity to the
French near Vyazma could not resist their desire to cut off and
break up two French corps, and by way of reporting their
intention to Kutuzov they sent him a blank sheet of paper in an
envelope.

And try as Kutuzov might to restrain the troops, our men
attacked, trying to bar the road. Infantry regiments, we are
told, advanced to the attack with music and with drums beating,
and killed and lost thousands of men.

But they did not cut off or overthrow anybody and the French
army, closing up more firmly at the danger, continued, while
steadily melting away, to pursue its fatal path to Smolensk.