\part*{Book Six: 1808 - 10}

% % % % % % % % % % % % % % % % % % % % % % % % % % % % % % % % %
% % % % % % % % % % % % % % % % % % % % % % % % % % % % % % % % %
% % % % % % % % % % % % % % % % % % % % % % % % % % % % % % % % %
% % % % % % % % % % % % % % % % % % % % % % % % % % % % % % % % %
% % % % % % % % % % % % % % % % % % % % % % % % % % % % % % % % %
% % % % % % % % % % % % % % % % % % % % % % % % % % % % % % % % %
% % % % % % % % % % % % % % % % % % % % % % % % % % % % % % % % %
% % % % % % % % % % % % % % % % % % % % % % % % % % % % % % % % %
% % % % % % % % % % % % % % % % % % % % % % % % % % % % % % % % %
% % % % % % % % % % % % % % % % % % % % % % % % % % % % % % % % %
% % % % % % % % % % % % % % % % % % % % % % % % % % % % % % % % %
% % % % % % % % % % % % % % % % % % % % % % % % % % % % % %

\chapter*{Chapter I}
\ifaudio
\marginpar{
\href{http://ia802205.us.archive.org/34/items/war_and_peace_06_0808_librivox/war_and_peace_06_01_tolstoy.mp3}{Audio}} 
\fi

\lettrine[lines=2, loversize=0.3, lraise=0]{\initfamily P}{rince}
Andrew had spent two years continuously in the country.

All the plans Pierre had attempted on his estates---and
constantly changing from one thing to another had never
accomplished---were carried out by Prince Andrew without display
and without perceptible difficulty.

He had in the highest degree a practical tenacity which Pierre
lacked, and without fuss or strain on his part this set things
going.

On one of his estates the three hundred serfs were liberated and
became free agricultural laborers---this being one of the first
examples of the kind in Russia. On other estates the serfs'
compulsory labor was commuted for a quitrent. A trained midwife
was engaged for Bogucharovo at his expense, and a priest was paid
to teach reading and writing to the children of the peasants and
household serfs.

Prince Andrew spent half his time at Bald Hills with his father
and his son, who was still in the care of nurses. The other half
he spent in \emph{Bogucharovo Cloister}, as his father called
Prince Andrew's estate.  Despite the indifference to the affairs
of the world he had expressed to Pierre, he diligently followed
all that went on, received many books, and to his surprise
noticed that when he or his father had visitors from Petersburg,
the very vortex of life, these people lagged behind himself---who
never left the country---in knowledge of what was happening in
home and foreign affairs.

Besides being occupied with his estates and reading a great
variety of books, Prince Andrew was at this time busy with a
critical survey of our last two unfortunate campaigns, and with
drawing up a proposal for a reform of the army rules and
regulations.

In the spring of 1809 he went to visit the Ryazan estates which
had been inherited by his son, whose guardian he was.

Warmed by the spring sunshine he sat in the caleche looking at
the new grass, the first leaves on the birches, and the first
puffs of white spring clouds floating across the clear blue
sky. He was not thinking of anything, but looked absent-mindedly
and cheerfully from side to side.

They crossed the ferry where he had talked with Pierre the year
before.  They went through the muddy village, past threshing
floors and green fields of winter rye, downhill where snow still
lodged near the bridge, uphill where the clay had been liquefied
by the rain, past strips of stubble land and bushes touched with
green here and there, and into a birch forest growing on both
sides of the road. In the forest it was almost hot, no wind could
be felt. The birches with their sticky green leaves were
motionless, and lilac-colored flowers and the first blades of
green grass were pushing up and lifting last year's leaves. The
coarse evergreen color of the small fir trees scattered here and
there among the birches was an unpleasant reminder of winter. On
entering the forest the horses began to snort and sweated
visibly.

Peter the footman made some remark to the coachman; the latter
assented.  But apparently the coachman's sympathy was not enough
for Peter, and he turned on the box toward his master.

``How pleasant it is, your excellency!'' he said with a
respectful smile.

``What?''

``It's pleasant, your excellency!''

``What is he talking about?'' thought Prince Andrew. ``Oh, the
spring, I suppose,'' he thought as he turned round. ``Yes, really
everything is green already... How early! The birches and cherry
and alders too are coming out... But the oaks show no sign
yet. Ah, here is one oak!''

At the edge of the road stood an oak. Probably ten times the age
of the birches that formed the forest, it was ten times as thick
and twice as tall as they. It was an enormous tree, its girth
twice as great as a man could embrace, and evidently long ago
some of its branches had been broken off and its bark
scarred. With its huge ungainly limbs sprawling unsymmetrically,
and its gnarled hands and fingers, it stood an aged, stern, and
scornful monster among the smiling birch trees. Only the
dead-looking evergreen firs dotted about in the forest, and this
oak, refused to yield to the charm of spring or notice either the
spring or the sunshine.

``Spring, love, happiness!'' this oak seemed to say. ``Are you
not weary of that stupid, meaningless, constantly repeated fraud?
Always the same and always a fraud? There is no spring, no sun,
no happiness! Look at those cramped dead firs, ever the same, and
at me too, sticking out my broken and barked fingers just where
they have grown, whether from my back or my sides: as they have
grown so I stand, and I do not believe in your hopes and your
lies.''

As he passed through the forest Prince Andrew turned several
times to look at that oak, as if expecting something from
it. Under the oak, too, were flowers and grass, but it stood
among them scowling, rigid, misshapen, and grim as ever.

``Yes, the oak is right, a thousand times right,'' thought Prince
Andrew.  ``Let others---the young---yield afresh to that fraud,
but we know life, our life is finished!''

A whole sequence of new thoughts, hopeless but mournfully
pleasant, rose in his soul in connection with that tree. During
this journey he, as it were, considered his life afresh and
arrived at his old conclusion, restful in its hopelessness: that
it was not for him to begin anything anew---but that he must live
out his life, content to do no harm, and not disturbing himself
or desiring anything.

% % % % % % % % % % % % % % % % % % % % % % % % % % % % % % % % %
% % % % % % % % % % % % % % % % % % % % % % % % % % % % % % % % %
% % % % % % % % % % % % % % % % % % % % % % % % % % % % % % % % %
% % % % % % % % % % % % % % % % % % % % % % % % % % % % % % % % %
% % % % % % % % % % % % % % % % % % % % % % % % % % % % % % % % %
% % % % % % % % % % % % % % % % % % % % % % % % % % % % % % % % %
% % % % % % % % % % % % % % % % % % % % % % % % % % % % % % % % %
% % % % % % % % % % % % % % % % % % % % % % % % % % % % % % % % %
% % % % % % % % % % % % % % % % % % % % % % % % % % % % % % % % %
% % % % % % % % % % % % % % % % % % % % % % % % % % % % % % % % %
% % % % % % % % % % % % % % % % % % % % % % % % % % % % % % % % %
% % % % % % % % % % % % % % % % % % % % % % % % % % % % % %

\chapter*{Chapter II}
\ifaudio     
\marginpar{
\href{http://ia802205.us.archive.org/34/items/war_and_peace_06_0808_librivox/war_and_peace_06_02_tolstoy.mp3}{Audio}} 
\fi

\lettrine[lines=2, loversize=0.3, lraise=0]{\initfamily P}{rince}
Andrew had to see the Marshal of the Nobility for the
district in connection with the affairs of the Ryazan estate of
which he was trustee. This Marshal was Count Ilya Rostov, and in
the middle of May Prince Andrew went to visit him.

It was now hot spring weather. The whole forest was already
clothed in green. It was dusty and so hot that on passing near
water one longed to bathe.

Prince Andrew, depressed and preoccupied with the business about
which he had to speak to the Marshal, was driving up the avenue
in the grounds of the Rostovs' house at Otradnoe. He heard merry
girlish cries behind some trees on the right and saw a group of
girls running to cross the path of his caleche. Ahead of the rest
and nearer to him ran a dark-haired, remarkably slim, pretty girl
in a yellow chintz dress, with a white handkerchief on her head
from under which loose locks of hair escaped. The girl was
shouting something but, seeing that he was a stranger, ran back
laughing without looking at him.

Suddenly, he did not know why, he felt a pang. The day was so
beautiful, the sun so bright, everything around so gay, but that
slim pretty girl did not know, or wish to know, of his existence
and was contented and cheerful in her own separate---probably
foolish---but bright and happy life. ``What is she so glad about?
What is she thinking of? Not of the military regulations or of
the arrangement of the Ryazan serfs' quitrents. Of what is she
thinking? Why is she so happy?'' Prince Andrew asked himself with
instinctive curiosity.

In 1809 Count Ilya Rostov was living at Otradnoe just as he had
done in former years, that is, entertaining almost the whole
province with hunts, theatricals, dinners, and music. He was glad
to see Prince Andrew, as he was to see any new visitor, and
insisted on his staying the night.

During the dull day, in the course of which he was entertained by
his elderly hosts and by the more important of the visitors (the
old count's house was crowded on account of an approaching name
day), Prince Andrew repeatedly glanced at Natasha, gay and
laughing among the younger members of the company, and asked
himself each time, ``What is she thinking about? Why is she so
glad?''

That night, alone in new surroundings, he was long unable to
sleep. He read awhile and then put out his candle, but relit
it. It was hot in the room, the inside shutters of which were
closed. He was cross with the stupid old man (as he called
Rostov), who had made him stay by assuring him that some
necessary documents had not yet arrived from town, and he was
vexed with himself for having stayed.

He got up and went to the window to open it. As soon as he opened
the shutters the moonlight, as if it had long been watching for
this, burst into the room. He opened the casement. The night was
fresh, bright, and very still. Just before the window was a row
of pollard trees, looking black on one side and with a silvery
light on the other. Beneath the trees grew some kind of lush,
wet, bushy vegetation with silver-lit leaves and stems here and
there. Farther back beyond the dark trees a roof glittered with
dew, to the right was a leafy tree with brilliantly white trunk
and branches, and above it shone the moon, nearly at its full, in
a pale, almost starless, spring sky. Prince Andrew leaned his
elbows on the window ledge and his eyes rested on that sky.

His room was on the first floor. Those in the rooms above were
also awake. He heard female voices overhead.

``Just once more,'' said a girlish voice above him which Prince
Andrew recognized at once.

``But when are you coming to bed?'' replied another voice.

``I won't, I can't sleep, what's the use? Come now for the last
time.''

Two girlish voices sang a musical passage---the end of some song.

``Oh, how lovely! Now go to sleep, and there's an end of it.''

``You go to sleep, but I can't,'' said the first voice, coming
nearer to the window. She was evidently leaning right out, for
the rustle of her dress and even her breathing could be
heard. Everything was stone-still, like the moon and its light
and the shadows. Prince Andrew, too, dared not stir, for fear of
betraying his unintentional presence.

``Sonya! Sonya!'' he again heard the first speaker. ``Oh, how can
you sleep? Only look how glorious it is! Ah, how glorious! Do
wake up, Sonya!'' she said almost with tears in her
voice. ``There never, never was such a lovely night before!''

Sonya made some reluctant reply.

``Do just come and see what a moon!... Oh, how lovely! Come
here...  Darling, sweetheart, come here! There, you see? I feel
like sitting down on my heels, putting my arms round my knees
like this, straining tight, as tight as possible, and flying
away! Like this...''

``Take care, you'll fall out.''

He heard the sound of a scuffle and Sonya's disapproving voice:
``It's past one o'clock.''

``Oh, you only spoil things for me. All right, go, go!''

Again all was silent, but Prince Andrew knew she was still
sitting there. From time to time he heard a soft rustle and at
times a sigh.

``O God, O God! What does it mean?'' she suddenly exclaimed. ``To
bed then, if it must be!'' and she slammed the casement.

``For her I might as well not exist!'' thought Prince Andrew
while he listened to her voice, for some reason expecting yet
fearing that she might say something about him. ``There she is
again! As if it were on purpose,'' thought he.

In his soul there suddenly arose such an unexpected turmoil of
youthful thoughts and hopes, contrary to the whole tenor of his
life, that unable to explain his condition to himself he lay down
and fell asleep at once.

% % % % % % % % % % % % % % % % % % % % % % % % % % % % % % % % %
% % % % % % % % % % % % % % % % % % % % % % % % % % % % % % % % %
% % % % % % % % % % % % % % % % % % % % % % % % % % % % % % % % %
% % % % % % % % % % % % % % % % % % % % % % % % % % % % % % % % %
% % % % % % % % % % % % % % % % % % % % % % % % % % % % % % % % %
% % % % % % % % % % % % % % % % % % % % % % % % % % % % % % % % %
% % % % % % % % % % % % % % % % % % % % % % % % % % % % % % % % %
% % % % % % % % % % % % % % % % % % % % % % % % % % % % % % % % %
% % % % % % % % % % % % % % % % % % % % % % % % % % % % % % % % %
% % % % % % % % % % % % % % % % % % % % % % % % % % % % % % % % %
% % % % % % % % % % % % % % % % % % % % % % % % % % % % % % % % %
% % % % % % % % % % % % % % % % % % % % % % % % % % % % % %

\chapter*{Chapter III}
\ifaudio     
\marginpar{
\href{http://ia802205.us.archive.org/34/items/war_and_peace_06_0808_librivox/war_and_peace_06_03_tolstoy.mp3}{Audio}} 
\fi

\lettrine[lines=2, loversize=0.3, lraise=0]{\initfamily N}{ext}
morning, having taken leave of no one but the count, and not
waiting for the ladies to appear, Prince Andrew set off for home.

It was already the beginning of June when on his return journey
he drove into the birch forest where the gnarled old oak had made
so strange and memorable an impression on him. In the forest the
harness bells sounded yet more muffled than they had done six
weeks before, for now all was thick, shady, and dense, and the
young firs dotted about in the forest did not jar on the general
beauty but, lending themselves to the mood around, were
delicately green with fluffy young shoots.

The whole day had been hot. Somewhere a storm was gathering, but
only a small cloud had scattered some raindrops lightly,
sprinkling the road and the sappy leaves. The left side of the
forest was dark in the shade, the right side glittered in the
sunlight, wet and shiny and scarcely swayed by the
breeze. Everything was in blossom, the nightingales trilled, and
their voices reverberated now near, now far away.

``Yes, here in this forest was that oak with which I agreed,''
thought Prince Andrew. ``But where is it?'' he again wondered,
gazing at the left side of the road, and without recognizing it
he looked with admiration at the very oak he sought. The old oak,
quite transfigured, spreading out a canopy of sappy dark-green
foliage, stood rapt and slightly trembling in the rays of the
evening sun. Neither gnarled fingers nor old scars nor old doubts
and sorrows were any of them in evidence now.  Through the hard
century-old bark, even where there were no twigs, leaves had
sprouted such as one could hardly believe the old veteran could
have produced.

``Yes, it is the same oak,'' thought Prince Andrew, and all at
once he was seized by an unreasoning springtime feeling of joy
and renewal. All the best moments of his life suddenly rose to
his memory. Austerlitz with the lofty heavens, his wife's dead
reproachful face, Pierre at the ferry, that girl thrilled by the
beauty of the night, and that night itself and the moon,
and... all this rushed suddenly to his mind.

``No, life is not over at thirty-one!'' Prince Andrew suddenly
decided finally and decisively. ``It is not enough for me to know
what I have in me---everyone must know it: Pierre, and that young
girl who wanted to fly away into the sky, everyone must know me,
so that my life may not be lived for myself alone while others
live so apart from it, but so that it may be reflected in them
all, and they and I may live in harmony!''

On reaching home Prince Andrew decided to go to Petersburg that
autumn and found all sorts of reasons for this decision. A whole
series of sensible and logical considerations showing it to be
essential for him to go to Petersburg, and even to re-enter the
service, kept springing up in his mind. He could not now
understand how he could ever even have doubted the necessity of
taking an active share in life, just as a month before he had not
understood how the idea of leaving the quiet country could ever
enter his head. It now seemed clear to him that all his
experience of life must be senselessly wasted unless he applied
it to some kind of work and again played an active part in
life. He did not even remember how formerly, on the strength of
similar wretched logical arguments, it had seemed obvious that he
would be degrading himself if he now, after the lessons he had
had in life, allowed himself to believe in the possibility of
being useful and in the possibility of happiness or love. Now
reason suggested quite the opposite. After that journey to Ryazan
he found the country dull; his former pursuits no longer
interested him, and often when sitting alone in his study he got
up, went to the mirror, and gazed a long time at his own
face. Then he would turn away to the portrait of his dead Lise,
who with hair curled a la grecque looked tenderly and gaily at
him out of the gilt frame. She did not now say those former
terrible words to him, but looked simply, merrily, and
inquisitively at him. And Prince Andrew, crossing his arms behind
him, long paced the room, now frowning, now smiling, as he
reflected on those irrational, inexpressible thoughts, secret as
a crime, which altered his whole life and were connected with
Pierre, with fame, with the girl at the window, the oak, and
woman's beauty and love.  And if anyone came into his room at
such moments he was particularly cold, stern, and above all
unpleasantly logical.

``My dear,'' Princess Mary entering at such a moment would say,
``little Nicholas can't go out today, it's very cold.''

``If it were hot,'' Prince Andrew would reply at such times very
dryly to his sister, ``he could go out in his smock, but as it is
cold he must wear warm clothes, which were designed for that
purpose. That is what follows from the fact that it is cold; and
not that a child who needs fresh air should remain at home,'' he
would add with extreme logic, as if punishing someone for those
secret illogical emotions that stirred within him.

At such moments Princess Mary would think how intellectual work
dries men up.

% % % % % % % % % % % % % % % % % % % % % % % % % % % % % % % % %
% % % % % % % % % % % % % % % % % % % % % % % % % % % % % % % % %
% % % % % % % % % % % % % % % % % % % % % % % % % % % % % % % % %
% % % % % % % % % % % % % % % % % % % % % % % % % % % % % % % % %
% % % % % % % % % % % % % % % % % % % % % % % % % % % % % % % % %
% % % % % % % % % % % % % % % % % % % % % % % % % % % % % % % % %
% % % % % % % % % % % % % % % % % % % % % % % % % % % % % % % % %
% % % % % % % % % % % % % % % % % % % % % % % % % % % % % % % % %
% % % % % % % % % % % % % % % % % % % % % % % % % % % % % % % % %
% % % % % % % % % % % % % % % % % % % % % % % % % % % % % % % % %
% % % % % % % % % % % % % % % % % % % % % % % % % % % % % % % % %
% % % % % % % % % % % % % % % % % % % % % % % % % % % % % %

\chapter*{Chapter IV}
\ifaudio     
\marginpar{
\href{http://ia802205.us.archive.org/34/items/war_and_peace_06_0808_librivox/war_and_peace_06_04_tolstoy.mp3}{Audio}} 
\fi

\lettrine[lines=2, loversize=0.3, lraise=0]{\initfamily P}{rince}
Andrew arrived in Petersburg in August, 1809. It was the
time when the youthful Speranski was at the zenith of his fame
and his reforms were being pushed forward with the greatest
energy. That same August the Emperor was thrown from his caleche,
injured his leg, and remained three weeks at Peterhof, receiving
Speranski every day and no one else. At that time the two famous
decrees were being prepared that so agitated society---abolishing
court ranks and introducing examinations to qualify for the
grades of Collegiate Assessor and State Councilor---and not
merely these but a whole state constitution, intended to change
the existing order of government in Russia: legal,
administrative, and financial, from the Council of State down to
the district tribunals. Now those vague liberal dreams with which
the Emperor Alexander had ascended the throne, and which he had
tried to put into effect with the aid of his associates,
Czartoryski, Novosiltsev, Kochubey, and Strogonov---whom he
himself in jest had called his Comite de salut public---were
taking shape and being realized.

Now all these men were replaced by Speranski on the civil side,
and Arakcheev on the military. Soon after his arrival Prince
Andrew, as a gentleman of the chamber, presented himself at court
and at a levee. The Emperor, though he met him twice, did not
favor him with a single word.  It had always seemed to Prince
Andrew before that he was antipathetic to the Emperor and that
the latter disliked his face and personality generally, and in
the cold, repellent glance the Emperor gave him, he now found
further confirmation of this surmise. The courtiers explained the
Emperor's neglect of him by His Majesty's displeasure at
Bolkonski's not having served since 1805.

``I know myself that one cannot help one's sympathies and
antipathies,'' thought Prince Andrew, ``so it will not do to
present my proposal for the reform of the army regulations to the
Emperor personally, but the project will speak for itself.''

He mentioned what he had written to an old field marshal, a
friend of his father's. The field marshal made an appointment to
see him, received him graciously, and promised to inform the
Emperor. A few days later Prince Andrew received notice that he
was to go to see the Minister of War, Count Arakcheev.

On the appointed day Prince Andrew entered Count Arakcheev's
waiting room at nine in the morning.

He did not know Arakcheev personally, had never seen him, and all
he had heard of him inspired him with but little respect for the
man.

``He is Minister of War, a man trusted by the Emperor, and I need
not concern myself about his personal qualities: he has been
commissioned to consider my project, so he alone can get it
adopted,'' thought Prince Andrew as he waited among a number of
important and unimportant people in Count Arakcheev's waiting
room.

During his service, chiefly as an adjutant, Prince Andrew had
seen the anterooms of many important men, and the different types
of such rooms were well known to him. Count Arakcheev's anteroom
had quite a special character. The faces of the unimportant
people awaiting their turn for an audience showed embarrassment
and servility; the faces of those of higher rank expressed a
common feeling of awkwardness, covered by a mask of unconcern and
ridicule of themselves, their situation, and the person for whom
they were waiting. Some walked thoughtfully up and down, others
whispered and laughed. Prince Andrew heard the nickname ``Sila
Andreevich'' and the words, ``Uncle will give it to us hot,'' in
reference to Count Arakcheev. One general (an important
personage), evidently feeling offended at having to wait so long,
sat crossing and uncrossing his legs and smiling contemptuously
to himself.

But the moment the door opened one feeling alone appeared on all
faces---that of fear. Prince Andrew for the second time asked the
adjutant on duty to take in his name, but received an ironical
look and was told that his turn would come in due course. After
some others had been shown in and out of the minister's room by
the adjutant on duty, an officer who struck Prince Andrew by his
humiliated and frightened air was admitted at that terrible
door. This officer's audience lasted a long time. Then suddenly
the grating sound of a harsh voice was heard from the other side
of the door, and the officer---with pale face and trembling
lips---came out and passed through the waiting room, clutching
his head.

After this Prince Andrew was conducted to the door and the
officer on duty said in a whisper, ``To the right, at the
window.''

Prince Andrew entered a plain tidy room and saw at the table a
man of forty with a long waist, a long closely cropped head, deep
wrinkles, scowling brows above dull greenish-hazel eyes and an
overhanging red nose. Arakcheev turned his head toward him
without looking at him.

``What is your petition?'' asked Arakcheev.

``I am not petitioning, your excellency,'' returned Prince Andrew
quietly.

Arakcheev's eyes turned toward him.

``Sit down,'' said he. ``Prince Bolkonski?''

``I am not petitioning about anything. His Majesty the Emperor
has deigned to send your excellency a project submitted by
me...''

``You see, my dear sir, I have read your project,'' interrupted
Arakcheev, uttering only the first words amiably and then---again
without looking at Prince Andrew---relapsing gradually into a
tone of grumbling contempt.  ``You are proposing new military
laws? There are many laws but no one to carry out the old
ones. Nowadays everybody designs laws, it is easier writing than
doing.''

``I came at His Majesty the Emperor's wish to learn from your
excellency how you propose to deal with the memorandum I have
presented,'' said Prince Andrew politely.

``I have endorsed a resolution on your memorandum and sent it to
the committee. I do not approve of it,'' said Arakcheev, rising
and taking a paper from his writing table. ``Here!'' and he
handed it to Prince Andrew.

Across the paper was scrawled in pencil, without capital letters,
misspelled, and without punctuation: ``Unsoundly constructed
because resembles an imitation of the French military code and
from the Articles of War needlessly deviating.''

``To what committee has the memorandum been referred?'' inquired
Prince Andrew.

``To the Committee on Army Regulations, and I have recommended
that your honor should be appointed a member, but without a
salary.''

Prince Andrew smiled.

``I don't want one.''

``A member without salary,'' repeated Arakcheev. ``I have the
honor... Eh!  Call the next one! Who else is there?'' he shouted,
bowing to Prince Andrew.

% % % % % % % % % % % % % % % % % % % % % % % % % % % % % % % % %
% % % % % % % % % % % % % % % % % % % % % % % % % % % % % % % % %
% % % % % % % % % % % % % % % % % % % % % % % % % % % % % % % % %
% % % % % % % % % % % % % % % % % % % % % % % % % % % % % % % % %
% % % % % % % % % % % % % % % % % % % % % % % % % % % % % % % % %
% % % % % % % % % % % % % % % % % % % % % % % % % % % % % % % % %
% % % % % % % % % % % % % % % % % % % % % % % % % % % % % % % % %
% % % % % % % % % % % % % % % % % % % % % % % % % % % % % % % % %
% % % % % % % % % % % % % % % % % % % % % % % % % % % % % % % % %
% % % % % % % % % % % % % % % % % % % % % % % % % % % % % % % % %
% % % % % % % % % % % % % % % % % % % % % % % % % % % % % % % % %
% % % % % % % % % % % % % % % % % % % % % % % % % % % % % %

\chapter*{Chapter V}
\ifaudio     
\marginpar{
\href{http://ia802205.us.archive.org/34/items/war_and_peace_06_0808_librivox/war_and_peace_06_05_tolstoy.mp3}{Audio}} 
\fi

\lettrine[lines=2, loversize=0.3, lraise=0]{\initfamily W}{hile}
waiting for the announcement of his appointment to the
committee Prince Andrew looked up his former acquaintances,
particularly those he knew to be in power and whose aid he might
need. In Petersburg he now experienced the same feeling he had
had on the eve of a battle, when troubled by anxious curiosity
and irresistibly attracted to the ruling circles where the
future, on which the fate of millions depended, was being
shaped. From the irritation of the older men, the curiosity of
the uninitiated, the reserve of the initiated, the hurry and
preoccupation of everyone, and the innumerable committees and
commissions of whose existence he learned every day, he felt that
now, in 1809, here in Petersburg a vast civil conflict was in
preparation, the commander in chief of which was a mysterious
person he did not know, but who was supposed to be a man of
genius---Speranski. And this movement of reconstruction of which
Prince Andrew had a vague idea, and Speranski its chief promoter,
began to interest him so keenly that the question of the army
regulations quickly receded to a secondary place in his
consciousness.

Prince Andrew was most favorably placed to secure good reception
in the highest and most diverse Petersburg circles of the
day. The reforming party cordially welcomed and courted him, in
the first place because he was reputed to be clever and very well
read, and secondly because by liberating his serfs he had
obtained the reputation of being a liberal.  The party of the old
and dissatisfied, who censured the innovations, turned to him
expecting his sympathy in their disapproval of the reforms,
simply because he was the son of his father. The feminine society
world welcomed him gladly, because he was rich, distinguished, a
good match, and almost a newcomer, with a halo of romance on
account of his supposed death and the tragic loss of his
wife. Besides this the general opinion of all who had known him
previously was that he had greatly improved during these last
five years, having softened and grown more manly, lost his former
affectation, pride, and contemptuous irony, and acquired the
serenity that comes with years. People talked about him, were
interested in him, and wanted to meet him.

The day after his interview with Count Arakcheev, Prince Andrew
spent the evening at Count Kochubey's. He told the count of his
interview with Sila Andreevich (Kochubey spoke of Arakcheev by
that nickname with the same vague irony Prince Andrew had noticed
in the Minister of War's anteroom).

``Mon cher, even in this case you can't do without Michael
Mikhay\-lo\-vich Speranski. He manages everything. I'll speak to
him. He has promised to come this evening.''

``What has Speranski to do with the army regulations?'' asked
Prince Andrew.

Kochubey shook his head smilingly, as if surprised at Bolkonski's
simplicity.

``We were talking to him about you a few days ago,'' Kochubey
continued, ``and about your freed plowmen.''

``Oh, is it you, Prince, who have freed your serfs?'' said an old
man of Catherine's day, turning contemptuously toward Bolkonski.

``It was a small estate that brought in no profit,'' replied
Prince Andrew, trying to extenuate his action so as not to
irritate the old man uselessly.

``Afraid of being late...'' said the old man, looking at
Kochubey.

``There's one thing I don't understand,'' he continued. ``Who
will plow the land if they are set free? It is easy to write
laws, but difficult to rule... Just the same as now---I ask you,
Count---who will be heads of the departments when everybody has
to pass examinations?''

``Those who pass the examinations, I suppose,'' replied Kochubey,
crossing his legs and glancing round.

``Well, I have Pryanichnikov serving under me, a splendid man, a
priceless man, but he's sixty. Is he to go up for examination?''

``Yes, that's a difficulty, as education is not at all general,
but...''

Count Kochubey did not finish. He rose, took Prince Andrew by the
arm, and went to meet a tall, bald, fair man of about forty with
a large open forehead and a long face of unusual and peculiar
whiteness, who was just entering. The newcomer wore a blue
swallow-tail coat with a cross suspended from his neck and a star
on his left breast. It was Speranski.  Prince Andrew recognized
him at once, and felt a throb within him, as happens at critical
moments of life. Whether it was from respect, envy, or
anticipation, he did not know. Speranski's whole figure was of a
peculiar type that made him easily recognizable. In the society
in which Prince Andrew lived he had never seen anyone who
together with awkward and clumsy gestures possessed such calmness
and self-assurance; he had never seen so resolute yet gentle an
expression as that in those half-closed, rather humid eyes, or so
firm a smile that expressed nothing; nor had he heard such a
refined, smooth, soft voice; above all he had never seen such
delicate whiteness of face or hands---hands which were broad, but
very plump, soft, and white. Such whiteness and softness Prince
Andrew had only seen on the faces of soldiers who had been long
in hospital. This was Speranski, Secretary of State, reporter to
the Emperor and his companion at Erfurt, where he had more than
once met and talked with Napoleon.

Speranski did not shift his eyes from one face to another as
people involuntarily do on entering a large company and was in no
hurry to speak. He spoke slowly, with assurance that he would be
listened to, and he looked only at the person with whom he was
conversing.

Prince Andrew followed Speranski's every word and movement with
particular attention. As happens to some people, especially to
men who judge those near to them severely, he always on meeting
anyone new---especially anyone whom, like Speranski, he knew by
reputation---expected to discover in him the perfection of human
qualities.

Speranski told Kochubey he was sorry he had been unable to come
sooner as he had been detained at the palace. He did not say that
the Emperor had kept him, and Prince Andrew noticed this
affectation of modesty.  When Kochubey introduced Prince Andrew,
Speranski slowly turned his eyes to Bolkonski with his customary
smile and looked at him in silence.

``I am very glad to make your acquaintance. I had heard of you,
as everyone has,'' he said after a pause.

Kochubey said a few words about the reception Arakcheev had given
Bolkonski. Speranski smiled more markedly.

``The chairman of the Committee on Army Regulations is my good
friend Monsieur Magnitski,'' he said, fully articulating every
word and syllable, ``and if you like I can put you in touch with
him.'' He paused at the full stop. ``I hope you will find him
sympathetic and ready to co-operate in promoting all that is
reasonable.''

A circle soon formed round Speranski, and the old man who had
talked about his subordinate Pryanichnikov addressed a question
to him.

Prince Andrew without joining in the conversation watched every
movement of Speranski's: this man, not long since an
insignificant divinity student, who now, Bolkonski thought, held
in his hands---those plump white hands---the fate of
Russia. Prince Andrew was struck by the extraordinarily
disdainful composure with which Speranski answered the old
man. He appeared to address condescending words to him from an
immeasurable height. When the old man began to speak too loud,
Speranski smiled and said he could not judge of the advantage or
disadvantage of what pleased the sovereign.

Having talked for a little while in the general circle, Speranski
rose and coming up to Prince Andrew took him along to the other
end of the room. It was clear that he thought it necessary to
interest himself in Bolkonski.

``I had no chance to talk with you, Prince, during the animated
conversation in which that venerable gentleman involved me,'' he
said with a mildly contemptuous smile, as if intimating by that
smile that he and Prince Andrew understood the insignificance of
the people with whom he had just been talking. This flattered
Prince Andrew. ``I have known of you for a long time: first from
your action with regard to your serfs, a first example, of which
it is very desirable that there should be more imitators; and
secondly because you are one of those gentlemen of the chamber
who have not considered themselves offended by the new decree
concerning the ranks allotted to courtiers, which is causing so
much gossip and tittle-tattle.''

``No,'' said Prince Andrew, ``my father did not wish me to take
advantage of the privilege. I began the service from the lower
grade.''

``Your father, a man of the last century, evidently stands above
our contemporaries who so condemn this measure which merely
reestablishes natural justice.''

``I think, however, that these condemnations have some ground,''
returned Prince Andrew, trying to resist Speranski's influence,
of which he began to be conscious. He did not like to agree with
him in everything and felt a wish to contradict. Though he
usually spoke easily and well, he felt a difficulty in expressing
himself now while talking with Speranski. He was too much
absorbed in observing the famous man's personality.

``Grounds of personal ambition maybe,'' Speranski put in quietly.

``And of state interest to some extent,'' said Prince Andrew.

``What do you mean?'' asked Speranski quietly, lowering his eyes.

``I am an admirer of Montesquieu,'' replied Prince Andrew, ``and
his idea that le principe des monarchies est l'honneur me parait
incontestable.  Certains droits et privileges de la noblesse me
paraissent etre des moyens de soutenir ce sentiment.''
\footnote{``The principle of monarchies is honor seems to me
incontestable.  Certain rights and privileges for the aristocracy
appear to me a means of maintaining that sentiment.''}

The smile vanished from Speranski's white face, which was much
improved by the change. Probably Prince Andrew's thought
interested him.

``Si vous envisagez la question sous ce point de
vue,''\footnote{``If you regard the question from that point of
view.''} he began, pronouncing French with evident difficulty,
and speaking even slower than in Russian but quite calmly.

Speranski went on to say that honor, l'honneur, cannot be upheld
by privileges harmful to the service; that honor, l'honneur, is
either a negative concept of not doing what is blameworthy or it
is a source of emulation in pursuit of commendation and rewards,
which recognize it.  His arguments were concise, simple, and
clear.

``An institution upholding honor, the source of emulation, is one
similar to the Legion d'honneur of the great Emperor Napoleon,
not harmful but helpful to the success of the service, but not a
class or court privilege.''

``I do not dispute that, but it cannot be denied that court
privileges have attained the same end,'' returned Prince
Andrew. ``Every courtier considers himself bound to maintain his
position worthily.''

``Yet you do not care to avail yourself of the privilege,
Prince,'' said Speranski, indicating by a smile that he wished to
finish amiably an argument which was embarrassing for his
companion. ``If you will do me the honor of calling on me on
Wednesday,'' he added, ``I will, after talking with Magnitski,
let you know what may interest you, and shall also have the
pleasure of a more detailed chat with you.''

Closing his eyes, he bowed a la francaise, without taking leave,
and trying to attract as little attention as possible, he left
the room.

% % % % % % % % % % % % % % % % % % % % % % % % % % % % % % % % %
% % % % % % % % % % % % % % % % % % % % % % % % % % % % % % % % %
% % % % % % % % % % % % % % % % % % % % % % % % % % % % % % % % %
% % % % % % % % % % % % % % % % % % % % % % % % % % % % % % % % %
% % % % % % % % % % % % % % % % % % % % % % % % % % % % % % % % %
% % % % % % % % % % % % % % % % % % % % % % % % % % % % % % % % %
% % % % % % % % % % % % % % % % % % % % % % % % % % % % % % % % %
% % % % % % % % % % % % % % % % % % % % % % % % % % % % % % % % %
% % % % % % % % % % % % % % % % % % % % % % % % % % % % % % % % %
% % % % % % % % % % % % % % % % % % % % % % % % % % % % % % % % %
% % % % % % % % % % % % % % % % % % % % % % % % % % % % % % % % %
% % % % % % % % % % % % % % % % % % % % % % % % % % % % % %

\chapter*{Chapter VI}
\ifaudio     
\marginpar{
\href{http://ia802205.us.archive.org/34/items/war_and_peace_06_0808_librivox/war_and_peace_06_06_tolstoy.mp3}{Audio}} 
\fi

\lettrine[lines=2, loversize=0.3, lraise=0]{\initfamily D}{uring}
the first weeks of his stay in Petersburg Prince Andrew
felt the whole trend of thought he had formed during his life of
seclusion quite overshadowed by the trifling cares that engrossed
him in that city.

On returning home in the evening he would jot down in his
notebook four or five necessary calls or appointments for certain
hours. The mechanism of life, the arrangement of the day so as to
be in time everywhere, absorbed the greater part of his vital
energy. He did nothing, did not even think or find time to think,
but only talked, and talked successfully, of what he had thought
while in the country.

He sometimes noticed with dissatisfaction that he repeated the
same remark on the same day in different circles. But he was so
busy for whole days together that he had no time to notice that
he was thinking of nothing.

As he had done on their first meeting at Kochubey's, Speranski
produced a strong impression on Prince Andrew on the Wednesday,
when he received him tête-à-tête at his own house and talked to
him long and confidentially.

To Bolkonski so many people appeared contemptible and
insignificant creatures, and he so longed to find in someone the
living ideal of that perfection toward which he strove, that he
readily believed that in Speranski he had found this ideal of a
perfectly rational and virtuous man. Had Speranski sprung from
the same class as himself and possessed the same breeding and
traditions, Bolkonski would soon have discovered his weak, human,
unheroic sides; but as it was, Speranski's strange and logical
turn of mind inspired him with respect all the more because he
did not quite understand him. Moreover, Speranski, either because
he appreciated the other's capacity or because he considered it
necessary to win him to his side, showed off his dispassionate
calm reasonableness before Prince Andrew and flattered him with
that subtle flattery which goes hand in hand with self-assurance
and consists in a tacit assumption that one's companion is the
only man besides oneself capable of understanding the folly of
the rest of mankind and the reasonableness and profundity of
one's own ideas.

During their long conversation on Wednesday evening, Speranski
more than once remarked: ``We regard everything that is above the
common level of rooted custom...'' or, with a smile: ``But we
want the wolves to be fed and the sheep to be safe...'' or:
``They cannot understand this...'' and all in a way that seemed
to say: ``We, you and I, understand what they are and who we
are.''

This first long conversation with Speranski only strengthened in
Prince Andrew the feeling he had experienced toward him at their
first meeting.  He saw in him a remarkable, clear-thinking man of
vast intellect who by his energy and persistence had attained
power, which he was using solely for the welfare of Russia. In
Prince Andrew's eyes Speranski was the man he would himself have
wished to be---one who explained all the facts of life
reasonably, considered important only what was rational, and was
capable of applying the standard of reason to
everything. Everything seemed so simple and clear in Speranski's
exposition that Prince Andrew involuntarily agreed with him about
everything. If he replied and argued, it was only because he
wished to maintain his independence and not submit to Speranski's
opinions entirely. Everything was right and everything was as it
should be: only one thing disconcerted Prince Andrew. This was
Speranski's cold, mirrorlike look, which did not allow one to
penetrate to his soul, and his delicate white hands, which Prince
Andrew involuntarily watched as one does watch the hands of those
who possess power. This mirrorlike gaze and those delicate hands
irritated Prince Andrew, he knew not why. He was unpleasantly
struck, too, by the excessive contempt for others that he
observed in Speranski, and by the diversity of lines of argument
he used to support his opinions. He made use of every kind of
mental device, except analogy, and passed too boldly, it seemed
to Prince Andrew, from one to another. Now he would take up the
position of a practical man and condemn dreamers; now that of a
satirist, and laugh ironically at his opponents; now grow
severely logical, or suddenly rise to the realm of
metaphysics. (This last resource was one he very frequently
employed.) He would transfer a question to metaphysical heights,
pass on to definitions of space, time, and thought, and, having
deduced the refutation he needed, would again descend to the
level of the original discussion.

In general the trait of Speranski's mentality which struck Prince
Andrew most was his absolute and unshakable belief in the power
and authority of reason. It was evident that the thought could
never occur to him which to Prince Andrew seemed so natural,
namely, that it is after all impossible to express all one
thinks; and that he had never felt the doubt, ``Is not all I
think and believe nonsense?'' And it was just this peculiarity of
Speranski's mind that particularly attracted Prince Andrew.

During the first period of their acquaintance Bolkonski felt a
passionate admiration for him similar to that which he had once
felt for Bonaparte. The fact that Speranski was the son of a
village priest, and that stupid people might meanly despise him
on account of his humble origin (as in fact many did), caused
Prince Andrew to cherish his sentiment for him the more, and
unconsciously to strengthen it.

On that first evening Bolkonski spent with him, having mentioned
the Commission for the Revision of the Code of Laws, Speranski
told him sarcastically that the Commission had existed for a
hundred and fifty years, had cost millions, and had done nothing
except that Rosenkampf had stuck labels on the corresponding
paragraphs of the different codes.

``And that is all the state has for the millions it has spent,''
said he.  ``We want to give the Senate new juridical powers, but
we have no laws.  That is why it is a sin for men like you,
Prince, not to serve in these times!''

Prince Andrew said that for that work an education in
jurisprudence was needed which he did not possess.

``But nobody possesses it, so what would you have? It is a
vicious circle from which we must break a way out.''

A week later Prince Andrew was a member of the Committee on Army
Regulations and---what he had not at all expected---was chairman
of a section of the committee for the revision of the laws. At
Speranski's request he took the first part of the Civil Code that
was being drawn up and, with the aid of the Code Napoleon and the
Institutes of Justinian, he worked at formulating the section on
Personal Rights.

% % % % % % % % % % % % % % % % % % % % % % % % % % % % % % % % %
% % % % % % % % % % % % % % % % % % % % % % % % % % % % % % % % %
% % % % % % % % % % % % % % % % % % % % % % % % % % % % % % % % %
% % % % % % % % % % % % % % % % % % % % % % % % % % % % % % % % %
% % % % % % % % % % % % % % % % % % % % % % % % % % % % % % % % %
% % % % % % % % % % % % % % % % % % % % % % % % % % % % % % % % %
% % % % % % % % % % % % % % % % % % % % % % % % % % % % % % % % %
% % % % % % % % % % % % % % % % % % % % % % % % % % % % % % % % %
% % % % % % % % % % % % % % % % % % % % % % % % % % % % % % % % %
% % % % % % % % % % % % % % % % % % % % % % % % % % % % % % % % %
% % % % % % % % % % % % % % % % % % % % % % % % % % % % % % % % %
% % % % % % % % % % % % % % % % % % % % % % % % % % % % % %

\chapter*{Chapter VII}
\ifaudio     
\marginpar{
\href{http://ia802205.us.archive.org/34/items/war_and_peace_06_0808_librivox/war_and_peace_06_07_tolstoy.mp3}{Audio}} 
\fi

\lettrine[lines=2, loversize=0.3, lraise=0]{\initfamily N}{early}
two years before this, in 1808, Pierre on returning to
Petersburg after visiting his estates had involuntarily found
himself in a leading position among the Petersburg Freemasons. He
arranged dining and funeral lodge meetings, enrolled new members,
and busied himself uniting various lodges and acquiring authentic
charters. He gave money for the erection of temples and
supplemented as far as he could the collection of alms, in regard
to which the majority of members were stingy and irregular. He
supported almost singlehanded a poorhouse the order had founded
in Petersburg.

His life meanwhile continued as before, with the same
infatuations and dissipations. He liked to dine and drink well,
and though he considered it immoral and humiliating could not
resist the temptations of the bachelor circles in which he moved.

Amid the turmoil of his activities and distractions, however,
Pierre at the end of a year began to feel that the more firmly he
tried to rest upon it, the more masonic ground on which he stood
gave way under him.  At the same time he felt that the deeper the
ground sank under him the closer bound he involuntarily became to
the order. When he had joined the Freemasons he had experienced
the feeling of one who confidently steps onto the smooth surface
of a bog. When he put his foot down it sank in. To make quite
sure of the firmness of the ground, he put his other foot down
and sank deeper still, became stuck in it, and involuntarily
waded knee-deep in the bog.

Joseph Alexeevich was not in Petersburg---he had of late stood
aside from the affairs of the Petersburg lodges, and lived almost
entirely in Moscow. All the members of the lodges were men Pierre
knew in ordinary life, and it was difficult for him to regard
them merely as Brothers in Freemasonry and not as Prince B. or
Ivan Vasilevich D., whom he knew in society mostly as weak and
insignificant men. Under the masonic aprons and insignia he saw
the uniforms and decorations at which they aimed in ordinary
life. Often after collecting alms, and reckoning up twenty to
thirty rubles received for the most part in promises from a dozen
members, of whom half were as well able to pay as himself, Pierre
remembered the masonic vow in which each Brother promised to
devote all his belongings to his neighbor, and doubts on which he
tried not to dwell arose in his soul.

He divided the Brothers he knew into four categories. In the
first he put those who did not take an active part in the affairs
of the lodges or in human affairs, but were exclusively occupied
with the mystical science of the order: with questions of the
threefold designation of God, the three primordial
elements---sulphur, mercury, and salt---or the meaning of the
square and all the various figures of the temple of
Solomon. Pierre respected this class of Brothers to which the
elder ones chiefly belonged, including, Pierre thought, Joseph
Alexeevich himself, but he did not share their interests. His
heart was not in the mystical aspect of Freemasonry.

In the second category Pierre reckoned himself and others like
him, seeking and vacillating, who had not yet found in
Freemasonry a straight and comprehensible path, but hoped to do
so.

In the third category he included those Brothers (the majority)
who saw nothing in Freemasonry but the external forms and
ceremonies, and prized the strict performance of these forms
without troubling about their purport or significance. Such were
Willarski and even the Grand Master of the principal lodge.

Finally, to the fourth category also a great many Brothers
belonged, particularly those who had lately joined. These
according to Pierre's observations were men who had no belief in
anything, nor desire for anything, but joined the Freemasons
merely to associate with the wealthy young Brothers who were
influential through their connections or rank, and of whom there
were very many in the lodge.

Pierre began to feel dissatisfied with what he was
doing. Freemasonry, at any rate as he saw it here, sometimes
seemed to him based merely on externals. He did not think of
doubting Freemasonry itself, but suspected that Russian Masonry
had taken a wrong path and deviated from its original
principles. And so toward the end of the year he went abroad to
be initiated into the higher secrets of the order.

In the summer of 1809 Pierre returned to Petersburg. Our
Freemasons knew from correspondence with those abroad that
Bezukhov had obtained the confidence of many highly placed
persons, had been initiated into many mysteries, had been raised
to a higher grade, and was bringing back with him much that might
conduce to the advantage of the masonic cause in Russia. The
Petersburg Freemasons all came to see him, tried to ingratiate
themselves with him, and it seemed to them all that he was
preparing something for them and concealing it.

A solemn meeting of the lodge of the second degree was convened,
at which Pierre promised to communicate to the Petersburg
Brothers what he had to deliver to them from the highest leaders
of their order. The meeting was a full one. After the usual
ceremonies Pierre rose and began his address.

``Dear Brothers,'' he began, blushing and stammering, with a
written speech in his hand, ``it is not sufficient to observe our
mysteries in the seclusion of our lodge---we must act---act! We
are drowsing, but we must act.'' Pierre raised his notebook and
began to read.

``For the dissemination of pure truth and to secure the triumph
of virtue,'' he read, ``we must cleanse men from prejudice,
diffuse principles in harmony with the spirit of the times,
undertake the education of the young, unite ourselves in
indissoluble bonds with the wisest men, boldly yet prudently
overcome superstitions, infidelity, and folly, and form of those
devoted to us a body linked together by unity of purpose and
possessed of authority and power.

''To attain this end we must secure a preponderance of virtue
over vice and must endeavor to secure that the honest man may,
even in this world, receive a lasting reward for his virtue. But
in these great endeavors we are gravely hampered by the political
institutions of today. What is to be done in these circumstances?
To favor revolutions, overthrow everything, repel force by
force?... No! We are very far from that.  Every violent reform
deserves censure, for it quite fails to remedy evil while men
remain what they are, and also because wisdom needs no violence.

``The whole plan of our order should be based on the idea of
preparing men of firmness and virtue bound together by unity of
conviction---aiming at the punishment of vice and folly, and
patronizing talent and virtue: raising worthy men from the dust
and attaching them to our Brotherhood.  Only then will our order
have the power unobtrusively to bind the hands of the protectors
of disorder and to control them without their being aware of
it. In a word, we must found a form of government holding
universal sway, which should be diffused over the whole world
without destroying the bonds of citizenship, and beside which all
other governments can continue in their customary course and do
everything except what impedes the great aim of our order, which
is to obtain for virtue the victory over vice. This aim was that
of Christianity itself.  It taught men to be wise and good and
for their own benefit to follow the example and instruction of
the best and wisest men.

''At that time, when everything was plunged in darkness,
preaching alone was of course sufficient. The novelty of Truth
endowed her with special strength, but now we need much more
powerful methods. It is now necessary that man, governed by his
senses, should find in virtue a charm palpable to those
senses. It is impossible to eradicate the passions; but we must
strive to direct them to a noble aim, and it is therefore
necessary that everyone should be able to satisfy his passions
within the limits of virtue. Our order should provide means to
that end.

``As soon as we have a certain number of worthy men in every
state, each of them again training two others and all being
closely united, everything will be possible for our order, which
has already in secret accomplished much for the welfare of
mankind.''

This speech not only made a strong impression, but created
excitement in the lodge. The majority of the Brothers, seeing in
it dangerous designs of Illuminism,\footnote{The Illuminati
sought to substitute republican for monarchical institutions.}
met it with a coldness that surprised Pierre. The Grand Master
began answering him, and Pierre began developing his views with
more and more warmth. It was long since there had been so stormy
a meeting. Parties were formed, some accusing Pierre of
Illuminism, others supporting him. At that meeting he was struck
for the first time by the endless variety of men's minds, which
prevents a truth from ever presenting itself identically to two
persons. Even those members who seemed to be on his side
understood him in their own way with limitations and alterations
he could not agree to, as what he always wanted most was to
convey his thought to others just as he himself understood it.

At the end of the meeting the Grand Master with irony and
ill-will reproved Bezukhov for his vehemence and said it was not
love of virtue alone, but also a love of strife that had moved
him in the dispute.  Pierre did not answer him and asked briefly
whether his proposal would be accepted. He was told that it would
not, and without waiting for the usual formalities he left the
lodge and went home.

% % % % % % % % % % % % % % % % % % % % % % % % % % % % % % % % %
% % % % % % % % % % % % % % % % % % % % % % % % % % % % % % % % %
% % % % % % % % % % % % % % % % % % % % % % % % % % % % % % % % %
% % % % % % % % % % % % % % % % % % % % % % % % % % % % % % % % %
% % % % % % % % % % % % % % % % % % % % % % % % % % % % % % % % %
% % % % % % % % % % % % % % % % % % % % % % % % % % % % % % % % %
% % % % % % % % % % % % % % % % % % % % % % % % % % % % % % % % %
% % % % % % % % % % % % % % % % % % % % % % % % % % % % % % % % %
% % % % % % % % % % % % % % % % % % % % % % % % % % % % % % % % %
% % % % % % % % % % % % % % % % % % % % % % % % % % % % % % % % %
% % % % % % % % % % % % % % % % % % % % % % % % % % % % % % % % %
% % % % % % % % % % % % % % % % % % % % % % % % % % % % % %

\chapter*{Chapter VIII}
\ifaudio     
\marginpar{
\href{http://ia802205.us.archive.org/34/items/war_and_peace_06_0808_librivox/war_and_peace_06_08_tolstoy.mp3}{Audio}} 
\fi

\lettrine[lines=2, loversize=0.3, lraise=0]{\initfamily A}{gain}
Pierre was overtaken by the depression he so dreaded. For
three days after the delivery of his speech at the lodge he lay
on a sofa at home receiving no one and going nowhere.

It was just then that he received a letter from his wife, who
implored him to see her, telling him how grieved she was about
him and how she wished to devote her whole life to him.

At the end of the letter she informed him that in a few days she
would return to Petersburg from abroad.

Following this letter one of the masonic Brothers whom Pierre
respected less than the others forced his way in to see him and,
turning the conversation upon Pierre's matrimonial affairs, by
way of fraternal advice expressed the opinion that his severity
to his wife was wrong and that he was neglecting one of the first
rules of Freemasonry by not forgiving the penitent.

At the same time his mother-in-law, Prince Vasili's wife, sent to
him imploring him to come if only for a few minutes to discuss a
most important matter. Pierre saw that there was a conspiracy
against him and that they wanted to reunite him with his wife,
and in the mood he then was, this was not even unpleasant to
him. Nothing mattered to him.  Nothing in life seemed to him of
much importance, and under the influence of the depression that
possessed him he valued neither his liberty nor his resolution to
punish his wife.

``No one is right and no one is to blame; so she too is not to
blame,'' he thought.

If he did not at once give his consent to a reunion with his
wife, it was only because in his state of depression he did not
feel able to take any step. Had his wife come to him, he would
not have turned her away.  Compared to what preoccupied him, was
it not a matter of indifference whether he lived with his wife or
not?

Without replying either to his wife or his mother-in-law, Pierre
late one night prepared for a journey and started for Moscow to
see Joseph Alexeevich. This is what he noted in his diary:

\begin{quote} \calli
Moscow, 17th November

I have just returned from my benefactor, and hasten to write down
what I have experienced. Joseph Alexeevich is living poorly and
has for three years been suffering from a painful disease of the
bladder. No one has ever heard him utter a groan or a word of
complaint. From morning till late at night, except when he eats
his very plain food, he is working at science. He received me
graciously and made me sit down on the bed on which he lay. I
made the sign of the Knights of the East and of Jerusalem, and he
responded in the same manner, asking me with a mild smile what I
had learned and gained in the Prussian and Scottish lodges.  I
told him everything as best I could, and told him what I had
proposed to our Petersburg lodge, of the bad reception I had
encountered, and of my rupture with the Brothers. Joseph
Alexeevich, having remained silent and thoughtful for a good
while, told me his view of the matter, which at once lit up for
me my whole past and the future path I should follow.  He
surprised me by asking whether I remembered the threefold aim of
the order: (1) The preservation and study of the mystery. (2) The
purification and reformation of oneself for its reception, and
(3) The improvement of the human race by striving for such
purification. Which is the principal aim of these three?
Certainly self-reformation and self-purification. Only to this
aim can we always strive independently of circumstances. But at
the same time just this aim demands the greatest efforts of us;
and so, led astray by pride, losing sight of this aim, we occupy
ourselves either with the mystery which in our impurity we are
unworthy to receive, or seek the reformation of the human race
while ourselves setting an example of baseness and
profligacy. Illuminism is not a pure doctrine, just because it is
attracted by social activity and puffed up by pride. On this
ground Joseph Alexeevich condemned my speech and my whole
activity, and in the depth of my soul I agreed with him. Talking
of my family affairs he said to me, ``the chief duty of a true
Mason, as I have told you, lies in perfecting himself. We often
think that by removing all the difficulties of our life we shall
more quickly reach our aim, but on the contrary, my dear sir, it
is only in the midst of worldly cares that we can attain our
three chief aims:
\begin{enumerate}
\item Self-knowledge---for man can only know himself by
  comparison,
\item Self-perfecting, which can only be attained by conflict,
  and
\item The attainment of the chief virtue---love of death.
\end{enumerate}
Only the vicissitudes of life can show us its vanity and develop
our innate love of death or of rebirth to a new life.'' These
words are all the more remarkable because, in spite of his great
physical sufferings, Joseph Alexeevich is never weary of life
though he loves death, for which---in spite of the purity and
loftiness of his inner man---he does not yet feel himself
sufficiently prepared. My benefactor then explained to me fully
the meaning of the Great Square of creation and pointed out to me
that the numbers three and seven are the basis of everything. He
advised me not to avoid intercourse with the Petersburg Brothers,
but to take up only second-grade posts in the lodge, to try,
while diverting the Brothers from pride, to turn them toward the
true path self-knowledge and self-perfecting. Besides this he
advised me for myself personally above all to keep a watch over
myself, and to that end he gave me a notebook, the one I am now
writing in and in which I will in future note down all my
actions.
\end{quote}

\begin{quote} \calli
Petersburg, 23rd November

I am again living with my wife. My mother-in-law came to me in
tears and said that Helene was here and that she implored me to
hear her; that she was innocent and unhappy at my desertion, and
much more. I knew that if I once let myself see her I should not
have strength to go on refusing what she wanted. In my perplexity
I did not know whose aid and advice to seek. Had my benefactor
been here he would have told me what to do. I went to my room and
reread Joseph Alexeevich's letters and recalled my conversations
with him, and deduced from it all that I ought not to refuse a
supplicant, and ought to reach a helping hand to
everyone---especially to one so closely bound to me---and that I
must bear my cross.  But if I forgive her for the sake of doing
right, then let union with her have only a spiritual aim. That is
what I decided, and what I wrote to Joseph Alexeevich. I told my
wife that I begged her to forget the past, to forgive me whatever
wrong I may have done her, and that I had nothing to forgive. It
gave me joy to tell her this. She need not know how hard it was
for me to see her again. I have settled on the upper floor of
this big house and am experiencing a happy feeling of
regeneration.
\end{quote}

% % % % % % % % % % % % % % % % % % % % % % % % % % % % % % % % %
% % % % % % % % % % % % % % % % % % % % % % % % % % % % % % % % %
% % % % % % % % % % % % % % % % % % % % % % % % % % % % % % % % %
% % % % % % % % % % % % % % % % % % % % % % % % % % % % % % % % %
% % % % % % % % % % % % % % % % % % % % % % % % % % % % % % % % %
% % % % % % % % % % % % % % % % % % % % % % % % % % % % % % % % %
% % % % % % % % % % % % % % % % % % % % % % % % % % % % % % % % %
% % % % % % % % % % % % % % % % % % % % % % % % % % % % % % % % %
% % % % % % % % % % % % % % % % % % % % % % % % % % % % % % % % %
% % % % % % % % % % % % % % % % % % % % % % % % % % % % % % % % %
% % % % % % % % % % % % % % % % % % % % % % % % % % % % % % % % %
% % % % % % % % % % % % % % % % % % % % % % % % % % % % % %

\chapter*{Chapter IX}
\ifaudio     
\marginpar{
\href{http://ia802205.us.archive.org/34/items/war_and_peace_06_0808_librivox/war_and_peace_06_09_tolstoy.mp3}{Audio}} 
\fi

\lettrine[lines=2, loversize=0.3, lraise=0]{\initfamily A}{t}
that time, as always happens, the highest society that met at
court and at the grand balls was divided into several circles,
each with its own particular tone. The largest of these was the
French circle of the Napoleonic alliance, the circle of Count
Rumyantsev and Caulaincourt. In this group Helene, as soon as she
had settled in Petersburg with her husband, took a very prominent
place. She was visited by the members of the French embassy and
by many belonging to that circle and noted for their intellect
and polished manners.

Helene had been at Erfurt during the famous meeting of the
Emperors and had brought from there these connections with the
Napoleonic notabilities. At Erfurt her success had been
brilliant. Napoleon himself had noticed her in the theater and
said of her: ``C'est un superbe animal.''\footnote{``That's a
superb animal.''} Her success as a beautiful and elegant woman
did not surprise Pierre, for she had become even handsomer than
before. What did surprise him was that during these last two
years his wife had succeeded in gaining the reputation ``d' une
femme charmante, aussi spirituelle que belle.''\footnote{``Of a
charming woman, as witty as she is lovely.''}  The distinguished
Prince de Ligne wrote her eight-page letters. Bilibin saved up
his epigrams to produce them in Countess Bezukhova's presence. To
be received in the Countess Bezukhova's salon was regarded as a
diploma of intellect. Young men read books before attending
Helene's evenings, to have something to say in her salon, and
secretaries of the embassy, and even ambassadors, confided
diplomatic secrets to her, so that in a way Helene was a
power. Pierre, who knew she was very stupid, sometimes attended,
with a strange feeling of perplexity and fear, her evenings and
dinner parties, where politics, poetry, and philosophy were
discussed. At these parties his feelings were like those of a
conjuror who always expects his trick to be found out at any
moment. But whether because stupidity was just what was needed to
run such a salon, or because those who were deceived found
pleasure in the deception, at any rate it remained unexposed and
Helene Bezukhova's reputation as a lovely and clever woman became
so firmly established that she could say the emptiest and
stupidest things and everybody would go into raptures over every
word of hers and look for a profound meaning in it of which she
herself had no conception.

Pierre was just the husband needed for a brilliant society
woman. He was that absent-minded crank, a grand seigneur husband
who was in no one's way, and far from spoiling the high tone and
general impression of the drawing room, he served, by the
contrast he presented to her, as an advantageous background to
his elegant and tactful wife. Pierre during the last two years,
as a result of his continual absorption in abstract interests and
his sincere contempt for all else, had acquired in his wife's
circle, which did not interest him, that air of unconcern,
indifference, and benevolence toward all, which cannot be
acquired artificially and therefore inspires involuntary
respect. He entered his wife's drawing room as one enters a
theater, was acquainted with everybody, equally pleased to see
everyone, and equally indifferent to them all. Sometimes he
joined in a conversation which interested him and, regardless of
whether any \emph{gentlemen of the embassy} were present or not,
lispingly expressed his views, which were sometimes not at all in
accord with the accepted tone of the moment. But the general
opinion concerning the queer husband of \emph{the most distinguished
woman in Petersburg} was so well established that no one took
his freaks seriously.

Among the many young men who frequented her house every day,
Boris Drubetskoy, who had already achieved great success in the
service, was the most intimate friend of the Bezukhov household
since Helene's return from Erfurt. Helene spoke of him as
\emph{mon page} and treated him like a child. Her smile for him
was the same as for everybody, but sometimes that smile made
Pierre uncomfortable. Toward him Boris behaved with a
particularly dignified and sad deference. This shade of deference
also disturbed Pierre. He had suffered so painfully three years
before from the mortification to which his wife had subjected him
that he now protected himself from the danger of its repetition,
first by not being a husband to his wife, and secondly by not
allowing himself to suspect.

``No, now that she has become a bluestocking she has finally
renounced her former infatuations,'' he told himself. ``There has
never been an instance of a bluestocking being carried away by
affairs of the heart''---a statement which, though gathered from
an unknown source, he believed implicitly. Yet strange to say
Boris' presence in his wife's drawing room (and he was almost
always there) had a physical effect upon Pierre; it constricted
his limbs and destroyed the unconsciousness and freedom of his
movements.

``What a strange antipathy,'' thought Pierre, ``yet I used to
like him very much.''

In the eyes of the world Pierre was a great gentleman, the rather
blind and absurd husband of a distinguished wife, a clever crank
who did nothing but harmed nobody and was a first-rate,
good-natured fellow. But a complex and difficult process of
internal development was taking place all this time in Pierre's
soul, revealing much to him and causing him many spiritual doubts
and joys.

% % % % % % % % % % % % % % % % % % % % % % % % % % % % % % % % %
% % % % % % % % % % % % % % % % % % % % % % % % % % % % % % % % %
% % % % % % % % % % % % % % % % % % % % % % % % % % % % % % % % %
% % % % % % % % % % % % % % % % % % % % % % % % % % % % % % % % %
% % % % % % % % % % % % % % % % % % % % % % % % % % % % % % % % %
% % % % % % % % % % % % % % % % % % % % % % % % % % % % % % % % %
% % % % % % % % % % % % % % % % % % % % % % % % % % % % % % % % %
% % % % % % % % % % % % % % % % % % % % % % % % % % % % % % % % %
% % % % % % % % % % % % % % % % % % % % % % % % % % % % % % % % %
% % % % % % % % % % % % % % % % % % % % % % % % % % % % % % % % %
% % % % % % % % % % % % % % % % % % % % % % % % % % % % % % % % %
% % % % % % % % % % % % % % % % % % % % % % % % % % % % % %

\chapter*{Chapter X}
\ifaudio     
\marginpar{
\href{http://ia802205.us.archive.org/34/items/war_and_peace_06_0808_librivox/war_and_peace_06_10_tolstoy.mp3}{Audio}} 
\fi

\lettrine[lines=2, loversize=0.3, lraise=0]{\initfamily P}{ierre}
went on with his diary, and this is what he wrote in it
during that time:

\begin{quote} \calli

24th November

Got up at eight, read the Scriptures, then went to my duties. (By
Joseph Alexeevich's advice Pierre had entered the service of the
state and served on one of the committees.) Returned home for
dinner and dined alone---the countess had many visitors I do not
like. I ate and drank moderately and after dinner copied out some
passages for the Brothers.  In the evening I went down to the
countess and told a funny story about B., and only remembered
that I ought not to have done so when everybody laughed loudly at
it.

I am going to bed with a happy and tranquil mind. Great God, help
me to walk in Thy paths,
\begin{enumerate}
\item to conquer anger by calmness and deliberation,
\item to vanquish lust by self-restraint and repulsion,
\item to withdraw from worldliness, but not avoid
\begin{enumerate}
\item the service of the state,
\item family duties,
\item relations with my friends, and the management of my
affairs.
\end{enumerate}
\end{enumerate}

27th November

I got up late. On waking I lay long in bed yielding to sloth. O
God, help and strengthen me that I may walk in Thy ways! Read the
Scriptures, but without proper feeling. Brother Urusov came and
we talked about worldly vanities. He told me of the Emperor's new
projects. I began to criticize them, but remembered my rules and
my benefactor's words---that a true Freemason should be a zealous
worker for the state when his aid is required and a quiet
onlooker when not called on to assist. My tongue is my
enemy. Brothers G. V. and O. visited me and we had a preliminary
talk about the reception of a new Brother. They laid on me the
duty of Rhetor. I feel myself weak and unworthy. Then our talk
turned to the interpretation of the seven pillars and steps of
the Temple, the seven sciences, the seven virtues, the seven
vices, and the seven gifts of the Holy Spirit. Brother O. was
very eloquent. In the evening the admission took place. The new
decoration of the Premises contributed much to the magnificence
of the spectacle. It was Boris Drubetskoy who was admitted.  I
nominated him and was the Rhetor. A strange feeling agitated me
all the time I was alone with him in the dark chamber. I caught
myself harboring a feeling of hatred toward him which I vainly
tried to overcome. That is why I should really like to save him
from evil and lead him into the path of truth, but evil thoughts
of him did not leave me. It seemed to me that his object in
entering the Brotherhood was merely to be intimate and in favor
with members of our lodge. Apart from the fact that he had asked
me several times whether N. and S. were members of our lodge (a
question to which I could not reply) and that according to my
observation he is incapable of feeling respect for our holy order
and is too preoccupied and satisfied with the outer man to desire
spiritual improvement, I had no cause to doubt him, but he seemed
to me insincere, and all the time I stood alone with him in the
dark temple it seemed to me that he was smiling contemptuously at
my words, and I wished really to stab his bare breast with the
sword I held to it.  I could not be eloquent, nor could I frankly
mention my doubts to the Brothers and to the Grand Master. Great
Architect of Nature, help me to find the true path out of the
labyrinth of lies!
\end{quote}

After this, three pages were left blank in the diary, and then
the following was written:

\begin{quote} \calli

I have had a long and instructive talk alone with Brother V., who
advised me to hold fast by Brother A. Though I am unworthy, much
was revealed to me. Adonai is the name of the creator of the
world. Elohim is the name of the ruler of all. The third name is
the name unutterable which means the All. Talks with Brother
V. strengthen, refresh, and support me in the path of virtue. In
his presence doubt has no place.  The distinction between the
poor teachings of mundane science and our sacred all-embracing
teaching is clear to me. Human sciences dissect everything to
comprehend it, and kill everything to examine it. In the holy
science of our order all is one, all is known in its entirety and
life. The Trinity---the three elements of matter---are sulphur,
mercury, and salt. Sulphur is of an oily and fiery nature; in
combination with salt by its fiery nature it arouses a desire in
the latter by means of which it attracts mercury, seizes it,
holds it, and in combination produces other bodies. Mercury is a
fluid, volatile, spiritual essence.  Christ, the Holy Spirit,
Him!...

3rd December

Awoke late, read the Scriptures but was apathetic. Afterwards
went and paced up and down the large hall. I wished to meditate,
but instead my imagination pictured an occurrence of four years
ago, when Dolokhov, meeting me in Moscow after our duel, said he
hoped I was enjoying perfect peace of mind in spite of my wife's
absence. At the time I gave him no answer. Now I recalled every
detail of that meeting and in my mind gave him the most
malevolent and bitter replies. I recollected myself and drove
away that thought only when I found myself glowing with anger,
but I did not sufficiently repent. Afterwards Boris Drubetskoy
came and began relating various adventures. His coming vexed me
from the first, and I said something disagreeable to him. He
replied. I flared up and said much that was unpleasant and even
rude to him. He became silent, and I recollected myself only when
it was too late. My God, I cannot get on with him at all. The
cause of this is my egotism. I set myself above him and so become
much worse than he, for he is lenient to my rudeness while I on
the contrary nourish contempt for him. O God, grant that in his
presence I may rather see my own vileness, and behave so that he
too may benefit. After dinner I fell asleep and as I was drowsing
off I clearly heard a voice saying in my left ear, ``Thy day!''

I dreamed that I was walking in the dark and was suddenly
surrounded by dogs, but I went on undismayed. Suddenly a smallish
dog seized my left thigh with its teeth and would not let go. I
began to throttle it with my hands. Scarcely had I torn it off
before another, a bigger one, began biting me. I lifted it up,
but the higher I lifted it the bigger and heavier it grew. And
suddenly Brother A. came and, taking my arm, led me to a building
to enter which we had to pass along a narrow plank. I stepped on
it, but it bent and gave way and I began to clamber up a fence
which I could scarcely reach with my hands. After much effort I
dragged myself up, so that my leg hung down on one side and my
body on the other. I looked round and saw Brother A. standing on
the fence and pointing me to a broad avenue and garden, and in
the garden was a large and beautiful building. I woke up. O Lord,
great Architect of Nature, help me to tear from myself these
dogs---my passions especially the last, which unites in itself
the strength of all the former ones, and aid me to enter that
temple of virtue to a vision of which I attained in my dream.

7th December

I dreamed that Joseph Alexeevich was sitting in my house, and
that I was very glad and wished to entertain him. It seemed as if
I chattered incessantly with other people and suddenly remembered
that this could not please him, and I wished to come close to him
and embrace him. But as soon as I drew near I saw that his face
had changed and grown young, and he was quietly telling me
something about the teaching of our order, but so softly that I
could not hear it. Then it seemed that we all left the room and
something strange happened. We were sitting or lying on the
floor. He was telling me something, and I wished to show him my
sensibility, and not listening to what he was saying I began
picturing to myself the condition of my inner man and the grace
of God sanctifying me. And tears came into my eyes, and I was
glad he noticed this. But he looked at me with vexation and
jumped up, breaking off his remarks. I felt abashed and asked
whether what he had been saying did not concern me; but he did
not reply, gave me a kind look, and then we suddenly found
ourselves in my bedroom where there is a double bed. He lay down
on the edge of it and I burned with longing to caress him and lie
down too. And he said, ``Tell me frankly what is your chief
temptation? Do you know it? I think you know it already.''
Abashed by this question, I replied that sloth was my chief
temptation. He shook his head incredulously; and even more
abashed, I said that though I was living with my wife as he
advised, I was not living with her as her husband. To this he
replied that one should not deprive a wife of one's embraces and
gave me to understand that that was my duty. But I replied that I
should be ashamed to do it, and suddenly everything vanished. And
I awoke and found in my mind the text from the Gospel: ``The life
was the light of men. And the light shineth in darkness; and the
darkness comprehended it not.'' Joseph Alexeevich's face had
looked young and bright. That day I received a letter from my
benefactor in which he wrote about ``conjugal duties.''

9th December

I had a dream from which I awoke with a throbbing heart. I saw
that I was in Moscow in my house, in the big sitting room, and
Joseph Alexeevich came in from the drawing room. I seemed to know
at once that the process of regeneration had already taken place
in him, and I rushed to meet him. I embraced him and kissed his
hands, and he said, ``Hast thou noticed that my face is
different?'' I looked at him, still holding him in my arms, and
saw that his face was young, but that he had no hair on his head
and his features were quite changed. And I said, ``I should have
known you had I met you by chance,'' and I thought to myself,
``Am I telling the truth?'' And suddenly I saw him lying like a
dead body; then he gradually recovered and went with me into my
study carrying a large book of sheets of drawing paper; I said,
``I drew that,'' and he answered by bowing his head. I opened the
book, and on all the pages there were excellent drawings. And in
my dream I knew that these drawings represented the love
adventures of the soul with its beloved. And on its pages I saw a
beautiful representation of a maiden in transparent garments and
with a transparent body, flying up to the clouds. And I seemed to
know that this maiden was nothing else than a representation of
the Song of Songs. And looking at those drawings I dreamed I felt
that I was doing wrong, but could not tear myself away from
them. Lord, help me! My God, if Thy forsaking me is Thy doing,
Thy will be done; but if I am myself the cause, teach me what I
should do! I shall perish of my debauchery if Thou utterly
desertest me!

\end{quote}

% % % % % % % % % % % % % % % % % % % % % % % % % % % % % % % % %
% % % % % % % % % % % % % % % % % % % % % % % % % % % % % % % % %
% % % % % % % % % % % % % % % % % % % % % % % % % % % % % % % % %
% % % % % % % % % % % % % % % % % % % % % % % % % % % % % % % % %
% % % % % % % % % % % % % % % % % % % % % % % % % % % % % % % % %
% % % % % % % % % % % % % % % % % % % % % % % % % % % % % % % % %
% % % % % % % % % % % % % % % % % % % % % % % % % % % % % % % % %
% % % % % % % % % % % % % % % % % % % % % % % % % % % % % % % % %
% % % % % % % % % % % % % % % % % % % % % % % % % % % % % % % % %
% % % % % % % % % % % % % % % % % % % % % % % % % % % % % % % % %
% % % % % % % % % % % % % % % % % % % % % % % % % % % % % % % % %
% % % % % % % % % % % % % % % % % % % % % % % % % % % % % %

\chapter*{Chapter XI}
\ifaudio     
\marginpar{
\href{http://ia802205.us.archive.org/34/items/war_and_peace_06_0808_librivox/war_and_peace_06_11_tolstoy.mp3}{Audio}} 
\fi

\lettrine[lines=2, loversize=0.3, lraise=0]{\initfamily T}{he}
Rostovs' monetary affairs had not improved during the two
years they had spent in the country.

Though Nicholas Rostov had kept firmly to his resolution and was
still serving modestly in an obscure regiment, spending
comparatively little, the way of life at Otradnoe---Mitenka's
management of affairs, in particular---was such that the debts
inevitably increased every year. The only resource obviously
presenting itself to the old count was to apply for an official
post, so he had come to Petersburg to look for one and also, as
he said, to let the lassies enjoy themselves for the last time.

Soon after their arrival in Petersburg Berg proposed to Vera and
was accepted.

Though in Moscow the Rostovs belonged to the best society without
themselves giving it a thought, yet in Petersburg their circle of
acquaintances was a mixed and indefinite one. In Petersburg they
were provincials, and the very people they had entertained in
Moscow without inquiring to what set they belonged, here looked
down on them.

The Rostovs lived in the same hospitable way in Petersburg as in
Moscow, and the most diverse people met at their suppers. Country
neighbors from Otradnoe, impoverished old squires and their
daughters, Peronskaya a maid of honor, Pierre Bezukhov, and the
son of their district postmaster who had obtained a post in
Petersburg. Among the men who very soon became frequent visitors
at the Rostovs' house in Petersburg were Boris, Pierre whom the
count had met in the street and dragged home with him, and Berg
who spent whole days at the Rostovs' and paid the eldest
daughter, Countess Vera, the attentions a young man pays when he
intends to propose.

Not in vain had Berg shown everybody his right hand wounded at
Austerlitz and held a perfectly unnecessary sword in his left. He
narrated that episode so persistently and with so important an
air that everyone believed in the merit and usefulness of his
deed, and he had obtained two decorations for Austerlitz.

In the Finnish war he also managed to distinguish himself. He had
picked up the scrap of a grenade that had killed an aide-de-camp
standing near the commander-in-chief and had taken it to his
commander. Just as he had done after Austerlitz, he related this
occurrence at such length and so insistently that everyone again
believed it had been necessary to do this, and he received two
decorations for the Finnish war also. In 1809 he was a captain in
the Guards, wore medals, and held some special lucrative posts in
Petersburg.

Though some skeptics smiled when told of Berg's merits, it could
not be denied that he was a painstaking and brave officer, on
excellent terms with his superiors, and a moral young man with a
brilliant career before him and an assured position in society.

Four years before, meeting a German comrade in the stalls of a
Moscow theater, Berg had pointed out Vera Rostova to him and had
said in German, ``das soll mein Weib werden,''\footnote{``That
girl shall be my wife.''} and from that moment had made up his
mind to marry her. Now in Petersburg, having considered the
Rostovs' position and his own, he decided that the time had come
to propose.

Berg's proposal was at first received with a perplexity that was
not flattering to him. At first it seemed strange that the son of
an obscure Livonian gentleman should propose marriage to a
Countess Rostova; but Berg's chief characteristic was such a
naive and good natured egotism that the Rostovs involuntarily
came to think it would be a good thing, since he himself was so
firmly convinced that it was good, indeed excellent. Moreover,
the Rostovs' affairs were seriously embarrassed, as the suitor
could not but know; and above all, Vera was twenty-four, had been
taken out everywhere, and though she was certainly good-looking
and sensible, no one up to now had proposed to her. So they gave
their consent.

``You see,'' said Berg to his comrade, whom he called ``friend''
only because he knew that everyone has friends, ``you see, I have
considered it all, and should not marry if I had not thought it
all out or if it were in any way unsuitable. But on the contrary,
my papa and mamma are now provided for---I have arranged that
rent for them in the Baltic Provinces---and I can live in
Petersburg on my pay, and with her fortune and my good management
we can get along nicely. I am not marrying for money---I consider
that dishonorable---but a wife should bring her share and a
husband his. I have my position in the service, she has
connections and some means. In our times that is worth something,
isn't it? But above all, she is a handsome, estimable girl, and
she loves me...''

Berg blushed and smiled.

``And I love her, because her character is sensible and very
good. Now the other sister, though they are the same family, is
quite different---an unpleasant character and has not the same
intelligence. She is so...  you know?... Unpleasant... But my
fiancee!... Well, you will be coming,'' he was going to say, ``to
dine,'' but changed his mind and said ``to take tea with us,''
and quickly doubling up his tongue he blew a small round ring of
tobacco smoke, perfectly embodying his dream of happiness.

After the first feeling of perplexity aroused in the parents by
Berg's proposal, the holiday tone of joyousness usual at such
times took possession of the family, but the rejoicing was
external and insincere.  In the family's feeling toward this
wedding a certain awkwardness and constraint was evident, as if
they were ashamed of not having loved Vera sufficiently and of
being so ready to get her off their hands. The old count felt
this most. He would probably have been unable to state the cause
of his embarrassment, but it resulted from the state of his
affairs. He did not know at all how much he had, what his debts
amounted to, or what dowry he could give Vera. When his daughters
were born he had assigned to each of them, for her dowry, an
estate with three hundred serfs; but one of these estates had
already been sold, and the other was mortgaged and the interest
so much in arrears that it would have to be sold, so that it was
impossible to give it to Vera. Nor had he any money.

Berg had already been engaged a month, and only a week remained
before the wedding, but the count had not yet decided in his own
mind the question of the dowry, nor spoken to his wife about
it. At one time the count thought of giving her the Ryazan estate
or of selling a forest, at another time of borrowing money on a
note of hand. A few days before the wedding Berg entered the
count's study early one morning and, with a pleasant smile,
respectfully asked his future father-in-law to let him know what
Vera's dowry would be. The count was so disconcerted by this
long-foreseen inquiry that without consideration he gave the
first reply that came into his head. ``I like your being
businesslike about it... I like it. You shall be satisfied...''

And patting Berg on the shoulder he got up, wishing to end the
conversation. But Berg, smiling pleasantly, explained that if he
did not know for certain how much Vera would have and did not
receive at least part of the dowry in advance, he would have to
break matters off.

``Because, consider, Count---if I allowed myself to marry now
without having definite means to maintain my wife, I should be
acting badly...''

The conversation ended by the count, who wished to be generous
and to avoid further importunity, saying that he would give a
note of hand for eighty thousand rubles. Berg smiled meekly,
kissed the count on the shoulder, and said that he was very
grateful, but that it was impossible for him to arrange his new
life without receiving thirty thousand in ready money. ``Or at
least twenty thousand, Count,'' he added, ``and then a note of
hand for only sixty thousand.''

``Yes, yes, all right!'' said the count hurriedly. ``Only excuse
me, my dear fellow, I'll give you twenty thousand and a note of
hand for eighty thousand as well. Yes, yes! Kiss me.''

% % % % % % % % % % % % % % % % % % % % % % % % % % % % % % % % %
% % % % % % % % % % % % % % % % % % % % % % % % % % % % % % % % %
% % % % % % % % % % % % % % % % % % % % % % % % % % % % % % % % %
% % % % % % % % % % % % % % % % % % % % % % % % % % % % % % % % %
% % % % % % % % % % % % % % % % % % % % % % % % % % % % % % % % %
% % % % % % % % % % % % % % % % % % % % % % % % % % % % % % % % %
% % % % % % % % % % % % % % % % % % % % % % % % % % % % % % % % %
% % % % % % % % % % % % % % % % % % % % % % % % % % % % % % % % %
% % % % % % % % % % % % % % % % % % % % % % % % % % % % % % % % %
% % % % % % % % % % % % % % % % % % % % % % % % % % % % % % % % %
% % % % % % % % % % % % % % % % % % % % % % % % % % % % % % % % %
% % % % % % % % % % % % % % % % % % % % % % % % % % % % % %

\chapter*{Chapter XII}
\ifaudio     
\marginpar{
\href{http://ia802205.us.archive.org/34/items/war_and_peace_06_0808_librivox/war_and_peace_06_12_tolstoy.mp3}{Audio}} 
\fi

\lettrine[lines=2, loversize=0.3, lraise=0]{\initfamily N}{atasha}
was sixteen and it was the year 1809, the very year to
which she had counted on her fingers with Boris after they had
kissed four years ago. Since then she had not seen him. Before
Sonya and her mother, if Boris happened to be mentioned, she
spoke quite freely of that episode as of some childish,
long-forgotten matter that was not worth mentioning. But in the
secret depths of her soul the question whether her engagement to
Boris was a jest or an important, binding promise tormented her.

Since Boris left Moscow in 1805 to join the army he had not seen
the Rostovs. He had been in Moscow several times, and had passed
near Otradnoe, but had never been to see them.

Sometimes it occurred to Natasha that he did not wish to see her,
and this conjecture was confirmed by the sad tone in which her
elders spoke of him.

``Nowadays old friends are not remembered,'' the countess would
say when Boris was mentioned.

Anna Mikhaylovna also had of late visited them less frequently,
seemed to hold herself with particular dignity, and always spoke
rapturously and gratefully of the merits of her son and the
brilliant career on which he had entered. When the Rostovs came
to Petersburg Boris called on them.

He drove to their house in some agitation. The memory of Natasha
was his most poetic recollection. But he went with the firm
intention of letting her and her parents feel that the childish
relations between himself and Natasha could not be binding either
on her or on him. He had a brilliant position in society thanks
to his intimacy with Countess Bezukhova, a brilliant position in
the service thanks to the patronage of an important personage
whose complete confidence he enjoyed, and he was beginning to
make plans for marrying one of the richest heiresses in
Petersburg, plans which might very easily be realized. When he
entered the Rostovs' drawing room Natasha was in her own
room. When she heard of his arrival she almost ran into the
drawing room, flushed and beaming with a more than cordial smile.

Boris remembered Natasha in a short dress, with dark eyes shining
from under her curls and boisterous, childish laughter, as he had
known her four years before; and so he was taken aback when quite
a different Natasha entered, and his face expressed rapturous
astonishment. This expression on his face pleased Natasha.

``Well, do you recognize your little madcap playmate?'' asked the
countess.

Boris kissed Natasha's hand and said that he was astonished at
the change in her.

``How handsome you have grown!''

``I should think so!'' replied Natasha's laughing eyes.

``And is Papa older?'' she asked.

Natasha sat down and, without joining in Boris' conversation with
the countess, silently and minutely studied her childhood's
suitor. He felt the weight of that resolute and affectionate
scrutiny and glanced at her occasionally.

Boris' uniform, spurs, tie, and the way his hair was brushed were
all comme il faut and in the latest fashion. This Natasha noticed
at once.  He sat rather sideways in the armchair next to the
countess, arranging with his right hand the cleanest of gloves
that fitted his left hand like a skin, and he spoke with a
particularly refined compression of his lips about the amusements
of the highest Petersburg society, recalling with mild irony old
times in Moscow and Moscow acquaintances. It was not
accidentally, Natasha felt, that he alluded, when speaking of the
highest aristocracy, to an ambassador's ball he had attended, and
to invitations he had received from N.N. and S.S.

All this time Natasha sat silent, glancing up at him from under
her brows. This gaze disturbed and confused Boris more and
more. He looked round more frequently toward her, and broke off
in what he was saying.  He did not stay more than ten minutes,
then rose and took his leave. The same inquisitive, challenging,
and rather mocking eyes still looked at him. After his first
visit Boris said to himself that Natasha attracted him just as
much as ever, but that he must not yield to that feeling, because
to marry her, a girl almost without fortune, would mean ruin to
his career, while to renew their former relations without
intending to marry her would be dishonorable. Boris made up his
mind to avoid meeting Natasha, but despite that resolution he
called again a few days later and began calling often and
spending whole days at the Rostovs'. It seemed to him that he
ought to have an explanation with Natasha and tell her that the
old times must be forgotten, that in spite of everything...  she
could not be his wife, that he had no means, and they would never
let her marry him. But he failed to do so and felt awkward about
entering on such an explanation. From day to day he became more
and more entangled. It seemed to her mother and Sonya that
Natasha was in love with Boris as of old. She sang him his
favorite songs, showed him her album, making him write in it, did
not allow him to allude to the past, letting it be understood how
delightful was the present; and every day he went away in a fog,
without having said what he meant to, and not knowing what he was
doing or why he came, or how it would all end. He left off
visiting Helene and received reproachful notes from her every
day, and yet he continued to spend whole days with the Rostovs.

% % % % % % % % % % % % % % % % % % % % % % % % % % % % % % % % %
% % % % % % % % % % % % % % % % % % % % % % % % % % % % % % % % %
% % % % % % % % % % % % % % % % % % % % % % % % % % % % % % % % %
% % % % % % % % % % % % % % % % % % % % % % % % % % % % % % % % %
% % % % % % % % % % % % % % % % % % % % % % % % % % % % % % % % %
% % % % % % % % % % % % % % % % % % % % % % % % % % % % % % % % %
% % % % % % % % % % % % % % % % % % % % % % % % % % % % % % % % %
% % % % % % % % % % % % % % % % % % % % % % % % % % % % % % % % %
% % % % % % % % % % % % % % % % % % % % % % % % % % % % % % % % %
% % % % % % % % % % % % % % % % % % % % % % % % % % % % % % % % %
% % % % % % % % % % % % % % % % % % % % % % % % % % % % % % % % %
% % % % % % % % % % % % % % % % % % % % % % % % % % % % % %

\chapter*{Chapter XIII}
\ifaudio     
\marginpar{
\href{http://ia802205.us.archive.org/34/items/war_and_peace_06_0808_librivox/war_and_peace_06_13_tolstoy.mp3}{Audio}} 
\fi

\lettrine[lines=2, loversize=0.3, lraise=0]{\initfamily O}{ne}
night when the old countess, in nightcap and dressing jacket,
without her false curls, and with her poor little knob of hair
showing under her white cotton cap, knelt sighing and groaning on
a rug and bowing to the ground in prayer, her door creaked and
Natasha, also in a dressing jacket with slippers on her bare feet
and her hair in curlpapers, ran in. The countess---her prayerful
mood dispelled---looked round and frowned. She was finishing her
last prayer: ``Can it be that this couch will be my grave?''
Natasha, flushed and eager, seeing her mother in prayer, suddenly
checked her rush, half sat down, and unconsciously put out her
tongue as if chiding herself. Seeing that her mother was still
praying she ran on tiptoe to the bed and, rapidly slipping one
little foot against the other, pushed off her slippers and jumped
onto the bed the countess had feared might become her grave. This
couch was high, with a feather bed and five pillows each smaller
than the one below. Natasha jumped on it, sank into the feather
bed, rolled over to the wall, and began snuggling up the
bedclothes as she settled down, raising her knees to her chin,
kicking out and laughing almost inaudibly, now covering herself
up head and all, and now peeping at her mother. The countess
finished her prayers and came to the bed with a stern face, but
seeing, that Natasha's head was covered, she smiled in her kind,
weak way.

``Now then, now then!'' said she.

``Mamma, can we have a talk? Yes?'' said Natasha. ``Now, just one
on your throat and another... that'll do!'' And seizing her
mother round the neck, she kissed her on the throat. In her
behavior to her mother Natasha seemed rough, but she was so
sensitive and tactful that however she clasped her mother she
always managed to do it without hurting her or making her feel
uncomfortable or displeased.

``Well, what is it tonight?'' said the mother, having arranged
her pillows and waited until Natasha, after turning over a couple
of times, had settled down beside her under the quilt, spread out
her arms, and assumed a serious expression.

These visits of Natasha's at night before the count returned from
his club were one of the greatest pleasures of both mother, and
daughter.

``What is it tonight?---But I have to tell you...''

Natasha put her hand on her mother's mouth.

``About Boris... I know,'' she said seriously; ``that's what I
have come about. Don't say it---I know. No, do tell me!'' and she
removed her hand.  ``Tell me, Mamma! He's nice?''

``Natasha, you are sixteen. At your age I was married. You say
Boris is nice. He is very nice, and I love him like a son. But
what then?... What are you thinking about? You have quite turned
his head, I can see that...''

As she said this the countess looked round at her
daughter. Natasha was lying looking steadily straight before her
at one of the mahogany sphinxes carved on the corners of the
bedstead, so that the countess only saw her daughter's face in
profile. That face struck her by its peculiarly serious and
concentrated expression.

Natasha was listening and considering.

``Well, what then?'' said she.

``You have quite turned his head, and why? What do you want of
him? You know you can't marry him.''

``Why not?'' said Natasha, without changing her position.

``Because he is young, because he is poor, because he is a
relation...  and because you yourself don't love him.''

``How do you know?''

``I know. It is not right, darling!''

``But if I want to...'' said Natasha.

``Leave off talking nonsense,'' said the countess.

``But if I want to...''

``Natasha, I am in earnest...''

Natasha did not let her finish. She drew the countess' large hand
to her, kissed it on the back and then on the palm, then again
turned it over and began kissing first one knuckle, then the
space between the knuckles, then the next knuckle, whispering,
``January, February, March, April, May. Speak, Mamma, why don't
you say anything? Speak!'' said she, turning to her mother, who
was tenderly gazing at her daughter and in that contemplation
seemed to have forgotten all she had wished to say.

``It won't do, my love! Not everyone will understand this
friendship dating from your childish days, and to see him so
intimate with you may injure you in the eyes of other young men
who visit us, and above all it torments him for nothing. He may
already have found a suitable and wealthy match, and now he's
half crazy.''

``Crazy?'' repeated Natasha.

``I'll tell you some things about myself. I had a cousin...''

``I know! Cyril Matveich... but he is old.''

``He was not always old. But this is what I'll do, Natasha, I'll
have a talk with Boris. He need not come so often...''

``Why not, if he likes to?''

``Because I know it will end in nothing...''

``How can you know? No, Mamma, don't speak to him! What
nonsense!'' said Natasha in the tone of one being deprived of her
property. ``Well, I won't marry, but let him come if he enjoys it
and I enjoy it.'' Natasha smiled and looked at her mother. ``Not
to marry, but just so,'' she added.

``How so, my pet?''

``Just so. There's no need for me to marry him. But... just so.''

``Just so, just so,'' repeated the countess, and shaking all
over, she went off into a good humored, unexpected, elderly
laugh.

``Don't laugh, stop!'' cried Natasha. ``You're shaking the whole
bed!  You're awfully like me, just such another
giggler... Wait...'' and she seized the countess' hands and
kissed a knuckle of the little finger, saying, ``June,'' and
continued, kissing, ``July, August,'' on the other hand. ``But,
Mamma, is he very much in love? What do you think? Was anybody
ever so much in love with you? And he's very nice, very, very
nice. Only not quite my taste---he is so narrow, like the
dining-room clock... Don't you understand? Narrow, you
know---gray, light gray...''

``What rubbish you're talking!'' said the countess.

Natasha continued: ``Don't you really understand? Nicholas would
understand... Bezukhov, now, is blue, dark-blue and red, and he
is square.''

``You flirt with him too,'' said the countess, laughing.

``No, he is a Freemason, I have found out. He is fine, dark-blue
and red... How can I explain it to you?''

``Little countess!'' the count's voice called from behind the
door.  ``You're not asleep?'' Natasha jumped up, snatched up her
slippers, and ran barefoot to her own room.

It was a long time before she could sleep. She kept thinking that
no one could understand all that she understood and all there was
in her.

``Sonya?'' she thought, glancing at that curled-up, sleeping
little kitten with her enormous plait of hair. ``No, how could
she? She's virtuous. She fell in love with Nicholas and does not
wish to know anything more. Even Mamma does not understand. It is
wonderful how clever I am and how...  charming she is,'' she went
on, speaking of herself in the third person, and imagining it was
some very wise man---the wisest and best of men---who was saying
it of her. ``There is everything, everything in her,'' continued
this man. ``She is unusually intelligent, charming... and then
she is pretty, uncommonly pretty, and agile---she swims and rides
splendidly... and her voice! One can really say it's a wonderful
voice!''

She hummed a scrap from her favorite opera by Cherubini, threw
herself on her bed, laughed at the pleasant thought that she
would immediately fall asleep, called Dunyasha the maid to put
out the candle, and before Dunyasha had left the room had already
passed into yet another happier world of dreams, where everything
was as light and beautiful as in reality, and even more so
because it was different.

Next day the countess called Boris aside and had a talk with him,
after which he ceased coming to the Rostovs'.

% % % % % % % % % % % % % % % % % % % % % % % % % % % % % % % % %
% % % % % % % % % % % % % % % % % % % % % % % % % % % % % % % % %
% % % % % % % % % % % % % % % % % % % % % % % % % % % % % % % % %
% % % % % % % % % % % % % % % % % % % % % % % % % % % % % % % % %
% % % % % % % % % % % % % % % % % % % % % % % % % % % % % % % % %
% % % % % % % % % % % % % % % % % % % % % % % % % % % % % % % % %
% % % % % % % % % % % % % % % % % % % % % % % % % % % % % % % % %
% % % % % % % % % % % % % % % % % % % % % % % % % % % % % % % % %
% % % % % % % % % % % % % % % % % % % % % % % % % % % % % % % % %
% % % % % % % % % % % % % % % % % % % % % % % % % % % % % % % % %
% % % % % % % % % % % % % % % % % % % % % % % % % % % % % % % % %
% % % % % % % % % % % % % % % % % % % % % % % % % % % % % %

\chapter*{Chapter XIV}
\ifaudio     
\marginpar{
\href{http://ia802205.us.archive.org/34/items/war_and_peace_06_0808_librivox/war_and_peace_06_14_tolstoy.mp3}{Audio}} 
\fi

\lettrine[lines=2, loversize=0.3, lraise=0]{\initfamily O}{n}
the thirty-first of December, New Year's Eve, 1809--10 an old
grandee of Catherine's day was giving a ball and midnight
supper. The diplomatic corps and the Emperor himself were to be
present.

The grandee's well-known mansion on the English Quay glittered
with innumerable lights. Police were stationed at the brightly
lit entrance which was carpeted with red baize, and not only
gendarmes but dozens of police officers and even the police
master himself stood at the porch.  Carriages kept driving away
and fresh ones arriving, with red-liveried footmen and footmen in
plumed hats. From the carriages emerged men wearing uniforms,
stars, and ribbons, while ladies in satin and ermine cautiously
descended the carriage steps which were let down for them with a
clatter, and then walked hurriedly and noiselessly over the baize
at the entrance.

Almost every time a new carriage drove up a whisper ran through
the crowd and caps were doffed.

``The Emperor?... No, a minister... prince... ambassador. Don't
you see the plumes?...'' was whispered among the crowd.

One person, better dressed than the rest, seemed to know everyone
and mentioned by name the greatest dignitaries of the day.

A third of the visitors had already arrived, but the Rostovs, who
were to be present, were still hurrying to get dressed.

There had been many discussions and preparations for this ball in
the Rostov family, many fears that the invitation would not
arrive, that the dresses would not be ready, or that something
would not be arranged as it should be.

Marya Ignatevna Peronskaya, a thin and shallow maid of honor at
the court of the Dowager Empress, who was a friend and relation
of the countess and piloted the provincial Rostovs in Petersburg
high society, was to accompany them to the ball.

They were to call for her at her house in the Taurida Gardens at
ten o'clock, but it was already five minutes to ten, and the
girls were not yet dressed.

Natasha was going to her first grand ball. She had got up at
eight that morning and had been in a fever of excitement and
activity all day. All her powers since morning had been
concentrated on ensuring that they all---she herself, Mamma, and
Sonya---should be as well dressed as possible. Sonya and her
mother put themselves entirely in her hands. The countess was to
wear a claret-colored velvet dress, and the two girls white gauze
over pink silk slips, with roses on their bodices and their hair
dressed a la grecque.

Everything essential had already been done; feet, hands, necks,
and ears washed, perfumed, and powdered, as befits a ball; the
openwork silk stockings and white satin shoes with ribbons were
already on; the hairdressing was almost done. Sonya was finishing
dressing and so was the countess, but Natasha, who had bustled
about helping them all, was behindhand. She was still sitting
before a looking-glass with a dressing jacket thrown over her
slender shoulders. Sonya stood ready dressed in the middle of the
room and, pressing the head of a pin till it hurt her dainty
finger, was fixing on a last ribbon that squeaked as the pin went
through it.

``That's not the way, that's not the way, Sonya!'' cried Natasha
turning her head and clutching with both hands at her hair which
the maid who was dressing it had not time to release. ``That bow
is not right. Come here!''

Sonya sat down and Natasha pinned the ribbon on differently.

``Allow me, Miss! I can't do it like that,'' said the maid who
was holding Natasha's hair.

``Oh, dear! Well then, wait. That's right, Sonya.''

``Aren't you ready? It is nearly ten,'' came the countess' voice.

``Directly! Directly! And you, Mamma?''

``I have only my cap to pin on.''

``Don't do it without me!'' called Natasha. ``You won't do it
right.''

``But it's already ten.''

They had decided to be at the ball by half past ten, and Natasha
had still to get dressed and they had to call at the Taurida
Gardens.

When her hair was done, Natasha, in her short petticoat from
under which her dancing shoes showed, and in her mother's
dressing jacket, ran up to Sonya, scrutinized her, and then ran
to her mother. Turning her mother's head this way and that, she
fastened on the cap and, hurriedly kissing her gray hair, ran
back to the maids who were turning up the hem of her skirt.

The cause of the delay was Natasha's skirt, which was too
long. Two maids were turning up the hem and hurriedly biting off
the ends of thread. A third with pins in her mouth was running
about between the countess and Sonya, and a fourth held the whole
of the gossamer garment up high on one uplifted hand.

``Mavra, quicker, darling!''

``Give me my thimble, Miss, from there...''

``Whenever will you be ready?'' asked the count coming to the
door. ``Here is some scent. Peronskaya must be tired of
waiting.''

``It's ready, Miss,'' said the maid, holding up the shortened
gauze dress with two fingers, and blowing and shaking something
off it, as if by this to express a consciousness of the airiness
and purity of what she held.

Natasha began putting on the dress.

``In a minute! In a minute! Don't come in, Papa!'' she cried to
her father as he opened the door---speaking from under the filmy
skirt which still covered her whole face.

Sonya slammed the door to. A minute later they let the count
in. He was wearing a blue swallow-tail coat, shoes and stockings,
and was perfumed and his hair pomaded.

``Oh, Papa! how nice you look! Charming!'' cried Natasha, as she
stood in the middle of the room smoothing out the folds of the
gauze.

``If you please, Miss! allow me,'' said the maid, who on her
knees was pulling the skirt straight and shifting the pins from
one side of her mouth to the other with her tongue.

``Say what you like,'' exclaimed Sonya, in a despairing voice as
she looked at Natasha, ``say what you like, it's still too
long.''

Natasha stepped back to look at herself in the pier glass. The
dress was too long.

``Really, madam, it is not at all too long,'' said Mavra,
crawling on her knees after her young lady.

``Well, if it's too long we'll tack it up... we'll tack it up in
one minute,'' said the resolute Dunyasha taking a needle that was
stuck on the front of her little shawl and, still kneeling on the
floor, set to work once more.

At that moment, with soft steps, the countess came in shyly, in
her cap and velvet gown.

``Oo-oo, my beauty!'' exclaimed the count, ``she looks better
than any of you!''

He would have embraced her but, blushing, she stepped aside
fearing to be rumpled.

``Mamma, your cap, more to this side,'' said Natasha. ``I'll
arrange it,'' and she rushed forward so that the maids who were
tacking up her skirt could not move fast enough and a piece of
gauze was torn off.

``Oh goodness! What has happened? Really it was not my fault!''

``Never mind, I'll run it up, it won't show,'' said Dunyasha.

``What a beauty---a very queen!'' said the nurse as she came to
the door.  ``And Sonya! They are lovely!''

At a quarter past ten they at last got into their carriages and
started.  But they had still to call at the Taurida Gardens.

Peronskaya was quite ready. In spite of her age and plainness she
had gone through the same process as the Rostovs, but with less
flurry---for to her it was a matter of routine. Her ugly old body
was washed, perfumed, and powdered in just the same way. She had
washed behind her ears just as carefully, and when she entered
her drawing room in her yellow dress, wearing her badge as maid
of honor, her old lady's maid was as full of rapturous admiration
as the Rostovs' servants had been.

She praised the Rostovs' toilets. They praised her taste and
toilet, and at eleven o'clock, careful of their coiffures and
dresses, they settled themselves in their carriages and drove
off.

% % % % % % % % % % % % % % % % % % % % % % % % % % % % % % % % %
% % % % % % % % % % % % % % % % % % % % % % % % % % % % % % % % %
% % % % % % % % % % % % % % % % % % % % % % % % % % % % % % % % %
% % % % % % % % % % % % % % % % % % % % % % % % % % % % % % % % %
% % % % % % % % % % % % % % % % % % % % % % % % % % % % % % % % %
% % % % % % % % % % % % % % % % % % % % % % % % % % % % % % % % %
% % % % % % % % % % % % % % % % % % % % % % % % % % % % % % % % %
% % % % % % % % % % % % % % % % % % % % % % % % % % % % % % % % %
% % % % % % % % % % % % % % % % % % % % % % % % % % % % % % % % %
% % % % % % % % % % % % % % % % % % % % % % % % % % % % % % % % %
% % % % % % % % % % % % % % % % % % % % % % % % % % % % % % % % %
% % % % % % % % % % % % % % % % % % % % % % % % % % % % % %

\chapter*{Chapter XV}
\ifaudio     
\marginpar{
\href{http://ia802205.us.archive.org/34/items/war_and_peace_06_0808_librivox/war_and_peace_06_15_tolstoy.mp3}{Audio}} 
\fi

\lettrine[lines=2, loversize=0.3, lraise=0]{\initfamily N}{atasha}
had not had a moment free since early morning and had not
once had time to think of what lay before her.

In the damp chill air and crowded closeness of the swaying
carriage, she for the first time vividly imagined what was in
store for her there at the ball, in those brightly lighted
rooms---with music, flowers, dances, the Emperor, and all the
brilliant young people of Petersburg. The prospect was so
splendid that she hardly believed it would come true, so out of
keeping was it with the chill darkness and closeness of the
carriage. She understood all that awaited her only when, after
stepping over the red baize at the entrance, she entered the
hall, took off her fur cloak, and, beside Sonya and in front of
her mother, mounted the brightly illuminated stairs between the
flowers. Only then did she remember how she must behave at a
ball, and tried to assume the majestic air she considered
indispensable for a girl on such an occasion. But, fortunately
for her, she felt her eyes growing misty, she saw nothing
clearly, her pulse beat a hundred to the minute, and the blood
throbbed at her heart. She could not assume that pose, which
would have made her ridiculous, and she moved on almost fainting
from excitement and trying with all her might to conceal it. And
this was the very attitude that became her best. Before and
behind them other visitors were entering, also talking in low
tones and wearing ball dresses. The mirrors on the landing
reflected ladies in white, pale-blue, and pink dresses, with
diamonds and pearls on their bare necks and arms.

Natasha looked in the mirrors and could not distinguish her
reflection from the others. All was blended into one brilliant
procession. On entering the ballroom the regular hum of voices,
footsteps, and greetings deafened Natasha, and the light and
glitter dazzled her still more. The host and hostess, who had
already been standing at the door for half an hour repeating the
same words to the various arrivals, ``Charme de vous
voir,''\footnote{``Delighted to see you.''} greeted the Rostovs
and Peronskaya in the same manner.

The two girls in their white dresses, each with a rose in her
black hair, both curtsied in the same way, but the hostess' eye
involuntarily rested longer on the slim Natasha. She looked at
her and gave her alone a special smile in addition to her usual
smile as hostess. Looking at her she may have recalled the
golden, irrecoverable days of her own girlhood and her own first
ball. The host also followed Natasha with his eyes and asked the
count which was his daughter.

``Charming!'' said he, kissing the tips of his fingers.

In the ballroom guests stood crowding at the entrance doors
awaiting the Emperor. The countess took up a position in one of
the front rows of that crowd. Natasha heard and felt that several
people were asking about her and looking at her. She realized
that those noticing her liked her, and this observation helped to
calm her.

``There are some like ourselves and some worse,'' she thought.

Peronskaya was pointing out to the countess the most important
people at the ball.

``That is the Dutch ambassador, do you see? That gray-haired
man,'' she said, indicating an old man with a profusion of
silver-gray curly hair, who was surrounded by ladies laughing at
something he said.

``Ah, here she is, the Queen of Petersburg, Countess Bezukhova,''
said Peronskaya, indicating Helene who had just entered. ``How
lovely! She is quite equal to Marya Antonovna. See how the men,
young and old, pay court to her. Beautiful and clever... they say
Prince---is quite mad about her. But see, those two, though not
good-looking, are even more run after.''

She pointed to a lady who was crossing the room followed by a
very plain daughter.

``She is a splendid match, a millionairess,'' said
Peronskaya. ``And look, here come her suitors.''

``That is Bezukhova's brother, Anatole Kuragin,'' she said,
indicating a handsome officer of the Horse Guards who passed by
them with head erect, looking at something over the heads of the
ladies. ``He's handsome, isn't he? I hear they will marry him to
that rich girl. But your cousin, Drubetskoy, is also very
attentive to her. They say she has millions. Oh yes, that's the
French ambassador himself!'' she replied to the countess' inquiry
about Caulaincourt. ``Looks as if he were a king! All the same,
the French are charming, very charming. No one more charming in
society.  Ah, here she is! Yes, she is still the most beautiful
of them all, our Marya Antonovna! And how simply she is dressed!
Lovely! And that stout one in spectacles is the universal
Freemason,'' she went on, indicating Pierre. ``Put him beside his
wife and he looks a regular buffoon!''

Pierre, swaying his stout body, advanced, making way through the
crowd and nodding to right and left as casually and
good-naturedly as if he were passing through a crowd at a
fair. He pushed through, evidently looking for someone.

Natasha looked joyfully at the familiar face of Pierre, ``the
buffoon,'' as Peronskaya had called him, and knew he was looking
for them, and for her in particular. He had promised to be at the
ball and introduce partners to her.

But before he reached them Pierre stopped beside a very handsome,
dark man of middle height, and in a white uniform, who stood by a
window talking to a tall man wearing stars and a ribbon. Natasha
at once recognized the shorter and younger man in the white
uniform: it was Bolkonski, who seemed to her to have grown much
younger, happier, and better-looking.

``There's someone else we know---Bolkonski, do you see, Mamma?''
said Natasha, pointing out Prince Andrew. ``You remember, he
stayed a night with us at Otradnoe.''

``Oh, you know him?'' said Peronskaya. ``I can't bear him. Il
fait a present la pluie et le beau temps.\footnote{''He is all
the rage just now.``} He's too proud for anything. Takes after
his father. And he's hand in glove with Speranski, writing some
project or other. Just look how he treats the ladies! There's one
talking to him and he has turned away,'' she said, pointing at
him. ``I'd give it to him if he treated me as he does those
ladies.''

% % % % % % % % % % % % % % % % % % % % % % % % % % % % % % % % %
% % % % % % % % % % % % % % % % % % % % % % % % % % % % % % % % %
% % % % % % % % % % % % % % % % % % % % % % % % % % % % % % % % %
% % % % % % % % % % % % % % % % % % % % % % % % % % % % % % % % %
% % % % % % % % % % % % % % % % % % % % % % % % % % % % % % % % %
% % % % % % % % % % % % % % % % % % % % % % % % % % % % % % % % %
% % % % % % % % % % % % % % % % % % % % % % % % % % % % % % % % %
% % % % % % % % % % % % % % % % % % % % % % % % % % % % % % % % %
% % % % % % % % % % % % % % % % % % % % % % % % % % % % % % % % %
% % % % % % % % % % % % % % % % % % % % % % % % % % % % % % % % %
% % % % % % % % % % % % % % % % % % % % % % % % % % % % % % % % %
% % % % % % % % % % % % % % % % % % % % % % % % % % % % % %

\chapter*{Chapter XVI}
\ifaudio     
\marginpar{
\href{http://ia802205.us.archive.org/34/items/war_and_peace_06_0808_librivox/war_and_peace_06_16_tolstoy.mp3}{Audio}} 
\fi

\lettrine[lines=2, loversize=0.3, lraise=0]{\initfamily S}{uddenly}
everybody stirred, began talking, and pressed forward
and then back, and between the two rows, which separated, the
Emperor entered to the sounds of music that had immediately
struck up. Behind him walked his host and hostess. He walked in
rapidly, bowing to right and left as if anxious to get the first
moments of the reception over. The band played the polonaise in
vogue at that time on account of the words that had been set to
it, beginning: ``Alexander, Elisaveta, all our hearts you ravish
quite...'' The Emperor passed on to the drawing room, the crowd
made a rush for the doors, and several persons with excited faces
hurried there and back again. Then the crowd hastily retired from
the drawing-room door, at which the Emperor reappeared talking to
the hostess. A young man, looking distraught, pounced down on the
ladies, asking them to move aside. Some ladies, with faces
betraying complete forgetfulness of all the rules of decorum,
pushed forward to the detriment of their toilets. The men began
to choose partners and take their places for the polonaise.

Everyone moved back, and the Emperor came smiling out of the
drawing room leading his hostess by the hand but not keeping time
to the music.  The host followed with Marya Antonovna Naryshkina;
then came ambassadors, ministers, and various generals, whom
Peronskaya diligently named. More than half the ladies already
had partners and were taking up, or preparing to take up, their
positions for the polonaise. Natasha felt that she would be left
with her mother and Sonya among a minority of women who crowded
near the wall, not having been invited to dance.  She stood with
her slender arms hanging down, her scarcely defined bosom rising
and falling regularly, and with bated breath and glittering,
frightened eyes gazed straight before her, evidently prepared for
the height of joy or misery. She was not concerned about the
Emperor or any of those great people whom Peronskaya was pointing
out---she had but one thought: ``Is it possible no one will ask
me, that I shall not be among the first to dance? Is it possible
that not one of all these men will notice me? They do not even
seem to see me, or if they do they look as if they were saying,
'Ah, she's not the one I'm after, so it's not worth looking at
her!' No, it's impossible,'' she thought. ``They must know how I
long to dance, how splendidly I dance, and how they would enjoy
dancing with me.''

The strains of the polonaise, which had continued for a
considerable time, had begun to sound like a sad reminiscence to
Natasha's ears. She wanted to cry. Peronskaya had left them. The
count was at the other end of the room. She and the countess and
Sonya were standing by themselves as in the depths of a forest
amid that crowd of strangers, with no one interested in them and
not wanted by anyone. Prince Andrew with a lady passed by,
evidently not recognizing them. The handsome Anatole was
smilingly talking to a partner on his arm and looked at Natasha
as one looks at a wall. Boris passed them twice and each time
turned away. Berg and his wife, who were not dancing, came up to
them.

This family gathering seemed humiliating to Natasha---as if there
were nowhere else for the family to talk but here at the
ball. She did not listen to or look at Vera, who was telling her
something about her own green dress.

At last the Emperor stopped beside his last partner (he had
danced with three) and the music ceased. A worried aide-de-camp
ran up to the Rostovs requesting them to stand farther back,
though as it was they were already close to the wall, and from
the gallery resounded the distinct, precise, enticingly
rhythmical strains of a waltz. The Emperor looked smilingly down
the room. A minute passed but no one had yet begun dancing. An
aide-de-camp, the Master of Ceremonies, went up to Countess
Bezukhova and asked her to dance. She smilingly raised her hand
and laid it on his shoulder without looking at him. The
aide-de-camp, an adept in his art, grasping his partner firmly
round her waist, with confident deliberation started smoothly,
gliding first round the edge of the circle, then at the corner of
the room he caught Helene's left hand and turned her, the only
sound audible, apart from the ever-quickening music, being the
rhythmic click of the spurs on his rapid, agile feet, while at
every third beat his partner's velvet dress spread out and seemed
to flash as she whirled round. Natasha gazed at them and was
ready to cry because it was not she who was dancing that first
turn of the waltz.

Prince Andrew, in the white uniform of a cavalry colonel, wearing
stockings and dancing shoes, stood looking animated and bright in
the front row of the circle not far from the Rostovs. Baron
Firhoff was talking to him about the first sitting of the Council
of State to be held next day. Prince Andrew, as one closely
connected with Speranski and participating in the work of the
legislative commission, could give reliable information about
that sitting, concerning which various rumors were current. But
not listening to what Firhoff was saying, he was gazing now at
the sovereign and now at the men intending to dance who had not
yet gathered courage to enter the circle.

Prince Andrew was watching these men abashed by the Emperor's
presence, and the women who were breathlessly longing to be asked
to dance.

Pierre came up to him and caught him by the arm.

``You always dance. I have a protegee, the young Rostova,
here. Ask her,'' he said.

``Where is she?'' asked Bolkonski. ``Excuse me!'' he added,
turning to the baron, ``we will finish this conversation
elsewhere---at a ball one must dance.'' He stepped forward in the
direction Pierre indicated. The despairing, dejected expression
of Natasha's face caught his eye. He recognized her, guessed her
feelings, saw that it was her debut, remembered her conversation
at the window, and with an expression of pleasure on his face
approached Countess Rostova.

``Allow me to introduce you to my daughter,'' said the countess,
with heightened color.

``I have the pleasure of being already acquainted, if the
countess remembers me,'' said Prince Andrew with a low and
courteous bow quite belying Peronskaya's remarks about his
rudeness, and approaching Natasha he held out his arm to grasp
her waist before he had completed his invitation. He asked her to
waltz. That tremulous expression on Natasha's face, prepared
either for despair or rapture, suddenly brightened into a happy,
grateful, childlike smile.

``I have long been waiting for you,'' that frightened happy
little girl seemed to say by the smile that replaced the
threatened tears, as she raised her hand to Prince Andrew's
shoulder. They were the second couple to enter the circle. Prince
Andrew was one of the best dancers of his day and Natasha danced
exquisitely. Her little feet in their white satin dancing shoes
did their work swiftly, lightly, and independently of herself,
while her face beamed with ecstatic happiness. Her slender bare
arms and neck were not beautiful---compared to Helene's her
shoulders looked thin and her bosom undeveloped. But Helene
seemed, as it were, hardened by a varnish left by the thousands
of looks that had scanned her person, while Natasha was like a
girl exposed for the first time, who would have felt very much
ashamed had she not been assured that this was absolutely
necessary.

Prince Andrew liked dancing, and wishing to escape as quickly as
possible from the political and clever talk which everyone
addressed to him, wishing also to break up the circle of
restraint he disliked, caused by the Emperor's presence, he
danced, and had chosen Natasha because Pierre pointed her out to
him and because she was the first pretty girl who caught his eye;
but scarcely had he embraced that slender supple figure and felt
her stirring so close to him and smiling so near him than the
wine of her charm rose to his head, and he felt himself revived
and rejuvenated when after leaving her he stood breathing deeply
and watching the other dancers.

% % % % % % % % % % % % % % % % % % % % % % % % % % % % % % % % %
% % % % % % % % % % % % % % % % % % % % % % % % % % % % % % % % %
% % % % % % % % % % % % % % % % % % % % % % % % % % % % % % % % %
% % % % % % % % % % % % % % % % % % % % % % % % % % % % % % % % %
% % % % % % % % % % % % % % % % % % % % % % % % % % % % % % % % %
% % % % % % % % % % % % % % % % % % % % % % % % % % % % % % % % %
% % % % % % % % % % % % % % % % % % % % % % % % % % % % % % % % %
% % % % % % % % % % % % % % % % % % % % % % % % % % % % % % % % %
% % % % % % % % % % % % % % % % % % % % % % % % % % % % % % % % %
% % % % % % % % % % % % % % % % % % % % % % % % % % % % % % % % %
% % % % % % % % % % % % % % % % % % % % % % % % % % % % % % % % %
% % % % % % % % % % % % % % % % % % % % % % % % % % % % % %

\chapter*{Chapter XVII}
\ifaudio     
\marginpar{
\href{http://ia802205.us.archive.org/34/items/war_and_peace_06_0808_librivox/war_and_peace_06_17_tolstoy.mp3}{Audio}} 
\fi

\lettrine[lines=2, loversize=0.3, lraise=0]{\initfamily A}{fter}
Prince Andrew, Boris came up to ask Natasha for a dance,
and then the aide-de-camp who had opened the ball, and several
other young men, so that, flushed and happy, and passing on her
superfluous partners to Sonya, she did not cease dancing all the
evening. She noticed and saw nothing of what occupied everyone
else. Not only did she fail to notice that the Emperor talked a
long time with the French ambassador, and how particularly
gracious he was to a certain lady, or that Prince So-and-so and
So-and-so did and said this and that, and that Helene had great
success and was honored by the special attention of So-and-so,
but she did not even see the Emperor, and only noticed that he
had gone because the ball became livelier after his
departure. For one of the merry cotillions before supper Prince
Andrew was again her partner. He reminded her of their first
encounter in the Otradnoe avenue, and how she had been unable to
sleep that moonlight night, and told her how he had involuntarily
overheard her. Natasha blushed at that recollection and tried to
excuse herself, as if there had been something to be ashamed of
in what Prince Andrew had overheard.

Like all men who have grown up in society, Prince Andrew liked
meeting someone there not of the conventional society stamp. And
such was Natasha, with her surprise, her delight, her shyness,
and even her mistakes in speaking French. With her he behaved
with special care and tenderness, sitting beside her and talking
of the simplest and most unimportant matters; he admired her shy
grace. In the middle of the cotillion, having completed one of
the figures, Natasha, still out of breath, was returning to her
seat when another dancer chose her. She was tired and panting and
evidently thought of declining, but immediately put her hand
gaily on the man's shoulder, smiling at Prince Andrew.

``I'd be glad to sit beside you and rest: I'm tired; but you see
how they keep asking me, and I'm glad of it, I'm happy and I love
everybody, and you and I understand it all,'' and much, much more
was said in her smile.  When her partner left her Natasha ran
across the room to choose two ladies for the figure.

``If she goes to her cousin first and then to another lady, she
will be my wife,'' said Prince Andrew to himself quite to his own
surprise, as he watched her. She did go first to her cousin.

``What rubbish sometimes enters one's head!'' thought Prince
Andrew, ``but what is certain is that that girl is so charming,
so original, that she won't be dancing here a month before she
will be married... Such as she are rare here,'' he thought, as
Natasha, readjusting a rose that was slipping on her bodice,
settled herself beside him.

When the cotillion was over the old count in his blue coat came
up to the dancers. He invited Prince Andrew to come and see them,
and asked his daughter whether she was enjoying herself. Natasha
did not answer at once but only looked up with a smile that said
reproachfully: ``How can you ask such a question?''

``I have never enjoyed myself so much before!'' she said, and
Prince Andrew noticed how her thin arms rose quickly as if to
embrace her father and instantly dropped again. Natasha was
happier than she had ever been in her life. She was at that
height of bliss when one becomes completely kind and good and
does not believe in the possibility of evil, unhappiness, or
sorrow.

At that ball Pierre for the first time felt humiliated by the
position his wife occupied in court circles. He was gloomy and
absent-minded. A deep furrow ran across his forehead, and
standing by a window he stared over his spectacles seeing no one.

On her way to supper Natasha passed him.

Pierre's gloomy, unhappy look struck her. She stopped in front of
him.  She wished to help him, to bestow on him the superabundance
of her own happiness.

``How delightful it is, Count!'' said she. ``Isn't it?''

Pierre smiled absent-mindedly, evidently not grasping what she
said.

``Yes, I am very glad,'' he said.

``How can people be dissatisfied with anything?'' thought
Natasha.  ``Especially such a capital fellow as Bezukhov!'' In
Natasha's eyes all the people at the ball alike were good, kind,
and splendid people, loving one another; none of them capable of
injuring another---and so they ought all to be happy.

% % % % % % % % % % % % % % % % % % % % % % % % % % % % % % % % %
% % % % % % % % % % % % % % % % % % % % % % % % % % % % % % % % %
% % % % % % % % % % % % % % % % % % % % % % % % % % % % % % % % %
% % % % % % % % % % % % % % % % % % % % % % % % % % % % % % % % %
% % % % % % % % % % % % % % % % % % % % % % % % % % % % % % % % %
% % % % % % % % % % % % % % % % % % % % % % % % % % % % % % % % %
% % % % % % % % % % % % % % % % % % % % % % % % % % % % % % % % %
% % % % % % % % % % % % % % % % % % % % % % % % % % % % % % % % %
% % % % % % % % % % % % % % % % % % % % % % % % % % % % % % % % %
% % % % % % % % % % % % % % % % % % % % % % % % % % % % % % % % %
% % % % % % % % % % % % % % % % % % % % % % % % % % % % % % % % %
% % % % % % % % % % % % % % % % % % % % % % % % % % % % % %

\chapter*{Chapter XVIII}
\ifaudio     
\marginpar{
\href{http://ia802205.us.archive.org/34/items/war_and_peace_06_0808_librivox/war_and_peace_06_18_tolstoy.mp3}{Audio}} 
\fi

\lettrine[lines=2, loversize=0.3, lraise=0]{\initfamily N}{ext}
day Prince Andrew thought of the ball, but his mind did not
dwell on it long. ``Yes, it was a very brilliant ball,'' and
then... ``Yes, that little Rostova is very charming. There's
something fresh, original, un-Petersburg-like about her that
distinguishes her.'' That was all he thought about yesterday's
ball, and after his morning tea he set to work.

But either from fatigue or want of sleep he was ill-disposed for
work and could get nothing done. He kept criticizing his own
work, as he often did, and was glad when he heard someone coming.

The visitor was Bitski, who served on various committees,
frequented all the societies in Petersburg, and a passionate
devotee of the new ideas and of Speranski, and a diligent
Petersburg newsmonger---one of those men who choose their
opinions like their clothes according to the fashion, but who for
that very reason appear to be the warmest partisans. Hardly had
he got rid of his hat before he ran into Prince Andrew's room
with a preoccupied air and at once began talking. He had just
heard particulars of that morning's sitting of the Council of
State opened by the Emperor, and he spoke of it
enthusiastically. The Emperor's speech had been extraordinary. It
had been a speech such as only constitutional monarchs
deliver. ``The Sovereign plainly said that the Council and Senate
are estates of the realm, he said that the government must rest
not on authority but on secure bases. The Emperor said that the
fiscal system must be reorganized and the accounts published,''
recounted Bitski, emphasizing certain words and opening his eyes
significantly.

``Ah, yes! Today's events mark an epoch, the greatest epoch in
our history,'' he concluded.

Prince Andrew listened to the account of the opening of the
Council of State, which he had so impatiently awaited and to
which he had attached such importance, and was surprised that
this event, now that it had taken place, did not affect him, and
even seemed quite insignificant. He listened with quiet irony to
Bitski's enthusiastic account of it. A very simple thought
occurred to him: ``What does it matter to me or to Bitski what
the Emperor was pleased to say at the Council? Can all that make
me any happier or better?''

And this simple reflection suddenly destroyed all the interest
Prince Andrew had felt in the impending reforms. He was going to
dine that evening at Speranski's, \emph{with only a few friends},
as the host had said when inviting him. The prospect of that
dinner in the intimate home circle of the man he so admired had
greatly interested Prince Andrew, especially as he had not yet
seen Speranski in his domestic surroundings, but now he felt
disinclined to go to it.

At the appointed hour, however, he entered the modest house
Speranski owned in the Taurida Gardens. In the parqueted dining
room this small house, remarkable for its extreme cleanliness
(suggesting that of a monastery), Prince Andrew, who was rather
late, found the friendly gathering of Speranski's intimate
acquaintances already assembled at five o'clock. There were no
ladies present except Speranski's little daughter (long-faced
like her father) and her governess. The other guests were
Gervais, Magnitski, and Stolypin. While still in the anteroom
Prince Andrew heard loud voices and a ringing staccato laugh---a
laugh such as one hears on the stage. Someone---it sounded like
Speranski---was distinctly ejaculating ha-ha-ha. Prince Andrew
had never before heard Speranski's famous laugh, and this
ringing, high-pitched laughter from a statesman made a strange
impression on him.

He entered the dining room. The whole company were standing
between two windows at a small table laid with
hors-d'oeuvres. Speranski, wearing a gray swallow-tail coat with
a star on the breast, and evidently still the same waistcoat and
high white stock he had worn at the meeting of the Council of
State, stood at the table with a beaming countenance. His guests
surrounded him. Magnitski, addressing himself to Speranski, was
relating an anecdote, and Speranski was laughing in advance at
what Magnitski was going to say. When Prince Andrew entered the
room Magnitski's words were again crowned by laughter. Stolypin
gave a deep bass guffaw as he munched a piece of bread and
cheese. Gervais laughed softly with a hissing chuckle, and
Speranski in a high-pitched staccato manner.

Still laughing, Speranski held out his soft white hand to Prince
Andrew.

``Very pleased to see you, Prince,'' he said. ``One moment...''
he went on, turning to Magnitski and interrupting his story. ``We
have agreed that this is a dinner for recreation, with not a word
about business!'' and turning again to the narrator he began to
laugh afresh.

Prince Andrew looked at the laughing Speranski with astonishment,
regret, and disillusionment. It seemed to him that this was not
Speranski but someone else. Everything that had formerly appeared
mysterious and fascinating in Speranski suddenly became plain and
unattractive.

At dinner the conversation did not cease for a moment and seemed
to consist of the contents of a book of funny anecdotes. Before
Magnitski had finished his story someone else was anxious to
relate something still funnier. Most of the anecdotes, if not
relating to the state service, related to people in the
service. It seemed that in this company the insignificance of
those people was so definitely accepted that the only possible
attitude toward them was one of good humored ridicule. Speranski
related how at the Council that morning a deaf dignitary, when
asked his opinion, replied that he thought so too.  Gervais gave
a long account of an official revision, remarkable for the
stupidity of everybody concerned. Stolypin, stuttering, broke
into the conversation and began excitedly talking of the abuses
that existed under the former order of things---threatening to
give a serious turn to the conversation. Magnitski starting
quizzing Stolypin about his vehemence. Gervais intervened with a
joke, and the talk reverted to its former lively tone.

Evidently Speranski liked to rest after his labors and find
amusement in a circle of friends, and his guests, understanding
his wish, tried to enliven him and amuse themselves. But their
gaiety seemed to Prince Andrew mirthless and
tiresome. Speranski's high-pitched voice struck him unpleasantly,
and the incessant laughter grated on him like a false
note. Prince Andrew did not laugh and feared that he would be a
damper on the spirits of the company, but no one took any notice
of his being out of harmony with the general mood. They all
seemed very gay.

He tried several times to join in the conversation, but his
remarks were tossed aside each time like a cork thrown out of the
water, and he could not jest with them.

There was nothing wrong or unseemly in what they said, it was
witty and might have been funny, but it lacked just that
something which is the salt of mirth, and they were not even
aware that such a thing existed.

After dinner Speranski's daughter and her governess rose. He
patted the little girl with his white hand and kissed her. And
that gesture, too, seemed unnatural to Prince Andrew.

The men remained at table over their port---English fashion. In
the midst of a conversation that was started about Napoleon's
Spanish affairs, which they all agreed in approving, Prince
Andrew began to express a contrary opinion. Speranski smiled and,
with an evident wish to prevent the conversation from taking an
unpleasant course, told a story that had no connection with the
previous conversation. For a few moments all were silent.

Having sat some time at table, Speranski corked a bottle of wine
and, remarking, ``Nowadays good wine rides in a carriage and
pair,'' passed it to the servant and got up. All rose and
continuing to talk loudly went into the drawing room. Two letters
brought by a courier were handed to Speranski and he took them to
his study. As soon as he had left the room the general merriment
stopped and the guests began to converse sensibly and quietly
with one another.

``Now for the recitation!'' said Speranski on returning from his
study. ``A wonderful talent!'' he said to Prince Andrew, and
Magnitski immediately assumed a pose and began reciting some
humorous verses in French which he had composed about various
well-known Petersburg people. He was interrupted several times by
applause. When the verses were finished Prince Andrew went up to
Speranski and took his leave.

``Where are you off to so early?'' asked Speranski.

``I promised to go to a reception.''

They said no more. Prince Andrew looked closely into those
mirrorlike, impenetrable eyes, and felt that it had been
ridiculous of him to have expected anything from Speranski and
from any of his own activities connected with him, or ever to
have attributed importance to what Speranski was doing. That
precise, mirthless laughter rang in Prince Andrew's ears long
after he had left the house.

When he reached home Prince Andrew began thinking of his life in
Petersburg during those last four months as if it were something
new. He recalled his exertions and solicitations, and the history
of his project of army reform, which had been accepted for
consideration and which they were trying to pass over in silence
simply because another, a very poor one, had already been
prepared and submitted to the Emperor. He thought of the meetings
of a committee of which Berg was a member. He remembered how
carefully and at what length everything relating to form and
procedure was discussed at those meetings, and how sedulously and
promptly all that related to the gist of the business was
evaded. He recalled his labors on the Legal Code, and how
painstakingly he had translated the articles of the Roman and
French codes into Russian, and he felt ashamed of himself. Then
he vividly pictured to himself Bogucharovo, his occupations in
the country, his journey to Ryazan; he remembered the peasants
and Dron the village elder, and mentally applying to them the
Personal Rights he had divided into paragraphs, he felt
astonished that he could have spent so much time on such useless
work.

% % % % % % % % % % % % % % % % % % % % % % % % % % % % % % % % %
% % % % % % % % % % % % % % % % % % % % % % % % % % % % % % % % %
% % % % % % % % % % % % % % % % % % % % % % % % % % % % % % % % %
% % % % % % % % % % % % % % % % % % % % % % % % % % % % % % % % %
% % % % % % % % % % % % % % % % % % % % % % % % % % % % % % % % %
% % % % % % % % % % % % % % % % % % % % % % % % % % % % % % % % %
% % % % % % % % % % % % % % % % % % % % % % % % % % % % % % % % %
% % % % % % % % % % % % % % % % % % % % % % % % % % % % % % % % %
% % % % % % % % % % % % % % % % % % % % % % % % % % % % % % % % %
% % % % % % % % % % % % % % % % % % % % % % % % % % % % % % % % %
% % % % % % % % % % % % % % % % % % % % % % % % % % % % % % % % %
% % % % % % % % % % % % % % % % % % % % % % % % % % % % % %

\chapter*{Chapter XIX}
\ifaudio     
\marginpar{
\href{http://ia802205.us.archive.org/34/items/war_and_peace_06_0808_librivox/war_and_peace_06_19_tolstoy.mp3}{Audio}} 
\fi

\lettrine[lines=2, loversize=0.3, lraise=0]{\initfamily N}{ext}
day Prince Andrew called at a few houses he had not visited
before, and among them at the Rostovs' with whom he had renewed
acquaintance at the ball. Apart from considerations of politeness
which demanded the call, he wanted to see that original, eager
girl who had left such a pleasant impression on his mind, in her
own home.

Natasha was one of the first to meet him. She was wearing a
dark-blue house dress in which Prince Andrew thought her even
prettier than in her ball dress. She and all the Rostov family
welcomed him as an old friend, simply and cordially. The whole
family, whom he had formerly judged severely, now seemed to him
to consist of excellent, simple, and kindly people. The old
count's hospitality and good nature, which struck one especially
in Petersburg as a pleasant surprise, were such that Prince
Andrew could not refuse to stay to dinner. ``Yes,'' he thought,
``they are capital people, who of course have not the slightest
idea what a treasure they possess in Natasha; but they are kindly
folk and form the best possible setting for this strikingly
poetic, charming girl, overflowing with life!''

In Natasha Prince Andrew was conscious of a strange world
completely alien to him and brimful of joys unknown to him, a
different world, that in the Otradnoe avenue and at the window
that moonlight night had already begun to disconcert him. Now
this world disconcerted him no longer and was no longer alien to
him, but he himself having entered it found in it a new
enjoyment.

After dinner Natasha, at Prince Andrew's request, went to the
clavichord and began singing. Prince Andrew stood by a window
talking to the ladies and listened to her. In the midst of a
phrase he ceased speaking and suddenly felt tears choking him, a
thing he had thought impossible for him. He looked at Natasha as
she sang, and something new and joyful stirred in his soul. He
felt happy and at the same time sad. He had absolutely nothing to
weep about yet he was ready to weep. What about?  His former
love? The little princess? His disillusionments?... His hopes for
the future?... Yes and no. The chief reason was a sudden, vivid
sense of the terrible contrast between something infinitely great
and illimitable within him and that limited and material
something that he, and even she, was. This contrast weighed on
and yet cheered him while she sang.

As soon as Natasha had finished she went up to him and asked how
he liked her voice. She asked this and then became confused,
feeling that she ought not to have asked it. He smiled, looking
at her, and said he liked her singing as he liked everything she
did.

Prince Andrew left the Rostovs' late in the evening. He went to
bed from habit, but soon realized that he could not sleep. Having
lit his candle he sat up in bed, then got up, then lay down again
not at all troubled by his sleeplessness: his soul was as fresh
and joyful as if he had stepped out of a stuffy room into God's
own fresh air. It did not enter his head that he was in love with
Natasha; he was not thinking about her, but only picturing her to
himself, and in consequence all life appeared in a new
light. ``Why do I strive, why do I toil in this narrow, confined
frame, when life, all life with all its joys, is open to me?''
said he to himself. And for the first time for a very long while
he began making happy plans for the future. He decided that he
must attend to his son's education by finding a tutor and putting
the boy in his charge, then he ought to retire from the service
and go abroad, and see England, Switzerland and Italy. ``I must
use my freedom while I feel so much strength and youth in me,''
he said to himself. ``Pierre was right when he said one must
believe in the possibility of happiness in order to be happy, and
now I do believe in it. Let the dead bury their dead, but while
one has life one must live and be happy!'' thought he.

% % % % % % % % % % % % % % % % % % % % % % % % % % % % % % % % %
% % % % % % % % % % % % % % % % % % % % % % % % % % % % % % % % %
% % % % % % % % % % % % % % % % % % % % % % % % % % % % % % % % %
% % % % % % % % % % % % % % % % % % % % % % % % % % % % % % % % %
% % % % % % % % % % % % % % % % % % % % % % % % % % % % % % % % %
% % % % % % % % % % % % % % % % % % % % % % % % % % % % % % % % %
% % % % % % % % % % % % % % % % % % % % % % % % % % % % % % % % %
% % % % % % % % % % % % % % % % % % % % % % % % % % % % % % % % %
% % % % % % % % % % % % % % % % % % % % % % % % % % % % % % % % %
% % % % % % % % % % % % % % % % % % % % % % % % % % % % % % % % %
% % % % % % % % % % % % % % % % % % % % % % % % % % % % % % % % %
% % % % % % % % % % % % % % % % % % % % % % % % % % % % % %

\chapter*{Chapter XX}
\ifaudio     
\marginpar{
\href{http://ia802205.us.archive.org/34/items/war_and_peace_06_0808_librivox/war_and_peace_06_20_tolstoy.mp3}{Audio}} 
\fi

\lettrine[lines=2, loversize=0.3, lraise=0]{\initfamily O}{ne}
morning Colonel Berg, whom Pierre knew as he knew everybody
in Moscow and Petersburg, came to see him. Berg arrived in an
immaculate brand-new uniform, with his hair pomaded and brushed
forward over his temples as the Emperor Alexander wore his hair.

``I have just been to see the countess, your wife. Unfortunately
she could not grant my request, but I hope, Count, I shall be
more fortunate with you,'' he said with a smile.

``What is it you wish, Colonel? I am at your service.''

``I have now quite settled in my new rooms, Count'' (Berg said
this with perfect conviction that this information could not but
be agreeable), ``and so I wish to arrange just a small party for
my own and my wife's friends.'' (He smiled still more
pleasantly.) ``I wished to ask the countess and you to do me the
honor of coming to tea and to supper.''

Only Countess Helene, considering the society of such people as
the Bergs beneath her, could be cruel enough to refuse such an
invitation.  Berg explained so clearly why he wanted to collect
at his house a small but select company, and why this would give
him pleasure, and why though he grudged spending money on cards
or anything harmful, he was prepared to run into some expense for
the sake of good society---that Pierre could not refuse, and
promised to come.

``But don't be late, Count, if I may venture to ask; about ten
minutes to eight, please. We shall make up a rubber. Our general
is coming. He is very good to me. We shall have supper, Count. So
you will do me the favor.''

Contrary to his habit of being late, Pierre on that day arrived
at the Bergs' house, not at ten but at fifteen minutes to eight.

Having prepared everything necessary for the party, the Bergs
were ready for their guests' arrival.

In their new, clean, and light study with its small busts and
pictures and new furniture sat Berg and his wife. Berg, closely
buttoned up in his new uniform, sat beside his wife explaining to
her that one always could and should be acquainted with people
above one, because only then does one get satisfaction from
acquaintances.

``You can get to know something, you can ask for something. See
how I managed from my first promotion.'' (Berg measured his life
not by years but by promotions.) ``My comrades are still
nobodies, while I am only waiting for a vacancy to command a
regiment, and have the happiness to be your husband.'' (He rose
and kissed Vera's hand, and on the way to her straightened out a
turned-up corner of the carpet.) ``And how have I obtained all
this? Chiefly by knowing how to choose my aquaintances. It goes
without saying that one must be conscientious and methodical.''

Berg smiled with a sense of his superiority over a weak woman,
and paused, reflecting that this dear wife of his was after all
but a weak woman who could not understand all that constitutes a
man's dignity, what it was \emph{ein Mann zu sein}.\footnote{To
be a man.} Vera at the same time smiling with a sense of
superiority over her good, conscientious husband, who all the
same understood life wrongly, as according to Vera all men
did. Berg, judging by his wife, thought all women weak and
foolish. Vera, judging only by her husband and generalizing from
that observation, supposed that all men, though they understand
nothing and are conceited and selfish, ascribe common sense to
themselves alone.

Berg rose and embraced his wife carefully, so as not to crush her
lace fichu for which he had paid a good price, kissing her
straight on the lips.

``The only thing is, we mustn't have children too soon,'' he
continued, following an unconscious sequence of ideas.

``Yes,'' answered Vera, ``I don't at all want that. We must live
for society.''

``Princess Yusupova wore one exactly like this,'' said Berg,
pointing to the fichu with a happy and kindly smile.

Just then Count Bezukhov was announced. Husband and wife glanced
at one another, both smiling with self-satisfaction, and each
mentally claiming the honor of this visit.

``This is what comes of knowing how to make acquaintances,''
thought Berg.  ``This is what comes of knowing how to conduct
oneself.''

``But please don't interrupt me when I am entertaining the
guests,'' said Vera, ``because I know what interests each of them
and what to say to different people.''

Berg smiled again.

``It can't be helped: men must sometimes have masculine
conversation,'' said he.

They received Pierre in their small, new drawing-room, where it
was impossible to sit down anywhere without disturbing its
symmetry, neatness, and order; so it was quite comprehensible and
not strange that Berg, having generously offered to disturb the
symmetry of an armchair or of the sofa for his dear guest, but
being apparently painfully undecided on the matter himself,
eventually left the visitor to settle the question of
selection. Pierre disturbed the symmetry by moving a chair for
himself, and Berg and Vera immediately began their evening party,
interrupting each other in their efforts to entertain their
guest.

Vera, having decided in her own mind that Pierre ought to be
entertained with conversation about the French embassy, at once
began accordingly.  Berg, having decided that masculine
conversation was required, interrupted his wife's remarks and
touched on the question of the war with Austria, and
unconsciously jumped from the general subject to personal
considerations as to the proposals made him to take part in the
Austrian campaign and the reasons why he had declined
them. Though the conversation was very incoherent and Vera was
angry at the intrusion of the masculine element, both husband and
wife felt with satisfaction that, even if only one guest was
present, their evening had begun very well and was as like as two
peas to every other evening party with its talk, tea, and lighted
candles.

Before long Boris, Berg's old comrade, arrived. There was a shade
of condescension and patronage in his treatment of Berg and
Vera. After Boris came a lady with the colonel, then the general
himself, then the Rostovs, and the party became unquestionably
exactly like all other evening parties. Berg and Vera could not
repress their smiles of satisfaction at the sight of all this
movement in their drawing room, at the sound of the disconnected
talk, the rustling of dresses, and the bowing and
scraping. Everything was just as everybody always has it,
especially so the general, who admired the apartment, patted Berg
on the shoulder, and with parental authority superintended the
setting out of the table for boston. The general sat down by
Count Ilya Rostov, who was next to himself the most important
guest. The old people sat with the old, the young with the young,
and the hostess at the tea table, on which stood exactly the same
kind of cakes in a silver cake basket as the Panins had at their
party. Everything was just as it was everywhere else.

% % % % % % % % % % % % % % % % % % % % % % % % % % % % % % % % %
% % % % % % % % % % % % % % % % % % % % % % % % % % % % % % % % %
% % % % % % % % % % % % % % % % % % % % % % % % % % % % % % % % %
% % % % % % % % % % % % % % % % % % % % % % % % % % % % % % % % %
% % % % % % % % % % % % % % % % % % % % % % % % % % % % % % % % %
% % % % % % % % % % % % % % % % % % % % % % % % % % % % % % % % %
% % % % % % % % % % % % % % % % % % % % % % % % % % % % % % % % %
% % % % % % % % % % % % % % % % % % % % % % % % % % % % % % % % %
% % % % % % % % % % % % % % % % % % % % % % % % % % % % % % % % %
% % % % % % % % % % % % % % % % % % % % % % % % % % % % % % % % %
% % % % % % % % % % % % % % % % % % % % % % % % % % % % % % % % %
% % % % % % % % % % % % % % % % % % % % % % % % % % % % % %

\chapter*{Chapter XXI}
\ifaudio     
\marginpar{
\href{http://ia802205.us.archive.org/34/items/war_and_peace_06_0808_librivox/war_and_peace_06_21_tolstoy.mp3}{Audio}} 
\fi

\lettrine[lines=2, loversize=0.3, lraise=0]{\initfamily P}{ierre}, 
as one of the principal guests, had to sit down to boston
with Count Rostov, the general, and the colonel. At the card
table he happened to be directly facing Natasha, and was struck
by a curious change that had come over her since the ball. She
was silent, and not only less pretty than at the ball, but only
redeemed from plainness by her look of gentle indifference to
everything around.

``What's the matter with her?'' thought Pierre, glancing at
her. She was sitting by her sister at the tea table, and
reluctantly, without looking at him, made some reply to Boris who
sat down beside her. After playing out a whole suit and to his
partner's delight taking five tricks, Pierre, hearing greetings
and the steps of someone who had entered the room while he was
picking up his tricks, glanced again at Natasha.

``What has happened to her?'' he asked himself with still greater
surprise.

Prince Andrew was standing before her, saying something to her
with a look of tender solicitude. She, having raised her head,
was looking up at him, flushed and evidently trying to master her
rapid breathing. And the bright glow of some inner fire that had
been suppressed was again alight in her. She was completely
transformed and from a plain girl had again become what she had
been at the ball.

Prince Andrew went up to Pierre, and the latter noticed a new and
youthful expression in his friend's face.

Pierre changed places several times during the game, sitting now
with his back to Natasha and now facing her, but during the whole
of the six rubbers he watched her and his friend.

``Something very important is happening between them,'' thought
Pierre, and a feeling that was both joyful and painful agitated
him and made him neglect the game.

After six rubbers the general got up, saying that it was no use
playing like that, and Pierre was released. Natasha on one side
was talking with Sonya and Boris, and Vera with a subtle smile
was saying something to Prince Andrew. Pierre went up to his
friend and, asking whether they were talking secrets, sat down
beside them. Vera, having noticed Prince Andrew's attentions to
Natasha, decided that at a party, a real evening party, subtle
allusions to the tender passion were absolutely necessary and,
seizing a moment when Prince Andrew was alone, began a
conversation with him about feelings in general and about her
sister. With so intellectual a guest as she considered Prince
Andrew to be, she felt that she had to employ her diplomatic
tact.

When Pierre went up to them he noticed that Vera was being
carried away by her self-satisfied talk, but that Prince Andrew
seemed embarrassed, a thing that rarely happened with him.

``What do you think?'' Vera was saying with an arch smile. ``You
are so discerning, Prince, and understand people's characters so
well at a glance. What do you think of Natalie? Could she be
constant in her attachments? Could she, like other women'' (Vera
meant herself), ``love a man once for all and remain true to him
forever? That is what I consider true love. What do you think,
Prince?''

``I know your sister too little,'' replied Prince Andrew, with a
sarcastic smile under which he wished to hide his embarrassment,
``to be able to solve so delicate a question, and then I have
noticed that the less attractive a woman is the more constant she
is likely to be,'' he added, and looked up at Pierre who was just
approaching them.

``Yes, that is true, Prince. In our days,'' continued
Vera---mentioning ``our days'' as people of limited intelligence
are fond of doing, imagining that they have discovered and
appraised the peculiarities of ``our days'' and that human
characteristics change with the times---``in our days a girl has
so much freedom that the pleasure of being courted often stifles
real feeling in her. And it must be confessed that Natalie is
very susceptible.'' This return to the subject of Natalie caused
Prince Andrew to knit his brows with discomfort: he was about to
rise, but Vera continued with a still more subtle smile:

``I think no one has been more courted than she,'' she went on,
``but till quite lately she never cared seriously for anyone. Now
you know, Count,'' she said to Pierre, ``even our dear cousin
Boris, who, between ourselves, was very far gone in the land of
tenderness...'' (alluding to a map of love much in vogue at that
time).

Prince Andrew frowned and remained silent.

``You are friendly with Boris, aren't you?'' asked Vera.

``Yes, I know him...''

``I expect he has told you of his childish love for Natasha?''

``Oh, there was childish love?'' suddenly asked Prince Andrew,
blushing unexpectedly.

``Yes, you know between cousins intimacy often leads to love. Le
cousinage est un dangereux voisinage.\footnote{''Cousinhood is a
dangerous neighborhood.``} Don't you think so?''

``Oh, undoubtedly!'' said Prince Andrew, and with sudden and
unnatural liveliness he began chaffing Pierre about the need to
be very careful with his fifty-year-old Moscow cousins, and in
the midst of these jesting remarks he rose, taking Pierre by the
arm, and drew him aside.

``Well?'' asked Pierre, seeing his friend's strange animation
with surprise, and noticing the glance he turned on Natasha as he
rose.

``I must... I must have a talk with you,'' said Prince
Andrew. ``You know that pair of women's gloves?'' (He referred to
the masonic gloves given to a newly initiated Brother to present
to the woman he loved.) ``I...  but no, I will talk to you later
on,'' and with a strange light in his eyes and restlessness in
his movements, Prince Andrew approached Natasha and sat down
beside her. Pierre saw how Prince Andrew asked her something and
how she flushed as she replied.

But at that moment Berg came to Pierre and began insisting that
he should take part in an argument between the general and the
colonel on the affairs in Spain.

Berg was satisfied and happy. The smile of pleasure never left
his face.  The party was very successful and quite like other
parties he had seen.  Everything was similar: the ladies' subtle
talk, the cards, the general raising his voice at the card table,
and the samovar and the tea cakes; only one thing was lacking
that he had always seen at the evening parties he wished to
imitate. They had not yet had a loud conversation among the men
and a dispute about something important and clever. Now the
general had begun such a discussion and so Berg drew Pierre to
it.

% % % % % % % % % % % % % % % % % % % % % % % % % % % % % % % % %
% % % % % % % % % % % % % % % % % % % % % % % % % % % % % % % % %
% % % % % % % % % % % % % % % % % % % % % % % % % % % % % % % % %
% % % % % % % % % % % % % % % % % % % % % % % % % % % % % % % % %
% % % % % % % % % % % % % % % % % % % % % % % % % % % % % % % % %
% % % % % % % % % % % % % % % % % % % % % % % % % % % % % % % % %
% % % % % % % % % % % % % % % % % % % % % % % % % % % % % % % % %
% % % % % % % % % % % % % % % % % % % % % % % % % % % % % % % % %
% % % % % % % % % % % % % % % % % % % % % % % % % % % % % % % % %
% % % % % % % % % % % % % % % % % % % % % % % % % % % % % % % % %
% % % % % % % % % % % % % % % % % % % % % % % % % % % % % % % % %
% % % % % % % % % % % % % % % % % % % % % % % % % % % % % %

\chapter*{Chapter XXII}
\ifaudio     
\marginpar{
\href{http://ia802205.us.archive.org/34/items/war_and_peace_06_0808_librivox/war_and_peace_06_22_tolstoy.mp3}{Audio}} 
\fi

\lettrine[lines=2, loversize=0.3, lraise=0]{\initfamily N}{ext}
day, having been invited by the count, Prince Andrew dined
with the Rostovs and spent the rest of the day there.

Everyone in the house realized for whose sake Prince Andrew came,
and without concealing it he tried to be with Natasha all
day. Not only in the soul of the frightened yet happy and
enraptured Natasha, but in the whole house, there was a feeling
of awe at something important that was bound to happen. The
countess looked with sad and sternly serious eyes at Prince
Andrew when he talked to Natasha and timidly started some
artificial conversation about trifles as soon as he looked her
way.  Sonya was afraid to leave Natasha and afraid of being in
the way when she was with them. Natasha grew pale, in a panic of
expectation, when she remained alone with him for a
moment. Prince Andrew surprised her by his timidity. She felt
that he wanted to say something to her but could not bring
himself to do so.

In the evening, when Prince Andrew had left, the countess went up
to Natasha and whispered: ``Well, what?''

``Mamma! For heaven's sake don't ask me anything now! One can't
talk about that,'' said Natasha.

But all the same that night Natasha, now agitated and now
frightened, lay a long time in her mother's bed gazing straight
before her. She told her how he had complimented her, how he told
her he was going abroad, asked her where they were going to spend
the summer, and then how he had asked her about Boris.

``But such a... such a... never happened to me before!'' she
said. ``Only I feel afraid in his presence. I am always afraid
when I'm with him. What does that mean? Does it mean that it's
the real thing? Yes? Mamma, are you asleep?''

``No, my love; I am frightened myself,'' answered her
mother. ``Now go!''

``All the same I shan't sleep. What silliness, to sleep! Mummy!
Mummy!  such a thing never happened to me before,'' she said,
surprised and alarmed at the feeling she was aware of in
herself. ``And could we ever have thought!...''

It seemed to Natasha that even at the time she first saw Prince
Andrew at Otradnoe she had fallen in love with him. It was as if
she feared this strange, unexpected happiness of meeting again
the very man she had then chosen (she was firmly convinced she
had done so) and of finding him, as it seemed, not indifferent to
her.

``And it had to happen that he should come specially to
Petersburg while we are here. And it had to happen that we should
meet at that ball. It is fate. Clearly it is fate that everything
led up to this! Already then, directly I saw him I felt something
peculiar.''

``What else did he say to you? What are those verses? Read
them...'' said her mother, thoughtfully, referring to some verses
Prince Andrew had written in Natasha's album.

``Mamma, one need not be ashamed of his being a widower?''

``Don't, Natasha! Pray to God. 'Marriages are made in
heaven,'\ '' said her mother.

``Darling Mummy, how I love you! How happy I am!'' cried Natasha,
shedding tears of joy and excitement and embracing her mother.

At that very time Prince Andrew was sitting with Pierre and
telling him of his love for Natasha and his firm resolve to make
her his wife.

That day Countess Helene had a reception at her house. The French
ambassador was there, and a foreign prince of the blood who had
of late become a frequent visitor of hers, and many brilliant
ladies and gentlemen. Pierre, who had come downstairs, walked
through the rooms and struck everyone by his preoccupied,
absent-minded, and morose air.

Since the ball he had felt the approach of a fit of nervous
depression and had made desperate efforts to combat it. Since the
intimacy of his wife with the royal prince, Pierre had
unexpectedly been made a gentleman of the bedchamber, and from
that time he had begun to feel oppressed and ashamed in court
society, and dark thoughts of the vanity of all things human came
to him oftener than before. At the same time the feeling he had
noticed between his protegee Natasha and Prince Andrew
accentuated his gloom by the contrast between his own position
and his friend's. He tried equally to avoid thinking about his
wife, and about Natasha and Prince Andrew; and again everything
seemed to him insignificant in comparison with eternity; again
the question: for what?  presented itself; and he forced himself
to work day and night at masonic labors, hoping to drive away the
evil spirit that threatened him. Toward midnight, after he had
left the countess' apartments, he was sitting upstairs in a
shabby dressing gown, copying out the original transaction of the
Scottish lodge of Freemasons at a table in his low room cloudy
with tobacco smoke, when someone came in. It was Prince Andrew.

``Ah, it's you!'' said Pierre with a preoccupied, dissatisfied
air. ``And I, you see, am hard at it.'' He pointed to his
manuscript book with that air of escaping from the ills of life
with which unhappy people look at their work.

Prince Andrew, with a beaming, ecstatic expression of renewed
life on his face, paused in front of Pierre and, not noticing his
sad look, smiled at him with the egotism of joy.

``Well, dear heart,'' said he, ``I wanted to tell you about it
yesterday and I have come to do so today. I never experienced
anything like it before. I am in love, my friend!''

Suddenly Pierre heaved a deep sigh and dumped his heavy person
down on the sofa beside Prince Andrew.

``With Natasha Rostova, yes?'' said he.

``Yes, yes! Who else should it be? I should never have believed
it, but the feeling is stronger than I. Yesterday I tormented
myself and suffered, but I would not exchange even that torment
for anything in the world, I have not lived till now. At last I
live, but I can't live without her! But can she love me?... I am
too old for her... Why don't you speak?''

``I? I? What did I tell you?'' said Pierre suddenly, rising and
beginning to pace up and down the room. ``I always thought
it... That girl is such a treasure... she is a rare girl... My
dear friend, I entreat you, don't philosophize, don't doubt,
marry, marry, marry... And I am sure there will not be a happier
man than you.''

``But what of her?''

``She loves you.''

``Don't talk rubbish...'' said Prince Andrew, smiling and looking
into Pierre's eyes.

``She does, I know,'' Pierre cried fiercely.

``But do listen,'' returned Prince Andrew, holding him by the
arm. ``Do you know the condition I am in? I must talk about it to
someone.''

``Well, go on, go on. I am very glad,'' said Pierre, and his face
really changed, his brow became smooth, and he listened gladly to
Prince Andrew. Prince Andrew seemed, and really was, quite a
different, quite a new man. Where was his spleen, his contempt
for life, his disillusionment? Pierre was the only person to whom
he made up his mind to speak openly; and to him he told all that
was in his soul. Now he boldly and lightly made plans for an
extended future, said he could not sacrifice his own happiness to
his father's caprice, and spoke of how he would either make his
father consent to this marriage and love her, or would do without
his consent; then he marveled at the feeling that had mastered
him as at something strange, apart from and independent of
himself.

``I should not have believed anyone who told me that I was
capable of such love,'' said Prince Andrew. ``It is not at all
the same feeling that I knew in the past. The whole world is now
for me divided into two halves: one half is she, and there all is
joy, hope, light: the other half is everything where she is not,
and there is all gloom and darkness...''

``Darkness and gloom,'' reiterated Pierre: ``yes, yes, I
understand that.''

``I cannot help loving the light, it is not my fault. And I am
very happy! You understand me? I know you are glad for my sake.''

``Yes, yes,'' Pierre assented, looking at his friend with a
touched and sad expression in his eyes. The brighter Prince
Andrew's lot appeared to him, the gloomier seemed his own.

% % % % % % % % % % % % % % % % % % % % % % % % % % % % % % % % %
% % % % % % % % % % % % % % % % % % % % % % % % % % % % % % % % %
% % % % % % % % % % % % % % % % % % % % % % % % % % % % % % % % %
% % % % % % % % % % % % % % % % % % % % % % % % % % % % % % % % %
% % % % % % % % % % % % % % % % % % % % % % % % % % % % % % % % %
% % % % % % % % % % % % % % % % % % % % % % % % % % % % % % % % %
% % % % % % % % % % % % % % % % % % % % % % % % % % % % % % % % %
% % % % % % % % % % % % % % % % % % % % % % % % % % % % % % % % %
% % % % % % % % % % % % % % % % % % % % % % % % % % % % % % % % %
% % % % % % % % % % % % % % % % % % % % % % % % % % % % % % % % %
% % % % % % % % % % % % % % % % % % % % % % % % % % % % % % % % %
% % % % % % % % % % % % % % % % % % % % % % % % % % % % % %

\chapter*{Chapter XXIII}
\ifaudio     
\marginpar{
\href{http://ia802205.us.archive.org/34/items/war_and_peace_06_0808_librivox/war_and_peace_06_23_tolstoy.mp3}{Audio}} 
\fi

\lettrine[lines=2, loversize=0.3, lraise=0]{\initfamily P}{rince}
Andrew needed his father's consent to his marriage, and to
obtain this he started for the country next day.

His father received his son's communication with external
composure, but inward wrath. He could not comprehend how anyone
could wish to alter his life or introduce anything new into it,
when his own life was already ending. ``If only they would let me
end my days as I want to,'' thought the old man, ``then they
might do as they please.'' With his son, however, he employed the
diplomacy he reserved for important occasions and, adopting a
quiet tone, discussed the whole matter.

In the first place the marriage was not a brilliant one as
regards birth, wealth, or rank. Secondly, Prince Andrew was no
longer as young as he had been and his health was poor (the old
man laid special stress on this), while she was very
young. Thirdly, he had a son whom it would be a pity to entrust
to a chit of a girl. ``Fourthly and finally,'' the father said,
looking ironically at his son, ``I beg you to put it off for a
year: go abroad, take a cure, look out as you wanted to for a
German tutor for Prince Nicholas. Then if your love or passion or
obstinacy---as you please---is still as great, marry! And that's
my last word on it.  Mind, the last...'' concluded the prince, in
a tone which showed that nothing would make him alter his
decision.

Prince Andrew saw clearly that the old man hoped that his
feelings, or his fiancee's, would not stand a year's test, or
that he (the old prince himself) would die before then, and he
decided to conform to his father's wish---to propose, and
postpone the wedding for a year.

Three weeks after the last evening he had spent with the Rostovs,
Prince Andrew returned to Petersburg.

Next day after her talk with her mother Natasha expected
Bolkonski all day, but he did not come. On the second and third
day it was the same.  Pierre did not come either and Natasha, not
knowing that Prince Andrew had gone to see his father, could not
explain his absence to herself.

Three weeks passed in this way. Natasha had no desire to go out
anywhere and wandered from room to room like a shadow, idle and
listless; she wept secretly at night and did not go to her mother
in the evenings. She blushed continually and was irritable. It
seemed to her that everybody knew about her disappointment and
was laughing at her and pitying her.  Strong as was her inward
grief, this wound to her vanity intensified her misery.

Once she came to her mother, tried to say something, and suddenly
began to cry. Her tears were those of an offended child who does
not know why it is being punished.

The countess began to soothe Natasha, who after first listening
to her mother's words, suddenly interrupted her:

``Leave off, Mamma! I don't think, and don't want to think about
it! He just came and then left off, left off...''

Her voice trembled, and she again nearly cried, but recovered and
went on quietly:

``And I don't at all want to get married. And I am afraid of him;
I have now become quite calm, quite calm.''

The day after this conversation Natasha put on the old dress
which she knew had the peculiar property of conducing to
cheerfulness in the mornings, and that day she returned to the
old way of life which she had abandoned since the ball. Having
finished her morning tea she went to the ballroom, which she
particularly liked for its loud resonance, and began singing her
solfeggio. When she had finished her first exercise she stood
still in the middle of the room and sang a musical phrase that
particularly pleased her. She listened joyfully (as though she
had not expected it) to the charm of the notes reverberating,
filling the whole empty ballroom, and slowly dying away; and all
at once she felt cheerful. ``What's the good of making so much of
it? Things are nice as it is,'' she said to herself, and she
began walking up and down the room, not stepping simply on the
resounding parquet but treading with each step from the heel to
the toe (she had on a new and favorite pair of shoes) and
listening to the regular tap of the heel and creak of the toe as
gladly as she had to the sounds of her own voice. Passing a
mirror she glanced into it. ``There, that's me!'' the expression
of her face seemed to say as she caught sight of herself. ``Well,
and very nice too!  I need nobody.''

A footman wanted to come in to clear away something in the room
but she would not let him, and having closed the door behind him
continued her walk. That morning she had returned to her favorite
mood---love of, and delight in, herself. ``How charming that
Natasha is!'' she said again, speaking as some third, collective,
male person. ``Pretty, a good voice, young, and in nobody's way
if only they leave her in peace.'' But however much they left her
in peace she could not now be at peace, and immediately felt
this.

In the hall the porch door opened, and someone asked, ``At
home?'' and then footsteps were heard. Natasha was looking at the
mirror, but did not see herself. She listened to the sounds in
the hall. When she saw herself, her face was pale. It was he. She
knew this for certain, though she hardly heard his voice through
the closed doors.

Pale and agitated, Natasha ran into the drawing room.

``Mamma! Bolkonski has come!'' she said. ``Mamma, it is awful, it
is unbearable! I don't want... to be tormented? What am I to
do?...''

Before the countess could answer, Prince Andrew entered the room
with an agitated and serious face. As soon as he saw Natasha his
face brightened. He kissed the countess' hand and Natasha's, and
sat down beside the sofa.

``It is long since we had the pleasure...'' began the countess,
but Prince Andrew interrupted her by answering her intended
question, obviously in haste to say what he had to.

``I have not been to see you all this time because I have been at
my father's. I had to talk over a very important matter with
him. I only got back last night,'' he said glancing at Natasha;
``I want to have a talk with you, Countess,'' he added after a
moment's pause.

The countess lowered her eyes, sighing deeply.

``I am at your disposal,'' she murmured.

Natasha knew that she ought to go away, but was unable to do so:
something gripped her throat, and regardless of manners she
stared straight at Prince Andrew with wide-open eyes.

``At once? This instant!... No, it can't be!'' she thought.

Again he glanced at her, and that glance convinced her that she
was not mistaken. Yes, at once, that very instant, her fate would
be decided.

``Go, Natasha! I will call you,'' said the countess in a whisper.

Natasha glanced with frightened imploring eyes at Prince Andrew
and at her mother and went out.

``I have come, Countess, to ask for your daughter's hand,'' said
Prince Andrew.

The countess' face flushed hotly, but she said nothing.

``Your offer...'' she began at last sedately. He remained silent,
looking into her eyes. ``Your offer...'' (she grew confused) ``is
agreeable to us, and I accept your offer. I am glad. And my
husband... I hope... but it will depend on her...''

``I will speak to her when I have your consent... Do you give it
to me?''  said Prince Andrew.

``Yes,'' replied the countess. She held out her hand to him, and
with a mixed feeling of estrangement and tenderness pressed her
lips to his forehead as he stooped to kiss her hand. She wished
to love him as a son, but felt that to her he was a stranger and
a terrifying man. ``I am sure my husband will consent,'' said the
countess, ``but your father...''

``My father, to whom I have told my plans, has made it an express
condition of his consent that the wedding is not to take place
for a year. And I wished to tell you of that,'' said Prince
Andrew.

``It is true that Natasha is still young, but---so long as
that?...''

``It is unavoidable,'' said Prince Andrew with a sigh.

``I will send her to you,'' said the countess, and left the room.

``Lord have mercy upon us!'' she repeated while seeking her
daughter.

Sonya said that Natasha was in her bedroom. Natasha was sitting
on the bed, pale and dry eyed, and was gazing at the icons and
whispering something as she rapidly crossed herself. Seeing her
mother she jumped up and flew to her.

``Well, Mamma?... Well?...''

``Go, go to him. He is asking for your hand,'' said the countess,
coldly it seemed to Natasha. ``Go... go,'' said the mother, sadly
and reproachfully, with a deep sigh, as her daughter ran away.

Natasha never remembered how she entered the drawing room. When
she came in and saw him she paused. ``Is it possible that this
stranger has now become everything to me?'' she asked herself,
and immediately answered, ``Yes, everything! He alone is now
dearer to me than everything in the world.'' Prince Andrew came
up to her with downcast eyes.

``I have loved you from the very first moment I saw you. May I
hope?''

He looked at her and was struck by the serious impassioned
expression of her face. Her face said: ``Why ask? Why doubt what
you cannot but know?  Why speak, when words cannot express what
one feels?''

She drew near to him and stopped. He took her hand and kissed it.

``Do you love me?''

``Yes, yes!'' Natasha murmured as if in vexation. Then she sighed
loudly and, catching her breath more and more quickly, began to
sob.

``What is it? What's the matter?''

``Oh, I am so happy!'' she replied, smiled through her tears,
bent over closer to him, paused for an instant as if asking
herself whether she might, and then kissed him.

Prince Andrew held her hands, looked into her eyes, and did not
find in his heart his former love for her. Something in him had
suddenly changed; there was no longer the former poetic and
mystic charm of desire, but there was pity for her feminine and
childish weakness, fear at her devotion and trustfulness, and an
oppressive yet joyful sense of the duty that now bound him to her
forever. The present feeling, though not so bright and poetic as
the former, was stronger and more serious.

``Did your mother tell you that it cannot be for a year?'' asked
Prince Andrew, still looking into her eyes.

``Is it possible that I---the 'chit of a girl,' as everybody
called me,'' thought Natasha---``is it possible that I am now to
be the wife and the equal of this strange, dear, clever man whom
even my father looks up to?  Can it be true? Can it be true that
there can be no more playing with life, that now I am grown up,
that on me now lies a responsibility for my every word and deed?
Yes, but what did he ask me?''

``No,'' she replied, but she had not understood his question.

``Forgive me!'' he said. ``But you are so young, and I have
already been through so much in life. I am afraid for you, you do
not yet know yourself.''

Natasha listened with concentrated attention, trying but failing
to take in the meaning of his words.

``Hard as this year which delays my happiness will be,''
continued Prince Andrew, ``it will give you time to be sure of
yourself. I ask you to make me happy in a year, but you are free:
our engagement shall remain a secret, and should you find that
you do not love me, or should you come to love...'' said Prince
Andrew with an unnatural smile.

``Why do you say that?'' Natasha interrupted him. ``You know that
from the very day you first came to Otradnoe I have loved you,''
she cried, quite convinced that she spoke the truth.

``In a year you will learn to know yourself...''

``A whole year!'' Natasha repeated suddenly, only now realizing
that the marriage was to be postponed for a year. ``But why a
year? Why a year?...''

Prince Andrew began to explain to her the reasons for this delay.
Natasha did not hear him.

``And can't it be helped?'' she asked. Prince Andrew did not
reply, but his face expressed the impossibility of altering that
decision.

``It's awful! Oh, it's awful! awful!'' Natasha suddenly cried,
and again burst into sobs. ``I shall die, waiting a year: it's
impossible, it's awful!'' She looked into her lover's face and
saw in it a look of commiseration and perplexity.

``No, no! I'll do anything!'' she said, suddenly checking her
tears. ``I am so happy.''

The father and mother came into the room and gave the betrothed
couple their blessing.

From that day Prince Andrew began to frequent the Rostovs' as
Natasha's affianced lover.

% % % % % % % % % % % % % % % % % % % % % % % % % % % % % % % % %
% % % % % % % % % % % % % % % % % % % % % % % % % % % % % % % % %
% % % % % % % % % % % % % % % % % % % % % % % % % % % % % % % % %
% % % % % % % % % % % % % % % % % % % % % % % % % % % % % % % % %
% % % % % % % % % % % % % % % % % % % % % % % % % % % % % % % % %
% % % % % % % % % % % % % % % % % % % % % % % % % % % % % % % % %
% % % % % % % % % % % % % % % % % % % % % % % % % % % % % % % % %
% % % % % % % % % % % % % % % % % % % % % % % % % % % % % % % % %
% % % % % % % % % % % % % % % % % % % % % % % % % % % % % % % % %
% % % % % % % % % % % % % % % % % % % % % % % % % % % % % % % % %
% % % % % % % % % % % % % % % % % % % % % % % % % % % % % % % % %
% % % % % % % % % % % % % % % % % % % % % % % % % % % % % %

\chapter*{Chapter XXIV}
\ifaudio
\marginpar{
\href{http://ia802205.us.archive.org/34/items/war_and_peace_06_0808_librivox/war_and_peace_06_24_tolstoy.mp3}{Audio}} 
\fi

\lettrine[lines=2, loversize=0.3, lraise=0]{\initfamily N}{o}
betrothal ceremony took place and Natasha's engagement to
Bolkonski was not announced; Prince Andrew insisted on that. He
said that as he was responsible for the delay he ought to bear
the whole burden of it; that he had given his word and bound
himself forever, but that he did not wish to bind Natasha and
gave her perfect freedom. If after six months she felt that she
did not love him she would have full right to reject
him. Naturally neither Natasha nor her parents wished to hear of
this, but Prince Andrew was firm. He came every day to the
Rostovs', but did not behave to Natasha as an affianced lover: he
did not use the familiar thou, but said you to her, and kissed
only her hand. After their engagement, quite different, intimate,
and natural relations sprang up between them. It was as if they
had not known each other till now. Both liked to recall how they
had regarded each other when as yet they were nothing to one
another; they felt themselves now quite different beings: then
they were artificial, now natural and sincere. At first the
family felt some constraint in intercourse with Prince Andrew; he
seemed a man from another world, and for a long time Natasha
trained the family to get used to him, proudly assuring them all
that he only appeared to be different, but was really just like
all of them, and that she was not afraid of him and no one else
ought to be. After a few days they grew accustomed to him, and
without restraint in his presence pursued their usual way of
life, in which he took his part. He could talk about rural
economy with the count, fashions with the countess and Natasha,
and about albums and fancywork with Sonya. Sometimes the
household both among themselves and in his presence expressed
their wonder at how it had all happened, and at the evident omens
there had been of it: Prince Andrew's coming to Otradnoe and
their coming to Petersburg, and the likeness between Natasha and
Prince Andrew which her nurse had noticed on his first visit, and
Andrew's encounter with Nicholas in 1805, and many other
incidents betokening that it had to be.

In the house that poetic dullness and quiet reigned which always
accompanies the presence of a betrothed couple. Often when all
sitting together everyone kept silent. Sometimes the others would
get up and go away and the couple, left alone, still remained
silent. They rarely spoke of their future life. Prince Andrew was
afraid and ashamed to speak of it. Natasha shared this as she did
all his feelings, which she constantly divined. Once she began
questioning him about his son. Prince Andrew blushed, as he often
did now---Natasha particularly liked it in him---and said that
his son would not live with them.

``Why not?'' asked Natasha in a frightened tone.

``I cannot take him away from his grandfather, and besides...''

``How I should have loved him!'' said Natasha, immediately
guessing his thought; ``but I know you wish to avoid any pretext
for finding fault with us.''

Sometimes the old count would come up, kiss Prince Andrew, and
ask his advice about Petya's education or Nicholas' service. The
old countess sighed as she looked at them; Sonya was always
getting frightened lest she should be in the way and tried to
find excuses for leaving them alone, even when they did not wish
it. When Prince Andrew spoke (he could tell a story very well),
Natasha listened to him with pride; when she spoke she noticed
with fear and joy that he gazed attentively and scrutinizingly at
her. She asked herself in perplexity: ``What does he look for in
me? He is trying to discover something by looking at me!  What if
what he seeks in me is not there?'' Sometimes she fell into one
of the mad, merry moods characteristic of her, and then she
particularly loved to hear and see how Prince Andrew laughed. He
seldom laughed, but when he did he abandoned himself entirely to
his laughter, and after such a laugh she always felt nearer to
him. Natasha would have been completely happy if the thought of
the separation awaiting her and drawing near had not terrified
her, just as the mere thought of it made him turn pale and cold.

On the eve of his departure from Petersburg Prince Andrew brought
with him Pierre, who had not been to the Rostovs' once since the
ball. Pierre seemed disconcerted and embarrassed. He was talking
to the countess, and Natasha sat down beside a little chess table
with Sonya, thereby inviting Prince Andrew to come too. He did
so.

``You have known Bezukhov a long time?'' he asked. ``Do you like
him?''

``Yes, he's a dear, but very absurd.''

And as usual when speaking of Pierre, she began to tell anecdotes
of his absent-mindedness, some of which had even been invented
about him.

``Do you know I have entrusted him with our secret? I have known
him from childhood. He has a heart of gold. I beg you, Natalie,''
Prince Andrew said with sudden seriousness---``I am going away
and heaven knows what may happen. You may cease to... all right,
I know I am not to say that. Only this, then: whatever may happen
to you when I am not here...''

``What can happen?''

``Whatever trouble may come,'' Prince Andrew continued, ``I beg
you, Mademoiselle Sophie, whatever may happen, to turn to him
alone for advice and help! He is a most absent-minded and absurd
fellow, but he has a heart of gold.''

Neither her father, nor her mother, nor Sonya, nor Prince Andrew
himself could have foreseen how the separation from her lover
would act on Natasha. Flushed and agitated she went about the
house all that day, dry-eyed, occupied with most trivial matters
as if not understanding what awaited her. She did not even cry
when, on taking leave, he kissed her hand for the last
time. ``Don't go!'' she said in a tone that made him wonder
whether he really ought not to stay and which he remembered long
afterwards. Nor did she cry when he was gone; but for several
days she sat in her room dry-eyed, taking no interest in anything
and only saying now and then, ``Oh, why did he go away?''

But a fortnight after his departure, to the surprise of those
around her, she recovered from her mental sickness just as
suddenly and became her old self again, but with a change in her
moral physiognomy, as a child gets up after a long illness with a
changed expression of face.

% % % % % % % % % % % % % % % % % % % % % % % % % % % % % % % % %
% % % % % % % % % % % % % % % % % % % % % % % % % % % % % % % % %
% % % % % % % % % % % % % % % % % % % % % % % % % % % % % % % % %
% % % % % % % % % % % % % % % % % % % % % % % % % % % % % % % % %
% % % % % % % % % % % % % % % % % % % % % % % % % % % % % % % % %
% % % % % % % % % % % % % % % % % % % % % % % % % % % % % % % % %
% % % % % % % % % % % % % % % % % % % % % % % % % % % % % % % % %
% % % % % % % % % % % % % % % % % % % % % % % % % % % % % % % % %
% % % % % % % % % % % % % % % % % % % % % % % % % % % % % % % % %
% % % % % % % % % % % % % % % % % % % % % % % % % % % % % % % % %
% % % % % % % % % % % % % % % % % % % % % % % % % % % % % % % % %
% % % % % % % % % % % % % % % % % % % % % % % % % % % % % %

\chapter*{Chapter XXV}
\ifaudio
\marginpar{
\href{http://ia802205.us.archive.org/34/items/war_and_peace_06_0808_librivox/war_and_peace_06_25_tolstoy.mp3}{Audio}} 
\fi

\lettrine[lines=2, loversize=0.3, lraise=0]{\initfamily D}{uring}
that year after his son's departure, Prince Nicholas
Bolkonski's health and temper became much worse. He grew still
more irritable, and it was Princess Mary who generally bore the
brunt of his frequent fits of unprovoked anger. He seemed
carefully to seek out her tender spots so as to torture her
mentally as harshly as possible. Princess Mary had two passions
and consequently two joys---her nephew, little Nicholas, and
religion---and these were the favorite subjects of the prince's
attacks and ridicule. Whatever was spoken of he would bring round
to the superstitiousness of old maids, or the petting and
spoiling of children.  ``You want to make him''---little
Nicholas---``into an old maid like yourself! A pity! Prince
Andrew wants a son and not an old maid,'' he would say. Or,
turning to Mademoiselle Bourienne, he would ask her in Princess
Mary's presence how she liked our village priests and icons and
would joke about them.

He continually hurt Princess Mary's feelings and tormented her,
but it cost her no effort to forgive him. Could he be to blame
toward her, or could her father, whom she knew loved her in spite
of it all, be unjust?  And what is justice? The princess never
thought of that proud word \emph{justice}. All the complex laws of
man centered for her in one clear and simple law---the law of
love and self-sacrifice taught us by Him who lovingly suffered
for mankind though He Himself was God. What had she to do with
the justice or injustice of other people? She had to endure and
love, and that she did.

During the winter Prince Andrew had come to Bald Hills and had
been gay, gentle, and more affectionate than Princess Mary had
known him for a long time past. She felt that something had
happened to him, but he said nothing to her about his
love. Before he left he had a long talk with his father about
something, and Princess Mary noticed that before his departure
they were dissatisfied with one another.

Soon after Prince Andrew had gone, Princess Mary wrote to her
friend Julie Karagina in Petersburg, whom she had dreamed (as all
girls dream) of marrying to her brother, and who was at that time
in mourning for her own brother, killed in Turkey.

\begin{quote} \calli

Sorrow, it seems, is our common lot, my dear, tender friend
Julie.

Your loss is so terrible that I can only explain it to myself as
a special providence of God who, loving you, wishes to try you
and your excellent mother. Oh, my friend! Religion, and religion
alone, can---I will not say comfort us---but save us from
despair. Religion alone can explain to us what without its help
man cannot comprehend: why, for what cause, kind and noble beings
able to find happiness in life---not merely harming no one but
necessary to the happiness of others---are called away to God,
while cruel, useless, harmful persons, or such as are a burden to
themselves and to others, are left living. The first death I saw,
and one I shall never forget---that of my dear
sister-in-law---left that impression on me. Just as you ask
destiny why your splendid brother had to die, so I asked why that
angel Lise, who not only never wronged anyone, but in whose soul
there were never any unkind thoughts, had to die. And what do you
think, dear friend? Five years have passed since then, and
already I, with my petty understanding, begin to see clearly why
she had to die, and in what way that death was but an expression
of the infinite goodness of the Creator, whose every action,
though generally incomprehensible to us, is but a manifestation
of His infinite love for His creatures. Perhaps, I often think,
she was too angelically innocent to have the strength to perform
all a mother's duties. As a young wife she was irreproachable;
perhaps she could not have been so as a mother. As it is, not
only has she left us, and particularly Prince Andrew, with the
purest regrets and memories, but probably she will there receive
a place I dare not hope for myself. But not to speak of her
alone, that early and terrible death has had the most beneficent
influence on me and on my brother in spite of all our
grief. Then, at the moment of our loss, these thoughts could not
occur to me; I should then have dismissed them with horror, but
now they are very clear and certain. I write all this to you,
dear friend, only to convince you of the Gospel truth which has
become for me a principle of life: not a single hair of our heads
will fall without His will. And His will is governed only by
infinite love for us, and so whatever befalls us is for our good.

You ask whether we shall spend next winter in Moscow. In spite of
my wish to see you, I do not think so and do not want to do
so. You will be surprised to hear that the reason for this is
Buonaparte! The case is this: my father's health is growing
noticeably worse, he cannot stand any contradiction and is
becoming irritable. This irritability is, as you know, chiefly
directed to political questions. He cannot endure the notion that
Buonaparte is negotiating on equal terms with all the sovereigns
of Europe and particularly with our own, the grandson of the
Great Catherine! As you know, I am quite indifferent to politics,
but from my father's remarks and his talks with Michael Ivanovich
I know all that goes on in the world and especially about the
honors conferred on Buonaparte, who only at Bald Hills in the
whole world, it seems, is not accepted as a great man, still less
as Emperor of France. And my father cannot stand this. It seems
to me that it is chiefly because of his political views that my
father is reluctant to speak of going to Moscow; for he foresees
the encounters that would result from his way of expressing his
views regardless of anybody. All the benefit he might derive from
a course of treatment he would lose as a result of the disputes
about Buonaparte which would be inevitable. In any case it will
be decided very shortly.

Our family life goes on in the old way except for my brother
Andrew's absence. He, as I wrote you before, has changed very
much of late. After his sorrow he only this year quite recovered
his spirits. He has again become as I used to know him when a
child: kind, affectionate, with that heart of gold to which I
know no equal. He has realized, it seems to me, that life is not
over for him. But together with this mental change he has grown
physically much weaker. He has become thinner and more nervous. I
am anxious about him and glad he is taking this trip abroad which
the doctors recommended long ago. I hope it will cure him. You
write that in Petersburg he is spoken of as one of the most
active, cultivated, and capable of the young men. Forgive my
vanity as a relation, but I never doubted it. The good he has
done to everybody here, from his peasants up to the gentry, is
incalculable. On his arrival in Petersburg he received only his
due. I always wonder at the way rumors fly from Petersburg to
Moscow, especially such false ones as that you write about---I
mean the report of my brother's betrothal to the little
Rostova. I do not think my brother will ever marry again, and
certainly not her; and this is why: first, I know that though he
rarely speaks about the wife he has lost, the grief of that loss
has gone too deep in his heart for him ever to decide to give her
a successor and our little angel a stepmother. Secondly because,
as far as I know, that girl is not the kind of girl who could
please Prince Andrew. I do not think he would choose her for a
wife, and frankly I do not wish it. But I am running on too long
and am at the end of my second sheet. Good-bye, my dear
friend. May God keep you in His holy and mighty care. My dear
friend, Mademoiselle Bourienne, sends you kisses.

\textsc{Mary}
\end{quote}

% % % % % % % % % % % % % % % % % % % % % % % % % % % % % % % % %
% % % % % % % % % % % % % % % % % % % % % % % % % % % % % % % % %
% % % % % % % % % % % % % % % % % % % % % % % % % % % % % % % % %
% % % % % % % % % % % % % % % % % % % % % % % % % % % % % % % % %
% % % % % % % % % % % % % % % % % % % % % % % % % % % % % % % % %
% % % % % % % % % % % % % % % % % % % % % % % % % % % % % % % % %
% % % % % % % % % % % % % % % % % % % % % % % % % % % % % % % % %
% % % % % % % % % % % % % % % % % % % % % % % % % % % % % % % % %
% % % % % % % % % % % % % % % % % % % % % % % % % % % % % % % % %
% % % % % % % % % % % % % % % % % % % % % % % % % % % % % % % % %
% % % % % % % % % % % % % % % % % % % % % % % % % % % % % % % % %
% % % % % % % % % % % % % % % % % % % % % % % % % % % % % %

\chapter*{Chapter XXVI}
\ifaudio     
\marginpar{
\href{http://ia802205.us.archive.org/34/items/war_and_peace_06_0808_librivox/war_and_peace_06_26_tolstoy.mp3}{Audio}} 
\fi

\lettrine[lines=2, loversize=0.3, lraise=0]{\initfamily I}{n}
the middle of the summer Princess Mary received an unexpected
letter from Prince Andrew in Switzerland in which he gave her
strange and surprising news. He informed her of his engagement to
Natasha Rostova.  The whole letter breathed loving rapture for
his betrothed and tender and confiding affection for his
sister. He wrote that he had never loved as he did now and that
only now did he understand and know what life was. He asked his
sister to forgive him for not having told her of his resolve when
he had last visited Bald Hills, though he had spoken of it to his
father. He had not done so for fear Princess Mary should ask her
father to give his consent, irritating him and having to bear the
brunt of his displeasure without attaining her
object. ``Besides,'' he wrote, ``the matter was not then so
definitely settled as it is now. My father then insisted on a
delay of a year and now already six months, half of that period,
have passed, and my resolution is firmer than ever. If the
doctors did not keep me here at the spas I should be back in
Russia, but as it is I have to postpone my return for three
months. You know me and my relations with Father. I want nothing
from him. I have been and always shall be independent; but to go
against his will and arouse his anger, now that he may perhaps
remain with us such a short time, would destroy half my
happiness. I am now writing to him about the same question, and
beg you to choose a good moment to hand him the letter and to let
me know how he looks at the whole matter and whether there is
hope that he may consent to reduce the term by four months.''

After long hesitations, doubts, and prayers, Princess Mary gave
the letter to her father. The next day the old prince said to her
quietly:

``Write and tell your brother to wait till I am dead... It won't
be long---I shall soon set him free.''

The princess was about to reply, but her father would not let her
speak and, raising his voice more and more, cried:

``Marry, marry, my boy!... A good family!... Clever people, eh?
Rich, eh?  Yes, a nice stepmother little Nicholas will have!
Write and tell him that he may marry tomorrow if he likes. She
will be little Nicholas' stepmother and I'll marry
Bourienne!... Ha, ha, ha! He mustn't be without a stepmother
either! Only one thing, no more women are wanted in my
house---let him marry and live by himself. Perhaps you will go
and live with him too?'' he added, turning to Princess Mary. ``Go
in heaven's name! Go out into the frost... the frost... the
frost!''

After this outburst the prince did not speak any more about the
matter.  But repressed vexation at his son's poor-spirited
behavior found expression in his treatment of his daughter. To
his former pretexts for irony a fresh one was now
added---allusions to stepmothers and amiabilities to Mademoiselle
Bourienne.

``Why shouldn't I marry her?'' he asked his daughter. ``She'll
make a splendid princess!''

And latterly, to her surprise and bewilderment, Princess Mary
noticed that her father was really associating more and more with
the Frenchwoman. She wrote to Prince Andrew about the reception
of his letter, but comforted him with hopes of reconciling their
father to the idea.

Little Nicholas and his education, her brother Andrew, and
religion were Princess Mary's joys and consolations; but besides
that, since everyone must have personal hopes, Princess Mary in
the profoundest depths of her heart had a hidden dream and hope
that supplied the chief consolation of her life. This comforting
dream and hope were given her by God's folk---the half-witted and
other pilgrims who visited her without the prince's
knowledge. The longer she lived, the more experience and
observation she had of life, the greater was her wonder at the
short-sightedness of men who seek enjoyment and happiness here on
earth: toiling, suffering, struggling, and harming one another,
to obtain that impossible, visionary, sinful happiness. Prince
Andrew had loved his wife, she died, but that was not enough: he
wanted to bind his happiness to another woman. Her father
objected to this because he wanted a more distinguished and
wealthier match for Andrew. And they all struggled and suffered
and tormented one another and injured their souls, their eternal
souls, for the attainment of benefits which endure but for an
instant. Not only do we know this ourselves, but Christ, the Son
of God, came down to earth and told us that this life is but for
a moment and is a probation; yet we cling to it and think to find
happiness in it. ``How is it that no one realizes this?'' thought
Princess Mary. ``No one except these despised God's folk who,
wallet on back, come to me by the back door, afraid of being seen
by the prince, not for fear of ill-usage by him but for fear of
causing him to sin. To leave family, home, and all the cares of
worldly welfare, in order without clinging to anything to wander
in hempen rags from place to place under an assumed name, doing
no one any harm but praying for all---for those who drive one
away as well as for those who protect one: higher than that life
and truth there is no life or truth!''

There was one pilgrim, a quiet pockmarked little woman of fifty
called Theodosia, who for over thirty years had gone about
barefoot and worn heavy chains. Princess Mary was particularly
fond of her. Once, when in a room with a lamp dimly lit before
the icon Theodosia was talking of her life, the thought that
Theodosia alone had found the true path of life suddenly came to
Princess Mary with such force that she resolved to become a
pilgrim herself. When Theodosia had gone to sleep Princess Mary
thought about this for a long time, and at last made up her mind
that, strange as it might seem, she must go on a pilgrimage. She
disclosed this thought to no one but to her confessor, Father
Akinfi, the monk, and he approved of her intention. Under guise
of a present for the pilgrims, Princess Mary prepared a pilgrim's
complete costume for herself: a coarse smock, bast shoes, a rough
coat, and a black kerchief.  Often, approaching the chest of
drawers containing this secret treasure, Princess Mary paused,
uncertain whether the time had not already come to put her
project into execution.

Often, listening to the pilgrims' tales, she was so stimulated by
their simple speech, mechanical to them but to her so full of
deep meaning, that several times she was on the point of
abandoning everything and running away from home. In imagination
she already pictured herself by Theodosia's side, dressed in
coarse rags, walking with a staff, a wallet on her back, along
the dusty road, directing her wanderings from one saint's shrine
to another, free from envy, earthly love, or desire, and reaching
at last the place where there is no more sorrow or sighing, but
eternal joy and bliss.

``I shall come to a place and pray there, and before having time
to get used to it or getting to love it, I shall go farther. I
will go on till my legs fail, and I'll lie down and die
somewhere, and shall at last reach that eternal, quiet haven,
where there is neither sorrow nor sighing...'' thought Princess
Mary.

But afterwards, when she saw her father and especially little
Koko (Nicholas), her resolve weakened. She wept quietly, and felt
that she was a sinner who loved her father and little nephew more
than God.