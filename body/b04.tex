\part*{Book Four: 1806}

% % % % % % % % % % % % % % % % % % % % % % % % % % % % % % % % %
% % % % % % % % % % % % % % % % % % % % % % % % % % % % % % % % %
% % % % % % % % % % % % % % % % % % % % % % % % % % % % % % % % %
% % % % % % % % % % % % % % % % % % % % % % % % % % % % % % % % %
% % % % % % % % % % % % % % % % % % % % % % % % % % % % % % % % %
% % % % % % % % % % % % % % % % % % % % % % % % % % % % % % % % %
% % % % % % % % % % % % % % % % % % % % % % % % % % % % % % % % %
% % % % % % % % % % % % % % % % % % % % % % % % % % % % % % % % %
% % % % % % % % % % % % % % % % % % % % % % % % % % % % % % % % %
% % % % % % % % % % % % % % % % % % % % % % % % % % % % % % % % %
% % % % % % % % % % % % % % % % % % % % % % % % % % % % % % % % %
% % % % % % % % % % % % % % % % % % % % % % % % % % % % % %

\chapter*{Chapter I}
\ifaudio
\marginpar{
\href{http://ia802609.us.archive.org/34/items/war_and_peace_04_0802_librivox/war_and_peace_04_01_tolstoy_64kb.mp3}{Audio}} 
\fi

\lettrine[lines=2, loversize=0.3, lraise=0]{\initfamily E}{arly}
in the year 1806 Nicholas Rostov returned home on
leave. Denisov was going home to Voronezh and Rostov persuaded
him to travel with him as far as Moscow and to stay with him
there. Meeting a comrade at the last post station but one before
Moscow, Denisov had drunk three bottles of wine with him and,
despite the jolting ruts across the snow-covered road, did not
once wake up on the way to Moscow, but lay at the bottom of the
sleigh beside Rostov, who grew more and more impatient the nearer
they got to Moscow.

``How much longer? How much longer? Oh, these insufferable
streets, shops, bakers' signboards, street lamps, and sleighs!''
thought Rostov, when their leave permits had been passed at the
town gate and they had entered Moscow.

``Denisov! We're here! He's asleep,'' he added, leaning forward
with his whole body as if in that position he hoped to hasten the
speed of the sleigh.

Denisov gave no answer.

``There's the corner at the crossroads, where the cabman, Zakhar,
has his stand, and there's Zakhar himself and still the same
horse! And here's the little shop where we used to buy
gingerbread! Can't you hurry up?  Now then!''

``Which house is it?'' asked the driver.

``Why, that one, right at the end, the big one. Don't you see?
That's our house,'' said Rostov. ``Of course, it's our house!
Denisov, Denisov! We're almost there!''

Denisov raised his head, coughed, and made no answer.

``Dmitri,'' said Rostov to his valet on the box, ``those lights
are in our house, aren't they?''

``Yes, sir, and there's a light in your father's study.''

``Then they've not gone to bed yet? What do you think? Mind now,
don't forget to put out my new coat,'' added Rostov, fingering
his new mustache. ``Now then, get on,'' he shouted to the
driver. ``Do wake up, Vaska!'' he went on, turning to Denisov,
whose head was again nodding.  ``Come, get on! You shall have
three rubles for vodka---get on!'' Rostov shouted, when the
sleigh was only three houses from his door. It seemed to him the
horses were not moving at all. At last the sleigh bore to the
right, drew up at an entrance, and Rostov saw overhead the old
familiar cornice with a bit of plaster broken off, the porch, and
the post by the side of the pavement. He sprang out before the
sleigh stopped, and ran into the hall. The house stood cold and
silent, as if quite regardless of who had come to it. There was
no one in the hall. ``Oh God! Is everyone all right?'' he
thought, stopping for a moment with a sinking heart, and then
immediately starting to run along the hall and up the warped
steps of the familiar staircase. The well-known old door handle,
which always angered the countess when it was not properly
cleaned, turned as loosely as ever. A solitary tallow candle
burned in the anteroom.

Old Michael was asleep on the chest. Prokofy, the footman, who
was so strong that he could lift the back of the carriage from
behind, sat plaiting slippers out of cloth selvedges. He looked
up at the opening door and his expression of sleepy indifference
suddenly changed to one of delighted amazement.

``Gracious heavens! The young count!'' he cried, recognizing his
young master. ``Can it be? My treasure!'' and Prokofy, trembling
with excitement, rushed toward the drawing-room door, probably in
order to announce him, but, changing his mind, came back and
stooped to kiss the young man's shoulder.

``All well?'' asked Rostov, drawing away his arm.

``Yes, God be thanked! Yes! They've just finished supper. Let me
have a look at you, your excellency.''

``Is everything quite all right?''

``The Lord be thanked, yes!''

Rostov, who had completely forgotten Denisov, not wishing anyone
to forestall him, threw off his fur coat and ran on tiptoe
through the large dark ballroom. All was the same: there were the
same old card tables and the same chandelier with a cover over
it; but someone had already seen the young master, and, before he
had reached the drawing room, something flew out from a side door
like a tornado and began hugging and kissing him. Another and yet
another creature of the same kind sprang from a second door and a
third; more hugging, more kissing, more outcries, and tears of
joy. He could not distinguish which was Papa, which Natasha, and
which Petya. Everyone shouted, talked, and kissed him at the same
time. Only his mother was not there, he noticed that.

``And I did not know... Nicholas... My darling!...''

``Here he is... our own... Kolya,\footnote{Nicholas.} dear
fellow... How he has changed!...  Where are the
candles?... Tea!...''

``And me, kiss me!''

``Dearest... and me!''

Sonya, Natasha, Petya, Anna Mikhaylovna, Vera, and the old count
were all hugging him, and the serfs, men and maids, flocked into
the room, exclaiming and oh-ing and ah-ing.

Petya, clinging to his legs, kept shouting, ``And me too!''

Natasha, after she had pulled him down toward her and covered his
face with kisses, holding him tight by the skirt of his coat,
sprang away and pranced up and down in one place like a goat and
shrieked piercingly.

All around were loving eyes glistening with tears of joy, and all
around were lips seeking a kiss.

Sonya too, all rosy red, clung to his arm and, radiant with
bliss, looked eagerly toward his eyes, waiting for the look for
which she longed. Sonya now was sixteen and she was very pretty,
especially at this moment of happy, rapturous excitement. She
gazed at him, not taking her eyes off him, and smiling and
holding her breath. He gave her a grateful look, but was still
expectant and looking for someone. The old countess had not yet
come. But now steps were heard at the door, steps so rapid that
they could hardly be his mother's.

Yet it was she, dressed in a new gown which he did not know, made
since he had left. All the others let him go, and he ran to
her. When they met, she fell on his breast, sobbing. She could
not lift her face, but only pressed it to the cold braiding of
his hussar's jacket. Denisov, who had come into the room
unnoticed by anyone, stood there and wiped his eyes at the sight.

``Vasili Denisov, your son's friend,'' he said, introducing
himself to the count, who was looking inquiringly at him.

``You are most welcome! I know, I know,'' said the count, kissing
and embracing Denisov. ``Nicholas wrote us... Natasha, Vera,
look! Here is Denisov!''

The same happy, rapturous faces turned to the shaggy figure of
Denisov.

``Darling Denisov!'' screamed Natasha, beside herself with
rapture, springing to him, putting her arms round him, and
kissing him. This escapade made everybody feel confused. Denisov
blushed too, but smiled and, taking Natasha's hand, kissed it.

Denisov was shown to the room prepared for him, and the Rostovs
all gathered round Nicholas in the sitting room.

The old countess, not letting go of his hand and kissing it every
moment, sat beside him: the rest, crowding round him, watched
every movement, word, or look of his, never taking their
blissfully adoring eyes off him. His brother and sisters
struggled for the places nearest to him and disputed with one
another who should bring him his tea, handkerchief, and pipe.

Rostov was very happy in the love they showed him; but the first
moment of meeting had been so beatific that his present joy
seemed insufficient, and he kept expecting something more, more
and yet more.

Next morning, after the fatigues of their journey, the travelers
slept till ten o'clock.

In the room next their bedroom there was a confusion of sabers,
satchels, sabretaches, open portmanteaus, and dirty boots. Two
freshly cleaned pairs with spurs had just been placed by the
wall. The servants were bringing in jugs and basins, hot water
for shaving, and their well-brushed clothes. There was a
masculine odor and a smell of tobacco.

``Hallo, Gwiska---my pipe!'' came Vasili Denisov's husky
voice. ``Wostov, get up!''

Rostov, rubbing his eyes that seemed glued together, raised his
disheveled head from the hot pillow.

``Why, is it late?''

``Late! It's nearly ten o'clock,'' answered Natasha's voice. A
rustle of starched petticoats and the whispering and laughter of
girls' voices came from the adjoining room. The door was opened a
crack and there was a glimpse of something blue, of ribbons,
black hair, and merry faces. It was Natasha, Sonya, and Petya,
who had come to see whether they were getting up.

``Nicholas! Get up!'' Natasha's voice was again heard at the
door.

``Directly!''

Meanwhile, Petya, having found and seized the sabers in the outer
room, with the delight boys feel at the sight of a military elder
brother, and forgetting that it was unbecoming for the girls to
see men undressed, opened the bedroom door.

``Is this your saber?'' he shouted.

The girls sprang aside. Denisov hid his hairy legs under the
blanket, looking with a scared face at his comrade for help. The
door, having let Petya in, closed again. A sound of laughter came
from behind it.

``Nicholas! Come out in your dressing gown!'' said Natasha's
voice.

``Is this your saber?'' asked Petya. ``Or is it yours?'' he said,
addressing the black-mustached Denisov with servile deference.

Rostov hurriedly put something on his feet, drew on his dressing
gown, and went out. Natasha had put on one spurred boot and was
just getting her foot into the other. Sonya, when he came in, was
twirling round and was about to expand her dresses into a balloon
and sit down. They were dressed alike, in new pale-blue frocks,
and were both fresh, rosy, and bright. Sonya ran away, but
Natasha, taking her brother's arm, led him into the sitting room,
where they began talking. They hardly gave one another time to
ask questions and give replies concerning a thousand little
matters which could not interest anyone but themselves. Natasha
laughed at every word he said or that she said herself, not
because what they were saying was amusing, but because she felt
happy and was unable to control her joy which expressed itself by
laughter.

``Oh, how nice, how splendid!'' she said to everything.

Rostov felt that, under the influence of the warm rays of love,
that childlike smile which had not once appeared on his face
since he left home now for the first time after eighteen months
again brightened his soul and his face.

``No, but listen,'' she said, ``now you are quite a man, aren't
you? I'm awfully glad you're my brother.'' She touched his
mustache. ``I want to know what you men are like. Are you the
same as we? No?''

``Why did Sonya run away?'' asked Rostov.

``Ah, yes! That's a whole long story! How are you going to speak
to her---thou or you?''

``As may happen,'' said Rostov.

``No, call her you, please! I'll tell you all about it some other
time.  No, I'll tell you now. You know Sonya's my dearest
friend. Such a friend that I burned my arm for her sake. Look
here!''

She pulled up her muslin sleeve and showed him a red scar on her
long, slender, delicate arm, high above the elbow on that part
that is covered even by a ball dress.

``I burned this to prove my love for her. I just heated a ruler
in the fire and pressed it there!''

Sitting on the sofa with the little cushions on its arms, in what
used to be his old schoolroom, and looking into Natasha's wildly
bright eyes, Rostov re-entered that world of home and childhood
which had no meaning for anyone else, but gave him some of the
best joys of his life; and the burning of an arm with a ruler as
a proof of love did not seem to him senseless, he understood and
was not surprised at it.

``Well, and is that all?'' he asked.

``We are such friends, such friends! All that ruler business was
just nonsense, but we are friends forever. She, if she loves
anyone, does it for life, but I don't understand that, I forget
quickly.''

``Well, what then?''

``Well, she loves me and you like that.''

Natasha suddenly flushed.

``Why, you remember before you went away?... Well, she says you
are to forget all that... She says: 'I shall love him always, but
let him be free.' Isn't that lovely and noble! Yes, very noble?
Isn't it?'' asked Natasha, so seriously and excitedly that it was
evident that what she was now saying she had talked of before,
with tears.

Rostov became thoughtful.

``I never go back on my word,'' he said. ``Besides, Sonya is so
charming that only a fool would renounce such happiness.''

``No, no!'' cried Natasha, ``she and I have already talked it
over. We knew you'd say so. But it won't do, because you see, if
you say that---if you consider yourself bound by your
promise---it will seem as if she had not meant it seriously. It
makes it as if you were marrying her because you must, and that
wouldn't do at all.''

Rostov saw that it had been well considered by them. Sonya had
already struck him by her beauty on the preceding day. Today,
when he had caught a glimpse of her, she seemed still more
lovely. She was a charming girl of sixteen, evidently
passionately in love with him (he did not doubt that for an
instant). Why should he not love her now, and even marry her,
Rostov thought, but just now there were so many other pleasures
and interests before him! ``Yes, they have taken a wise
decision,'' he thought, ``I must remain free.''

``Well then, that's excellent,'' said he. ``We'll talk it over
later on.  Oh, how glad I am to have you!''

``Well, and are you still true to Boris?'' he continued.

``Oh, what nonsense!'' cried Natasha, laughing. ``I don't think
about him or anyone else, and I don't want anything of the
kind.''

``Dear me! Then what are you up to now?''

``Now?'' repeated Natasha, and a happy smile lit up her
face. ``Have you seen Duport?''

``No.''

``Not seen Duport---the famous dancer? Well then, you won't
understand.  That's what I'm up to.''

Curving her arms, Natasha held out her skirts as dancers do, ran
back a few steps, turned, cut a caper, brought her little feet
sharply together, and made some steps on the very tips of her
toes.

``See, I'm standing! See!'' she said, but could not maintain
herself on her toes any longer. ``So that's what I'm up to! I'll
never marry anyone, but will be a dancer. Only don't tell
anyone.''

Rostov laughed so loud and merrily that Denisov, in his bedroom,
felt envious and Natasha could not help joining in.

``No, but don't you think it's nice?'' she kept repeating.

``Nice! And so you no longer wish to marry Boris?''

Natasha flared up. ``I don't want to marry anyone. And I'll tell
him so when I see him!''

``Dear me!'' said Rostov.

``But that's all rubbish,'' Natasha chattered on. ``And is
Denisov nice?''  she asked.

``Yes, indeed!''

``Oh, well then, good-by: go and dress. Is he very terrible,
Denisov?''

``Why terrible?'' asked Nicholas. ``No, Vaska is a splendid
fellow.''

``You call him Vaska? That's funny! And is he very nice?''

``Very.''

``Well then, be quick. We'll all have breakfast together.''

And Natasha rose and went out of the room on tiptoe, like a
ballet dancer, but smiling as only happy girls of fifteen can
smile. When Rostov met Sonya in the drawing room, he reddened. He
did not know how to behave with her. The evening before, in the
first happy moment of meeting, they had kissed each other, but
today they felt it could not be done; he felt that everybody,
including his mother and sisters, was looking inquiringly at him
and watching to see how he would behave with her. He kissed her
hand and addressed her not as thou but as you---Sonya.  But their
eyes met and said thou, and exchanged tender kisses. Her looks
asked him to forgive her for having dared, by Natasha's
intermediacy, to remind him of his promise, and then thanked him
for his love. His looks thanked her for offering him his freedom
and told her that one way or another he would never cease to love
her, for that would be impossible.

``How strange it is,'' said Vera, selecting a moment when all
were silent, ``that Sonya and Nicholas now say you to one another
and meet like strangers.''

Vera's remark was correct, as her remarks always were, but, like
most of her observations, it made everyone feel uncomfortable,
not only Sonya, Nicholas, and Natasha, but even the old countess,
who---dreading this love affair which might hinder Nicholas from
making a brilliant match---blushed like a girl.

Denisov, to Rostov's surprise, appeared in the drawing room with
pomaded hair, perfumed, and in a new uniform, looking just as
smart as he made himself when going into battle, and he was more
amiable to the ladies and gentlemen than Rostov had ever expected
to see him.

% % % % % % % % % % % % % % % % % % % % % % % % % % % % % % % % %
% % % % % % % % % % % % % % % % % % % % % % % % % % % % % % % % %
% % % % % % % % % % % % % % % % % % % % % % % % % % % % % % % % %
% % % % % % % % % % % % % % % % % % % % % % % % % % % % % % % % %
% % % % % % % % % % % % % % % % % % % % % % % % % % % % % % % % %
% % % % % % % % % % % % % % % % % % % % % % % % % % % % % % % % %
% % % % % % % % % % % % % % % % % % % % % % % % % % % % % % % % %
% % % % % % % % % % % % % % % % % % % % % % % % % % % % % % % % %
% % % % % % % % % % % % % % % % % % % % % % % % % % % % % % % % %
% % % % % % % % % % % % % % % % % % % % % % % % % % % % % % % % %
% % % % % % % % % % % % % % % % % % % % % % % % % % % % % % % % %
% % % % % % % % % % % % % % % % % % % % % % % % % % % % % %

\chapter*{Chapter II}
\ifaudio     
\marginpar{
\href{http://ia802609.us.archive.org/34/items/war_and_peace_04_0802_librivox/war_and_peace_04_02_tolstoy_64kb.mp3}{Audio}} 
\fi

\lettrine[lines=2, loversize=0.3, lraise=0]{\initfamily O}{n}
his return to Moscow from the army, Nicholas Rostov was
welcomed by his home circle as the best of sons, a hero, and
their darling Nikolenka; by his relations as a charming,
attractive, and polite young man; by his acquaintances as a
handsome lieutenant of hussars, a good dancer, and one of the
best matches in the city.

The Rostovs knew everybody in Moscow. The old count had money
enough that year, as all his estates had been remortgaged, and so
Nicholas, acquiring a trotter of his own, very stylish riding
breeches of the latest cut, such as no one else yet had in
Moscow, and boots of the latest fashion, with extremely pointed
toes and small silver spurs, passed his time very gaily. After a
short period of adapting himself to the old conditions of life,
Nicholas found it very pleasant to be at home again. He felt that
he had grown up and matured very much. His despair at failing in
a Scripture examination, his borrowing money from Gavril to pay a
sleigh driver, his kissing Sonya on the sly---he now recalled all
this as childishness he had left immeasurably behind. Now he was
a lieutenant of hussars, in a jacket laced with silver, and
wearing the Cross of St. George, awarded to soldiers for bravery
in action, and in the company of well-known, elderly, and
respected racing men was training a trotter of his own for a
race. He knew a lady on one of the boulevards whom he visited of
an evening. He led the mazurka at the Arkharovs' ball, talked
about the war with Field Marshal Kamenski, visited the English
Club, and was on intimate terms with a colonel of forty to whom
Denisov had introduced him.

His passion for the Emperor had cooled somewhat in Moscow. But
still, as he did not see him and had no opportunity of seeing
him, he often spoke about him and about his love for him, letting
it be understood that he had not told all and that there was
something in his feelings for the Emperor not everyone could
understand, and with his whole soul he shared the adoration then
common in Moscow for the Emperor, who was spoken of as the
\emph{angel incarnate}.

During Rostov's short stay in Moscow, before rejoining the army,
he did not draw closer to Sonya, but rather drifted away from
her. She was very pretty and sweet, and evidently deeply in love
with him, but he was at the period of youth when there seems so
much to do that there is no time for that sort of thing and a
young man fears to bind himself and prizes his freedom which he
needs for so many other things. When he thought of Sonya, during
this stay in Moscow, he said to himself, ``Ah, there will be, and
there are, many more such girls somewhere whom I do not yet
know. There will be time enough to think about love when I want
to, but now I have no time.'' Besides, it seemed to him that the
society of women was rather derogatory to his manhood. He went to
balls and into ladies' society with an affectation of doing so
against his will. The races, the English Club, sprees with
Denisov, and visits to a certain house---that was another matter
and quite the thing for a dashing young hussar!

At the beginning of March, old Count Ilya Rostov was very busy
arranging a dinner in honor of Prince Bagration at the English
Club.

The count walked up and down the hall in his dressing gown,
giving orders to the club steward and to the famous Feoktist, the
club's head cook, about asparagus, fresh cucumbers, strawberries,
veal, and fish for this dinner. The count had been a member and
on the committee of the club from the day it was founded. To him
the club entrusted the arrangement of the festival in honor of
Bagration, for few men knew so well how to arrange a feast on an
open-handed, hospitable scale, and still fewer men would be so
well able and willing to make up out of their own resources what
might be needed for the success of the fete.  The club cook and
the steward listened to the count's orders with pleased faces,
for they knew that under no other management could they so easily
extract a good profit for themselves from a dinner costing
several thousand rubles.

``Well then, mind and have cocks' comb in the turtle soup, you
know!''

``Shall we have three cold dishes then?'' asked the cook.

The count considered.

``We can't have less---yes, three... the mayonnaise, that's
one,'' said he, bending down a finger.

``Then am I to order those large sterlets?'' asked the steward.

``Yes, it can't be helped if they won't take less. Ah, dear me! I
was forgetting. We must have another entree. Ah, goodness
gracious!'' he clutched at his head. ``Who is going to get me the
flowers? Dmitri! Eh, Dmitri! Gallop off to our Moscow estate,''
he said to the factotum who appeared at his call. ``Hurry off and
tell Maksim, the gardener, to set the serfs to work. Say that
everything out of the hothouses must be brought here well wrapped
up in felt. I must have two hundred pots here on Friday.''

Having given several more orders, he was about to go to his
``little countess'' to have a rest, but remembering something
else of importance, he returned again, called back the cook and
the club steward, and again began giving orders. A light footstep
and the clinking of spurs were heard at the door, and the young
count, handsome, rosy, with a dark little mustache, evidently
rested and made sleeker by his easy life in Moscow, entered the
room.

``Ah, my boy, my head's in a whirl!'' said the old man with a
smile, as if he felt a little confused before his son. ``Now, if
you would only help a bit! I must have singers too. I shall have
my own orchestra, but shouldn't we get the gypsy singers as well?
You military men like that sort of thing.''

``Really, Papa, I believe Prince Bagration worried himself less
before the battle of Schon Grabern than you do now,'' said his
son with a smile.

The old count pretended to be angry.

``Yes, you talk, but try it yourself!''

And the count turned to the cook, who, with a shrewd and
respectful expression, looked observantly and sympathetically at
the father and son.

``What have the young people come to nowadays, eh, Feoktist?''
said he.  ``Laughing at us old fellows!''

``That's so, your excellency, all they have to do is to eat a
good dinner, but providing it and serving it all up, that's not
their business!''

``That's it, that's it!'' exclaimed the count, and gaily seizing
his son by both hands, he cried, ``Now I've got you, so take the
sleigh and pair at once, and go to Bezukhov's, and tell him
'Count Ilya has sent you to ask for strawberries and fresh
pineapples.' We can't get them from anyone else. He's not there
himself, so you'll have to go in and ask the princesses; and from
there go on to the Rasgulyay---the coachman Ipatka knows---and
look up the gypsy Ilyushka, the one who danced at Count Orlov's,
you remember, in a white Cossack coat, and bring him along to
me.''

``And am I to bring the gypsy girls along with him?'' asked
Nicholas, laughing. ``Dear, dear!...''

At that moment, with noiseless footsteps and with the
businesslike, preoccupied, yet meekly Christian look which never
left her face, Anna Mikhaylovna entered the hall. Though she came
upon the count in his dressing gown every day, he invariably
became confused and begged her to excuse his costume.

``No matter at all, my dear count,'' she said, meekly closing her
eyes.  ``But I'll go to Bezukhov's myself. Pierre has arrived,
and now we shall get anything we want from his hothouses. I have
to see him in any case.  He has forwarded me a letter from
Boris. Thank God, Boris is now on the staff.''

The count was delighted at Anna Mikhaylovna's taking upon herself
one of his commissions and ordered the small closed carriage for
her.

``Tell Bezukhov to come. I'll put his name down. Is his wife with
him?''  he asked.

Anna Mikhaylovna turned up her eyes, and profound sadness was
depicted on her face.

``Ah, my dear friend, he is very unfortunate,'' she said. ``If
what we hear is true, it is dreadful. How little we dreamed of
such a thing when we were rejoicing at his happiness! And such a
lofty angelic soul as young Bezukhov! Yes, I pity him from my
heart, and shall try to give him what consolation I can.''

``Wh-what is the matter?'' asked both the young and old Rostov.

Anna Mikhaylovna sighed deeply.

``Dolokhov, Mary Ivanovna's son,'' she said in a mysterious
whisper, ``has compromised her completely, they say. Pierre took
him up, invited him to his house in Petersburg, and now... she
has come here and that daredevil after her!'' said Anna
Mikhaylovna, wishing to show her sympathy for Pierre, but by
involuntary intonations and a half smile betraying her sympathy
for the ``daredevil,'' as she called Dolokhov. ``They say Pierre
is quite broken by his misfortune.''

``Dear, dear! But still tell him to come to the club---it will
all blow over. It will be a tremendous banquet.''

Next day, the third of March, soon after one o'clock, two hundred
and fifty members of the English Club and fifty guests were
awaiting the guest of honor and hero of the Austrian campaign,
Prince Bagration, to dinner.

On the first arrival of the news of the battle of Austerlitz,
Moscow had been bewildered. At that time, the Russians were so
used to victories that on receiving news of the defeat some would
simply not believe it, while others sought some extraordinary
explanation of so strange an event. In the English Club, where
all who were distinguished, important, and well informed
foregathered when the news began to arrive in December, nothing
was said about the war and the last battle, as though all were in
a conspiracy of silence. The men who set the tone in
conversation---Count Rostopchin, Prince Yuri Dolgorukov, Valuev,
Count Markov, and Prince Vyazemski---did not show themselves at
the club, but met in private houses in intimate circles, and the
Moscovites who took their opinions from others---Ilya Rostov
among them---remained for a while without any definite opinion on
the subject of the war and without leaders. The Moscovites felt
that something was wrong and that to discuss the bad news was
difficult, and so it was best to be silent. But after a while,
just as a jury comes out of its room, the bigwigs who guided the
club's opinion reappeared, and everybody began speaking clearly
and definitely. Reasons were found for the incredible,
unheard-of, and impossible event of a Russian defeat, everything
became clear, and in all corners of Moscow the same things began
to be said. These reasons were the treachery of the Austrians, a
defective commissariat, the treachery of the Pole Przebyszewski
and of the Frenchman Langeron, Kutuzov's incapacity, and (it was
whispered) the youth and inexperience of the sovereign, who had
trusted worthless and insignificant people.  But the army, the
Russian army, everyone declared, was extraordinary and had
achieved miracles of valor. The soldiers, officers, and generals
were heroes. But the hero of heroes was Prince Bagration,
distinguished by his Schon Grabern affair and by the retreat from
Austerlitz, where he alone had withdrawn his column unbroken and
had all day beaten back an enemy force twice as numerous as his
own. What also conduced to Bagration's being selected as Moscow's
hero was the fact that he had no connections in the city and was
a stranger there. In his person, honor was shown to a simple
fighting Russian soldier without connections and intrigues, and
to one who was associated by memories of the Italian campaign
with the name of Suvorov. Moreover, paying such honor to
Bagration was the best way of expressing disapproval and dislike
of Kutuzov.

``Had there been no Bagration, it would have been necessary to
invent him,'' said the wit Shinshin, parodying the words of
Voltaire. Kutuzov no one spoke of, except some who abused him in
whispers, calling him a court weathercock and an old satyr.

All Moscow repeated Prince Dolgorukov's saying: ``If you go on
modeling and modeling you must get smeared with clay,''
suggesting consolation for our defeat by the memory of former
victories; and the words of Rostopchin, that French soldiers have
to be incited to battle by highfalutin words, and Germans by
logical arguments to show them that it is more dangerous to run
away than to advance, but that Russian soldiers only need to be
restrained and held back! On all sides, new and fresh anecdotes
were heard of individual examples of heroism shown by our
officers and men at Austerlitz. One had saved a standard, another
had killed five Frenchmen, a third had loaded five cannon
singlehanded. Berg was mentioned, by those who did not know him,
as having, when wounded in the right hand, taken his sword in the
left, and gone forward. Of Bolkonski, nothing was said, and only
those who knew him intimately regretted that he had died so
young, leaving a pregnant wife with his eccentric father.

% % % % % % % % % % % % % % % % % % % % % % % % % % % % % % % % %
% % % % % % % % % % % % % % % % % % % % % % % % % % % % % % % % %
% % % % % % % % % % % % % % % % % % % % % % % % % % % % % % % % %
% % % % % % % % % % % % % % % % % % % % % % % % % % % % % % % % %
% % % % % % % % % % % % % % % % % % % % % % % % % % % % % % % % %
% % % % % % % % % % % % % % % % % % % % % % % % % % % % % % % % %
% % % % % % % % % % % % % % % % % % % % % % % % % % % % % % % % %
% % % % % % % % % % % % % % % % % % % % % % % % % % % % % % % % %
% % % % % % % % % % % % % % % % % % % % % % % % % % % % % % % % %
% % % % % % % % % % % % % % % % % % % % % % % % % % % % % % % % %
% % % % % % % % % % % % % % % % % % % % % % % % % % % % % % % % %
% % % % % % % % % % % % % % % % % % % % % % % % % % % % % %

\chapter*{Chapter III}
\ifaudio     
\marginpar{
\href{http://ia802609.us.archive.org/34/items/war_and_peace_04_0802_librivox/war_and_peace_04_03_tolstoy_64kb.mp3}{Audio}} 
\fi

\lettrine[lines=2, loversize=0.3, lraise=0]{\initfamily O}{n}
that third of March, all the rooms in the English Club were
filled with a hum of conversation, like the hum of bees swarming
in springtime.  The members and guests of the club wandered
hither and thither, sat, stood, met, and separated, some in
uniform and some in evening dress, and a few here and there with
powdered hair and in Russian kaftans.  Powdered footmen, in
livery with buckled shoes and smart stockings, stood at every
door anxiously noting visitors' every movement in order to offer
their services. Most of those present were elderly, respected men
with broad, self-confident faces, fat fingers, and resolute
gestures and voices. This class of guests and members sat in
certain habitual places and met in certain habitual groups. A
minority of those present were casual guests---chiefly young men,
among whom were Denisov, Rostov, and Dolokhov---who was now again
an officer in the Semenov regiment. The faces of these young
people, especially those who were military men, bore that
expression of condescending respect for their elders which seems
to say to the older generation, ``We are prepared to respect and
honor you, but all the same remember that the future belongs to
us.''

Nesvitski was there as an old member of the club. Pierre, who at
his wife's command had let his hair grow and abandoned his
spectacles, went about the rooms fashionably dressed but looking
sad and dull. Here, as elsewhere, he was surrounded by an
atmosphere of subservience to his wealth, and being in the habit
of lording it over these people, he treated them with
absent-minded contempt.

By his age he should have belonged to the younger men, but by his
wealth and connections he belonged to the groups of old and
honored guests, and so he went from one group to another. Some of
the most important old men were the center of groups which even
strangers approached respectfully to hear the voices of
well-known men. The largest circles formed round Count
Rostopchin, Valuev, and Naryshkin. Rostopchin was describing how
the Russians had been overwhelmed by flying Austrians and had had
to force their way through them with bayonets.

Valuev was confidentially telling that Uvarov had been sent from
Petersburg to ascertain what Moscow was thinking about
Austerlitz.

In the third circle, Naryshkin was speaking of the meeting of the
Austrian Council of War at which Suvorov crowed like a cock in
reply to the nonsense talked by the Austrian generals. Shinshin,
standing close by, tried to make a joke, saying that Kutuzov had
evidently failed to learn from Suvorov even so simple a thing as
the art of crowing like a cock, but the elder members glanced
severely at the wit, making him feel that in that place and on
that day, it was improper to speak so of Kutuzov.

Count Ilya Rostov, hurried and preoccupied, went about in his
soft boots between the dining and drawing rooms, hastily greeting
the important and unimportant, all of whom he knew, as if they
were all equals, while his eyes occasionally sought out his fine
well-set-up young son, resting on him and winking joyfully at
him. Young Rostov stood at a window with Dolokhov, whose
acquaintance he had lately made and highly valued. The old count
came up to them and pressed Dolokhov's hand.

``Please come and visit us... you know my brave boy... been
together out there... both playing the hero... Ah, Vasili
Ignatovich... How d'ye do, old fellow?'' he said, turning to an
old man who was passing, but before he had finished his greeting
there was a general stir, and a footman who had run in announced,
with a frightened face: ``He's arrived!''

Bells rang, the stewards rushed forward, and---like rye shaken
together in a shovel---the guests who had been scattered about in
different rooms came together and crowded in the large drawing
room by the door of the ballroom.

Bagration appeared in the doorway of the anteroom without hat or
sword, which, in accord with the club custom, he had given up to
the hall porter. He had no lambskin cap on his head, nor had he a
loaded whip over his shoulder, as when Rostov had seen him on the
eve of the battle of Austerlitz, but wore a tight new uniform
with Russian and foreign Orders, and the Star of St. George on
his left breast. Evidently just before coming to the dinner he
had had his hair and whiskers trimmed, which changed his
appearance for the worse. There was something naively festive in
his air, which, in conjunction with his firm and virile features,
gave him a rather comical expression. Bekleshev and Theodore
Uvarov, who had arrived with him, paused at the doorway to allow
him, as the guest of honor, to enter first. Bagration was
embarrassed, not wishing to avail himself of their courtesy, and
this caused some delay at the doors, but after all he did at last
enter first. He walked shyly and awkwardly over the parquet floor
of the reception room, not knowing what to do with his hands; he
was more accustomed to walk over a plowed field under fire, as he
had done at the head of the Kursk regiment at Schon Grabern---and
he would have found that easier. The committeemen met him at the
first door and, expressing their delight at seeing such a highly
honored guest, took possession of him as it were, without waiting
for his reply, surrounded him, and led him to the drawing
room. It was at first impossible to enter the drawing-room door
for the crowd of members and guests jostling one another and
trying to get a good look at Bagration over each other's
shoulders, as if he were some rare animal.  Count Ilya Rostov,
laughing and repeating the words, ``Make way, dear boy! Make way,
make way!'' pushed through the crowd more energetically than
anyone, led the guests into the drawing room, and seated them on
the center sofa. The bigwigs, the most respected members of the
club, beset the new arrivals. Count Ilya, again thrusting his way
through the crowd, went out of the drawing room and reappeared a
minute later with another committeeman, carrying a large silver
salver which he presented to Prince Bagration. On the salver lay
some verses composed and printed in the hero's honor. Bagration,
on seeing the salver, glanced around in dismay, as though seeking
help. But all eyes demanded that he should submit. Feeling
himself in their power, he resolutely took the salver with both
hands and looked sternly and reproachfully at the count who had
presented it to him. Someone obligingly took the dish from
Bagration (or he would, it seemed, have held it till evening and
have gone in to dinner with it) and drew his attention to the
verses.

``Well, I will read them, then!'' Bagration seemed to say, and,
fixing his weary eyes on the paper, began to read them with a
fixed and serious expression. But the author himself took the
verses and began reading them aloud. Bagration bowed his head and
listened:

Bring glory then to Alexander's reign And on the throne our Titus
shield. A dreaded foe be thou, kindhearted as a man, A Rhipheus
at home, a Caesar in the field! E'en fortunate Napoleon Knows by
experience, now, Bagration, And dare not Herculean Russians
trouble...

But before he had finished reading, a stentorian major-domo
announced that dinner was ready! The door opened, and from the
dining room came the resounding strains of the polonaise:

Conquest's joyful thunder waken, Triumph, valiant Russians,
now!...

and Count Rostov, glancing angrily at the author who went on
reading his verses, bowed to Bagration. Everyone rose, feeling
that dinner was more important than verses, and Bagration, again
preceding all the rest, went in to dinner. He was seated in the
place of honor between two Alexanders---Bekleshev and
Naryshkin---which was a significant allusion to the name of the
sovereign. Three hundred persons took their seats in the dining
room, according to their rank and importance: the more important
nearer to the honored guest, as naturally as water flows deepest
where the land lies lowest.

Just before dinner, Count Ilya Rostov presented his son to
Bagration, who recognized him and said a few words to him,
disjointed and awkward, as were all the words he spoke that day,
and Count Ilya looked joyfully and proudly around while Bagration
spoke to his son.

Nicholas Rostov, with Denisov and his new acquaintance, Dolokhov,
sat almost at the middle of the table. Facing them sat Pierre,
beside Prince Nesvitski. Count Ilya Rostov with the other members
of the committee sat facing Bagration and, as the very
personification of Moscow hospitality, did the honors to the
prince.

His efforts had not been in vain. The dinner, both the Lenten and
the other fare, was splendid, yet he could not feel quite at ease
till the end of the meal. He winked at the butler, whispered
directions to the footmen, and awaited each expected dish with
some anxiety. Everything was excellent. With the second course, a
gigantic sterlet (at sight of which Ilya Rostov blushed with
self-conscious pleasure), the footmen began popping corks and
filling the champagne glasses. After the fish, which made a
certain sensation, the count exchanged glances with the other
committeemen. ``There will be many toasts, it's time to begin,''
he whispered, and taking up his glass, he rose. All were silent,
waiting for what he would say.

``To the health of our Sovereign, the Emperor!'' he cried, and at
the same moment his kindly eyes grew moist with tears of joy and
enthusiasm. The band immediately struck up ``Conquest's joyful
thunder waken...'' All rose and cried ``Hurrah!'' Bagration also
rose and shouted ``Hurrah!'' in exactly the same voice in which
he had shouted it on the field at Schon Grabern.  Young Rostov's
ecstatic voice could be heard above the three hundred others. He
nearly wept. ``To the health of our Sovereign, the Emperor!''  he
roared, ``Hurrah!'' and emptying his glass at one gulp he dashed
it to the floor. Many followed his example, and the loud shouting
continued for a long time. When the voices subsided, the footmen
cleared away the broken glass and everybody sat down again,
smiling at the noise they had made and exchanging remarks. The
old count rose once more, glanced at a note lying beside his
plate, and proposed a toast, ``To the health of the hero of our
last campaign, Prince Peter Ivanovich Bagration!'' and again his
blue eyes grew moist. ``Hurrah!'' cried the three hundred voices
again, but instead of the band a choir began singing a cantata
composed by Paul Ivanovich Kutuzov:

Russians! O'er all barriers on! Courage conquest guarantees; Have
we not Bagration? He brings foe men to their knees,... etc.

As soon as the singing was over, another and another toast was
proposed and Count Ilya Rostov became more and more moved, more
glass was smashed, and the shouting grew louder. They drank to
Bekleshev, Naryshkin, Uvarov, Dolgorukov, Apraksin, Valuev, to
the committee, to all the club members and to all the club
guests, and finally to Count Ilya Rostov separately, as the
organizer of the banquet. At that toast, the count took out his
handkerchief and, covering his face, wept outright.

% % % % % % % % % % % % % % % % % % % % % % % % % % % % % % % % %
% % % % % % % % % % % % % % % % % % % % % % % % % % % % % % % % %
% % % % % % % % % % % % % % % % % % % % % % % % % % % % % % % % %
% % % % % % % % % % % % % % % % % % % % % % % % % % % % % % % % %
% % % % % % % % % % % % % % % % % % % % % % % % % % % % % % % % %
% % % % % % % % % % % % % % % % % % % % % % % % % % % % % % % % %
% % % % % % % % % % % % % % % % % % % % % % % % % % % % % % % % %
% % % % % % % % % % % % % % % % % % % % % % % % % % % % % % % % %
% % % % % % % % % % % % % % % % % % % % % % % % % % % % % % % % %
% % % % % % % % % % % % % % % % % % % % % % % % % % % % % % % % %
% % % % % % % % % % % % % % % % % % % % % % % % % % % % % % % % %
% % % % % % % % % % % % % % % % % % % % % % % % % % % % % %

\chapter*{Chapter IV}
\ifaudio     
\marginpar{
\href{http://ia802609.us.archive.org/34/items/war_and_peace_04_0802_librivox/war_and_peace_04_04_tolstoy_64kb.mp3}{Audio}} 
\fi

\lettrine[lines=2, loversize=0.3, lraise=0]{\initfamily P}{ierre}
sat opposite Dolokhov and Nicholas Rostov. As usual, he
ate and drank much, and eagerly. But those who knew him
intimately noticed that some great change had come over him that
day. He was silent all through dinner and looked about, blinking
and scowling, or, with fixed eyes and a look of complete
absent-mindedness, kept rubbing the bridge of his nose. His face
was depressed and gloomy. He seemed to see and hear nothing of
what was going on around him and to be absorbed by some
depressing and unsolved problem.

The unsolved problem that tormented him was caused by hints given
by the princess, his cousin, at Moscow, concerning Dolokhov's
intimacy with his wife, and by an anonymous letter he had
received that morning, which in the mean jocular way common to
anonymous letters said that he saw badly through his spectacles,
but that his wife's connection with Dolokhov was a secret to no
one but himself. Pierre absolutely disbelieved both the princess'
hints and the letter, but he feared now to look at Dolokhov, who
was sitting opposite him. Every time he chanced to meet
Dolokhov's handsome insolent eyes, Pierre felt something terrible
and monstrous rising in his soul and turned quickly
away. Involuntarily recalling his wife's past and her relations
with Dolokhov, Pierre saw clearly that what was said in the
letter might be true, or might at least seem to be true had it
not referred to his wife. He involuntarily remembered how
Dolokhov, who had fully recovered his former position after the
campaign, had returned to Petersburg and come to him. Availing
himself of his friendly relations with Pierre as a boon
companion, Dolokhov had come straight to his house, and Pierre
had put him up and lent him money. Pierre recalled how Helene had
smilingly expressed disapproval of Dolokhov's living at their
house, and how cynically Dolokhov had praised his wife's beauty
to him and from that time till they came to Moscow had not left
them for a day.

``Yes, he is very handsome,'' thought Pierre, ``and I know
him. It would be particularly pleasant to him to dishonor my name
and ridicule me, just because I have exerted myself on his
behalf, befriended him, and helped him. I know and understand
what a spice that would add to the pleasure of deceiving me, if
it really were true. Yes, if it were true, but I do not believe
it. I have no right to, and can't, believe it.'' He remembered
the expression Dolokhov's face assumed in his moments of cruelty,
as when tying the policeman to the bear and dropping them into
the water, or when he challenged a man to a duel without any
reason, or shot a post-boy's horse with a pistol. That expression
was often on Dolokhov's face when looking at him. ``Yes, he is a
bully,'' thought Pierre, ``to kill a man means nothing to him. It
must seem to him that everyone is afraid of him, and that must
please him. He must think that I, too, am afraid of him---and in
fact I am afraid of him,'' he thought, and again he felt
something terrible and monstrous rising in his soul.  Dolokhov,
Denisov, and Rostov were now sitting opposite Pierre and seemed
very gay. Rostov was talking merrily to his two friends, one of
whom was a dashing hussar and the other a notorious duelist and
rake, and every now and then he glanced ironically at Pierre,
whose preoccupied, absent-minded, and massive figure was a very
noticeable one at the dinner. Rostov looked inimically at Pierre,
first because Pierre appeared to his hussar eyes as a rich
civilian, the husband of a beauty, and in a word---an old woman;
and secondly because Pierre in his preoccupation and
absent-mindedness had not recognized Rostov and had not responded
to his greeting. When the Emperor's health was drunk, Pierre,
lost in thought, did not rise or lift his glass.

``What are you about?'' shouted Rostov, looking at him in an
ecstasy of exasperation. ``Don't you hear it's His Majesty the
Emperor's health?''

Pierre sighed, rose submissively, emptied his glass, and, waiting
till all were seated again, turned with his kindly smile to
Rostov.

``Why, I didn't recognize you!'' he said. But Rostov was
otherwise engaged; he was shouting ``Hurrah!''

``Why don't you renew the acquaintance?'' said Dolokhov to
Rostov.

``Confound him, he's a fool!'' said Rostov.

``One should make up to the husbands of pretty women,'' said
Denisov.

Pierre did not catch what they were saying, but knew they were
talking about him. He reddened and turned away.

``Well, now to the health of handsome women!'' said Dolokhov, and
with a serious expression, but with a smile lurking at the
corners of his mouth, he turned with his glass to Pierre.

``Here's to the health of lovely women, Peterkin---and their
lovers!'' he added.

Pierre, with downcast eyes, drank out of his glass without
looking at Dolokhov or answering him. The footman, who was
distributing leaflets with Kutuzov's cantata, laid one before
Pierre as one of the principal guests. He was just going to take
it when Dolokhov, leaning across, snatched it from his hand and
began reading it. Pierre looked at Dolokhov and his eyes dropped,
the something terrible and monstrous that had tormented him all
dinnertime rose and took possession of him. He leaned his whole
massive body across the table.

``How dare you take it?'' he shouted.

Hearing that cry and seeing to whom it was addressed, Nesvitski
and the neighbor on his right quickly turned in alarm to
Bezukhov.

``Don't! Don't! What are you about?'' whispered their frightened
voices.

Dolokhov looked at Pierre with clear, mirthful, cruel eyes, and
that smile of his which seemed to say, ``Ah! This is what I
like!''

``You shan't have it!'' he said distinctly.

Pale, with quivering lips, Pierre snatched the copy.

``You...! you... scoundrel! I challenge you!'' he ejaculated,
and, pushing back his chair, he rose from the table.

At the very instant he did this and uttered those words, Pierre
felt that the question of his wife's guilt which had been
tormenting him the whole day was finally and indubitably answered
in the affirmative. He hated her and was forever sundered from
her. Despite Denisov's request that he would take no part in the
matter, Rostov agreed to be Dolokhov's second, and after dinner
he discussed the arrangements for the duel with Nesvitski,
Bezukhov's second. Pierre went home, but Rostov with Dolokhov and
Denisov stayed on at the club till late, listening to the gypsies
and other singers.

``Well then, till tomorrow at Sokolniki,'' said Dolokhov, as he
took leave of Rostov in the club porch.

``And do you feel quite calm?'' Rostov asked.

Dolokhov paused.

``Well, you see, I'll tell you the whole secret of dueling in two
words.  If you are going to fight a duel, and you make a will and
write affectionate letters to your parents, and if you think you
may be killed, you are a fool and are lost for certain. But go
with the firm intention of killing your man as quickly and surely
as possible, and then all will be right, as our bear huntsman at
Kostroma used to tell me. 'Everyone fears a bear,' he says, 'but
when you see one your fear's all gone, and your only thought is
not to let him get away!' And that's how it is with me. A demain,
mon cher.''\footnote{Till tomorrow, my dear fellow.}

Next day, at eight in the morning, Pierre and Nesvitski drove to
the Sokolniki forest and found Dolokhov, Denisov, and Rostov
already there.  Pierre had the air of a man preoccupied with
considerations which had no connection with the matter in
hand. His haggard face was yellow. He had evidently not slept
that night. He looked about distractedly and screwed up his eyes
as if dazzled by the sun. He was entirely absorbed by two
considerations: his wife's guilt, of which after his sleepless
night he had not the slightest doubt, and the guiltlessness of
Dolokhov, who had no reason to preserve the honor of a man who
was nothing to him... ``I should perhaps have done the same thing
in his place,'' thought Pierre.  ``It's even certain that I
should have done the same, then why this duel, this murder?
Either I shall kill him, or he will hit me in the head, or elbow,
or knee. Can't I go away from here, run away, bury myself
somewhere?'' passed through his mind. But just at moments when
such thoughts occurred to him, he would ask in a particularly
calm and absent-minded way, which inspired the respect of the
onlookers, ``Will it be long? Are things ready?''

When all was ready, the sabers stuck in the snow to mark the
barriers, and the pistols loaded, Nesvitski went up to Pierre.

``I should not be doing my duty, Count,'' he said in timid tones,
``and should not justify your confidence and the honor you have
done me in choosing me for your second, if at this grave, this
very grave, moment I did not tell you the whole truth. I think
there is no sufficient ground for this affair, or for blood to be
shed over it... You were not right, not quite in the right, you
were impetuous...''

``Oh yes, it is horribly stupid,'' said Pierre.

``Then allow me to express your regrets, and I am sure your
opponent will accept them,'' said Nesvitski (who like the others
concerned in the affair, and like everyone in similar cases, did
not yet believe that the affair had come to an actual
duel). ``You know, Count, it is much more honorable to admit
one's mistake than to let matters become irreparable.  There was
no insult on either side. Allow me to convey...''

``No! What is there to talk about?'' said Pierre. ``It's all the
same...  Is everything ready?'' he added. ``Only tell me where to
go and where to shoot,'' he said with an unnaturally gentle
smile.

He took the pistol in his hand and began asking about the working
of the trigger, as he had not before held a pistol in his
hand---a fact that he did not wish to confess.

``Oh yes, like that, I know, I only forgot,'' said he.

``No apologies, none whatever,'' said Dolokhov to Denisov (who on
his side had been attempting a reconciliation), and he also went
up to the appointed place.

The spot chosen for the duel was some eighty paces from the road,
where the sleighs had been left, in a small clearing in the pine
forest covered with melting snow, the frost having begun to break
up during the last few days. The antagonists stood forty paces
apart at the farther edge of the clearing. The seconds, measuring
the paces, left tracks in the deep wet snow between the place
where they had been standing and Nesvitski's and Dolokhov's
sabers, which were stuck into the ground ten paces apart to mark
the barrier. It was thawing and misty; at forty paces' distance
nothing could be seen. For three minutes all had been ready, but
they still delayed and all were silent.

% % % % % % % % % % % % % % % % % % % % % % % % % % % % % % % % %
% % % % % % % % % % % % % % % % % % % % % % % % % % % % % % % % %
% % % % % % % % % % % % % % % % % % % % % % % % % % % % % % % % %
% % % % % % % % % % % % % % % % % % % % % % % % % % % % % % % % %
% % % % % % % % % % % % % % % % % % % % % % % % % % % % % % % % %
% % % % % % % % % % % % % % % % % % % % % % % % % % % % % % % % %
% % % % % % % % % % % % % % % % % % % % % % % % % % % % % % % % %
% % % % % % % % % % % % % % % % % % % % % % % % % % % % % % % % %
% % % % % % % % % % % % % % % % % % % % % % % % % % % % % % % % %
% % % % % % % % % % % % % % % % % % % % % % % % % % % % % % % % %
% % % % % % % % % % % % % % % % % % % % % % % % % % % % % % % % %
% % % % % % % % % % % % % % % % % % % % % % % % % % % % % %

\chapter*{Chapter V}
\ifaudio     
\marginpar{
\href{http://ia802609.us.archive.org/34/items/war_and_peace_04_0802_librivox/war_and_peace_04_05_tolstoy_64kb.mp3}{Audio}} 
\fi

\lettrine[lines=2, loversize=0.3, lraise=0]{``\initfamily W}{ell}
begin!'' said Dolokhov.

``All right,'' said Pierre, still smiling in the same way. A
feeling of dread was in the air. It was evident that the affair
so lightly begun could no longer be averted but was taking its
course independently of men's will.

Denisov first went to the barrier and announced: ``As the
adve'sawies have wefused a weconciliation, please pwoceed. Take
your pistols, and at the word thwee begin to advance.''

``O-ne! T-wo! Thwee!'' he shouted angrily and stepped aside.

The combatants advanced along the trodden tracks, nearer and
nearer to one another, beginning to see one another through the
mist. They had the right to fire when they liked as they
approached the barrier. Dolokhov walked slowly without raising
his pistol, looking intently with his bright, sparkling blue eyes
into his antagonist's face. His mouth wore its usual semblance of
a smile.

``So I can fire when I like!'' said Pierre, and at the word
``three,'' he went quickly forward, missing the trodden path and
stepping into the deep snow. He held the pistol in his right hand
at arm's length, apparently afraid of shooting himself with
it. His left hand he held carefully back, because he wished to
support his right hand with it and knew he must not do so. Having
advanced six paces and strayed off the track into the snow,
Pierre looked down at his feet, then quickly glanced at Dolokhov
and, bending his finger as he had been shown, fired.  Not at all
expecting so loud a report, Pierre shuddered at the sound and
then, smiling at his own sensations, stood still. The smoke,
rendered denser by the mist, prevented him from seeing anything
for an instant, but there was no second report as he had
expected. He only heard Dolokhov's hurried steps, and his figure
came in view through the smoke.  He was pressing one hand to his
left side, while the other clutched his drooping pistol. His face
was pale. Rostov ran toward him and said something.

``No-o-o!'' muttered Dolokhov through his teeth, ``no, it's not
over.'' And after stumbling a few staggering steps right up to
the saber, he sank on the snow beside it. His left hand was
bloody; he wiped it on his coat and supported himself with
it. His frowning face was pallid and quivered.

``Plea...'' began Dolokhov, but could not at first pronounce the
word.

``Please,'' he uttered with an effort.

Pierre, hardly restraining his sobs, began running toward
Dolokhov and was about to cross the space between the barriers,
when Dolokhov cried:

``To your barrier!'' and Pierre, grasping what was meant, stopped
by his saber. Only ten paces divided them. Dolokhov lowered his
head to the snow, greedily bit at it, again raised his head,
adjusted himself, drew in his legs and sat up, seeking a firm
center of gravity. He sucked and swallowed the cold snow, his
lips quivered but his eyes, still smiling, glittered with effort
and exasperation as he mustered his remaining strength. He raised
his pistol and aimed.

``Sideways! Cover yourself with your pistol!'' ejaculated
Nesvitski.

``Cover yourself!'' even Denisov cried to his adversary.

Pierre, with a gentle smile of pity and remorse, his arms and
legs helplessly spread out, stood with his broad chest directly
facing Dolokhov looked sorrowfully at him. Denisov, Rostov, and
Nesvitski closed their eyes. At the same instant they heard a
report and Dolokhov's angry cry.

``Missed!'' shouted Dolokhov, and he lay helplessly, face
downwards on the snow.

Pierre clutched his temples, and turning round went into the
forest, trampling through the deep snow, and muttering incoherent
words:

``Folly... folly! Death... lies...'' he repeated, puckering his
face.

Nesvitski stopped him and took him home.

Rostov and Denisov drove away with the wounded Dolokhov.

The latter lay silent in the sleigh with closed eyes and did not
answer a word to the questions addressed to him. But on entering
Moscow he suddenly came to and, lifting his head with an effort,
took Rostov, who was sitting beside him, by the hand. Rostov was
struck by the totally altered and unexpectedly rapturous and
tender expression on Dolokhov's face.

``Well? How do you feel?'' he asked.

``Bad! But it's not that, my friend-'' said Dolokhov with a
gasping voice.  ``Where are we? In Moscow, I know. I don't
matter, but I have killed her, killed... She won't get over it!
She won't survive...''

``Who?'' asked Rostov.

``My mother! My mother, my angel, my adored angel mother,'' and
Dolokhov pressed Rostov's hand and burst into tears.

When he had become a little quieter, he explained to Rostov that
he was living with his mother, who, if she saw him dying, would
not survive it.  He implored Rostov to go on and prepare her.

Rostov went on ahead to do what was asked, and to his great
surprise learned that Dolokhov the brawler, Dolokhov the bully,
lived in Moscow with an old mother and a hunchback sister, and
was the most affectionate of sons and brothers.

% % % % % % % % % % % % % % % % % % % % % % % % % % % % % % % % %
% % % % % % % % % % % % % % % % % % % % % % % % % % % % % % % % %
% % % % % % % % % % % % % % % % % % % % % % % % % % % % % % % % %
% % % % % % % % % % % % % % % % % % % % % % % % % % % % % % % % %
% % % % % % % % % % % % % % % % % % % % % % % % % % % % % % % % %
% % % % % % % % % % % % % % % % % % % % % % % % % % % % % % % % %
% % % % % % % % % % % % % % % % % % % % % % % % % % % % % % % % %
% % % % % % % % % % % % % % % % % % % % % % % % % % % % % % % % %
% % % % % % % % % % % % % % % % % % % % % % % % % % % % % % % % %
% % % % % % % % % % % % % % % % % % % % % % % % % % % % % % % % %
% % % % % % % % % % % % % % % % % % % % % % % % % % % % % % % % %
% % % % % % % % % % % % % % % % % % % % % % % % % % % % % %

\chapter*{Chapter VI}
\ifaudio     
\marginpar{
\href{http://ia802609.us.archive.org/34/items/war_and_peace_04_0802_librivox/war_and_peace_04_06_tolstoy_64kb.mp3}{Audio}} 
\fi

\lettrine[lines=2, loversize=0.3, lraise=0]{\initfamily P}{ierre}
had of late rarely seen his wife alone. Both in Petersburg
and in Moscow their house was always full of visitors. The night
after the duel he did not go to his bedroom but, as he often did,
remained in his father's room, that huge room in which Count
Bezukhov had died.

He lay down on the sofa meaning to fall asleep and forget all
that had happened to him, but could not do so. Such a storm of
feelings, thoughts, and memories suddenly arose within him that
he could not fall asleep, nor even remain in one place, but had
to jump up and pace the room with rapid steps. Now he seemed to
see her in the early days of their marriage, with bare shoulders
and a languid, passionate look on her face, and then immediately
he saw beside her Dolokhov's handsome, insolent, hard, and
mocking face as he had seen it at the banquet, and then that same
face pale, quivering, and suffering, as it had been when he
reeled and sank on the snow.

``What has happened?'' he asked himself. ``I have killed her
lover, yes, killed my wife's lover. Yes, that was it! And why?
How did I come to do it?''---``Because you married her,''
answered an inner voice.

``But in what was I to blame?'' he asked. ``In marrying her
without loving her; in deceiving yourself and her.'' And he
vividly recalled that moment after supper at Prince Vasili's,
when he spoke those words he had found so difficult to utter: ``I
love you.'' ``It all comes from that! Even then I felt it,'' he
thought. ``I felt then that it was not so, that I had no right to
do it. And so it turns out.''

He remembered his honeymoon and blushed at the recollection.
Particularly vivid, humiliating, and shameful was the
recollection of how one day soon after his marriage he came out
of the bedroom into his study a little before noon in his silk
dressing gown and found his head steward there, who, bowing
respectfully, looked into his face and at his dressing gown and
smiled slightly, as if expressing respectful understanding of his
employer's happiness.

``But how often I have felt proud of her, proud of her majestic
beauty and social tact,'' thought he; ``been proud of my house,
in which she received all Petersburg, proud of her
unapproachability and beauty. So this is what I was proud of! I
then thought that I did not understand her. How often when
considering her character I have told myself that I was to blame
for not understanding her, for not understanding that constant
composure and complacency and lack of all interests or desires,
and the whole secret lies in the terrible truth that she is a
depraved woman. Now I have spoken that terrible word to myself
all has become clear.''

``Anatole used to come to borrow money from her and used to kiss
her naked shoulders. She did not give him the money, but let
herself be kissed. Her father in jest tried to rouse her
jealousy, and she replied with a calm smile that she was not so
stupid as to be jealous: 'Let him do what he pleases,' she used
to say of me. One day I asked her if she felt any symptoms of
pregnancy. She laughed contemptuously and said she was not a fool
to want to have children, and that she was not going to have any
children by me.''

Then he recalled the coarseness and bluntness of her thoughts and
the vulgarity of the expressions that were natural to her, though
she had been brought up in the most aristocratic circles.

``I'm not such a fool... Just you try it on... Allez-vous
promener,''\footnote{``You clear out of this.''}  she used to
say. Often seeing the success she had with young and old men and
women Pierre could not understand why he did not love her.

``Yes, I never loved her,'' said he to himself; ``I knew she was
a depraved woman,'' he repeated, ``but dared not admit it to
myself. And now there's Dolokhov sitting in the snow with a
forced smile and perhaps dying, while meeting my remorse with
some forced bravado!''

Pierre was one of those people who, in spite of an appearance of
what is called weak character, do not seek a confidant in their
troubles. He digested his sufferings alone.

``It is all, all her fault,'' he said to himself; ``but what of
that? Why did I bind myself to her? Why did I say 'Je vous
aime'\footnote{I love you.} to her, which was a lie, and worse
than a lie? I am guilty and must endure... what? A slur on my
name? A misfortune for life? Oh, that's nonsense,'' he
thought. ``The slur on my name and honor---that's all apart from
myself.''

``Louis XVI was executed because they said he was dishonorable
and a criminal,'' came into Pierre's head, ``and from their point
of view they were right, as were those too who canonized him and
died a martyr's death for his sake. Then Robespierre was beheaded
for being a despot.  Who is right and who is wrong? No one! But
if you are alive---live: tomorrow you'll die as I might have died
an hour ago. And is it worth tormenting oneself, when one has
only a moment of life in comparison with eternity?''

But at the moment when he imagined himself calmed by such
reflections, she suddenly came into his mind as she was at the
moments when he had most strongly expressed his insincere love
for her, and he felt the blood rush to his heart and had again to
get up and move about and break and tear whatever came to his
hand. ``Why did I tell her that 'Je vous aime'?'' he kept
repeating to himself. And when he had said it for the tenth time,
Moliere's words: ``Mais que diable allait-il faire dans cette
galere?''\footnote{``But what the devil was he doing in that
galley?''}  occurred to him, and he began to laugh at himself.

In the night he called his valet and told him to pack up to go to
Petersburg. He could not imagine how he could speak to her
now. He resolved to go away next day and leave a letter informing
her of his intention to part from her forever.

Next morning when the valet came into the room with his coffee,
Pierre was lying asleep on the ottoman with an open book in his
hand.

He woke up and looked round for a while with a startled
expression, unable to realize where he was.

``The countess told me to inquire whether your excellency was at
home,'' said the valet.

But before Pierre could decide what answer he would send, the
countess herself in a white satin dressing gown embroidered with
silver and with simply dressed hair (two immense plaits twice
round her lovely head like a coronet) entered the room, calm and
majestic, except that there was a wrathful wrinkle on her rather
prominent marble brow. With her imperturbable calm she did not
begin to speak in front of the valet. She knew of the duel and
had come to speak about it. She waited till the valet had set
down the coffee things and left the room. Pierre looked at her
timidly over his spectacles, and like a hare surrounded by hounds
who lays back her ears and continues to crouch motionless before
her enemies, he tried to continue reading. But feeling this to be
senseless and impossible, he again glanced timidly at her. She
did not sit down but looked at him with a contemptuous smile,
waiting for the valet to go.

``Well, what's this now? What have you been up to now, I should
like to know?'' she asked sternly.

``I? What have I...?'' stammered Pierre.

``So it seems you're a hero, eh? Come now, what was this duel
about? What is it meant to prove? What? I ask you.''

Pierre turned over heavily on the ottoman and opened his mouth,
but could not reply.

``If you won't answer, I'll tell you...'' Helene went on. ``You
believe everything you're told. You were told...'' Helene
laughed, ``that Dolokhov was my lover,'' she said in French with
her coarse plainness of speech, uttering the word amant as
casually as any other word, ``and you believed it! Well, what
have you proved? What does this duel prove? That you're a fool,
que vous etes un sot, but everybody knew that. What will be the
result? That I shall be the laughingstock of all Moscow, that
everyone will say that you, drunk and not knowing what you were
about, challenged a man you are jealous of without cause.''
Helene raised her voice and became more and more excited, ``A man
who's a better man than you in every way...''

``Hm... Hm...!'' growled Pierre, frowning without looking at her,
and not moving a muscle.

``And how could you believe he was my lover? Why? Because I like
his company? If you were cleverer and more agreeable, I should
prefer yours.''

``Don't speak to me... I beg you,'' muttered Pierre hoarsely.

``Why shouldn't I speak? I can speak as I like, and I tell you
plainly that there are not many wives with husbands such as you
who would not have taken lovers (des amants), but I have not done
so,'' said she.

Pierre wished to say something, looked at her with eyes whose
strange expression she did not understand, and lay down again. He
was suffering physically at that moment, there was a weight on
his chest and he could not breathe. He knew that he must do
something to put an end to this suffering, but what he wanted to
do was too terrible.

``We had better separate,'' he muttered in a broken voice.

``Separate? Very well, but only if you give me a fortune,'' said
Helene.  ``Separate! That's a thing to frighten me with!''

Pierre leaped up from the sofa and rushed staggering toward her.

``I'll kill you!'' he shouted, and seizing the marble top of a
table with a strength he had never before felt, he made a step
toward her brandishing the slab.

Helene's face became terrible, she shrieked and sprang aside. His
father's nature showed itself in Pierre. He felt the fascination
and delight of frenzy. He flung down the slab, broke it, and
swooping down on her with outstretched hands shouted, ``Get
out!'' in such a terrible voice that the whole house heard it
with horror. God knows what he would have done at that moment had
Helene not fled from the room.

A week later Pierre gave his wife full power to control all his
estates in Great Russia, which formed the larger part of his
property, and left for Petersburg alone.

% % % % % % % % % % % % % % % % % % % % % % % % % % % % % % % % %
% % % % % % % % % % % % % % % % % % % % % % % % % % % % % % % % %
% % % % % % % % % % % % % % % % % % % % % % % % % % % % % % % % %
% % % % % % % % % % % % % % % % % % % % % % % % % % % % % % % % %
% % % % % % % % % % % % % % % % % % % % % % % % % % % % % % % % %
% % % % % % % % % % % % % % % % % % % % % % % % % % % % % % % % %
% % % % % % % % % % % % % % % % % % % % % % % % % % % % % % % % %
% % % % % % % % % % % % % % % % % % % % % % % % % % % % % % % % %
% % % % % % % % % % % % % % % % % % % % % % % % % % % % % % % % %
% % % % % % % % % % % % % % % % % % % % % % % % % % % % % % % % %
% % % % % % % % % % % % % % % % % % % % % % % % % % % % % % % % %
% % % % % % % % % % % % % % % % % % % % % % % % % % % % % %

\chapter*{Chapter VII}
\ifaudio     
\marginpar{
\href{http://ia802609.us.archive.org/34/items/war_and_peace_04_0802_librivox/war_and_peace_04_07_tolstoy_64kb.mp3}{Audio}} 
\fi

\lettrine[lines=2, loversize=0.3, lraise=0]{\initfamily T}{wo}
months had elapsed since the news of the battle of Austerlitz
and the loss of Prince Andrew had reached Bald Hills, and in
spite of the letters sent through the embassy and all the
searches made, his body had not been found nor was he on the list
of prisoners. What was worst of all for his relations was the
fact that there was still a possibility of his having been picked
up on the battlefield by the people of the place and that he
might now be lying, recovering or dying, alone among strangers
and unable to send news of himself. The gazettes from which the
old prince first heard of the defeat at Austerlitz stated, as
usual very briefly and vaguely, that after brilliant engagements
the Russians had had to retreat and had made their withdrawal in
perfect order. The old prince understood from this official
report that our army had been defeated. A week after the gazette
report of the battle of Austerlitz came a letter from Kutuzov
informing the prince of the fate that had befallen his son.

``Your son,'' wrote Kutuzov, ``fell before my eyes, a standard in
his hand and at the head of a regiment---he fell as a hero,
worthy of his father and his fatherland. To the great regret of
myself and of the whole army it is still uncertain whether he is
alive or not. I comfort myself and you with the hope that your
son is alive, for otherwise he would have been mentioned among
the officers found on the field of battle, a list of whom has
been sent me under flag of truce.''

After receiving this news late in the evening, when he was alone
in his study, the old prince went for his walk as usual next
morning, but he was silent with his steward, the gardener, and
the architect, and though he looked very grim he said nothing to
anyone.

When Princess Mary went to him at the usual hour he was working
at his lathe and, as usual, did not look round at her.

``Ah, Princess Mary!'' he said suddenly in an unnatural voice,
throwing down his chisel. (The wheel continued to revolve by its
own impetus, and Princess Mary long remembered the dying creak of
that wheel, which merged in her memory with what followed.)

She approached him, saw his face, and something gave way within
her. Her eyes grew dim. By the expression of her father's face,
not sad, not crushed, but angry and working unnaturally, she saw
that hanging over her and about to crush her was some terrible
misfortune, the worst in life, one she had not yet experienced,
irreparable and incomprehensible---the death of one she loved.

``Father! Andrew!''---said the ungraceful, awkward princess with
such an indescribable charm of sorrow and self-forgetfulness that
her father could not bear her look but turned away with a sob.

``Bad news! He's not among the prisoners nor among the killed!
Kutuzov writes...'' and he screamed as piercingly as if he wished
to drive the princess away by that scream... ``Killed!''

The princess did not fall down or faint. She was already pale,
but on hearing these words her face changed and something
brightened in her beautiful, radiant eyes. It was as if joy---a
supreme joy apart from the joys and sorrows of this
world---overflowed the great grief within her.  She forgot all
fear of her father, went up to him, took his hand, and drawing
him down put her arm round his thin, scraggy neck.

``Father,'' she said, ``do not turn away from me, let us weep
together.''

``Scoundrels! Blackguards!'' shrieked the old man, turning his
face away from her. ``Destroying the army, destroying the men!
And why? Go, go and tell Lise.''

The princess sank helplessly into an armchair beside her father
and wept. She saw her brother now as he had been at the moment
when he took leave of her and of Lise, his look tender yet
proud. She saw him tender and amused as he was when he put on the
little icon. ``Did he believe?  Had he repented of his unbelief?
Was he now there? There in the realms of eternal peace and
blessedness?'' she thought.

``Father, tell me how it happened,'' she asked through her tears.

``Go! Go! Killed in battle, where the best of Russian men and
Russia's glory were led to destruction. Go, Princess Mary. Go and
tell Lise. I will follow.''

When Princess Mary returned from her father, the little princess
sat working and looked up with that curious expression of inner,
happy calm peculiar to pregnant women. It was evident that her
eyes did not see Princess Mary but were looking within... into
herself... at something joyful and mysterious taking place within
her.

``Mary,'' she said, moving away from the embroidery frame and
lying back, ``give me your hand.'' She took her sister-in-law's
hand and held it below her waist.

Her eyes were smiling expectantly, her downy lip rose and
remained lifted in childlike happiness.

Princess Mary knelt down before her and hid her face in the folds
of her sister-in-law's dress.

``There, there! Do you feel it? I feel so strange. And do you
know, Mary, I am going to love him very much,'' said Lise,
looking with bright and happy eyes at her sister-in-law.

Princess Mary could not lift her head, she was weeping.

``What is the matter, Mary?''

``Nothing... only I feel sad... sad about Andrew,'' she said,
wiping away her tears on her sister-in-law's knee.

Several times in the course of the morning Princess Mary began
trying to prepare her sister-in-law, and every time began to
cry. Unobservant as was the little princess, these tears, the
cause of which she did not understand, agitated her. She said
nothing but looked about uneasily as if in search of
something. Before dinner the old prince, of whom she was always
afraid, came into her room with a peculiarly restless and malign
expression and went out again without saying a word. She looked
at Princess Mary, then sat thinking for a while with that
expression of attention to something within her that is only seen
in pregnant women, and suddenly began to cry.

``Has anything come from Andrew?'' she asked.

``No, you know it's too soon for news. But my father is anxious
and I feel afraid.''

``So there's nothing?''

``Nothing,'' answered Princess Mary, looking firmly with her
radiant eyes at her sister-in-law.

She had determined not to tell her and persuaded her father to
hide the terrible news from her till after her confinement, which
was expected within a few days. Princess Mary and the old prince
each bore and hid their grief in their own way. The old prince
would not cherish any hope: he made up his mind that Prince
Andrew had been killed, and though he sent an official to Austria
to seek for traces of his son, he ordered a monument from Moscow
which he intended to erect in his own garden to his memory, and
he told everybody that his son had been killed. He tried not to
change his former way of life, but his strength failed him. He
walked less, ate less, slept less, and became weaker every
day. Princess Mary hoped. She prayed for her brother as living
and was always awaiting news of his return.

% % % % % % % % % % % % % % % % % % % % % % % % % % % % % % % % %
% % % % % % % % % % % % % % % % % % % % % % % % % % % % % % % % %
% % % % % % % % % % % % % % % % % % % % % % % % % % % % % % % % %
% % % % % % % % % % % % % % % % % % % % % % % % % % % % % % % % %
% % % % % % % % % % % % % % % % % % % % % % % % % % % % % % % % %
% % % % % % % % % % % % % % % % % % % % % % % % % % % % % % % % %
% % % % % % % % % % % % % % % % % % % % % % % % % % % % % % % % %
% % % % % % % % % % % % % % % % % % % % % % % % % % % % % % % % %
% % % % % % % % % % % % % % % % % % % % % % % % % % % % % % % % %
% % % % % % % % % % % % % % % % % % % % % % % % % % % % % % % % %
% % % % % % % % % % % % % % % % % % % % % % % % % % % % % % % % %
% % % % % % % % % % % % % % % % % % % % % % % % % % % % % %

\chapter*{Chapter VIII}
\ifaudio     
\marginpar{
\href{http://ia802609.us.archive.org/34/items/war_and_peace_04_0802_librivox/war_and_peace_04_08_tolstoy_64kb.mp3}{Audio}} 
\fi

\lettrine[lines=2, loversize=0.3, lraise=0]{``\initfamily D}%
{earest},''
said the little princess after breakfast on the
morning of the nineteenth March, and her downy little lip rose
from old habit, but as sorrow was manifest in every smile, the
sound of every word, and even every footstep in that house since
the terrible news had come, so now the smile of the little
princess---influenced by the general mood though without knowing
its cause---was such as to remind one still more of the general
sorrow.

``Dearest, I'm afraid this morning's
fruschtique\footnote{Fruhstuck: breakfast.}---as Foka the cook
calls it---has disagreed with me.''

``What is the matter with you, my darling? You look pale. Oh, you
are very pale!'' said Princess Mary in alarm, running with her
soft, ponderous steps up to her sister-in-law.

``Your excellency, should not Mary Bogdanovna be sent for?'' said
one of the maids who was present. (Mary Bogdanovna was a midwife
from the neighboring town, who had been at Bald Hills for the
last fortnight.)

``Oh yes,'' assented Princess Mary, ``perhaps that's it. I'll
go. Courage, my angel.'' She kissed Lise and was about to leave
the room.

``Oh, no, no!'' And besides the pallor and the physical suffering
on the little princess' face, an expression of childish fear of
inevitable pain showed itself.

``No, it's only indigestion?... Say it's only indigestion, say
so, Mary!  Say...'' And the little princess began to cry
capriciously like a suffering child and to wring her little hands
even with some affectation. Princess Mary ran out of the room to
fetch Mary Bogdanovna.

``Mon Dieu! Mon Dieu! Oh!'' she heard as she left the room.

The midwife was already on her way to meet her, rubbing her
small, plump white hands with an air of calm importance.

``Mary Bogdanovna, I think it's beginning!'' said Princess Mary
looking at the midwife with wide-open eyes of alarm.

``Well, the Lord be thanked, Princess,'' said Mary Bogdanovna,
not hastening her steps. ``You young ladies should not know
anything about it.''

``But how is it the doctor from Moscow is not here yet?'' said
the princess. (In accordance with Lise's and Prince Andrew's
wishes they had sent in good time to Moscow for a doctor and were
expecting him at any moment.)

``No matter, Princess, don't be alarmed,'' said Mary
Bogdanovna. ``We'll manage very well without a doctor.''

Five minutes later Princess Mary from her room heard something
heavy being carried by. She looked out. The men servants were
carrying the large leather sofa from Prince Andrew's study into
the bedroom. On their faces was a quiet and solemn look.

Princess Mary sat alone in her room listening to the sounds in
the house, now and then opening her door when someone passed and
watching what was going on in the passage. Some women passing
with quiet steps in and out of the bedroom glanced at the
princess and turned away. She did not venture to ask any
questions, and shut the door again, now sitting down in her easy
chair, now taking her prayer book, now kneeling before the icon
stand. To her surprise and distress she found that her prayers
did not calm her excitement. Suddenly her door opened softly and
her old nurse, Praskovya Savishna, who hardly ever came to that
room as the old prince had forbidden it, appeared on the
threshold with a shawl round her head.

``I've come to sit with you a bit, Masha,'' said the nurse, ``and
here I've brought the prince's wedding candles to light before
his saint, my angel,'' she said with a sigh.

``Oh, nurse, I'm so glad!''

``God is merciful, birdie.''

The nurse lit the gilt candles before the icons and sat down by
the door with her knitting. Princess Mary took a book and began
reading. Only when footsteps or voices were heard did they look
at one another, the princess anxious and inquiring, the nurse
encouraging. Everyone in the house was dominated by the same
feeling that Princess Mary experienced as she sat in her
room. But owing to the superstition that the fewer the people who
know of it the less a woman in travail suffers, everyone tried to
pretend not to know; no one spoke of it, but apart from the
ordinary staid and respectful good manners habitual in the
prince's household, a common anxiety, a softening of the heart,
and a consciousness that something great and mysterious was being
accomplished at that moment made itself felt.

There was no laughter in the maids' large hall. In the men
servants' hall all sat waiting, silently and alert. In the
outlying serfs' quarters torches and candles were burning and no
one slept. The old prince, stepping on his heels, paced up and
down his study and sent Tikhon to ask Mary Bogdanovna what
news.---``Say only that 'the prince told me to ask,' and come and
tell me her answer.''

``Inform the prince that labor has begun,'' said Mary Bogdanovna,
giving the messenger a significant look.

Tikhon went and told the prince.

``Very good!'' said the prince closing the door behind him, and
Tikhon did not hear the slightest sound from the study after
that.

After a while he re-entered it as if to snuff the candles, and,
seeing the prince was lying on the sofa, looked at him, noticed
his perturbed face, shook his head, and going up to him silently
kissed him on the shoulder and left the room without snuffing the
candles or saying why he had entered. The most solemn mystery in
the world continued its course.  Evening passed, night came, and
the feeling of suspense and softening of heart in the presence of
the unfathomable did not lessen but increased.  No one slept.

It was one of those March nights when winter seems to wish to
resume its sway and scatters its last snows and storms with
desperate fury. A relay of horses had been sent up the highroad
to meet the German doctor from Moscow who was expected every
moment, and men on horseback with lanterns were sent to the
crossroads to guide him over the country road with its hollows
and snow-covered pools of water.

Princess Mary had long since put aside her book: she sat silent,
her luminous eyes fixed on her nurse's wrinkled face (every line
of which she knew so well), on the lock of gray hair that escaped
from under the kerchief, and the loose skin that hung under her
chin.

Nurse Savishna, knitting in hand, was telling in low tones,
scarcely hearing or understanding her own words, what she had
told hundreds of times before: how the late princess had given
birth to Princess Mary in Kishenev with only a Moldavian peasant
woman to help instead of a midwife.

``God is merciful, doctors are never needed,'' she said.

Suddenly a gust of wind beat violently against the casement of
the window, from which the double frame had been removed (by
order of the prince, one window frame was removed in each room as
soon as the larks returned), and, forcing open a loosely closed
latch, set the damask curtain flapping and blew out the candle
with its chill, snowy draft.  Princess Mary shuddered; her nurse,
putting down the stocking she was knitting, went to the window
and leaning out tried to catch the open casement. The cold wind
flapped the ends of her kerchief and her loose locks of gray
hair.

``Princess, my dear, there's someone driving up the avenue!'' she
said, holding the casement and not closing it. ``With
lanterns. Most likely the doctor.''

``Oh, my God! thank God!'' said Princess Mary. ``I must go and
meet him, he does not know Russian.''

Princess Mary threw a shawl over her head and ran to meet the
newcomer.  As she was crossing the anteroom she saw through the
window a carriage with lanterns, standing at the entrance. She
went out on the stairs. On a banister post stood a tallow candle
which guttered in the draft. On the landing below, Philip, the
footman, stood looking scared and holding another candle. Still
lower, beyond the turn of the staircase, one could hear the
footstep of someone in thick felt boots, and a voice that seemed
familiar to Princess Mary was saying something.

``Thank God!'' said the voice. ``And Father?''

``Gone to bed,'' replied the voice of Demyan the house steward,
who was downstairs.

Then the voice said something more, Demyan replied, and the steps
in the felt boots approached the unseen bend of the staircase
more rapidly.

``It's Andrew!'' thought Princess Mary. ``No it can't be, that
would be too extraordinary,'' and at the very moment she thought
this, the face and figure of Prince Andrew, in a fur cloak the
deep collar of which covered with snow, appeared on the landing
where the footman stood with the candle. Yes, it was he, pale,
thin, with a changed and strangely softened but agitated
expression on his face. He came up the stairs and embraced his
sister.

``You did not get my letter?'' he asked, and not waiting for a
reply---which he would not have received, for the princess was
unable to speak---he turned back, rapidly mounted the stairs
again with the doctor who had entered the hall after him (they
had met at the last post station), and again embraced his sister.

``What a strange fate, Masha darling!'' And having taken off his
cloak and felt boots, he went to the little princess' apartment.

% % % % % % % % % % % % % % % % % % % % % % % % % % % % % % % % %
% % % % % % % % % % % % % % % % % % % % % % % % % % % % % % % % %
% % % % % % % % % % % % % % % % % % % % % % % % % % % % % % % % %
% % % % % % % % % % % % % % % % % % % % % % % % % % % % % % % % %
% % % % % % % % % % % % % % % % % % % % % % % % % % % % % % % % %
% % % % % % % % % % % % % % % % % % % % % % % % % % % % % % % % %
% % % % % % % % % % % % % % % % % % % % % % % % % % % % % % % % %
% % % % % % % % % % % % % % % % % % % % % % % % % % % % % % % % %
% % % % % % % % % % % % % % % % % % % % % % % % % % % % % % % % %
% % % % % % % % % % % % % % % % % % % % % % % % % % % % % % % % %
% % % % % % % % % % % % % % % % % % % % % % % % % % % % % % % % %
% % % % % % % % % % % % % % % % % % % % % % % % % % % % % %

\chapter*{Chapter IX}
\ifaudio     
\marginpar{
\href{http://ia802609.us.archive.org/34/items/war_and_peace_04_0802_librivox/war_and_peace_04_09_tolstoy_64kb.mp3}{Audio}} 
\fi

\lettrine[lines=2, loversize=0.3, lraise=0]{\initfamily T}{he}
\emph{little princess} lay supported by pillows, with a white cap on
her head (the pains had just left her). Strands of her black hair
lay round her inflamed and perspiring cheeks, her charming rosy
mouth with its downy lip was open and she was smiling
joyfully. Prince Andrew entered and paused facing her at the foot
of the sofa on which she was lying.  Her glittering eyes, filled
with childlike fear and excitement, rested on him without
changing their expression. ``I love you all and have done no harm
to anyone; why must I suffer so? Help me!'' her look seemed to
say. She saw her husband, but did not realize the significance of
his appearance before her now. Prince Andrew went round the sofa
and kissed her forehead.

``My darling!'' he said---a word he had never used to her
before. ``God is merciful...''

She looked at him inquiringly and with childlike reproach.

``I expected help from you and I get none, none from you
either!'' said her eyes. She was not surprised at his having
come; she did not realize that he had come. His coming had
nothing to do with her sufferings or with their relief. The pangs
began again and Mary Bogdanovna advised Prince Andrew to leave
the room.

The doctor entered. Prince Andrew went out and, meeting Princess
Mary, again joined her. They began talking in whispers, but their
talk broke off at every moment. They waited and listened.

``Go, dear,'' said Princess Mary.

Prince Andrew went again to his wife and sat waiting in the room
next to hers. A woman came from the bedroom with a frightened
face and became confused when she saw Prince Andrew. He covered
his face with his hands and remained so for some
minutes. Piteous, helpless, animal moans came through the
door. Prince Andrew got up, went to the door, and tried to open
it. Someone was holding it shut.

``You can't come in! You can't!'' said a terrified voice from
within.

He began pacing the room. The screaming ceased, and a few more
seconds went by. Then suddenly a terrible shriek---it could not
be hers, she could not scream like that---came from the
bedroom. Prince Andrew ran to the door; the scream ceased and he
heard the wail of an infant.

``What have they taken a baby in there for?'' thought Prince
Andrew in the first second. ``A baby? What baby...? Why is there
a baby there? Or is the baby born?''

Then suddenly he realized the joyful significance of that wail;
tears choked him, and leaning his elbows on the window sill be
began to cry, sobbing like a child. The door opened. The doctor
with his shirt sleeves tucked up, without a coat, pale and with a
trembling jaw, came out of the room. Prince Andrew turned to him,
but the doctor gave him a bewildered look and passed by without a
word. A woman rushed out and seeing Prince Andrew stopped,
hesitating on the threshold. He went into his wife's room. She
was lying dead, in the same position he had seen her in five
minutes before and, despite the fixed eyes and the pallor of the
cheeks, the same expression was on her charming childlike face
with its upper lip covered with tiny black hair.

``I love you all, and have done no harm to anyone; and what have
you done to me?''---said her charming, pathetic, dead face.

In a corner of the room something red and tiny gave a grunt and
squealed in Mary Bogdanovna's trembling white hands.

Two hours later Prince Andrew, stepping softly, went into his
father's room. The old man already knew everything. He was
standing close to the door and as soon as it opened his rough old
arms closed like a vise round his son's neck, and without a word
he began to sob like a child.

Three days later the little princess was buried, and Prince
Andrew went up the steps to where the coffin stood, to give her
the farewell kiss.  And there in the coffin was the same face,
though with closed eyes. ``Ah, what have you done to me?'' it
still seemed to say, and Prince Andrew felt that something gave
way in his soul and that he was guilty of a sin he could neither
remedy nor forget. He could not weep. The old man too came up and
kissed the waxen little hands that lay quietly crossed one on the
other on her breast, and to him, too, her face seemed to say:
``Ah, what have you done to me, and why?'' And at the sight the
old man turned angrily away.

Another five days passed, and then the young Prince Nicholas
Andreevich was baptized. The wet nurse supported the coverlet
with her chin, while the priest with a goose feather anointed the
boy's little red and wrinkled soles and palms.

His grandfather, who was his godfather, trembling and afraid of
dropping him, carried the infant round the battered tin font and
handed him over to the godmother, Princess Mary. Prince Andrew
sat in another room, faint with fear lest the baby should be
drowned in the font, and awaited the termination of the
ceremony. He looked up joyfully at the baby when the nurse
brought it to him and nodded approval when she told him that the
wax with the baby's hair had not sunk in the font but had
floated.

% % % % % % % % % % % % % % % % % % % % % % % % % % % % % % % % %
% % % % % % % % % % % % % % % % % % % % % % % % % % % % % % % % %
% % % % % % % % % % % % % % % % % % % % % % % % % % % % % % % % %
% % % % % % % % % % % % % % % % % % % % % % % % % % % % % % % % %
% % % % % % % % % % % % % % % % % % % % % % % % % % % % % % % % %
% % % % % % % % % % % % % % % % % % % % % % % % % % % % % % % % %
% % % % % % % % % % % % % % % % % % % % % % % % % % % % % % % % %
% % % % % % % % % % % % % % % % % % % % % % % % % % % % % % % % %
% % % % % % % % % % % % % % % % % % % % % % % % % % % % % % % % %
% % % % % % % % % % % % % % % % % % % % % % % % % % % % % % % % %
% % % % % % % % % % % % % % % % % % % % % % % % % % % % % % % % %
% % % % % % % % % % % % % % % % % % % % % % % % % % % % % %

\chapter*{Chapter X}
\ifaudio     
\marginpar{
\href{http://ia802609.us.archive.org/34/items/war_and_peace_04_0802_librivox/war_and_peace_04_10_tolstoy_64kb.mp3}{Audio}} 
\fi

\lettrine[lines=2, loversize=0.3, lraise=0]{\initfamily R}{ostov}'s
share in Dolokhov's duel with Bezukhov was hushed up by
the efforts of the old count, and instead of being degraded to
the ranks as he expected he was appointed an adjutant to the
governor general of Moscow. As a result he could not go to the
country with the rest of the family, but was kept all summer in
Moscow by his new duties. Dolokhov recovered, and Rostov became
very friendly with him during his convalescence. Dolokhov lay ill
at his mother's who loved him passionately and tenderly, and old
Mary Ivanovna, who had grown fond of Rostov for his friendship to
her Fedya, often talked to him about her son.

``Yes, Count,'' she would say, ``he is too noble and pure-souled
for our present, depraved world. No one now loves virtue; it
seems like a reproach to everyone. Now tell me, Count, was it
right, was it honorable, of Bezukhov? And Fedya, with his noble
spirit, loved him and even now never says a word against
him. Those pranks in Petersburg when they played some tricks on a
policeman, didn't they do it together? And there! Bezukhov got
off scotfree, while Fedya had to bear the whole burden on his
shoulders. Fancy what he had to go through! It's true he has been
reinstated, but how could they fail to do that? I think there
were not many such gallant sons of the fatherland out there as
he. And now---this duel! Have these people no feeling, or honor?
Knowing him to be an only son, to challenge him and shoot so
straight! It's well God had mercy on us. And what was it for? Who
doesn't have intrigues nowadays? Why, if he was so jealous, as I
see things he should have shown it sooner, but he lets it go on
for months. And then to call him out, reckoning on Fedya not
fighting because he owed him money! What baseness! What meanness!
I know you understand Fedya, my dear count; that, believe me, is
why I am so fond of you. Few people do understand him. He is such
a lofty, heavenly soul!''

Dolokhov himself during his convalescence spoke to Rostov in a
way no one would have expected of him.

``I know people consider me a bad man!'' he said. ``Let them! I
don't care a straw about anyone but those I love; but those I
love, I love so that I would give my life for them, and the
others I'd throttle if they stood in my way. I have an adored, a
priceless mother, and two or three friends---you among them---and
as for the rest I only care about them in so far as they are
harmful or useful. And most of them are harmful, especially the
women. Yes, dear boy,'' he continued, ``I have met loving, noble,
high-minded men, but I have not yet met any women---countesses or
cooks---who were not venal. I have not yet met that divine purity
and devotion I look for in women. If I found such a one I'd give
my life for her! But those!...'' and he made a gesture of
contempt. ``And believe me, if I still value my life it is only
because I still hope to meet such a divine creature, who will
regenerate, purify, and elevate me. But you don't understand
it.''

``Oh, yes, I quite understand,'' answered Rostov, who was under
his new friend's influence.

In the autumn the Rostovs returned to Moscow. Early in the winter
Denisov also came back and stayed with them. The first half of
the winter of 1806, which Nicholas Rostov spent in Moscow, was
one of the happiest, merriest times for him and the whole
family. Nicholas brought many young men to his parents'
house. Vera was a handsome girl of twenty; Sonya a girl of
sixteen with all the charm of an opening flower; Natasha, half
grown up and half child, was now childishly amusing, now
girlishly enchanting.

At that time in the Rostovs' house there prevailed an amorous
atmosphere characteristic of homes where there are very young and
very charming girls. Every young man who came to the
house---seeing those impressionable, smiling young faces (smiling
probably at their own happiness), feeling the eager bustle around
him, and hearing the fitful bursts of song and music and the
inconsequent but friendly prattle of young girls ready for
anything and full of hope---experienced the same feeling; sharing
with the young folk of the Rostovs' household a readiness to fall
in love and an expectation of happiness.

Among the young men introduced by Rostov one of the first was
Dolokhov, whom everyone in the house liked except Natasha. She
almost quarreled with her brother about him. She insisted that he
was a bad man, and that in the duel with Bezukhov, Pierre was
right and Dolokhov wrong, and further that he was disagreeable
and unnatural.

``There's nothing for me to understand,'' she cried out with
resolute self-will, ``he is wicked and heartless. There now, I
like your Denisov though he is a rake and all that, still I like
him; so you see I do understand. I don't know how to put
it... with this one everything is calculated, and I don't like
that. But Denisov...''

``Oh, Denisov is quite different,'' replied Nicholas, implying
that even Denisov was nothing compared to Dolokhov---``you must
understand what a soul there is in Dolokhov, you should see him
with his mother. What a heart!''

``Well, I don't know about that, but I am uncomfortable with
him. And do you know he has fallen in love with Sonya?''

``What nonsense...''

``I'm certain of it; you'll see.''

Natasha's prediction proved true. Dolokhov, who did not usually
care for the society of ladies, began to come often to the house,
and the question for whose sake he came (though no one spoke of
it) was soon settled. He came because of Sonya. And Sonya, though
she would never have dared to say so, knew it and blushed scarlet
every time Dolokhov appeared.

Dolokhov often dined at the Rostovs', never missed a performance
at which they were present, and went to Iogel's balls for young
people which the Rostovs always attended. He was pointedly
attentive to Sonya and looked at her in such a way that not only
could she not bear his glances without coloring, but even the old
countess and Natasha blushed when they saw his looks.

It was evident that this strange, strong man was under the
irresistible influence of the dark, graceful girl who loved
another.

Rostov noticed something new in Dolokhov's relations with Sonya,
but he did not explain to himself what these new relations
were. ``They're always in love with someone,'' he thought of
Sonya and Natasha. But he was not as much at ease with Sonya and
Dolokhov as before and was less frequently at home.

In the autumn of 1806 everybody had again begun talking of the
war with Napoleon with even greater warmth than the year
before. Orders were given to raise recruits, ten men in every
thousand for the regular army, and besides this, nine men in
every thousand for the militia. Everywhere Bonaparte was
anathematized and in Moscow nothing but the coming war was talked
of. For the Rostov family the whole interest of these
preparations for war lay in the fact that Nicholas would not hear
of remaining in Moscow, and only awaited the termination of
Denisov's furlough after Christmas to return with him to their
regiment. His approaching departure did not prevent his amusing
himself, but rather gave zest to his pleasures. He spent the
greater part of his time away from home, at dinners, parties, and
balls.

% % % % % % % % % % % % % % % % % % % % % % % % % % % % % % % % %
% % % % % % % % % % % % % % % % % % % % % % % % % % % % % % % % %
% % % % % % % % % % % % % % % % % % % % % % % % % % % % % % % % %
% % % % % % % % % % % % % % % % % % % % % % % % % % % % % % % % %
% % % % % % % % % % % % % % % % % % % % % % % % % % % % % % % % %
% % % % % % % % % % % % % % % % % % % % % % % % % % % % % % % % %
% % % % % % % % % % % % % % % % % % % % % % % % % % % % % % % % %
% % % % % % % % % % % % % % % % % % % % % % % % % % % % % % % % %
% % % % % % % % % % % % % % % % % % % % % % % % % % % % % % % % %
% % % % % % % % % % % % % % % % % % % % % % % % % % % % % % % % %
% % % % % % % % % % % % % % % % % % % % % % % % % % % % % % % % %
% % % % % % % % % % % % % % % % % % % % % % % % % % % % % %

\chapter*{Chapter XI}
\ifaudio     
\marginpar{
\href{http://ia802609.us.archive.org/34/items/war_and_peace_04_0802_librivox/war_and_peace_04_11_tolstoy_64kb.mp3}{Audio}} 
\fi

\lettrine[lines=2, loversize=0.3, lraise=0]{\initfamily O}{n}
the third day after Christmas Nicholas dined at home, a thing
he had rarely done of late. It was a grand farewell dinner, as he
and Denisov were leaving to join their regiment after
Epiphany. About twenty people were present, including Dolokhov
and Denisov.

Never had love been so much in the air, and never had the amorous
atmosphere made itself so strongly felt in the Rostovs' house as
at this holiday time. ``Seize the moments of happiness, love and
be loved! That is the only reality in the world, all else is
folly. It is the one thing we are interested in here,'' said the
spirit of the place.

Nicholas, having as usual exhausted two pairs of horses, without
visiting all the places he meant to go to and where he had been
invited, returned home just before dinner. As soon as he entered
he noticed and felt the tension of the amorous air in the house,
and also noticed a curious embarrassment among some of those
present. Sonya, Dolokhov, and the old countess were especially
disturbed, and to a lesser degree Natasha. Nicholas understood
that something must have happened between Sonya and Dolokhov
before dinner, and with the kindly sensitiveness natural to him
was very gentle and wary with them both at dinner. On that same
evening there was to be one of the balls that Iogel (the dancing
master) gave for his pupils during the holidays.

``Nicholas, will you come to Iogel's? Please do!'' said
Natasha. ``He asked you, and Vasili Dmitrich\footnote{Denisov.}
is also going.''

``Where would I not go at the countess' command!'' said Denisov,
who at the Rostovs' had jocularly assumed the role of Natasha's
knight. ``I'm even weady to dance the pas de chale.''

``If I have time,'' answered Nicholas. ``But I promised the
Arkharovs; they have a party.''

``And you?'' he asked Dolokhov, but as soon as he had asked the
question he noticed that it should not have been put.

``Perhaps,'' coldly and angrily replied Dolokhov, glancing at
Sonya, and, scowling, he gave Nicholas just such a look as he had
given Pierre at the club dinner.

``There is something up,'' thought Nicholas, and he was further
confirmed in this conclusion by the fact that Dolokhov left
immediately after dinner. He called Natasha and asked her what
was the matter.

``And I was looking for you,'' said Natasha running out to
him. ``I told you, but you would not believe it,'' she said
triumphantly. ``He has proposed to Sonya!''

Little as Nicholas had occupied himself with Sonya of late,
something seemed to give way within him at this news. Dolokhov
was a suitable and in some respects a brilliant match for the
dowerless, orphan girl. From the point of view of the old
countess and of society it was out of the question for her to
refuse him. And therefore Nicholas' first feeling on hearing the
news was one of anger with Sonya... He tried to say, ``That's
capital; of course she'll forget her childish promises and accept
the offer,'' but before he had time to say it Natasha began
again.

``And fancy! she refused him quite definitely!'' adding, after a
pause, ``she told him she loved another.''

``Yes, my Sonya could not have done otherwise!'' thought
Nicholas.

``Much as Mamma pressed her, she refused, and I know she won't
change once she has said...''

``And Mamma pressed her!'' said Nicholas reproachfully.

``Yes,'' said Natasha. ``Do you know, Nicholas---don't be
angry---but I know you will not marry her. I know, heaven knows
how, but I know for certain that you won't marry her.''

``Now you don't know that at all!'' said Nicholas. ``But I must
talk to her. What a darling Sonya is!'' he added with a smile.

``Ah, she is indeed a darling! I'll send her to you.''

And Natasha kissed her brother and ran away.

A minute later Sonya came in with a frightened, guilty, and
scared look.  Nicholas went up to her and kissed her hand. This
was the first time since his return that they had talked alone
and about their love.

``Sophie,'' he began, timidly at first and then more and more
boldly, ``if you wish to refuse one who is not only a brilliant
and advantageous match but a splendid, noble fellow... he is my
friend...''

Sonya interrupted him.

``I have already refused,'' she said hurriedly.

``If you are refusing for my sake, I am afraid that I...''

Sonya again interrupted. She gave him an imploring, frightened
look.

``Nicholas, don't tell me that!'' she said.

``No, but I must. It may be arrogant of me, but still it is best
to say it. If you refuse him on my account, I must tell you the
whole truth. I love you, and I think I love you more than anyone
else...''

``That is enough for me,'' said Sonya, blushing.

``No, but I have been in love a thousand times and shall fall in
love again, though for no one have I such a feeling of
friendship, confidence, and love as I have for you. Then I am
young. Mamma does not wish it. In a word, I make no promise. And
I beg you to consider Dolokhov's offer,'' he said, articulating
his friend's name with difficulty.

``Don't say that to me! I want nothing. I love you as a brother
and always shall, and I want nothing more.''

``You are an angel: I am not worthy of you, but I am afraid of
misleading you.''

And Nicholas again kissed her hand.

% % % % % % % % % % % % % % % % % % % % % % % % % % % % % % % % %
% % % % % % % % % % % % % % % % % % % % % % % % % % % % % % % % %
% % % % % % % % % % % % % % % % % % % % % % % % % % % % % % % % %
% % % % % % % % % % % % % % % % % % % % % % % % % % % % % % % % %
% % % % % % % % % % % % % % % % % % % % % % % % % % % % % % % % %
% % % % % % % % % % % % % % % % % % % % % % % % % % % % % % % % %
% % % % % % % % % % % % % % % % % % % % % % % % % % % % % % % % %
% % % % % % % % % % % % % % % % % % % % % % % % % % % % % % % % %
% % % % % % % % % % % % % % % % % % % % % % % % % % % % % % % % %
% % % % % % % % % % % % % % % % % % % % % % % % % % % % % % % % %
% % % % % % % % % % % % % % % % % % % % % % % % % % % % % % % % %
% % % % % % % % % % % % % % % % % % % % % % % % % % % % % %

\chapter*{Chapter XII}
\ifaudio     
\marginpar{
\href{http://ia802609.us.archive.org/34/items/war_and_peace_04_0802_librivox/war_and_peace_04_12_tolstoy_64kb.mp3}{Audio}} 
\fi

\lettrine[lines=2, loversize=0.3, lraise=0]{\initfamily I}{ogel}'s
were the most enjoyable balls in Moscow. So said the
mothers as they watched their young people executing their newly
learned steps, and so said the youths and maidens themselves as
they danced till they were ready to drop, and so said the
grown-up young men and women who came to these balls with an air
of condescension and found them most enjoyable.  That year two
marriages had come of these balls. The two pretty young
Princesses Gorchakov met suitors there and were married and so
further increased the fame of these dances. What distinguished
them from others was the absence of host or hostess and the
presence of the good-natured Iogel, flying about like a feather
and bowing according to the rules of his art, as he collected the
tickets from all his visitors. There was the fact that only those
came who wished to dance and amuse themselves as girls of
thirteen and fourteen do who are wearing long dresses for the
first time. With scarcely any exceptions they all were, or seemed
to be, pretty---so rapturous were their smiles and so sparkling
their eyes.  Sometimes the best of the pupils, of whom Natasha,
who was exceptionally graceful, was first, even danced the pas de
chale, but at this last ball only the ecossaise, the anglaise,
and the mazurka, which was just coming into fashion, were
danced. Iogel had taken a ballroom in Bezukhov's house, and the
ball, as everyone said, was a great success. There were many
pretty girls and the Rostov girls were among the prettiest. They
were both particularly happy and gay. That evening, proud of
Dolokhov's proposal, her refusal, and her explanation with
Nicholas, Sonya twirled about before she left home so that the
maid could hardly get her hair plaited, and she was transparently
radiant with impulsive joy.

Natasha no less proud of her first long dress and of being at a
real ball was even happier. They were both dressed in white
muslin with pink ribbons.

Natasha fell in love the very moment she entered the
ballroom. She was not in love with anyone in particular, but with
everyone. Whatever person she happened to look at she was in love
with for that moment.

``Oh, how delightful it is!'' she kept saying, running up to
Sonya.

Nicholas and Denisov were walking up and down, looking with
kindly patronage at the dancers.

``How sweet she is---she will be a weal beauty!'' said Denisov.

``Who?''

``Countess Natasha,'' answered Denisov.

``And how she dances! What gwace!'' he said again after a pause.

``Who are you talking about?''

``About your sister,'' ejaculated Denisov testily.

Rostov smiled.

``My dear count, you were one of my best pupils---you must
dance,'' said little Iogel coming up to Nicholas. ``Look how many
charming young ladies-'' He turned with the same request to
Denisov who was also a former pupil of his.

``No, my dear fellow, I'll be a wallflower,'' said
Denisov. ``Don't you wecollect what bad use I made of your
lessons?''

``Oh no!'' said Iogel, hastening to reassure him. ``You were only
inattentive, but you had talent---oh yes, you had talent!''

The band struck up the newly introduced mazurka. Nicholas could
not refuse Iogel and asked Sonya to dance. Denisov sat down by
the old ladies and, leaning on his saber and beating time with
his foot, told them something funny and kept them amused, while
he watched the young people dancing, Iogel with Natasha, his
pride and his best pupil, were the first couple. Noiselessly,
skillfully stepping with his little feet in low shoes, Iogel flew
first across the hall with Natasha, who, though shy, went on
carefully executing her steps. Denisov did not take his eyes off
her and beat time with his saber in a way that clearly indicated
that if he was not dancing it was because he would not and not
because he could not. In the middle of a figure he beckoned to
Rostov who was passing:

``This is not at all the thing,'' he said. ``What sort of Polish
mazuwka is this? But she does dance splendidly.''

Knowing that Denisov had a reputation even in Poland for the
masterly way in which he danced the mazurka, Nicholas ran up to
Natasha:

``Go and choose Denisov. He is a real dancer, a wonder!'' he
said.

When it came to Natasha's turn to choose a partner, she rose and,
tripping rapidly across in her little shoes trimmed with bows,
ran timidly to the corner where Denisov sat. She saw that
everybody was looking at her and waiting. Nicholas saw that
Denisov was refusing though he smiled delightedly. He ran up to
them.

``Please, Vasili Dmitrich,'' Natasha was saying, ``do come!''

``Oh no, let me off, Countess,'' Denisov replied.

``Now then, Vaska,'' said Nicholas.

``They coax me as if I were Vaska the cat!'' said Denisov
jokingly.

``I'll sing for you a whole evening,'' said Natasha.

``Oh, the faiwy! She can do anything with me!'' said Denisov, and
he unhooked his saber. He came out from behind the chairs,
clasped his partner's hand firmly, threw back his head, and
advanced his foot, waiting for the beat. Only on horse back and
in the mazurka was Denisov's short stature not noticeable and he
looked the fine fellow he felt himself to be. At the right beat
of the music he looked sideways at his partner with a merry and
triumphant air, suddenly stamped with one foot, bounded from the
floor like a ball, and flew round the room taking his partner
with him. He glided silently on one foot half across the room,
and seeming not to notice the chairs was dashing straight at
them, when suddenly, clinking his spurs and spreading out his
legs, he stopped short on his heels, stood so a second, stamped
on the spot clanking his spurs, whirled rapidly round, and,
striking his left heel against his right, flew round again in a
circle. Natasha guessed what he meant to do, and abandoning
herself to him followed his lead hardly knowing how.  First he
spun her round, holding her now with his left, now with his right
hand, then falling on one knee he twirled her round him, and
again jumping up, dashed so impetuously forward that it seemed as
if he would rush through the whole suite of rooms without drawing
breath, and then he suddenly stopped and performed some new and
unexpected steps. When at last, smartly whirling his partner
round in front of her chair, he drew up with a click of his spurs
and bowed to her, Natasha did not even make him a curtsy. She
fixed her eyes on him in amazement, smiling as if she did not
recognize him.

``What does this mean?'' she brought out.

Although Iogel did not acknowledge this to be the real mazurka,
everyone was delighted with Denisov's skill, he was asked again
and again as a partner, and the old men began smilingly to talk
about Poland and the good old days. Denisov, flushed after the
mazurka and mopping himself with his handkerchief, sat down by
Natasha and did not leave her for the rest of the evening.

% % % % % % % % % % % % % % % % % % % % % % % % % % % % % % % % %
% % % % % % % % % % % % % % % % % % % % % % % % % % % % % % % % %
% % % % % % % % % % % % % % % % % % % % % % % % % % % % % % % % %
% % % % % % % % % % % % % % % % % % % % % % % % % % % % % % % % %
% % % % % % % % % % % % % % % % % % % % % % % % % % % % % % % % %
% % % % % % % % % % % % % % % % % % % % % % % % % % % % % % % % %
% % % % % % % % % % % % % % % % % % % % % % % % % % % % % % % % %
% % % % % % % % % % % % % % % % % % % % % % % % % % % % % % % % %
% % % % % % % % % % % % % % % % % % % % % % % % % % % % % % % % %
% % % % % % % % % % % % % % % % % % % % % % % % % % % % % % % % %
% % % % % % % % % % % % % % % % % % % % % % % % % % % % % % % % %
% % % % % % % % % % % % % % % % % % % % % % % % % % % % % %

\chapter*{Chapter XIII}
\ifaudio     
\marginpar{
\href{http://ia802609.us.archive.org/34/items/war_and_peace_04_0802_librivox/war_and_peace_04_13_tolstoy_64kb.mp3}{Audio}} 
\fi

\lettrine[lines=2, loversize=0.3, lraise=0]{\initfamily F}{or}
two days after that Rostov did not see Dolokhov at his own or
at Dolokhov's home: on the third day he received a note from him:

As I do not intend to be at your house again for reasons you know
of, and am going to rejoin my regiment, I am giving a farewell
supper tonight to my friends---come to the English Hotel.

About ten o'clock Rostov went to the English Hotel straight from
the theater, where he had been with his family and Denisov. He
was at once shown to the best room, which Dolokhov had taken for
that evening. Some twenty men were gathered round a table at
which Dolokhov sat between two candles. On the table was a pile
of gold and paper money, and he was keeping the bank. Rostov had
not seen him since his proposal and Sonya's refusal and felt
uncomfortable at the thought of how they would meet.

Dolokhov's clear, cold glance met Rostov as soon as he entered
the door, as though he had long expected him.

``It's a long time since we met,'' he said. ``Thanks for
coming. I'll just finish dealing, and then Ilyushka will come
with his chorus.''

``I called once or twice at your house,'' said Rostov, reddening.

Dolokhov made no reply.

``You may punt,'' he said.

Rostov recalled at that moment a strange conversation he had once
had with Dolokhov. ``None but fools trust to luck in play,''
Dolokhov had then said.

``Or are you afraid to play with me?'' Dolokhov now asked as if
guessing Rostov's thought.

Beneath his smile Rostov saw in him the mood he had shown at the
club dinner and at other times, when as if tired of everyday life
he had felt a need to escape from it by some strange, and usually
cruel, action.

Rostov felt ill at ease. He tried, but failed, to find some joke
with which to reply to Dolokhov's words. But before he had
thought of anything, Dolokhov, looking straight in his face, said
slowly and deliberately so that everyone could hear:

``Do you remember we had a talk about cards... 'He's a fool who
trusts to luck, one should make certain,' and I want to try.''

``To try his luck or the certainty?'' Rostov asked himself.

``Well, you'd better not play,'' Dolokhov added, and springing a
new pack of cards said: ``Bank, gentlemen!''

Moving the money forward he prepared to deal. Rostov sat down by
his side and at first did not play. Dolokhov kept glancing at
him.

``Why don't you play?'' he asked.

And strange to say Nicholas felt that he could not help taking up
a card, putting a small stake on it, and beginning to play.

``I have no money with me,'' he said.

``I'll trust you.''

Rostov staked five rubles on a card and lost, staked again, and
again lost. Dolokhov ``killed,'' that is, beat, ten cards of
Rostov's running.

``Gentlemen,'' said Dolokhov after he had dealt for some
time. ``Please place your money on the cards or I may get muddled
in the reckoning.''

One of the players said he hoped he might be trusted.

``Yes, you might, but I am afraid of getting the accounts
mixed. So I ask you to put the money on your cards,'' replied
Dolokhov. ``Don't stint yourself, we'll settle afterwards,'' he
added, turning to Rostov.

The game continued; a waiter kept handing round champagne.

All Rostov's cards were beaten and he had eight hundred rubles
scored up against him. He wrote ``800 rubles'' on a card, but
while the waiter filled his glass he changed his mind and altered
it to his usual stake of twenty rubles.

``Leave it,'' said Dolokhov, though he did not seem to be even
looking at Rostov, ``you'll win it back all the sooner. I lose to
the others but win from you. Or are you afraid of me?'' he asked
again.

Rostov submitted. He let the eight hundred remain and laid down a
seven of hearts with a torn corner, which he had picked up from
the floor. He well remembered that seven afterwards. He laid down
the seven of hearts, on which with a broken bit of chalk he had
written \emph{800 rubles} in clear upright figures; he emptied the
glass of warm champagne that was handed him, smiled at Dolokhov's
words, and with a sinking heart, waiting for a seven to turn up,
gazed at Dolokhov's hands which held the pack. Much depended on
Rostov's winning or losing on that seven of hearts. On the
previous Sunday the old count had given his son two thousand
rubles, and though he always disliked speaking of money
difficulties had told Nicholas that this was all he could let him
have till May, and asked him to be more economical this
time. Nicholas had replied that it would be more than enough for
him and that he gave his word of honor not to take anything more
till the spring. Now only twelve hundred rubles was left of that
money, so that this seven of hearts meant for him not only the
loss of sixteen hundred rubles, but the necessity of going back
on his word. With a sinking heart he watched Dolokhov's hands and
thought, ``Now then, make haste and let me have this card and
I'll take my cap and drive home to supper with Denisov, Natasha,
and Sonya, and will certainly never touch a card again.'' At that
moment his home life, jokes with Petya, talks with Sonya, duets
with Natasha, piquet with his father, and even his comfortable
bed in the house on the Povarskaya rose before him with such
vividness, clearness, and charm that it seemed as if it were all
a lost and unappreciated bliss, long past. He could not conceive
that a stupid chance, letting the seven be dealt to the right
rather than to the left, might deprive him of all this happiness,
newly appreciated and newly illumined, and plunge him into the
depths of unknown and undefined misery. That could not be, yet he
awaited with a sinking heart the movement of Dolokhov's
hands. Those broad, reddish hands, with hairy wrists visible from
under the shirt cuffs, laid down the pack and took up a glass and
a pipe that were handed him.

``So you are not afraid to play with me?'' repeated Dolokhov, and
as if about to tell a good story he put down the cards, leaned
back in his chair, and began deliberately with a smile:

``Yes, gentlemen, I've been told there's a rumor going about
Moscow that I'm a sharper, so I advise you to be careful.''

``Come now, deal!'' exclaimed Rostov.

``Oh, those Moscow gossips!'' said Dolokhov, and he took up the
cards with a smile.

``Aah!'' Rostov almost screamed lifting both hands to his
head. The seven he needed was lying uppermost, the first card in
the pack. He had lost more than he could pay.

``Still, don't ruin yourself!'' said Dolokhov with a side glance
at Rostov as he continued to deal.

% % % % % % % % % % % % % % % % % % % % % % % % % % % % % % % % %
% % % % % % % % % % % % % % % % % % % % % % % % % % % % % % % % %
% % % % % % % % % % % % % % % % % % % % % % % % % % % % % % % % %
% % % % % % % % % % % % % % % % % % % % % % % % % % % % % % % % %
% % % % % % % % % % % % % % % % % % % % % % % % % % % % % % % % %
% % % % % % % % % % % % % % % % % % % % % % % % % % % % % % % % %
% % % % % % % % % % % % % % % % % % % % % % % % % % % % % % % % %
% % % % % % % % % % % % % % % % % % % % % % % % % % % % % % % % %
% % % % % % % % % % % % % % % % % % % % % % % % % % % % % % % % %
% % % % % % % % % % % % % % % % % % % % % % % % % % % % % % % % %
% % % % % % % % % % % % % % % % % % % % % % % % % % % % % % % % %
% % % % % % % % % % % % % % % % % % % % % % % % % % % % % %

\chapter*{Chapter XIV}
\ifaudio     
\marginpar{
\href{http://ia802609.us.archive.org/34/items/war_and_peace_04_0802_librivox/war_and_peace_04_14_tolstoy_64kb.mp3}{Audio}} 
\fi

\lettrine[lines=2, loversize=0.3, lraise=0]{\initfamily A}{n}
hour and a half later most of the players were but little
interested in their own play.

The whole interest was concentrated on Rostov. Instead of sixteen
hundred rubles he had a long column of figures scored against
him, which he had reckoned up to ten thousand, but that now, as
he vaguely supposed, must have risen to fifteen thousand. In
reality it already exceeded twenty thousand rubles. Dolokhov was
no longer listening to stories or telling them, but followed
every movement of Rostov's hands and occasionally ran his eyes
over the score against him. He had decided to play until that
score reached forty-three thousand. He had fixed on that number
because forty-three was the sum of his and Sonya's joint
ages. Rostov, leaning his head on both hands, sat at the table
which was scrawled over with figures, wet with spilled wine, and
littered with cards. One tormenting impression did not leave him:
that those broad-boned reddish hands with hairy wrists visible
from under the shirt sleeves, those hands which he loved and
hated, held him in their power.

``Six hundred rubles, ace, a corner, a nine... winning it back's
impossible... Oh, how pleasant it was at home!... The knave,
double or quits... it can't be!... And why is he doing this to
me?'' Rostov pondered. Sometimes he staked a large sum, but
Dolokhov refused to accept it and fixed the stake
himself. Nicholas submitted to him, and at one moment prayed to
God as he had done on the battlefield at the bridge over the
Enns, and then guessed that the card that came first to hand from
the crumpled heap under the table would save him, now counted the
cords on his coat and took a card with that number and tried
staking the total of his losses on it, then he looked round for
aid from the other players, or peered at the now cold face of
Dolokhov and tried to read what was passing in his mind.

``He knows of course what this loss means to me. He can't want my
ruin.  Wasn't he my friend? Wasn't I fond of him? But it's not
his fault.  What's he to do if he has such luck?... And it's not
my fault either,'' he thought to himself, ``I have done nothing
wrong. Have I killed anyone, or insulted or wished harm to
anyone? Why such a terrible misfortune?  And when did it begin?
Such a little while ago I came to this table with the thought of
winning a hundred rubles to buy that casket for Mamma's name day
and then going home. I was so happy, so free, so lighthearted!
And I did not realize how happy I was! When did that end and when
did this new, terrible state of things begin? What marked the
change? I sat all the time in this same place at this table,
chose and placed cards, and watched those broad-boned agile hands
in the same way. When did it happen and what has happened? I am
well and strong and still the same and in the same place. No, it
can't be! Surely it will all end in nothing!''

He was flushed and bathed in perspiration, though the room was
not hot.  His face was terrible and piteous to see, especially
from its helpless efforts to seem calm.

The score against him reached the fateful sum of forty-three
thousand.  Rostov had just prepared a card, by bending the corner
of which he meant to double the three thousand just put down to
his score, when Dolokhov, slamming down the pack of cards, put it
aside and began rapidly adding up the total of Rostov's debt,
breaking the chalk as he marked the figures in his clear, bold
hand.

``Supper, it's time for supper! And here are the gypsies!''

Some swarthy men and women were really entering from the cold
outside and saying something in their gypsy accents. Nicholas
understood that it was all over; but he said in an indifferent
tone:

``Well, won't you go on? I had a splendid card all ready,'' as if
it were the fun of the game which interested him most.

``It's all up! I'm lost!'' thought he. ``Now a bullet through my
brain---that's all that's left me!'' And at the same time he said
in a cheerful voice:

``Come now, just this one more little card!''

``All right!'' said Dolokhov, having finished the addition. ``All
right!  Twenty-one rubles,'' he said, pointing to the figure
twenty-one by which the total exceeded the round sum of
forty-three thousand; and taking up a pack he prepared to
deal. Rostov submissively unbent the corner of his card and,
instead of the six thousand he had intended, carefully wrote
twenty-one.

``It's all the same to me,'' he said. ``I only want to see
whether you will let me win this ten, or beat it.''

Dolokhov began to deal seriously. Oh, how Rostov detested at that
moment those hands with their short reddish fingers and hairy
wrists, which held him in their power... The ten fell to him.

``You owe forty-three thousand, Count,'' said Dolokhov, and
stretching himself he rose from the table. ``One does get tired
sitting so long,'' he added.

``Yes, I'm tired too,'' said Rostov.

Dolokhov cut him short, as if to remind him that it was not for
him to jest.

``When am I to receive the money, Count?''

Rostov, flushing, drew Dolokhov into the next room.

``I cannot pay it all immediately. Will you take an I.O.U.?'' he
said.

``I say, Rostov,'' said Dolokhov clearly, smiling and looking
Nicholas straight in the eyes, ``you know the saying, 'Lucky in
love, unlucky at cards.' Your cousin is in love with you, I
know.''

``Oh, it's terrible to feel oneself so in this man's power,''
thought Rostov. He knew what a shock he would inflict on his
father and mother by the news of this loss, he knew what a relief
it would be to escape it all, and felt that Dolokhov knew that he
could save him from all this shame and sorrow, but wanted now to
play with him as a cat does with a mouse.

``Your cousin...'' Dolokhov started to say, but Nicholas
interrupted him.

``My cousin has nothing to do with this and it's not necessary to
mention her!'' he exclaimed fiercely.

``Then when am I to have it?''

``Tomorrow,'' replied Rostov and left the room.

% % % % % % % % % % % % % % % % % % % % % % % % % % % % % % % % %
% % % % % % % % % % % % % % % % % % % % % % % % % % % % % % % % %
% % % % % % % % % % % % % % % % % % % % % % % % % % % % % % % % %
% % % % % % % % % % % % % % % % % % % % % % % % % % % % % % % % %
% % % % % % % % % % % % % % % % % % % % % % % % % % % % % % % % %
% % % % % % % % % % % % % % % % % % % % % % % % % % % % % % % % %
% % % % % % % % % % % % % % % % % % % % % % % % % % % % % % % % %
% % % % % % % % % % % % % % % % % % % % % % % % % % % % % % % % %
% % % % % % % % % % % % % % % % % % % % % % % % % % % % % % % % %
% % % % % % % % % % % % % % % % % % % % % % % % % % % % % % % % %
% % % % % % % % % % % % % % % % % % % % % % % % % % % % % % % % %
% % % % % % % % % % % % % % % % % % % % % % % % % % % % % %

\chapter*{Chapter XV}
\ifaudio     
\marginpar{
\href{http://ia802609.us.archive.org/34/items/war_and_peace_04_0802_librivox/war_and_peace_04_15_tolstoy_64kb.mp3}{Audio}} 
\fi

\lettrine[lines=2, loversize=0.3, lraise=0]{\initfamily T}{o}
say \emph{tomorrow} and keep up a dignified tone was not
difficult, but to go home alone, see his sisters, brother,
mother, and father, confess and ask for money he had no right to
after giving his word of honor, was terrible.

At home, they had not yet gone to bed. The young people, after
returning from the theater, had had supper and were grouped round
the clavichord.  As soon as Nicholas entered, he was enfolded in
that poetic atmosphere of love which pervaded the Rostov
household that winter and, now after Dolokhov's proposal and
Iogel's ball, seemed to have grown thicker round Sonya and
Natasha as the air does before a thunderstorm. Sonya and Natasha,
in the light-blue dresses they had worn at the theater, looking
pretty and conscious of it, were standing by the clavichord,
happy and smiling. Vera was playing chess with Shinshin in the
drawing room. The old countess, waiting for the return of her
husband and son, sat playing patience with the old gentlewoman
who lived in their house. Denisov, with sparkling eyes and
ruffled hair, sat at the clavichord striking chords with his
short fingers, his legs thrown back and his eyes rolling as he
sang, with his small, husky, but true voice, some verses called
\emph{Enchantress}, which he had composed, and to which he was
trying to fit music:

Enchantress, say, to my forsaken lyre What magic power is this
recalls me still? What spark has set my inmost soul on fire, What
is this bliss that makes my fingers thrill?

He was singing in passionate tones, gazing with his sparkling
black-agate eyes at the frightened and happy Natasha.

``Splendid! Excellent!'' exclaimed Natasha. ``Another verse,''
she said, without noticing Nicholas.

``Everything's still the same with them,'' thought Nicholas,
glancing into the drawing room, where he saw Vera and his mother
with the old lady.

``Ah, and here's Nicholas!'' cried Natasha, running up to him.

``Is Papa at home?'' he asked.

``I am so glad you've come!'' said Natasha, without answering
him. ``We are enjoying ourselves! Vasili Dmitrich is staying a
day longer for my sake!  Did you know?''

``No, Papa is not back yet,'' said Sonya.

``Nicholas, have you come? Come here, dear!'' called the old
countess from the drawing room.

Nicholas went to her, kissed her hand, and sitting down silently
at her table began to watch her hands arranging the cards. From
the dancing room, they still heard the laughter and merry voices
trying to persuade Natasha to sing.

``All wight! All wight!'' shouted Denisov. ``It's no good making
excuses now! It's your turn to sing the ba'cawolla---I entweat
you!''

The countess glanced at her silent son.

``What is the matter?'' she asked.

``Oh, nothing,'' said he, as if weary of being continually asked
the same question. ``Will Papa be back soon?''

``I expect so.''

``Everything's the same with them. They know nothing about it!
Where am I to go?'' thought Nicholas, and went again into the
dancing room where the clavichord stood.

Sonya was sitting at the clavichord, playing the prelude to
Denisov's favorite barcarolle. Natasha was preparing to
sing. Denisov was looking at her with enraptured eyes.

Nicholas began pacing up and down the room.

``Why do they want to make her sing? How can she sing? There's
nothing to be happy about!'' thought he.

Sonya struck the first chord of the prelude.

``My God, I'm a ruined and dishonored man! A bullet through my
brain is the only thing left me---not singing!'' his thoughts ran
on. ``Go away? But where to? It's one---let them sing!''

He continued to pace the room, looking gloomily at Denisov and
the girls and avoiding their eyes.

``Nikolenka, what is the matter?'' Sonya's eyes fixed on him
seemed to ask. She noticed at once that something had happened to
him.

Nicholas turned away from her. Natasha too, with her quick
instinct, had instantly noticed her brother's condition. But,
though she noticed it, she was herself in such high spirits at
that moment, so far from sorrow, sadness, or self-reproach, that
she purposely deceived herself as young people often do. ``No, I
am too happy now to spoil my enjoyment by sympathy with anyone's
sorrow,'' she felt, and she said to herself: ``No, I must be
mistaken, he must be feeling happy, just as I am.''

``Now, Sonya!'' she said, going to the very middle of the room,
where she considered the resonance was best.

Having lifted her head and let her arms droop lifelessly, as
ballet dancers do, Natasha, rising energetically from her heels
to her toes, stepped to the middle of the room and stood still.

``Yes, that's me!'' she seemed to say, answering the rapt gaze
with which Denisov followed her.

``And what is she so pleased about?'' thought Nicholas, looking
at his sister. ``Why isn't she dull and ashamed?''

Natasha took the first note, her throat swelled, her chest rose,
her eyes became serious. At that moment she was oblivious of her
surroundings, and from her smiling lips flowed sounds which
anyone may produce at the same intervals and hold for the same
time, but which leave you cold a thousand times and the thousand
and first time thrill you and make you weep.

Natasha, that winter, had for the first time begun to sing
seriously, mainly because Denisov so delighted in her
singing. She no longer sang as a child, there was no longer in
her singing that comical, childish, painstaking effect that had
been in it before; but she did not yet sing well, as all the
connoisseurs who heard her said: ``It is not trained, but it is a
beautiful voice that must be trained.'' Only they generally said
this some time after she had finished singing. While that
untrained voice, with its incorrect breathing and labored
transitions, was sounding, even the connoisseurs said nothing,
but only delighted in it and wished to hear it again. In her
voice there was a virginal freshness, an unconsciousness of her
own powers, and an as yet untrained velvety softness, which so
mingled with her lack of art in singing that it seemed as if
nothing in that voice could be altered without spoiling it.

``What is this?'' thought Nicholas, listening to her with widely
opened eyes. ``What has happened to her? How she is singing
today!'' And suddenly the whole world centered for him on
anticipation of the next note, the next phrase, and everything in
the world was divided into three beats: ``Oh mio crudele
affetto.''... One, two, three... one, two, three...  One... ``Oh
mio crudele affetto.''... One, two, three... One. ``Oh, this
senseless life of ours!'' thought Nicholas. ``All this misery,
and money, and Dolokhov, and anger, and honor---it's all
nonsense... but this is real... Now then, Natasha, now then,
dearest! Now then, darling! How will she take that si? She's
taken it! Thank God!'' And without noticing that he was singing,
to strengthen the si he sung a second, a third below the high
note. ``Ah, God! How fine! Did I really take it? How fortunate!''
he thought.

Oh, how that chord vibrated, and how moved was something that was
finest in Rostov's soul! And this something was apart from
everything else in the world and above everything in the
world. ``What were losses, and Dolokhov, and words of
honor?... All nonsense! One might kill and rob and yet be
happy...''

% % % % % % % % % % % % % % % % % % % % % % % % % % % % % % % % %
% % % % % % % % % % % % % % % % % % % % % % % % % % % % % % % % %
% % % % % % % % % % % % % % % % % % % % % % % % % % % % % % % % %
% % % % % % % % % % % % % % % % % % % % % % % % % % % % % % % % %
% % % % % % % % % % % % % % % % % % % % % % % % % % % % % % % % %
% % % % % % % % % % % % % % % % % % % % % % % % % % % % % % % % %
% % % % % % % % % % % % % % % % % % % % % % % % % % % % % % % % %
% % % % % % % % % % % % % % % % % % % % % % % % % % % % % % % % %
% % % % % % % % % % % % % % % % % % % % % % % % % % % % % % % % %
% % % % % % % % % % % % % % % % % % % % % % % % % % % % % % % % %
% % % % % % % % % % % % % % % % % % % % % % % % % % % % % % % % %
% % % % % % % % % % % % % % % % % % % % % % % % % % % % % %

\chapter*{Chapter XVI}
\ifaudio
\marginpar{
\href{http://ia802609.us.archive.org/34/items/war_and_peace_04_0802_librivox/war_and_peace_04_16_tolstoy_64kb.mp3}{Audio}} 
\fi

\lettrine[lines=2, loversize=0.3, lraise=0]{\initfamily I}{t}
was long since Rostov had felt such enjoyment from music as he
did that day. But no sooner had Natasha finished her barcarolle
than reality again presented itself. He got up without saying a
word and went downstairs to his own room. A quarter of an hour
later the old count came in from his club, cheerful and
contented. Nicholas, hearing him drive up, went to meet him.

``Well---had a good time?'' said the old count, smiling gaily and
proudly at his son.

Nicholas tried to say ``Yes,'' but could not: and he nearly burst
into sobs. The count was lighting his pipe and did not notice his
son's condition.

``Ah, it can't be avoided!'' thought Nicholas, for the first and
last time. And suddenly, in the most casual tone, which made him
feel ashamed of himself, he said, as if merely asking his father
to let him have the carriage to drive to town:

``Papa, I have come on a matter of business. I was nearly
forgetting. I need some money.''

``Dear me!'' said his father, who was in a specially good
humor. ``I told you it would not be enough. How much?''

``Very much,'' said Nicholas flushing, and with a stupid careless
smile, for which he was long unable to forgive himself, ``I have
lost a little, I mean a good deal, a great deal---forty three
thousand.''

``What! To whom?... Nonsense!'' cried the count, suddenly
reddening with an apoplectic flush over neck and nape as old
people do.

``I promised to pay tomorrow,'' said Nicholas.

``Well!...'' said the old count, spreading out his arms and
sinking helplessly on the sofa.

``It can't be helped It happens to everyone!'' said the son, with
a bold, free, and easy tone, while in his soul he regarded
himself as a worthless scoundrel whose whole life could not atone
for his crime. He longed to kiss his father's hands and kneel to
beg his forgiveness, but said, in a careless and even rude voice,
that it happens to everyone!

The old count cast down his eyes on hearing his son's words and
began bustlingly searching for something.

``Yes, yes,'' he muttered, ``it will be difficult, I fear,
difficult to raise... happens to everybody! Yes, who has not done
it?''

And with a furtive glance at his son's face, the count went out
of the room... Nicholas had been prepared for resistance, but had
not at all expected this.

``Papa! Pa-pa!'' he called after him, sobbing, ``forgive me!''
And seizing his father's hand, he pressed it to his lips and
burst into tears.

While father and son were having their explanation, the mother
and daughter were having one not less important. Natasha came
running to her mother, quite excited.

``Mamma!... Mamma!... He has made me...''

``Made what?''

``Made, made me an offer, Mamma! Mamma!'' she exclaimed.

The countess did not believe her ears. Denisov had proposed. To
whom? To this chit of a girl, Natasha, who not so long ago was
playing with dolls and who was still having lessons.

``Don't, Natasha! What nonsense!'' she said, hoping it was a
joke.

``Nonsense, indeed! I am telling you the fact,'' said Natasha
indignantly.  ``I come to ask you what to do, and you call it
'nonsense!'''

The countess shrugged her shoulders.

``If it is true that Monsieur Denisov has made you a proposal,
tell him he is a fool, that's all!''

``No, he's not a fool!'' replied Natasha indignantly and
seriously.

``Well then, what do you want? You're all in love nowadays. Well,
if you are in love, marry him!'' said the countess, with a laugh
of annoyance.  ``Good luck to you!''

``No, Mamma, I'm not in love with him, I suppose I'm not in love
with him.''

``Well then, tell him so.''

``Mamma, are you cross? Don't be cross, dear! Is it my fault?''

``No, but what is it, my dear? Do you want me to go and tell
him?'' said the countess smiling.

``No, I will do it myself, only tell me what to say. It's all
very well for you,'' said Natasha, with a responsive smile. ``You
should have seen how he said it! I know he did not mean to say
it, but it came out accidently.''

``Well, all the same, you must refuse him.''

``No, I mustn't. I am so sorry for him! He's so nice.''

``Well then, accept his offer. It's high time for you to be
married,'' answered the countess sharply and sarcastically.

``No, Mamma, but I'm so sorry for him. I don't know how I'm to
say it.''

``And there's nothing for you to say. I shall speak to him
myself,'' said the countess, indignant that they should have
dared to treat this little Natasha as grown up.

``No, not on any account! I will tell him myself, and you'll
listen at the door,'' and Natasha ran across the drawing room to
the dancing hall, where Denisov was sitting on the same chair by
the clavichord with his face in his hands.

He jumped up at the sound of her light step.

``Nataly,'' he said, moving with rapid steps toward her, ``decide
my fate.  It is in your hands.''

``Vasili Dmitrich, I'm so sorry for you!... No, but you are so
nice...  but it won't do...not that... but as a friend, I shall
always love you.''

Denisov bent over her hand and she heard strange sounds she did
not understand. She kissed his rough curly black head. At this
instant, they heard the quick rustle of the countess' dress. She
came up to them.

``Vasili Dmitrich, I thank you for the honor,'' she said, with an
embarrassed voice, though it sounded severe to Denisov---``but my
daughter is so young, and I thought that, as my son's friend, you
would have addressed yourself first to me. In that case you would
not have obliged me to give this refusal.''

``Countess...'' said Denisov, with downcast eyes and a guilty
face. He tried to say more, but faltered.

Natasha could not remain calm, seeing him in such a plight. She
began to sob aloud.

``Countess, I have done w'ong,'' Denisov went on in an unsteady
voice, ``but believe me, I so adore your daughter and all your
family that I would give my life twice over...'' He looked at the
countess, and seeing her severe face said: ``Well, good-by,
Countess,'' and kissing her hand, he left the room with quick
resolute strides, without looking at Natasha.

Next day Rostov saw Denisov off. He did not wish to stay another
day in Moscow. All Denisov's Moscow friends gave him a farewell
entertainment at the gypsies', with the result that he had no
recollection of how he was put in the sleigh or of the first
three stages of his journey.

After Denisov's departure, Rostov spent another fortnight in
Moscow, without going out of the house, waiting for the money his
father could not at once raise, and he spent most of his time in
the girls' room.

Sonya was more tender and devoted to him than ever. It was as if
she wanted to show him that his losses were an achievement that
made her love him all the more, but Nicholas now considered
himself unworthy of her.

He filled the girls' albums with verses and music, and having at
last sent Dolokhov the whole forty-three thousand rubles and
received his receipt, he left at the end of November, without
taking leave of any of his acquaintances, to overtake his
regiment which was already in Poland.
