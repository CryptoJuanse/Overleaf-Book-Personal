\part*{Book Eleven: 1812}

% % % % % % % % % % % % % % % % % % % % % % % % % % % % % % % % %
% % % % % % % % % % % % % % % % % % % % % % % % % % % % % % % % %
% % % % % % % % % % % % % % % % % % % % % % % % % % % % % % % % %
% % % % % % % % % % % % % % % % % % % % % % % % % % % % % % % % %
% % % % % % % % % % % % % % % % % % % % % % % % % % % % % % % % %
% % % % % % % % % % % % % % % % % % % % % % % % % % % % % % % % %
% % % % % % % % % % % % % % % % % % % % % % % % % % % % % % % % %
% % % % % % % % % % % % % % % % % % % % % % % % % % % % % % % % %
% % % % % % % % % % % % % % % % % % % % % % % % % % % % % % % % %
% % % % % % % % % % % % % % % % % % % % % % % % % % % % % % % % %
% % % % % % % % % % % % % % % % % % % % % % % % % % % % % % % % %
% % % % % % % % % % % % % % % % % % % % % % % % % % % % % %

\chapter*{Chapter I} \ifaudio \marginpar{
\href{http://ia600205.us.archive.org/6/items/war_and_peace_11_0909/war_and_peace_11_01_tolstoy_64kb.mp3}{Audio}}
\fi

\initial{A}{bsolute} continuity of motion is not comprehensible to the human
mind.  Laws of motion of any kind become comprehensible to man
only when he examines arbitrarily selected elements of that
motion; but at the same time, a large proportion of human error
comes from the arbitrary division of continuous motion into
discontinuous elements. There is a well known, so-called sophism
of the ancients consisting in this, that Achilles could never
catch up with a tortoise he was following, in spite of the fact
that he traveled ten times as fast as the tortoise. By the time
Achilles has covered the distance that separated him from the
tortoise, the tortoise has covered one tenth of that distance
ahead of him: when Achilles has covered that tenth, the tortoise
has covered another one hundredth, and so on forever. This
problem seemed to the ancients insoluble. The absurd answer (that
Achilles could never overtake the tortoise) resulted from this:
that motion was arbitrarily divided into discontinuous elements,
whereas the motion both of Achilles and of the tortoise was
continuous.

By adopting smaller and smaller elements of motion we only
approach a solution of the problem, but never reach it. Only when
we have admitted the conception of the infinitely small, and the
resulting geometrical progression with a common ratio of one
tenth, and have found the sum of this progression to infinity, do
we reach a solution of the problem.

A modern branch of mathematics having achieved the art of dealing
with the infinitely small can now yield solutions in other more
complex problems of motion which used to appear insoluble.

This modern branch of mathematics, unknown to the ancients, when
dealing with problems of motion admits the conception of the
infinitely small, and so conforms to the chief condition of
motion (absolute continuity) and thereby corrects the inevitable
error which the human mind cannot avoid when it deals with
separate elements of motion instead of examining continuous
motion.

In seeking the laws of historical movement just the same thing
happens.  The movement of humanity, arising as it does from
innumerable arbitrary human wills, is continuous.

To understand the laws of this continuous movement is the aim of
history. But to arrive at these laws, resulting from the sum of
all those human wills, man's mind postulates arbitrary and
disconnected units. The first method of history is to take an
arbitrarily selected series of continuous events and examine it
apart from others, though there is and can be no beginning to any
event, for one event always flows uninterruptedly from another.

The second method is to consider the actions of some one man---a
king or a commander---as equivalent to the sum of many individual
wills; whereas the sum of individual wills is never expressed by
the activity of a single historic personage.

Historical science in its endeavor to draw nearer to truth
continually takes smaller and smaller units for examination. But
however small the units it takes, we feel that to take any unit
disconnected from others, or to assume a beginning of any
phenomenon, or to say that the will of many men is expressed by
the actions of any one historic personage, is in itself false.

It needs no critical exertion to reduce utterly to dust any
deductions drawn from history. It is merely necessary to select
some larger or smaller unit as the subject of observation---as
criticism has every right to do, seeing that whatever unit
history observes must always be arbitrarily selected.

Only by taking infinitesimally small units for observation (the
differential of history, that is, the individual tendencies of
men) and attaining to the art of integrating them (that is,
finding the sum of these infinitesimals) can we hope to arrive at
the laws of history.

The first fifteen years of the nineteenth century in Europe
present an extraordinary movement of millions of people. Men
leave their customary pursuits, hasten from one side of Europe to
the other, plunder and slaughter one another, triumph and are
plunged in despair, and for some years the whole course of life
is altered and presents an intensive movement which first
increases and then slackens. What was the cause of this movement,
by what laws was it governed? asks the mind of man.

The historians, replying to this question, lay before us the
sayings and doings of a few dozen men in a building in the city
of Paris, calling these sayings and doings \emph{the Revolution};
then they give a detailed biography of Napoleon and of certain
people favorable or hostile to him; tell of the influence some of
these people had on others, and say: that is why this movement
took place and those are its laws.

But the mind of man not only refuses to believe this explanation,
but plainly says that this method of explanation is fallacious,
because in it a weaker phenomenon is taken as the cause of a
stronger. The sum of human wills produced the Revolution and
Napoleon, and only the sum of those wills first tolerated and
then destroyed them.

``But every time there have been conquests there have been
conquerors; every time there has been a revolution in any state
there have been great men,'' says history. And, indeed, human
reason replies: every time conquerors appear there have been
wars, but this does not prove that the conquerors caused the wars
and that it is possible to find the laws of a war in the personal
activity of a single man. Whenever I look at my watch and its
hands point to ten, I hear the bells of the neighboring church;
but because the bells begin to ring when the hands of the clock
reach ten, I have no right to assume that the movement of the
bells is caused by the position of the hands of the watch.

Whenever I see the movement of a locomotive I hear the whistle
and see the valves opening and wheels turning; but I have no
right to conclude that the whistling and the turning of wheels
are the cause of the movement of the engine.

The peasants say that a cold wind blows in late spring because
the oaks are budding, and really every spring cold winds do blow
when the oak is budding. But though I do not know what causes the
cold winds to blow when the oak buds unfold, I cannot agree with
the peasants that the unfolding of the oak buds is the cause of
the cold wind, for the force of the wind is beyond the influence
of the buds. I see only a coincidence of occurrences such as
happens with all the phenomena of life, and I see that however
much and however carefully I observe the hands of the watch, and
the valves and wheels of the engine, and the oak, I shall not
discover the cause of the bells ringing, the engine moving, or of
the winds of spring. To that I must entirely change my point of
view and study the laws of the movement of steam, of the bells,
and of the wind. History must do the same. And attempts in this
direction have already been made.

To study the laws of history we must completely change the
subject of our observation, must leave aside kings, ministers,
and generals, and study the common, infinitesimally small
elements by which the masses are moved. No one can say in how far
it is possible for man to advance in this way toward an
understanding of the laws of history; but it is evident that only
along that path does the possibility of discovering the laws of
history lie, and that as yet not a millionth part as much mental
effort has been applied in this direction by historians as has
been devoted to describing the actions of various kings,
commanders, and ministers and propounding the historians' own
reflections concerning these actions.

% % % % % % % % % % % % % % % % % % % % % % % % % % % % % % % % %
% % % % % % % % % % % % % % % % % % % % % % % % % % % % % % % % %
% % % % % % % % % % % % % % % % % % % % % % % % % % % % % % % % %
% % % % % % % % % % % % % % % % % % % % % % % % % % % % % % % % %
% % % % % % % % % % % % % % % % % % % % % % % % % % % % % % % % %
% % % % % % % % % % % % % % % % % % % % % % % % % % % % % % % % %
% % % % % % % % % % % % % % % % % % % % % % % % % % % % % % % % %
% % % % % % % % % % % % % % % % % % % % % % % % % % % % % % % % %
% % % % % % % % % % % % % % % % % % % % % % % % % % % % % % % % %
% % % % % % % % % % % % % % % % % % % % % % % % % % % % % % % % %
% % % % % % % % % % % % % % % % % % % % % % % % % % % % % % % % %
% % % % % % % % % % % % % % % % % % % % % % % % % % % % % %

\chapter*{Chapter II} \ifaudio \marginpar{
\href{http://ia600205.us.archive.org/6/items/war_and_peace_11_0909/war_and_peace_11_02_tolstoy_64kb.mp3}{Audio}}
\fi

\initial{T}{he} forces of a dozen European nations burst into Russia. The
Russian army and people avoided a collision till Smolensk was
reached, and again from Smolensk to Borodino. The French army
pushed on to Moscow, its goal, its impetus ever increasing as it
neared its aim, just as the velocity of a falling body increases
as it approaches the earth. Behind it were seven hundred miles of
hunger-stricken, hostile country; ahead were a few dozen miles
separating it from its goal. Every soldier in Napoleon's army
felt this and the invasion moved on by its own momentum.

The more the Russian army retreated the more fiercely a spirit of
hatred of the enemy flared up, and while it retreated the army
increased and consolidated. At Borodino a collision took
place. Neither army was broken up, but the Russian army retreated
immediately after the collision as inevitably as a ball recoils
after colliding with another having a greater momentum, and with
equal inevitability the ball of invasion that had advanced with
such momentum rolled on for some distance, though the collision
had deprived it of all its force.

The Russians retreated eighty miles---to beyond Moscow---and the
French reached Moscow and there came to a standstill. For five
weeks after that there was not a single battle. The French did
not move. As a bleeding, mortally wounded animal licks its
wounds, they remained inert in Moscow for five weeks, and then
suddenly, with no fresh reason, fled back: they made a dash for
the Kaluga road, and (after a victory---for at Malo-Yaroslavets
the field of conflict again remained theirs) without undertaking
a single serious battle, they fled still more rapidly back to
Smolensk, beyond Smolensk, beyond the Berezina, beyond Vilna, and
farther still.

On the evening of the twenty-sixth of August, Kutuzov and the
whole Russian army were convinced that the battle of Borodino was
a victory.  Kutuzov reported so to the Emperor. He gave orders to
prepare for a fresh conflict to finish the enemy and did this not
to deceive anyone, but because he knew that the enemy was beaten,
as everyone who had taken part in the battle knew it.

But all that evening and next day reports came in one after
another of unheard-of losses, of the loss of half the army, and a
fresh battle proved physically impossible.

It was impossible to give battle before information had been
collected, the wounded gathered in, the supplies of ammunition
replenished, the slain reckoned up, new officers appointed to
replace those who had been killed, and before the men had had
food and sleep. And meanwhile, the very next morning after the
battle, the French army advanced of itself upon the Russians,
carried forward by the force of its own momentum now seemingly
increased in inverse proportion to the square of the distance
from its aim. Kutuzov's wish was to attack next day, and the
whole army desired to do so. But to make an attack the wish to do
so is not sufficient, there must also be a possibility of doing
it, and that possibility did not exist. It was impossible not to
retreat a day's march, and then in the same way it was impossible
not to retreat another and a third day's march, and at last, on
the first of September when the army drew near Moscow---despite
the strength of the feeling that had arisen in all ranks---the
force of circumstances compelled it to retire beyond Moscow. And
the troops retired one more, last, day's march, and abandoned
Moscow to the enemy.

For people accustomed to think that plans of campaign and battles
are made by generals---as any one of us sitting over a map in his
study may imagine how he would have arranged things in this or
that battle---the questions present themselves: Why did Kutuzov
during the retreat not do this or that? Why did he not take up a
position before reaching Fili?  Why did he not retire at once by
the Kaluga road, abandoning Moscow? and so on. People accustomed
to think in that way forget, or do not know, the inevitable
conditions which always limit the activities of any commander in
chief. The activity of a commander-in-chief does not at all
resemble the activity we imagine to ourselves when we sit at ease
in our studies examining some campaign on the map, with a certain
number of troops on this and that side in a certain known
locality, and begin our plans from some given moment. A
commander-in-chief is never dealing with the beginning of any
event---the position from which we always contemplate it. The
commander-in-chief is always in the midst of a series of shifting
events and so he never can at any moment consider the whole
import of an event that is occurring. Moment by moment the event
is imperceptibly shaping itself, and at every moment of this
continuous, uninterrupted shaping of events the
commander-in-chief is in the midst of a most complex play of
intrigues, worries, contingencies, authorities, projects,
counsels, threats, and deceptions and is continually obliged to
reply to innumerable questions addressed to him, which constantly
conflict with one another.

Learned military authorities quite seriously tell us that Kutuzov
should have moved his army to the Kaluga road long before
reaching Fili, and that somebody actually submitted such a
proposal to him. But a commander in chief, especially at a
difficult moment, has always before him not one proposal but
dozens simultaneously. And all these proposals, based on
strategics and tactics, contradict each other.

A commander-in-chief's business, it would seem, is simply to
choose one of these projects. But even that he cannot do. Events
and time do not wait. For instance, on the twenty-eighth it is
suggested to him to cross to the Kaluga road, but just then an
adjutant gallops up from Miloradovich asking whether he is to
engage the French or retire. An order must be given him at once,
that instant. And the order to retreat carries us past the turn
to the Kaluga road. And after the adjutant comes the commissary
general asking where the stores are to be taken, and the chief of
the hospitals asks where the wounded are to go, and a courier
from Petersburg brings a letter from the sovereign which does not
admit of the possibility of abandoning Moscow, and the
commander-in-chief's rival, the man who is undermining him (and
there are always not merely one but several such), presents a new
project diametrically opposed to that of turning to the Kaluga
road, and the commander-in-chief himself needs sleep and
refreshment to maintain his energy and a respectable general who
has been overlooked in the distribution of rewards comes to
complain, and the inhabitants of the district pray to be
defended, and an officer sent to inspect the locality comes in
and gives a report quite contrary to what was said by the officer
previously sent; and a spy, a prisoner, and a general who has
been on reconnaissance, all describe the position of the enemy's
army differently. People accustomed to misunderstand or to forget
these inevitable conditions of a commander-in-chief's actions
describe to us, for instance, the position of the army at Fili
and assume that the commander-in-chief could, on the first of
September, quite freely decide whether to abandon Moscow or
defend it; whereas, with the Russian army less than four miles
from Moscow, no such question existed. When had that question
been settled? At Drissa and at Smolensk and most palpably of all
on the twenty-fourth of August at Shevardino and on the
twenty-sixth at Borodino, and each day and hour and minute of the
retreat from Borodino to Fili.

% % % % % % % % % % % % % % % % % % % % % % % % % % % % % % % % %
% % % % % % % % % % % % % % % % % % % % % % % % % % % % % % % % %
% % % % % % % % % % % % % % % % % % % % % % % % % % % % % % % % %
% % % % % % % % % % % % % % % % % % % % % % % % % % % % % % % % %
% % % % % % % % % % % % % % % % % % % % % % % % % % % % % % % % %
% % % % % % % % % % % % % % % % % % % % % % % % % % % % % % % % %
% % % % % % % % % % % % % % % % % % % % % % % % % % % % % % % % %
% % % % % % % % % % % % % % % % % % % % % % % % % % % % % % % % %
% % % % % % % % % % % % % % % % % % % % % % % % % % % % % % % % %
% % % % % % % % % % % % % % % % % % % % % % % % % % % % % % % % %
% % % % % % % % % % % % % % % % % % % % % % % % % % % % % % % % %
% % % % % % % % % % % % % % % % % % % % % % % % % % % % % %

\chapter*{Chapter III} \ifaudio \marginpar{
\href{http://ia600205.us.archive.org/6/items/war_and_peace_11_0909/war_and_peace_11_03_tolstoy_64kb.mp3}{Audio}}
\fi

\initial{W}{hen} Ermolov, having been sent by Kutuzov to inspect the
position, told the field marshal that it was impossible to fight
there before Moscow and that they must retreat, Kutuzov looked at
him in silence.

``Give me your hand,'' said he and, turning it over so as to feel
the pulse, added: ``You are not well, my dear fellow. Think what
you are saying!''

Kutuzov could not yet admit the possibility of retreating beyond
Moscow without a battle.

On the Poklonny Hill, four miles from the Dorogomilov gate of
Moscow, Kutuzov got out of his carriage and sat down on a bench
by the roadside.  A great crowd of generals gathered round him,
and Count Rostopchin, who had come out from Moscow, joined
them. This brilliant company separated into several groups who
all discussed the advantages and disadvantages of the position,
the state of the army, the plans suggested, the situation of
Moscow, and military questions generally. Though they had not
been summoned for the purpose, and though it was not so called,
they all felt that this was really a council of war. The
conversations all dealt with public questions. If anyone gave or
asked for personal news, it was done in a whisper and they
immediately reverted to general matters. No jokes, or laughter,
or smiles even, were seen among all these men. They evidently all
made an effort to hold themselves at the height the situation
demanded. And all these groups, while talking among themselves,
tried to keep near the commander-in-chief (whose bench formed the
center of the gathering) and to speak so that he might overhear
them. The commander in chief listened to what was being said and
sometimes asked them to repeat their remarks, but did not himself
take part in the conversations or express any opinion. After
hearing what was being said by one or other of these groups he
generally turned away with an air of disappointment, as though
they were not speaking of anything he wished to hear. Some
discussed the position that had been chosen, criticizing not the
position itself so much as the mental capacity of those who had
chosen it. Others argued that a mistake had been made earlier and
that a battle should have been fought two days before. Others
again spoke of the battle of Salamanca, which was described by
Crosart, a newly arrived Frenchman in a Spanish uniform.  (This
Frenchman and one of the German princes serving with the Russian
army were discussing the siege of Saragossa and considering the
possibility of defending Moscow in a similar manner.) Count
Rostopchin was telling a fourth group that he was prepared to die
with the city train bands under the walls of the capital, but
that he still could not help regretting having been left in
ignorance of what was happening, and that had he known it sooner
things would have been different... A fifth group, displaying the
profundity of their strategic perceptions, discussed the
direction the troops would now have to take. A sixth group was
talking absolute nonsense. Kutuzov's expression grew more and
more preoccupied and gloomy. From all this talk he saw only one
thing: that to defend Moscow was a physical impossibility in the
full meaning of those words, that is to say, so utterly
impossible that if any senseless commander were to give orders to
fight, confusion would result but the battle would still not take
place. It would not take place because the commanders not merely
all recognized the position to be impossible, but in their
conversations were only discussing what would happen after its
inevitable abandonment. How could the commanders lead their
troops to a field of battle they considered impossible to hold?
The lower-grade officers and even the soldiers (who too reason)
also considered the position impossible and therefore could not
go to fight, fully convinced as they were of defeat. If Bennigsen
insisted on the position being defended and others still
discussed it, the question was no longer important in itself but
only as a pretext for disputes and intrigue.  This Kutuzov knew
well.

Bennigsen, who had chosen the position, warmly displayed his
Russian patriotism (Kutuzov could not listen to this without
wincing) by insisting that Moscow must be defended. His aim was
as clear as daylight to Kutuzov: if the defense failed, to throw
the blame on Kutuzov who had brought the army as far as the
Sparrow Hills without giving battle; if it succeeded, to claim
the success as his own; or if battle were not given, to clear
himself of the crime of abandoning Moscow. But this intrigue did
not now occupy the old man's mind. One terrible question absorbed
him and to that question he heard no reply from anyone. The
question for him now was: ``Have I really allowed Napoleon to
reach Moscow, and when did I do so? When was it decided? Can it
have been yesterday when I ordered Platov to retreat, or was it
the evening before, when I had a nap and told Bennigsen to issue
orders? Or was it earlier still?... When, when was this terrible
affair decided? Moscow must be abandoned. The army must retreat
and the order to do so must be given.'' To give that terrible
order seemed to him equivalent to resigning the command of the
army. And not only did he love power to which he was accustomed
(the honours awarded to Prince Prozorovski, under whom he had
served in Turkey, galled him), but he was convinced that he was
destined to save Russia and that that was why, against the
Emperor's wish and by the will of the people, he had been chosen
commander-in-chief. He was convinced that he alone could maintain
command of the army in these difficult circumstances, and that in
all the world he alone could encounter the invincible Napoleon
without fear, and he was horrified at the thought of the order he
had to issue. But something had to be decided, and these
conversations around him which were assuming too free a character
must be stopped.

He called the most important generals to him.

``My head, be it good or bad, must depend on itself,'' said he,
rising from the bench, and he rode to Fili where his carriages
were waiting.

% % % % % % % % % % % % % % % % % % % % % % % % % % % % % % % % %
% % % % % % % % % % % % % % % % % % % % % % % % % % % % % % % % %
% % % % % % % % % % % % % % % % % % % % % % % % % % % % % % % % %
% % % % % % % % % % % % % % % % % % % % % % % % % % % % % % % % %
% % % % % % % % % % % % % % % % % % % % % % % % % % % % % % % % %
% % % % % % % % % % % % % % % % % % % % % % % % % % % % % % % % %
% % % % % % % % % % % % % % % % % % % % % % % % % % % % % % % % %
% % % % % % % % % % % % % % % % % % % % % % % % % % % % % % % % %
% % % % % % % % % % % % % % % % % % % % % % % % % % % % % % % % %
% % % % % % % % % % % % % % % % % % % % % % % % % % % % % % % % %
% % % % % % % % % % % % % % % % % % % % % % % % % % % % % % % % %
% % % % % % % % % % % % % % % % % % % % % % % % % % % % % %

\chapter*{Chapter IV} \ifaudio \marginpar{
\href{http://ia600205.us.archive.org/6/items/war_and_peace_11_0909/war_and_peace_11_04_tolstoy_64kb.mp3}{Audio}}
\fi

\initial{T}{he} Council of War began to assemble at two in the afternoon in
the better and roomier part of Andrew Savostyanov's hut. The men,
women, and children of the large peasant family crowded into the
back room across the passage. Only Malasha, Andrew's six-year-old
granddaughter whom his Serene Highness had petted and to whom he
had given a lump of sugar while drinking his tea, remained on the
top of the brick oven in the larger room. Malasha looked down
from the oven with shy delight at the faces, uniforms, and
decorations of the generals, who one after another came into the
room and sat down on the broad benches in the corner under the
icons. \emph{Granddad} himself, as Malasha in her own mind called
Kutuzov, sat apart in a dark corner behind the oven. He sat, sunk
deep in a folding armchair, and continually cleared his throat
and pulled at the collar of his coat which, though it was
unbuttoned, still seemed to pinch his neck. Those who entered
went up one by one to the field marshal; he pressed the hands of
some and nodded to others. His adjutant Kaysarov was about to
draw back the curtain of the window facing Kutuzov, but the
latter moved his hand angrily and Kaysarov understood that his
Serene Highness did not wish his face to be seen.

Round the peasant's deal table, on which lay maps, plans,
pencils, and papers, so many people gathered that the orderlies
brought in another bench and put it beside the table. Ermolov,
Kaysarov, and Toll, who had just arrived, sat down on this
bench. In the foremost place, immediately under the icons, sat
Barclay de Tolly, his high forehead merging into his bald
crown. He had a St. George's Cross round his neck and looked pale
and ill. He had been feverish for two days and was now shivering
and in pain. Beside him sat Uvarov, who with rapid gesticulations
was giving him some information, speaking in low tones as they
all did.  Chubby little Dokhturov was listening attentively with
eyebrows raised and arms folded on his stomach. On the other side
sat Count Ostermann-Tolstoy, seemingly absorbed in his own
thoughts. His broad head with its bold features and glittering
eyes was resting on his hand. Raevski, twitching forward the
black hair on his temples as was his habit, glanced now at
Kutuzov and now at the door with a look of impatience.
Konovnitsyn's firm, handsome, and kindly face was lit up by a
tender, sly smile. His glance met Malasha's, and the expression
of his eyes caused the little girl to smile.

They were all waiting for Bennigsen, who on the pretext of
inspecting the position was finishing his savory dinner. They
waited for him from four till six o'clock and did not begin their
deliberations all that time but talked in low tones of other
matters.

Only when Bennigsen had entered the hut did Kutuzov leave his
corner and draw toward the table, but not near enough for the
candles that had been placed there to light up his face.

Bennigsen opened the council with the question: ``Are we to
abandon Russia's ancient and sacred capital without a struggle,
or are we to defend it?'' A prolonged and general silence
followed. There was a frown on every face and only Kutuzov's
angry grunts and occasional cough broke the silence. All eyes
were gazing at him. Malasha too looked at \emph{Granddad}. She
was nearest to him and saw how his face puckered; he seemed about
to cry, but this did not last long.

``Russia's ancient and sacred capital!'' he suddenly said,
repeating Bennigsen's words in an angry voice and thereby drawing
attention to the false note in them. ``Allow me to tell you, your
excellency, that that question has no meaning for a Russian.''
(He lurched his heavy body forward.) ``Such a question cannot be
put; it is senseless! The question I have asked these gentlemen
to meet to discuss is a military one. The question is that of
saving Russia. Is it better to give up Moscow without a battle,
or by accepting battle to risk losing the army as well as Moscow?
That is the question on which I want your opinion,'' and he sank
back in his chair.

The discussion began. Bennigsen did not yet consider his game
lost.  Admitting the view of Barclay and others that a defensive
battle at Fili was impossible, but imbued with Russian patriotism
and the love of Moscow, he proposed to move troops from the right
to the left flank during the night and attack the French right
flank the following day.  Opinions were divided, and arguments
were advanced for and against that project. Ermolov, Dokhturov,
and Raevski agreed with Bennigsen. Whether feeling it necessary
to make a sacrifice before abandoning the capital or guided by
other, personal considerations, these generals seemed not to
understand that this council could not alter the inevitable
course of events and that Moscow was in effect already
abandoned. The other generals, however, understood it and,
leaving aside the question of Moscow, spoke of the direction the
army should take in its retreat.  Malasha, who kept her eyes
fixed on what was going on before her, understood the meaning of
the council differently. It seemed to her that it was only a
personal struggle between \emph{Granddad} and \emph{Long-coat} as
she termed Bennigsen. She saw that they grew spiteful when they
spoke to one another, and in her heart she sided with
\emph{Granddad}. In the midst of the conversation she noticed
\emph{Granddad} give Bennigsen a quick, subtle glance, and then
to her joys she saw that \emph{Granddad} said something to
\emph{Long-coat} which settled him. Bennigsen suddenly reddened
and paced angrily up and down the room. What so affected him was
Kutuzov's calm and quiet comment on the advantage or disadvantage
of Bennigsen's proposal to move troops by night from the right to
the left flank to attack the French right wing.

``Gentlemen,'' said Kutuzov, ``I cannot approve of the count's
plan. Moving troops in close proximity to an enemy is always
dangerous, and military history supports that view. For
instance...'' Kutuzov seemed to reflect, searching for an
example, then with a clear, naive look at Bennigsen he added:
``Oh yes; take the battle of Friedland, which I think the count
well remembers, and which was... not fully successful, only
because our troops were rearranged too near the enemy...''

There followed a momentary pause, which seemed very long to them
all.

The discussion recommenced, but pauses frequently occurred and
they all felt that there was no more to be said.

During one of these pauses Kutuzov heaved a deep sigh as if
preparing to speak. They all looked at him.

``Well, gentlemen, I see that it is I who will have to pay for
the broken crockery,'' said he, and rising slowly he moved to the
table. ``Gentlemen, I have heard your views. Some of you will not
agree with me. But I,'' he paused, ``by the authority entrusted
to me by my Sovereign and country, order a retreat.''

After that the generals began to disperse with the solemnity and
circumspect silence of people who are leaving, after a funeral.

Some of the generals, in low tones and in a strain very different
from the way they had spoken during the council, communicated
something to their commander-in-chief.

Malasha, who had long been expected for supper, climbed carefully
backwards down from the oven, her bare little feet catching at
its projections, and slipping between the legs of the generals
she darted out of the room.

When he had dismissed the generals Kutuzov sat a long time with
his elbows on the table, thinking always of the same terrible
question: ``When, when did the abandonment of Moscow become
inevitable? When was that done which settled the matter? And who
was to blame for it?''

``I did not expect this,'' said he to his adjutant Schneider when
the latter came in late that night. ``I did not expect this! I
did not think this would happen.''

``You should take some rest, your Serene Highness,'' replied
Schneider.

``But no! They shall eat horseflesh yet, like the Turks!''
exclaimed Kutuzov without replying, striking the table with his
podgy fist. ``They shall too, if only...''

% % % % % % % % % % % % % % % % % % % % % % % % % % % % % % % % %
% % % % % % % % % % % % % % % % % % % % % % % % % % % % % % % % %
% % % % % % % % % % % % % % % % % % % % % % % % % % % % % % % % %
% % % % % % % % % % % % % % % % % % % % % % % % % % % % % % % % %
% % % % % % % % % % % % % % % % % % % % % % % % % % % % % % % % %
% % % % % % % % % % % % % % % % % % % % % % % % % % % % % % % % %
% % % % % % % % % % % % % % % % % % % % % % % % % % % % % % % % %
% % % % % % % % % % % % % % % % % % % % % % % % % % % % % % % % %
% % % % % % % % % % % % % % % % % % % % % % % % % % % % % % % % %
% % % % % % % % % % % % % % % % % % % % % % % % % % % % % % % % %
% % % % % % % % % % % % % % % % % % % % % % % % % % % % % % % % %
% % % % % % % % % % % % % % % % % % % % % % % % % % % % % %

\chapter*{Chapter V} \ifaudio \marginpar{
\href{http://ia600205.us.archive.org/6/items/war_and_peace_11_0909/war_and_peace_11_05_tolstoy_64kb.mp3}{Audio}}
\fi

\initial{A}{t} that very time, in circumstances even more important than
retreating without a battle, namely the evacuation and burning of
Moscow, Rostopchin, who is usually represented as being the
instigator of that event, acted in an altogether different manner
from Kutuzov.

After the battle of Borodino the abandonment and burning of
Moscow was as inevitable as the retreat of the army beyond Moscow
without fighting.

Every Russian might have predicted it, not by reasoning but by
the feeling implanted in each of us and in our fathers.

The same thing that took place in Moscow had happened in all the
towns and villages on Russian soil beginning with Smolensk,
without the participation of Count Rostopchin and his
broadsheets. The people awaited the enemy unconcernedly, did not
riot or become excited or tear anyone to pieces, but faced its
fate, feeling within it the strength to find what it should do at
that most difficult moment. And as soon as the enemy drew near
the wealthy classes went away abandoning their property, while
the poorer remained and burned and destroyed what was left.

The consciousness that this would be so and would always be so
was and is present in the heart of every Russian. And a
consciousness of this, and a foreboding that Moscow would be
taken, was present in Russian Moscow society in 1812. Those who
had quitted Moscow already in July and at the beginning of August
showed that they expected this. Those who went away, taking what
they could and abandoning their houses and half their belongings,
did so from the latent patriotism which expresses itself not by
phrases or by giving one's children to save the fatherland and
similar unnatural exploits, but unobtrusively, simply,
organically, and therefore in the way that always produces the
most powerful results.

``It is disgraceful to run away from danger; only cowards are
running away from Moscow,'' they were told. In his broadsheets
Rostopchin impressed on them that to leave Moscow was
shameful. They were ashamed to be called cowards, ashamed to
leave, but still they left, knowing it had to be done. Why did
they go? It is impossible to suppose that Rostopchin had scared
them by his accounts of horrors Napoleon had committed in
conquered countries. The first people to go away were the rich
educated people who knew quite well that Vienna and Berlin had
remained intact and that during Napoleon's occupation the
inhabitants had spent their time pleasantly in the company of the
charming Frenchmen whom the Russians, and especially the Russian
ladies, then liked so much.

They went away because for Russians there could be no question as
to whether things would go well or ill under French rule in
Moscow. It was out of the question to be under French rule, it
would be the worst thing that could happen. They went away even
before the battle of Borodino and still more rapidly after it,
despite Rostopchin's calls to defend Moscow or the announcement
of his intention to take the wonder-working icon of the Iberian
Mother of God and go to fight, or of the balloons that were to
destroy the French, and despite all the nonsense Rostopchin wrote
in his broadsheets. They knew that it was for the army to fight,
and that if it could not succeed it would not do to take young
ladies and house serfs to the Three Hills quarter of Moscow to
fight Napoleon, and that they must go away, sorry as they were to
abandon their property to destruction. They went away without
thinking of the tremendous significance of that immense and
wealthy city being given over to destruction, for a great city
with wooden buildings was certain when abandoned by its
inhabitants to be burned. They went away each on his own account,
and yet it was only in consequence of their going away that the
momentous event was accomplished that will always remain the
greatest glory of the Russian people. The lady who, afraid of
being stopped by Count Rostopchin's orders, had already in June
moved with her Negroes and her women jesters from Moscow to her
Saratov estate, with a vague consciousness that she was not
Bonaparte's servant, was really, simply, and truly carrying out
the great work which saved Russia. But Count Rostopchin, who now
taunted those who left Moscow and now had the government offices
removed; now distributed quite useless weapons to the drunken
rabble; now had processions displaying the icons, and now forbade
Father Augustin to remove icons or the relics of saints; now
seized all the private carts in Moscow and on one hundred and
thirty-six of them removed the balloon that was being constructed
by Leppich; now hinted that he would burn Moscow and related how
he had set fire to his own house; now wrote a proclamation to the
French solemnly upbraiding them for having destroyed his
Orphanage; now claimed the glory of having hinted that he would
burn Moscow and now repudiated the deed; now ordered the people
to catch all spies and bring them to him, and now reproached them
for doing so; now expelled all the French residents from Moscow,
and now allowed Madame Aubert-Chalme (the center of the whole
French colony in Moscow) to remain, but ordered the venerable old
postmaster Klyucharev to be arrested and exiled for no particular
offense; now assembled the people at the Three Hills to fight the
French and now, to get rid of them, handed over to them a man to
be killed and himself drove away by a back gate; now declared
that he would not survive the fall of Moscow, and now wrote
French verses in albums concerning his share in the affair---this
man did not understand the meaning of what was happening but
merely wanted to do something himself that would astonish people,
to perform some patriotically heroic feat; and like a child he
made sport of the momentous, and unavoidable event---the
abandonment and burning of Moscow---and tried with his puny hand
now to speed and now to stay the enormous, popular tide that bore
him along with it.

% % % % % % % % % % % % % % % % % % % % % % % % % % % % % % % % %
% % % % % % % % % % % % % % % % % % % % % % % % % % % % % % % % %
% % % % % % % % % % % % % % % % % % % % % % % % % % % % % % % % %
% % % % % % % % % % % % % % % % % % % % % % % % % % % % % % % % %
% % % % % % % % % % % % % % % % % % % % % % % % % % % % % % % % %
% % % % % % % % % % % % % % % % % % % % % % % % % % % % % % % % %
% % % % % % % % % % % % % % % % % % % % % % % % % % % % % % % % %
% % % % % % % % % % % % % % % % % % % % % % % % % % % % % % % % %
% % % % % % % % % % % % % % % % % % % % % % % % % % % % % % % % %
% % % % % % % % % % % % % % % % % % % % % % % % % % % % % % % % %
% % % % % % % % % % % % % % % % % % % % % % % % % % % % % % % % %
% % % % % % % % % % % % % % % % % % % % % % % % % % % % % %

\chapter*{Chapter VI} \ifaudio \marginpar{
\href{http://ia600205.us.archive.org/6/items/war_and_peace_11_0909/war_and_peace_11_06_tolstoy_64kb.mp3}{Audio}}
\fi

\initial{H}{elene}, having returned with the court from Vilna to Petersburg,
found herself in a difficult position.

In Petersburg she had enjoyed the special protection of a grandee
who occupied one of the highest posts in the Empire. In Vilna she
had formed an intimacy with a young foreign prince. When she
returned to Petersburg both the magnate and the prince were
there, and both claimed their rights. Helene was faced by a new
problem---how to preserve her intimacy with both without
offending either.

What would have seemed difficult or even impossible to another
woman did not cause the least embarrassment to Countess
Bezukhova, who evidently deserved her reputation of being a very
clever woman. Had she attempted concealment, or tried to
extricate herself from her awkward position by cunning, she would
have spoiled her case by acknowledging herself guilty. But
Helene, like a really great man who can do whatever he pleases,
at once assumed her own position to be correct, as she sincerely
believed it to be, and that everyone else was to blame.

The first time the young foreigner allowed himself to reproach
her, she lifted her beautiful head and, half turning to him, said
firmly: ``That's just like a man---selfish and cruel! I expected
nothing else. A woman sacrifices herself for you, she suffers,
and this is her reward! What right have you, monseigneur, to
demand an account of my attachments and friendships? He is a man
who has been more than a father to me!'' The prince was about to
say something, but Helene interrupted him.

``Well, yes,'' said she, ``it may be that he has other sentiments
for me than those of a father, but that is not a reason for me to
shut my door on him. I am not a man, that I should repay kindness
with ingratitude!  Know, monseigneur, that in all that relates to
my intimate feelings I render account only to God and to my
conscience,'' she concluded, laying her hand on her beautiful,
fully expanded bosom and looking up to heaven.

``But for heaven's sake listen to me!''

``Marry me, and I will be your slave!''

``But that's impossible.''

``You won't deign to demean yourself by marrying me, you...''
said Helene, beginning to cry.

The prince tried to comfort her, but Helene, as if quite
distraught, said through her tears that there was nothing to
prevent her marrying, that there were precedents (there were up
to that time very few, but she mentioned Napoleon and some other
exalted personages), that she had never been her husband's wife,
and that she had been sacrificed.

``But the law, religion...'' said the prince, already yielding.

``The law, religion... What have they been invented for if they
can't arrange that?'' said Helene.

The prince was surprised that so simple an idea had not occurred
to him, and he applied for advice to the holy brethren of the
Society of Jesus, with whom he was on intimate terms.

A few days later at one of those enchanting fetes which Helene
gave at her country house on the Stone Island, the charming
Monsieur de Jobert, a man no longer young, with snow white hair
and brilliant black eyes, a Jesuit a robe courte\footnote{Lay
member of the Society of Jesus.}  was presented to her, and in
the garden by the light of the illuminations and to the sound of
music talked to her for a long time of the love of God, of
Christ, of the Sacred Heart, and of the consolations the one true
Catholic religion affords in this world and the next. Helene was
touched, and more than once tears rose to her eyes and to those
of Monsieur de Jobert and their voices trembled. A dance, for
which her partner came to seek her, put an end to her discourse
with her future directeur de conscience, but the next evening
Monsieur de Jobert came to see Helene when she was alone, and
after that often came again.

One day he took the countess to a Roman Catholic church, where
she knelt down before the altar to which she was led. The
enchanting, middle-aged Frenchman laid his hands on her head and,
as she herself afterward described it, she felt something like a
fresh breeze wafted into her soul. It was explained to her that
this was la grace.

After that a long-frocked abbe was brought to her. She confessed
to him, and he absolved her from her sins. Next day she received
a box containing the Sacred Host, which was left at her house for
her to partake of. A few days later Helene learned with pleasure
that she had now been admitted to the true Catholic Church and
that in a few days the Pope himself would hear of her and would
send her a certain document.

All that was done around her and to her at this time, all the
attention devoted to her by so many clever men and expressed in
such pleasant, refined ways, and the state of dove-like purity
she was now in (she wore only white dresses and white ribbons all
that time) gave her pleasure, but her pleasure did not cause her
for a moment to forget her aim. And as it always happens in
contests of cunning that a stupid person gets the better of
cleverer ones, Helene---having realized that the main object of
all these words and all this trouble was, after converting her to
Catholicism, to obtain money from her for Jesuit institutions (as
to which she received indications)-before parting with her money
insisted that the various operations necessary to free her from
her husband should be performed. In her view the aim of every
religion was merely to preserve certain proprieties while
affording satisfaction to human desires. And with this aim, in
one of her talks with her Father Confessor, she insisted on an
answer to the question, in how far was she bound by her marriage?

They were sitting in the twilight by a window in the drawing
room. The scent of flowers came in at the window. Helene was
wearing a white dress, transparent over her shoulders and
bosom. The abbe, a well-fed man with a plump, clean-shaven chin,
a pleasant firm mouth, and white hands meekly folded on his
knees, sat close to Helene and, with a subtle smile on his lips
and a peaceful look of delight at her beauty, occasionally
glanced at her face as he explained his opinion on the
subject. Helene with an uneasy smile looked at his curly hair and
his plump, clean-shaven, blackish cheeks and every moment
expected the conversation to take a fresh turn. But the abbe,
though he evidently enjoyed the beauty of his companion, was
absorbed in his mastery of the matter.

The course of the Father Confessor's arguments ran as follows:
``Ignorant of the import of what you were undertaking, you made a
vow of conjugal fidelity to a man who on his part, by entering
the married state without faith in the religious significance of
marriage, committed an act of sacrilege. That marriage lacked the
dual significance it should have had. Yet in spite of this your
vow was binding. You swerved from it.  What did you commit by so
acting? A venial, or a mortal, sin? A venial sin, for you acted
without evil intention. If now you married again with the object
of bearing children, your sin might be forgiven. But the question
is again a twofold one: firstly...''

But suddenly Helene, who was getting bored, said with one of her
bewitching smiles: ``But I think that having espoused the true
religion I cannot be bound by what a false religion laid upon
me.''

The director of her conscience was astounded at having the case
presented to him thus with the simplicity of Columbus' egg. He
was delighted at the unexpected rapidity of his pupil's progress,
but could not abandon the edifice of argument he had laboriously
constructed.

``Let us understand one another, Countess,'' said he with a
smile, and began refuting his spiritual daughter's arguments.

% % % % % % % % % % % % % % % % % % % % % % % % % % % % % % % % %
% % % % % % % % % % % % % % % % % % % % % % % % % % % % % % % % %
% % % % % % % % % % % % % % % % % % % % % % % % % % % % % % % % %
% % % % % % % % % % % % % % % % % % % % % % % % % % % % % % % % %
% % % % % % % % % % % % % % % % % % % % % % % % % % % % % % % % %
% % % % % % % % % % % % % % % % % % % % % % % % % % % % % % % % %
% % % % % % % % % % % % % % % % % % % % % % % % % % % % % % % % %
% % % % % % % % % % % % % % % % % % % % % % % % % % % % % % % % %
% % % % % % % % % % % % % % % % % % % % % % % % % % % % % % % % %
% % % % % % % % % % % % % % % % % % % % % % % % % % % % % % % % %
% % % % % % % % % % % % % % % % % % % % % % % % % % % % % % % % %
% % % % % % % % % % % % % % % % % % % % % % % % % % % % % %

\chapter*{Chapter VII} \ifaudio \marginpar{
\href{http://ia600205.us.archive.org/6/items/war_and_peace_11_0909/war_and_peace_11_07_tolstoy_64kb.mp3}{Audio}}
\fi

\initial{H}{elene} understood that the question was very simple and easy from
the ecclesiastical point of view, and that her directors were
making difficulties only because they were apprehensive as to how
the matter would be regarded by the secular authorities.

So she decided that it was necessary to prepare the opinion of
society.  She provoked the jealousy of the elderly magnate and
told him what she had told her other suitor; that is, she put the
matter so that the only way for him to obtain a right over her
was to marry her. The elderly magnate was at first as much taken
aback by this suggestion of marriage with a woman whose husband
was alive, as the younger man had been, but Helene's
imperturbable conviction that it was as simple and natural as
marrying a maiden had its effect on him too. Had Helene herself
shown the least sign of hesitation, shame, or secrecy, her cause
would certainly have been lost; but not only did she show no
signs of secrecy or shame, on the contrary, with good-natured
naivete she told her intimate friends (and these were all
Petersburg) that both the prince and the magnate had proposed to
her and that she loved both and was afraid of grieving either.

A rumor immediately spread in Petersburg, not that Helene wanted
to be divorced from her husband (had such a report spread many
would have opposed so illegal an intention) but simply that the
unfortunate and interesting Helene was in doubt which of the two
men she should marry.  The question was no longer whether this
was possible, but only which was the better match and how the
matter would be regarded at court. There were, it is true, some
rigid individuals unable to rise to the height of such a
question, who saw in the project a desecration of the sacrament
of marriage, but there were not many such and they remained
silent, while the majority were interested in Helene's good
fortune and in the question which match would be the more
advantageous. Whether it was right or wrong to remarry while one
had a husband living they did not discuss, for that question had
evidently been settled by people \emph{wiser than you or me}, as
they said, and to doubt the correctness of that decision would be
to risk exposing one's stupidity and incapacity to live in
society.

Only Marya Dmitrievna Akhrosimova, who had come to Petersburg
that summer to see one of her sons, allowed herself plainly to
express an opinion contrary to the general one. Meeting Helene at
a ball she stopped her in the middle of the room and, amid
general silence, said in her gruff voice: ``So wives of living
men have started marrying again!  Perhaps you think you have
invented a novelty? You have been forestalled, my dear! It was
thought of long ago. It is done in all the brothels,'' and with
these words Marya Dmitrievna, turning up her wide sleeves with
her usual threatening gesture and glancing sternly round, moved
across the room.

Though people were afraid of Marya Dmitrievna she was regarded in
Petersburg as a buffoon, and so of what she had said they only
noticed, and repeated in a whisper, the one coarse word she had
used, supposing the whole sting of her remark to lie in that
word.

Prince Vasili, who of late very often forgot what he had said and
repeated one and the same thing a hundred times, remarked to his
daughter whenever he chanced to see her:

``Helene, I have a word to say to you,'' and he would lead her
aside, drawing her hand downward. ``I have heard of certain
projects concerning... you know. Well my dear child, you know how
your father's heart rejoices to know that you... You have
suffered so much... But, my dear child, consult only your own
heart. That is all I have to say,'' and concealing his unvarying
emotion he would press his cheek against his daughter's and move
away.

Bilibin, who had not lost his reputation of an exceedingly clever
man, and who was one of the disinterested friends so brilliant a
woman as Helene always has---men friends who can never change
into lovers---once gave her his view of the matter at a small and
intimate gathering.

``Listen, Bilibin,'' said Helene (she always called friends of
that sort by their surnames), and she touched his coat sleeve
with her white, beringed fingers. ``Tell me, as you would a
sister, what I ought to do.  Which of the two?''

Bilibin wrinkled up the skin over his eyebrows and pondered, with
a smile on his lips.

``You are not taking me unawares, you know,'' said he. ``As a
true friend, I have thought and thought again about your
affair. You see, if you marry the prince''---he meant the younger
man---and he crooked one finger, ``you forever lose the chance of
marrying the other, and you will displease the court
besides. (You know there is some kind of connection.) But if you
marry the old count you will make his last days happy, and as
widow of the Grand... the prince would no longer be making a
mesalliance by marrying you,'' and Bilibin smoothed out his
forehead.

``That's a true friend!'' said Helene beaming, and again touching
Bilibin's sleeve. ``But I love them, you know, and don't want to
distress either of them. I would give my life for the happiness
of them both.''

Bilibin shrugged his shoulders, as much as to say that not even
he could help in that difficulty.

``Une maitresse-femme!\footnote{A masterly woman.} That's what is
called putting things squarely.  She would like to be married to
all three at the same time,'' thought he.

``But tell me, how will your husband look at the matter?''
Bilibin asked, his reputation being so well established that he
did not fear to ask so naive a question. ``Will he agree?''

``Oh, he loves me so!'' said Helene, who for some reason imagined
that Pierre too loved her. ``He will do anything for me.''

Bilibin puckered his skin in preparation for something witty.

``Even divorce you?'' said he.

Helene laughed.

Among those who ventured to doubt the justifiability of the
proposed marriage was Helene's mother, Princess Kuragina. She was
continually tormented by jealousy of her daughter, and now that
jealousy concerned a subject near to her own heart, she could not
reconcile herself to the idea. She consulted a Russian priest as
to the possibility of divorce and remarriage during a husband's
lifetime, and the priest told her that it was impossible, and to
her delight showed her a text in the Gospel which (as it seemed
to him) plainly forbids remarriage while the husband is alive.

Armed with these arguments, which appeared to her unanswerable,
she drove to her daughter's early one morning so as to find her
alone.

Having listened to her mother's objections, Helene smiled blandly
and ironically.

``But it says plainly: 'Whosoever shall marry her that is
divorced...'''  said the old princess.

``Ah, Maman, ne dites pas de betises. Vous ne comprenez
rien. Dans ma position j'ai des devoirs,''\footnote{``Oh, Mamma,
don't talk nonsense!  You don't understand anything. In my
position I have obligations.''} said Helene changing from
Russian, in which language she always felt that her case did not
sound quite clear, into French which suited it better.

``But, my dear...''

``Oh, Mamma, how is it you don't understand that the Holy Father,
who has the right to grant dispensations...''

Just then the lady companion who lived with Helene came in to
announce that His Highness was in the ballroom and wished to see
her.

``Non, dites-lui que je ne veux pas le voir, que je suis furieuse
contre lui, parce qu'il m'a manque parole.''\footnote{``No, tell
him I don't wish to see him, I am furious with him for not
keeping his word to me.''}

``Comtesse, a tout peche misericorde,''\footnote{``Countess,
there is mercy for every sin.''} said a fair-haired young man
with a long face and nose, as he entered the room.

The old princess rose respectfully and curtsied. The young man
who had entered took no notice of her. The princess nodded to her
daughter and sidled out of the room.

``Yes, she is right,'' thought the old princess, all her
convictions dissipated by the appearance of His Highness. ``She
is right, but how is it that we in our irrecoverable youth did
not know it? Yet it is so simple,'' she thought as she got into
her carriage.

By the beginning of August Helene's affairs were clearly defined
and she wrote a letter to her husband---who, as she imagined,
loved her very much---informing him of her intention to marry
N.N. and of her having embraced the one true faith, and asking
him to carry out all the formalities necessary for a divorce,
which would be explained to him by the bearer of the letter.

And so I pray God to have you, my friend, in His holy and
powerful keeping---Your friend Helene.

This letter was brought to Pierre's house when he was on the
field of Borodino.

% % % % % % % % % % % % % % % % % % % % % % % % % % % % % % % % %
% % % % % % % % % % % % % % % % % % % % % % % % % % % % % % % % %
% % % % % % % % % % % % % % % % % % % % % % % % % % % % % % % % %
% % % % % % % % % % % % % % % % % % % % % % % % % % % % % % % % %
% % % % % % % % % % % % % % % % % % % % % % % % % % % % % % % % %
% % % % % % % % % % % % % % % % % % % % % % % % % % % % % % % % %
% % % % % % % % % % % % % % % % % % % % % % % % % % % % % % % % %
% % % % % % % % % % % % % % % % % % % % % % % % % % % % % % % % %
% % % % % % % % % % % % % % % % % % % % % % % % % % % % % % % % %
% % % % % % % % % % % % % % % % % % % % % % % % % % % % % % % % %
% % % % % % % % % % % % % % % % % % % % % % % % % % % % % % % % %
% % % % % % % % % % % % % % % % % % % % % % % % % % % % % %

\chapter*{Chapter VIII} \ifaudio \marginpar{
\href{http://ia600205.us.archive.org/6/items/war_and_peace_11_0909/war_and_peace_11_08_tolstoy_64kb.mp3}{Audio}}
\fi

\initial{T}{oward} the end of the battle of Borodino, Pierre, having run down
from Raevski's battery a second time, made his way through a
gully to Knyazkovo with a crowd of soldiers, reached the dressing
station, and seeing blood and hearing cries and groans hurried
on, still entangled in the crowds of soldiers.

The one thing he now desired with his whole soul was to get away
quickly from the terrible sensations amid which he had lived that
day and return to ordinary conditions of life and sleep quietly
in a room in his own bed. He felt that only in the ordinary
conditions of life would he be able to understand himself and all
he had seen and felt. But such ordinary conditions of life were
nowhere to be found.

Though shells and bullets did not whistle over the road along
which he was going, still on all sides there was what there had
been on the field of battle. There were still the same suffering,
exhausted, and sometimes strangely indifferent faces, the same
blood, the same soldiers' overcoats, the same sounds of firing
which, though distant now, still aroused terror, and besides this
there were the foul air and the dust.

Having gone a couple of miles along the Mozhaysk road, Pierre sat
down by the roadside.

Dusk had fallen, and the roar of guns died away. Pierre lay
leaning on his elbow for a long time, gazing at the shadows that
moved past him in the darkness. He was continually imagining that
a cannon ball was flying toward him with a terrific whizz, and
then he shuddered and sat up. He had no idea how long he had been
there. In the middle of the night three soldiers, having brought
some firewood, settled down near him and began lighting a fire.

The soldiers, who threw sidelong glances at Pierre, got the fire
to burn and placed an iron pot on it into which they broke some
dried bread and put a little dripping. The pleasant odor of
greasy viands mingled with the smell of smoke. Pierre sat up and
sighed. The three soldiers were eating and talking among
themselves, taking no notice of him.

``And who may you be?'' one of them suddenly asked Pierre,
evidently meaning what Pierre himself had in mind, namely: ``If
you want to eat we'll give you some food, only let us know
whether you are an honest man.''

``I, I...'' said Pierre, feeling it necessary to minimize his
social position as much as possible so as to be nearer to the
soldiers and better understood by them. ``By rights I am a
militia officer, but my men are not here. I came to the battle
and have lost them.''

``There now!'' said one of the soldiers.

Another shook his head.

``Would you like a little mash?'' the first soldier asked, and
handed Pierre a wooden spoon after licking it clean.

Pierre sat down by the fire and began eating the mash, as they
called the food in the cauldron, and he thought it more delicious
than any food he had ever tasted. As he sat bending greedily over
it, helping himself to large spoonfuls and chewing one after
another, his face was lit up by the fire and the soldiers looked
at him in silence.

``Where have you to go to? Tell us!'' said one of them.

``To Mozhaysk.''

``You're a gentleman, aren't you?''

``Yes.''

``And what's your name?''

``Peter Kirilych.''

``Well then, Peter Kirilych, come along with us, we'll take you
there.''

In the total darkness the soldiers walked with Pierre to
Mozhaysk.

By the time they got near Mozhaysk and began ascending the steep
hill into the town, the cocks were already crowing. Pierre went
on with the soldiers, quite forgetting that his inn was at the
bottom of the hill and that he had already passed it. He would
not soon have remembered this, such was his state of
forgetfulness, had he not halfway up the hill stumbled upon his
groom, who had been to look for him in the town and was returning
to the inn. The groom recognized Pierre in the darkness by his
white hat.

``Your excellency!'' he said. ``Why, we were beginning to
despair! How is it you are on foot? And where are you going,
please?''

``Oh, yes!'' said Pierre.

The soldiers stopped.

``So you've found your folk?'' said one of them. ``Well, good-by,
Peter Kirilych---isn't it?''

``Good-bye, Peter Kirilych!'' Pierre heard the other voices
repeat.

``Good-bye!'' he said and turned with his groom toward the inn.

``I ought to give them something!'' he thought, and felt in his
pocket.  ``No, better not!'' said another, inner voice.

There was not a room to be had at the inn, they were all
occupied.  Pierre went out into the yard and, covering himself up
head and all, lay down in his carriage.

% % % % % % % % % % % % % % % % % % % % % % % % % % % % % % % % %
% % % % % % % % % % % % % % % % % % % % % % % % % % % % % % % % %
% % % % % % % % % % % % % % % % % % % % % % % % % % % % % % % % %
% % % % % % % % % % % % % % % % % % % % % % % % % % % % % % % % %
% % % % % % % % % % % % % % % % % % % % % % % % % % % % % % % % %
% % % % % % % % % % % % % % % % % % % % % % % % % % % % % % % % %
% % % % % % % % % % % % % % % % % % % % % % % % % % % % % % % % %
% % % % % % % % % % % % % % % % % % % % % % % % % % % % % % % % %
% % % % % % % % % % % % % % % % % % % % % % % % % % % % % % % % %
% % % % % % % % % % % % % % % % % % % % % % % % % % % % % % % % %
% % % % % % % % % % % % % % % % % % % % % % % % % % % % % % % % %
% % % % % % % % % % % % % % % % % % % % % % % % % % % % % %

\chapter*{Chapter IX} \ifaudio \marginpar{
\href{http://ia600205.us.archive.org/6/items/war_and_peace_11_0909/war_and_peace_11_09_tolstoy_64kb.mp3}{Audio}}
\fi

\initial{S}{carcely} had Pierre laid his head on the pillow before he felt
himself falling asleep, but suddenly, almost with the
distinctness of reality, he heard the boom, boom, boom of firing,
the thud of projectiles, groans and cries, and smelled blood and
powder, and a feeling of horror and dread of death seized
him. Filled with fright he opened his eyes and lifted his head
from under his cloak. All was tranquil in the yard. Only
someone's orderly passed through the gateway, splashing through
the mud, and talked to the innkeeper. Above Pierre's head some
pigeons, disturbed by the movement he had made in sitting up,
fluttered under the dark roof of the penthouse. The whole
courtyard was permeated by a strong peaceful smell of stable
yards, delightful to Pierre at that moment. He could see the
clear starry sky between the dark roofs of two penthouses.

``Thank God, there is no more of that!'' he thought, covering up
his head again. ``Oh, what a terrible thing is fear, and how
shamefully I yielded to it! But they... they were steady and calm
all the time, to the end...'' thought he.

They, in Pierre's mind, were the soldiers, those who had been at
the battery, those who had given him food, and those who had
prayed before the icon. They, those strange men he had not
previously known, stood out clearly and sharply from everyone
else.

``To be a soldier, just a soldier!'' thought Pierre as he fell
asleep, ``to enter communal life completely, to be imbued by what
makes them what they are. But how cast off all the superfluous,
devilish burden of my outer man? There was a time when I could
have done it. I could have run away from my father, as I wanted
to. Or I might have been sent to serve as a soldier after the
duel with Dolokhov.'' And the memory of the dinner at the English
Club when he had challenged Dolokhov flashed through Pierre's
mind, and then he remembered his benefactor at Torzhok. And now a
picture of a solemn meeting of the lodge presented itself to his
mind.  It was taking place at the English Club and someone near
and dear to him sat at the end of the table. ``Yes, that is he!
It is my benefactor. But he died!'' thought Pierre. ``Yes, he
died, and I did not know he was alive. How sorry I am that he
died, and how glad I am that he is alive again!'' On one side of
the table sat Anatole, Dolokhov, Nesvitski, Denisov, and others
like them (in his dream the category to which these men belonged
was as clearly defined in his mind as the category of those he
termed they), and he heard those people, Anatole and Dolokhov,
shouting and singing loudly; yet through their shouting the voice
of his benefactor was heard speaking all the time and the sound
of his words was as weighty and uninterrupted as the booming on
the battlefield, but pleasant and comforting. Pierre did not
understand what his benefactor was saying, but he knew (the
categories of thoughts were also quite distinct in his dream)
that he was talking of goodness and the possibility of being what
they were. And they with their simple, kind, firm faces
surrounded his benefactor on all sides. But though they were
kindly they did not look at Pierre and did not know him. Wishing
to speak and to attract their attention, he got up, but at that
moment his legs grew cold and bare.

He felt ashamed, and with one arm covered his legs from which his
cloak had in fact slipped. For a moment as he was rearranging his
cloak Pierre opened his eyes and saw the same penthouse roofs,
posts, and yard, but now they were all bluish, lit up, and
glittering with frost or dew.

``It is dawn,'' thought Pierre. ``But that's not what I want. I
want to hear and understand my benefactor's words.'' Again he
covered himself up with his cloak, but now neither the lodge nor
his benefactor was there.  There were only thoughts clearly
expressed in words, thoughts that someone was uttering or that he
himself was formulating.

Afterwards when he recalled those thoughts Pierre was convinced
that someone outside himself had spoken them, though the
impressions of that day had evoked them. He had never, it seemed
to him, been able to think and express his thoughts like that
when awake.

``To endure war is the most difficult subordination of man's
freedom to the law of God,'' the voice had said. ``Simplicity is
submission to the will of God; you cannot escape from Him. And
they are simple. They do not talk, but act. The spoken word is
silver but the unspoken is golden.  Man can be master of nothing
while he fears death, but he who does not fear it possesses
all. If there were no suffering, man would not know his
limitations, would not know himself. The hardest thing (Pierre
went on thinking, or hearing, in his dream) is to be able in your
soul to unite the meaning of all. To unite all?'' he asked
himself. ``No, not to unite. Thoughts cannot be united, but to
harness all these thoughts together is what we need! Yes, one
must harness them, must harness them!'' he repeated to himself
with inward rapture, feeling that these words and they alone
expressed what he wanted to say and solved the question that
tormented him.

``Yes, one must harness, it is time to harness.''

``Time to harness, time to harness, your excellency! Your
excellency!''  some voice was repeating. ``We must harness, it is
time to harness...''

It was the voice of the groom, trying to wake him. The sun shone
straight into Pierre's face. He glanced at the dirty innyard in
the middle of which soldiers were watering their lean horses at
the pump while carts were passing out of the gate. Pierre turned
away with repugnance, and closing his eyes quickly fell back on
the carriage seat.  ``No, I don't want that, I don't want to see
and understand that. I want to understand what was revealing
itself to me in my dream. One second more and I should have
understood it all! But what am I to do? Harness, but how can I
harness everything?'' and Pierre felt with horror that the
meaning of all he had seen and thought in the dream had been
destroyed.

The groom, the coachman, and the innkeeper told Pierre that an
officer had come with news that the French were already near
Mozhaysk and that our men were leaving it.

Pierre got up and, having told them to harness and overtake him,
went on foot through the town.

The troops were moving on, leaving about ten thousand wounded
behind them. There were wounded in the yards, at the windows of
the houses, and the streets were crowded with them. In the
streets, around carts that were to take some of the wounded away,
shouts, curses, and blows could be heard. Pierre offered the use
of his carriage, which had overtaken him, to a wounded general he
knew, and drove with him to Moscow. On the way Pierre was told of
the death of his brother-in-law Anatole and of that of Prince
Andrew.

% % % % % % % % % % % % % % % % % % % % % % % % % % % % % % % % %
% % % % % % % % % % % % % % % % % % % % % % % % % % % % % % % % %
% % % % % % % % % % % % % % % % % % % % % % % % % % % % % % % % %
% % % % % % % % % % % % % % % % % % % % % % % % % % % % % % % % %
% % % % % % % % % % % % % % % % % % % % % % % % % % % % % % % % %
% % % % % % % % % % % % % % % % % % % % % % % % % % % % % % % % %
% % % % % % % % % % % % % % % % % % % % % % % % % % % % % % % % %
% % % % % % % % % % % % % % % % % % % % % % % % % % % % % % % % %
% % % % % % % % % % % % % % % % % % % % % % % % % % % % % % % % %
% % % % % % % % % % % % % % % % % % % % % % % % % % % % % % % % %
% % % % % % % % % % % % % % % % % % % % % % % % % % % % % % % % %
% % % % % % % % % % % % % % % % % % % % % % % % % % % % % %

\chapter*{Chapter X} \ifaudio \marginpar{
\href{http://ia600205.us.archive.org/6/items/war_and_peace_11_0909/war_and_peace_11_10_tolstoy_64kb.mp3}{Audio}}
\fi

\initial{O}{n} the thirteenth of August Pierre reached Moscow. Close to the
gates of the city he was met by Count Rostopchin's adjutant.

``We have been looking for you everywhere,'' said the
adjutant. ``The count wants to see you particularly. He asks you
to come to him at once on a very important matter.''

Without going home, Pierre took a cab and drove to see the Moscow
commander-in-chief.

Count Rostopchin had only that morning returned to town from his
summer villa at Sokolniki. The anteroom and reception room of his
house were full of officials who had been summoned or had come
for orders.  Vasilchikov and Platov had already seen the count
and explained to him that it was impossible to defend Moscow and
that it would have to be surrendered. Though this news was being
concealed from the inhabitants, the officials---the heads of the
various government departments---knew that Moscow would soon be
in the enemy's hands, just as Count Rostopchin himself knew it,
and to escape personal responsibility they had all come to the
governor to ask how they were to deal with their various
departments.

As Pierre was entering the reception room a courier from the army
came out of Rostopchin's private room.

In answer to questions with which he was greeted, the courier
made a despairing gesture with his hand and passed through the
room.

While waiting in the reception room Pierre with weary eyes
watched the various officials, old and young, military and
civilian, who were there.  They all seemed dissatisfied and
uneasy. Pierre went up to a group of men, one of whom he
knew. After greeting Pierre they continued their conversation.

``If they're sent out and brought back again later on it will do
no harm, but as things are now one can't answer for anything.''

``But you see what he writes...'' said another, pointing to a
printed sheet he held in his hand.

``That's another matter. That's necessary for the people,'' said
the first.

``What is it?'' asked Pierre.

``Oh, it's a fresh broadsheet.''

Pierre took it and began reading.

His Serene Highness has passed through Mozhaysk in order to join
up with the troops moving toward him and has taken up a strong
position where the enemy will not soon attack him. Forty eight
guns with ammunition have been sent him from here, and his Serene
Highness says he will defend Moscow to the last drop of blood and
is even ready to fight in the streets. Do not be upset, brothers,
that the law courts are closed; things have to be put in order,
and we will deal with villains in our own way! When the time
comes I shall want both town and peasant lads and will raise the
cry a day or two beforehand, but they are not wanted yet so I
hold my peace. An ax will be useful, a hunting spear not bad, but
a three-pronged fork will be best of all: a Frenchman is no
heavier than a sheaf of rye. Tomorrow after dinner I shall take
the Iberian icon of the Mother of God to the wounded in the
Catherine Hospital where we will have some water blessed. That
will help them to get well quicker. I, too, am well now: one of
my eyes was sore but now I am on the lookout with both.

``But military men have told me that it is impossible to fight in
the town,'' said Pierre, ``and that the position...''

``Well, of course! That's what we were saying,'' replied the
first speaker.

``And what does he mean by 'One of my eyes was sore but now I am
on the lookout with both'?'' asked Pierre.

``The count had a sty,'' replied the adjutant smiling, ``and was
very much upset when I told him people had come to ask what was
the matter with him. By the by, Count,'' he added suddenly,
addressing Pierre with a smile, ``we heard that you have family
troubles and that the countess, your wife...''

``I have heard nothing,'' Pierre replied unconcernedly. ``But
what have you heard?''

``Oh, well, you know people often invent things. I only say what
I heard.''

``But what did you hear?''

``Well, they say,'' continued the adjutant with the same smile,
``that the countess, your wife, is preparing to go abroad. I
expect it's nonsense...''

``Possibly,'' remarked Pierre, looking about him
absent-mindedly. ``And who is that?'' he asked, indicating a
short old man in a clean blue peasant overcoat, with a big
snow-white beard and eyebrows and a ruddy face.

``He? That's a tradesman, that is to say, he's the restaurant
keeper, Vereshchagin. Perhaps you have heard of that affair with
the proclamation.''

``Oh, so that is Vereshchagin!'' said Pierre, looking at the
firm, calm face of the old man and seeking any indication of his
being a traitor.

``That's not he himself, that's the father of the fellow who
wrote the proclamation,'' said the adjutant. ``The young man is
in prison and I expect it will go hard with him.''

An old gentleman wearing a star and another official, a German
wearing a cross round his neck, approached the speaker.

``It's a complicated story, you know,'' said the adjutant. ``That
proclamation appeared about two months ago. The count was
informed of it. He gave orders to investigate the matter. Gabriel
Ivanovich here made the inquiries. The proclamation had passed
through exactly sixty-three hands. He asked one, 'From whom did
you get it?' 'From so-and-so.'  He went to the next one. 'From
whom did you get it?' and so on till he reached Vereshchagin, a
half educated tradesman, you know, 'a pet of a trader,''' said
the adjutant smiling. ``They asked him, 'Who gave it you?'  And
the point is that we knew whom he had it from. He could only have
had it from the Postmaster. But evidently they had come to some
understanding. He replied: 'From no one; I made it up myself.'
They threatened and questioned him, but he stuck to that: 'I made
it up myself.' And so it was reported to the count, who sent for
the man.  'From whom did you get the proclamation?' 'I wrote it
myself.' Well, you know the count,'' said the adjutant
cheerfully, with a smile of pride, ``he flared up
dreadfully---and just think of the fellow's audacity, lying, and
obstinacy!''

``And the count wanted him to say it was from Klyucharev? I
understand!''  said Pierre.

``Not at all,'' rejoined the adjutant in dismay. ``Klyucharev had
his own sins to answer for without that and that is why he has
been banished.  But the point is that the count was much
annoyed. 'How could you have written it yourself?' said he, and
he took up the Hamburg Gazette that was lying on the table. 'Here
it is! You did not write it yourself but translated it, and
translated it abominably, because you don't even know French, you
fool.' And what do you think? 'No,' said he, 'I have not read any
papers, I made it up myself.' 'If that's so, you're a traitor and
I'll have you tried, and you'll be hanged! Say from whom you had
it.' 'I have seen no papers, I made it up myself.' And that was
the end of it. The count had the father fetched, but the fellow
stuck to it. He was sent for trial and condemned to hard labor, I
believe. Now the father has come to intercede for him. But he's a
good-for-nothing lad!  You know that sort of tradesman's son, a
dandy and lady-killer. He attended some lectures somewhere and
imagines that the devil is no match for him. That's the sort of
fellow he is. His father keeps a cookshop here by the Stone
Bridge, and you know there was a large icon of God Almighty
painted with a scepter in one hand and an orb in the other.
Well, he took that icon home with him for a few days and what did
he do?  He found some scoundrel of a painter...''

% % % % % % % % % % % % % % % % % % % % % % % % % % % % % % % % %
% % % % % % % % % % % % % % % % % % % % % % % % % % % % % % % % %
% % % % % % % % % % % % % % % % % % % % % % % % % % % % % % % % %
% % % % % % % % % % % % % % % % % % % % % % % % % % % % % % % % %
% % % % % % % % % % % % % % % % % % % % % % % % % % % % % % % % %
% % % % % % % % % % % % % % % % % % % % % % % % % % % % % % % % %
% % % % % % % % % % % % % % % % % % % % % % % % % % % % % % % % %
% % % % % % % % % % % % % % % % % % % % % % % % % % % % % % % % %
% % % % % % % % % % % % % % % % % % % % % % % % % % % % % % % % %
% % % % % % % % % % % % % % % % % % % % % % % % % % % % % % % % %
% % % % % % % % % % % % % % % % % % % % % % % % % % % % % % % % %
% % % % % % % % % % % % % % % % % % % % % % % % % % % % % %

\chapter*{Chapter XI} \ifaudio \marginpar{
\href{http://ia600205.us.archive.org/6/items/war_and_peace_11_0909/war_and_peace_11_11_tolstoy_64kb.mp3}{Audio}}
\fi

\initial{I}{n} the middle of this fresh tale Pierre was summoned to the
commander in chief.

When he entered the private room Count Rostopchin, puckering his
face, was rubbing his forehead and eyes with his hand. A short
man was saying something, but when Pierre entered he stopped
speaking and went out.

``Ah, how do you do, great warrior?'' said Rostopchin as soon as
the short man had left the room. ``We have heard of your
prowess. But that's not the point. Between ourselves, mon cher,
do you belong to the Masons?'' he went on severely, as though
there were something wrong about it which he nevertheless
intended to pardon. Pierre remained silent. ``I am well informed,
my friend, but I am aware that there are Masons and I hope that
you are not one of those who on pretense of saving mankind wish
to ruin Russia.''

``Yes, I am a Mason,'' Pierre replied.

``There, you see, mon cher! I expect you know that
Messrs. Speranski and Magnitski have been deported to their
proper place. Mr. Klyucharev has been treated in the same way,
and so have others who on the plea of building up the temple of
Solomon have tried to destroy the temple of their fatherland. You
can understand that there are reasons for this and that I could
not have exiled the Postmaster had he not been a harmful
person. It has now come to my knowledge that you lent him your
carriage for his removal from town, and that you have even
accepted papers from him for safe custody. I like you and don't
wish you any harm and---as you are only half my age---I advise
you, as a father would, to cease all communication with men of
that stamp and to leave here as soon as possible.''

``But what did Klyucharev do wrong, Count?'' asked Pierre.

``That is for me to know, but not for you to ask,'' shouted
Rostopchin.

``If he is accused of circulating Napoleon's proclamation it is
not proved that he did so,'' said Pierre without looking at
Rostopchin, ``and Vereshchagin...''

``There we are!'' Rostopchin shouted at Pierre louder than
before, frowning suddenly. ``Vereshchagin is a renegade and a
traitor who will be punished as he deserves,'' said he with the
vindictive heat with which people speak when recalling an
insult. ``But I did not summon you to discuss my actions, but to
give you advice---or an order if you prefer it. I beg you to
leave the town and break off all communication with such men as
Klyucharev. And I will knock the nonsense out of anybody''---but
probably realizing that he was shouting at Bezukhov who so far
was not guilty of anything, he added, taking Pierre's hand in a
friendly manner, ``We are on the eve of a public disaster and I
haven't time to be polite to everybody who has business with
me. My head is sometimes in a whirl. Well, mon cher, what are you
doing personally?''

``Why, nothing,'' answered Pierre without raising his eyes or
changing the thoughtful expression of his face.

The count frowned.

``A word of friendly advice, mon cher. Be off as soon as you can,
that's all I have to tell you. Happy he who has ears to
hear. Good-bye, my dear fellow. Oh, by the by!'' he shouted
through the doorway after Pierre, ``is it true that the countess
has fallen into the clutches of the holy fathers of the Society
of Jesus?''

Pierre did not answer and left Rostopchin's room more sullen and
angry than he had ever before shown himself.

When he reached home it was already getting dark. Some eight
people had come to see him that evening: the secretary of a
committee, the colonel of his battalion, his steward, his
major-domo, and various petitioners.  They all had business with
Pierre and wanted decisions from him. Pierre did not understand
and was not interested in any of these questions and only
answered them in order to get rid of these people. When left
alone at last he opened and read his wife's letter.

``They, the soldiers at the battery, Prince Andrew killed... that
old man... Simplicity is submission to God. Suffering is
necessary... the meaning of all... one must harness... my wife is
getting married... One must forget and understand...'' And going
to his bed he threw himself on it without undressing and
immediately fell asleep.

When he awoke next morning the major-domo came to inform him that
a special messenger, a police officer, had come from Count
Rostopchin to know whether Count Bezukhov had left or was leaving
the town.

A dozen persons who had business with Pierre were awaiting him in
the drawing room. Pierre dressed hurriedly and, instead of going
to see them, went to the back porch and out through the gate.

From that time till the end of the destruction of Moscow no one
of Bezukhov's household, despite all the search they made, saw
Pierre again or knew where he was.

% % % % % % % % % % % % % % % % % % % % % % % % % % % % % % % % %
% % % % % % % % % % % % % % % % % % % % % % % % % % % % % % % % %
% % % % % % % % % % % % % % % % % % % % % % % % % % % % % % % % %
% % % % % % % % % % % % % % % % % % % % % % % % % % % % % % % % %
% % % % % % % % % % % % % % % % % % % % % % % % % % % % % % % % %
% % % % % % % % % % % % % % % % % % % % % % % % % % % % % % % % %
% % % % % % % % % % % % % % % % % % % % % % % % % % % % % % % % %
% % % % % % % % % % % % % % % % % % % % % % % % % % % % % % % % %
% % % % % % % % % % % % % % % % % % % % % % % % % % % % % % % % %
% % % % % % % % % % % % % % % % % % % % % % % % % % % % % % % % %
% % % % % % % % % % % % % % % % % % % % % % % % % % % % % % % % %
% % % % % % % % % % % % % % % % % % % % % % % % % % % % % %

\chapter*{Chapter XII} \ifaudio \marginpar{
\href{http://ia600205.us.archive.org/6/items/war_and_peace_11_0909/war_and_peace_11_12_tolstoy_64kb.mp3}{Audio}}
\fi

\initial{T}{he} Rostovs remained in Moscow till the first of September, that
is, till the eve of the enemy's entry into the city.

After Petya had joined Obolenski's regiment of Cossacks and left
for Belaya Tserkov where that regiment was forming, the countess
was seized with terror. The thought that both her sons were at
the war, had both gone from under her wing, that today or
tomorrow either or both of them might be killed like the three
sons of one of her acquaintances, struck her that summer for the
first time with cruel clearness. She tried to get Nicholas back
and wished to go herself to join Petya, or to get him an
appointment somewhere in Petersburg, but neither of these proved
possible. Petya could not return unless his regiment did so or
unless he was transferred to another regiment on active
service. Nicholas was somewhere with the army and had not sent a
word since his last letter, in which he had given a detailed
account of his meeting with Princess Mary. The countess did not
sleep at night, or when she did fall asleep dreamed that she saw
her sons lying dead. After many consultations and conversations,
the count at last devised means to tranquillize her. He got Petya
transferred from Obolenski's regiment to Bezukhov's, which was in
training near Moscow. Though Petya would remain in the service,
this transfer would give the countess the consolation of seeing
at least one of her sons under her wing, and she hoped to arrange
matters for her Petya so as not to let him go again, but always
get him appointed to places where he could not possibly take part
in a battle. As long as Nicholas alone was in danger the countess
imagined that she loved her first-born more than all her other
children and even reproached herself for it; but when her
youngest: the scapegrace who had been bad at lessons, was always
breaking things in the house and making himself a nuisance to
everybody, that snub-nosed Petya with his merry black eyes and
fresh rosy cheeks where soft down was just beginning to
show---when he was thrown amid those big, dreadful, cruel men who
were fighting somewhere about something and apparently finding
pleasure in it---then his mother thought she loved him more, much
more, than all her other children. The nearer the time came for
Petya to return, the more uneasy grew the countess. She began to
think she would never live to see such happiness. The presence of
Sonya, of her beloved Natasha, or even of her husband irritated
her. ``What do I want with them? I want no one but Petya,'' she
thought.

At the end of August the Rostovs received another letter from
Nicholas.  He wrote from the province of Voronezh where he had
been sent to procure remounts, but that letter did not set the
countess at ease. Knowing that one son was out of danger she
became the more anxious about Petya.

Though by the twentieth of August nearly all the Rostovs'
acquaintances had left Moscow, and though everybody tried to
persuade the countess to get away as quickly as possible, she
would not hear of leaving before her treasure, her adored Petya,
returned. On the twenty-eighth of August he arrived. The
passionate tenderness with which his mother received him did not
please the sixteen-year-old officer. Though she concealed from
him her intention of keeping him under her wing, Petya guessed
her designs, and instinctively fearing that he might give way to
emotion when with her---might \emph{become womanish} as he termed
it to himself---he treated her coldly, avoided her, and during
his stay in Moscow attached himself exclusively to Natasha for
whom he had always had a particularly brotherly tenderness,
almost lover-like.

Owing to the count's customary carelessness nothing was ready for
their departure by the twenty-eighth of August and the carts that
were to come from their Ryazan and Moscow estates to remove their
household belongings did not arrive till the thirtieth.

From the twenty-eighth till the thirty-first all Moscow was in a
bustle and commotion. Every day thousands of men wounded at
Borodino were brought in by the Dorogomilov gate and taken to
various parts of Moscow, and thousands of carts conveyed the
inhabitants and their possessions out by the other gates. In
spite of Rostopchin's broadsheets, or because of them or
independently of them, the strangest and most contradictory
rumors were current in the town. Some said that no one was to be
allowed to leave the city, others on the contrary said that all
the icons had been taken out of the churches and everybody was to
be ordered to leave.  Some said there had been another battle
after Borodino at which the French had been routed, while others
on the contrary reported that the Russian army had been
destroyed. Some talked about the Moscow militia which, preceded
by the clergy, would go to the Three Hills; others whispered that
Augustin had been forbidden to leave, that traitors had been
seized, that the peasants were rioting and robbing people on
their way from Moscow, and so on. But all this was only talk; in
reality (though the Council of Fili, at which it was decided to
abandon Moscow, had not yet been held) both those who went away
and those who remained behind felt, though they did not show it,
that Moscow would certainly be abandoned, and that they ought to
get away as quickly as possible and save their belongings. It was
felt that everything would suddenly break up and change, but up
to the first of September nothing had done so. As a criminal who
is being led to execution knows that he must die immediately, but
yet looks about him and straightens the cap that is awry on his
head, so Moscow involuntarily continued its wonted life, though
it knew that the time of its destruction was near when the
conditions of life to which its people were accustomed to submit
would be completely upset.

During the three days preceding the occupation of Moscow the
whole Rostov family was absorbed in various activities. The head
of the family, Count Ilya Rostov, continually drove about the
city collecting the current rumors from all sides and gave
superficial and hasty orders at home about the preparations for
their departure.

The countess watched the things being packed, was dissatisfied
with everything, was constantly in pursuit of Petya who was
always running away from her, and was jealous of Natasha with
whom he spent all his time. Sonya alone directed the practical
side of matters by getting things packed. But of late Sonya had
been particularly sad and silent.  Nicholas' letter in which he
mentioned Princess Mary had elicited, in her presence, joyous
comments from the countess, who saw an intervention of Providence
in this meeting of the princess and Nicholas.

``I was never pleased at Bolkonski's engagement to Natasha,''
said the countess, ``but I always wanted Nicholas to marry the
princess, and had a presentiment that it would happen. What a
good thing it would be!''

Sonya felt that this was true: that the only possibility of
retrieving the Rostovs' affairs was by Nicholas marrying a rich
woman, and that the princess was a good match. It was very bitter
for her. But despite her grief, or perhaps just because of it,
she took on herself all the difficult work of directing the
storing and packing of their things and was busy for whole
days. The count and countess turned to her when they had any
orders to give. Petya and Natasha on the contrary, far from
helping their parents, were generally a nuisance and a hindrance
to everyone. Almost all day long the house resounded with their
running feet, their cries, and their spontaneous laughter. They
laughed and were gay not because there was any reason to laugh,
but because gaiety and mirth were in their hearts and so
everything that happened was a cause for gaiety and laughter to
them. Petya was in high spirits because having left home a boy he
had returned (as everybody told him) a fine young man, because he
was at home, because he had left Belaya Tserkov where there was
no hope of soon taking part in a battle and had come to Moscow
where there was to be fighting in a few days, and chiefly because
Natasha, whose lead he always followed, was in high
spirits. Natasha was gay because she had been sad too long and
now nothing reminded her of the cause of her sadness, and because
she was feeling well. She was also happy because she had someone
to adore her: the adoration of others was a lubricant the wheels
of her machine needed to make them run freely---and Petya adored
her. Above all, they were gay because there was a war near
Moscow, there would be fighting at the town gates, arms were
being given out, everybody was escaping---going away somewhere,
and in general something extraordinary was happening, and that is
always exciting, especially to the young.

% % % % % % % % % % % % % % % % % % % % % % % % % % % % % % % % %
% % % % % % % % % % % % % % % % % % % % % % % % % % % % % % % % %
% % % % % % % % % % % % % % % % % % % % % % % % % % % % % % % % %
% % % % % % % % % % % % % % % % % % % % % % % % % % % % % % % % %
% % % % % % % % % % % % % % % % % % % % % % % % % % % % % % % % %
% % % % % % % % % % % % % % % % % % % % % % % % % % % % % % % % %
% % % % % % % % % % % % % % % % % % % % % % % % % % % % % % % % %
% % % % % % % % % % % % % % % % % % % % % % % % % % % % % % % % %
% % % % % % % % % % % % % % % % % % % % % % % % % % % % % % % % %
% % % % % % % % % % % % % % % % % % % % % % % % % % % % % % % % %
% % % % % % % % % % % % % % % % % % % % % % % % % % % % % % % % %
% % % % % % % % % % % % % % % % % % % % % % % % % % % % % %

\chapter*{Chapter XIII} \ifaudio \marginpar{
\href{http://ia600205.us.archive.org/6/items/war_and_peace_11_0909/war_and_peace_11_13_tolstoy_64kb.mp3}{Audio}}
\fi

\initial{O}{n} Saturday, the thirty-first of August, everything in the
Rostovs' house seemed topsy-turvy. All the doors were open, all
the furniture was being carried out or moved about, and the
mirrors and pictures had been taken down. There were trunks in
the rooms, and hay, wrapping paper, and ropes were scattered
about. The peasants and house serfs carrying out the things were
treading heavily on the parquet floors. The yard was crowded with
peasant carts, some loaded high and already corded up, others
still empty.

The voices and footsteps of the many servants and of the peasants
who had come with the carts resounded as they shouted to one
another in the yard and in the house. The count had been out
since morning. The countess had a headache brought on by all the
noise and turmoil and was lying down in the new sitting room with
a vinegar compress on her head.  Petya was not at home, he had
gone to visit a friend with whom he meant to obtain a transfer
from the militia to the active army. Sonya was in the ballroom
looking after the packing of the glass and china. Natasha was
sitting on the floor of her dismantled room with dresses,
ribbons, and scarves strewn all about her, gazing fixedly at the
floor and holding in her hands the old ball dress (already out of
fashion) which she had worn at her first Petersburg ball.

Natasha was ashamed of doing nothing when everyone else was so
busy, and several times that morning had tried to set to work,
but her heart was not in it, and she could not and did not know
how to do anything except with all her heart and all her
might. For a while she had stood beside Sonya while the china was
being packed and tried to help, but soon gave it up and went to
her room to pack her own things. At first she found it amusing to
give away dresses and ribbons to the maids, but when that was
done and what was left had still to be packed, she found it dull.

``Dunyasha, you pack! You will, won't you, dear?'' And when
Dunyasha willingly promised to do it all for her, Natasha sat
down on the floor, took her old ball dress, and fell into a
reverie quite unrelated to what ought to have occupied her
thoughts now. She was roused from her reverie by the talk of the
maids in the next room (which was theirs) and by the sound of
their hurried footsteps going to the back porch. Natasha got up
and looked out of the window. An enormously long row of carts
full of wounded men had stopped in the street.

The housekeeper, the old nurse, the cooks, coachmen, maids,
footmen, postilions, and scullions stood at the gate, staring at
the wounded.

Natasha, throwing a clean pocket handkerchief over her hair and
holding an end of it in each hand, went out into the street.

The former housekeeper, old Mavra Kuzminichna, had stepped out of
the crowd by the gate, gone up to a cart with a hood constructed
of bast mats, and was speaking to a pale young officer who lay
inside. Natasha moved a few steps forward and stopped shyly,
still holding her handkerchief, and listened to what the
housekeeper was saying.

``Then you have nobody in Moscow?'' she was saying. ``You would
be more comfortable somewhere in a house... in ours, for
instance... the family are leaving.''

``I don't know if it would be allowed,'' replied the officer in a
weak voice. ``Here is our commanding officer... ask him,'' and he
pointed to a stout major who was walking back along the street
past the row of carts.

Natasha glanced with frightened eyes at the face of the wounded
officer and at once went to meet the major.

``May the wounded men stay in our house?'' she asked.

The major raised his hand to his cap with a smile.

``Which one do you want, Ma'am'selle?'' said he, screwing up his
eyes and smiling.

Natasha quietly repeated her question, and her face and whole
manner were so serious, though she was still holding the ends of
her handkerchief, that the major ceased smiling and after some
reflection---as if considering in how far the thing was
possible---replied in the affirmative.

``Oh yes, why not? They may,'' he said.

With a slight inclination of her head, Natasha stepped back
quickly to Mavra Kuzminichna, who stood talking compassionately
to the officer.

``They may. He says they may!'' whispered Natasha.

The cart in which the officer lay was turned into the Rostovs'
yard, and dozens of carts with wounded men began at the
invitation of the townsfolk to turn into the yards and to draw up
at the entrances of the houses in Povarskaya Street. Natasha was
evidently pleased to be dealing with new people outside the
ordinary routine of her life. She and Mavra Kuzminichna tried to
get as many of the wounded as possible into their yard.

``Your Papa must be told, though,'' said Mavra Kuzminichna.

``Never mind, never mind, what does it matter? For one day we can
move into the drawing room. They can have all our half of the
house.''

``There now, young lady, you do take things into your head! Even
if we put them into the wing, the men's room, or the nurse's
room, we must ask permission.''

``Well, I'll ask.''

Natasha ran into the house and went on tiptoe through the
half-open door into the sitting room, where there was a smell of
vinegar and Hoffman's drops.

``Are you asleep, Mamma?''

``Oh, what sleep-?'' said the countess, waking up just as she was
dropping into a doze.

``Mamma darling!'' said Natasha, kneeling by her mother and
bringing her face close to her mother's, ``I am sorry, forgive
me, I'll never do it again; I woke you up! Mavra Kuzminichna has
sent me: they have brought some wounded here---officers. Will you
let them come? They have nowhere to go. I knew you'd let them
come!'' she said quickly all in one breath.

``What officers? Whom have they brought? I don't understand
anything about it,'' said the countess.

Natasha laughed, and the countess too smiled slightly.

``I knew you'd give permission... so I'll tell them,'' and,
having kissed her mother, Natasha got up and went to the door.

In the hall she met her father, who had returned with bad news.

``We've stayed too long!'' said the count with involuntary
vexation. ``The club is closed and the police are leaving.''

``Papa, is it all right---I've invited some of the wounded into
the house?'' said Natasha.

``Of course it is,'' he answered absently. ``That's not the
point. I beg you not to indulge in trifles now, but to help to
pack, and tomorrow we must go, go, go!...''

And the count gave a similar order to the major-domo and the
servants.

At dinner Petya having returned home told them the news he had
heard. He said the people had been getting arms in the Kremlin,
and that though Rostopchin's broadsheet had said that he would
sound a call two or three days in advance, the order had
certainly already been given for everyone to go armed to the
Three Hills tomorrow, and that there would be a big battle there.

The countess looked with timid horror at her son's eager, excited
face as he said this. She realized that if she said a word about
his not going to the battle (she knew he enjoyed the thought of
the impending engagement) he would say something about men,
honor, and the fatherland---something senseless, masculine, and
obstinate which there would be no contradicting, and her plans
would be spoiled; and so, hoping to arrange to leave before then
and take Petya with her as their protector and defender, she did
not answer him, but after dinner called the count aside and
implored him with tears to take her away quickly, that very night
if possible. With a woman's involuntary loving cunning she, who
till then had not shown any alarm, said that she would die of
fright if they did not leave that very night. Without any
pretense she was now afraid of everything.

% % % % % % % % % % % % % % % % % % % % % % % % % % % % % % % % %
% % % % % % % % % % % % % % % % % % % % % % % % % % % % % % % % %
% % % % % % % % % % % % % % % % % % % % % % % % % % % % % % % % %
% % % % % % % % % % % % % % % % % % % % % % % % % % % % % % % % %
% % % % % % % % % % % % % % % % % % % % % % % % % % % % % % % % %
% % % % % % % % % % % % % % % % % % % % % % % % % % % % % % % % %
% % % % % % % % % % % % % % % % % % % % % % % % % % % % % % % % %
% % % % % % % % % % % % % % % % % % % % % % % % % % % % % % % % %
% % % % % % % % % % % % % % % % % % % % % % % % % % % % % % % % %
% % % % % % % % % % % % % % % % % % % % % % % % % % % % % % % % %
% % % % % % % % % % % % % % % % % % % % % % % % % % % % % % % % %
% % % % % % % % % % % % % % % % % % % % % % % % % % % % % %

\chapter*{Chapter XIV} \ifaudio \marginpar{
\href{http://ia600205.us.archive.org/6/items/war_and_peace_11_0909/war_and_peace_11_14_tolstoy_64kb.mp3}{Audio}}
\fi

\initial{M}{adame} Schoss, who had been out to visit her daughter, increased
the countess' fears still more by telling what she had seen at a
spirit dealer's in Myasnitski Street. When returning by that
street she had been unable to pass because of a drunken crowd
rioting in front of the shop. She had taken a cab and driven home
by a side street and the cabman had told her that the people were
breaking open the barrels at the drink store, having received
orders to do so.

After dinner the whole Rostov household set to work with
enthusiastic haste packing their belongings and preparing for
their departure. The old count, suddenly setting to work, kept
passing from the yard to the house and back again, shouting
confused instructions to the hurrying people, and flurrying them
still more. Petya directed things in the yard. Sonya, owing to
the count's contradictory orders, lost her head and did not know
what to do. The servants ran noisily about the house and yard,
shouting and disputing. Natasha, with the ardor characteristic of
all she did suddenly set to work too. At first her intervention
in the business of packing was received skeptically. Everybody
expected some prank from her and did not wish to obey her; but
she resolutely and passionately demanded obedience, grew angry
and nearly cried because they did not heed her, and at last
succeeded in making them believe her.  Her first exploit, which
cost her immense effort and established her authority, was the
packing of the carpets. The count had valuable Gobelin tapestries
and Persian carpets in the house. When Natasha set to work two
cases were standing open in the ballroom, one almost full up with
crockery, the other with carpets. There was also much china
standing on the tables, and still more was being brought in from
the storeroom. A third case was needed and servants had gone to
fetch it.

``Sonya, wait a bit---we'll pack everything into these,'' said
Natasha.

``You can't, Miss, we have tried to,'' said the butler's
assistant.

``No, wait a minute, please.''

And Natasha began rapidly taking out of the case dishes and
plates wrapped in paper.

``The dishes must go in here among the carpets,'' said she.

``Why, it's a mercy if we can get the carpets alone into three
cases,'' said the butler's assistant.

``Oh, wait, please!'' And Natasha began rapidly and deftly
sorting out the things. ``These aren't needed,'' said she,
putting aside some plates of Kiev ware. ``These---yes, these must
go among the carpets,'' she said, referring to the Saxony china
dishes.

``Don't, Natasha! Leave it alone! We'll get it all packed,''
urged Sonya reproachfully.

``What a young lady she is!'' remarked the major-domo.

But Natasha would not give in. She turned everything out and
began quickly repacking, deciding that the inferior Russian
carpets and unnecessary crockery should not be taken at all. When
everything had been taken out of the cases, they recommenced
packing, and it turned out that when the cheaper things not worth
taking had nearly all been rejected, the valuable ones really did
all go into the two cases. Only the lid of the case containing
the carpets would not shut down. A few more things might have
been taken out, but Natasha insisted on having her own way. She
packed, repacked, pressed, made the butler's assistant and
Petya---whom she had drawn into the business of packing---press
on the lid, and made desperate efforts herself.

``That's enough, Natasha,'' said Sonya. ``I see you were right,
but just take out the top one.''

``I won't!'' cried Natasha, with one hand holding back the hair
that hung over her perspiring face, while with the other she
pressed down the carpets. ``Now press, Petya! Press, Vasilich,
press hard!'' she cried.

The carpets yielded and the lid closed; Natasha, clapping her
hands, screamed with delight and tears fell from her eyes. But
this only lasted a moment. She at once set to work afresh and
they now trusted her completely. The count was not angry even
when they told him that Natasha had countermanded an order of
his, and the servants now came to her to ask whether a cart was
sufficiently loaded, and whether it might be corded up. Thanks to
Natasha's directions the work now went on expeditiously,
unnecessary things were left, and the most valuable packed as
compactly as possible.

But hard as they all worked till quite late that night, they
could not get everything packed. The countess had fallen asleep
and the count, having put off their departure till next morning,
went to bed.

Sonya and Natasha slept in the sitting room without undressing.

That night another wounded man was driven down the Povarskaya,
and Mavra Kuzminichna, who was standing at the gate, had him
brought into the Rostovs' yard. Mavra Kuzminichna concluded that
he was a very important man. He was being conveyed in a caleche
with a raised hood, and was quite covered by an apron. On the box
beside the driver sat a venerable old attendant. A doctor and two
soldiers followed the carriage in a cart.

``Please come in here. The masters are going away and the whole
house will be empty,'' said the old woman to the old attendant.

``Well, perhaps,'' said he with a sigh. ``We don't expect to get
him home alive! We have a house of our own in Moscow, but it's a
long way from here, and there's nobody living in it.''

``Do us the honor to come in, there's plenty of everything in the
master's house. Come in,'' said Mavra Kuzminichna. ``Is he very
ill?'' she asked.

The attendant made a hopeless gesture.

``We don't expect to get him home! We must ask the doctor.''

And the old servant got down from the box and went up to the
cart.

``All right!'' said the doctor.

The old servant returned to the caleche, looked into it, shook
his head disconsolately, told the driver to turn into the yard,
and stopped beside Mavra Kuzminichna.

``O, Lord Jesus Christ!'' she murmured.

She invited them to take the wounded man into the house.

``The masters won't object...'' she said.

But they had to avoid carrying the man upstairs, and so they took
him into the wing and put him in the room that had been Madame
Schoss'.

This wounded man was Prince Andrew Bolkonski.

% % % % % % % % % % % % % % % % % % % % % % % % % % % % % % % % %
% % % % % % % % % % % % % % % % % % % % % % % % % % % % % % % % %
% % % % % % % % % % % % % % % % % % % % % % % % % % % % % % % % %
% % % % % % % % % % % % % % % % % % % % % % % % % % % % % % % % %
% % % % % % % % % % % % % % % % % % % % % % % % % % % % % % % % %
% % % % % % % % % % % % % % % % % % % % % % % % % % % % % % % % %
% % % % % % % % % % % % % % % % % % % % % % % % % % % % % % % % %
% % % % % % % % % % % % % % % % % % % % % % % % % % % % % % % % %
% % % % % % % % % % % % % % % % % % % % % % % % % % % % % % % % %
% % % % % % % % % % % % % % % % % % % % % % % % % % % % % % % % %
% % % % % % % % % % % % % % % % % % % % % % % % % % % % % % % % %
% % % % % % % % % % % % % % % % % % % % % % % % % % % % % %

\chapter*{Chapter XV} \ifaudio \marginpar{
\href{http://ia600205.us.archive.org/6/items/war_and_peace_11_0909/war_and_peace_11_15_tolstoy_64kb.mp3}{Audio}}
\fi

\initial{M}{oscow}'s last day had come. It was a clear bright autumn day, a
Sunday.  The church bells everywhere were ringing for service,
just as usual on Sundays. Nobody seemed yet to realize what
awaited the city.

Only two things indicated the social condition of Moscow---the
rabble, that is the poor people, and the price of commodities. An
enormous crowd of factory hands, house serfs, and peasants, with
whom some officials, seminarists, and gentry were mingled, had
gone early that morning to the Three Hills. Having waited there
for Rostopchin who did not turn up, they became convinced that
Moscow would be surrendered, and then dispersed all about the
town to the public houses and cookshops. Prices too that day
indicated the state of affairs. The price of weapons, of gold, of
carts and horses, kept rising, but the value of paper money and
city articles kept falling, so that by midday there were
instances of carters removing valuable goods, such as cloth, and
receiving in payment a half of what they carted, while peasant
horses were fetching five hundred rubles each, and furniture,
mirrors, and bronzes were being given away for nothing.

In the Rostovs' staid old-fashioned house the dissolution of
former conditions of life was but little noticeable. As to the
serfs the only indication was that three out of their huge
retinue disappeared during the night, but nothing was stolen; and
as to the value of their possessions, the thirty peasant carts
that had come in from their estates and which many people envied
proved to be extremely valuable and they were offered enormous
sums of money for them. Not only were huge sums offered for the
horses and carts, but on the previous evening and early in the
morning of the first of September, orderlies and servants sent by
wounded officers came to the Rostovs' and wounded men dragged
themselves there from the Rostovs' and from neighboring houses
where they were accommodated, entreating the servants to try to
get them a lift out of Moscow. The major-domo to whom these
entreaties were addressed, though he was sorry for the wounded,
resolutely refused, saying that he dare not even mention the
matter to the count. Pity these wounded men as one might, it was
evident that if they were given one cart there would be no reason
to refuse another, or all the carts and one's own carriages as
well. Thirty carts could not save all the wounded and in the
general catastrophe one could not disregard oneself and one's own
family. So thought the major-domo on his master's behalf.

On waking up that morning Count Ilya Rostov left his bedroom
softly, so as not to wake the countess who had fallen asleep only
toward morning, and came out to the porch in his lilac silk
dressing gown. In the yard stood the carts ready corded. The
carriages were at the front porch. The major-domo stood at the
porch talking to an elderly orderly and to a pale young officer
with a bandaged arm. On seeing the count the major-domo made a
significant and stern gesture to them both to go away.

``Well, Vasilich, is everything ready?'' asked the count, and
stroking his bald head he looked good-naturedly at the officer
and the orderly and nodded to them. (He liked to see new faces.)

``We can harness at once, your excellency.''

``Well, that's right. As soon as the countess wakes we'll be off,
God willing! What is it, gentlemen?'' he added, turning to the
officer. ``Are you staying in my house?''

The officer came nearer and suddenly his face flushed crimson.

``Count, be so good as to allow me... for God's sake, to get into
some corner of one of your carts! I have nothing here with
me... I shall be all right on a loaded cart...''

Before the officer had finished speaking the orderly made the
same request on behalf of his master.

``Oh, yes, yes, yes!'' said the count hastily. ``I shall be very
pleased, very pleased. Vasilich, you'll see to it. Just unload
one or two carts.  Well, what of it... do what's necessary...''
said the count, muttering some indefinite order.

But at the same moment an expression of warm gratitude on the
officer's face had already sealed the order. The count looked
around him. In the yard, at the gates, at the window of the
wings, wounded officers and their orderlies were to be seen. They
were all looking at the count and moving toward the porch.

``Please step into the gallery, your excellency,'' said the
major-domo.  ``What are your orders about the pictures?''

The count went into the house with him, repeating his order not
to refuse the wounded who asked for a lift.

``Well, never mind, some of the things can be unloaded,'' he
added in a soft, confidential voice, as though afraid of being
overheard.

At nine o'clock the countess woke up, and Matrena Timofeevna, who
had been her lady's maid before her marriage and now performed a
sort of chief gendarme's duty for her, came to say that Madame
Schoss was much offended and the young ladies' summer dresses
could not be left behind.  On inquiry, the countess learned that
Madame Schoss was offended because her trunk had been taken down
from its cart, and all the loads were being uncorded and the
luggage taken out of the carts to make room for wounded men whom
the count in the simplicity of his heart had ordered that they
should take with them. The countess sent for her husband.

``What is this, my dear? I hear that the luggage is being
unloaded.''

``You know, love, I wanted to tell you... Countess dear... an
officer came to me to ask for a few carts for the wounded. After
all, ours are things that can be bought but think what being left
behind means to them!... Really now, in our own yard---we asked
them in ourselves and there are officers among them... You know,
I think, my dear... let them be taken... where's the hurry?''

The count spoke timidly, as he always did when talking of money
matters.  The countess was accustomed to this tone as a precursor
of news of something detrimental to the children's interests,
such as the building of a new gallery or conservatory, the
inauguration of a private theater or an orchestra. She was
accustomed always to oppose anything announced in that timid tone
and considered it her duty to do so.

She assumed her dolefully submissive manner and said to her
husband: ``Listen to me, Count, you have managed matters so that
we are getting nothing for the house, and now you wish to throw
away all our---all the children's property! You said yourself
that we have a hundred thousand rubles' worth of things in the
house. I don't consent, my dear, I don't!  Do as you please! It's
the government's business to look after the wounded; they know
that. Look at the Lopukhins opposite, they cleared out everything
two days ago. That's what other people do. It's only we who are
such fools. If you have no pity on me, have some for the
children.''

Flourishing his arms in despair the count left the room without
replying.

``Papa, what are you doing that for?'' asked Natasha, who had
followed him into her mother's room.

``Nothing! What business is it of yours?'' muttered the count
angrily.

``But I heard,'' said Natasha. ``Why does Mamma object?''

``What business is it of yours?'' cried the count.

Natasha stepped up to the window and pondered.

``Papa! Here's Berg coming to see us,'' said she, looking out of
the window.

% % % % % % % % % % % % % % % % % % % % % % % % % % % % % % % % %
% % % % % % % % % % % % % % % % % % % % % % % % % % % % % % % % %
% % % % % % % % % % % % % % % % % % % % % % % % % % % % % % % % %
% % % % % % % % % % % % % % % % % % % % % % % % % % % % % % % % %
% % % % % % % % % % % % % % % % % % % % % % % % % % % % % % % % %
% % % % % % % % % % % % % % % % % % % % % % % % % % % % % % % % %
% % % % % % % % % % % % % % % % % % % % % % % % % % % % % % % % %
% % % % % % % % % % % % % % % % % % % % % % % % % % % % % % % % %
% % % % % % % % % % % % % % % % % % % % % % % % % % % % % % % % %
% % % % % % % % % % % % % % % % % % % % % % % % % % % % % % % % %
% % % % % % % % % % % % % % % % % % % % % % % % % % % % % % % % %
% % % % % % % % % % % % % % % % % % % % % % % % % % % % % %

\chapter*{Chapter XVI} \ifaudio \marginpar{
\href{http://ia600205.us.archive.org/6/items/war_and_peace_11_0909/war_and_peace_11_16_tolstoy_64kb.mp3}{Audio}}
\fi

\initial{B}{erg}, the Rostovs' son-in-law, was already a colonel wearing the
orders of Vladimir and Anna, and he still filled the quiet and
agreeable post of assistant to the head of the staff of the
assistant commander of the first division of the Second Army.

On the first of September he had come to Moscow from the army.

He had nothing to do in Moscow, but he had noticed that everyone
in the army was asking for leave to visit Moscow and had
something to do there.  So he considered it necessary to ask for
leave of absence for family and domestic reasons.

Berg drove up to his father-in-law's house in his spruce little
trap with a pair of sleek roans, exactly like those of a certain
prince. He looked attentively at the carts in the yard and while
going up to the porch took out a clean pocket handkerchief and
tied a knot in it.

From the anteroom Berg ran with smooth though impatient steps
into the drawing room, where he embraced the count, kissed the
hands of Natasha and Sonya, and hastened to inquire after
\emph{Mamma's} health.

``Health, at a time like this?'' said the count. ``Come, tell us
the news!  Is the army retreating or will there be another
battle?''

``God Almighty alone can decide the fate of our fatherland,
Papa,'' said Berg. ``The army is burning with a spirit of heroism
and the leaders, so to say, have now assembled in council. No one
knows what is coming. But in general I can tell you, Papa, that
such a heroic spirit, the truly antique valor of the Russian
army, which they---which it'' (he corrected himself) ``has shown
or displayed in the battle of the twenty-sixth---there are no
words worthy to do it justice! I tell you, Papa'' (he smote
himself on the breast as a general he had heard speaking had
done, but Berg did it a trifle late for he should have struck his
breast at the words \emph{Russian army}), ``I tell you frankly
that we, the commanders, far from having to urge the men on or
anything of that kind, could hardly restrain
those... those... yes, those exploits of antique valor,'' he went
on rapidly. ``General Barclay de Tolly risked his life everywhere
at the head of the troops, I can assure you. Our corps was
stationed on a hillside. You can imagine!''

And Berg related all that he remembered of the various tales he
had heard those days. Natasha watched him with an intent gaze
that confused him, as if she were trying to find in his face the
answer to some question.

``Altogether such heroism as was displayed by the Russian
warriors cannot be imagined or adequately praised!'' said Berg,
glancing round at Natasha, and as if anxious to conciliate her,
replying to her intent look with a smile. ``'Russia is not in
Moscow, she lives in the hearts of her sons!' Isn't it so,
Papa?'' said he.

Just then the countess came in from the sitting room with a weary
and dissatisfied expression. Berg hurriedly jumped up, kissed her
hand, asked about her health, and, swaying his head from side to
side to express sympathy, remained standing beside her.

``Yes, Mamma, I tell you sincerely that these are hard and sad
times for every Russian. But why are you so anxious? You have
still time to get away...''

``I can't think what the servants are about,'' said the countess,
turning to her husband. ``I have just been told that nothing is
ready yet.  Somebody after all must see to things. One misses
Mitenka at such times.  There won't be any end to it.''

The count was about to say something, but evidently restrained
himself.  He got up from his chair and went to the door.

At that moment Berg drew out his handkerchief as if to blow his
nose and, seeing the knot in it, pondered, shaking his head sadly
and significantly.

``And I have a great favor to ask of you, Papa,'' said he.

``Hm...'' said the count, and stopped.

``I was driving past Yusupov's house just now,'' said Berg with a
laugh, ``when the steward, a man I know, ran out and asked me
whether I wouldn't buy something. I went in out of curiosity, you
know, and there is a small chiffonier and a dressing table. You
know how dear Vera wanted a chiffonier like that and how we had a
dispute about it.'' (At the mention of the chiffonier and
dressing table Berg involuntarily changed his tone to one of
pleasure at his admirable domestic arrangements.) ``And it's such
a beauty! It pulls out and has a secret English drawer, you know!
And dear Vera has long wanted one. I wish to give her a surprise,
you see. I saw so many of those peasant carts in your
yard. Please let me have one, I will pay the man well, and...''

The count frowned and coughed.

``Ask the countess, I don't give orders.''

``If it's inconvenient, please don't,'' said Berg. ``Only I so
wanted it, for dear Vera's sake.''

``Oh, go to the devil, all of you! To the devil, the devil, the
devil...''  cried the old count. ``My head's in a whirl!''

And he left the room. The countess began to cry.

``Yes, Mamma! Yes, these are very hard times!'' said Berg.

Natasha left the room with her father and, as if finding it
difficult to reach some decision, first followed him and then ran
downstairs.

Petya was in the porch, engaged in giving out weapons to the
servants who were to leave Moscow. The loaded carts were still
standing in the yard. Two of them had been uncorded and a wounded
officer was climbing into one of them helped by an orderly.

``Do you know what it's about?'' Petya asked Natasha.

She understood that he meant what were their parents quarreling
about.  She did not answer.

``It's because Papa wanted to give up all the carts to the
wounded,'' said Petya. ``Vasilich told me. I consider...''

``I consider,'' Natasha suddenly almost shouted, turning her
angry face to Petya, ``I consider it so horrid, so abominable,
so... I don't know what.  Are we despicable Germans?''

Her throat quivered with convulsive sobs and, afraid of weakening
and letting the force of her anger run to waste, she turned and
rushed headlong up the stairs.

Berg was sitting beside the countess consoling her with the
respectful attention of a relative. The count, pipe in hand, was
pacing up and down the room, when Natasha, her face distorted by
anger, burst in like a tempest and approached her mother with
rapid steps.

``It's horrid! It's abominable!'' she screamed. ``You can't
possibly have ordered it!''

Berg and the countess looked at her, perplexed and
frightened. The count stood still at the window and listened.

``Mamma, it's impossible: see what is going on in the yard!'' she
cried.  ``They will be left!...''

``What's the matter with you? Who are 'they'? What do you want?''

``Why, the wounded! It's impossible, Mamma. It's
monstrous!... No, Mamma darling, it's not the thing. Please
forgive me, darling... Mamma, what does it matter what we take
away? Only look what is going on in the yard... Mamma!... It's
impossible!''

The count stood by the window and listened without turning round.
Suddenly he sniffed and put his face closer to the window.

The countess glanced at her daughter, saw her face full of shame
for her mother, saw her agitation, and understood why her husband
did not turn to look at her now, and she glanced round quite
disconcerted.

``Oh, do as you like! Am I hindering anyone?'' she said, not
surrendering at once.

``Mamma, darling, forgive me!''

But the countess pushed her daughter away and went up to her
husband.

``My dear, you order what is right... You know I don't understand
about it,'' said she, dropping her eyes shamefacedly.

``The eggs... the eggs are teaching the hen,'' muttered the count
through tears of joy, and he embraced his wife who was glad to
hide her look of shame on his breast.

``Papa! Mamma! May I see to it? May I?...'' asked Natasha. ``We
will still take all the most necessary things.''

The count nodded affirmatively, and Natasha, at the rapid pace at
which she used to run when playing at tag, ran through the
ballroom to the anteroom and downstairs into the yard.

The servants gathered round Natasha, but could not believe the
strange order she brought them until the count himself, in his
wife's name, confirmed the order to give up all the carts to the
wounded and take the trunks to the storerooms. When they
understood that order the servants set to work at this new task
with pleasure and zeal. It no longer seemed strange to them but
on the contrary it seemed the only thing that could be done, just
as a quarter of an hour before it had not seemed strange to
anyone that the wounded should be left behind and the goods
carted away but that had seemed the only thing to do.

The whole household, as if to atone for not having done it
sooner, set eagerly to work at the new task of placing the
wounded in the carts. The wounded dragged themselves out of their
rooms and stood with pale but happy faces round the carts. The
news that carts were to be had spread to the neighboring houses,
from which wounded men began to come into the Rostovs' yard. Many
of the wounded asked them not to unload the carts but only to let
them sit on the top of the things. But the work of unloading,
once started, could not be arrested. It seemed not to matter
whether all or only half the things were left behind. Cases full
of china, bronzes, pictures, and mirrors that had been so
carefully packed the night before now lay about the yard, and
still they went on searching for and finding possibilities of
unloading this or that and letting the wounded have another and
yet another cart.

``We can take four more men,'' said the steward. ``They can have
my trap, or else what is to become of them?''

``Let them have my wardrobe cart,'' said the countess. ``Dunyasha
can go with me in the carriage.''

They unloaded the wardrobe cart and sent it to take wounded men
from a house two doors off. The whole household, servants
included, was bright and animated. Natasha was in a state of
rapturous excitement such as she had not known for a long time.

``What could we fasten this onto?'' asked the servants, trying to
fix a trunk on the narrow footboard behind a carriage. ``We must
keep at least one cart.''

``What's in it?'' asked Natasha.

``The count's books.''

``Leave it, Vasilich will put it away. It's not wanted.''

The phaeton was full of people and there was a doubt as to where
Count Peter could sit.

``On the box. You'll sit on the box, won't you, Petya?'' cried
Natasha.

Sonya too was busy all this time, but the aim of her efforts was
quite different from Natasha's. She was putting away the things
that had to be left behind and making a list of them as the
countess wished, and she tried to get as much taken away with
them as possible.

% % % % % % % % % % % % % % % % % % % % % % % % % % % % % % % % %
% % % % % % % % % % % % % % % % % % % % % % % % % % % % % % % % %
% % % % % % % % % % % % % % % % % % % % % % % % % % % % % % % % %
% % % % % % % % % % % % % % % % % % % % % % % % % % % % % % % % %
% % % % % % % % % % % % % % % % % % % % % % % % % % % % % % % % %
% % % % % % % % % % % % % % % % % % % % % % % % % % % % % % % % %
% % % % % % % % % % % % % % % % % % % % % % % % % % % % % % % % %
% % % % % % % % % % % % % % % % % % % % % % % % % % % % % % % % %
% % % % % % % % % % % % % % % % % % % % % % % % % % % % % % % % %
% % % % % % % % % % % % % % % % % % % % % % % % % % % % % % % % %
% % % % % % % % % % % % % % % % % % % % % % % % % % % % % % % % %
% % % % % % % % % % % % % % % % % % % % % % % % % % % % % %

\chapter*{Chapter XVII} \ifaudio \marginpar{
\href{http://ia600205.us.archive.org/6/items/war_and_peace_11_0909/war_and_peace_11_17_tolstoy_64kb.mp3}{Audio}}
\fi

\initial{B}{efore} two o'clock in the afternoon the Rostovs' four carriages,
packed full and with the horses harnessed, stood at the front
door. One by one the carts with the wounded had moved out of the
yard.

The caleche in which Prince Andrew was being taken attracted
Sonya's attention as it passed the front porch. With the help of
a maid she was arranging a seat for the countess in the huge high
coach that stood at the entrance.

``Whose caleche is that?'' she inquired, leaning out of the
carriage window.

``Why, didn't you know, Miss?'' replied the maid. ``The wounded
prince: he spent the night in our house and is going with us.''

``But who is it? What's his name?''

``It's our intended that was---Prince Bolkonski himself! They say
he is dying,'' replied the maid with a sigh.

Sonya jumped out of the coach and ran to the countess. The
countess, tired out and already dressed in shawl and bonnet for
her journey, was pacing up and down the drawing room, waiting for
the household to assemble for the usual silent prayer with closed
doors before starting.  Natasha was not in the room.

``Mamma,'' said Sonya, ``Prince Andrew is here, mortally
wounded. He is going with us.''

The countess opened her eyes in dismay and, seizing Sonya's arm,
glanced around.

``Natasha?'' she murmured.

At that moment this news had only one significance for both of
them.  They knew their Natasha, and alarm as to what would happen
if she heard this news stifled all sympathy for the man they both
liked.

``Natasha does not know yet, but he is going with us,'' said
Sonya.

``You say he is dying?''

Sonya nodded.

The countess put her arms around Sonya and began to cry.

``The ways of God are past finding out!'' she thought, feeling
that the Almighty Hand, hitherto unseen, was becoming manifest in
all that was now taking place.

``Well, Mamma? Everything is ready. What's the matter?'' asked
Natasha, as with animated face she ran into the room.

``Nothing,'' answered the countess. ``If everything is ready let
us start.''

And the countess bent over her reticule to hide her agitated
face. Sonya embraced Natasha and kissed her.

Natasha looked at her inquiringly.

``What is it? What has happened?''

``Nothing... No...''

``Is it something very bad for me? What is it?'' persisted
Natasha with her quick intuition.

Sonya sighed and made no reply. The count, Petya, Madame Schoss,
Mavra Kuzminichna, and Vasilich came into the drawing room and,
having closed the doors, they all sat down and remained for some
moments silently seated without looking at one another.

The count was the first to rise, and with a loud sigh crossed
himself before the icon. All the others did the same. Then the
count embraced Mavra Kuzminichna and Vasilich, who were to remain
in Moscow, and while they caught at his hand and kissed his
shoulder he patted their backs lightly with some vaguely
affectionate and comforting words. The countess went into the
oratory and there Sonya found her on her knees before the icons
that had been left here and there hanging on the wall.  (The most
precious ones, with which some family tradition was connected,
were being taken with them.)

In the porch and in the yard the men whom Petya had armed with
swords and daggers, with trousers tucked inside their high boots
and with belts and girdles tightened, were taking leave of those
remaining behind.

As is always the case at a departure, much had been forgotten or
put in the wrong place, and for a long time two menservants stood
one on each side of the open door and the carriage steps waiting
to help the countess in, while maids rushed with cushions and
bundles from the house to the carriages, the caleche, the
phaeton, and back again.

``They always will forget everything!'' said the
countess. ``Don't you know I can't sit like that?''

And Dunyasha, with clenched teeth, without replying but with an
aggrieved look on her face, hastily got into the coach to
rearrange the seat.

``Oh, those servants!'' said the count, swaying his head.

Efim, the old coachman, who was the only one the countess trusted
to drive her, sat perched up high on the box and did not so much
as glance round at what was going on behind him. From thirty
years' experience he knew it would be some time yet before the
order, \emph{Be off, in God's name!} would be given him: and he
knew that even when it was said he would be stopped once or twice
more while they sent back to fetch something that had been
forgotten, and even after that he would again be stopped and the
countess herself would lean out of the window and beg him for the
love of heaven to drive carefully down the hill. He knew all this
and therefore waited calmly for what would happen, with more
patience than the horses, especially the near one, the chestnut
Falcon, who was pawing the ground and champing his bit. At last
all were seated, the carriage steps were folded and pulled up,
the door was shut, somebody was sent for a traveling case, and
the countess leaned out and said what she had to say. Then Efim
deliberately doffed his hat and began crossing himself. The
postilion and all the other servants did the same. ``Off, in
God's name!'' said Efim, putting on his hat. ``Start!'' The
postilion started the horses, the off pole horse tugged at his
collar, the high springs creaked, and the body of the coach
swayed. The footman sprang onto the box of the moving coach which
jolted as it passed out of the yard onto the uneven roadway; the
other vehicles jolted in their turn, and the procession of
carriages moved up the street. In the carriages, the caleche, and
the phaeton, all crossed themselves as they passed the church
opposite the house. Those who were to remain in Moscow walked on
either side of the vehicles seeing the travelers off.

Rarely had Natasha experienced so joyful a feeling as now,
sitting in the carriage beside the countess and gazing at the
slowly receding walls of forsaken, agitated Moscow. Occasionally
she leaned out of the carriage window and looked back and then
forward at the long train of wounded in front of them. Almost at
the head of the line she could see the raised hood of Prince
Andrew's caleche. She did not know who was in it, but each time
she looked at the procession her eyes sought that caleche. She
knew it was right in front.

In Kudrino, from the Nikitski, Presnya, and Podnovinsk Streets
came several other trains of vehicles similar to the Rostovs',
and as they passed along the Sadovaya Street the carriages and
carts formed two rows abreast.

As they were going round the Sukharev water tower Natasha, who
was inquisitively and alertly scrutinizing the people driving or
walking past, suddenly cried out in joyful surprise:

``Dear me! Mamma, Sonya, look, it's he!''

``Who? Who?''

``Look! Yes, on my word, it's Bezukhov!'' said Natasha, putting
her head out of the carriage and staring at a tall, stout man in
a coachman's long coat, who from his manner of walking and moving
was evidently a gentleman in disguise, and who was passing under
the arch of the Sukharev tower accompanied by a small,
sallow-faced, beardless old man in a frieze coat.

``Yes, it really is Bezukhov in a coachman's coat, with a
queer-looking old boy. Really,'' said Natasha, ``look, look!''

``No, it's not he. How can you talk such nonsense?''

``Mamma,'' screamed Natasha, ``I'll stake my head it's he! I
assure you!  Stop, stop!'' she cried to the coachman.

But the coachman could not stop, for from the Meshchanski Street
came more carts and carriages, and the Rostovs were being shouted
at to move on and not block the way.

In fact, however, though now much farther off than before, the
Rostovs all saw Pierre---or someone extraordinarily like him---in
a coachman's coat, going down the street with head bent and a
serious face beside a small, beardless old man who looked like a
footman. That old man noticed a face thrust out of the carriage
window gazing at them, and respectfully touching Pierre's elbow
said something to him and pointed to the carriage. Pierre,
evidently engrossed in thought, could not at first understand
him. At length when he had understood and looked in the direction
the old man indicated, he recognized Natasha, and following his
first impulse stepped instantly and rapidly toward the coach. But
having taken a dozen steps he seemed to remember something and
stopped.

Natasha's face, leaning out of the window, beamed with quizzical
kindliness.

``Peter Kirilovich, come here! We have recognized you! This is
wonderful!'' she cried, holding out her hand to him. ``What are
you doing?  Why are you like this?''

Pierre took her outstretched hand and kissed it awkwardly as he
walked along beside her while the coach still moved on.

``What is the matter, Count?'' asked the countess in a surprised
and commiserating tone.

``What? What? Why? Don't ask me,'' said Pierre, and looked round
at Natasha whose radiant, happy expression---of which he was
conscious without looking at her---filled him with enchantment.

``Are you remaining in Moscow, then?''

Pierre hesitated.

``In Moscow?'' he said in a questioning tone. ``Yes, in
Moscow. Good-bye!''

``Ah, if only I were a man! I'd certainly stay with you. How
splendid!''  said Natasha. ``Mamma, if you'll let me, I'll
stay!''

Pierre glanced absently at Natasha and was about to say
something, but the countess interrupted him.

``You were at the battle, we heard.''

``Yes, I was,'' Pierre answered. ``There will be another battle
tomorrow...'' he began, but Natasha interrupted him.

``But what is the matter with you, Count? You are not like
yourself...''

``Oh, don't ask me, don't ask me! I don't know
myself. Tomorrow... But no! Good-bye, good-by!'' he
muttered. ``It's an awful time!'' and dropping behind the
carriage he stepped onto the pavement.

Natasha continued to lean out of the window for a long time,
beaming at him with her kindly, slightly quizzical, happy smile.

% % % % % % % % % % % % % % % % % % % % % % % % % % % % % % % % %
% % % % % % % % % % % % % % % % % % % % % % % % % % % % % % % % %
% % % % % % % % % % % % % % % % % % % % % % % % % % % % % % % % %
% % % % % % % % % % % % % % % % % % % % % % % % % % % % % % % % %
% % % % % % % % % % % % % % % % % % % % % % % % % % % % % % % % %
% % % % % % % % % % % % % % % % % % % % % % % % % % % % % % % % %
% % % % % % % % % % % % % % % % % % % % % % % % % % % % % % % % %
% % % % % % % % % % % % % % % % % % % % % % % % % % % % % % % % %
% % % % % % % % % % % % % % % % % % % % % % % % % % % % % % % % %
% % % % % % % % % % % % % % % % % % % % % % % % % % % % % % % % %
% % % % % % % % % % % % % % % % % % % % % % % % % % % % % % % % %
% % % % % % % % % % % % % % % % % % % % % % % % % % % % % %

\chapter*{Chapter XVIII} \ifaudio \marginpar{
\href{http://ia600205.us.archive.org/6/items/war_and_peace_11_0909/war_and_peace_11_18_tolstoy_64kb.mp3}{Audio}}
\fi

\initial{F}{or} the last two days, ever since leaving home, Pierre had been
living in the empty house of his deceased benefactor,
Bazdeev. This is how it happened.

When he woke up on the morning after his return to Moscow and his
interview with Count Rostopchin, he could not for some time make
out where he was and what was expected of him. When he was
informed that among others awaiting him in his reception room
there was a Frenchman who had brought a letter from his wife, the
Countess Helene, he felt suddenly overcome by that sense of
confusion and hopelessness to which he was apt to succumb. He
felt that everything was now at an end, all was in confusion and
crumbling to pieces, that nobody was right or wrong, the future
held nothing, and there was no escape from this position. Smiling
unnaturally and muttering to himself, he first sat down on the
sofa in an attitude of despair, then rose, went to the door of
the reception room and peeped through the crack, returned
flourishing his arms, and took up a book. His major-domo came in
a second time to say that the Frenchman who had brought the
letter from the countess was very anxious to see him if only for
a minute, and that someone from Bazdeev's widow had called to ask
Pierre to take charge of her husband's books, as she herself was
leaving for the country.

``Oh, yes, in a minute; wait... or no! No, of course... go and
say I will come directly,'' Pierre replied to the major-domo.

But as soon as the man had left the room Pierre took up his hat
which was lying on the table and went out of his study by the
other door.  There was no one in the passage. He went along the
whole length of this passage to the stairs and, frowning and
rubbing his forehead with both hands, went down as far as the
first landing. The hall porter was standing at the front
door. From the landing where Pierre stood there was a second
staircase leading to the back entrance. He went down that
staircase and out into the yard. No one had seen him. But there
were some carriages waiting, and as soon as Pierre stepped out of
the gate the coachmen and the yard porter noticed him and raised
their caps to him. When he felt he was being looked at he behaved
like an ostrich which hides its head in a bush in order not to be
seen: he hung his head and quickening his pace went down the
street.

Of all the affairs awaiting Pierre that day the sorting of Joseph
Bazdeev's books and papers appeared to him the most necessary.

He hired the first cab he met and told the driver to go to the
Patriarch's Ponds, where the widow Bazdeev's house was.

Continually turning round to look at the rows of loaded carts
that were making their way from all sides out of Moscow, and
balancing his bulky body so as not to slip out of the ramshackle
old vehicle, Pierre, experiencing the joyful feeling of a boy
escaping from school, began to talk to his driver.

The man told him that arms were being distributed today at the
Kremlin and that tomorrow everyone would be sent out beyond the
Three Hills gates and a great battle would be fought there.

Having reached the Patriarch's Ponds Pierre found the Bazdeevs'
house, where he had not been for a long time past. He went up to
the gate.  Gerasim, that sallow beardless old man Pierre had seen
at Torzhok five years before with Joseph Bazdeev, came out in
answer to his knock.

``At home?'' asked Pierre.

``Owing to the present state of things Sophia Danilovna has gone
to the Torzhok estate with the children, your excellency.''

``I will come in all the same, I have to look through the
books,'' said Pierre.

``Be so good as to step in. Makar Alexeevich, the brother of my
late master---may the kingdom of heaven be his---has remained
here, but he is in a weak state as you know,'' said the old
servant.

Pierre knew that Makar Alexeevich was Joseph Bazdeev's
half-insane brother and a hard drinker.

``Yes, yes, I know. Let us go in...'' said Pierre and entered the
house.

A tall, bald-headed old man with a red nose, wearing a dressing
gown and with galoshes on his bare feet, stood in the
anteroom. On seeing Pierre he muttered something angrily and went
away along the passage.

``He was a very clever man but has now grown quite feeble, as
your honor sees,'' said Gerasim. ``Will you step into the
study?'' Pierre nodded. ``As it was sealed up so it has remained,
but Sophia Danilovna gave orders that if anyone should come from
you they were to have the books.''

Pierre went into that gloomy study which he had entered with such
trepidation in his benefactor's lifetime. The room, dusty and
untouched since the death of Joseph Bazdeev was now even
gloomier.

Gerasim opened one of the shutters and left the room on
tiptoe. Pierre went round the study, approached the cupboard in
which the manuscripts were kept, and took out what had once been
one of the most important, the holy of holies of the order. This
was the authentic Scotch Acts with Bazdeev's notes and
explanations. He sat down at the dusty writing table, and, having
laid the manuscripts before him, opened them out, closed them,
finally pushed them away, and resting his head on his hand sank
into meditation.

Gerasim looked cautiously into the study several times and saw
Pierre always sitting in the same attitude.

More than two hours passed and Gerasim took the liberty of making
a slight noise at the door to attract his attention, but Pierre
did not hear him.

``Is the cabman to be discharged, your honor?''

``Oh yes!'' said Pierre, rousing himself and rising
hurriedly. ``Look here,'' he added, taking Gerasim by a button of
his coat and looking down at the old man with moist, shining, and
ecstatic eyes, ``I say, do you know that there is going to be a
battle tomorrow?''

``We heard so,'' replied the man.

``I beg you not to tell anyone who I am, and to do what I ask
you.''

``Yes, your excellency,'' replied Gerasim. ``Will you have
something to eat?''

``No, but I want something else. I want peasant clothes and a
pistol,'' said Pierre, unexpectedly blushing.

``Yes, your excellency,'' said Gerasim after thinking for a
moment.

All the rest of that day Pierre spent alone in his benefactor's
study, and Gerasim heard him pacing restlessly from one corner to
another and talking to himself. And he spent the night on a bed
made up for him there.

Gerasim, being a servant who in his time had seen many strange
things, accepted Pierre's taking up his residence in the house
without surprise, and seemed pleased to have someone to wait
on. That same evening---without even asking himself what they
were wanted for---he procured a coachman's coat and cap for
Pierre, and promised to get him the pistol next day. Makar
Alexeevich came twice that evening shuffling along in his
galoshes as far as the door and stopped and looked ingratiatingly
at Pierre. But as soon as Pierre turned toward him he wrapped his
dressing gown around him with a shamefaced and angry look and
hurried away. It was when Pierre (wearing the coachman's coat
which Gerasim had procured for him and had disinfected by steam)
was on his way with the old man to buy the pistol at the Sukharev
market that he met the Rostovs.

% % % % % % % % % % % % % % % % % % % % % % % % % % % % % % % % %
% % % % % % % % % % % % % % % % % % % % % % % % % % % % % % % % %
% % % % % % % % % % % % % % % % % % % % % % % % % % % % % % % % %
% % % % % % % % % % % % % % % % % % % % % % % % % % % % % % % % %
% % % % % % % % % % % % % % % % % % % % % % % % % % % % % % % % %
% % % % % % % % % % % % % % % % % % % % % % % % % % % % % % % % %
% % % % % % % % % % % % % % % % % % % % % % % % % % % % % % % % %
% % % % % % % % % % % % % % % % % % % % % % % % % % % % % % % % %
% % % % % % % % % % % % % % % % % % % % % % % % % % % % % % % % %
% % % % % % % % % % % % % % % % % % % % % % % % % % % % % % % % %
% % % % % % % % % % % % % % % % % % % % % % % % % % % % % % % % %
% % % % % % % % % % % % % % % % % % % % % % % % % % % % % %

\chapter*{Chapter XIX} \ifaudio \marginpar{
\href{http://ia600205.us.archive.org/6/items/war_and_peace_11_0909/war_and_peace_11_19_tolstoy_64kb.mp3}{Audio}}
\fi

\initial{K}{utuzov}'s order to retreat through Moscow to the Ryazan road was
issued at night on the first of September.

The first troops started at once, and during the night they
marched slowly and steadily without hurry. At daybreak, however,
those nearing the town at the Dorogomilov bridge saw ahead of
them masses of soldiers crowding and hurrying across the bridge,
ascending on the opposite side and blocking the streets and
alleys, while endless masses of troops were bearing down on them
from behind, and an unreasoning hurry and alarm overcame
them. They all rushed forward to the bridge, onto it, and to the
fords and the boats. Kutuzov himself had driven round by side
streets to the other side of Moscow.

By ten o'clock in the morning of the second of September, only
the rear guard remained in the Dorogomilov suburb, where they had
ample room. The main army was on the other side of Moscow or
beyond it.

At that very time, at ten in the morning of the second of
September, Napoleon was standing among his troops on the Poklonny
Hill looking at the panorama spread out before him. From the
twenty-sixth of August to the second of September, that is from
the battle of Borodino to the entry of the French into Moscow,
during the whole of that agitating, memorable week, there had
been the extraordinary autumn weather that always comes as a
surprise, when the sun hangs low and gives more heat than in
spring, when everything shines so brightly in the rare clear
atmosphere that the eyes smart, when the lungs are strengthened
and refreshed by inhaling the aromatic autumn air, when even the
nights are warm, and when in those dark warm nights, golden stars
startle and delight us continually by falling from the sky.

At ten in the morning of the second of September this weather
still held.

The brightness of the morning was magical. Moscow seen from the
Poklonny Hill lay spaciously spread out with her river, her
gardens, and her churches, and she seemed to be living her usual
life, her cupolas glittering like stars in the sunlight.

The view of the strange city with its peculiar architecture, such
as he had never seen before, filled Napoleon with the rather
envious and uneasy curiosity men feel when they see an alien form
of life that has no knowledge of them. This city was evidently
living with the full force of its own life. By the indefinite
signs which, even at a distance, distinguish a living body from a
dead one, Napoleon from the Poklonny Hill perceived the throb of
life in the town and felt, as it were, the breathing of that
great and beautiful body.

Every Russian looking at Moscow feels her to be a mother; every
foreigner who sees her, even if ignorant of her significance as
the mother city, must feel her feminine character, and Napoleon
felt it.

``Cette ville asiatique aux innombrables eglises, Moscou la
sainte. La voila done enfin, cette fameuse ville! Il etait
temps,''\footnote{``That Asiatic city of the innumerable
churches, holy Moscow! Here it is then at last, that famous
city. It was high time.''} said he, and dismounting he ordered a
plan of Moscow to be spread out before him, and summoned Lelorgne
d'Ideville, the interpreter.

``A town captured by the enemy is like a maid who has lost her
honor,'' thought he (he had said so to Tuchkov at Smolensk). From
that point of view he gazed at the Oriental beauty he had not
seen before. It seemed strange to him that his long-felt wish,
which had seemed unattainable, had at last been realized. In the
clear morning light he gazed now at the city and now at the plan,
considering its details, and the assurance of possessing it
agitated and awed him.

``But could it be otherwise?'' he thought. ``Here is this capital
at my feet. Where is Alexander now, and of what is he thinking? A
strange, beautiful, and majestic city; and a strange and majestic
moment! In what light must I appear to them!'' thought he,
thinking of his troops. ``Here she is, the reward for all those
fainthearted men,'' he reflected, glancing at those near him and
at the troops who were approaching and forming up. ``One word
from me, one movement of my hand, and that ancient capital of the
Tsars would perish. But my clemency is always ready to descend
upon the vanquished. I must be magnanimous and truly great. But
no, it can't be true that I am in Moscow,'' he suddenly
thought. ``Yet here she is lying at my feet, with her golden
domes and crosses scintillating and twinkling in the
sunshine. But I shall spare her. On the ancient monuments of
barbarism and despotism I will inscribe great words of justice
and mercy... It is just this which Alexander will feel most
painfully, I know him.'' (It seemed to Napoleon that the chief
import of what was taking place lay in the personal struggle
between himself and Alexander.) ``From the height of the
Kremlin---yes, there is the Kremlin, yes---I will give them just
laws; I will teach them the meaning of true civilization, I will
make generations of boyars remember their conqueror with love. I
will tell the deputation that I did not, and do not, desire war,
that I have waged war only against the false policy of their
court; that I love and respect Alexander and that in Moscow I
will accept terms of peace worthy of myself and of my people. I
do not wish to utilize the fortunes of war to humiliate an
honored monarch. 'Boyars,' I will say to them, 'I do not desire
war, I desire the peace and welfare of all my subjects.' However,
I know their presence will inspire me, and I shall speak to them
as I always do: clearly, impressively, and majestically. But can
it be true that I am in Moscow? Yes, there she lies.''

``Qu'on m'amene les boyars,''\footnote{``Bring the boyars to
me.''}  said he to his suite.

A general with a brilliant suite galloped off at once to fetch
the boyars.

Two hours passed. Napoleon had lunched and was again standing in
the same place on the Poklonny Hill awaiting the deputation. His
speech to the boyars had already taken definite shape in his
imagination. That speech was full of dignity and greatness as
Napoleon understood it.

He was himself carried away by the tone of magnanimity he
intended to adopt toward Moscow. In his imagination he appointed
days for assemblies at the palace of the Tsars, at which Russian
notables and his own would mingle. He mentally appointed a
governor, one who would win the hearts of the people. Having
learned that there were many charitable institutions in Moscow he
mentally decided that he would shower favors on them all. He
thought that, as in Africa he had to put on a burnoose and sit in
a mosque, so in Moscow he must be beneficent like the Tsars.  And
in order finally to touch the hearts of the Russians---and being
like all Frenchmen unable to imagine anything sentimental without
a reference to ma chere, ma tendre, ma pauvre mere\footnote{``My
dear, my tender, my poor mother.''} ---he decided that he would
place an inscription on all these establishments in large
letters: ``This establishment is dedicated to my dear mother.''
Or no, it should be simply: Maison de ma Mere,\footnote{``House
of my Mother.''} he concluded. ``But am I really in Moscow? Yes,
here it lies before me, but why is the deputation from the city
so long in appearing?'' he wondered.

Meanwhile an agitated consultation was being carried on in
whispers among his generals and marshals at the rear of his
suite. Those sent to fetch the deputation had returned with the
news that Moscow was empty, that everyone had left it. The faces
of those who were not conferring together were pale and
perturbed. They were not alarmed by the fact that Moscow had been
abandoned by its inhabitants (grave as that fact seemed), but by
the question how to tell the Emperor---without putting him in the
terrible position of appearing ridiculous---that he had been
awaiting the boyars so long in vain: that there were drunken mobs
left in Moscow but no one else. Some said that a deputation of
some sort must be scraped together, others disputed that opinion
and maintained that the Emperor should first be carefully and
skillfully prepared, and then told the truth.

``He will have to be told, all the same,'' said some gentlemen of
the suite. ``But, gentlemen...''

The position was the more awkward because the Emperor, meditating
upon his magnanimous plans, was pacing patiently up and down
before the outspread map, occasionally glancing along the road to
Moscow from under his lifted hand with a bright and proud smile.

``But it's impossible...'' declared the gentlemen of the suite,
shrugging their shoulders but not venturing to utter the implied
word---le ridicule...

At last the Emperor, tired of futile expectation, his actor's
instinct suggesting to him that the sublime moment having been
too long drawn out was beginning to lose its sublimity, gave a
sign with his hand. A single report of a signaling gun followed,
and the troops, who were already spread out on different sides of
Moscow, moved into the city through Tver, Kaluga, and Dorogomilov
gates. Faster and faster, vying with one another, they moved at
the double or at a trot, vanishing amid the clouds of dust they
raised and making the air ring with a deafening roar of mingling
shouts.

Drawn on by the movement of his troops Napoleon rode with them as
far as the Dorogomilov gate, but there again stopped and,
dismounting from his horse, paced for a long time by the
Kammer-Kollezski rampart, awaiting the deputation.

% % % % % % % % % % % % % % % % % % % % % % % % % % % % % % % % %
% % % % % % % % % % % % % % % % % % % % % % % % % % % % % % % % %
% % % % % % % % % % % % % % % % % % % % % % % % % % % % % % % % %
% % % % % % % % % % % % % % % % % % % % % % % % % % % % % % % % %
% % % % % % % % % % % % % % % % % % % % % % % % % % % % % % % % %
% % % % % % % % % % % % % % % % % % % % % % % % % % % % % % % % %
% % % % % % % % % % % % % % % % % % % % % % % % % % % % % % % % %
% % % % % % % % % % % % % % % % % % % % % % % % % % % % % % % % %
% % % % % % % % % % % % % % % % % % % % % % % % % % % % % % % % %
% % % % % % % % % % % % % % % % % % % % % % % % % % % % % % % % %
% % % % % % % % % % % % % % % % % % % % % % % % % % % % % % % % %
% % % % % % % % % % % % % % % % % % % % % % % % % % % % % %

\chapter*{Chapter XX} \ifaudio \marginpar{
\href{http://ia600205.us.archive.org/6/items/war_and_peace_11_0909/war_and_peace_11_20_tolstoy_64kb.mp3}{Audio}}
\fi

\initial{M}{eanwhile} Moscow was empty. There were still people in it,
perhaps a fiftieth part of its former inhabitants had remained,
but it was empty.  It was empty in the sense that a dying
queenless hive is empty.

In a queenless hive no life is left though to a superficial
glance it seems as much alive as other hives.

The bees circle round a queenless hive in the hot beams of the
midday sun as gaily as around the living hives; from a distance
it smells of honey like the others, and bees fly in and out in
the same way. But one has only to observe that hive to realize
that there is no longer any life in it. The bees do not fly in
the same way, the smell and the sound that meet the beekeeper are
not the same. To the beekeeper's tap on the wall of the sick
hive, instead of the former instant unanimous humming of tens of
thousands of bees with their abdomens threateningly compressed,
and producing by the rapid vibration of their wings an aerial
living sound, the only reply is a disconnected buzzing from
different parts of the deserted hive. From the alighting board,
instead of the former spirituous fragrant smell of honey and
venom, and the warm whiffs of crowded life, comes an odor of
emptiness and decay mingling with the smell of honey. There are
no longer sentinels sounding the alarm with their abdomens
raised, and ready to die in defense of the hive. There is no
longer the measured quiet sound of throbbing activity, like the
sound of boiling water, but diverse discordant sounds of
disorder. In and out of the hive long black robber bees smeared
with honey fly timidly and shiftily. They do not sting, but crawl
away from danger. Formerly only bees laden with honey flew into
the hive, and they flew out empty; now they fly out laden. The
beekeeper opens the lower part of the hive and peers in. Instead
of black, glossy bees---tamed by toil, clinging to one another's
legs and drawing out the wax, with a ceaseless hum of
labor---that used to hang in long clusters down to the floor of
the hive, drowsy shriveled bees crawl about separately in various
directions on the floor and walls of the hive. Instead of a
neatly glued floor, swept by the bees with the fanning of their
wings, there is a floor littered with bits of wax, excrement,
dying bees scarcely moving their legs, and dead ones that have
not been cleared away.

The beekeeper opens the upper part of the hive and examines the
super.  Instead of serried rows of bees sealing up every gap in
the combs and keeping the brood warm, he sees the skillful
complex structures of the combs, but no longer in their former
state of purity. All is neglected and foul. Black robber bees are
swiftly and stealthily prowling about the combs, and the short
home bees, shriveled and listless as if they were old, creep
slowly about without trying to hinder the robbers, having lost
all motive and all sense of life. Drones, bumblebees, wasps, and
butterflies knock awkwardly against the walls of the hive in
their flight. Here and there among the cells containing dead
brood and honey an angry buzzing can sometimes be heard. Here and
there a couple of bees, by force of habit and custom cleaning out
the brood cells, with efforts beyond their strength laboriously
drag away a dead bee or bumblebee without knowing why they do
it. In another corner two old bees are languidly fighting, or
cleaning themselves, or feeding one another, without themselves
knowing whether they do it with friendly or hostile intent. In a
third place a crowd of bees, crushing one another, attack some
victim and fight and smother it, and the victim, enfeebled or
killed, drops from above slowly and lightly as a feather, among
the heap of corpses. The keeper opens the two center partitions
to examine the brood cells. In place of the former close dark
circles formed by thousands of bees sitting back to back and
guarding the high mystery of generation, he sees hundreds of
dull, listless, and sleepy shells of bees. They have almost all
died unawares, sitting in the sanctuary they had guarded and
which is now no more. They reek of decay and death. Only a few of
them still move, rise, and feebly fly to settle on the enemy's
hand, lacking the spirit to die stinging him; the rest are dead
and fall as lightly as fish scales. The beekeeper closes the
hive, chalks a mark on it, and when he has time tears out its
contents and burns it clean.

So in the same way Moscow was empty when Napoleon, weary, uneasy,
and morose, paced up and down in front of the Kammer-Kollezski
rampart, awaiting what to his mind was a necessary, if but
formal, observance of the proprieties---a deputation.

In various corners of Moscow there still remained a few people
aimlessly moving about, following their old habits and hardly
aware of what they were doing.

When with due circumspection Napoleon was informed that Moscow
was empty, he looked angrily at his informant, turned away, and
silently continued to walk to and fro.

``My carriage!'' he said.

He took his seat beside the aide-de-camp on duty and drove into
the suburb. ``Moscow deserted!'' he said to himself. ``What an
incredible event!''

He did not drive into the town, but put up at an inn in the
Dorogomilov suburb.

The coup de theatre had not come off.

% % % % % % % % % % % % % % % % % % % % % % % % % % % % % % % % %
% % % % % % % % % % % % % % % % % % % % % % % % % % % % % % % % %
% % % % % % % % % % % % % % % % % % % % % % % % % % % % % % % % %
% % % % % % % % % % % % % % % % % % % % % % % % % % % % % % % % %
% % % % % % % % % % % % % % % % % % % % % % % % % % % % % % % % %
% % % % % % % % % % % % % % % % % % % % % % % % % % % % % % % % %
% % % % % % % % % % % % % % % % % % % % % % % % % % % % % % % % %
% % % % % % % % % % % % % % % % % % % % % % % % % % % % % % % % %
% % % % % % % % % % % % % % % % % % % % % % % % % % % % % % % % %
% % % % % % % % % % % % % % % % % % % % % % % % % % % % % % % % %
% % % % % % % % % % % % % % % % % % % % % % % % % % % % % % % % %
% % % % % % % % % % % % % % % % % % % % % % % % % % % % % %

\chapter*{Chapter XXI} \ifaudio \marginpar{
\href{http://ia600205.us.archive.org/6/items/war_and_peace_11_0909/war_and_peace_11_21_tolstoy_64kb.mp3}{Audio}}
\fi

\initial{T}{he} Russian troops were passing through Moscow from two o'clock
at night till two in the afternoon and bore away with them the
wounded and the last of the inhabitants who were leaving.

The greatest crush during the movement of the troops took place
at the Stone, Moskva, and Yauza bridges.

While the troops, dividing into two parts when passing around the
Kremlin, were thronging the Moskva and the Stone bridges, a great
many soldiers, taking advantage of the stoppage and congestion,
turned back from the bridges and slipped stealthily and silently
past the church of Vasili the Beatified and under the Borovitski
gate, back up the hill to the Red Square where some instinct told
them they could easily take things not belonging to them. Crowds
of the kind seen at cheap sales filled all the passages and
alleys of the Bazaar. But there were no dealers with voices of
ingratiating affability inviting customers to enter; there were
no hawkers, nor the usual motley crowd of female purchasers---but
only soldiers, in uniforms and overcoats though without muskets,
entering the Bazaar empty-handed and silently making their way
out through its passages with bundles. Tradesmen and their
assistants (of whom there were but few) moved about among the
soldiers quite bewildered. They unlocked their shops and locked
them up again, and themselves carried goods away with the help of
their assistants. On the square in front of the Bazaar were
drummers beating the muster call. But the roll of the drums did
not make the looting soldiers run in the direction of the drum as
formerly, but made them, on the contrary, run farther away. Among
the soldiers in the shops and passages some men were to be seen
in gray coats, with closely shaven heads. Two officers, one with
a scarf over his uniform and mounted on a lean, dark-gray horse,
the other in an overcoat and on foot, stood at the corner of
Ilyinka Street, talking. A third officer galloped up to them.

``The general orders them all to be driven out at once, without
fail.  This is outrageous! Half the men have dispersed.''

``Where are you off to?... Where?...'' he shouted to three
infantrymen without muskets who, holding up the skirts of their
overcoats, were slipping past him into the Bazaar
passage. ``Stop, you rascals!''

``But how are you going to stop them?'' replied another
officer. ``There is no getting them together. The army should
push on before the rest bolt, that's all!''

``How can one push on? They are stuck there, wedged on the
bridge, and don't move. Shouldn't we put a cordon round to
prevent the rest from running away?''

``Come, go in there and drive them out!'' shouted the senior
officer.

The officer in the scarf dismounted, called up a drummer, and
went with him into the arcade. Some soldiers started running away
in a group. A shopkeeper with red pimples on his cheeks near the
nose, and a calm, persistent, calculating expression on his plump
face, hurriedly and ostentatiously approached the officer,
swinging his arms.

``Your honor!'' said he. ``Be so good as to protect us! We won't
grudge trifles, you are welcome to anything---we shall be
delighted! Pray!...  I'll fetch a piece of cloth at once for such
an honorable gentleman, or even two pieces with pleasure. For we
feel how it is; but what's all this---sheer robbery! If you
please, could not guards be placed if only to let us close the
shop...''

Several shopkeepers crowded round the officer.

``Eh, what twaddle!'' said one of them, a thin, stern-looking
man. ``When one's head is gone one doesn't weep for one's hair!
Take what any of you like!'' And flourishing his arm
energetically he turned sideways to the officer.

``It's all very well for you, Ivan Sidorych, to talk,'' said the
first tradesman angrily. ``Please step inside, your honor!''

``Talk indeed!'' cried the thin one. ``In my three shops here I
have a hundred thousand rubles' worth of goods. Can they be saved
when the army has gone? Eh, what people! 'Against God's might our
hands can't fight.'''

``Come inside, your honor!'' repeated the tradesman, bowing.

The officer stood perplexed and his face showed indecision.

``It's not my business!'' he exclaimed, and strode on quickly
down one of the passages.

From one open shop came the sound of blows and vituperation, and
just as the officer came up to it a man in a gray coat with a
shaven head was flung out violently.

This man, bent double, rushed past the tradesman and the
officer. The officer pounced on the soldiers who were in the
shops, but at that moment fearful screams reached them from the
huge crowd on the Moskva bridge and the officer ran out into the
square.

``What is it? What is it?'' he asked, but his comrade was already
galloping off past Vasili the Beatified in the direction from
which the screams came.

The officer mounted his horse and rode after him. When he reached
the bridge he saw two unlimbered guns, the infantry crossing the
bridge, several overturned carts, and frightened and laughing
faces among the troops. Beside the cannon a cart was standing to
which two horses were harnessed. Four borzois with collars were
pressing close to the wheels.  The cart was loaded high, and at
the very top, beside a child's chair with its legs in the air,
sat a peasant woman uttering piercing and desperate shrieks. He
was told by his fellow officers that the screams of the crowd and
the shrieks of the woman were due to the fact that General
Ermolov, coming up to the crowd and learning that soldiers were
dispersing among the shops while crowds of civilians blocked the
bridge, had ordered two guns to be unlimbered and made a show of
firing at the bridge. The crowd, crushing one another, upsetting
carts, and shouting and squeezing desperately, had cleared off
the bridge and the troops were now moving forward.

% % % % % % % % % % % % % % % % % % % % % % % % % % % % % % % % %
% % % % % % % % % % % % % % % % % % % % % % % % % % % % % % % % %
% % % % % % % % % % % % % % % % % % % % % % % % % % % % % % % % %
% % % % % % % % % % % % % % % % % % % % % % % % % % % % % % % % %
% % % % % % % % % % % % % % % % % % % % % % % % % % % % % % % % %
% % % % % % % % % % % % % % % % % % % % % % % % % % % % % % % % %
% % % % % % % % % % % % % % % % % % % % % % % % % % % % % % % % %
% % % % % % % % % % % % % % % % % % % % % % % % % % % % % % % % %
% % % % % % % % % % % % % % % % % % % % % % % % % % % % % % % % %
% % % % % % % % % % % % % % % % % % % % % % % % % % % % % % % % %
% % % % % % % % % % % % % % % % % % % % % % % % % % % % % % % % %
% % % % % % % % % % % % % % % % % % % % % % % % % % % % % %

\chapter*{Chapter XXII} \ifaudio \marginpar{
\href{http://ia600205.us.archive.org/6/items/war_and_peace_11_0909/war_and_peace_11_22_tolstoy_64kb.mp3}{Audio}}
\fi

\initial{M}{eanwhile}, the city itself was deserted. There was hardly anyone
in the streets. The gates and shops were all closed, only here
and there round the taverns solitary shouts or drunken songs
could be heard. Nobody drove through the streets and footsteps
were rarely heard. The Povarskaya was quite still and
deserted. The huge courtyard of the Rostovs' house was littered
with wisps of hay and with dung from the horses, and not a soul
was to be seen there. In the great drawing room of the house,
which had been left with all it contained, were two people. They
were the yard porter Ignat, and the page boy Mishka, Vasilich's
grandson who had stayed in Moscow with his grandfather.  Mishka
had opened the clavichord and was strumming on it with one
finger. The yard porter, his arms akimbo, stood smiling with
satisfaction before the large mirror.

``Isn't it fine, eh, Uncle Ignat?'' said the boy, suddenly
beginning to strike the keyboard with both hands.

``Only fancy!'' answered Ignat, surprised at the broadening grin
on his face in the mirror.

``Impudence! Impudence!'' they heard behind them the voice of
Mavra Kuzminichna who had entered silently. ``How he's grinning,
the fat mug!  Is that what you're here for? Nothing's cleared
away down there and Vasilich is worn out. Just you wait a bit!''

Ignat left off smiling, adjusted his belt, and went out of the
room with meekly downcast eyes.

``Aunt, I did it gently,'' said the boy.

``I'll give you something gently, you monkey you!'' cried Mavra
Kuzminichna, raising her arm threateningly. ``Go and get the
samovar to boil for your grandfather.''

Mavra Kuzminichna flicked the dust off the clavichord and closed
it, and with a deep sigh left the drawing room and locked its
main door.

Going out into the yard she paused to consider where she should
go next---to drink tea in the servants' wing with Vasilich, or
into the storeroom to put away what still lay about.

She heard the sound of quick footsteps in the quiet
street. Someone stopped at the gate, and the latch rattled as
someone tried to open it.  Mavra Kuzminichna went to the gate.

``Who do you want?''

``The count---Count Ilya Andreevich Rostov.''

``And who are you?''

``An officer, I have to see him,'' came the reply in a pleasant,
well-bred Russian voice.

Mavra Kuzminichna opened the gate and an officer of eighteen,
with the round face of a Rostov, entered the yard.

``They have gone away, sir. Went away yesterday at vespertime,''
said Mavra Kuzminichna cordially.

The young officer standing in the gateway, as if hesitating
whether to enter or not, clicked his tongue.

``Ah, how annoying!'' he muttered. ``I should have come
yesterday... Ah, what a pity.''

Meanwhile, Mavra Kuzminichna was attentively and sympathetically
examining the familiar Rostov features of the young man's face,
his tattered coat and trodden-down boots.

``What did you want to see the count for?'' she asked.

``Oh well... it can't be helped!'' said he in a tone of vexation
and placed his hand on the gate as if to leave.

He again paused in indecision.

``You see,'' he suddenly said, ``I am a kinsman of the count's
and he has been very kind to me. As you see'' (he glanced with an
amused air and good-natured smile at his coat and boots) ``my
things are worn out and I have no money, so I was going to ask
the count...''

Mavra Kuzminichna did not let him finish.

``Just wait a minute, sir. One little moment,'' said she.

And as soon as the officer let go of the gate handle she turned
and, hurrying away on her old legs, went through the back yard to
the servants' quarters.

While Mavra Kuzminichna was running to her room the officer
walked about the yard gazing at his worn-out boots with lowered
head and a faint smile on his lips. ``What a pity I've missed
Uncle! What a nice old woman! Where has she run off to? And how
am I to find the nearest way to overtake my regiment, which must
by now be getting near the Rogozhski gate?'' thought he. Just
then Mavra Kuzminichna appeared from behind the corner of the
house with a frightened yet resolute look, carrying a rolled-up
check kerchief in her hand. While still a few steps from the
officer she unfolded the kerchief and took out of it a
white-twenty-five-ruble assignat and hastily handed it to him.

``If his excellency had been at home, as a kinsman he would of
course...  but as it is...''

Mavra Kuzminichna grew abashed and confused. The officer did not
decline, but took the note quietly and thanked her.

``If the count had been at home...'' Mavra Kuzminichna went on
apologetically. ``Christ be with you, sir! May God preserve
you!'' said she, bowing as she saw him out.

Swaying his head and smiling as if amused at himself, the officer
ran almost at a trot through the deserted streets toward the
Yauza bridge to overtake his regiment.

But Mavra Kuzminichna stood at the closed gate for some time with
moist eyes, pensively swaying her head and feeling an unexpected
flow of motherly tenderness and pity for the unknown young
officer.

% % % % % % % % % % % % % % % % % % % % % % % % % % % % % % % % %
% % % % % % % % % % % % % % % % % % % % % % % % % % % % % % % % %
% % % % % % % % % % % % % % % % % % % % % % % % % % % % % % % % %
% % % % % % % % % % % % % % % % % % % % % % % % % % % % % % % % %
% % % % % % % % % % % % % % % % % % % % % % % % % % % % % % % % %
% % % % % % % % % % % % % % % % % % % % % % % % % % % % % % % % %
% % % % % % % % % % % % % % % % % % % % % % % % % % % % % % % % %
% % % % % % % % % % % % % % % % % % % % % % % % % % % % % % % % %
% % % % % % % % % % % % % % % % % % % % % % % % % % % % % % % % %
% % % % % % % % % % % % % % % % % % % % % % % % % % % % % % % % %
% % % % % % % % % % % % % % % % % % % % % % % % % % % % % % % % %
% % % % % % % % % % % % % % % % % % % % % % % % % % % % % %

\chapter*{Chapter XXIII} \ifaudio \marginpar{
\href{http://ia600205.us.archive.org/6/items/war_and_peace_11_0909/war_and_peace_11_23_tolstoy_64kb.mp3}{Audio}}
\fi

\initial{F}{rom} an unfinished house on the Varvarka, the ground floor of
which was a dramshop, came drunken shouts and songs. On benches
round the tables in a dirty little room sat some ten factory
hands. Tipsy and perspiring, with dim eyes and wide-open mouths,
they were all laboriously singing some song or other. They were
singing discordantly, arduously, and with great effort, evidently
not because they wished to sing, but because they wanted to show
they were drunk and on a spree. One, a tall, fair-haired lad in a
clean blue coat, was standing over the others. His face with its
fine straight nose would have been handsome had it not been for
his thin, compressed, twitching lips and dull, gloomy, fixed
eyes.  Evidently possessed by some idea, he stood over those who
were singing, and solemnly and jerkily flourished above their
heads his white arm with the sleeve turned up to the elbow,
trying unnaturally to spread out his dirty fingers. The sleeve of
his coat kept slipping down and he always carefully rolled it up
again with his left hand, as if it were most important that the
sinewy white arm he was flourishing should be bare.  In the midst
of the song cries were heard, and fighting and blows in the
passage and porch. The tall lad waved his arm.

``Stop it!'' he exclaimed peremptorily. ``There's a fight,
lads!'' And, still rolling up his sleeve, he went out to the
porch.

The factory hands followed him. These men, who under the
leadership of the tall lad were drinking in the dramshop that
morning, had brought the publican some skins from the factory and
for this had had drink served them. The blacksmiths from a
neighboring smithy, hearing the sounds of revelry in the tavern
and supposing it to have been broken into, wished to force their
way in too and a fight in the porch had resulted.

The publican was fighting one of the smiths at the door, and when
the workmen came out the smith, wrenching himself free from the
tavern keeper, fell face downward on the pavement.

Another smith tried to enter the doorway, pressing against the
publican with his chest.

The lad with the turned-up sleeve gave the smith a blow in the
face and cried wildly: ``They're fighting us, lads!''

At that moment the first smith got up and, scratching his bruised
face to make it bleed, shouted in a tearful voice: ``Police!
Murder!...  They've killed a man, lads!''

``Oh, gracious me, a man beaten to death---killed!...'' screamed
a woman coming out of a gate close by.

A crowd gathered round the bloodstained smith.

``Haven't you robbed people enough---taking their last shirts?''
said a voice addressing the publican. ``What have you killed a
man for, you thief?''

The tall lad, standing in the porch, turned his bleared eyes from
the publican to the smith and back again as if considering whom
he ought to fight now.

``Murderer!'' he shouted suddenly to the publican. ``Bind him,
lads!''

``I daresay you would like to bind me!'' shouted the publican,
pushing away the men advancing on him, and snatching his cap from
his head he flung it on the ground.

As if this action had some mysterious and menacing significance,
the workmen surrounding the publican paused in indecision.

``I know the law very well, mates! I'll take the matter to the
captain of police. You think I won't get to him? Robbery is not
permitted to anybody now a days!'' shouted the publican, picking
up his cap.

``Come along then! Come along then!'' the publican and the tall
young fellow repeated one after the other, and they moved up the
street together.

The bloodstained smith went beside them. The factory hands and
others followed behind, talking and shouting.

At the corner of the Moroseyka, opposite a large house with
closed shutters and bearing a bootmaker's signboard, stood a
score of thin, worn-out, gloomy-faced bootmakers, wearing
overalls and long tattered coats.

``He should pay folks off properly,'' a thin workingman, with
frowning brows and a straggly beard, was saying.

``But he's sucked our blood and now he thinks he's quit of
us. He's been misleading us all the week and now that he's
brought us to this pass he's made off.''

On seeing the crowd and the bloodstained man the workman ceased
speaking, and with eager curiosity all the bootmakers joined the
moving crowd.

``Where are all the folks going?''

``Why, to the police, of course!''

``I say, is it true that we have been beaten?'' ``And what did
you think?  Look what folks are saying.''

Questions and answers were heard. The publican, taking advantage
of the increased crowd, dropped behind and returned to his
tavern.

The tall youth, not noticing the disappearance of his foe, waved
his bare arm and went on talking incessantly, attracting general
attention to himself. It was around him that the people chiefly
crowded, expecting answers from him to the questions that
occupied all their minds.

``He must keep order, keep the law, that's what the government is
there for. Am I not right, good Christians?'' said the tall
youth, with a scarcely perceptible smile. ``He thinks there's no
government! How can one do without government? Or else there
would be plenty who'd rob us.''

``Why talk nonsense?'' rejoined voices in the crowd. ``Will they
give up Moscow like this? They told you that for fun, and you
believed it!  Aren't there plenty of troops on the march? Let him
in, indeed! That's what the government is for. You'd better
listen to what people are saying,'' said some of the mob pointing
to the tall youth.

By the wall of China-Town a smaller group of people were gathered
round a man in a frieze coat who held a paper in his hand.

``An ukase, they are reading an ukase! Reading an ukase!'' cried
voices in the crowd, and the people rushed toward the reader.

The man in the frieze coat was reading the broadsheet of August
31. When the crowd collected round him he seemed confused, but at
the demand of the tall lad who had pushed his way up to him, he
began in a rather tremulous voice to read the sheet from the
beginning.

``Early tomorrow I shall go to his Serene Highness,'' he read
(``Sirin Highness,'' said the tall fellow with a triumphant smile
on his lips and a frown on his brow), ``to consult with him to
act, and to aid the army to exterminate these scoundrels. We too
will take part...'' the reader went on, and then paused (``Do you
see,'' shouted the youth victoriously, ``he's going to clear up
the whole affair for you...''), ``in destroying them, and will
send these visitors to the devil. I will come back to dinner, and
we'll set to work. We will do, completely do, and undo these
scoundrels.''

The last words were read out in the midst of complete
silence. The tall lad hung his head gloomily. It was evident that
no one had understood the last part. In particular, the words ``I
will come back to dinner,'' evidently displeased both reader and
audience. The people's minds were tuned to a high pitch and this
was too simple and needlessly comprehensible---it was what any
one of them might have said and therefore was what an ukase
emanating from the highest authority should not say.

They all stood despondent and silent. The tall youth moved his
lips and swayed from side to side.

``We should ask him... that's he himself?''... ``Yes, ask him
indeed!...  Why not? He'll explain''... voices in the rear of the
crowd were suddenly heard saying, and the general attention
turned to the police superintendent's trap which drove into the
square attended by two mounted dragoons.

The superintendent of police, who had gone that morning by Count
Rostopchin's orders to burn the barges and had in connection with
that matter acquired a large sum of money which was at that
moment in his pocket, on seeing a crowd bearing down upon him
told his coachman to stop.

``What people are these?'' he shouted to the men, who were moving
singly and timidly in the direction of his trap.

``What people are these?'' he shouted again, receiving no answer.

``Your honor...'' replied the shopman in the frieze coat, ``your
honor, in accord with the proclamation of his highest excellency
the count, they desire to serve, not sparing their lives, and it
is not any kind of riot, but as his highest excellence said...''

``The count has not left, he is here, and an order will be issued
concerning you,'' said the superintendent of police. ``Go on!''
he ordered his coachman.

The crowd halted, pressing around those who had heard what the
superintendent had said, and looking at the departing trap.

The superintendent of police turned round at that moment with a
scared look, said something to his coachman, and his horses
increased their speed.

``It's a fraud, lads! Lead the way to him, himself!'' shouted the
tall youth. ``Don't let him go, lads! Let him answer us! Keep
him!'' shouted different people and the people dashed in pursuit
of the trap.

Following the superintendent of police and talking loudly the
crowd went in the direction of the Lubyanka Street.

``There now, the gentry and merchants have gone away and left us
to perish. Do they think we're dogs?'' voices in the crowd were
heard saying more and more frequently.

% % % % % % % % % % % % % % % % % % % % % % % % % % % % % % % % %
% % % % % % % % % % % % % % % % % % % % % % % % % % % % % % % % %
% % % % % % % % % % % % % % % % % % % % % % % % % % % % % % % % %
% % % % % % % % % % % % % % % % % % % % % % % % % % % % % % % % %
% % % % % % % % % % % % % % % % % % % % % % % % % % % % % % % % %
% % % % % % % % % % % % % % % % % % % % % % % % % % % % % % % % %
% % % % % % % % % % % % % % % % % % % % % % % % % % % % % % % % %
% % % % % % % % % % % % % % % % % % % % % % % % % % % % % % % % %
% % % % % % % % % % % % % % % % % % % % % % % % % % % % % % % % %
% % % % % % % % % % % % % % % % % % % % % % % % % % % % % % % % %
% % % % % % % % % % % % % % % % % % % % % % % % % % % % % % % % %
% % % % % % % % % % % % % % % % % % % % % % % % % % % % % %

\chapter*{Chapter XXIV} \ifaudio \marginpar{
\href{http://ia600205.us.archive.org/6/items/war_and_peace_11_0909/war_and_peace_11_24_tolstoy_64kb.mp3}{Audio}}
\fi

\initial{O}{n} the evening of the first of September, after his interview
with Kutuzov, Count Rostopchin had returned to Moscow mortified
and offended because he had not been invited to attend the
council of war, and because Kutuzov had paid no attention to his
offer to take part in the defense of the city; amazed also at the
novel outlook revealed to him at the camp, which treated the
tranquillity of the capital and its patriotic fervor as not
merely secondary but quite irrelevant and unimportant
matters. Distressed, offended, and surprised by all this,
Rostopchin had returned to Moscow. After supper he lay down on a
sofa without undressing, and was awakened soon after midnight by
a courier bringing him a letter from Kutuzov. This letter
requested the count to send police officers to guide the troops
through the town, as the army was retreating to the Ryazan road
beyond Moscow. This was not news to Rostopchin. He had known that
Moscow would be abandoned not merely since his interview the
previous day with Kutuzov on the Poklonny Hill but ever since the
battle of Borodino, for all the generals who came to Moscow after
that battle had said unanimously that it was impossible to fight
another battle, and since then the government property had been
removed every night, and half the inhabitants had left the city
with Rostopchin's own permission. Yet all the same this
information astonished and irritated the count, coming as it did
in the form of a simple note with an order from Kutuzov, and
received at night, breaking in on his beauty sleep.

When later on in his memoirs Count Rostopchin explained his
actions at this time, he repeatedly says that he was then
actuated by two important considerations: to maintain
tranquillity in Moscow and expedite the departure of the
inhabitants. If one accepts this twofold aim all Rostopchin's
actions appear irreproachable. ``Why were the holy relics, the
arms, ammunition, gunpowder, and stores of corn not removed? Why
were thousands of inhabitants deceived into believing that Moscow
would not be given up---and thereby ruined?'' ``To preserve the
tranquillity of the city,'' explains Count Rostopchin. ``Why were
bundles of useless papers from the government offices, and
Leppich's balloon and other articles removed?'' ``To leave the
town empty,'' explains Count Rostopchin.  One need only admit
that public tranquillity is in danger and any action finds a
justification.

All the horrors of the reign of terror were based only on
solicitude for public tranquillity.

On what, then, was Count Rostopchin's fear for the tranquillity
of Moscow based in 1812? What reason was there for assuming any
probability of an uprising in the city? The inhabitants were
leaving it and the retreating troops were filling it. Why should
that cause the masses to riot?

Neither in Moscow nor anywhere in Russia did anything resembling
an insurrection ever occur when the enemy entered a town. More
than ten thousand people were still in Moscow on the first and
second of September, and except for a mob in the governor's
courtyard, assembled there at his bidding, nothing happened. It
is obvious that there would have been even less reason to expect
a disturbance among the people if after the battle of Borodino,
when the surrender of Moscow became certain or at least probable,
Rostopchin instead of exciting the people by distributing arms
and broadsheets had taken steps to remove all the holy relics,
the gunpowder, munitions, and money, and had told the population
plainly that the town would be abandoned.

Rostopchin, though he had patriotic sentiments, was a sanguine
and impulsive man who had always moved in the highest
administrative circles and had no understanding at all of the
people he supposed himself to be guiding. Ever since the enemy's
entry into Smolensk he had in imagination been playing the role
of director of the popular feeling of \emph{the heart of Russia}.
Not only did it seem to him (as to all administrators) that he
controlled the external actions of Moscow's inhabitants, but he
also thought he controlled their mental attitude by means of his
broadsheets and posters, written in a coarse tone which the
people despise in their own class and do not understand from
those in authority. Rostopchin was so pleased with the fine role
of leader of popular feeling, and had grown so used to it, that
the necessity of relinquishing that role and abandoning Moscow
without any heroic display took him unawares and he suddenly felt
the ground slip away from under his feet, so that he positively
did not know what to do. Though he knew it was coming, he did not
till the last moment wholeheartedly believe that Moscow would be
abandoned, and did not prepare for it. The inhabitants left
against his wishes. If the government offices were removed, this
was only done on the demand of officials to whom the count
yielded reluctantly. He was absorbed in the role he had created
for himself. As is often the case with those gifted with an
ardent imagination, though he had long known that Moscow would be
abandoned he knew it only with his intellect, he did not believe
it in his heart and did not adapt himself mentally to this new
position of affairs.

All his painstaking and energetic activity (in how far it was
useful and had any effect on the people is another question) had
been simply directed toward arousing in the masses his own
feeling of patriotic hatred of the French.

But when events assumed their true historical character, when
expressing hatred for the French in words proved insufficient,
when it was not even possible to express that hatred by fighting
a battle, when self-confidence was of no avail in relation to the
one question before Moscow, when the whole population streamed
out of Moscow as one man, abandoning their belongings and proving
by that negative action all the depth of their national feeling,
then the role chosen by Rostopchin suddenly appeared
senseless. He unexpectedly felt himself ridiculous, weak, and
alone, with no ground to stand on.

When, awakened from his sleep, he received that cold, peremptory
note from Kutuzov, he felt the more irritated the more he felt
himself to blame. All that he had been specially put in charge
of, the state property which he should have removed, was still in
Moscow and it was no longer possible to take the whole of it
away.

``Who is to blame for it? Who has let things come to such a
pass?'' he ruminated. ``Not I, of course. I had everything
ready. I had Moscow firmly in hand. And this is what they have
let it come to! Villains!  Traitors!'' he thought, without
clearly defining who the villains and traitors were, but feeling
it necessary to hate those traitors whoever they might be who
were to blame for the false and ridiculous position in which he
found himself.

All that night Count Rostopchin issued orders, for which people
came to him from all parts of Moscow. Those about him had never
seen the count so morose and irritable.

``Your excellency, the Director of the Registrar's Department has
sent for instructions... From the Consistory, from the Senate,
from the University, from the Foundling Hospital, the Suffragan
has sent...  asking for information... What are your orders about
the Fire Brigade?  From the governor of the prison... from the
superintendent of the lunatic asylum...'' All night long such
announcements were continually being received by the count.

To all these inquiries he gave brief and angry replies indicating
that orders from him were not now needed, that the whole affair,
carefully prepared by him, had now been ruined by somebody, and
that that somebody would have to bear the whole responsibility
for all that might happen.

``Oh, tell that blockhead,'' he said in reply to the question
from the Registrar's Department, ``that he should remain to guard
his documents.  Now why are you asking silly questions about the
Fire Brigade? They have horses, let them be off to Vladimir, and
not leave them to the French.''

``Your excellency, the superintendent of the lunatic asylum has
come: what are your commands?''

``My commands? Let them go away, that's all... And let the
lunatics out into the town. When lunatics command our armies God
evidently means these other madmen to be free.''

In reply to an inquiry about the convicts in the prison, Count
Rostopchin shouted angrily at the governor:

``Do you expect me to give you two battalions---which we have not
got---for a convoy? Release them, that's all about it!''

``Your excellency, there are some political prisoners, Meshkov,
Vereshchagin...''

``Vereshchagin! Hasn't he been hanged yet?'' shouted
Rostopchin. ``Bring him to me!''

% % % % % % % % % % % % % % % % % % % % % % % % % % % % % % % % %
% % % % % % % % % % % % % % % % % % % % % % % % % % % % % % % % %
% % % % % % % % % % % % % % % % % % % % % % % % % % % % % % % % %
% % % % % % % % % % % % % % % % % % % % % % % % % % % % % % % % %
% % % % % % % % % % % % % % % % % % % % % % % % % % % % % % % % %
% % % % % % % % % % % % % % % % % % % % % % % % % % % % % % % % %
% % % % % % % % % % % % % % % % % % % % % % % % % % % % % % % % %
% % % % % % % % % % % % % % % % % % % % % % % % % % % % % % % % %
% % % % % % % % % % % % % % % % % % % % % % % % % % % % % % % % %
% % % % % % % % % % % % % % % % % % % % % % % % % % % % % % % % %
% % % % % % % % % % % % % % % % % % % % % % % % % % % % % % % % %
% % % % % % % % % % % % % % % % % % % % % % % % % % % % % %

\chapter*{Chapter XXV} \ifaudio \marginpar{
\href{http://ia600205.us.archive.org/6/items/war_and_peace_11_0909/war_and_peace_11_25_tolstoy_64kb.mp3}{Audio}}
\fi

\initial{T}{oward} nine o'clock in the morning, when the troops were already
moving through Moscow, nobody came to the count any more for
instructions.  Those who were able to get away were going of
their own accord, those who remained behind decided for
themselves what they must do.

The count ordered his carriage that he might drive to Sokolniki,
and sat in his study with folded hands, morose, sallow, and
taciturn.

In quiet and untroubled times it seems to every administrator
that it is only by his efforts that the whole population under
his rule is kept going, and in this consciousness of being
indispensable every administrator finds the chief reward of his
labor and efforts. While the sea of history remains calm the
ruler-administrator in his frail bark, holding on with a boat
hook to the ship of the people and himself moving, naturally
imagines that his efforts move the ship he is holding on to. But
as soon as a storm arises and the sea begins to heave and the
ship to move, such a delusion is no longer possible. The ship
moves independently with its own enormous motion, the boat hook
no longer reaches the moving vessel, and suddenly the
administrator, instead of appearing a ruler and a source of
power, becomes an insignificant, useless, feeble man.

Rostopchin felt this, and it was this which exasperated him.

The superintendent of police, whom the crowd had stopped, went in
to see him at the same time as an adjutant who informed the count
that the horses were harnessed. They were both pale, and the
superintendent of police, after reporting that he had executed
the instructions he had received, informed the count that an
immense crowd had collected in the courtyard and wished to see
him.

Without saying a word Rostopchin rose and walked hastily to his
light, luxurious drawing room, went to the balcony door, took
hold of the handle, let it go again, and went to the window from
which he had a better view of the whole crowd. The tall lad was
standing in front, flourishing his arm and saying something with
a stern look. The blood-stained smith stood beside him with a
gloomy face. A drone of voices was audible through the closed
window.

``Is my carriage ready?'' asked Rostopchin, stepping back from
the window.

``It is, your excellency,'' replied the adjutant.

Rostopchin went again to the balcony door.

``But what do they want?'' he asked the superintendent of police.

``Your excellency, they say they have got ready, according to
your orders, to go against the French, and they shouted something
about treachery. But it is a turbulent crowd, your excellency---I
hardly managed to get away from it. Your excellency, I venture to
suggest...''

``You may go. I don't need you to tell me what to do!'' exclaimed
Rostopchin angrily.

He stood by the balcony door looking at the crowd.

``This is what they have done with Russia! This is what they have
done with me!'' thought he, full of an irrepressible fury that
welled up within him against the someone to whom what was
happening might be attributed. As often happens with passionate
people, he was mastered by anger but was still seeking an object
on which to vent it. ``Here is that mob, the dregs of the
people,'' he thought as he gazed at the crowd: ``this rabble they
have roused by their folly! They want a victim,'' he thought as
he looked at the tall lad flourishing his arm. And this thought
occurred to him just because he himself desired a victim,
something on which to vent his rage.

``Is the carriage ready?'' he asked again.

``Yes, your excellency. What are your orders about Vereshchagin?
He is waiting at the porch,'' said the adjutant.

``Ah!'' exclaimed Rostopchin, as if struck by an unexpected
recollection.

And rapidly opening the door he went resolutely out onto the
balcony.  The talking instantly ceased, hats and caps were
doffed, and all eyes were raised to the count.

``Good morning, lads!'' said the count briskly and
loudly. ``Thank you for coming. I'll come out to you in a moment,
but we must first settle with the villain. We must punish the
villain who has caused the ruin of Moscow. Wait for me!''

And the count stepped as briskly back into the room and slammed
the door behind him.

A murmur of approbation and satisfaction ran through the
crowd. ``He'll settle with all the villains, you'll see! And you
said the French...  He'll show you what law is!'' the mob were
saying as if reproving one another for their lack of confidence.

A few minutes later an officer came hurriedly out of the front
door, gave an order, and the dragoons formed up in line. The
crowd moved eagerly from the balcony toward the
porch. Rostopchin, coming out there with quick angry steps,
looked hastily around as if seeking someone.

``Where is he?'' he inquired. And as he spoke he saw a young man
coming round the corner of the house between two dragoons. He had
a long thin neck, and his head, that had been half shaved, was
again covered by short hair. This young man was dressed in a
threadbare blue cloth coat lined with fox fur, that had once been
smart, and dirty hempen convict trousers, over which were pulled
his thin, dirty, trodden-down boots. On his thin, weak legs were
heavy chains which hampered his irresolute movements.

``Ah!'' said Rostopchin, hurriedly turning away his eyes from the
young man in the fur-lined coat and pointing to the bottom step
of the porch.  ``Put him there.''

The young man in his clattering chains stepped clumsily to the
spot indicated, holding away with one finger the coat collar
which chafed his neck, turned his long neck twice this way and
that, sighed, and submissively folded before him his thin hands,
unused to work.

For several seconds while the young man was taking his place on
the step the silence continued. Only among the back rows of the
people, who were all pressing toward the one spot, could sighs,
groans, and the shuffling of feet be heard.

While waiting for the young man to take his place on the step
Rostopchin stood frowning and rubbing his face with his hand.

``Lads!'' said he, with a metallic ring in his voice. ``This man,
Vereshchagin, is the scoundrel by whose doing Moscow is
perishing.''

The young man in the fur-lined coat, stooping a little, stood in
a submissive attitude, his fingers clasped before him. His
emaciated young face, disfigured by the half-shaven head, hung
down hopelessly. At the count's first words he raised it slowly
and looked up at him as if wishing to say something or at least
to meet his eye. But Rostopchin did not look at him. A vein in
the young man's long thin neck swelled like a cord and went blue
behind the ear, and suddenly his face flushed.

All eyes were fixed on him. He looked at the crowd, and rendered
more hopeful by the expression he read on the faces there, he
smiled sadly and timidly, and lowering his head shifted his feet
on the step.

``He has betrayed his Tsar and his country, he has gone over to
Bonaparte. He alone of all the Russians has disgraced the Russian
name, he has caused Moscow to perish,'' said Rostopchin in a
sharp, even voice, but suddenly he glanced down at Vereshchagin
who continued to stand in the same submissive attitude. As if
inflamed by the sight, he raised his arm and addressed the
people, almost shouting:

``Deal with him as you think fit! I hand him over to you.''

The crowd remained silent and only pressed closer and closer to
one another. To keep one another back, to breathe in that
stifling atmosphere, to be unable to stir, and to await something
unknown, uncomprehended, and terrible, was becoming
unbearable. Those standing in front, who had seen and heard what
had taken place before them, all stood with wide-open eyes and
mouths, straining with all their strength, and held back the
crowd that was pushing behind them.

``Beat him!... Let the traitor perish and not disgrace the
Russian name!''  shouted Rostopchin. ``Cut him down. I command
it.''

Hearing not so much the words as the angry tone of Rostopchin's
voice, the crowd moaned and heaved forward, but again paused.

``Count!'' exclaimed the timid yet theatrical voice of
Vereshchagin in the midst of the momentary silence that ensued,
``Count! One God is above us both...'' He lifted his head and
again the thick vein in his thin neck filled with blood and the
color rapidly came and went in his face.

He did not finish what he wished to say.

``Cut him down! I command it...'' shouted Rostopchin, suddenly
growing pale like Vereshchagin.

``Draw sabers!'' cried the dragoon officer, drawing his own.

Another still stronger wave flowed through the crowd and reaching
the front ranks carried it swaying to the very steps of the
porch. The tall youth, with a stony look on his face, and rigid
and uplifted arm, stood beside Vereshchagin.

``Saber him!'' the dragoon officer almost whispered.

And one of the soldiers, his face all at once distorted with
fury, struck Vereshchagin on the head with the blunt side of his
saber.

``Ah!'' cried Vereshchagin in meek surprise, looking round with a
frightened glance as if not understanding why this was done to
him. A similar moan of surprise and horror ran through the
crowd. ``O Lord!''  exclaimed a sorrowful voice.

But after the exclamation of surprise that had escaped from
Vereshchagin he uttered a plaintive cry of pain, and that cry was
fatal. The barrier of human feeling, strained to the utmost, that
had held the crowd in check suddenly broke. The crime had begun
and must now be completed. The plaintive moan of reproach was
drowned by the threatening and angry roar of the crowd. Like the
seventh and last wave that shatters a ship, that last
irresistible wave burst from the rear and reached the front
ranks, carrying them off their feet and engulfing them all. The
dragoon was about to repeat his blow. Vereshchagin with a cry of
horror, covering his head with his hands, rushed toward the
crowd. The tall youth, against whom he stumbled, seized his thin
neck with his hands and, yelling wildly, fell with him under the
feet of the pressing, struggling crowd.

Some beat and tore at Vereshchagin, others at the tall youth. And
the screams of those that were being trampled on and of those who
tried to rescue the tall lad only increased the fury of the
crowd. It was a long time before the dragoons could extricate the
bleeding youth, beaten almost to death. And for a long time,
despite the feverish haste with which the mob tried to end the
work that had been begun, those who were hitting, throttling, and
tearing at Vereshchagin were unable to kill him, for the crowd
pressed from all sides, swaying as one mass with them in the
center and rendering it impossible for them either to kill him or
let him go.

``Hit him with an ax, eh!... Crushed?... Traitor, he sold
Christ...  Still alive... tenacious... serves him right! Torture
serves a thief right. Use the hatchet!... What---still alive?''

Only when the victim ceased to struggle and his cries changed to
a long-drawn, measured death rattle did the crowd around his
prostrate, bleeding corpse begin rapidly to change places. Each
one came up, glanced at what had been done, and with horror,
reproach, and astonishment pushed back again.

``O Lord! The people are like wild beasts! How could he be
alive?'' voices in the crowd could be heard saying. ``Quite a
young fellow too... must have been a merchant's son. What
men!... and they say he's not the right one... How not the right
one?... O Lord! And there's another has been beaten too---they
say he's nearly done for... Oh, the people... Aren't they afraid
of sinning?...'' said the same mob now, looking with pained
distress at the dead body with its long, thin, half-severed neck
and its livid face stained with blood and dust.

A painstaking police officer, considering the presence of a
corpse in his excellency's courtyard unseemly, told the dragoons
to take it away.  Two dragoons took it by its distorted legs and
dragged it along the ground. The gory, dust-stained, half-shaven
head with its long neck trailed twisting along the ground. The
crowd shrank back from it.

At the moment when Vereshchagin fell and the crowd closed in with
savage yells and swayed about him, Rostopchin suddenly turned
pale and, instead of going to the back entrance where his
carriage awaited him, went with hurried steps and bent head, not
knowing where and why, along the passage leading to the rooms on
the ground floor. The count's face was white and he could not
control the feverish twitching of his lower jaw.

``This way, your excellency... Where are you going?... This way,
please...'' said a trembling, frightened voice behind him.

Count Rostopchin was unable to reply and, turning obediently,
went in the direction indicated. At the back entrance stood his
caleche. The distant roar of the yelling crowd was audible even
there. He hastily took his seat and told the coachman to drive
him to his country house in Sokolniki.

When they reached the Myasnitski Street and could no longer hear
the shouts of the mob, the count began to repent. He remembered
with dissatisfaction the agitation and fear he had betrayed
before his subordinates. ``The mob is terrible---disgusting,'' he
said to himself in French. ``They are like wolves whom nothing
but flesh can appease.''  ``Count! One God is above us
both!''---Vereshchagin's words suddenly recurred to him, and a
disagreeable shiver ran down his back. But this was only a
momentary feeling and Count Rostopchin smiled disdainfully at
himself. ``I had other duties,'' thought he. ``The people had to
be appeased. Many other victims have perished and are perishing
for the public good''---and he began thinking of his social
duties to his family and to the city entrusted to him, and of
himself---not himself as Theodore Vasilyevich Rostopchin (he
fancied that Theodore Vasilyevich Rostopchin was sacrificing
himself for the public good) but himself as governor, the
representative of authority and of the Tsar. ``Had I been simply
Theodore Vasilyevich my course of action would have been quite
different, but it was my duty to safeguard my life and dignity as
commander-in-chief.''

Lightly swaying on the flexible springs of his carriage and no
longer hearing the terrible sounds of the crowd, Rostopchin grew
physically calm and, as always happens, as soon as he became
physically tranquil his mind devised reasons why he should be
mentally tranquil too. The thought which tranquillized Rostopchin
was not a new one. Since the world began and men have killed one
another no one has ever committed such a crime against his fellow
man without comforting himself with this same idea. This idea is
le bien public, the hypothetical welfare of other people.

To a man not swayed by passion that welfare is never certain, but
he who commits such a crime always knows just where that welfare
lies. And Rostopchin now knew it.

Not only did his reason not reproach him for what he had done,
but he even found cause for self-satisfaction in having so
successfully contrived to avail himself of a convenient
opportunity to punish a criminal and at the same time pacify the
mob.

``Vereshchagin was tried and condemned to death,'' thought
Rostopchin (though the Senate had only condemned Vereshchagin to
hard labor), ``he was a traitor and a spy. I could not let him go
unpunished and so I have killed two birds with one stone: to
appease the mob I gave them a victim and at the same time
punished a miscreant.''

Having reached his country house and begun to give orders about
domestic arrangements, the count grew quite tranquil.

Half an hour later he was driving with his fast horses across the
Sokolniki field, no longer thinking of what had occurred but
considering what was to come. He was driving to the Yauza bridge
where he had heard that Kutuzov was. Count Rostopchin was
mentally preparing the angry and stinging reproaches he meant to
address to Kutuzov for his deception. He would make that foxy old
courtier feel that the responsibility for all the calamities that
would follow the abandonment of the city and the ruin of Russia
(as Rostopchin regarded it) would fall upon his doting old
head. Planning beforehand what he would say to Kutuzov,
Rostopchin turned angrily in his caleche and gazed sternly from
side to side.

The Sokolniki field was deserted. Only at the end of it, in front
of the almshouse and the lunatic asylum, could be seen some
people in white and others like them walking singly across the
field shouting and gesticulating.

One of these was running to cross the path of Count Rostopchin's
carriage, and the count himself, his coachman, and his dragoons
looked with vague horror and curiosity at these released lunatics
and especially at the one running toward them.

Swaying from side to side on his long, thin legs in his
fluttering dressing gown, this lunatic was running impetuously,
his gaze fixed on Rostopchin, shouting something in a hoarse
voice and making signs to him to stop. The lunatic's solemn,
gloomy face was thin and yellow, with its beard growing in uneven
tufts. His black, agate pupils with saffron-yellow whites moved
restlessly near the lower eyelids.

``Stop! Pull up, I tell you!'' he cried in a piercing voice, and
again shouted something breathlessly with emphatic intonations
and gestures.

Coming abreast of the caleche he ran beside it.

``Thrice have they slain me, thrice have I risen from the
dead. They stoned me, crucified me... I shall rise... shall
rise... shall rise.  They have torn my body. The kingdom of God
will be overthrown... Thrice will I overthrow it and thrice
re-establish it!'' he cried, raising his voice higher and higher.

Count Rostopchin suddenly grew pale as he had done when the crowd
closed in on Vereshchagin. He turned away. ``Go fas... faster!''
he cried in a trembling voice to his coachman. The caleche flew
over the ground as fast as the horses could draw it, but for a
long time Count Rostopchin still heard the insane despairing
screams growing fainter in the distance, while his eyes saw
nothing but the astonished, frightened, bloodstained face of
``the traitor'' in the fur-lined coat.

Recent as that mental picture was, Rostopchin already felt that
it had cut deep into his heart and drawn blood. Even now he felt
clearly that the gory trace of that recollection would not pass
with time, but that the terrible memory would, on the contrary,
dwell in his heart ever more cruelly and painfully to the end of
his life. He seemed still to hear the sound of his own words:
``Cut him down! I command it...''

``Why did I utter those words? It was by some accident I said
them... I need not have said them,'' he thought. ``And then
nothing would have happened.'' He saw the frightened and then
infuriated face of the dragoon who dealt the blow, the look of
silent, timid reproach that boy in the fur-lined coat had turned
upon him. ``But I did not do it for my own sake. I was bound to
act that way... The mob, the traitor... the public welfare,''
thought he.

Troops were still crowding at the Yauza bridge. It was
hot. Kutuzov, dejected and frowning, sat on a bench by the bridge
toying with his whip in the sand when a caleche dashed up
noisily. A man in a general's uniform with plumes in his hat went
up to Kutuzov and said something in French. It was Count
Rostopchin. He told Kutuzov that he had come because Moscow, the
capital, was no more and only the army remained.

``Things would have been different if your Serene Highness had
not told me that you would not abandon Moscow without another
battle; all this would not have happened,'' he said.

Kutuzov looked at Rostopchin as if, not grasping what was said to
him, he was trying to read something peculiar written at that
moment on the face of the man addressing him. Rostopchin grew
confused and became silent. Kutuzov slightly shook his head and
not taking his penetrating gaze from Rostopchin's face muttered
softly:

``No! I shall not give up Moscow without a battle!''

Whether Kutuzov was thinking of something entirely different when
he spoke those words, or uttered them purposely, knowing them to
be meaningless, at any rate Rostopchin made no reply and hastily
left him.  And strange to say, the Governor of Moscow, the proud
Count Rostopchin, took up a Cossack whip and went to the bridge
where he began with shouts to drive on the carts that blocked the
way.

% % % % % % % % % % % % % % % % % % % % % % % % % % % % % % % % %
% % % % % % % % % % % % % % % % % % % % % % % % % % % % % % % % %
% % % % % % % % % % % % % % % % % % % % % % % % % % % % % % % % %
% % % % % % % % % % % % % % % % % % % % % % % % % % % % % % % % %
% % % % % % % % % % % % % % % % % % % % % % % % % % % % % % % % %
% % % % % % % % % % % % % % % % % % % % % % % % % % % % % % % % %
% % % % % % % % % % % % % % % % % % % % % % % % % % % % % % % % %
% % % % % % % % % % % % % % % % % % % % % % % % % % % % % % % % %
% % % % % % % % % % % % % % % % % % % % % % % % % % % % % % % % %
% % % % % % % % % % % % % % % % % % % % % % % % % % % % % % % % %
% % % % % % % % % % % % % % % % % % % % % % % % % % % % % % % % %
% % % % % % % % % % % % % % % % % % % % % % % % % % % % % %

\chapter*{Chapter XXVI} \ifaudio \marginpar{
\href{http://ia600205.us.archive.org/6/items/war_and_peace_11_0909/war_and_peace_11_26_tolstoy_64kb.mp3}{Audio}}
\fi

\initial{T}{oward} four o'clock in the afternoon Murat's troops were entering
Moscow. In front rode a detachment of Wurttemberg hussars and
behind them rode the King of Naples himself accompanied by a
numerous suite.

About the middle of the Arbat Street, near the Church of the
Miraculous Icon of St. Nicholas, Murat halted to await news from
the advanced detachment as to the condition in which they had
found the citadel, le Kremlin.

Around Murat gathered a group of those who had remained in
Moscow. They all stared in timid bewilderment at the strange,
long-haired commander dressed up in feathers and gold.

``Is that their Tsar himself? He's not bad!'' low voices could be
heard saying.

An interpreter rode up to the group.

``Take off your cap... your caps!'' These words went from one to
another in the crowd. The interpreter addressed an old porter and
asked if it was far to the Kremlin. The porter, listening in
perplexity to the unfamiliar Polish accent and not realizing that
the interpreter was speaking Russian, did not understand what was
being said to him and slipped behind the others.

Murat approached the interpreter and told him to ask where the
Russian army was. One of the Russians understood what was asked
and several voices at once began answering the interpreter. A
French officer, returning from the advanced detachment, rode up
to Murat and reported that the gates of the citadel had been
barricaded and that there was probably an ambuscade there.

``Good!'' said Murat and, turning to one of the gentlemen in his
suite, ordered four light guns to be moved forward to fire at the
gates.

The guns emerged at a trot from the column following Murat and
advanced up the Arbat. When they reached the end of the
Vozdvizhenka Street they halted and drew in the Square. Several
French officers superintended the placing of the guns and looked
at the Kremlin through field glasses.

The bells in the Kremlin were ringing for vespers, and this sound
troubled the French. They imagined it to be a call to arms. A few
infantrymen ran to the Kutafyev Gate. Beams and wooden screens
had been put there, and two musket shots rang out from under the
gate as soon as an officer and men began to run toward it. A
general who was standing by the guns shouted some words of
command to the officer, and the latter ran back again with his
men.

The sound of three more shots came from the gate.

One shot struck a French soldier's foot, and from behind the
screens came the strange sound of a few voices
shouting. Instantly as at a word of command the expression of
cheerful serenity on the faces of the French general, officers,
and men changed to one of determined concentrated readiness for
strife and suffering. To all of them from the marshal to the
least soldier, that place was not the Vozdvizhenka, Mokhavaya, or
Kutafyev Street, nor the Troitsa Gate (places familiar in
Moscow), but a new battlefield which would probably prove
sanguinary.  And all made ready for that battle. The cries from
the gates ceased. The guns were advanced, the artillerymen blew
the ash off their linstocks, and an officer gave the word
``Fire!'' This was followed by two whistling sounds of canister
shot, one after another. The shot rattled against the stone of
the gate and upon the wooden beams and screens, and two wavering
clouds of smoke rose over the Square.

A few instants after the echo of the reports resounding over the
stone-built Kremlin had died away the French heard a strange
sound above their head. Thousands of crows rose above the walls
and circled in the air, cawing and noisily flapping their
wings. Together with that sound came a solitary human cry from
the gateway and amid the smoke appeared the figure of a
bareheaded man in a peasant's coat. He grasped a musket and took
aim at the French. ``Fire!'' repeated the officer once more, and
the reports of a musket and of two cannon shots were heard
simultaneously.  The gate was again hidden by smoke.

Nothing more stirred behind the screens and the French infantry
soldiers and officers advanced to the gate. In the gateway lay
three wounded and four dead. Two men in peasant coats ran away at
the foot of the wall, toward the Znamenka.

``Clear that away!'' said the officer, pointing to the beams and
the corpses, and the French soldiers, after dispatching the
wounded, threw the corpses over the parapet.

Who these men were nobody knew. ``Clear that away!'' was all that
was said of them, and they were thrown over the parapet and
removed later on that they might not stink. Thiers alone
dedicates a few eloquent lines to their memory: ``These wretches
had occupied the sacred citadel, having supplied themselves with
guns from the arsenal, and fired'' (the wretches) ``at the
French. Some of them were sabered and the Kremlin was purged of
their presence.''

Murat was informed that the way had been cleared. The French
entered the gates and began pitching their camp in the Senate
Square. Out of the windows of the Senate House the soldiers threw
chairs into the Square for fuel and kindled fires there.

Other detachments passed through the Kremlin and encamped along
the Moroseyka, the Lubyanka, and Pokrovka Streets. Others
quartered themselves along the Vozdvizhenka, the Nikolski, and
the Tverskoy Streets. No masters of the houses being found
anywhere, the French were not billeted on the inhabitants as is
usual in towns but lived in it as in a camp.

Though tattered, hungry, worn out, and reduced to a third of
their original number, the French entered Moscow in good marching
order. It was a weary and famished, but still a fighting and
menacing army. But it remained an army only until its soldiers
had dispersed into their different lodgings. As soon as the men
of the various regiments began to disperse among the wealthy and
deserted houses, the army was lost forever and there came into
being something nondescript, neither citizens nor soldiers but
what are known as marauders. When five weeks later these same men
left Moscow, they no longer formed an army. They were a mob of
marauders, each carrying a quantity of articles which seemed to
him valuable or useful. The aim of each man when he left Moscow
was no longer, as it had been, to conquer, but merely to keep
what he had acquired. Like a monkey which puts its paw into the
narrow neck of a jug, and having seized a handful of nuts will
not open its fist for fear of losing what it holds, and therefore
perishes, the French when they left Moscow had inevitably to
perish because they carried their loot with them, yet to abandon
what they had stolen was as impossible for them as it is for the
monkey to open its paw and let go of its nuts. Ten minutes after
each regiment had entered a Moscow district, not a soldier or
officer was left. Men in military uniforms and Hessian boots
could be seen through the windows, laughing and walking through
the rooms. In cellars and storerooms similar men were busy among
the provisions, and in the yards unlocking or breaking open coach
house and stable doors, lighting fires in kitchens and kneading
and baking bread with rolled-up sleeves, and cooking; or
frightening, amusing, or caressing women and children. There were
many such men both in the shops and houses---but there was no
army.

Order after order was issued by the French commanders that day
forbidding the men to disperse about the town, sternly forbidding
any violence to the inhabitants or any looting, and announcing a
roll call for that very evening. But despite all these measures
the men, who had till then constituted an army, flowed all over
the wealthy, deserted city with its comforts and plentiful
supplies. As a hungry herd of cattle keeps well together when
crossing a barren field, but gets out of hand and at once
disperses uncontrollably as soon as it reaches rich pastures, so
did the army disperse all over the wealthy city.

No residents were left in Moscow, and the soldiers---like water
percolating through sand---spread irresistibly through the city
in all directions from the Kremlin into which they had first
marched. The cavalry, on entering a merchant's house that had
been abandoned and finding there stabling more than sufficient
for their horses, went on, all the same, to the next house which
seemed to them better. Many of them appropriated several houses,
chalked their names on them, and quarreled and even fought with
other companies for them. Before they had had time to secure
quarters the soldiers ran out into the streets to see the city
and, hearing that everything had been abandoned, rushed to places
where valuables were to be had for the taking. The officers
followed to check the soldiers and were involuntarily drawn into
doing the same. In Carriage Row carriages had been left in the
shops, and generals flocked there to select caleches and coaches
for themselves.  The few inhabitants who had remained invited
commanding officers to their houses, hoping thereby to secure
themselves from being plundered.  There were masses of wealth and
there seemed no end to it. All around the quarters occupied by
the French were other regions still unexplored and unoccupied
where, they thought, yet greater riches might be found.  And
Moscow engulfed the army ever deeper and deeper. When water is
spilled on dry ground both the dry ground and the water disappear
and mud results; and in the same way the entry of the famished
army into the rich and deserted city resulted in fires and
looting and the destruction of both the army and the wealthy
city.

The French attributed the Fire of Moscow au patriotisme feroce de
Rostopchine,\footnote{To Rostopchin's ferocious patriotism.} the
Russians to the barbarity of the French. In reality, however, it
was not, and could not be, possible to explain the burning of
Moscow by making any individual, or any group of people,
responsible for it. Moscow was burned because it found itself in
a position in which any town built of wood was bound to burn,
quite apart from whether it had, or had not, a hundred and thirty
inferior fire engines. Deserted Moscow had to burn as inevitably
as a heap of shavings has to burn on which sparks continually
fall for several days. A town built of wood, where scarcely a day
passes without conflagrations when the house owners are in
residence and a police force is present, cannot help burning when
its inhabitants have left it and it is occupied by soldiers who
smoke pipes, make campfires of the Senate chairs in the Senate
Square, and cook themselves meals twice a day. In peacetime it is
only necessary to billet troops in the villages of any district
and the number of fires in that district immediately
increases. How much then must the probability of fire be
increased in an abandoned, wooden town where foreign troops are
quartered. \emph{Le patriotisme feroce de
Rostopchine}\footnote{Rostopchin's fierce patriotism} and the
barbarity of the French were not to blame in the matter. Moscow
was set on fire by the soldiers' pipes, kitchens, and campfires,
and by the carelessness of enemy soldiers occupying houses they
did not own. Even if there was any arson (which is very doubtful,
for no one had any reason to burn the houses---in any case a
troublesome and dangerous thing to do), arson cannot be regarded
as the cause, for the same thing would have happened without any
incendiarism.

However tempting it might be for the French to blame Rostopchin's
ferocity and for Russians to blame the scoundrel Bonaparte, or
later on to place an heroic torch in the hands of their own
people, it is impossible not to see that there could be no such
direct cause of the fire, for Moscow had to burn as every
village, factory, or house must burn which is left by its owners
and in which strangers are allowed to live and cook their
porridge. Moscow was burned by its inhabitants, it is true, but
by those who had abandoned it and not by those who remained in
it. Moscow when occupied by the enemy did not remain intact like
Berlin, Vienna, and other towns, simply because its inhabitants
abandoned it and did not welcome the French with bread and salt,
nor bring them the keys of the city.

% % % % % % % % % % % % % % % % % % % % % % % % % % % % % % % % %
% % % % % % % % % % % % % % % % % % % % % % % % % % % % % % % % %
% % % % % % % % % % % % % % % % % % % % % % % % % % % % % % % % %
% % % % % % % % % % % % % % % % % % % % % % % % % % % % % % % % %
% % % % % % % % % % % % % % % % % % % % % % % % % % % % % % % % %
% % % % % % % % % % % % % % % % % % % % % % % % % % % % % % % % %
% % % % % % % % % % % % % % % % % % % % % % % % % % % % % % % % %
% % % % % % % % % % % % % % % % % % % % % % % % % % % % % % % % %
% % % % % % % % % % % % % % % % % % % % % % % % % % % % % % % % %
% % % % % % % % % % % % % % % % % % % % % % % % % % % % % % % % %
% % % % % % % % % % % % % % % % % % % % % % % % % % % % % % % % %
% % % % % % % % % % % % % % % % % % % % % % % % % % % % % %

\chapter*{Chapter XXVII} \ifaudio \marginpar{
\href{http://ia600205.us.archive.org/6/items/war_and_peace_11_0909/war_and_peace_11_27_tolstoy_64kb.mp3}{Audio}}
\fi

\initial{T}{he} absorption of the French by Moscow, radiating starwise as it
did, only reached the quarter where Pierre was staying by the
evening of the second of September.

After the last two days spent in solitude and unusual
circumstances, Pierre was in a state bordering on insanity. He
was completely obsessed by one persistent thought. He did not
know how or when this thought had taken such possession of him,
but he remembered nothing of the past, understood nothing of the
present, and all he saw and heard appeared to him like a dream.

He had left home only to escape the intricate tangle of life's
demands that enmeshed him, and which in his present condition he
was unable to unravel. He had gone to Joseph Alexeevich's house,
on the plea of sorting the deceased's books and papers, only in
search of rest from life's turmoil, for in his mind the memory of
Joseph Alexeevich was connected with a world of eternal, solemn,
and calm thoughts, quite contrary to the restless confusion into
which he felt himself being drawn. He sought a quiet refuge, and
in Joseph Alexeevich's study he really found it. When he sat with
his elbows on the dusty writing table in the deathlike stillness
of the study, calm and significant memories of the last few days
rose one after another in his imagination, particularly of the
battle of Borodino and of that vague sense of his own
insignificance and insincerity compared with the truth,
simplicity, and strength of the class of men he mentally classed
as they. When Gerasim roused him from his reverie the idea
occurred to him of taking part in the popular defense of Moscow
which he knew was projected. And with that object he had asked
Gerasim to get him a peasant's coat and a pistol, confiding to
him his intentions of remaining in Joseph Alexeevich's house and
keeping his name secret. Then during the first day spent in
inaction and solitude (he tried several times to fix his
attention on the masonic manuscripts, but was unable to do so)
the idea that had previously occurred to him of the cabalistic
significance of his name in connection with Bonaparte's more than
once vaguely presented itself. But the idea that he, L'russe
Besuhof, was destined to set a limit to the power of the Beast
was as yet only one of the fancies that often passed through his
mind and left no trace behind.

When, having bought the coat merely with the object of taking
part among the people in the defense of Moscow, Pierre had met
the Rostovs and Natasha had said to him: ``Are you remaining in
Moscow?... How splendid!''  the thought flashed into his mind
that it really would be a good thing, even if Moscow were taken,
for him to remain there and do what he was predestined to do.

Next day, with the sole idea of not sparing himself and not
lagging in any way behind them, Pierre went to the Three Hills
gate. But when he returned to the house convinced that Moscow
would not be defended, he suddenly felt that what before had
seemed to him merely a possibility had now become absolutely
necessary and inevitable. He must remain in Moscow, concealing
his name, and must meet Napoleon and kill him, and either perish
or put an end to the misery of all Europe---which it seemed to
him was solely due to Napoleon.

Pierre knew all the details of the attempt on Bonaparte's life in
1809 by a German student in Vienna, and knew that the student had
been shot.  And the risk to which he would expose his life by
carrying out his design excited him still more.

Two equally strong feelings drew Pierre irresistibly to this
purpose.  The first was a feeling of the necessity of sacrifice
and suffering in view of the common calamity, the same feeling
that had caused him to go to Mozhaysk on the twenty-fifth and to
make his way to the very thick of the battle and had now caused
him to run away from his home and, in place of the luxury and
comfort to which he was accustomed, to sleep on a hard sofa
without undressing and eat the same food as Gerasim. The other
was that vague and quite Russian feeling of contempt for
everything conventional, artificial, and human---for everything
the majority of men regard as the greatest good in the
world. Pierre had first experienced this strange and fascinating
feeling at the Sloboda Palace, when he had suddenly felt that
wealth, power, and life---all that men so painstakingly acquire
and guard---if it has any worth has so only by reason of the joy
with which it can all be renounced.

It was the feeling that induces a volunteer recruit to spend his
last penny on drink, and a drunken man to smash mirrors or
glasses for no apparent reason and knowing that it will cost him
all the money he possesses: the feeling which causes a man to
perform actions which from an ordinary point of view are insane,
to test, as it were, his personal power and strength, affirming
the existence of a higher, nonhuman criterion of life.

From the very day Pierre had experienced this feeling for the
first time at the Sloboda Palace he had been continuously under
its influence, but only now found full satisfaction for
it. Moreover, at this moment Pierre was supported in his design
and prevented from renouncing it by what he had already done in
that direction. If he were now to leave Moscow like everyone
else, his flight from home, the peasant coat, the pistol, and his
announcement to the Rostovs that he would remain in Moscow would
all become not merely meaningless but contemptible and
ridiculous, and to this Pierre was very sensitive.

Pierre's physical condition, as is always the case, corresponded
to his mental state. The unaccustomed coarse food, the vodka he
drank during those days, the absence of wine and cigars, his
dirty unchanged linen, two almost sleepless nights passed on a
short sofa without bedding---all this kept him in a state of
excitement bordering on insanity.

It was two o'clock in the afternoon. The French had already
entered Moscow. Pierre knew this, but instead of acting he only
thought about his undertaking, going over its minutest details in
his mind. In his fancy he did not clearly picture to himself
either the striking of the blow or the death of Napoleon, but
with extraordinary vividness and melancholy enjoyment imagined
his own destruction and heroic endurance.

``Yes, alone, for the sake of all, I must do it or perish!'' he
thought.  ``Yes, I will approach... and then suddenly... with
pistol or dagger? But that is all the same! 'It is not I but the
hand of Providence that punishes thee,' I shall say,'' thought
he, imagining what he would say when killing Napoleon. ``Well
then, take me and execute me!'' he went on, speaking to himself
and bowing his head with a sad but firm expression.

While Pierre, standing in the middle of the room, was talking to
himself in this way, the study door opened and on the threshold
appeared the figure of Makar Alexeevich, always so timid before
but now quite transformed.

His dressing gown was unfastened, his face red and distorted. He
was obviously drunk. On seeing Pierre he grew confused at first,
but noticing embarrassment on Pierre's face immediately grew bold
and, staggering on his thin legs, advanced into the middle of the
room.

``They're frightened,'' he said confidentially in a hoarse
voice. ``I say I won't surrender, I say... Am I not right, sir?''

He paused and then suddenly seeing the pistol on the table seized
it with unexpected rapidity and ran out into the corridor.

Gerasim and the porter, who had followed Makar Alexeevich,
stopped him in the vestibule and tried to take the pistol from
him. Pierre, coming out into the corridor, looked with pity and
repulsion at the half-crazy old man. Makar Alexeevich, frowning
with exertion, held on to the pistol and screamed hoarsely,
evidently with some heroic fancy in his head.

``To arms! Board them! No, you shan't get it,'' he yelled.

``That will do, please, that will do. Have the goodness---please,
sir, to let go! Please, sir...'' pleaded Gerasim, trying
carefully to steer Makar Alexeevich by the elbows back to the
door.

``Who are you? Bonaparte!...'' shouted Makar Alexeevich.

``That's not right, sir. Come to your room, please, and
rest. Allow me to have the pistol.''

``Be off, thou base slave! Touch me not! See this?'' shouted
Makar Alexeevich, brandishing the pistol. ``Board them!''

``Catch hold!'' whispered Gerasim to the porter.

They seized Makar Alexeevich by the arms and dragged him to the
door.

The vestibule was filled with the discordant sounds of a struggle
and of a tipsy, hoarse voice.

Suddenly a fresh sound, a piercing feminine scream, reverberated
from the porch and the cook came running into the vestibule.

``It's them! Gracious heavens! O Lord, four of them, horsemen!''
she cried.

Gerasim and the porter let Makar Alexeevich go, and in the now
silent corridor the sound of several hands knocking at the front
door could be heard.

% % % % % % % % % % % % % % % % % % % % % % % % % % % % % % % % %
% % % % % % % % % % % % % % % % % % % % % % % % % % % % % % % % %
% % % % % % % % % % % % % % % % % % % % % % % % % % % % % % % % %
% % % % % % % % % % % % % % % % % % % % % % % % % % % % % % % % %
% % % % % % % % % % % % % % % % % % % % % % % % % % % % % % % % %
% % % % % % % % % % % % % % % % % % % % % % % % % % % % % % % % %
% % % % % % % % % % % % % % % % % % % % % % % % % % % % % % % % %
% % % % % % % % % % % % % % % % % % % % % % % % % % % % % % % % %
% % % % % % % % % % % % % % % % % % % % % % % % % % % % % % % % %
% % % % % % % % % % % % % % % % % % % % % % % % % % % % % % % % %
% % % % % % % % % % % % % % % % % % % % % % % % % % % % % % % % %
% % % % % % % % % % % % % % % % % % % % % % % % % % % % % %

\chapter*{Chapter XXVIII} \ifaudio \marginpar{
\href{http://ia600205.us.archive.org/6/items/war_and_peace_11_0909/war_and_peace_11_28_tolstoy_64kb.mp3}{Audio}}
\fi

\initial{P}{ierre}, having decided that until he had carried out his design
he would disclose neither his identity nor his knowledge of
French, stood at the half-open door of the corridor, intending to
conceal himself as soon as the French entered. But the French
entered and still Pierre did not retire---an irresistible
curiosity kept him there.

There were two of them. One was an officer---a tall, soldierly,
handsome man---the other evidently a private or an orderly,
sunburned, short, and thin, with sunken cheeks and a dull
expression. The officer walked in front, leaning on a stick and
slightly limping. When he had advanced a few steps he stopped,
having apparently decided that these were good quarters, turned
round to the soldiers standing at the entrance, and in a loud
voice of command ordered them to put up the horses. Having done
that, the officer, lifting his elbow with a smart gesture,
stroked his mustache and lightly touched his hat.

``Bonjour, la compagnie!''\footnote{``Good day, everybody!''}
said he gaily, smiling and looking about him.

No one gave any reply.

``Vous etes le bourgeois?''\footnote{ ``Are you the master
here?''}  the officer asked Gerasim.

Gerasim gazed at the officer with an alarmed and inquiring look.

``Quartier, quartier, logement!'' said the officer, looking down
at the little man with a condescending and good-natured
smile. ``Les francais sont de bons enfants. Que diable! Voyons!
Ne nous fachons pas, mon vieux!''\footnote{``Quarters, quarters,
lodgings! The French are good fellows. What the devil! There,
don't let us be cross, old fellow!''}  added he, clapping the
scared and silent Gerasim on the shoulder. ``Well, does no one
speak French in this establishment?'' he asked again in French,
looking around and meeting Pierre's eyes. Pierre moved away from
the door.

Again the officer turned to Gerasim and asked him to show him the
rooms in the house.

``Master, not here---don't understand... me, you...'' said
Gerasim, trying to render his words more comprehensible by
contorting them.

Still smiling, the French officer spread out his hands before
Gerasim's nose, intimating that he did not understand him either,
and moved, limping, to the door at which Pierre was
standing. Pierre wished to go away and conceal himself, but at
that moment he saw Makar Alexeevich appearing at the open kitchen
door with the pistol in his hand. With a madman's cunning, Makar
Alexeevich eyed the Frenchman, raised his pistol, and took aim.

``Board them!'' yelled the tipsy man, trying to press the
trigger. Hearing the yell the officer turned round, and at the
same moment Pierre threw himself on the drunkard. Just when
Pierre snatched at and struck up the pistol Makar Alexeevich at
last got his fingers on the trigger, there was a deafening
report, and all were enveloped in a cloud of smoke. The Frenchman
turned pale and rushed to the door.

Forgetting his intention of concealing his knowledge of French,
Pierre, snatching away the pistol and throwing it down, ran up to
the officer and addressed him in French.

``You are not wounded?'' he asked.

``I think not,'' answered the Frenchman, feeling himself
over. ``But I have had a lucky escape this time,'' he added,
pointing to the damaged plaster of the wall. ``Who is that man?''
said he, looking sternly at Pierre.

``Oh, I am really in despair at what has occurred,'' said Pierre
rapidly, quite forgetting the part he had intended to play. ``He
is an unfortunate madman who did not know what he was doing.''

The officer went up to Makar Alexeevich and took him by the
collar.

Makar Alexeevich was standing with parted lips, swaying, as if
about to fall asleep, as he leaned against the wall.

``Brigand! You shall pay for this,'' said the Frenchman, letting
go of him. ``We French are merciful after victory, but we do not
pardon traitors,'' he added, with a look of gloomy dignity and a
fine energetic gesture.

Pierre continued, in French, to persuade the officer not to hold
that drunken imbecile to account. The Frenchman listened in
silence with the same gloomy expression, but suddenly turned to
Pierre with a smile. For a few seconds he looked at him in
silence. His handsome face assumed a melodramatically gentle
expression and he held out his hand.

``You have saved my life. You are French,'' said he.

For a Frenchman that deduction was indubitable. Only a Frenchman
could perform a great deed, and to save his life---the life of
M. Ramballe, captain of the 13th Light Regiment---was undoubtedly
a very great deed.

But however indubitable that conclusion and the officer's
conviction based upon it, Pierre felt it necessary to disillusion
him.

``I am Russian,'' he said quickly.

``Tut, tut, tut! Tell that to others,'' said the officer, waving
his finger before his nose and smiling. ``You shall tell me all
about that presently. I am delighted to meet a compatriot. Well,
and what are we to do with this man?'' he added, addressing
himself to Pierre as to a brother.

Even if Pierre were not a Frenchman, having once received that
loftiest of human appellations he could not renounce it, said the
officer's look and tone. In reply to his last question Pierre
again explained who Makar Alexeevich was and how just before
their arrival that drunken imbecile had seized the loaded pistol
which they had not had time to recover from him, and begged the
officer to let the deed go unpunished.

The Frenchman expanded his chest and made a majestic gesture with
his arm.

``You have saved my life! You are French. You ask his pardon? I
grant it you. Lead that man away!'' said he quickly and
energetically, and taking the arm of Pierre whom he had promoted
to be a Frenchman for saving his life, he went with him into the
room.

The soldiers in the yard, hearing the shot, came into the passage
asking what had happened, and expressed their readiness to punish
the culprits, but the officer sternly checked them.

``You will be called in when you are wanted,'' he said.

The soldiers went out again, and the orderly, who had meanwhile
had time to visit the kitchen, came up to his officer.

``Captain, there is soup and a leg of mutton in the kitchen,''
said he.  ``Shall I serve them up?''

``Yes, and some wine,'' answered the captain.

% % % % % % % % % % % % % % % % % % % % % % % % % % % % % % % % %
% % % % % % % % % % % % % % % % % % % % % % % % % % % % % % % % %
% % % % % % % % % % % % % % % % % % % % % % % % % % % % % % % % %
% % % % % % % % % % % % % % % % % % % % % % % % % % % % % % % % %
% % % % % % % % % % % % % % % % % % % % % % % % % % % % % % % % %
% % % % % % % % % % % % % % % % % % % % % % % % % % % % % % % % %
% % % % % % % % % % % % % % % % % % % % % % % % % % % % % % % % %
% % % % % % % % % % % % % % % % % % % % % % % % % % % % % % % % %
% % % % % % % % % % % % % % % % % % % % % % % % % % % % % % % % %
% % % % % % % % % % % % % % % % % % % % % % % % % % % % % % % % %
% % % % % % % % % % % % % % % % % % % % % % % % % % % % % % % % %
% % % % % % % % % % % % % % % % % % % % % % % % % % % % % %

\chapter*{Chapter XXIX} \ifaudio \marginpar{
\href{http://ia600205.us.archive.org/6/items/war_and_peace_11_0909/war_and_peace_11_29_tolstoy_64kb.mp3}{Audio}}
\fi

\initial{W}{hen} the French officer went into the room with Pierre the latter
again thought it his duty to assure him that he was not French
and wished to go away, but the officer would not hear of it. He
was so very polite, amiable, good-natured, and genuinely grateful
to Pierre for saving his life that Pierre had not the heart to
refuse, and sat down with him in the parlor---the first room they
entered. To Pierre's assurances that he was not a Frenchman, the
captain, evidently not understanding how anyone could decline so
flattering an appellation, shrugged his shoulders and said that
if Pierre absolutely insisted on passing for a Russian let it be
so, but for all that he would be forever bound to Pierre by
gratitude for saving his life.

Had this man been endowed with the slightest capacity for
perceiving the feelings of others, and had he at all understood
what Pierre's feelings were, the latter would probably have left
him, but the man's animated obtuseness to everything other than
himself disarmed Pierre.

``A Frenchman or a Russian prince incognito,'' said the officer,
looking at Pierre's fine though dirty linen and at the ring on
his finger. ``I owe my life to you and offer you my friendship. A
Frenchman never forgets either an insult or a service. I offer
you my friendship. That is all I can say.''

There was so much good nature and nobility (in the French sense
of the word) in the officer's voice, in the expression of his
face and in his gestures, that Pierre, unconsciously smiling in
response to the Frenchman's smile, pressed the hand held out to
him.

``Captain Ramballe, of the 13th Light Regiment, Chevalier of the
Legion of Honor for the affair on the seventh of September,'' he
introduced himself, a self-satisfied irrepressible smile
puckering his lips under his mustache. ``Will you now be so good
as to tell me with whom I have the honor of conversing so
pleasantly, instead of being in the ambulance with that maniac's
bullet in my body?''

Pierre replied that he could not tell him his name and, blushing,
began to try to invent a name and to say something about his
reason for concealing it, but the Frenchman hastily interrupted
him.

``Oh, please!'' said he. ``I understand your reasons. You are an
officer...  a superior officer perhaps. You have borne arms
against us. That's not my business. I owe you my life. That is
enough for me. I am quite at your service. You belong to the
gentry?'' he concluded with a shade of inquiry in his
tone. Pierre bent his head. ``Your baptismal name, if you
please. That is all I ask. Monsieur Pierre, you say... That's all
I want to know.''

When the mutton and an omelet had been served and a samovar and
vodka brought, with some wine which the French had taken from a
Russian cellar and brought with them, Ramballe invited Pierre to
share his dinner, and himself began to eat greedily and quickly
like a healthy and hungry man, munching his food rapidly with his
strong teeth, continually smacking his lips, and
repeating---``Excellent! Delicious!'' His face grew red and was
covered with perspiration. Pierre was hungry and shared the
dinner with pleasure. Morel, the orderly, brought some hot water
in a saucepan and placed a bottle of claret in it. He also
brought a bottle of kvass, taken from the kitchen for them to
try. That beverage was already known to the French and had been
given a special name. They called it limonade de cochon (pig's
lemonade), and Morel spoke well of the limonade de cochon he had
found in the kitchen. But as the captain had the wine they had
taken while passing through Moscow, he left the kvass to Morel
and applied himself to the bottle of Bordeaux. He wrapped the
bottle up to its neck in a table napkin and poured out wine for
himself and for Pierre. The satisfaction of his hunger and the
wine rendered the captain still more lively and he chatted
incessantly all through dinner.

``Yes, my dear Monsieur Pierre, I owe you a fine votive candle
for saving me from that maniac... You see, I have bullets enough
in my body already. Here is one I got at Wagram'' (he touched his
side) ``and a second at Smolensk''---he showed a scar on his
cheek---``and this leg which as you see does not want to march, I
got that on the seventh at the great battle of la Moskowa. Sacre
Dieu! It was splendid! That deluge of fire was worth seeing. It
was a tough job you set us there, my word! You may be proud of
it! And on my honor, in spite of the cough I caught there, I
should be ready to begin again. I pity those who did not see
it.''

``I was there,'' said Pierre.

``Bah, really? So much the better! You are certainly brave
foes. The great redoubt held out well, by my pipe!'' continued
the Frenchman. ``And you made us pay dear for it. I was at it
three times---sure as I sit here. Three times we reached the guns
and three times we were thrown back like cardboard figures. Oh,
it was beautiful, Monsieur Pierre! Your grenadiers were splendid,
by heaven! I saw them close up their ranks six times in
succession and march as if on parade. Fine fellows! Our King of
Naples, who knows what's what, cried 'Bravo!' Ha, ha! So you are
one of us soldiers!'' he added, smiling, after a momentary
pause. ``So much the better, so much the better, Monsieur Pierre!
Terrible in battle...  gallant... with the fair'' (he winked and
smiled), ``that's what the French are, Monsieur Pierre, aren't
they?''

The captain was so naively and good-humoredly gay, so real, and
so pleased with himself that Pierre almost winked back as he
looked merrily at him. Probably the word ``gallant'' turned the
captain's thoughts to the state of Moscow.

``Apropos, tell me please, is it true that the women have all
left Moscow? What a queer idea! What had they to be afraid of?''

``Would not the French ladies leave Paris if the Russians entered
it?''  asked Pierre.

``Ha, ha, ha!'' The Frenchman emitted a merry, sanguine chuckle,
patting Pierre on the shoulder. ``What a thing to say!'' he
exclaimed. ``Paris?...  But Paris, Paris...''

``Paris---the capital of the world,'' Pierre finished his remark
for him.

The captain looked at Pierre. He had a habit of stopping short in
the middle of his talk and gazing intently with his laughing,
kindly eyes.

``Well, if you hadn't told me you were Russian, I should have
wagered that you were Parisian! You have that... I don't know
what, that...'' and having uttered this compliment, he again
gazed at him in silence.

``I have been in Paris. I spent years there,'' said Pierre.

``Oh yes, one sees that plainly. Paris!... A man who doesn't know
Paris is a savage. You can tell a Parisian two leagues off. Paris
is Talma, la Duchenois, Potier, the Sorbonne, the boulevards,''
and noticing that his conclusion was weaker than what had gone
before, he added quickly: ``There is only one Paris in the
world. You have been to Paris and have remained Russian. Well, I
don't esteem you the less for it.''

Under the influence of the wine he had drunk, and after the days
he had spent alone with his depressing thoughts, Pierre
involuntarily enjoyed talking with this cheerful and good-natured
man.

``To return to your ladies---I hear they are lovely. What a
wretched idea to go and bury themselves in the steppes when the
French army is in Moscow. What a chance those girls have missed!
Your peasants, now---that's another thing; but you civilized
people, you ought to know us better than that. We took Vienna,
Berlin, Madrid, Naples, Rome, Warsaw, all the world's
capitals... We are feared, but we are loved. We are nice to
know. And then the Emperor...'' he began, but Pierre interrupted
him.

``The Emperor,'' Pierre repeated, and his face suddenly became
sad and embarrassed, ``is the Emperor...?''

``The Emperor? He is generosity, mercy, justice, order,
genius---that's what the Emperor is! It is I, Ramballe, who tell
you so... I assure you I was his enemy eight years ago. My father
was an emigrant count... But that man has vanquished me. He has
taken hold of me. I could not resist the sight of the grandeur
and glory with which he has covered France.  When I understood
what he wanted---when I saw that he was preparing a bed of
laurels for us, you know, I said to myself: 'That is a monarch,'
and I devoted myself to him! So there! Oh yes, mon cher, he is
the greatest man of the ages past or future.''

``Is he in Moscow?'' Pierre stammered with a guilty look.

The Frenchman looked at his guilty face and smiled.

``No, he will make his entry tomorrow,'' he replied, and
continued his talk.

Their conversation was interrupted by the cries of several voices
at the gate and by Morel, who came to say that some Wurttemberg
hussars had come and wanted to put up their horses in the yard
where the captain's horses were. This difficulty had arisen
chiefly because the hussars did not understand what was said to
them in French.

The captain had their senior sergeant called in, and in a stern
voice asked him to what regiment he belonged, who was his
commanding officer, and by what right he allowed himself to claim
quarters that were already occupied. The German who knew little
French, answered the two first questions by giving the names of
his regiment and of his commanding officer, but in reply to the
third question which he did not understand said, introducing
broken French into his own German, that he was the quartermaster
of the regiment and his commander had ordered him to occupy all
the houses one after another. Pierre, who knew German, translated
what the German said to the captain and gave the captain's reply
to the Wurttemberg hussar in German. When he had understood what
was said to him, the German submitted and took his men
elsewhere. The captain went out into the porch and gave some
orders in a loud voice.

When he returned to the room Pierre was sitting in the same place
as before, with his head in his hands. His face expressed
suffering. He really was suffering at that moment. When the
captain went out and he was left alone, suddenly he came to
himself and realized the position he was in. It was not that
Moscow had been taken or that the happy conquerors were masters
in it and were patronizing him. Painful as that was it was not
that which tormented Pierre at the moment. He was tormented by
the consciousness of his own weakness. The few glasses of wine he
had drunk and the conversation with this good-natured man had
destroyed the mood of concentrated gloom in which he had spent
the last few days and which was essential for the execution of
his design. The pistol, dagger, and peasant coat were
ready. Napoleon was to enter the town next day. Pierre still
considered that it would be a useful and worthy action to slay
the evildoer, but now he felt that he would not do it. He did not
know why, but he felt a foreboding that he would not carry out
his intention. He struggled against the confession of his
weakness but dimly felt that he could not overcome it and that
his former gloomy frame of mind, concerning vengeance, killing,
and self-sacrifice, had been dispersed like dust by contact with
the first man he met.

The captain returned to the room, limping slightly and whistling
a tune.

The Frenchman's chatter which had previously amused Pierre now
repelled him. The tune he was whistling, his gait, and the
gesture with which he twirled his mustache, all now seemed
offensive. ``I will go away immediately. I won't say another word
to him,'' thought Pierre. He thought this, but still sat in the
same place. A strange feeling of weakness tied him to the spot;
he wished to get up and go away, but could not do so.

The captain, on the other hand, seemed very cheerful. He paced up
and down the room twice. His eyes shone and his mustache twitched
as if he were smiling to himself at some amusing thought.

``The colonel of those Wurttembergers is delightful,'' he
suddenly said.  ``He's a German, but a nice fellow all the
same... But he's a German.''  He sat down facing Pierre. ``By the
way, you know German, then?''

Pierre looked at him in silence.

``What is the German for 'shelter'?''

``Shelter?'' Pierre repeated. ``The German for shelter is
Unterkunft.''

``How do you say it?'' the captain asked quickly and doubtfully.

``Unterkunft,'' Pierre repeated.

``Onterkoff,'' said the captain and looked at Pierre for some
seconds with laughing eyes. ``These Germans are first-rate fools,
don't you think so, Monsieur Pierre?'' he concluded.

``Well, let's have another bottle of this Moscow Bordeaux, shall
we?  Morel will warm us up another little bottle. Morel!'' he
called out gaily.

Morel brought candles and a bottle of wine. The captain looked at
Pierre by the candlelight and was evidently struck by the
troubled expression on his companion's face. Ramballe, with
genuine distress and sympathy in his face, went up to Pierre and
bent over him.

``There now, we're sad,'' said he, touching Pierre's hand. ``Have
I upset you? No, really, have you anything against me?'' he asked
Pierre.  ``Perhaps it's the state of affairs?''

Pierre did not answer, but looked cordially into the Frenchman's
eyes whose expression of sympathy was pleasing to him.

``Honestly, without speaking of what I owe you, I feel friendship
for you. Can I do anything for you? Dispose of me. It is for life
and death.  I say it with my hand on my heart!'' said he,
striking his chest.

``Thank you,'' said Pierre.

The captain gazed intently at him as he had done when he learned
that ``shelter'' was Unterkunft in German, and his face suddenly
brightened.

``Well, in that case, I drink to our friendship!'' he cried
gaily, filling two glasses with wine.

Pierre took one of the glasses and emptied it. Ramballe emptied
his too, again pressed Pierre's hand, and leaned his elbows on
the table in a pensive attitude.

``Yes, my dear friend,'' he began, ``such is fortune's
caprice. Who would have said that I should be a soldier and a
captain of dragoons in the service of Bonaparte, as we used to
call him? Yet here I am in Moscow with him. I must tell you, mon
cher,'' he continued in the sad and measured tones of a man who
intends to tell a long story, ``that our name is one of the most
ancient in France.''

And with a Frenchman's easy and naive frankness the captain told
Pierre the story of his ancestors, his childhood, youth, and
manhood, and all about his relations and his financial and family
affairs, ``ma pauvre mere'' playing of course an important part
in the story.

``But all that is only life's setting, the real thing is
love---love! Am I not right, Monsieur Pierre?'' said he, growing
animated. ``Another glass?''

Pierre again emptied his glass and poured himself out a third.

``Oh, women, women!'' and the captain, looking with glistening
eyes at Pierre, began talking of love and of his love affairs.

There were very many of these, as one could easily believe,
looking at the officer's handsome, self-satisfied face, and
noting the eager enthusiasm with which he spoke of women. Though
all Ramballe's love stories had the sensual character which
Frenchmen regard as the special charm and poetry of love, yet he
told his story with such sincere conviction that he alone had
experienced and known all the charm of love and he described
women so alluringly that Pierre listened to him with curiosity.

It was plain that l'amour which the Frenchman was so fond of was
not that low and simple kind that Pierre had once felt for his
wife, nor was it the romantic love stimulated by himself that he
experienced for Natasha. (Ramballe despised both these kinds of
love equally: the one he considered the \emph{love of
clodhoppers} and the other the \emph{love of simpletons}.)
L'amour which the Frenchman worshiped consisted principally in
the unnaturalness of his relation to the woman and in a
combination of incongruities giving the chief charm to the
feeling.

Thus the captain touchingly recounted the story of his love for a
fascinating marquise of thirty-five and at the same time for a
charming, innocent child of seventeen, daughter of the bewitching
marquise. The conflict of magnanimity between the mother and the
daughter, ending in the mother's sacrificing herself and offering
her daughter in marriage to her lover, even now agitated the
captain, though it was the memory of a distant past. Then he
recounted an episode in which the husband played the part of the
lover, and he---the lover---assumed the role of the husband, as
well as several droll incidents from his recollections of
Germany, where \emph{shelter} is called Unterkunft and where the
husbands eat sauerkraut and the young girls are \emph{too
blonde}.

Finally, the latest episode in Poland still fresh in the
captain's memory, and which he narrated with rapid gestures and
glowing face, was of how he had saved the life of a Pole (in
general, the saving of life continually occurred in the captain's
stories) and the Pole had entrusted to him his enchanting wife
(parisienne de coeur) while himself entering the French
service. The captain was happy, the enchanting Polish lady wished
to elope with him, but, prompted by magnanimity, the captain
restored the wife to the husband, saying as he did so: ``I have
saved your life, and I save your honor!'' Having repeated these
words the captain wiped his eyes and gave himself a shake, as if
driving away the weakness which assailed him at this touching
recollection.

Listening to the captain's tales, Pierre---as often happens late
in the evening and under the influence of wine---followed all
that was told him, understood it all, and at the same time
followed a train of personal memories which, he knew not why,
suddenly arose in his mind. While listening to these love stories
his own love for Natasha unexpectedly rose to his mind, and going
over the pictures of that love in his imagination he mentally
compared them with Ramballe's tales. Listening to the story of
the struggle between love and duty, Pierre saw before his eyes
every minutest detail of his last meeting with the object of his
love at the Sukharev water tower. At the time of that meeting it
had not produced an effect upon him---he had not even once
recalled it. But now it seemed to him that that meeting had had
in it something very important and poetic.

``Peter Kirilovich, come here! We have recognized you,'' he now
seemed to hear the words she had uttered and to see before him
her eyes, her smile, her traveling hood, and a stray lock of her
hair... and there seemed to him something pathetic and touching
in all this.

Having finished his tale about the enchanting Polish lady, the
captain asked Pierre if he had ever experienced a similar impulse
to sacrifice himself for love and a feeling of envy of the
legitimate husband.

Challenged by this question Pierre raised his head and felt a
need to express the thoughts that filled his mind. He began to
explain that he understood love for a women somewhat
differently. He said that in all his life he had loved and still
loved only one woman, and that she could never be his.

``Tiens!'' said the captain.

Pierre then explained that he had loved this woman from his
earliest years, but that he had not dared to think of her because
she was too young, and because he had been an illegitimate son
without a name.  Afterwards when he had received a name and
wealth he dared not think of her because he loved her too well,
placing her far above everything in the world, and especially
therefore above himself.

When he had reached this point, Pierre asked the captain whether
he understood that.

The captain made a gesture signifying that even if he did not
understand it he begged Pierre to continue.

``Platonic love, clouds...'' he muttered.

Whether it was the wine he had drunk, or an impulse of frankness,
or the thought that this man did not, and never would, know any
of those who played a part in his story, or whether it was all
these things together, something loosened Pierre's
tongue. Speaking thickly and with a faraway look in his shining
eyes, he told the whole story of his life: his marriage,
Natasha's love for his best friend, her betrayal of him, and all
his own simple relations with her. Urged on by Ramballe's
questions he also told what he had at first concealed---his own
position and even his name.

More than anything else in Pierre's story the captain was
impressed by the fact that Pierre was very rich, had two mansions
in Moscow, and that he had abandoned everything and not left the
city, but remained there concealing his name and station.

When it was late at night they went out together into the
street. The night was warm and light. To the left of the house on
the Pokrovka a fire glowed---the first of those that were
beginning in Moscow. To the right and high up in the sky was the
sickle of the waning moon and opposite to it hung that bright
comet which was connected in Pierre's heart with his love. At the
gate stood Gerasim, the cook, and two Frenchmen. Their laughter
and their mutually incomprehensible remarks in two languages
could be heard. They were looking at the glow seen in the town.

There was nothing terrible in the one small, distant fire in the
immense city.

Gazing at the high starry sky, at the moon, at the comet, and at
the glow from the fire, Pierre experienced a joyful
emotion. ``There now, how good it is, what more does one need?''
thought he. And suddenly remembering his intention he grew dizzy
and felt so faint that he leaned against the fence to save
himself from falling.

Without taking leave of his new friend, Pierre left the gate with
unsteady steps and returning to his room lay down on the sofa and
immediately fell asleep.

% % % % % % % % % % % % % % % % % % % % % % % % % % % % % % % % %
% % % % % % % % % % % % % % % % % % % % % % % % % % % % % % % % %
% % % % % % % % % % % % % % % % % % % % % % % % % % % % % % % % %
% % % % % % % % % % % % % % % % % % % % % % % % % % % % % % % % %
% % % % % % % % % % % % % % % % % % % % % % % % % % % % % % % % %
% % % % % % % % % % % % % % % % % % % % % % % % % % % % % % % % %
% % % % % % % % % % % % % % % % % % % % % % % % % % % % % % % % %
% % % % % % % % % % % % % % % % % % % % % % % % % % % % % % % % %
% % % % % % % % % % % % % % % % % % % % % % % % % % % % % % % % %
% % % % % % % % % % % % % % % % % % % % % % % % % % % % % % % % %
% % % % % % % % % % % % % % % % % % % % % % % % % % % % % % % % %
% % % % % % % % % % % % % % % % % % % % % % % % % % % % % %

\chapter*{Chapter XXX} \ifaudio \marginpar{
\href{http://ia600205.us.archive.org/6/items/war_and_peace_11_0909/war_and_peace_11_30_tolstoy_64kb.mp3}{Audio}}
\fi

\initial{T}{he} glow of the first fire that began on the second of September
was watched from the various roads by the fugitive Muscovites and
by the retreating troops, with many different feelings.

The Rostov party spent the night at Mytishchi, fourteen miles
from Moscow. They had started so late on the first of September,
the road had been so blocked by vehicles and troops, so many
things had been forgotten for which servants were sent back, that
they had decided to spend that night at a place three miles out
of Moscow. The next morning they woke late and were again delayed
so often that they only got as far as Great Mytishchi. At ten
o'clock that evening the Rostov family and the wounded traveling
with them were all distributed in the yards and huts of that
large village. The Rostovs' servants and coachmen and the
orderlies of the wounded officers, after attending to their
masters, had supper, fed the horses, and came out into the
porches.

In a neighboring hut lay Raevski's adjutant with a fractured
wrist. The awful pain he suffered made him moan incessantly and
piteously, and his moaning sounded terrible in the darkness of
the autumn night. He had spent the first night in the same yard
as the Rostovs. The countess said she had been unable to close
her eyes on account of his moaning, and at Mytishchi she moved
into a worse hut simply to be farther away from the wounded man.

In the darkness of the night one of the servants noticed, above
the high body of a coach standing before the porch, the small
glow of another fire. One glow had long been visible and
everybody knew that it was Little Mytishchi burning---set on fire
by Mamonov's Cossacks.

``But look here, brothers, there's another fire!'' remarked an
orderly.

All turned their attention to the glow.

``But they told us Little Mytishchi had been set on fire by
Mamonov's Cossacks.''

``But that's not Mytishchi, it's farther away.''

``Look, it must be in Moscow!''

Two of the gazers went round to the other side of the coach and
sat down on its steps.

``It's more to the left, why, Little Mytishchi is over there, and
this is right on the other side.''

Several men joined the first two.

``See how it's flaring,'' said one. ``That's a fire in Moscow:
either in the Sushchevski or the Rogozhski quarter.''

No one replied to this remark and for some time they all gazed
silently at the spreading flames of the second fire in the
distance.

Old Daniel Terentich, the count's valet (as he was called), came
up to the group and shouted at Mishka.

``What are you staring at, you good-for-nothing?... The count
will be calling and there's nobody there; go and gather the
clothes together.''

``I only ran out to get some water,'' said Mishka.

``But what do you think, Daniel Terentich? Doesn't it look as if
that glow were in Moscow?'' remarked one of the footmen.

Daniel Terentich made no reply, and again for a long time they
were all silent. The glow spread, rising and falling, farther and
farther still.

``God have mercy... It's windy and dry...'' said another voice.

``Just look! See what it's doing now. O Lord! You can even see
the crows flying. Lord have mercy on us sinners!''

``They'll put it out, no fear!''

``Who's to put it out?'' Daniel Terentich, who had hitherto been
silent, was heard to say. His voice was calm and
deliberate. ``Moscow it is, brothers,'' said he. ``Mother Moscow,
the white...'' his voice faltered, and he gave way to an old
man's sob.

And it was as if they had all only waited for this to realize the
significance for them of the glow they were watching. Sighs were
heard, words of prayer, and the sobbing of the count's old valet.

% % % % % % % % % % % % % % % % % % % % % % % % % % % % % % % % %
% % % % % % % % % % % % % % % % % % % % % % % % % % % % % % % % %
% % % % % % % % % % % % % % % % % % % % % % % % % % % % % % % % %
% % % % % % % % % % % % % % % % % % % % % % % % % % % % % % % % %
% % % % % % % % % % % % % % % % % % % % % % % % % % % % % % % % %
% % % % % % % % % % % % % % % % % % % % % % % % % % % % % % % % %
% % % % % % % % % % % % % % % % % % % % % % % % % % % % % % % % %
% % % % % % % % % % % % % % % % % % % % % % % % % % % % % % % % %
% % % % % % % % % % % % % % % % % % % % % % % % % % % % % % % % %
% % % % % % % % % % % % % % % % % % % % % % % % % % % % % % % % %
% % % % % % % % % % % % % % % % % % % % % % % % % % % % % % % % %
% % % % % % % % % % % % % % % % % % % % % % % % % % % % % %

\chapter*{Chapter XXXI} \ifaudio \marginpar{
\href{http://ia600205.us.archive.org/6/items/war_and_peace_11_0909/war_and_peace_11_31_tolstoy_64kb.mp3}{Audio}}
\fi

\initial{T}{he} valet, returning to the cottage, informed the count that
Moscow was burning. The count donned his dressing gown and went
out to look. Sonya and Madame Schoss, who had not yet undressed,
went out with him. Only Natasha and the countess remained in the
room. Petya was no longer with the family, he had gone on with
his regiment which was making for Troitsa.

The countess, on hearing that Moscow was on fire, began to
cry. Natasha, pale, with a fixed look, was sitting on the bench
under the icons just where she had sat down on arriving and paid
no attention to her father's words. She was listening to the
ceaseless moaning of the adjutant, three houses off.

``Oh, how terrible,'' said Sonya returning from the yard chilled
and frightened. ``I believe the whole of Moscow will burn,
there's an awful glow! Natasha, do look! You can see it from the
window,'' she said to her cousin, evidently wishing to distract
her mind.

But Natasha looked at her as if not understanding what was said
to her and again fixed her eyes on the corner of the stove. She
had been in this condition of stupor since the morning, when
Sonya, to the surprise and annoyance of the countess, had for
some unaccountable reason found it necessary to tell Natasha of
Prince Andrew's wound and of his being with their party. The
countess had seldom been so angry with anyone as she was with
Sonya. Sonya had cried and begged to be forgiven and now, as if
trying to atone for her fault, paid unceasing attention to her
cousin.

``Look, Natasha, how dreadfully it is burning!'' said she.

``What's burning?'' asked Natasha. ``Oh, yes, Moscow.''

And as if in order not to offend Sonya and to get rid of her, she
turned her face to the window, looked out in such a way that it
was evident that she could not see anything, and again settled
down in her former attitude.

``But you didn't see it!''

``Yes, really I did,'' Natasha replied in a voice that pleaded to
be left in peace.

Both the countess and Sonya understood that, naturally, neither
Moscow nor the burning of Moscow nor anything else could seem of
importance to Natasha.

The count returned and lay down behind the partition. The
countess went up to her daughter and touched her head with the
back of her hand as she was wont to do when Natasha was ill, then
touched her forehead with her lips as if to feel whether she was
feverish, and finally kissed her.

``You are cold. You are trembling all over. You'd better lie
down,'' said the countess.

``Lie down? All right, I will. I'll lie down at once,'' said
Natasha.

When Natasha had been told that morning that Prince Andrew was
seriously wounded and was traveling with their party, she had at
first asked many questions: Where was he going? How was he
wounded? Was it serious? And could she see him? But after she had
been told that she could not see him, that he was seriously
wounded but that his life was not in danger, she ceased to ask
questions or to speak at all, evidently disbelieving what they
told her, and convinced that say what she might she would still
be told the same. All the way she had sat motionless in a corner
of the coach with wide open eyes, and the expression in them
which the countess knew so well and feared so much, and now she
sat in the same way on the bench where she had seated herself on
arriving. She was planning something and either deciding or had
already decided something in her mind. The countess knew this,
but what it might be she did not know, and this alarmed and
tormented her.

``Natasha, undress, darling; lie down on my bed.''

A bed had been made on a bedstead for the countess only. Madame
Schoss and the two girls were to sleep on some hay on the floor.

``No, Mamma, I will lie down here on the floor,'' Natasha replied
irritably and she went to the window and opened it. Through the
open window the moans of the adjutant could be heard more
distinctly. She put her head out into the damp night air, and the
countess saw her slim neck shaking with sobs and throbbing
against the window frame. Natasha knew it was not Prince Andrew
who was moaning. She knew Prince Andrew was in the same yard as
themselves and in a part of the hut across the passage; but this
dreadful incessant moaning made her sob. The countess exchanged a
look with Sonya.

``Lie down, darling; lie down, my pet,'' said the countess,
softly touching Natasha's shoulders. ``Come, lie down.''

``Oh, yes... I'll lie down at once,'' said Natasha, and began
hurriedly undressing, tugging at the tapes of her petticoat.

When she had thrown off her dress and put on a dressing jacket,
she sat down with her foot under her on the bed that had been
made up on the floor, jerked her thin and rather short plait of
hair to the front, and began replaiting it. Her long, thin,
practiced fingers rapidly unplaited, replaited, and tied up her
plait. Her head moved from side to side from habit, but her eyes,
feverishly wide, looked fixedly before her. When her toilet for
the night was finished she sank gently onto the sheet spread over
the hay on the side nearest the door.

``Natasha, you'd better lie in the middle,'' said Sonya.

``I'll stay here,'' muttered Natasha. ``Do lie down,'' she added
crossly, and buried her face in the pillow.

The countess, Madame Schoss, and Sonya undressed hastily and lay
down.  The small lamp in front of the icons was the only light
left in the room. But in the yard there was a light from the fire
at Little Mytishchi a mile and a half away, and through the night
came the noise of people shouting at a tavern Mamonov's Cossacks
had set up across the street, and the adjutant's unceasing moans
could still be heard.

For a long time Natasha listened attentively to the sounds that
reached her from inside and outside the room and did not
move. First she heard her mother praying and sighing and the
creaking of her bed under her, then Madame Schoss' familiar
whistling snore and Sonya's gentle breathing. Then the countess
called to Natasha. Natasha did not answer.

``I think she's asleep, Mamma,'' said Sonya softly.

After a short silence the countess spoke again but this time no
one replied.

Soon after that Natasha heard her mother's even
breathing. Natasha did not move, though her little bare foot,
thrust out from under the quilt, was growing cold on the bare
floor.

As if to celebrate a victory over everybody, a cricket chirped in
a crack in the wall. A cock crowed far off and another replied
near by.  The shouting in the tavern had died down; only the
moaning of the adjutant was heard. Natasha sat up.

``Sonya, are you asleep? Mamma?'' she whispered.

No one replied. Natasha rose slowly and carefully, crossed
herself, and stepped cautiously on the cold and dirty floor with
her slim, supple, bare feet. The boards of the floor
creaked. Stepping cautiously from one foot to the other she ran
like a kitten the few steps to the door and grasped the cold door
handle.

It seemed to her that something heavy was beating rhythmically
against all the walls of the room: it was her own heart, sinking
with alarm and terror and overflowing with love.

She opened the door and stepped across the threshold and onto the
cold, damp earthen floor of the passage. The cold she felt
refreshed her. With her bare feet she touched a sleeping man,
stepped over him, and opened the door into the part of the hut
where Prince Andrew lay. It was dark in there. In the farthest
corner, on a bench beside a bed on which something was lying,
stood a tallow candle with a long, thick, and smoldering wick.

From the moment she had been told that morning of Prince Andrew's
wound and his presence there, Natasha had resolved to see
him. She did not know why she had to, she knew the meeting would
be painful, but felt the more convinced that it was necessary.

All day she had lived only in hope of seeing him that night. But
now that the moment had come she was filled with dread of what
she might see. How was he maimed? What was left of him? Was he
like that incessant moaning of the adjutant's? Yes, he was
altogether like that. In her imagination he was that terrible
moaning personified. When she saw an indistinct shape in the
corner, and mistook his knees raised under the quilt for his
shoulders, she imagined a horrible body there, and stood still in
terror. But an irresistible impulse drew her forward. She
cautiously took one step and then another, and found herself in
the middle of a small room containing baggage. Another
man---Timokhin---was lying in a corner on the benches beneath the
icons, and two others---the doctor and a valet---lay on the
floor.

The valet sat up and whispered something. Timokhin, kept awake by
the pain in his wounded leg, gazed with wide-open eyes at this
strange apparition of a girl in a white chemise, dressing jacket,
and nightcap.  The valet's sleepy, frightened exclamation, ``What
do you want? What's the matter?'' made Natasha approach more
swiftly to what was lying in the corner. Horribly unlike a man as
that body looked, she must see him. She passed the valet, the
snuff fell from the candle wick, and she saw Prince Andrew
clearly with his arms outside the quilt, and such as she had
always seen him.

He was the same as ever, but the feverish color of his face, his
glittering eyes rapturously turned toward her, and especially his
neck, delicate as a child's, revealed by the turn-down collar of
his shirt, gave him a peculiarly innocent, childlike look, such
as she had never seen on him before. She went up to him and with
a swift, flexible, youthful movement dropped on her knees.

He smiled and held out his hand to her.

% % % % % % % % % % % % % % % % % % % % % % % % % % % % % % % % %
% % % % % % % % % % % % % % % % % % % % % % % % % % % % % % % % %
% % % % % % % % % % % % % % % % % % % % % % % % % % % % % % % % %
% % % % % % % % % % % % % % % % % % % % % % % % % % % % % % % % %
% % % % % % % % % % % % % % % % % % % % % % % % % % % % % % % % %
% % % % % % % % % % % % % % % % % % % % % % % % % % % % % % % % %
% % % % % % % % % % % % % % % % % % % % % % % % % % % % % % % % %
% % % % % % % % % % % % % % % % % % % % % % % % % % % % % % % % %
% % % % % % % % % % % % % % % % % % % % % % % % % % % % % % % % %
% % % % % % % % % % % % % % % % % % % % % % % % % % % % % % % % %
% % % % % % % % % % % % % % % % % % % % % % % % % % % % % % % % %
% % % % % % % % % % % % % % % % % % % % % % % % % % % % % %

\chapter*{Chapter XXXII} \ifaudio \marginpar{
\href{http://ia600205.us.archive.org/6/items/war_and_peace_11_0909/war_and_peace_11_32_tolstoy_64kb.mp3}{Audio}}
\fi

\initial{S}{even} days had passed since Prince Andrew found himself in the
ambulance station on the field of Borodino. His feverish state
and the inflammation of his bowels, which were injured, were in
the doctor's opinion sure to carry him off. But on the seventh
day he ate with pleasure a piece of bread with some tea, and the
doctor noticed that his temperature was lower. He had regained
consciousness that morning. The first night after they left
Moscow had been fairly warm and he had remained in the caleche,
but at Mytishchi the wounded man himself asked to be taken out
and given some tea. The pain caused by his removal into the hut
had made him groan aloud and again lose consciousness. When he
had been placed on his camp bed he lay for a long time motionless
with closed eyes. Then he opened them and whispered softly: ``And
the tea?''  His remembering such a small detail of everyday life
astonished the doctor. He felt Prince Andrew's pulse, and to his
surprise and dissatisfaction found it had improved. He was
dissatisfied because he knew by experience that if his patient
did not die now, he would do so a little later with greater
suffering. Timokhin, the red-nosed major of Prince Andrew's
regiment, had joined him in Moscow and was being taken along with
him, having been wounded in the leg at the battle of
Borodino. They were accompanied by a doctor, Prince Andrew's
valet, his coachman, and two orderlies.

They gave Prince Andrew some tea. He drank it eagerly, looking
with feverish eyes at the door in front of him as if trying to
understand and remember something.

``I don't want any more. Is Timokhin here?'' he asked.

Timokhin crept along the bench to him.

``I am here, your excellency.''

``How's your wound?''

``Mine, sir? All right. But how about you?''

Prince Andrew again pondered as if trying to remember something.

``Couldn't one get a book?'' he asked.

``What book?''

``The Gospels. I haven't one.''

The doctor promised to procure it for him and began to ask how he
was feeling. Prince Andrew answered all his questions reluctantly
but reasonably, and then said he wanted a bolster placed under
him as he was uncomfortable and in great pain. The doctor and
valet lifted the cloak with which he was covered and, making wry
faces at the noisome smell of mortifying flesh that came from the
wound, began examining that dreadful place. The doctor was very
much displeased about something and made a change in the
dressings, turning the wounded man over so that he groaned again
and grew unconscious and delirious from the agony. He kept asking
them to get him the book and put it under him.

``What trouble would it be to you?'' he said. ``I have not got
one. Please get it for me and put it under for a moment,'' he
pleaded in a piteous voice.

The doctor went into the passage to wash his hands.

``You fellows have no conscience,'' said he to the valet who was
pouring water over his hands. ``For just one moment I didn't look
after you...  It's such pain, you know, that I wonder how he can
bear it.''

``By the Lord Jesus Christ, I thought we had put something under
him!''  said the valet.

The first time Prince Andrew understood where he was and what was
the matter with him and remembered being wounded and how was when
he asked to be carried into the hut after his caleche had stopped
at Mytishchi.  After growing confused from pain while being
carried into the hut he again regained consciousness, and while
drinking tea once more recalled all that had happened to him, and
above all vividly remembered the moment at the ambulance station
when, at the sight of the sufferings of a man he disliked, those
new thoughts had come to him which promised him happiness. And
those thoughts, though now vague and indefinite, again possessed
his soul. He remembered that he had now a new source of happiness
and that this happiness had something to do with the Gospels.
That was why he asked for a copy of them. The uncomfortable
position in which they had put him and turned him over again
confused his thoughts, and when he came to himself a third time
it was in the complete stillness of the night. Everybody near him
was sleeping. A cricket chirped from across the passage; someone
was shouting and singing in the street; cockroaches rustled on
the table, on the icons, and on the walls, and a big fly flopped
at the head of the bed and around the candle beside him, the wick
of which was charred and had shaped itself like a mushroom.

His mind was not in a normal state. A healthy man usually thinks
of, feels, and remembers innumerable things simultaneously, but
has the power and will to select one sequence of thoughts or
events on which to fix his whole attention. A healthy man can
tear himself away from the deepest reflections to say a civil
word to someone who comes in and can then return again to his own
thoughts. But Prince Andrew's mind was not in a normal state in
that respect. All the powers of his mind were more active and
clearer than ever, but they acted apart from his will. Most
diverse thoughts and images occupied him simultaneously. At times
his brain suddenly began to work with a vigor, clearness, and
depth it had never reached when he was in health, but suddenly in
the midst of its work it would turn to some unexpected idea and
he had not the strength to turn it back again.

``Yes, a new happiness was revealed to me of which man cannot be
deprived,'' he thought as he lay in the semidarkness of the quiet
hut, gazing fixedly before him with feverish wide open eyes. ``A
happiness lying beyond material forces, outside the material
influences that act on man---a happiness of the soul alone, the
happiness of loving. Every man can understand it, but to conceive
it and enjoin it was possible only for God. But how did God
enjoin that law? And why was the Son...?''

And suddenly the sequence of these thoughts broke off, and Prince
Andrew heard (without knowing whether it was a delusion or
reality) a soft whispering voice incessantly and rhythmically
repeating \emph{piti-piti-piti}, and then \emph{titi}, and then
again \emph{piti-piti-piti}, and \emph{ti-ti} once more. At the
same time he felt that above his face, above the very middle of
it, some strange airy structure was being erected out of slender
needles or splinters, to the sound of this whispered music. He
felt that he had to balance carefully (though it was difficult)
so that this airy structure should not collapse; but nevertheless
it kept collapsing and again slowly rising to the sound of
whispered rhythmic music---``it stretches, stretches, spreading
out and stretching,'' said Prince Andrew to himself. While
listening to this whispering and feeling the sensation of this
drawing out and the construction of this edifice of needles, he
also saw by glimpses a red halo round the candle, and heard the
rustle of the cockroaches and the buzzing of the fly that flopped
against his pillow and his face. Each time the fly touched his
face it gave him a burning sensation and yet to his surprise it
did not destroy the structure, though it knocked against the very
region of his face where it was rising. But besides this there
was something else of importance. It was something white by the
door---the statue of a sphinx, which also oppressed him.

``But perhaps that's my shirt on the table,'' he thought, ``and
that's my legs, and that is the door, but why is it always
stretching and drawing itself out, and 'piti-piti-piti' and
'ti-ti' and 'piti-piti-piti'...?  That's enough, please leave
off!'' Prince Andrew painfully entreated someone. And suddenly
thoughts and feelings again swam to the surface of his mind with
peculiar clearness and force.

``Yes---love,'' he thought again quite clearly. ``But not love
which loves for something, for some quality, for some purpose, or
for some reason, but the love which I---while dying---first
experienced when I saw my enemy and yet loved him. I experienced
that feeling of love which is the very essence of the soul and
does not require an object. Now again I feel that bliss. To love
one's neighbors, to love one's enemies, to love everything, to
love God in all His manifestations. It is possible to love
someone dear to you with human love, but an enemy can only be
loved by divine love. That is why I experienced such joy when I
felt that I loved that man. What has become of him? Is he
alive?...''

``When loving with human love one may pass from love to hatred,
but divine love cannot change. No, neither death nor anything
else can destroy it. It is the very essence of the soul. Yet how
many people have I hated in my life? And of them all, I loved and
hated none as I did her.'' And he vividly pictured to himself
Natasha, not as he had done in the past with nothing but her
charms which gave him delight, but for the first time picturing
to himself her soul. And he understood her feelings, her
sufferings, shame, and remorse. He now understood for the first
time all the cruelty of his rejection of her, the cruelty of his
rupture with her. ``If only it were possible for me to see her
once more!  Just once, looking into those eyes to say...''

\emph{Piti-piti-piti and ti-ti and piti-piti-piti boom!} flopped
the fly...  And his attention was suddenly carried into another
world, a world of reality and delirium in which something
particular was happening. In that world some structure was still
being erected and did not fall, something was still stretching
out, and the candle with its red halo was still burning, and the
same shirtlike sphinx lay near the door; but besides all this
something creaked, there was a whiff of fresh air, and a new
white sphinx appeared, standing at the door. And that sphinx had
the pale face and shining eyes of the very Natasha of whom he had
just been thinking.

``Oh, how oppressive this continual delirium is,'' thought Prince
Andrew, trying to drive that face from his imagination. But the
face remained before him with the force of reality and drew
nearer. Prince Andrew wished to return to that former world of
pure thought, but he could not, and delirium drew him back into
its domain. The soft whispering voice continued its rhythmic
murmur, something oppressed him and stretched out, and the
strange face was before him. Prince Andrew collected all his
strength in an effort to recover his senses, he moved a little,
and suddenly there was a ringing in his ears, a dimness in his
eyes, and like a man plunged into water he lost
consciousness. When he came to himself, Natasha, that same living
Natasha whom of all people he most longed to love with this new
pure divine love that had been revealed to him, was kneeling
before him. He realized that it was the real living Natasha, and
he was not surprised but quietly happy. Natasha, motionless on
her knees (she was unable to stir), with frightened eyes riveted
on him, was restraining her sobs. Her face was pale and
rigid. Only in the lower part of it something quivered.

Prince Andrew sighed with relief, smiled, and held out his hand.

``You?'' he said. ``How fortunate!''

With a rapid but careful movement Natasha drew nearer to him on
her knees and, taking his hand carefully, bent her face over it
and began kissing it, just touching it lightly with her lips.

``Forgive me!'' she whispered, raising her head and glancing at
him.  ``Forgive me!''

``I love you,'' said Prince Andrew.

``Forgive...!''

``Forgive what?'' he asked.

``Forgive me for what I ha-ve do-ne!'' faltered Natasha in a
scarcely audible, broken whisper, and began kissing his hand more
rapidly, just touching it with her lips.

``I love you more, better than before,'' said Prince Andrew,
lifting her face with his hand so as to look into her eyes.

Those eyes, filled with happy tears, gazed at him timidly,
compassionately, and with joyous love. Natasha's thin pale face,
with its swollen lips, was more than plain---it was dreadful. But
Prince Andrew did not see that, he saw her shining eyes which
were beautiful.  They heard the sound of voices behind them.

Peter the valet, who was now wide awake, had roused the doctor.
Timokhin, who had not slept at all because of the pain in his
leg, had long been watching all that was going on, carefully
covering his bare body with the sheet as he huddled up on his
bench.

``What's this?'' said the doctor, rising from his bed. ``Please
go away, madam!''

At that moment a maid sent by the countess, who had noticed her
daughter's absence, knocked at the door.

Like a somnambulist aroused from her sleep Natasha went out of
the room and, returning to her hut, fell sobbing on her bed.

From that time, during all the rest of the Rostovs' journey, at
every halting place and wherever they spent a night, Natasha
never left the wounded Bolkonski, and the doctor had to admit
that he had not expected from a young girl either such firmness
or such skill in nursing a wounded man.

Dreadful as the countess imagined it would be should Prince
Andrew die in her daughter's arms during the journey---as,
judging by what the doctor said, it seemed might easily
happen---she could not oppose Natasha. Though with the intimacy
now established between the wounded man and Natasha the thought
occurred that should he recover their former engagement would be
renewed, no one---least of all Natasha and Prince Andrew---spoke
of this: the unsettled question of life and death, which hung not
only over Bolkonski but over all Russia, shut out all other
considerations.

% % % % % % % % % % % % % % % % % % % % % % % % % % % % % % % % %
% % % % % % % % % % % % % % % % % % % % % % % % % % % % % % % % %
% % % % % % % % % % % % % % % % % % % % % % % % % % % % % % % % %
% % % % % % % % % % % % % % % % % % % % % % % % % % % % % % % % %
% % % % % % % % % % % % % % % % % % % % % % % % % % % % % % % % %
% % % % % % % % % % % % % % % % % % % % % % % % % % % % % % % % %
% % % % % % % % % % % % % % % % % % % % % % % % % % % % % % % % %
% % % % % % % % % % % % % % % % % % % % % % % % % % % % % % % % %
% % % % % % % % % % % % % % % % % % % % % % % % % % % % % % % % %
% % % % % % % % % % % % % % % % % % % % % % % % % % % % % % % % %
% % % % % % % % % % % % % % % % % % % % % % % % % % % % % % % % %
% % % % % % % % % % % % % % % % % % % % % % % % % % % % % %

\chapter*{Chapter XXXIII} \ifaudio \marginpar{
\href{http://ia600205.us.archive.org/6/items/war_and_peace_11_0909/war_and_peace_11_33_tolstoy_64kb.mp3}{Audio}}
\fi

\initial{O}{n} the third of September Pierre awoke late. His head was aching,
the clothes in which he had slept without undressing felt
uncomfortable on his body, and his mind had a dim consciousness
of something shameful he had done the day before. That something
shameful was his yesterday's conversation with Captain Ramballe.

It was eleven by the clock, but it seemed peculiarly dark out of
doors.  Pierre rose, rubbed his eyes, and seeing the pistol with
an engraved stock which Gerasim had replaced on the writing
table, he remembered where he was and what lay before him that
very day.

``Am I not too late?'' he thought. ``No, probably he won't make
his entry into Moscow before noon.''

Pierre did not allow himself to reflect on what lay before him,
but hastened to act.

After arranging his clothes, he took the pistol and was about to
go out.  But it then occurred to him for the first time that he
certainly could not carry the weapon in his hand through the
streets. It was difficult to hide such a big pistol even under
his wide coat. He could not carry it unnoticed in his belt or
under his arm. Besides, it had been discharged, and he had not
had time to reload it. ``No matter, dagger will do,'' he said to
himself, though when planning his design he had more than once
come to the conclusion that the chief mistake made by the student
in 1809 had been to try to kill Napoleon with a dagger. But as
his chief aim consisted not in carrying out his design, but in
proving to himself that he would not abandon his intention and
was doing all he could to achieve it, Pierre hastily took the
blunt jagged dagger in a green sheath which he had bought at the
Sukharev market with the pistol, and hid it under his waistcoat.

Having tied a girdle over his coat and pulled his cap low on his
head, Pierre went down the corridor, trying to avoid making a
noise or meeting the captain, and passed out into the street.

The conflagration, at which he had looked with so much
indifference the evening before, had greatly increased during the
night. Moscow was on fire in several places. The buildings in
Carriage Row, across the river, in the Bazaar and the Povarskoy,
as well as the barges on the Moskva River and the timber yards by
the Dorogomilov Bridge, were all ablaze.

Pierre's way led through side streets to the Povarskoy and from
there to the church of St. Nicholas on the Arbat, where he had
long before decided that the deed should be done. The gates of
most of the houses were locked and the shutters up. The streets
and lanes were deserted.  The air was full of smoke and the smell
of burning. Now and then he met Russians with anxious and timid
faces, and Frenchmen with an air not of the city but of the camp,
walking in the middle of the streets. Both the Russians and the
French looked at Pierre with surprise. Besides his height and
stoutness, and the strange morose look of suffering in his face
and whole figure, the Russians stared at Pierre because they
could not make out to what class he could belong. The French
followed him with astonishment in their eyes chiefly because
Pierre, unlike all the other Russians who gazed at the French
with fear and curiosity, paid no attention to them. At the gate
of one house three Frenchmen, who were explaining something to
some Russians who did not understand them, stopped Pierre asking
if he did not know French.

Pierre shook his head and went on. In another side street a
sentinel standing beside a green caisson shouted at him, but only
when the shout was threateningly repeated and he heard the click
of the man's musket as he raised it did Pierre understand that he
had to pass on the other side of the street. He heard nothing and
saw nothing of what went on around him. He carried his resolution
within himself in terror and haste, like something dreadful and
alien to him, for, after the previous night's experience, he was
afraid of losing it. But he was not destined to bring his mood
safely to his destination. And even had he not been hindered by
anything on the way, his intention could not now have been
carried out, for Napoleon had passed the Arbat more than four
hours previously on his way from the Dorogomilov suburb to the
Kremlin, and was now sitting in a very gloomy frame of mind in a
royal study in the Kremlin, giving detailed and exact orders as
to measures to be taken immediately to extinguish the fire, to
prevent looting, and to reassure the inhabitants. But Pierre did
not know this; he was entirely absorbed in what lay before him,
and was tortured---as those are who obstinately undertake a task
that is impossible for them not because of its difficulty but
because of its incompatibility with their natures---by the fear
of weakening at the decisive moment and so losing his
self-esteem.

Though he heard and saw nothing around him he found his way by
instinct and did not go wrong in the side streets that led to the
Povarskoy.

As Pierre approached that street the smoke became denser and
denser---he even felt the heat of the fire. Occasionally curly
tongues of flame rose from under the roofs of the houses. He met
more people in the streets and they were more excited. But
Pierre, though he felt that something unusual was happening
around him, did not realize that he was approaching the fire. As
he was going along a foot path across a wide-open space adjoining
the Povarskoy on one side and the gardens of Prince Gruzinski's
house on the other, Pierre suddenly heard the desperate weeping
of a woman close to him. He stopped as if awakening from a dream
and lifted his head.

By the side of the path, on the dusty dry grass, all sorts of
household goods lay in a heap: featherbeds, a samovar, icons, and
trunks. On the ground, beside the trunks, sat a thin woman no
longer young, with long, prominent upper teeth, and wearing a
black cloak and cap. This woman, swaying to and fro and muttering
something, was choking with sobs. Two girls of about ten and
twelve, dressed in dirty short frocks and cloaks, were staring at
their mother with a look of stupefaction on their pale frightened
faces. The youngest child, a boy of about seven, who wore an
overcoat and an immense cap evidently not his own, was crying in
his old nurse's arms. A dirty, barefooted maid was sitting on a
trunk, and, having undone her pale-colored plait, was pulling it
straight and sniffing at her singed hair. The woman's husband, a
short, round-shouldered man in the undress uniform of a civilian
official, with sausage-shaped whiskers and showing under his
square-set cap the hair smoothly brushed forward over his
temples, with expressionless face was moving the trunks, which
were placed one on another, and was dragging some garments from
under them.

As soon as she saw Pierre, the woman almost threw herself at his
feet.

``Dear people, good Christians, save me, help me, dear
friends... help us, somebody,'' she muttered between her
sobs. ``My girl... My daughter!  My youngest daughter is left
behind. She's burned! Ooh! Was it for this I nursed you... Ooh!''

``Don't, Mary Nikolievna!'' said her husband to her in a low
voice, evidently only to justify himself before the
stranger. ``Sister must have taken her, or else where can she
be?'' he added.

``Monster! Villain!'' shouted the woman angrily, suddenly ceasing
to weep.  ``You have no heart, you don't feel for your own child!
Another man would have rescued her from the fire. But this is a
monster and neither a man nor a father! You, honored sir, are a
noble man,'' she went on, addressing Pierre rapidly between her
sobs. ``The fire broke out alongside, and blew our way, the maid
called out 'Fire!' and we rushed to collect our things. We ran
out just as we were... This is what we have brought away... The
icons, and my dowry bed, all the rest is lost.  We seized the
children. But not Katie! Ooh! O Lord!...'' and again she began to
sob. ``My child, my dear one! Burned, burned!''

``But where was she left?'' asked Pierre.

From the expression of his animated face the woman saw that this
man might help her.

``Oh, dear sir!'' she cried, seizing him by the legs. ``My
benefactor, set my heart at ease... Aniska, go, you horrid girl,
show him the way!'' she cried to the maid, angrily opening her
mouth and still farther exposing her long teeth.

``Show me the way, show me, I... I'll do it,'' gasped Pierre
rapidly.

The dirty maidservant stepped from behind the trunk, put up her
plait, sighed, and went on her short, bare feet along the
path. Pierre felt as if he had come back to life after a heavy
swoon. He held his head higher, his eyes shone with the light of
life, and with swift steps he followed the maid, overtook her,
and came out on the Povarskoy. The whole street was full of
clouds of black smoke. Tongues of flame here and there broke
through that cloud. A great number of people crowded in front of
the conflagration. In the middle of the street stood a French
general saying something to those around him. Pierre, accompanied
by the maid, was advancing to the spot where the general stood,
but the French soldiers stopped him.

``On ne passe pas!''\footnote{``You can't pass!''} cried a voice.

``This way, uncle,'' cried the girl. ``We'll pass through the
side street, by the Nikulins'!''

Pierre turned back, giving a spring now and then to keep up with
her.  She ran across the street, turned down a side street to the
left, and, passing three houses, turned into a yard on the right.

``It's here, close by,'' said she and, running across the yard,
opened a gate in a wooden fence and, stopping, pointed out to him
a small wooden wing of the house, which was burning brightly and
fiercely. One of its sides had fallen in, another was on fire,
and bright flames issued from the openings of the windows and
from under the roof.

As Pierre passed through the fence gate, he was enveloped by hot
air and involuntarily stopped.

``Which is it? Which is your house?'' he asked.

``Ooh!'' wailed the girl, pointing to the wing. ``That's it, that
was our lodging. You've burned to death, our treasure, Katie, my
precious little missy! Ooh!'' lamented Aniska, who at the sight
of the fire felt that she too must give expression to her
feelings.

Pierre rushed to the wing, but the heat was so great that he
involuntarily passed round in a curve and came upon the large
house that was as yet burning only at one end, just below the
roof, and around which swarmed a crowd of Frenchmen. At first
Pierre did not realize what these men, who were dragging
something out, were about; but seeing before him a Frenchman
hitting a peasant with a blunt saber and trying to take from him
a fox-fur coat, he vaguely understood that looting was going on
there, but he had no time to dwell on that idea.

The sounds of crackling and the din of falling walls and
ceilings, the whistle and hiss of the flames, the excited shouts
of the people, and the sight of the swaying smoke, now gathering
into thick black clouds and now soaring up with glittering
sparks, with here and there dense sheaves of flame (now red and
now like golden fish scales creeping along the walls), and the
heat and smoke and rapidity of motion, produced on Pierre the
usual animating effects of a conflagration. It had a peculiarly
strong effect on him because at the sight of the fire he felt
himself suddenly freed from the ideas that had weighed him
down. He felt young, bright, adroit, and resolute. He ran round
to the other side of the lodge and was about to dash into that
part of it which was still standing, when just above his head he
heard several voices shouting and then a cracking sound and the
ring of something heavy falling close beside him.

Pierre looked up and saw at a window of the large house some
Frenchmen who had just thrown out the drawer of a chest, filled
with metal articles. Other French soldiers standing below went up
to the drawer.

``What does this fellow want?'' shouted one of them referring to
Pierre.

``There's a child in that house. Haven't you seen a child?''
cried Pierre.

``What's he talking about? Get along!'' said several voices, and
one of the soldiers, evidently afraid that Pierre might want to
take from them some of the plate and bronzes that were in the
drawer, moved threateningly toward him.

``A child?'' shouted a Frenchman from above. ``I did hear
something squealing in the garden. Perhaps it's his brat that the
fellow is looking for. After all, one must be human, you
know...''

``Where is it? Where?'' said Pierre.

``There! There!'' shouted the Frenchman at the window, pointing
to the garden at the back of the house. ``Wait a bit---I'm coming
down.''

And a minute or two later the Frenchman, a black-eyed fellow with
a spot on his cheek, in shirt sleeves, really did jump out of a
window on the ground floor, and clapping Pierre on the shoulder
ran with him into the garden.

``Hurry up, you others!'' he called out to his comrades. ``It's
getting hot.''

When they reached a gravel path behind the house the Frenchman
pulled Pierre by the arm and pointed to a round, graveled space
where a three-year-old girl in a pink dress was lying under a
seat.

``There is your child! Oh, a girl, so much the better!'' said the
Frenchman. ``Good-bye, Fatty. We must be human, we are all mortal
you know!'' and the Frenchman with the spot on his cheek ran back
to his comrades.

Breathless with joy, Pierre ran to the little girl and was going
to take her in his arms. But seeing a stranger the sickly,
scrofulous-looking child, unattractively like her mother, began
to yell and run away.  Pierre, however, seized her and lifted her
in his arms. She screamed desperately and angrily and tried with
her little hands to pull Pierre's hands away and to bite them
with her slobbering mouth. Pierre was seized by a sense of horror
and repulsion such as he had experienced when touching some nasty
little animal. But he made an effort not to throw the child down
and ran with her to the large house. It was now, however,
impossible to get back the way he had come; the maid, Aniska, was
no longer there, and Pierre with a feeling of pity and disgust
pressed the wet, painfully sobbing child to himself as tenderly
as he could and ran with her through the garden seeking another
way out.

% % % % % % % % % % % % % % % % % % % % % % % % % % % % % % % % %
% % % % % % % % % % % % % % % % % % % % % % % % % % % % % % % % %
% % % % % % % % % % % % % % % % % % % % % % % % % % % % % % % % %
% % % % % % % % % % % % % % % % % % % % % % % % % % % % % % % % %
% % % % % % % % % % % % % % % % % % % % % % % % % % % % % % % % %
% % % % % % % % % % % % % % % % % % % % % % % % % % % % % % % % %
% % % % % % % % % % % % % % % % % % % % % % % % % % % % % % % % %
% % % % % % % % % % % % % % % % % % % % % % % % % % % % % % % % %
% % % % % % % % % % % % % % % % % % % % % % % % % % % % % % % % %
% % % % % % % % % % % % % % % % % % % % % % % % % % % % % % % % %
% % % % % % % % % % % % % % % % % % % % % % % % % % % % % % % % %
% % % % % % % % % % % % % % % % % % % % % % % % % % % % % %

\chapter*{Chapter XXXIV} \ifaudio \marginpar{
\href{http://ia600205.us.archive.org/6/items/war_and_peace_11_0909/war_and_peace_11_34_tolstoy_64kb.mp3}{Audio}}
\fi

\initial{H}{aving} run through different yards and side streets, Pierre got
back with his little burden to the Gruzinski garden at the corner
of the Povarskoy. He did not at first recognize the place from
which he had set out to look for the child, so crowded was it now
with people and goods that had been dragged out of the
houses. Besides Russian families who had taken refuge here from
the fire with their belongings, there were several French
soldiers in a variety of clothing. Pierre took no notice of
them. He hurried to find the family of that civil servant in
order to restore the daughter to her mother and go to save
someone else. Pierre felt that he had still much to do and to do
quickly. Glowing with the heat and from running, he felt at that
moment more strongly than ever the sense of youth, animation, and
determination that had come on him when he ran to save the
child. She had now become quiet and, clinging with her little
hands to Pierre's coat, sat on his arm gazing about her like some
little wild animal. He glanced at her occasionally with a slight
smile. He fancied he saw something pathetically innocent in that
frightened, sickly little face.

He did not find the civil servant or his wife where he had left
them. He walked among the crowd with rapid steps, scanning the
various faces he met. Involuntarily he noticed a Georgian or
Armenian family consisting of a very handsome old man of Oriental
type, wearing a new, cloth-covered, sheepskin coat and new boots,
an old woman of similar type, and a young woman. That very young
woman seemed to Pierre the perfection of Oriental beauty, with
her sharply outlined, arched, black eyebrows and the
extraordinarily soft, bright color of her long, beautiful,
expressionless face. Amid the scattered property and the crowd on
the open space, she, in her rich satin cloak with a bright lilac
shawl on her head, suggested a delicate exotic plant thrown out
onto the snow.  She was sitting on some bundles a little behind
the old woman, and looked from under her long lashes with
motionless, large, almond-shaped eyes at the ground before
her. Evidently she was aware of her beauty and fearful because of
it. Her face struck Pierre and, hurrying along by the fence, he
turned several times to look at her. When he had reached the
fence, still without finding those he sought, he stopped and
looked about him.

With the child in his arms his figure was now more conspicuous
than before, and a group of Russians, both men and women,
gathered about him.

``Have you lost anyone, my dear fellow? You're of the gentry
yourself, aren't you? Whose child is it?'' they asked him.

Pierre replied that the child belonged to a woman in a black coat
who had been sitting there with her other children, and he asked
whether anyone knew where she had gone.

``Why, that must be the Anferovs,'' said an old deacon,
addressing a pockmarked peasant woman. ``Lord have mercy, Lord
have mercy!'' he added in his customary bass.

``The Anferovs? No,'' said the woman. ``They left in the
morning. That must be either Mary Nikolievna's or the Ivanovs'!''

``He says 'a woman,' and Mary Nikolievna is a lady,'' remarked a
house serf.

``Do you know her? She's thin, with long teeth,'' said Pierre.

``That's Mary Nikolievna! They went inside the garden when these
wolves swooped down,'' said the woman, pointing to the French
soldiers.

``O Lord, have mercy!'' added the deacon.

``Go over that way, they're there. It's she! She kept on
lamenting and crying,'' continued the woman. ``It's she. Here,
this way!''

But Pierre was not listening to the woman. He had for some
seconds been intently watching what was going on a few steps
away. He was looking at the Armenian family and at two French
soldiers who had gone up to them.  One of these, a nimble little
man, was wearing a blue coat tied round the waist with a rope. He
had a nightcap on his head and his feet were bare. The other,
whose appearance particularly struck Pierre, was a long, lank,
round-shouldered, fair-haired man, slow in his movements and with
an idiotic expression of face. He wore a woman's loose gown of
frieze, blue trousers, and large torn Hessian boots. The little
barefooted Frenchman in the blue coat went up to the Armenians
and, saying something, immediately seized the old man by his legs
and the old man at once began pulling off his boots. The other in
the frieze gown stopped in front of the beautiful Armenian girl
and with his hands in his pockets stood staring at her,
motionless and silent.

``Here, take the child!'' said Pierre peremptorily and hurriedly
to the woman, handing the little girl to her. ``Give her back to
them, give her back!'' he almost shouted, putting the child, who
began screaming, on the ground, and again looking at the
Frenchman and the Armenian family.

The old man was already sitting barefoot. The little Frenchman
had secured his second boot and was slapping one boot against the
other. The old man was saying something in a voice broken by
sobs, but Pierre caught but a glimpse of this, his whole
attention was directed to the Frenchman in the frieze gown who
meanwhile, swaying slowly from side to side, had drawn nearer to
the young woman and taking his hands from his pockets had seized
her by the neck.

The beautiful Armenian still sat motionless and in the same
attitude, with her long lashes drooping as if she did not see or
feel what the soldier was doing to her.

While Pierre was running the few steps that separated him from
the Frenchman, the tall marauder in the frieze gown was already
tearing from her neck the necklace the young Armenian was
wearing, and the young woman, clutching at her neck, screamed
piercingly.

``Let that woman alone!'' exclaimed Pierre hoarsely in a furious
voice, seizing the soldier by his round shoulders and throwing
him aside.

The soldier fell, got up, and ran away. But his comrade, throwing
down the boots and drawing his sword, moved threateningly toward
Pierre.

``Voyons, Pas de betises!''\footnote{``Look here, no nonsense!''}
he cried.

Pierre was in such a transport of rage that he remembered nothing
and his strength increased tenfold. He rushed at the barefooted
Frenchman and, before the latter had time to draw his sword,
knocked him off his feet and hammered him with his fists. Shouts
of approval were heard from the crowd around, and at the same
moment a mounted patrol of French uhlans appeared from round the
corner. The uhlans came up at a trot to Pierre and the Frenchman
and surrounded them. Pierre remembered nothing of what happened
after that. He only remembered beating someone and being beaten
and finally feeling that his hands were bound and that a crowd of
French soldiers stood around him and were searching him.

``Lieutenant, he has a dagger,'' were the first words Pierre
understood.

``Ah, a weapon?'' said the officer and turned to the barefooted
soldier who had been arrested with Pierre. ``All right, you can
tell all about it at the court-martial.'' Then he turned to
Pierre. ``Do you speak French?''

Pierre looked around him with bloodshot eyes and did not
reply. His face probably looked very terrible, for the officer
said something in a whisper and four more uhlans left the ranks
and placed themselves on both sides of Pierre.

``Do you speak French?'' the officer asked again, keeping at a
distance from Pierre. ``Call the interpreter.''

A little man in Russian civilian clothes rode out from the ranks,
and by his clothes and manner of speaking Pierre at once knew him
to be a French salesman from one of the Moscow shops.

``He does not look like a common man,'' said the interpreter,
after a searching look at Pierre.

``Ah, he looks very much like an incendiary,'' remarked the
officer. ``And ask him who he is,'' he added.

``Who are you?'' asked the interpreter in poor Russian. ``You
must answer the chief.''

``I will not tell you who I am. I am your prisoner---take me!''
Pierre suddenly replied in French.

``Ah, ah!'' muttered the officer with a frown. ``Well then,
march!''

A crowd had collected round the uhlans. Nearest to Pierre stood
the pockmarked peasant woman with the little girl, and when the
patrol started she moved forward.

``Where are they taking you to, you poor dear?'' said she. ``And
the little girl, the little girl, what am I to do with her if
she's not theirs?''  said the woman.

``What does that woman want?'' asked the officer.

Pierre was as if intoxicated. His elation increased at the sight
of the little girl he had saved.

``What does she want?'' he murmured. ``She is bringing me my
daughter whom I have just saved from the flames,'' said
he. ``Good-bye!'' And without knowing how this aimless lie had
escaped him, he went along with resolute and triumphant steps
between the French soldiers.

The French patrol was one of those sent out through the various
streets of Moscow by Durosnel's order to put a stop to the
pillage, and especially to catch the incendiaries who, according
to the general opinion which had that day originated among the
higher French officers, were the cause of the
conflagrations. After marching through a number of streets the
patrol arrested five more Russian suspects: a small shopkeeper,
two seminary students, a peasant, and a house serf, besides
several looters. But of all these various suspected characters,
Pierre was considered to be the most suspicious of all. When they
had all been brought for the night to a large house on the Zubov
Rampart that was being used as a guardhouse, Pierre was placed
apart under strict guard.

