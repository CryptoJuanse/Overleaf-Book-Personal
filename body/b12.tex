\part*{Book Twelve: 1812}

% % % % % % % % % % % % % % % % % % % % % % % % % % % % % % % % %
% % % % % % % % % % % % % % % % % % % % % % % % % % % % % % % % %
% % % % % % % % % % % % % % % % % % % % % % % % % % % % % % % % %
% % % % % % % % % % % % % % % % % % % % % % % % % % % % % % % % %
% % % % % % % % % % % % % % % % % % % % % % % % % % % % % % % % %
% % % % % % % % % % % % % % % % % % % % % % % % % % % % % % % % %
% % % % % % % % % % % % % % % % % % % % % % % % % % % % % % % % %
% % % % % % % % % % % % % % % % % % % % % % % % % % % % % % % % %
% % % % % % % % % % % % % % % % % % % % % % % % % % % % % % % % %
% % % % % % % % % % % % % % % % % % % % % % % % % % % % % % % % %
% % % % % % % % % % % % % % % % % % % % % % % % % % % % % % % % %
% % % % % % % % % % % % % % % % % % % % % % % % % % % % % %

\chapter*{Chapter I}
\ifaudio 
\marginpar{
\href{http://ia800203.us.archive.org/13/items/war_and_peace_12_0911_librivox/war_and_peace_12_01_tolstoy_64kb.mp3}{Audio}}
\fi

\initial{I}{n} Petersburg at that time a complicated struggle was being
carried on with greater heat than ever in the highest circles,
between the parties of Rumyantsev, the French, Marya Fedorovna,
the Tsarevich, and others, drowned as usual by the buzzing of the
court drones. But the calm, luxurious life of Petersburg,
concerned only about phantoms and reflections of real life, went
on in its old way and made it hard, except by a great effort, to
realize the danger and the difficult position of the Russian
people. There were the same receptions and balls, the same French
theater, the same court interests and service interests and
intrigues as usual. Only in the very highest circles were
attempts made to keep in mind the difficulties of the actual
position.  Stories were whispered of how differently the two
Empresses behaved in these difficult circumstances. The Empress
Marya, concerned for the welfare of the charitable and
educational institutions under her patronage, had given
directions that they should all be removed to Kazan, and the
things belonging to these institutions had already been packed
up. The Empress Elisabeth, however, when asked what instructions
she would be pleased to give---with her characteristic Russian
patriotism had replied that she could give no directions about
state institutions for that was the affair of the sovereign, but
as far as she personally was concerned she would be the last to
quit Petersburg.

At Anna Pavlovna's on the twenty-sixth of August, the very day of
the battle of Borodino, there was a soiree, the chief feature of
which was to be the reading of a letter from His Lordship the
Bishop when sending the Emperor an icon of the Venerable
Sergius. It was regarded as a model of ecclesiastical, patriotic
eloquence. Prince Vasili himself, famed for his elocution, was to
read it. (He used to read at the Empress'.) The art of his
reading was supposed to lie in rolling out the words, quite
independently of their meaning, in a loud and singsong voice
alternating between a despairing wail and a tender murmur, so
that the wail fell quite at random on one word and the murmur on
another. This reading, as was always the case at Anna Pavlovna's
soirees, had a political significance. That evening she expected
several important personages who had to be made ashamed of their
visits to the French theater and aroused to a patriotic temper. A
good many people had already arrived, but Anna Pavlovna, not yet
seeing all those whom she wanted in her drawing room, did not let
the reading begin but wound up the springs of a general
conversation.

The news of the day in Petersburg was the illness of Countess
Bezukhova.  She had fallen ill unexpectedly a few days
previously, had missed several gatherings of which she was
usually ornament, and was said to be receiving no one, and
instead of the celebrated Petersburg doctors who usually attended
her had entrusted herself to some Italian doctor who was treating
her in some new and unusual way.

They all knew very well that the enchanting countess' illness
arose from an inconvenience resulting from marrying two husbands
at the same time, and that the Italian's cure consisted in
removing such inconvenience; but in Anna Pavlovna's presence no
one dared to think of this or even appear to know it.

``They say the poor countess is very ill. The doctor says it is
angina pectoris.''

``Angina? Oh, that's a terrible illness!''

``They say that the rivals are reconciled, thanks to the
angina...'' and the word angina was repeated with great
satisfaction.

``The count is pathetic, they say. He cried like a child when the
doctor told him the case was dangerous.''

``Oh, it would be a terrible loss, she is an enchanting woman.''

``You are speaking of the poor countess?'' said Anna Pavlovna,
coming up just then. ``I sent to ask for news, and hear that she
is a little better. Oh, she is certainly the most charming woman
in the world,'' she went on, with a smile at her own
enthusiasm. ``We belong to different camps, but that does not
prevent my esteeming her as she deserves. She is very
unfortunate!'' added Anna Pavlovna.

Supposing that by these words Anna Pavlovna was somewhat lifting
the veil from the secret of the countess' malady, an unwary young
man ventured to express surprise that well known doctors had not
been called in and that the countess was being attended by a
charlatan who might employ dangerous remedies.

``Your information may be better than mine,'' Anna Pavlovna
suddenly and venomously retorted on the inexperienced young man,
``but I know on good authority that this doctor is a very learned
and able man. He is private physician to the Queen of Spain.''

And having thus demolished the young man, Anna Pavlovna turned to
another group where Bilibin was talking about the Austrians:
having wrinkled up his face he was evidently preparing to smooth
it out again and utter one of his mots.

``I think it is delightful,'' he said, referring to a diplomatic
note that had been sent to Vienna with some Austrian banners
captured from the French by Wittgenstein, ``the hero of
Petropol'' as he was then called in Petersburg.

``What? What's that?'' asked Anna Pavlovna, securing silence for
the mot, which she had heard before.

And Bilibin repeated the actual words of the diplomatic dispatch,
which he had himself composed.

``The Emperor returns these Austrian banners,'' said Bilibin,
``friendly banners gone astray and found on a wrong path,'' and
his brow became smooth again.

``Charming, charming!'' observed Prince Vasili.

``The path to Warsaw, perhaps,'' Prince Hippolyte remarked loudly
and unexpectedly. Everybody looked at him, understanding what he
meant.  Prince Hippolyte himself glanced around with amused
surprise. He knew no more than the others what his words
meant. During his diplomatic career he had more than once noticed
that such utterances were received as very witty, and at every
opportunity he uttered in that way the first words that entered
his head. ``It may turn out very well,'' he thought, ``but if
not, they'll know how to arrange matters.'' And really, during
the awkward silence that ensued, that insufficiently patriotic
person entered whom Anna Pavlovna had been waiting for and wished
to convert, and she, smiling and shaking a finger at Hippolyte,
invited Prince Vasili to the table and bringing him two candles
and the manuscript begged him to begin. Everyone became silent.

``Most Gracious Sovereign and Emperor!'' Prince Vasili sternly
declaimed, looking round at his audience as if to inquire whether
anyone had anything to say to the contrary. But no one said
anything. ``Moscow, our ancient capital, the New Jerusalem,
receives her Christ''---he placed a sudden emphasis on the word
her---``as a mother receives her zealous sons into her arms, and
through the gathering mists, foreseeing the brilliant glory of
thy rule, sings in exultation, 'Hosanna, blessed is he that
cometh!'{}''

Prince Vasili pronounced these last words in a tearful voice.

Bilibin attentively examined his nails, and many of those present
appeared intimidated, as if asking in what they were to
blame. Anna Pavlovna whispered the next words in advance, like an
old woman muttering the prayer at Communion: ``Let the bold and
insolent Goliath...'' she whispered.

Prince Vasili continued.

``Let the bold and insolent Goliath from the borders of France
encompass the realms of Russia with death-bearing terrors; humble
Faith, the sling of the Russian David, shall suddenly smite his
head in his bloodthirsty pride. This icon of the Venerable
Sergius, the servant of God and zealous champion of old of our
country's weal, is offered to Your Imperial Majesty. I grieve
that my waning strength prevents rejoicing in the sight of your
most gracious presence. I raise fervent prayers to Heaven that
the Almighty may exalt the race of the just, and mercifully
fulfill the desires of Your Majesty.''

``What force! What a style!'' was uttered in approval both of
reader and of author.

Animated by that address Anna Pavlovna's guests talked for a long
time of the state of the fatherland and offered various
conjectures as to the result of the battle to be fought in a few
days.

``You will see,'' said Anna Pavlovna, ``that tomorrow, on the
Emperor's birthday, we shall receive news. I have a favorable
presentiment!''

% % % % % % % % % % % % % % % % % % % % % % % % % % % % % % % % %
% % % % % % % % % % % % % % % % % % % % % % % % % % % % % % % % %
% % % % % % % % % % % % % % % % % % % % % % % % % % % % % % % % %
% % % % % % % % % % % % % % % % % % % % % % % % % % % % % % % % %
% % % % % % % % % % % % % % % % % % % % % % % % % % % % % % % % %
% % % % % % % % % % % % % % % % % % % % % % % % % % % % % % % % %
% % % % % % % % % % % % % % % % % % % % % % % % % % % % % % % % %
% % % % % % % % % % % % % % % % % % % % % % % % % % % % % % % % %
% % % % % % % % % % % % % % % % % % % % % % % % % % % % % % % % %
% % % % % % % % % % % % % % % % % % % % % % % % % % % % % % % % %
% % % % % % % % % % % % % % % % % % % % % % % % % % % % % % % % %
% % % % % % % % % % % % % % % % % % % % % % % % % % % % % %

\chapter*{Chapter II} \ifaudio \marginpar{
\href{http://ia800203.us.archive.org/13/items/war_and_peace_12_0911_librivox/war_and_peace_12_02_tolstoy_64kb.mp3}{Audio}}
\fi

\initial{A}{nna} Pavlovna's presentiment was in fact fulfilled. Next day
during the service at the palace church in honor of the Emperor's
birthday, Prince Volkonski was called out of the church and
received a dispatch from Prince Kutuzov. It was Kutuzov's report,
written from Tatarinova on the day of the battle. Kutuzov wrote
that the Russians had not retreated a step, that the French
losses were much heavier than ours, and that he was writing in
haste from the field of battle before collecting full
information. It followed that there must have been a victory. And
at once, without leaving the church, thanks were rendered to the
Creator for His help and for the victory.

Anna Pavlovna's presentiment was justified, and all that morning
a joyously festive mood reigned in the city. Everyone believed
the victory to have been complete, and some even spoke of
Napoleon's having been captured, of his deposition, and of the
choice of a new ruler for France.

It is very difficult for events to be reflected in their real
strength and completeness amid the conditions of court life and
far from the scene of action. General events involuntarily group
themselves around some particular incident. So now the courtiers'
pleasure was based as much on the fact that the news had arrived
on the Emperor's birthday as on the fact of the victory
itself. It was like a successfully arranged surprise. Mention was
made in Kutuzov's report of the Russian losses, among which
figured the names of Tuchkov, Bagration, and Kutaysov. In the
Petersburg world this sad side of the affair again involuntarily
centered round a single incident: Kutaysov's death. Everybody
knew him, the Emperor liked him, and he was young and
interesting. That day everyone met with the words:

``What a wonderful coincidence! Just during the service. But what
a loss Kutaysov is! How sorry I am!''

``What did I tell about Kutuzov?'' Prince Vasili now said with a
prophet's pride. ``I always said he was the only man capable of
defeating Napoleon.''

But next day no news arrived from the army and the public mood
grew anxious. The courtiers suffered because of the suffering the
suspense occasioned the Emperor.

``Fancy the Emperor's position!'' said they, and instead of
extolling Kutuzov as they had done the day before, they condemned
him as the cause of the Emperor's anxiety. That day Prince Vasili
no longer boasted of his protege Kutuzov, but remained silent
when the commander-in-chief was mentioned. Moreover, toward
evening, as if everything conspired to make Petersburg society
anxious and uneasy, a terrible piece of news was added. Countess
Helene Bezukhova had suddenly died of that terrible malady it had
been so agreeable to mention. Officially, at large gatherings,
everyone said that Countess Bezukhova had died of a terrible
attack of angina pectoris, but in intimate circles details were
mentioned of how the private physician of the Queen of Spain had
prescribed small doses of a certain drug to produce a certain
effect; but Helene, tortured by the fact that the old count
suspected her and that her husband to whom she had written (that
wretched, profligate Pierre) had not replied, had suddenly taken
a very large dose of the drug, and had died in agony before
assistance could be rendered her. It was said that Prince Vasili
and the old count had turned upon the Italian, but the latter had
produced such letters from the unfortunate deceased that they had
immediately let the matter drop.

Talk in general centered round three melancholy facts: the
Emperor's lack of news, the loss of Kutaysov, and the death of
Helene.

On the third day after Kutuzov's report a country gentleman
arrived from Moscow, and news of the surrender of Moscow to the
French spread through the whole town. This was terrible! What a
position for the Emperor to be in! Kutuzov was a traitor, and
Prince Vasili during the visits of condolence paid to him on the
occasion of his daughter's death said of Kutuzov, whom he had
formerly praised (it was excusable for him in his grief to forget
what he had said), that it was impossible to expect anything else
from a blind and depraved old man.

``I only wonder that the fate of Russia could have been entrusted
to such a man.''

As long as this news remained unofficial it was possible to doubt
it, but the next day the following communication was received
from Count Rostopchin:

Prince Kutuzov's adjutant has brought me a letter in which he
demands police officers to guide the army to the Ryazan road. He
writes that he is regretfully abandoning Moscow. Sire! Kutuzov's
action decides the fate of the capital and of your empire! Russia
will shudder to learn of the abandonment of the city in which her
greatness is centered and in which lie the ashes of your
ancestors! I shall follow the army. I have had everything
removed, and it only remains for me to weep over the fate of my
fatherland.

On receiving this dispatch the Emperor sent Prince Volkonski to
Kutuzov with the following rescript:

Prince Michael Ilarionovich! Since the twenty-ninth of August I
have received no communication from you, yet on the first of
September I received from the commander-in-chief of Moscow, via
Yaroslavl, the sad news that you, with the army, have decided to
abandon Moscow. You can yourself imagine the effect this news has
had on me, and your silence increases my astonishment. I am
sending this by Adjutant-General Prince Volkonski, to hear from
you the situation of the army and the reasons that have induced
you to take this melancholy decision.

% % % % % % % % % % % % % % % % % % % % % % % % % % % % % % % % %
% % % % % % % % % % % % % % % % % % % % % % % % % % % % % % % % %
% % % % % % % % % % % % % % % % % % % % % % % % % % % % % % % % %
% % % % % % % % % % % % % % % % % % % % % % % % % % % % % % % % %
% % % % % % % % % % % % % % % % % % % % % % % % % % % % % % % % %
% % % % % % % % % % % % % % % % % % % % % % % % % % % % % % % % %
% % % % % % % % % % % % % % % % % % % % % % % % % % % % % % % % %
% % % % % % % % % % % % % % % % % % % % % % % % % % % % % % % % %
% % % % % % % % % % % % % % % % % % % % % % % % % % % % % % % % %
% % % % % % % % % % % % % % % % % % % % % % % % % % % % % % % % %
% % % % % % % % % % % % % % % % % % % % % % % % % % % % % % % % %
% % % % % % % % % % % % % % % % % % % % % % % % % % % % % %

\chapter*{Chapter III} \ifaudio \marginpar{
\href{http://ia800203.us.archive.org/13/items/war_and_peace_12_0911_librivox/war_and_peace_12_03_tolstoy_64kb.mp3}{Audio}}
\fi 

\initial{N}{ine} days after the abandonment of Moscow, a messenger from
Kutuzov reached Petersburg with the official announcement of that
event. This messenger was Michaud, a Frenchman who did not know
Russian, but who was quoique etranger, russe de coeur et
d'ame,\footnote{Though a foreigner, Russian in heart and soul.}
as he said of himself.

The Emperor at once received this messenger in his study at the
palace on Stone Island. Michaud, who had never seen Moscow before
the campaign and who did not know Russian, yet felt deeply moved
(as he wrote) when he appeared before notre tres gracieux
souverain\footnote{Our most gracious sovereign.} with the news of
the burning of Moscow, dont les flammes eclairaient sa
route.\footnote{Whose flames illumined his route.}

Though the source of M. Michaud's chagrin must have been
different from that which caused Russians to grieve, he had such
a sad face when shown into the Emperor's study that the latter at
once asked:

``Have you brought me sad news, Colonel?''

``Very sad, sire,'' replied Michaud, lowering his eyes with a
sigh. ``The abandonment of Moscow.''

``Have they surrendered my ancient capital without a battle?''
asked the Emperor quickly, his face suddenly flushing.

Michaud respectfully delivered the message Kutuzov had entrusted
to him, which was that it had been impossible to fight before
Moscow, and that as the only remaining choice was between losing
the army as well as Moscow, or losing Moscow alone, the field
marshal had to choose the latter.

The Emperor listened in silence, not looking at Michaud.

``Has the enemy entered the city?'' he asked.

``Yes, sire, and Moscow is now in ashes. I left it all in
flames,'' replied Michaud in a decided tone, but glancing at the
Emperor he was frightened by what he had done.

The Emperor began to breathe heavily and rapidly, his lower lip
trembled, and tears instantly appeared in his fine blue eyes.

But this lasted only a moment. He suddenly frowned, as if blaming
himself for his weakness, and raising his head addressed Michaud
in a firm voice:

``I see, Colonel, from all that is happening, that Providence
requires great sacrifices of us... I am ready to submit myself in
all things to His will; but tell me, Michaud, how did you leave
the army when it saw my ancient capital abandoned without a
battle? Did you not notice discouragement?...''

Seeing that his most gracious ruler was calm once more, Michaud
also grew calm, but was not immediately ready to reply to the
Emperor's direct and relevant question which required a direct
answer.

``Sire, will you allow me to speak frankly as befits a loyal
soldier?'' he asked to gain time.

``Colonel, I always require it,'' replied the Emperor. ``Conceal
nothing from me, I wish to know absolutely how things are.''

``Sire!'' said Michaud with a subtle, scarcely perceptible smile
on his lips, having now prepared a well-phrased reply, ``sire, I
left the whole army, from its chiefs to the lowest soldier,
without exception in desperate and agonized terror...''

``How is that?'' the Emperor interrupted him, frowning
sternly. ``Would misfortune make my Russians lose
heart?... Never!''

Michaud had only waited for this to bring out the phrase he had
prepared.

``Sire,'' he said, with respectful playfulness, ``they are only
afraid lest Your Majesty, in the goodness of your heart, should
allow yourself to be persuaded to make peace. They are burning
for the combat,'' declared this representative of the Russian
nation, ``and to prove to Your Majesty by the sacrifice of their
lives how devoted they are...''

``Ah!'' said the Emperor reassured, and with a kindly gleam in
his eyes, he patted Michaud on the shoulder. ``You set me at
ease, Colonel.''

He bent his head and was silent for some time.

``Well, then, go back to the army,'' he said, drawing himself up
to his full height and addressing Michaud with a gracious and
majestic gesture, ``and tell our brave men and all my good
subjects wherever you go that when I have not a soldier left I
shall put myself at the head of my beloved nobility and my good
peasants and so use the last resources of my empire. It still
offers me more than my enemies suppose,'' said the Emperor
growing more and more animated; ``but should it ever be ordained
by Divine Providence,'' he continued, raising to heaven his fine
eyes shining with emotion, ``that my dynasty should cease to
reign on the throne of my ancestors, then after exhausting all
the means at my command, I shall let my beard grow to here'' (he
pointed halfway down his chest) ``and go and eat potatoes with
the meanest of my peasants, rather than sign the disgrace of my
country and of my beloved people whose sacrifices I know how to
appreciate.''

Having uttered these words in an agitated voice the Emperor
suddenly turned away as if to hide from Michaud the tears that
rose to his eyes, and went to the further end of his
study. Having stood there a few moments, he strode back to
Michaud and pressed his arm below the elbow with a vigorous
movement. The Emperor's mild and handsome face was flushed and
his eyes gleamed with resolution and anger.

``Colonel Michaud, do not forget what I say to you here, perhaps
we may recall it with pleasure someday... Napoleon or I,'' said
the Emperor, touching his breast. ``We can no longer both reign
together. I have learned to know him, and he will not deceive me
any more...''

And the Emperor paused, with a frown.

When he heard these words and saw the expression of firm
resolution in the Emperor's eyes, Michaud---quoique etranger,
russe de coeur et d'ame---at that solemn moment felt himself
enraptured by all that he had heard (as he used afterwards to
say), and gave expression to his own feelings and those of the
Russian people whose representative he considered himself to be,
in the following words:

``Sire!'' said he, ``Your Majesty is at this moment signing the
glory of the nation and the salvation of Europe!''

With an inclination of the head the Emperor dismissed him.

% % % % % % % % % % % % % % % % % % % % % % % % % % % % % % % % %
% % % % % % % % % % % % % % % % % % % % % % % % % % % % % % % % %
% % % % % % % % % % % % % % % % % % % % % % % % % % % % % % % % %
% % % % % % % % % % % % % % % % % % % % % % % % % % % % % % % % %
% % % % % % % % % % % % % % % % % % % % % % % % % % % % % % % % %
% % % % % % % % % % % % % % % % % % % % % % % % % % % % % % % % %
% % % % % % % % % % % % % % % % % % % % % % % % % % % % % % % % %
% % % % % % % % % % % % % % % % % % % % % % % % % % % % % % % % %
% % % % % % % % % % % % % % % % % % % % % % % % % % % % % % % % %
% % % % % % % % % % % % % % % % % % % % % % % % % % % % % % % % %
% % % % % % % % % % % % % % % % % % % % % % % % % % % % % % % % %
% % % % % % % % % % % % % % % % % % % % % % % % % % % % % %

\chapter*{Chapter IV} \ifaudio \marginpar{
\href{http://ia800203.us.archive.org/13/items/war_and_peace_12_0911_librivox/war_and_peace_12_04_tolstoy_64kb.mp3}{Audio}}
\fi

\initial{I}{t} is natural for us who were not living in those days to imagine
that when half Russia had been conquered and the inhabitants were
fleeing to distant provinces, and one levy after another was
being raised for the defense of the fatherland, all Russians from
the greatest to the least were solely engaged in sacrificing
themselves, saving their fatherland, or weeping over its
downfall. The tales and descriptions of that time without
exception speak only of the self-sacrifice, patriotic devotion,
despair, grief, and the heroism of the Russians. But it was not
really so. It appears so to us because we see only the general
historic interest of that time and do not see all the personal
human interests that people had. Yet in reality those personal
interests of the moment so much transcend the general interests
that they always prevent the public interest from being felt or
even noticed. Most of the people at that time paid no attention
to the general progress of events but were guided only by their
private interests, and they were the very people whose activities
at that period were most useful.

Those who tried to understand the general course of events and to
take part in it by self-sacrifice and heroism were the most
useless members of society, they saw everything upside down, and
all they did for the common good turned out to be useless and
foolish---like Pierre's and Mamonov's regiments which looted
Russian villages, and the lint the young ladies prepared and that
never reached the wounded, and so on.  Even those, fond of
intellectual talk and of expressing their feelings, who discussed
Russia's position at the time involuntarily introduced into their
conversation either a shade of pretense and falsehood or useless
condemnation and anger directed against people accused of actions
no one could possibly be guilty of. In historic events the rule
forbidding us to eat of the fruit of the Tree of Knowledge is
specially applicable. Only unconscious action bears fruit, and he
who plays a part in an historic event never understands its
significance. If he tries to realize it his efforts are
fruitless.

The more closely a man was engaged in the events then taking
place in Russia the less did he realize their significance. In
Petersburg and in the provinces at a distance from Moscow,
ladies, and gentlemen in militia uniforms, wept for Russia and
its ancient capital and talked of self-sacrifice and so on; but
in the army which retired beyond Moscow there was little talk or
thought of Moscow, and when they caught sight of its burned ruins
no one swore to be avenged on the French, but they thought about
their next pay, their next quarters, of Matreshka the vivandiere,
and like matters.

As the war had caught him in the service, Nicholas Rostov took a
close and prolonged part in the defense of his country, but did
so casually, without any aim at self-sacrifice, and he therefore
looked at what was going on in Russia without despair and without
dismally racking his brains over it. Had he been asked what he
thought of the state of Russia, he would have said that it was
not his business to think about it, that Kutuzov and others were
there for that purpose, but that he had heard that the regiments
were to be made up to their full strength, that fighting would
probably go on for a long time yet, and that things being so it
was quite likely he might be in command of a regiment in a couple
of years' time.

As he looked at the matter in this way, he learned that he was
being sent to Voronezh to buy remounts for his division, not only
without regret at being prevented from taking part in the coming
battle, but with the greatest pleasure---which he did not conceal
and which his comrades fully understood.

A few days before the battle of Borodino, Nicholas received the
necessary money and warrants, and having sent some hussars on in
advance, he set out with post horses for Voronezh.

Only a man who has experienced it---that is, has passed some
months continuously in an atmosphere of campaigning and war---can
understand the delight Nicholas felt when he escaped from the
region covered by the army's foraging operations, provision
trains, and hospitals. When---free from soldiers, wagons, and the
filthy traces of a camp---he saw villages with peasants and
peasant women, gentlemen's country houses, fields where cattle
were grazing, posthouses with stationmasters asleep in them, he
rejoiced as though seeing all this for the first time. What for a
long while specially surprised and delighted him were the women,
young and healthy, without a dozen officers making up to each of
them; women, too, who were pleased and flattered that a passing
officer should joke with them.

In the highest spirits Nicholas arrived at night at a hotel in
Voronezh, ordered things he had long been deprived of in camp,
and next day, very clean-shaven and in a full-dress uniform he
had not worn for a long time, went to present himself to the
authorities.

The commander of the militia was a civilian general, an old man
who was evidently pleased with his military designation and
rank. He received Nicholas brusquely (imagining this to be
characteristically military) and questioned him with an important
air, as if considering the general progress of affairs and
approving and disapproving with full right to do so. Nicholas was
in such good spirits that this merely amused him.

From the commander of the militia he drove to the governor. The
governor was a brisk little man, very simple and affable. He
indicated the stud farms at which Nicholas might procure horses,
recommended to him a horse dealer in the town and a landowner
fourteen miles out of town who had the best horses, and promised
to assist him in every way.

``You are Count Ilya Rostov's son? My wife was a great friend of
your mother's. We are at home on Thursdays---today is Thursday,
so please come and see us quite informally,'' said the governor,
taking leave of him.

Immediately on leaving the governor's, Nicholas hired post horses
and, taking his squadron quartermaster with him, drove at a
gallop to the landowner, fourteen miles away, who had the
stud. Everything seemed to him pleasant and easy during that
first part of his stay in Voronezh and, as usually happens when a
man is in a pleasant state of mind, everything went well and
easily.

The landowner to whom Nicholas went was a bachelor, an old
cavalryman, a horse fancier, a sportsman, the possessor of some
century-old brandy and some old Hungarian wine, who had a
snuggery where he smoked, and who owned some splendid horses.

In very few words Nicholas bought seventeen picked stallions for
six thousand rubles---to serve, as he said, as samples of his
remounts. After dining and taking rather too much of the
Hungarian wine, Nicholas---having exchanged kisses with the
landowner, with whom he was already on the friendliest
terms---galloped back over abominable roads, in the brightest
frame of mind, continually urging on the driver so as to be in
time for the governor's party.

When he had changed, poured water over his head, and scented
himself, Nicholas arrived at the governor's rather late, but with
the phrase \emph{better late than never} on his lips.

It was not a ball, nor had dancing been announced, but everyone
knew that Catherine Petrovna would play valses and the ecossaise
on the clavichord and that there would be dancing, and so
everyone had come as to a ball.

Provincial life in 1812 went on very much as usual, but with this
difference, that it was livelier in the towns in consequence of
the arrival of many wealthy families from Moscow, and as in
everything that went on in Russia at that time a special
recklessness was noticeable, an \emph{'in for a penny, in for a
pound---who cares?'} spirit, and the inevitable small talk,
instead of turning on the weather and mutual acquaintances, now
turned on Moscow, the army, and Napoleon.

The society gathered together at the governor's was the best in
Voronezh.

There were a great many ladies and some of Nicholas' Moscow
acquaintances, but there were no men who could at all vie with
the cavalier of St. George, the hussar remount officer, the
good-natured and well-bred Count Rostov. Among the men was an
Italian prisoner, an officer of the French army; and Nicholas
felt that the presence of that prisoner enhanced his own
importance as a Russian hero. The Italian was, as it were, a war
trophy. Nicholas felt this, it seemed to him that everyone
regarded the Italian in the same light, and he treated him
cordially though with dignity and restraint.

As soon as Nicholas entered in his hussar uniform, diffusing
around him a fragrance of perfume and wine, and had uttered the
words \emph{better late than never} and heard them repeated
several times by others, people clustered around him; all eyes
turned on him, and he felt at once that he had entered into his
proper position in the province---that of a universal favorite: a
very pleasant position, and intoxicatingly so after his long
privations. At posting stations, at inns, and in the landowner's
snuggery, maidservants had been flattered by his notice, and here
too at the governor's party there were (as it seemed to Nicholas)
an inexhaustible number of pretty young women, married and
unmarried, impatiently awaiting his notice. The women and girls
flirted with him and, from the first day, the people concerned
themselves to get this fine young daredevil of an hussar married
and settled down. Among these was the governor's wife herself,
who welcomed Rostov as a near relative and called him
``Nicholas.''

Catherine Petrovna did actually play valses and the ecossaise,
and dancing began in which Nicholas still further captivated the
provincial society by his agility. His particularly free manner
of dancing even surprised them all. Nicholas was himself rather
surprised at the way he danced that evening. He had never danced
like that in Moscow and would even have considered such a very
free and easy manner improper and in bad form, but here he felt
it incumbent on him to astonish them all by something unusual,
something they would have to accept as the regular thing in the
capital though new to them in the provinces.

All the evening Nicholas paid attention to a blue-eyed, plump and
pleasing little blonde, the wife of one of the provincial
officials.  With the naive conviction of young men in a merry
mood that other men's wives were created for them, Rostov did not
leave the lady's side and treated her husband in a friendly and
conspiratorial style, as if, without speaking of it, they knew
how capitally Nicholas and the lady would get on together. The
husband, however, did not seem to share that conviction and tried
to behave morosely with Rostov. But the latter's good-natured
naivete was so boundless that sometimes even he involuntarily
yielded to Nicholas' good humor. Toward the end of the evening,
however, as the wife's face grew more flushed and animated, the
husband's became more and more melancholy and solemn, as though
there were but a given amount of animation between them and as
the wife's share increased the husband's diminished.

% % % % % % % % % % % % % % % % % % % % % % % % % % % % % % % % %
% % % % % % % % % % % % % % % % % % % % % % % % % % % % % % % % %
% % % % % % % % % % % % % % % % % % % % % % % % % % % % % % % % %
% % % % % % % % % % % % % % % % % % % % % % % % % % % % % % % % %
% % % % % % % % % % % % % % % % % % % % % % % % % % % % % % % % %
% % % % % % % % % % % % % % % % % % % % % % % % % % % % % % % % %
% % % % % % % % % % % % % % % % % % % % % % % % % % % % % % % % %
% % % % % % % % % % % % % % % % % % % % % % % % % % % % % % % % %
% % % % % % % % % % % % % % % % % % % % % % % % % % % % % % % % %
% % % % % % % % % % % % % % % % % % % % % % % % % % % % % % % % %
% % % % % % % % % % % % % % % % % % % % % % % % % % % % % % % % %
% % % % % % % % % % % % % % % % % % % % % % % % % % % % % %

\chapter*{Chapter V} \ifaudio \marginpar{
\href{http://ia800203.us.archive.org/13/items/war_and_peace_12_0911_librivox/war_and_peace_12_05_tolstoy_64kb.mp3}{Audio}}
\fi

\initial{N}{icholas} sat leaning slightly forward in an armchair, bending
closely over the blonde lady and paying her mythological
compliments with a smile that never left his face. Jauntily
shifting the position of his legs in their tight riding breeches,
diffusing an odor of perfume, and admiring his partner, himself,
and the fine outlines of his legs in their well-fitting Hessian
boots, Nicholas told the blonde lady that he wished to run away
with a certain lady here in Voronezh.

``Which lady?''

``A charming lady, a divine one. Her eyes'' (Nicholas looked at
his partner) ``are blue, her mouth coral and ivory; her figure''
(he glanced at her shoulders) ``like Diana's...''

The husband came up and sullenly asked his wife what she was
talking about.

``Ah, Nikita Ivanych!'' cried Nicholas, rising politely, and as
if wishing Nikita Ivanych to share his joke, he began to tell him
of his intention to elope with a blonde lady.

The husband smiled gloomily, the wife gaily. The governor's
good-natured wife came up with a look of disapproval.

``Anna Ignatyevna wants to see you, Nicholas,'' said she,
pronouncing the name so that Nicholas at once understood that
Anna Ignatyevna was a very important person. ``Come, Nicholas!
You know you let me call you so?''

``Oh, yes, Aunt. Who is she?''

``Anna Ignatyevna Malvintseva. She has heard from her niece how
you rescued her... Can you guess?''

``I rescued such a lot of them!'' said Nicholas.

``Her niece, Princess Bolkonskaya. She is here in Voronezh with
her aunt.  Oho! How you blush. Why, are...?''

``Not a bit! Please don't, Aunt!''

``Very well, very well!... Oh, what a fellow you are!''

The governor's wife led him up to a tall and very stout old lady
with a blue headdress, who had just finished her game of cards
with the most important personages of the town. This was
Malvintseva, Princess Mary's aunt on her mother's side, a rich,
childless widow who always lived in Voronezh. When Rostov
approached her she was standing settling up for the game. She
looked at him and, screwing up her eyes sternly, continued to
upbraid the general who had won from her.

``Very pleased, mon cher,'' she then said, holding out her hand
to Nicholas. ``Pray come and see me.''

After a few words about Princess Mary and her late father, whom
Malvintseva had evidently not liked, and having asked what
Nicholas knew of Prince Andrew, who also was evidently no
favorite of hers, the important old lady dismissed Nicholas after
repeating her invitation to come to see her.

Nicholas promised to come and blushed again as he bowed. At the
mention of Princess Mary he experienced a feeling of shyness and
even of fear, which he himself did not understand.

When he had parted from Malvintseva Nicholas wished to return to
the dancing, but the governor's little wife placed her plump hand
on his sleeve and, saying that she wanted to have a talk with
him, led him to her sitting room, from which those who were there
immediately withdrew so as not to be in her way.

``Do you know, dear boy,'' began the governor's wife with a
serious expression on her kind little face, ``that really would
be the match for you: would you like me to arrange it?''

``Whom do you mean, Aunt?'' asked Nicholas.

``I will make a match for you with the princess. Catherine
Petrovna speaks of Lily, but I say, no---the princess! Do you
want me to do it? I am sure your mother will be grateful to
me. What a charming girl she is, really! And she is not at all so
plain, either.''

``Not at all,'' replied Nicholas as if offended at the idea. ``As
befits a soldier, Aunt, I don't force myself on anyone or refuse
anything,'' he said before he had time to consider what he was
saying.

``Well then, remember, this is not a joke!''

``Of course not!''

``Yes, yes,'' the governor's wife said as if talking to
herself. ``But, my dear boy, among other things you are too
attentive to the other, the blonde. One is sorry for the husband,
really...''

``Oh no, we are good friends with him,'' said Nicholas in the
simplicity of his heart; it did not enter his head that a pastime
so pleasant to himself might not be pleasant to someone else.

``But what nonsense I have been saying to the governor's wife!''
thought Nicholas suddenly at supper. ``She will really begin to
arrange a match... and Sonya...?'' And on taking leave of the
governor's wife, when she again smilingly said to him, ``Well
then, remember!'' he drew her aside.

``But see here, to tell the truth, Aunt...''

``What is it, my dear? Come, let's sit down here,'' said she.

Nicholas suddenly felt a desire and need to tell his most
intimate thoughts (which he would not have told to his mother,
his sister, or his friend) to this woman who was almost a
stranger. When he afterwards recalled that impulse to unsolicited
and inexplicable frankness which had very important results for
him, it seemed to him---as it seems to everyone in such
cases---that it was merely some silly whim that seized him: yet
that burst of frankness, together with other trifling events, had
immense consequences for him and for all his family.

``You see, Aunt, Mamma has long wanted me to marry an heiress,
but the very idea of marrying for money is repugnant to me.''

``Oh yes, I understand,'' said the governor's wife.

``But Princess Bolkonskaya---that's another matter. I will tell
you the truth. In the first place I like her very much, I feel
drawn to her; and then, after I met her under such
circumstances---so strangely, the idea often occurred to me:
'This is fate.' Especially if you remember that Mamma had long
been thinking of it; but I had never happened to meet her before,
somehow it had always happened that we did not meet. And as long
as my sister Natasha was engaged to her brother it was of course
out of the question for me to think of marrying her. And it must
needs happen that I should meet her just when Natasha's
engagement had been broken off... and then everything... So you
see... I never told this to anyone and never will, only to you.''

The governor's wife pressed his elbow gratefully.

``You know Sonya, my cousin? I love her, and promised to marry
her, and will do so... So you see there can be no question
about-'' said Nicholas incoherently and blushing.

``My dear boy, what a way to look at it! You know Sonya has
nothing and you yourself say your Papa's affairs are in a very
bad way. And what about your mother? It would kill her, that's
one thing. And what sort of life would it be for Sonya---if she's
a girl with a heart? Your mother in despair, and you all
ruined... No, my dear, you and Sonya ought to understand that.''

Nicholas remained silent. It comforted him to hear these
arguments.

``All the same, Aunt, it is impossible,'' he rejoined with a
sigh, after a short pause. ``Besides, would the princess have me?
And besides, she is now in mourning. How can one think of it!''

``But you don't suppose I'm going to get you married at once?
There is always a right way of doing things,'' replied the
governor's wife.

``What a matchmaker you are, Aunt...'' said Nicholas, kissing her
plump little hand.

% % % % % % % % % % % % % % % % % % % % % % % % % % % % % % % % %
% % % % % % % % % % % % % % % % % % % % % % % % % % % % % % % % %
% % % % % % % % % % % % % % % % % % % % % % % % % % % % % % % % %
% % % % % % % % % % % % % % % % % % % % % % % % % % % % % % % % %
% % % % % % % % % % % % % % % % % % % % % % % % % % % % % % % % %
% % % % % % % % % % % % % % % % % % % % % % % % % % % % % % % % %
% % % % % % % % % % % % % % % % % % % % % % % % % % % % % % % % %
% % % % % % % % % % % % % % % % % % % % % % % % % % % % % % % % %
% % % % % % % % % % % % % % % % % % % % % % % % % % % % % % % % %
% % % % % % % % % % % % % % % % % % % % % % % % % % % % % % % % %
% % % % % % % % % % % % % % % % % % % % % % % % % % % % % % % % %
% % % % % % % % % % % % % % % % % % % % % % % % % % % % % %

\chapter*{Chapter VI} \ifaudio \marginpar{
\href{http://ia800203.us.archive.org/13/items/war_and_peace_12_0911_librivox/war_and_peace_12_06_tolstoy_64kb.mp3}{Audio}}
\fi

\initial{O}{n} reaching Moscow after her meeting with Rostov, Princess Mary
had found her nephew there with his tutor, and a letter from
Prince Andrew giving her instructions how to get to her Aunt
Malvintseva at Voronezh.  That feeling akin to temptation which
had tormented her during her father's illness, since his death,
and especially since her meeting with Rostov was smothered by
arrangements for the journey, anxiety about her brother, settling
in a new house, meeting new people, and attending to her nephew's
education. She was sad. Now, after a month passed in quiet
surroundings, she felt more and more deeply the loss of her
father which was associated in her mind with the ruin of
Russia. She was agitated and incessantly tortured by the thought
of the dangers to which her brother, the only intimate person now
remaining to her, was exposed. She was worried too about her
nephew's education for which she had always felt herself
incompetent, but in the depths of her soul she felt at peace---a
peace arising from consciousness of having stifled those personal
dreams and hopes that had been on the point of awakening within
her and were related to her meeting with Rostov.

The day after her party the governor's wife came to see
Malvintseva and, after discussing her plan with the aunt,
remarked that though under present circumstances a formal
betrothal was, of course, not to be thought of, all the same the
young people might be brought together and could get to know one
another. Malvintseva expressed approval, and the governor's wife
began to speak of Rostov in Mary's presence, praising him and
telling how he had blushed when Princess Mary's name was
mentioned. But Princess Mary experienced a painful rather than a
joyful feeling---her mental tranquillity was destroyed, and
desires, doubts, self-reproach, and hopes reawoke.

During the two days that elapsed before Rostov called, Princess
Mary continually thought of how she ought to behave to him. First
she decided not to come to the drawing room when he called to see
her aunt---that it would not be proper for her, in her deep
mourning, to receive visitors; then she thought this would be
rude after what he had done for her; then it occurred to her that
her aunt and the governor's wife had intentions concerning
herself and Rostov---their looks and words at times seemed to
confirm this supposition---then she told herself that only she,
with her sinful nature, could think this of them: they could not
forget that situated as she was, while still wearing deep
mourning, such matchmaking would be an insult to her and to her
father's memory. Assuming that she did go down to see him,
Princess Mary imagined the words he would say to her and what she
would say to him, and these words sometimes seemed undeservedly
cold and then to mean too much. More than anything she feared
lest the confusion she felt might overwhelm her and betray her as
soon as she saw him.

But when on Sunday after church the footman announced in the
drawing room that Count Rostov had called, the princess showed no
confusion, only a slight blush suffused her cheeks and her eyes
lit up with a new and radiant light.

``You have met him, Aunt?'' said she in a calm voice, unable
herself to understand that she could be outwardly so calm and
natural.

When Rostov entered the room, the princess dropped her eyes for
an instant, as if to give the visitor time to greet her aunt, and
then just as Nicholas turned to her she raised her head and met
his look with shining eyes. With a movement full of dignity and
grace she half rose with a smile of pleasure, held out her
slender, delicate hand to him, and began to speak in a voice in
which for the first time new deep womanly notes
vibrated. Mademoiselle Bourienne, who was in the drawing room,
looked at Princess Mary in bewildered surprise. Herself a
consummate coquette, she could not have maneuvered better on
meeting a man she wished to attract.

``Either black is particularly becoming to her or she really has
greatly improved without my having noticed it. And above all,
what tact and grace!'' thought Mademoiselle Bourienne.

Had Princess Mary been capable of reflection at that moment, she
would have been more surprised than Mademoiselle Bourienne at the
change that had taken place in herself. From the moment she
recognized that dear, loved face, a new life force took
possession of her and compelled her to speak and act apart from
her own will. From the time Rostov entered, her face became
suddenly transformed. It was as if a light had been kindled in a
carved and painted lantern and the intricate, skillful, artistic
work on its sides, that previously seemed dark, coarse, and
meaningless, was suddenly shown up in unexpected and striking
beauty. For the first time all that pure, spiritual, inward
travail through which she had lived appeared on the surface. All
her inward labor, her dissatisfaction with herself, her
sufferings, her strivings after goodness, her meekness, love, and
self-sacrifice---all this now shone in those radiant eyes, in her
delicate smile, and in every trait of her gentle face.

Rostov saw all this as clearly as if he had known her whole
life. He felt that the being before him was quite different from,
and better than, anyone he had met before, and above all better
than himself.

Their conversation was very simple and unimportant. They spoke of
the war, and like everyone else unconsciously exaggerated their
sorrow about it; they spoke of their last meeting---Nicholas
trying to change the subject---they talked of the governor's kind
wife, of Nicholas' relations, and of Princess Mary's.

She did not talk about her brother, diverting the conversation as
soon as her aunt mentioned Andrew. Evidently she could speak of
Russia's misfortunes with a certain artificiality, but her
brother was too near her heart and she neither could nor would
speak lightly of him. Nicholas noticed this, as he noticed every
shade of Princess Mary's character with an observation unusual to
him, and everything confirmed his conviction that she was a quite
unusual and extraordinary being.  Nicholas blushed and was
confused when people spoke to him about the princess (as she did
when he was mentioned) and even when he thought of her, but in
her presence he felt quite at ease, and said not at all what he
had prepared, but what, quite appropriately, occurred to him at
the moment.

When a pause occurred during his short visit, Nicholas, as is
usual when there are children, turned to Prince Andrew's little
son, caressing him and asking whether he would like to be an
hussar. He took the boy on his knee, played with him, and looked
round at Princess Mary. With a softened, happy, timid look she
watched the boy she loved in the arms of the man she
loved. Nicholas also noticed that look and, as if understanding
it, flushed with pleasure and began to kiss the boy with good
natured playfulness.

As she was in mourning Princess Mary did not go out into society,
and Nicholas did not think it the proper thing to visit her
again; but all the same the governor's wife went on with her
matchmaking, passing on to Nicholas the flattering things
Princess Mary said of him and vice versa, and insisting on his
declaring himself to Princess Mary. For this purpose she arranged
a meeting between the young people at the bishop's house before
Mass.

Though Rostov told the governor's wife that he would not make any
declaration to Princess Mary, he promised to go.

As at Tilsit Rostov had not allowed himself to doubt that what
everybody considered right was right, so now, after a short but
sincere struggle between his effort to arrange his life by his
own sense of justice, and in obedient submission to
circumstances, he chose the latter and yielded to the power he
felt irresistibly carrying him he knew not where. He knew that
after his promise to Sonya it would be what he deemed base to
declare his feelings to Princess Mary. And he knew that he would
never act basely. But he also knew (or rather felt at the bottom
of his heart) that by resigning himself now to the force of
circumstances and to those who were guiding him, he was not only
doing nothing wrong, but was doing something very
important---more important than anything he had ever done in his
life.

After meeting Princess Mary, though the course of his life went
on externally as before, all his former amusements lost their
charm for him and he often thought about her. But he never
thought about her as he had thought of all the young ladies
without exception whom he had met in society, nor as he had for a
long time, and at one time rapturously, thought about Sonya. He
had pictured each of those young ladies as almost all
honest-hearted young men do, that is, as a possible wife,
adapting her in his imagination to all the conditions of married
life: a white dressing gown, his wife at the tea table, his
wife's carriage, little ones, Mamma and Papa, their relations to
her, and so on---and these pictures of the future had given him
pleasure. But with Princess Mary, to whom they were trying to get
him engaged, he could never picture anything of future married
life. If he tried, his pictures seemed incongruous and false. It
made him afraid.

% % % % % % % % % % % % % % % % % % % % % % % % % % % % % % % % %
% % % % % % % % % % % % % % % % % % % % % % % % % % % % % % % % %
% % % % % % % % % % % % % % % % % % % % % % % % % % % % % % % % %
% % % % % % % % % % % % % % % % % % % % % % % % % % % % % % % % %
% % % % % % % % % % % % % % % % % % % % % % % % % % % % % % % % %
% % % % % % % % % % % % % % % % % % % % % % % % % % % % % % % % %
% % % % % % % % % % % % % % % % % % % % % % % % % % % % % % % % %
% % % % % % % % % % % % % % % % % % % % % % % % % % % % % % % % %
% % % % % % % % % % % % % % % % % % % % % % % % % % % % % % % % %
% % % % % % % % % % % % % % % % % % % % % % % % % % % % % % % % %
% % % % % % % % % % % % % % % % % % % % % % % % % % % % % % % % %
% % % % % % % % % % % % % % % % % % % % % % % % % % % % % %

\chapter*{Chapter VII} \ifaudio \marginpar{
\href{http://ia800203.us.archive.org/13/items/war_and_peace_12_0911_librivox/war_and_peace_12_07_tolstoy_64kb.mp3}{Audio}}
\fi

\initial{T}{he} dreadful news of the battle of Borodino, of our losses in
killed and wounded, and the still more terrible news of the loss
of Moscow reached Voronezh in the middle of September. Princess
Mary, having learned of her brother's wound only from the Gazette
and having no definite news of him, prepared (so Nicholas heard,
he had not seen her again himself) to set off in search of Prince
Andrew.

When he received the news of the battle of Borodino and the
abandonment of Moscow, Rostov was not seized with despair, anger,
the desire for vengeance, or any feeling of that kind, but
everything in Voronezh suddenly seemed to him dull and tiresome,
and he experienced an indefinite feeling of shame and
awkwardness. The conversations he heard seemed to him insincere;
he did not know how to judge all these affairs and felt that only
in the regiment would everything again become clear to him. He
made haste to finish buying the horses, and often became
unreasonably angry with his servant and squadron quartermaster.

A few days before his departure a special thanksgiving, at which
Nicholas was present, was held in the cathedral for the Russian
victory.  He stood a little behind the governor and held himself
with military decorum through the service, meditating on a great
variety of subjects.  When the service was over the governor's
wife beckoned him to her.

``Have you seen the princess?'' she asked, indicating with a
movement of her head a lady standing on the opposite side, beyond
the choir.

Nicholas immediately recognized Princess Mary not so much by the
profile he saw under her bonnet as by the feeling of solicitude,
timidity, and pity that immediately overcame him. Princess Mary,
evidently engrossed by her thoughts, was crossing herself for the
last time before leaving the church.

Nicholas looked at her face with surprise. It was the same face
he had seen before, there was the same general expression of
refined, inner, spiritual labor, but now it was quite differently
lit up. There was a pathetic expression of sorrow, prayer, and
hope in it. As had occurred before when she was present, Nicholas
went up to her without waiting to be prompted by the governor's
wife and not asking himself whether or not it was right and
proper to address her here in church, and told her he had heard
of her trouble and sympathized with his whole soul. As soon as
she heard his voice a vivid glow kindled in her face, lighting up
both her sorrow and her joy.

``There is one thing I wanted to tell you, Princess,'' said
Rostov. ``It is that if your brother, Prince Andrew Nikolievich,
were not living, it would have been at once announced in the
Gazette, as he is a colonel.''

The princess looked at him, not grasping what he was saying, but
cheered by the expression of regretful sympathy on his face.

``And I have known so many cases of a splinter wound'' (the
Gazette said it was a shell) ``either proving fatal at once or
being very slight,'' continued Nicholas. ``We must hope for the
best, and I am sure...''

Princess Mary interrupted him.

``Oh, that would be so dread...'' she began and, prevented by
agitation from finishing, she bent her head with a movement as
graceful as everything she did in his presence and, looking up at
him gratefully, went out, following her aunt.

That evening Nicholas did not go out, but stayed at home to
settle some accounts with the horse dealers. When he had finished
that business it was already too late to go anywhere but still
too early to go to bed, and for a long time he paced up and down
the room, reflecting on his life, a thing he rarely did.

Princess Mary had made an agreeable impression on him when he had
met her in Smolensk province. His having encountered her in such
exceptional circumstances, and his mother having at one time
mentioned her to him as a good match, had drawn his particular
attention to her. When he met her again in Voronezh the
impression she made on him was not merely pleasing but
powerful. Nicholas had been struck by the peculiar moral beauty
he observed in her at this time. He was, however, preparing to go
away and it had not entered his head to regret that he was thus
depriving himself of chances of meeting her. But that day's
encounter in church had, he felt, sunk deeper than was desirable
for his peace of mind. That pale, sad, refined face, that radiant
look, those gentle graceful gestures, and especially the deep and
tender sorrow expressed in all her features agitated him and
evoked his sympathy. In men Rostov could not bear to see the
expression of a higher spiritual life (that was why he did not
like Prince Andrew) and he referred to it contemptuously as
philosophy and dreaminess, but in Princess Mary that very sorrow
which revealed the depth of a whole spiritual world foreign to
him was an irresistible attraction.

``She must be a wonderful woman. A real angel!'' he said to
himself. ``Why am I not free? Why was I in such a hurry with
Sonya?'' And he involuntarily compared the two: the lack of
spirituality in the one and the abundance of it in the other---a
spirituality he himself lacked and therefore valued most
highly. He tried to picture what would happen were he free. How
he would propose to her and how she would become his wife.  But
no, he could not imagine that. He felt awed, and no clear picture
presented itself to his mind. He had long ago pictured to himself
a future with Sonya, and that was all clear and simple just
because it had all been thought out and he knew all there was in
Sonya, but it was impossible to picture a future with Princess
Mary, because he did not understand her but simply loved her.

Reveries about Sonya had had something merry and playful in them,
but to dream of Princess Mary was always difficult and a little
frightening.

``How she prayed!'' he thought. ``It was plain that her whole
soul was in her prayer. Yes, that was the prayer that moves
mountains, and I am sure her prayer will be answered. Why don't I
pray for what I want?'' he suddenly thought. ``What do I want? To
be free, released from Sonya...  She was right,'' he thought,
remembering what the governor's wife had said: ``Nothing but
misfortune can come of marrying Sonya. Muddles, grief for
Mamma... business difficulties... muddles, terrible muddles!
Besides, I don't love her---not as I should. O, God! release me
from this dreadful, inextricable position!'' he suddenly began to
pray. ``Yes, prayer can move mountains, but one must have faith
and not pray as Natasha and I used to as children, that the snow
might turn into sugar---and then run out into the yard to see
whether it had done so. No, but I am not praying for trifles
now,'' he thought as he put his pipe down in a corner, and
folding his hands placed himself before the icon. Softened by
memories of Princess Mary he began to pray as he had not done for
a long time. Tears were in his eyes and in his throat when the
door opened and Lavrushka came in with some papers.

``Blockhead! Why do you come in without being called?'' cried
Nicholas, quickly changing his attitude.

``From the governor,'' said Lavrushka in a sleepy voice. ``A
courier has arrived and there's a letter for you.''

``Well, all right, thanks. You can go!''

Nicholas took the two letters, one of which was from his mother
and the other from Sonya. He recognized them by the handwriting
and opened Sonya's first. He had read only a few lines when he
turned pale and his eyes opened wide with fear and joy.

``No, it's not possible!'' he cried aloud.

Unable to sit still he paced up and down the room holding the
letter and reading it. He glanced through it, then read it again,
and then again, and standing still in the middle of the room he
raised his shoulders, stretching out his hands, with his mouth
wide open and his eyes fixed.  What he had just been praying for
with confidence that God would hear him had come to pass; but
Nicholas was as much astonished as if it were something
extraordinary and unexpected, and as if the very fact that it had
happened so quickly proved that it had not come from God to whom
he had prayed, but by some ordinary coincidence.

This unexpected and, as it seemed to Nicholas, quite voluntary
letter from Sonya freed him from the knot that fettered him and
from which there had seemed no escape. She wrote that the last
unfortunate events---the loss of almost the whole of the Rostovs'
Moscow property---and the countess' repeatedly expressed wish
that Nicholas should marry Princess Bolkonskaya, together with
his silence and coldness of late, had all combined to make her
decide to release him from his promise and set him completely
free.  

\begin{quote} \calli 
It would be too painful to me to
think that I might be a cause of sorrow or discord in the family
that has been so good to me (she wrote), and my love has no aim
but the happiness of those I love; so, Nicholas, I beg you to
consider yourself free, and to be assured that, in spite of
everything, no one can love you more than does

  Your Sonya 
\end{quote} 
  
  Both letters were written from
Troitsa. The other, from the countess, described their last days
in Moscow, their departure, the fire, and the destruction of all
their property. In this letter the countess also mentioned that
Prince Andrew was among the wounded traveling with them; his
state was very critical, but the doctor said there was now more
hope. Sonya and Natasha were nursing him.

Next day Nicholas took his mother's letter and went to see
Princess Mary. Neither he nor she said a word about what
\emph{Natasha nursing him} might mean, but thanks to this letter
Nicholas suddenly became almost as intimate with the princess as
if they were relations.

The following day he saw Princess Mary off on her journey to
Yaroslavl, and a few days later left to rejoin his regiment.

% % % % % % % % % % % % % % % % % % % % % % % % % % % % % % % % %
% % % % % % % % % % % % % % % % % % % % % % % % % % % % % % % % %
% % % % % % % % % % % % % % % % % % % % % % % % % % % % % % % % %
% % % % % % % % % % % % % % % % % % % % % % % % % % % % % % % % %
% % % % % % % % % % % % % % % % % % % % % % % % % % % % % % % % %
% % % % % % % % % % % % % % % % % % % % % % % % % % % % % % % % %
% % % % % % % % % % % % % % % % % % % % % % % % % % % % % % % % %
% % % % % % % % % % % % % % % % % % % % % % % % % % % % % % % % %
% % % % % % % % % % % % % % % % % % % % % % % % % % % % % % % % %
% % % % % % % % % % % % % % % % % % % % % % % % % % % % % % % % %
% % % % % % % % % % % % % % % % % % % % % % % % % % % % % % % % %
% % % % % % % % % % % % % % % % % % % % % % % % % % % % % %

\chapter*{Chapter VIII} \ifaudio \marginpar{
\href{http://ia800203.us.archive.org/13/items/war_and_peace_12_0911_librivox/war_and_peace_12_08_tolstoy_64kb.mp3}{Audio}}
\fi

\initial{S}{onya}'s letter written from Troitsa, which had come as an answer
to Nicholas' prayer, was prompted by this: the thought of getting
Nicholas married to an heiress occupied the old countess' mind
more and more. She knew that Sonya was the chief obstacle to this
happening, and Sonya's life in the countess' house had grown
harder and harder, especially after they had received a letter
from Nicholas telling of his meeting with Princess Mary in
Bogucharovo. The countess let no occasion slip of making
humiliating or cruel allusions to Sonya.

But a few days before they left Moscow, moved and excited by all
that was going on, she called Sonya to her and, instead of
reproaching and making demands on her, tearfully implored her to
sacrifice herself and repay all that the family had done for her
by breaking off her engagement with Nicholas.

``I shall not be at peace till you promise me this.''

Sonya burst into hysterical tears and replied through her sobs
that she would do anything and was prepared for anything, but
gave no actual promise and could not bring herself to decide to
do what was demanded of her. She must sacrifice herself for the
family that had reared and brought her up. To sacrifice herself
for others was Sonya's habit. Her position in the house was such
that only by sacrifice could she show her worth, and she was
accustomed to this and loved doing it. But in all her former acts
of self-sacrifice she had been happily conscious that they raised
her in her own esteem and in that of others, and so made her more
worthy of Nicholas whom she loved more than anything in the
world. But now they wanted her to sacrifice the very thing that
constituted the whole reward for her self-sacrifice and the whole
meaning of her life.  And for the first time she felt bitterness
against those who had been her benefactors only to torture her
the more painfully; she felt jealous of Natasha who had never
experienced anything of this sort, had never needed to sacrifice
herself, but made others sacrifice themselves for her and yet was
beloved by everybody. And for the first time Sonya felt that out
of her pure, quiet love for Nicholas a passionate feeling was
beginning to grow up which was stronger than principle, virtue,
or religion. Under the influence of this feeling Sonya, whose
life of dependence had taught her involuntarily to be secretive,
having answered the countess in vague general terms, avoided
talking with her and resolved to wait till she should see
Nicholas, not in order to set him free but on the contrary at
that meeting to bind him to her forever.

The bustle and terror of the Rostovs' last days in Moscow stifled
the gloomy thoughts that oppressed Sonya. She was glad to find
escape from them in practical activity. But when she heard of
Prince Andrew's presence in their house, despite her sincere pity
for him and for Natasha, she was seized by a joyful and
superstitious feeling that God did not intend her to be separated
from Nicholas. She knew that Natasha loved no one but Prince
Andrew and had never ceased to love him. She knew that being
thrown together again under such terrible circumstances they
would again fall in love with one another, and that Nicholas
would then not be able to marry Princess Mary as they would be
within the prohibited degrees of affinity. Despite all the terror
of what had happened during those last days and during the first
days of their journey, this feeling that Providence was
intervening in her personal affairs cheered Sonya.

At the Troitsa monastery the Rostovs first broke their journey
for a whole day.

Three large rooms were assigned to them in the monastery
hostelry, one of which was occupied by Prince Andrew. The wounded
man was much better that day and Natasha was sitting with him. In
the next room sat the count and countess respectfully conversing
with the prior, who was calling on them as old acquaintances and
benefactors of the monastery.  Sonya was there too, tormented by
curiosity as to what Prince Andrew and Natasha were talking
about. She heard the sound of their voices through the door. That
door opened and Natasha came out, looking excited. Not noticing
the monk, who had risen to greet her and was drawing back the
wide sleeve on his right arm, she went up to Sonya and took her
hand.

``Natasha, what are you about? Come here!'' said the countess.

Natasha went up to the monk for his blessing, and he advised her
to pray for aid to God and His saint.

As soon as the prior withdrew, Natasha took her friend by the
hand and went with her into the unoccupied room.

``Sonya, will he live?'' she asked. ``Sonya, how happy I am, and
how unhappy!... Sonya, dovey, everything is as it used to be. If
only he lives! He cannot... because... because... of'' and
Natasha burst into tears.

``Yes! I knew it! Thank God!'' murmured Sonya. ``He will live.''

Sonya was not less agitated than her friend by the latter's fear
and grief and by her own personal feelings which she shared with
no one.  Sobbing, she kissed and comforted Natasha. ``If only he
lives!'' she thought. Having wept, talked, and wiped away their
tears, the two friends went together to Prince Andrew's
door. Natasha opened it cautiously and glanced into the room,
Sonya standing beside her at the half-open door.

Prince Andrew was lying raised high on three pillows. His pale
face was calm, his eyes closed, and they could see his regular
breathing.

``O, Natasha!'' Sonya suddenly almost screamed, catching her
companion's arm and stepping back from the door.

``What? What is it?'' asked Natasha.

``It's that, that...'' said Sonya, with a white face and
trembling lips.

Natasha softly closed the door and went with Sonya to the window,
not yet understanding what the latter was telling her.

``You remember,'' said Sonya with a solemn and frightened
expression. ``You remember when I looked in the mirror for
you... at Otradnoe at Christmas? Do you remember what I saw?''

``Yes, yes!'' cried Natasha opening her eyes wide, and vaguely
recalling that Sonya had told her something about Prince Andrew
whom she had seen lying down.

``You remember?'' Sonya went on. ``I saw it then and told
everybody, you and Dunyasha. I saw him lying on a bed,'' said
she, making a gesture with her hand and a lifted finger at each
detail, ``and that he had his eyes closed and was covered just
with a pink quilt, and that his hands were folded,'' she
concluded, convincing herself that the details she had just seen
were exactly what she had seen in the mirror.

She had in fact seen nothing then but had mentioned the first
thing that came into her head, but what she had invented then
seemed to her now as real as any other recollection. She not only
remembered what she had then said---that he turned to look at her
and smiled and was covered with something red---but was firmly
convinced that she had then seen and said that he was covered
with a pink quilt and that his eyes were closed.

``Yes, yes, it really was pink!'' cried Natasha, who now thought
she too remembered the word pink being used, and saw in this the
most extraordinary and mysterious part of the prediction.

``But what does it mean?'' she added meditatively.

``Oh, I don't know, it is all so strange,'' replied Sonya,
clutching at her head.

A few minutes later Prince Andrew rang and Natasha went to him,
but Sonya, feeling unusually excited and touched, remained at the
window thinking about the strangeness of what had occurred.

They had an opportunity that day to send letters to the army, and
the countess was writing to her son.

``Sonya!'' said the countess, raising her eyes from her letter as
her niece passed, ``Sonya, won't you write to Nicholas?'' She
spoke in a soft, tremulous voice, and in the weary eyes that
looked over her spectacles Sonya read all that the countess meant
to convey with these words. Those eyes expressed entreaty, shame
at having to ask, fear of a refusal, and readiness for relentless
hatred in case of such refusal.

Sonya went up to the countess and, kneeling down, kissed her
hand.

``Yes, Mamma, I will write,'' said she.

Sonya was softened, excited, and touched by all that had occurred
that day, especially by the mysterious fulfillment she had just
seen of her vision. Now that she knew that the renewal of
Natasha's relations with Prince Andrew would prevent Nicholas
from marrying Princess Mary, she was joyfully conscious of a
return of that self-sacrificing spirit in which she was
accustomed to live and loved to live. So with a joyful
consciousness of performing a magnanimous deed---interrupted
several times by the tears that dimmed her velvety black
eyes---she wrote that touching letter the arrival of which had so
amazed Nicholas.

% % % % % % % % % % % % % % % % % % % % % % % % % % % % % % % % %
% % % % % % % % % % % % % % % % % % % % % % % % % % % % % % % % %
% % % % % % % % % % % % % % % % % % % % % % % % % % % % % % % % %
% % % % % % % % % % % % % % % % % % % % % % % % % % % % % % % % %
% % % % % % % % % % % % % % % % % % % % % % % % % % % % % % % % %
% % % % % % % % % % % % % % % % % % % % % % % % % % % % % % % % %
% % % % % % % % % % % % % % % % % % % % % % % % % % % % % % % % %
% % % % % % % % % % % % % % % % % % % % % % % % % % % % % % % % %
% % % % % % % % % % % % % % % % % % % % % % % % % % % % % % % % %
% % % % % % % % % % % % % % % % % % % % % % % % % % % % % % % % %
% % % % % % % % % % % % % % % % % % % % % % % % % % % % % % % % %
% % % % % % % % % % % % % % % % % % % % % % % % % % % % % %

\chapter*{Chapter IX} \ifaudio \marginpar{
\href{http://ia800203.us.archive.org/13/items/war_and_peace_12_0911_librivox/war_and_peace_12_09_tolstoy_64kb.mp3}{Audio}}
\fi

\initial{T}{he} officer and soldiers who had arrested Pierre treated him with
hostility but yet with respect, in the guardhouse to which he was
taken.  In their attitude toward him could still be felt both
uncertainty as to who he might be---perhaps a very important
person---and hostility as a result of their recent personal
conflict with him.

But when the guard was relieved next morning, Pierre felt that
for the new guard---both officers and men---he was not as
interesting as he had been to his captors; and in fact the guard
of the second day did not recognize in this big, stout man in a
peasant coat the vigorous person who had fought so desperately
with the marauder and the convoy and had uttered those solemn
words about saving a child; they saw in him only No. 17 of the
captured Russians, arrested and detained for some reason by order
of the Higher Command. If they noticed anything remarkable about
Pierre, it was only his unabashed, meditative concentration and
thoughtfulness, and the way he spoke French, which struck them as
surprisingly good. In spite of this he was placed that day with
the other arrested suspects, as the separate room he had occupied
was required by an officer.

All the Russians confined with Pierre were men of the lowest
class and, recognizing him as a gentleman, they all avoided him,
more especially as he spoke French. Pierre felt sad at hearing
them making fun of him.

That evening he learned that all these prisoners (he, probably,
among them) were to be tried for incendiarism. On the third day
he was taken with the others to a house where a French general
with a white mustache sat with two colonels and other Frenchmen
with scarves on their arms.  With the precision and definiteness
customary in addressing prisoners, and which is supposed to
preclude human frailty, Pierre like the others was questioned as
to who he was, where he had been, with what object, and so on.

These questions, like questions put at trials generally, left the
essence of the matter aside, shut out the possibility of that
essence's being revealed, and were designed only to form a
channel through which the judges wished the answers of the
accused to flow so as to lead to the desired result, namely a
conviction. As soon as Pierre began to say anything that did not
fit in with that aim, the channel was removed and the water could
flow to waste. Pierre felt, moreover, what the accused always
feel at their trial, perplexity as to why these questions were
put to him. He had a feeling that it was only out of
condescension or a kind of civility that this device of placing a
channel was employed. He knew he was in these men's power, that
only by force had they brought him there, that force alone gave
them the right to demand answers to their questions, and that the
sole object of that assembly was to inculpate him. And so, as
they had the power and wish to inculpate him, this expedient of
an inquiry and trial seemed unnecessary. It was evident that any
answer would lead to conviction. When asked what he was doing
when he was arrested, Pierre replied in a rather tragic manner
that he was restoring to its parents a child he had saved from
the flames. Why had he fought the marauder? Pierre answered that
he \emph{was protecting a woman}, and that \emph{to protect a
woman who was being insulted was the duty of every man; that...}
They interrupted him, for this was not to the point. Why was he
in the yard of a burning house where witnesses had seen him? He
replied that he had gone out to see what was happening in
Moscow. Again they interrupted him: they had not asked where he
was going, but why he was found near the fire? Who was he? they
asked, repeating their first question, which he had declined to
answer.  Again he replied that he could not answer it.

``Put that down, that's bad... very bad,'' sternly remarked the
general with the white mustache and red flushed face.

On the fourth day fires broke out on the Zubovski rampart.

Pierre and thirteen others were moved to the coach house of a
merchant's house near the Crimean bridge. On his way through the
streets Pierre felt stifled by the smoke which seemed to hang
over the whole city.  Fires were visible on all sides. He did not
then realize the significance of the burning of Moscow, and
looked at the fires with horror.

He passed four days in the coach house near the Crimean bridge
and during that time learned, from the talk of the French
soldiers, that all those confined there were awaiting a decision
which might come any day from the marshal. What marshal this was,
Pierre could not learn from the soldiers. Evidently for them
\emph{the marshal} represented a very high and rather mysterious
power.

These first days, before the eighth of September when the
prisoners were had up for a second examination, were the hardest
of all for Pierre.

% % % % % % % % % % % % % % % % % % % % % % % % % % % % % % % % %
% % % % % % % % % % % % % % % % % % % % % % % % % % % % % % % % %
% % % % % % % % % % % % % % % % % % % % % % % % % % % % % % % % %
% % % % % % % % % % % % % % % % % % % % % % % % % % % % % % % % %
% % % % % % % % % % % % % % % % % % % % % % % % % % % % % % % % %
% % % % % % % % % % % % % % % % % % % % % % % % % % % % % % % % %
% % % % % % % % % % % % % % % % % % % % % % % % % % % % % % % % %
% % % % % % % % % % % % % % % % % % % % % % % % % % % % % % % % %
% % % % % % % % % % % % % % % % % % % % % % % % % % % % % % % % %
% % % % % % % % % % % % % % % % % % % % % % % % % % % % % % % % %
% % % % % % % % % % % % % % % % % % % % % % % % % % % % % % % % %
% % % % % % % % % % % % % % % % % % % % % % % % % % % % % %

\chapter*{Chapter X} \ifaudio \marginpar{
\href{http://ia800203.us.archive.org/13/items/war_and_peace_12_0911_librivox/war_and_peace_12_10_tolstoy_64kb.mp3}{Audio}}
\fi

\initial{O}{n} the eighth of September an officer---a very important one
judging by the respect the guards showed him---entered the coach
house where the prisoners were. This officer, probably someone on
the staff, was holding a paper in his hand, and called over all
the Russians there, naming Pierre as \emph{the man who does not
give his name}. Glancing indolently and indifferently at all the
prisoners, he ordered the officer in charge to have them decently
dressed and tidied up before taking them to the marshal. An hour
later a squad of soldiers arrived and Pierre with thirteen others
was led to the Virgin's Field. It was a fine day, sunny after
rain, and the air was unusually pure. The smoke did not hang low
as on the day when Pierre had been taken from the guardhouse on
the Zubovski rampart, but rose through the pure air in
columns. No flames were seen, but columns of smoke rose on all
sides, and all Moscow as far as Pierre could see was one vast
charred ruin. On all sides there were waste spaces with only
stoves and chimney stacks still standing, and here and there the
blackened walls of some brick houses. Pierre gazed at the ruins
and did not recognize districts he had known well. Here and there
he could see churches that had not been burned. The Kremlin,
which was not destroyed, gleamed white in the distance with its
towers and the belfry of Ivan the Great. The domes of the New
Convent of the Virgin glittered brightly and its bells were
ringing particularly clearly.  These bells reminded Pierre that
it was Sunday and the feast of the Nativity of the Virgin. But
there seemed to be no one to celebrate this holiday: everywhere
were blackened ruins, and the few Russians to be seen were
tattered and frightened people who tried to hide when they saw
the French.

It was plain that the Russian nest was ruined and destroyed, but
in place of the Russian order of life that had been destroyed,
Pierre unconsciously felt that a quite different, firm, French
order had been established over this ruined nest. He felt this in
the looks of the soldiers who, marching in regular ranks briskly
and gaily, were escorting him and the other criminals; he felt it
in the looks of an important French official in a carriage and
pair driven by a soldier, whom they met on the way. He felt it in
the merry sounds of regimental music he heard from the left side
of the field, and felt and realized it especially from the list
of prisoners the French officer had read out when he came that
morning. Pierre had been taken by one set of soldiers and led
first to one and then to another place with dozens of other men,
and it seemed that they might have forgotten him, or confused him
with the others. But no: the answers he had given when questioned
had come back to him in his designation as ``the man who does not
give his name,'' and under that appellation, which to Pierre
seemed terrible, they were now leading him somewhere with
unhesitating assurance on their faces that he and all the other
prisoners were exactly the ones they wanted and that they were
being taken to the proper place. Pierre felt himself to be an
insignificant chip fallen among the wheels of a machine whose
action he did not understand but which was working well.

He and the other prisoners were taken to the right side of the
Virgin's Field, to a large white house with an immense garden not
far from the convent. This was Prince Shcherbitov's house, where
Pierre had often been in other days, and which, as he learned
from the talk of the soldiers, was now occupied by the marshal,
the Duke of Eckmuhl (Davout).

They were taken to the entrance and led into the house one by
one.  Pierre was the sixth to enter. He was conducted through a
glass gallery, an anteroom, and a hall, which were familiar to
him, into a long low study at the door of which stood an
adjutant.

Davout, spectacles on nose, sat bent over a table at the further
end of the room. Pierre went close up to him, but Davout,
evidently consulting a paper that lay before him, did not look
up. Without raising his eyes, he said in a low voice:

``Who are you?''

Pierre was silent because he was incapable of uttering a word. To
him Davout was not merely a French general, but a man notorious
for his cruelty. Looking at his cold face, as he sat like a stern
schoolmaster who was prepared to wait awhile for an answer,
Pierre felt that every instant of delay might cost him his life;
but he did not know what to say. He did not venture to repeat
what he had said at his first examination, yet to disclose his
rank and position was dangerous and embarrassing. So he was
silent. But before he had decided what to do, Davout raised his
head, pushed his spectacles back on his forehead, screwed up his
eyes, and looked intently at him.

``I know that man,'' he said in a cold, measured tone, evidently
calculated to frighten Pierre.

The chill that had been running down Pierre's back now seized his
head as in a vise.

``You cannot know me, General, I have never seen you...''

``He is a Russian spy,'' Davout interrupted, addressing another
general who was present, but whom Pierre had not noticed.

Davout turned away. With an unexpected reverberation in his voice
Pierre rapidly began:

``No, monseigneur,'' he said, suddenly remembering that Davout
was a duke.  ``No, monseigneur, you cannot have known me. I am a
militia officer and have not quitted Moscow.''

``Your name?'' asked Davout.

``Bezukhov.''

``What proof have I that you are not lying?''

``Monseigneur!'' exclaimed Pierre, not in an offended but in a
pleading voice.

Davout looked up and gazed intently at him. For some seconds they
looked at one another, and that look saved Pierre. Apart from
conditions of war and law, that look established human relations
between the two men. At that moment an immense number of things
passed dimly through both their minds, and they realized that
they were both children of humanity and were brothers.

At the first glance, when Davout had only raised his head from
the papers where human affairs and lives were indicated by
numbers, Pierre was merely a circumstance, and Davout could have
shot him without burdening his conscience with an evil deed, but
now he saw in him a human being. He reflected for a moment.

``How can you show me that you are telling the truth?'' said
Davout coldly.

Pierre remembered Ramballe, and named him and his regiment and
the street where the house was.

``You are not what you say,'' returned Davout.

In a trembling, faltering voice Pierre began adducing proofs of
the truth of his statements.

But at that moment an adjutant entered and reported something to
Davout.

Davout brightened up at the news the adjutant brought, and began
buttoning up his uniform. It seemed that he had quite forgotten
Pierre.

When the adjutant reminded him of the prisoner, he jerked his
head in Pierre's direction with a frown and ordered him to be led
away. But where they were to take him Pierre did not know: back
to the coach house or to the place of execution his companions
had pointed out to him as they crossed the Virgin's Field.

He turned his head and saw that the adjutant was putting another
question to Davout.

``Yes, of course!'' replied Davout, but what this \emph{yes}
meant, Pierre did not know.

Pierre could not afterwards remember how he went, whether it was
far, or in which direction. His faculties were quite numbed, he
was stupefied, and noticing nothing around him went on moving his
legs as the others did till they all stopped and he stopped
too. The only thought in his mind at that time was: who was it
that had really sentenced him to death? Not the men on the
commission that had first examined him---not one of them wished
to or, evidently, could have done it. It was not Davout, who had
looked at him in so human a way. In another moment Davout would
have realized that he was doing wrong, but just then the adjutant
had come in and interrupted him. The adjutant, also, had
evidently had no evil intent though he might have refrained from
coming in. Then who was executing him, killing him, depriving him
of life---him, Pierre, with all his memories, aspirations, hopes,
and thoughts? Who was doing this? And Pierre felt that it was no
one.

It was a system---a concurrence of circumstances.

A system of some sort was killing him---Pierre---depriving him of
life, of everything, annihilating him.

% % % % % % % % % % % % % % % % % % % % % % % % % % % % % % % % %
% % % % % % % % % % % % % % % % % % % % % % % % % % % % % % % % %
% % % % % % % % % % % % % % % % % % % % % % % % % % % % % % % % %
% % % % % % % % % % % % % % % % % % % % % % % % % % % % % % % % %
% % % % % % % % % % % % % % % % % % % % % % % % % % % % % % % % %
% % % % % % % % % % % % % % % % % % % % % % % % % % % % % % % % %
% % % % % % % % % % % % % % % % % % % % % % % % % % % % % % % % %
% % % % % % % % % % % % % % % % % % % % % % % % % % % % % % % % %
% % % % % % % % % % % % % % % % % % % % % % % % % % % % % % % % %
% % % % % % % % % % % % % % % % % % % % % % % % % % % % % % % % %
% % % % % % % % % % % % % % % % % % % % % % % % % % % % % % % % %
% % % % % % % % % % % % % % % % % % % % % % % % % % % % % %

\chapter*{Chapter XI} \ifaudio \marginpar{
\href{http://ia800203.us.archive.org/13/items/war_and_peace_12_0911_librivox/war_and_peace_12_11_tolstoy_64kb.mp3}{Audio}}
\fi

\initial{F}{rom} Prince Shcherbatov's house the prisoners were led straight
down the Virgin's Field, to the left of the nunnery, as far as a
kitchen garden in which a post had been set up. Beyond that post
a fresh pit had been dug in the ground, and near the post and the
pit a large crowd stood in a semicircle. The crowd consisted of a
few Russians and many of Napoleon's soldiers who were not on
duty---Germans, Italians, and Frenchmen, in a variety of
uniforms. To the right and left of the post stood rows of French
troops in blue uniforms with red epaulets and high boots and
shakos.

The prisoners were placed in a certain order, according to the
list (Pierre was sixth), and were led to the post. Several drums
suddenly began to beat on both sides of them, and at that sound
Pierre felt as if part of his soul had been torn away. He lost
the power of thinking or understanding. He could only hear and
see. And he had only one wish---that the frightful thing that had
to happen should happen quickly.  Pierre looked round at his
fellow prisoners and scrutinized them.

The two first were convicts with shaven heads. One was tall and
thin, the other dark, shaggy, and sinewy, with a flat nose. The
third was a domestic serf, about forty-five years old, with
grizzled hair and a plump, well-nourished body. The fourth was a
peasant, a very handsome man with a broad, light-brown beard and
black eyes. The fifth was a factory hand, a thin, sallow-faced
lad of eighteen in a loose coat.

Pierre heard the French consulting whether to shoot them
separately or two at a time. ``In couples,'' replied the officer
in command in a calm voice. There was a stir in the ranks of the
soldiers and it was evident that they were all hurrying---not as
men hurry to do something they understand, but as people hurry to
finish a necessary but unpleasant and incomprehensible task.

A French official wearing a scarf came up to the right of the row
of prisoners and read out the sentence in Russian and in French.

Then two pairs of Frenchmen approached the criminals and at the
officer's command took the two convicts who stood first in the
row. The convicts stopped when they reached the post and, while
sacks were being brought, looked dumbly around as a wounded beast
looks at an approaching huntsman. One crossed himself
continually, the other scratched his back and made a movement of
the lips resembling a smile. With hurried hands the soldiers
blindfolded them, drawing the sacks over their heads, and bound
them to the post.

Twelve sharpshooters with muskets stepped out of the ranks with a
firm regular tread and halted eight paces from the post. Pierre
turned away to avoid seeing what was going to happen. Suddenly a
crackling, rolling noise was heard which seemed to him louder
than the most terrific thunder, and he looked round. There was
some smoke, and the Frenchmen were doing something near the pit,
with pale faces and trembling hands.  Two more prisoners were led
up. In the same way and with similar looks, these two glanced
vainly at the onlookers with only a silent appeal for protection
in their eyes, evidently unable to understand or believe what was
going to happen to them. They could not believe it because they
alone knew what their life meant to them, and so they neither
understood nor believed that it could be taken from them.

Again Pierre did not wish to look and again turned away; but
again the sound as of a frightful explosion struck his ear, and
at the same moment he saw smoke, blood, and the pale, scared
faces of the Frenchmen who were again doing something by the
post, their trembling hands impeding one another. Pierre,
breathing heavily, looked around as if asking what it meant. The
same question was expressed in all the looks that met his.

On the faces of all the Russians and of the French soldiers and
officers without exception, he read the same dismay, horror, and
conflict that were in his own heart. ``But who, after all, is
doing this? They are all suffering as I am. Who then is it?
Who?''  flashed for an instant through his mind.

``Sharpshooters of the 86th, forward!'' shouted someone. The
fifth prisoner, the one next to Pierre, was led
away---alone. Pierre did not understand that he was saved, that
he and the rest had been brought there only to witness the
execution. With ever-growing horror, and no sense of joy or
relief, he gazed at what was taking place. The fifth man was the
factory lad in the loose cloak. The moment they laid hands on him
he sprang aside in terror and clutched at Pierre. (Pierre
shuddered and shook himself free.) The lad was unable to
walk. They dragged him along, holding him up under the arms, and
he screamed. When they got him to the post he grew quiet, as if
he suddenly understood something.  Whether he understood that
screaming was useless or whether he thought it incredible that
men should kill him, at any rate he took his stand at the post,
waiting to be blindfolded like the others, and like a wounded
animal looked around him with glittering eyes.

Pierre was no longer able to turn away and close his eyes. His
curiosity and agitation, like that of the whole crowd, reached
the highest pitch at this fifth murder. Like the others this
fifth man seemed calm; he wrapped his loose cloak closer and
rubbed one bare foot with the other.

When they began to blindfold him he himself adjusted the knot
which hurt the back of his head; then when they propped him
against the bloodstained post, he leaned back and, not being
comfortable in that position, straightened himself, adjusted his
feet, and leaned back again more comfortably. Pierre did not take
his eyes from him and did not miss his slightest movement.

Probably a word of command was given and was followed by the
reports of eight muskets; but try as he would Pierre could not
afterwards remember having heard the slightest sound of the
shots. He only saw how the workman suddenly sank down on the
cords that held him, how blood showed itself in two places, how
the ropes slackened under the weight of the hanging body, and how
the workman sat down, his head hanging unnaturally and one leg
bent under him. Pierre ran up to the post. No one hindered
him. Pale, frightened people were doing something around the
workman.  The lower jaw of an old Frenchman with a thick mustache
trembled as he untied the ropes. The body collapsed. The soldiers
dragged it awkwardly from the post and began pushing it into the
pit.

They all plainly and certainly knew that they were criminals who
must hide the traces of their guilt as quickly as possible.

Pierre glanced into the pit and saw that the factory lad was
lying with his knees close up to his head and one shoulder higher
than the other.  That shoulder rose and fell rhythmically and
convulsively, but spadefuls of earth were already being thrown
over the whole body. One of the soldiers, evidently suffering,
shouted gruffly and angrily at Pierre to go back. But Pierre did
not understand him and remained near the post, and no one drove
him away.

When the pit had been filled up a command was given. Pierre was
taken back to his place, and the rows of troops on both sides of
the post made a half turn and went past it at a measured
pace. The twenty-four sharpshooters with discharged muskets,
standing in the center of the circle, ran back to their places as
the companies passed by.

Pierre gazed now with dazed eyes at these sharpshooters who ran
in couples out of the circle. All but one rejoined their
companies. This one, a young soldier, his face deadly pale, his
shako pushed back, and his musket resting on the ground, still
stood near the pit at the spot from which he had fired. He swayed
like a drunken man, taking some steps forward and back to save
himself from falling. An old, noncommissioned officer ran out of
the ranks and taking him by the elbow dragged him to his
company. The crowd of Russians and Frenchmen began to
disperse. They all went away silently and with drooping heads.

``That will teach them to start fires,'' said one of the
Frenchmen.

Pierre glanced round at the speaker and saw that it was a soldier
who was trying to find some relief after what had been done, but
was not able to do so. Without finishing what he had begun to say
he made a hopeless movement with his arm and went away.

% % % % % % % % % % % % % % % % % % % % % % % % % % % % % % % % %
% % % % % % % % % % % % % % % % % % % % % % % % % % % % % % % % %
% % % % % % % % % % % % % % % % % % % % % % % % % % % % % % % % %
% % % % % % % % % % % % % % % % % % % % % % % % % % % % % % % % %
% % % % % % % % % % % % % % % % % % % % % % % % % % % % % % % % %
% % % % % % % % % % % % % % % % % % % % % % % % % % % % % % % % %
% % % % % % % % % % % % % % % % % % % % % % % % % % % % % % % % %
% % % % % % % % % % % % % % % % % % % % % % % % % % % % % % % % %
% % % % % % % % % % % % % % % % % % % % % % % % % % % % % % % % %
% % % % % % % % % % % % % % % % % % % % % % % % % % % % % % % % %
% % % % % % % % % % % % % % % % % % % % % % % % % % % % % % % % %
% % % % % % % % % % % % % % % % % % % % % % % % % % % % % %

\chapter*{Chapter XII} \ifaudio \marginpar{
\href{http://ia800203.us.archive.org/13/items/war_and_peace_12_0911_librivox/war_and_peace_12_12_tolstoy_64kb.mp3}{Audio}}
\fi

\initial{A}{fter} the execution Pierre was separated from the rest of the
prisoners and placed alone in a small, ruined, and befouled
church.

Toward evening a noncommissioned officer entered with two
soldiers and told him that he had been pardoned and would now go
to the barracks for the prisoners of war. Without understanding
what was said to him, Pierre got up and went with the
soldiers. They took him to the upper end of the field, where
there were some sheds built of charred planks, beams, and
battens, and led him into one of them. In the darkness some
twenty different men surrounded Pierre. He looked at them without
understanding who they were, why they were there, or what they
wanted of him. He heard what they said, but did not understand
the meaning of the words and made no kind of deduction from or
application of them. He replied to questions they put to him, but
did not consider who was listening to his replies, nor how they
would understand them. He looked at their faces and figures, but
they all seemed to him equally meaningless.

From the moment Pierre had witnessed those terrible murders
committed by men who did not wish to commit them, it was as if
the mainspring of his life, on which everything depended and
which made everything appear alive, had suddenly been wrenched
out and everything had collapsed into a heap of meaningless
rubbish. Though he did not acknowledge it to himself, his faith
in the right ordering of the universe, in humanity, in his own
soul, and in God, had been destroyed. He had experienced this
before, but never so strongly as now. When similar doubts had
assailed him before, they had been the result of his own
wrongdoing, and at the bottom of his heart he had felt that
relief from his despair and from those doubts was to be found
within himself. But now he felt that the universe had crumbled
before his eyes and only meaningless ruins remained, and this not
by any fault of his own. He felt that it was not in his power to
regain faith in the meaning of life.

Around him in the darkness men were standing and evidently
something about him interested them greatly. They were telling
him something and asking him something. Then they led him away
somewhere, and at last he found himself in a corner of the shed
among men who were laughing and talking on all sides.

``Well, then, mates... that very prince who...'' some voice at
the other end of the shed was saying, with a strong emphasis on
the word who.

Sitting silent and motionless on a heap of straw against the
wall, Pierre sometimes opened and sometimes closed his eyes. But
as soon as he closed them he saw before him the dreadful face of
the factory lad---especially dreadful because of its
simplicity---and the faces of the murderers, even more dreadful
because of their disquiet. And he opened his eyes again and
stared vacantly into the darkness around him.

Beside him in a stooping position sat a small man of whose
presence he was first made aware by a strong smell of
perspiration which came from him every time he moved. This man
was doing something to his legs in the darkness, and though
Pierre could not see his face he felt that the man continually
glanced at him. On growing used to the darkness Pierre saw that
the man was taking off his leg bands, and the way he did it
aroused Pierre's interest.

Having unwound the string that tied the band on one leg, he
carefully coiled it up and immediately set to work on the other
leg, glancing up at Pierre. While one hand hung up the first
string the other was already unwinding the band on the second
leg. In this way, having carefully removed the leg bands by deft
circular motions of his arm following one another
uninterruptedly, the man hung the leg bands up on some pegs fixed
above his head. Then he took out a knife, cut something, closed
the knife, placed it under the head of his bed, and, seating
himself comfortably, clasped his arms round his lifted knees and
fixed his eyes on Pierre. The latter was conscious of something
pleasant, comforting, and well-rounded in these deft movements,
in the man's well-ordered arrangements in his corner, and even in
his very smell, and he looked at the man without taking his eyes
from him.

``You've seen a lot of trouble, sir, eh?'' the little man
suddenly said.

And there was so much kindliness and simplicity in his singsong
voice that Pierre tried to reply, but his jaw trembled and he
felt tears rising to his eyes. The little fellow, giving Pierre
no time to betray his confusion, instantly continued in the same
pleasant tones:

``Eh, lad, don't fret!'' said he, in the tender singsong
caressing voice old Russian peasant women employ. ``Don't fret,
friend---'suffer an hour, live for an age!' that's how it is, my
dear fellow. And here we live, thank heaven, without
offense. Among these folk, too, there are good men as well as
bad,'' said he, and still speaking, he turned on his knees with a
supple movement, got up, coughed, and went off to another part of
the shed.

``Eh, you rascal!'' Pierre heard the same kind voice saying at
the other end of the shed. ``So you've come, you rascal? She
remembers... Now, now, that'll do!''

And the soldier, pushing away a little dog that was jumping up at
him, returned to his place and sat down. In his hands he had
something wrapped in a rag.

``Here, eat a bit, sir,'' said he, resuming his former respectful
tone as he unwrapped and offered Pierre some baked potatoes. ``We
had soup for dinner and the potatoes are grand!''

Pierre had not eaten all day and the smell of the potatoes seemed
extremely pleasant to him. He thanked the soldier and began to
eat.

``Well, are they all right?'' said the soldier with a
smile. ``You should do like this.''

He took a potato, drew out his clasp knife, cut the potato into
two equal halves on the palm of his hand, sprinkled some salt on
it from the rag, and handed it to Pierre.

``The potatoes are grand!'' he said once more. ``Eat some like
that!''

Pierre thought he had never eaten anything that tasted better.

``Oh, I'm all right,'' said he, ``but why did they shoot those
poor fellows? The last one was hardly twenty.''

``Tss, tt...!'' said the little man. ``Ah, what a sin... what a
sin!'' he added quickly, and as if his words were always waiting
ready in his mouth and flew out involuntarily he went on: ``How
was it, sir, that you stayed in Moscow?''

``I didn't think they would come so soon. I stayed
accidentally,'' replied Pierre.

``And how did they arrest you, dear lad? At your house?''

``No, I went to look at the fire, and they arrested me there, and
tried me as an incendiary.''

``Where there's law there's injustice,'' put in the little man.

``And have you been here long?'' Pierre asked as he munched the
last of the potato.

``I? It was last Sunday they took me, out of a hospital in
Moscow.''

``Why, are you a soldier then?''

``Yes, we are soldiers of the Apsheron regiment. I was dying of
fever. We weren't told anything. There were some twenty of us
lying there. We had no idea, never guessed at all.''

``And do you feel sad here?'' Pierre inquired.

``How can one help it, lad? My name is Platon, and the surname is
Karataev,'' he added, evidently wishing to make it easier for
Pierre to address him. ``They call me 'little falcon' in the
regiment. How is one to help feeling sad? Moscow---she's the
mother of cities. How can one see all this and not feel sad? But
'the maggot gnaws the cabbage, yet dies first'; that's what the
old folks used to tell us,'' he added rapidly.

``What? What did you say?'' asked Pierre.

``Who? I?'' said Karataev. ``I say things happen not as we plan
but as God judges,'' he replied, thinking that he was repeating
what he had said before, and immediately continued:

``Well, and you, have you a family estate, sir? And a house? So
you have abundance, then? And a housewife? And your old parents,
are they still living?'' he asked.

And though it was too dark for Pierre to see, he felt that a
suppressed smile of kindliness puckered the soldier's lips as he
put these questions. He seemed grieved that Pierre had no
parents, especially that he had no mother.

``A wife for counsel, a mother-in-law for welcome, but there's
none as dear as one's own mother!'' said he. ``Well, and have you
little ones?'' he went on asking.

Again Pierre's negative answer seemed to distress him, and he
hastened to add:

``Never mind! You're young folks yet, and please God may still
have some.  The great thing is to live in harmony...''

``But it's all the same now,'' Pierre could not help saying.

``Ah, my dear fellow!'' rejoined Karataev, ``never decline a
prison or a beggar's sack!''

He seated himself more comfortably and coughed, evidently
preparing to tell a long story.

``Well, my dear fellow, I was still living at home,'' he
began. ``We had a well-to-do homestead, plenty of land, we
peasants lived well and our house was one to thank God for. When
Father and we went out mowing there were seven of us. We lived
well. We were real peasants. It so happened...''

And Platon Karataev told a long story of how he had gone into
someone's copse to take wood, how he had been caught by the
keeper, had been tried, flogged, and sent to serve as a soldier.

``Well, lad,'' and a smile changed the tone of his voice ``we
thought it was a misfortune but it turned out a blessing! If it
had not been for my sin, my brother would have had to go as a
soldier. But he, my younger brother, had five little ones, while
I, you see, only left a wife behind. We had a little girl, but
God took her before I went as a soldier. I come home on leave and
I'll tell you how it was, I look and see that they are living
better than before. The yard full of cattle, the women at home,
two brothers away earning wages, and only Michael the youngest,
at home. Father, he says, 'All my children are the same to me: it
hurts the same whichever finger gets bitten. But if Platon hadn't
been shaved for a soldier, Michael would have had to go.' called
us all to him and, will you believe it, placed us in front of the
icons.  'Michael,' he says, 'come here and bow down to his feet;
and you, young woman, you bow down too; and you, grandchildren,
also bow down before him! Do you understand?' he says. That's how
it is, dear fellow. Fate looks for a head. But we are always
judging, 'that's not well---that's not right!' Our luck is like
water in a dragnet: you pull at it and it bulges, but when you've
drawn it out it's empty! That's how it is.''

And Platon shifted his seat on the straw.

After a short silence he rose.

``Well, I think you must be sleepy,'' said he, and began rapidly
crossing himself and repeating:

``Lord Jesus Christ, holy Saint Nicholas, Frola and Lavra! Lord
Jesus Christ, holy Saint Nicholas, Frola and Lavra! Lord Jesus
Christ, have mercy on us and save us!'' he concluded, then bowed
to the ground, got up, sighed, and sat down again on his heap of
straw. ``That's the way.  Lay me down like a stone, O God, and
raise me up like a loaf,'' he muttered as he lay down, pulling
his coat over him.

``What prayer was that you were saying?'' asked Pierre.

``Eh?'' murmured Platon, who had almost fallen asleep. ``What was
I saying?  I was praying. Don't you pray?''

``Yes, I do,'' said Pierre. ``But what was that you said: Frola
and Lavra?''

``Well, of course,'' replied Platon quickly, ``the horses'
saints. One must pity the animals too. Eh, the rascal! Now you've
curled up and got warm, you daughter of a bitch!'' said Karataev,
touching the dog that lay at his feet, and again turning over he
fell asleep immediately.

Sounds of crying and screaming came from somewhere in the
distance outside, and flames were visible through the cracks of
the shed, but inside it was quiet and dark. For a long time
Pierre did not sleep, but lay with eyes open in the darkness,
listening to the regular snoring of Platon who lay beside him,
and he felt that the world that had been shattered was once more
stirring in his soul with a new beauty and on new and unshakable
foundations.

% % % % % % % % % % % % % % % % % % % % % % % % % % % % % % % % %
% % % % % % % % % % % % % % % % % % % % % % % % % % % % % % % % %
% % % % % % % % % % % % % % % % % % % % % % % % % % % % % % % % %
% % % % % % % % % % % % % % % % % % % % % % % % % % % % % % % % %
% % % % % % % % % % % % % % % % % % % % % % % % % % % % % % % % %
% % % % % % % % % % % % % % % % % % % % % % % % % % % % % % % % %
% % % % % % % % % % % % % % % % % % % % % % % % % % % % % % % % %
% % % % % % % % % % % % % % % % % % % % % % % % % % % % % % % % %
% % % % % % % % % % % % % % % % % % % % % % % % % % % % % % % % %
% % % % % % % % % % % % % % % % % % % % % % % % % % % % % % % % %
% % % % % % % % % % % % % % % % % % % % % % % % % % % % % % % % %
% % % % % % % % % % % % % % % % % % % % % % % % % % % % % %

\chapter*{Chapter XIII} \ifaudio \marginpar{
\href{http://ia800203.us.archive.org/13/items/war_and_peace_12_0911_librivox/war_and_peace_12_13_tolstoy_64kb.mp3}{Audio}}
\fi

\initial{T}{wenty-three} soldiers, three officers, and two officials were
confined in the shed in which Pierre had been placed and where he
remained for four weeks.

When Pierre remembered them afterwards they all seemed misty
figures to him except Platon Karataev, who always remained in his
mind a most vivid and precious memory and the personification of
everything Russian, kindly, and round. When Pierre saw his
neighbor next morning at dawn the first impression of him, as of
something round, was fully confirmed: Platon's whole figure---in
a French overcoat girdled with a cord, a soldier's cap, and bast
shoes---was round. His head was quite round, his back, chest,
shoulders, and even his arms, which he held as if ever ready to
embrace something, were rounded, his pleasant smile and his
large, gentle brown eyes were also round.

Platon Karataev must have been fifty, judging by his stories of
campaigns he had been in, told as by an old soldier. He did not
himself know his age and was quite unable to determine it. But
his brilliantly white, strong teeth which showed in two unbroken
semicircles when he laughed---as he often did---were all sound
and good, there was not a gray hair in his beard or on his head,
and his whole body gave an impression of suppleness and
especially of firmness and endurance.

His face, despite its fine, rounded wrinkles, had an expression
of innocence and youth, his voice was pleasant and musical. But
the chief peculiarity of his speech was its directness and
appositeness. It was evident that he never considered what he had
said or was going to say, and consequently the rapidity and
justice of his intonation had an irresistible persuasiveness.

His physical strength and agility during the first days of his
imprisonment were such that he seemed not to know what fatigue
and sickness meant. Every night before lying down, he said:
``Lord, lay me down as a stone and raise me up as a loaf!'' and
every morning on getting up, he said: ``I lay down and curled up,
I get up and shake myself.'' And indeed he only had to lie down,
to fall asleep like a stone, and he only had to shake himself, to
be ready without a moment's delay for some work, just as children
are ready to play directly they awake. He could do everything,
not very well but not badly. He baked, cooked, sewed, planed, and
mended boots. He was always busy, and only at night allowed
himself conversation---of which he was fond---and songs. He did
not sing like a trained singer who knows he is listened to, but
like the birds, evidently giving vent to the sounds in the same
way that one stretches oneself or walks about to get rid of
stiffness, and the sounds were always high-pitched, mournful,
delicate, and almost feminine, and his face at such times was
very serious.

Having been taken prisoner and allowed his beard to grow, he
seemed to have thrown off all that had been forced upon
him---everything military and alien to himself---and had returned
to his former peasant habits.

``A soldier on leave---a shirt outside breeches,'' he would say.

He did not like talking about his life as a soldier, though he
did not complain, and often mentioned that he had not been
flogged once during the whole of his army service. When he
related anything it was generally some old and evidently precious
memory of his \emph{Christian} life, as he called his peasant
existence. The proverbs, of which his talk was full, were for the
most part not the coarse and indecent saws soldiers employ, but
those folk sayings which taken without a context seem so
insignificant, but when used appositely suddenly acquire a
significance of profound wisdom.

He would often say the exact opposite of what he had said on a
previous occasion, yet both would be right. He liked to talk and
he talked well, adorning his speech with terms of endearment and
with folk sayings which Pierre thought he invented himself, but
the chief charm of his talk lay in the fact that the commonest
events---sometimes just such as Pierre had witnessed without
taking notice of them---assumed in Karataev's a character of
solemn fitness. He liked to hear the folk tales one of the
soldiers used to tell of an evening (they were always the same),
but most of all he liked to hear stories of real life. He would
smile joyfully when listening to such stories, now and then
putting in a word or asking a question to make the moral beauty
of what he was told clear to himself. Karataev had no
attachments, friendships, or love, as Pierre understood them, but
loved and lived affectionately with everything life brought him
in contact with, particularly with man---not any particular man,
but those with whom he happened to be. He loved his dog, his
comrades, the French, and Pierre who was his neighbor, but Pierre
felt that in spite of Karataev's affectionate tenderness for him
(by which he unconsciously gave Pierre's spiritual life its due)
he would not have grieved for a moment at parting from him. And
Pierre began to feel in the same way toward Karataev.

To all the other prisoners Platon Karataev seemed a most ordinary
soldier. They called him \emph{little falcon} or \emph{Platosha},
chaffed him good-naturedly, and sent him on errands. But to
Pierre he always remained what he had seemed that first night: an
unfathomable, rounded, eternal personification of the spirit of
simplicity and truth.

Platon Karataev knew nothing by heart except his prayers. When he
began to speak he seemed not to know how he would conclude.

Sometimes Pierre, struck by the meaning of his words, would ask
him to repeat them, but Platon could never recall what he had
said a moment before, just as he never could repeat to Pierre the
words of his favorite song: native and birch tree and my heart is
sick occurred in it, but when spoken and not sung, no meaning
could be got out of it. He did not, and could not, understand the
meaning of words apart from their context. Every word and action
of his was the manifestation of an activity unknown to him, which
was his life. But his life, as he regarded it, had no meaning as
a separate thing. It had meaning only as part of a whole of which
he was always conscious. His words and actions flowed from him as
evenly, inevitably, and spontaneously as fragrance exhales from a
flower. He could not understand the value or significance of any
word or deed taken separately.

% % % % % % % % % % % % % % % % % % % % % % % % % % % % % % % % %
% % % % % % % % % % % % % % % % % % % % % % % % % % % % % % % % %
% % % % % % % % % % % % % % % % % % % % % % % % % % % % % % % % %
% % % % % % % % % % % % % % % % % % % % % % % % % % % % % % % % %
% % % % % % % % % % % % % % % % % % % % % % % % % % % % % % % % %
% % % % % % % % % % % % % % % % % % % % % % % % % % % % % % % % %
% % % % % % % % % % % % % % % % % % % % % % % % % % % % % % % % %
% % % % % % % % % % % % % % % % % % % % % % % % % % % % % % % % %
% % % % % % % % % % % % % % % % % % % % % % % % % % % % % % % % %
% % % % % % % % % % % % % % % % % % % % % % % % % % % % % % % % %
% % % % % % % % % % % % % % % % % % % % % % % % % % % % % % % % %
% % % % % % % % % % % % % % % % % % % % % % % % % % % % % %

\chapter*{Chapter XIV} \ifaudio \marginpar{
\href{http://ia800203.us.archive.org/13/items/war_and_peace_12_0911_librivox/war_and_peace_12_14_tolstoy_64kb.mp3}{Audio}}
\fi

\initial{W}{hen} Princess Mary heard from Nicholas that her brother was with
the Rostovs at Yaroslavl she at once prepared to go there, in
spite of her aunt's efforts to dissuade her---and not merely to
go herself but to take her nephew with her. Whether it were
difficult or easy, possible or impossible, she did not ask and
did not want to know: it was her duty, not only to herself, to be
near her brother who was perhaps dying, but to do everything
possible to take his son to him, and so she prepared to set
off. That she had not heard from Prince Andrew himself, Princess
Mary attributed to his being too weak to write or to his
considering the long journey too hard and too dangerous for her
and his son.

In a few days Princess Mary was ready to start. Her equipages
were the huge family coach in which she had traveled to Voronezh,
a semiopen trap, and a baggage cart. With her traveled
Mademoiselle Bourienne, little Nicholas and his tutor, her old
nurse, three maids, Tikhon, and a young footman and courier her
aunt had sent to accompany her.

The usual route through Moscow could not be thought of, and the
roundabout way Princess Mary was obliged to take through Lipetsk,
Ryazan, Vladimir, and Shuya was very long and, as post horses
were not everywhere obtainable, very difficult, and near Ryazan
where the French were said to have shown themselves was even
dangerous.

During this difficult journey Mademoiselle Bourienne, Dessalles,
and Princess Mary's servants were astonished at her energy and
firmness of spirit. She went to bed later and rose earlier than
any of them, and no difficulties daunted her. Thanks to her
activity and energy, which infected her fellow travelers, they
approached Yaroslavl by the end of the second week.

The last days of her stay in Voronezh had been the happiest of
her life.  Her love for Rostov no longer tormented or agitated
her. It filled her whole soul, had become an integral part of
herself, and she no longer struggled against it. Latterly she had
become convinced that she loved and was beloved, though she never
said this definitely to herself in words. She had become
convinced of it at her last interview with Nicholas, when he had
come to tell her that her brother was with the Rostovs. Not by a
single word had Nicholas alluded to the fact that Prince Andrew's
relations with Natasha might, if he recovered, be renewed, but
Princess Mary saw by his face that he knew and thought of this.

Yet in spite of that, his relation to her---considerate,
delicate, and loving---not only remained unchanged, but it
sometimes seemed to Princess Mary that he was even glad that the
family connection between them allowed him to express his
friendship more freely. She knew that she loved for the first and
only time in her life and felt that she was beloved, and was
happy in regard to it.

But this happiness on one side of her spiritual nature did not
prevent her feeling grief for her brother with full force; on the
contrary, that spiritual tranquility on the one side made it the
more possible for her to give full play to her feeling for her
brother. That feeling was so strong at the moment of leaving
Voronezh that those who saw her off, as they looked at her
careworn, despairing face, felt sure she would fall ill on the
journey. But the very difficulties and preoccupations of the
journey, which she took so actively in hand, saved her for a
while from her grief and gave her strength.

As always happens when traveling, Princess Mary thought only of
the journey itself, forgetting its object. But as she approached
Yaroslavl the thought of what might await her there---not after
many days, but that very evening---again presented itself to her
and her agitation increased to its utmost limit.

The courier who had been sent on in advance to find out where the
Rostovs were staying in Yaroslavl, and in what condition Prince
Andrew was, when he met the big coach just entering the town
gates was appalled by the terrible pallor of the princess' face
that looked out at him from the window.

``I have found out everything, your excellency: the Rostovs are
staying at the merchant Bronnikov's house, in the Square not far
from here, right above the Volga,'' said the courier.

Princess Mary looked at him with frightened inquiry, not
understanding why he did not reply to what she chiefly wanted to
know: how was her brother? Mademoiselle Bourienne put that
question for her.

``How is the prince?'' she asked.

``His excellency is staying in the same house with them.''

``Then he is alive,'' thought Princess Mary, and asked in a low
voice: ``How is he?''

``The servants say he is still the same.''

What \emph{still the same} might mean Princess Mary did not ask,
but with an unnoticed glance at little seven-year-old Nicholas,
who was sitting in front of her looking with pleasure at the
town, she bowed her head and did not raise it again till the
heavy coach, rumbling, shaking and swaying, came to a stop. The
carriage steps clattered as they were let down.

The carriage door was opened. On the left there was water---a
great river---and on the right a porch. There were people at the
entrance: servants, and a rosy girl with a large plait of black
hair, smiling as it seemed to Princess Mary in an unpleasantly
affected way. (This was Sonya.) Princess Mary ran up the
steps. ``This way, this way!'' said the girl, with the same
artificial smile, and the princess found herself in the hall
facing an elderly woman of Oriental type, who came rapidly to
meet her with a look of emotion. This was the countess. She
embraced Princess Mary and kissed her.

``Mon enfant!'' she muttered, ``je vous aime et vous connais
depuis longtemps.''\footnote{``My child! I love you and have
known you a long time.''}

Despite her excitement, Princess Mary realized that this was the
countess and that it was necessary to say something to
her. Hardly knowing how she did it, she contrived to utter a few
polite phrases in French in the same tone as those that had been
addressed to her, and asked: ``How is he?''

``The doctor says that he is not in danger,'' said the countess,
but as she spoke she raised her eyes with a sigh, and her gesture
conveyed a contradiction of her words.

``Where is he? Can I see him---can I?'' asked the princess.

``One moment, Princess, one moment, my dear! Is this his son?''
said the countess, turning to little Nicholas who was coming in
with Dessalles.  ``There will be room for everybody, this is a
big house. Oh, what a lovely boy!''

The countess took Princess Mary into the drawing room, where
Sonya was talking to Mademoiselle Bourienne. The countess
caressed the boy, and the old count came in and welcomed the
princess. He had changed very much since Princess Mary had last
seen him. Then he had been a brisk, cheerful, self-assured old
man; now he seemed a pitiful, bewildered person. While talking to
Princess Mary he continually looked round as if asking everyone
whether he was doing the right thing. After the destruction of
Moscow and of his property, thrown out of his accustomed groove
he seemed to have lost the sense of his own significance and to
feel that there was no longer a place for him in life.

In spite of her one desire to see her brother as soon as
possible, and her vexation that at the moment when all she wanted
was to see him they should be trying to entertain her and
pretending to admire her nephew, the princess noticed all that
was going on around her and felt the necessity of submitting, for
a time, to this new order of things which she had entered. She
knew it to be necessary, and though it was hard for her she was
not vexed with these people.

``This is my niece,'' said the count, introducing Sonya---``You
don't know her, Princess?''

Princess Mary turned to Sonya and, trying to stifle the hostile
feeling that arose in her toward the girl, she kissed her. But
she felt oppressed by the fact that the mood of everyone around
her was so far from what was in her own heart.

``Where is he?'' she asked again, addressing them all.

``He is downstairs. Natasha is with him,'' answered Sonya,
flushing. ``We have sent to ask. I think you must be tired,
Princess.''

Tears of vexation showed themselves in Princess Mary's eyes. She
turned away and was about to ask the countess again how to go to
him, when light, impetuous, and seemingly buoyant steps were
heard at the door.  The princess looked round and saw Natasha
coming in, almost running---that Natasha whom she had liked so
little at their meeting in Moscow long since.

But hardly had the princess looked at Natasha's face before she
realized that here was a real comrade in her grief, and
consequently a friend.  She ran to meet her, embraced her, and
began to cry on her shoulder.

As soon as Natasha, sitting at the head of Prince Andrew's bed,
heard of Princess Mary's arrival, she softly left his room and
hastened to her with those swift steps that had sounded buoyant
to Princess Mary.

There was only one expression on her agitated face when she ran
into the drawing room---that of love---boundless love for him,
for her, and for all that was near to the man she loved; and of
pity, suffering for others, and passionate desire to give herself
entirely to helping them. It was plain that at that moment there
was in Natasha's heart no thought of herself or of her own
relations with Prince Andrew.

Princess Mary, with her acute sensibility, understood all this at
the first glance at Natasha's face, and wept on her shoulder with
sorrowful pleasure.

``Come, come to him, Mary,'' said Natasha, leading her into the
other room.

Princess Mary raised her head, dried her eyes, and turned to
Natasha.  She felt that from her she would be able to understand
and learn everything.

``How...'' she began her question but stopped short.

She felt that it was impossible to ask, or to answer, in words.
Natasha's face and eyes would have to tell her all more clearly
and profoundly.

Natasha was gazing at her, but seemed afraid and in doubt whether
to say all she knew or not; she seemed to feel that before those
luminous eyes which penetrated into the very depths of her heart,
it was impossible not to tell the whole truth which she saw. And
suddenly, Natasha's lips twitched, ugly wrinkles gathered round
her mouth, and covering her face with her hands she burst into
sobs.

Princess Mary understood.

But she still hoped, and asked, in words she herself did not
trust:

``But how is his wound? What is his general condition?''

``You, you... will see,'' was all Natasha could say.

They sat a little while downstairs near his room till they had
left off crying and were able to go to him with calm faces.

``How has his whole illness gone? Is it long since he grew worse?
When did this happen?'' Princess Mary inquired.

Natasha told her that at first there had been danger from his
feverish condition and the pain he suffered, but at Troitsa that
had passed and the doctor had only been afraid of gangrene. That
danger had also passed. When they reached Yaroslavl the wound had
begun to fester (Natasha knew all about such things as festering)
and the doctor had said that the festering might take a normal
course. Then fever set in, but the doctor had said the fever was
not very serious.

``But two days ago this suddenly happened,'' said Natasha,
struggling with her sobs. ``I don't know why, but you will see
what he is like.''

``Is he weaker? Thinner?'' asked the princess.

``No, it's not that, but worse. You will see. O, Mary, he is too
good, he cannot, cannot live, because...''

% % % % % % % % % % % % % % % % % % % % % % % % % % % % % % % % %
% % % % % % % % % % % % % % % % % % % % % % % % % % % % % % % % %
% % % % % % % % % % % % % % % % % % % % % % % % % % % % % % % % %
% % % % % % % % % % % % % % % % % % % % % % % % % % % % % % % % %
% % % % % % % % % % % % % % % % % % % % % % % % % % % % % % % % %
% % % % % % % % % % % % % % % % % % % % % % % % % % % % % % % % %
% % % % % % % % % % % % % % % % % % % % % % % % % % % % % % % % %
% % % % % % % % % % % % % % % % % % % % % % % % % % % % % % % % %
% % % % % % % % % % % % % % % % % % % % % % % % % % % % % % % % %
% % % % % % % % % % % % % % % % % % % % % % % % % % % % % % % % %
% % % % % % % % % % % % % % % % % % % % % % % % % % % % % % % % %
% % % % % % % % % % % % % % % % % % % % % % % % % % % % % %

\chapter*{Chapter XV} \ifaudio \marginpar{
\href{http://ia800203.us.archive.org/13/items/war_and_peace_12_0911_librivox/war_and_peace_12_15_tolstoy_64kb.mp3}{Audio}}
\fi

\initial{W}{hen} Natasha opened Prince Andrew's door with a familiar movement
and let Princess Mary pass into the room before her, the princess
felt the sobs in her throat. Hard as she had tried to prepare
herself, and now tried to remain tranquil, she knew that she
would be unable to look at him without tears.

The princess understood what Natasha had meant by the words:
\emph{two days ago this suddenly happened}. She understood those
words to mean that he had suddenly softened and that this
softening and gentleness were signs of approaching death. As she
stepped to the door she already saw in imagination Andrew's face
as she remembered it in childhood, a gentle, mild, sympathetic
face which he had rarely shown, and which therefore affected her
very strongly. She was sure he would speak soft, tender words to
her such as her father had uttered before his death, and that she
would not be able to bear it and would burst into sobs in his
presence. Yet sooner or later it had to be, and she went in. The
sobs rose higher and higher in her throat as she more and more
clearly distinguished his form and her shortsighted eyes tried to
make out his features, and then she saw his face and met his
gaze.

He was lying in a squirrel-fur dressing gown on a divan,
surrounded by pillows. He was thin and pale. In one thin,
translucently white hand he held a handkerchief, while with the
other he stroked the delicate mustache he had grown, moving his
fingers slowly. His eyes gazed at them as they entered.

On seeing his face and meeting his eyes Princess Mary's pace
suddenly slackened, she felt her tears dry up and her sobs
ceased. She suddenly felt guilty and grew timid on catching the
expression of his face and eyes.

``But in what am I to blame?'' she asked herself. And his cold,
stern look replied: ``Because you are alive and thinking of the
living, while I...''

In the deep gaze that seemed to look not outwards but inwards
there was an almost hostile expression as he slowly regarded his
sister and Natasha.

He kissed his sister, holding her hand in his as was their wont.

``How are you, Mary? How did you manage to get here?'' said he in
a voice as calm and aloof as his look.

Had he screamed in agony, that scream would not have struck such
horror into Princess Mary's heart as the tone of his voice.

``And have you brought little Nicholas?'' he asked in the same
slow, quiet manner and with an obvious effort to remember.

``How are you now?'' said Princess Mary, herself surprised at
what she was saying.

``That, my dear, you must ask the doctor,'' he replied, and again
making an evident effort to be affectionate, he said with his
lips only (his words clearly did not correspond to his thoughts):

``Merci, chere amie, d'etre venue.''\footnote{``Thank you for
  coming, my dear.''}

Princess Mary pressed his hand. The pressure made him wince just
perceptibly. He was silent, and she did not know what to say. She
now understood what had happened to him two days before. In his
words, his tone, and especially in that calm, almost antagonistic
look could be felt an estrangement from everything belonging to
this world, terrible in one who is alive. Evidently only with an
effort did he understand anything living; but it was obvious that
he failed to understand, not because he lacked the power to do so
but because he understood something else---something the living
did not and could not understand---and which wholly occupied his
mind.

``There, you see how strangely fate has brought us together,''
said he, breaking the silence and pointing to Natasha. ``She
looks after me all the time.''

Princess Mary heard him and did not understand how he could say
such a thing. He, the sensitive, tender Prince Andrew, how could
he say that, before her whom he loved and who loved him? Had he
expected to live he could not have said those words in that
offensively cold tone. If he had not known that he was dying, how
could he have failed to pity her and how could he speak like that
in her presence? The only explanation was that he was
indifferent, because something else, much more important, had
been revealed to him.

The conversation was cold and disconnected and continually broke
off.

``Mary came by way of Ryazan,'' said Natasha.

Prince Andrew did not notice that she called his sister Mary, and
only after calling her so in his presence did Natasha notice it
herself.

``Really?'' he asked.

``They told her that all Moscow has been burned down, and
that...''

Natasha stopped. It was impossible to talk. It was plain that he
was making an effort to listen, but could not do so.

``Yes, they say it's burned,'' he said. ``It's a great pity,''
and he gazed straight before him, absently stroking his mustache
with his fingers.

``And so you have met Count Nicholas, Mary?'' Prince Andrew
suddenly said, evidently wishing to speak pleasantly to
them. ``He wrote here that he took a great liking to you,'' he
went on simply and calmly, evidently unable to understand all the
complex significance his words had for living people. ``If you
liked him too, it would be a good thing for you to get married,''
he added rather more quickly, as if pleased at having found words
he had long been seeking.

Princess Mary heard his words but they had no meaning for her,
except as a proof of how far away he now was from everything
living.

``Why talk of me?'' she said quietly and glanced at Natasha.

Natasha, who felt her glance, did not look at her. All three were
again silent.

``Andrew, would you like...'' Princess Mary suddenly said in a
trembling voice, ``would you like to see little Nicholas? He is
always talking about you!''

Prince Andrew smiled just perceptibly and for the first time, but
Princess Mary, who knew his face so well, saw with horror that he
did not smile with pleasure or affection for his son, but with
quiet, gentle irony because he thought she was trying what she
believed to be the last means of arousing him.

``Yes, I shall be very glad to see him. Is he quite well?''

When little Nicholas was brought into Prince Andrew's room he
looked at his father with frightened eyes, but did not cry,
because no one else was crying. Prince Andrew kissed him and
evidently did not know what to say to him.

When Nicholas had been led away, Princess Mary again went up to
her brother, kissed him, and unable to restrain her tears any
longer began to cry.

He looked at her attentively.

``Is it about Nicholas?'' he asked.

Princess Mary nodded her head, weeping.

``Mary, you know the Gosp...'' but he broke off.

``What did you say?''

``Nothing. You mustn't cry here,'' he said, looking at her with
the same cold expression.

When Princess Mary began to cry, he understood that she was
crying at the thought that little Nicholas would be left without
a father. With a great effort he tried to return to life and to
see things from their point of view.

``Yes, to them it must seem sad!'' he thought. ``But how simple
it is.''

``The fowls of the air sow not, neither do they reap, yet your
Father feedeth them,'' he said to himself and wished to say to
Princess Mary; ``but no, they will take it their own way, they
won't understand! They can't understand that all those feelings
they prize so---all our feelings, all those ideas that seem so
important to us, are unnecessary.  We cannot understand one
another,'' and he remained silent.

Prince Andrew's little son was seven. He could scarcely read, and
knew nothing. After that day he lived through many things,
gaining knowledge, observation, and experience, but had he
possessed all the faculties he afterwards acquired, he could not
have had a better or more profound understanding of the meaning
of the scene he had witnessed between his father, Mary, and
Natasha, than he had then. He understood it completely, and,
leaving the room without crying, went silently up to Natasha who
had come out with him and looked shyly at her with his beautiful,
thoughtful eyes, then his uplifted, rosy upper lip trembled and
leaning his head against her he began to cry.

After that he avoided Dessalles and the countess who caressed him
and either sat alone or came timidly to Princess Mary, or to
Natasha of whom he seemed even fonder than of his aunt, and clung
to them quietly and shyly.

When Princess Mary had left Prince Andrew she fully understood
what Natasha's face had told her. She did not speak any more to
Natasha of hopes of saving his life. She took turns with her
beside his sofa, and did not cry any more, but prayed
continually, turning in soul to that Eternal and Unfathomable,
whose presence above the dying man was now so evident.

% % % % % % % % % % % % % % % % % % % % % % % % % % % % % % % % %
% % % % % % % % % % % % % % % % % % % % % % % % % % % % % % % % %
% % % % % % % % % % % % % % % % % % % % % % % % % % % % % % % % %
% % % % % % % % % % % % % % % % % % % % % % % % % % % % % % % % %
% % % % % % % % % % % % % % % % % % % % % % % % % % % % % % % % %
% % % % % % % % % % % % % % % % % % % % % % % % % % % % % % % % %
% % % % % % % % % % % % % % % % % % % % % % % % % % % % % % % % %
% % % % % % % % % % % % % % % % % % % % % % % % % % % % % % % % %
% % % % % % % % % % % % % % % % % % % % % % % % % % % % % % % % %
% % % % % % % % % % % % % % % % % % % % % % % % % % % % % % % % %
% % % % % % % % % % % % % % % % % % % % % % % % % % % % % % % % %
% % % % % % % % % % % % % % % % % % % % % % % % % % % % % %

 
\chapter*{Chapter XVI} \ifaudio \marginpar{
\href{http://ia800203.us.archive.org/13/items/war_and_peace_12_0911_librivox/war_and_peace_12_16_tolstoy_64kb.mp3}{Audio}}
\fi

\initial{N}{ot} only did Prince Andrew know he would die, but he felt that he
was dying and was already half dead. He was conscious of an
aloofness from everything earthly and a strange and joyous
lightness of existence.  Without haste or agitation he awaited
what was coming. That inexorable, eternal, distant, and unknown
the presence of which he had felt continually all his life---was
now near to him and, by the strange lightness he experienced,
almost comprehensible and palpable...

Formerly he had feared the end. He had twice experienced that
terribly tormenting fear of death---the end---but now he no
longer understood that fear.

He had felt it for the first time when the shell spun like a top
before him, and he looked at the fallow field, the bushes, and
the sky, and knew that he was face to face with death. When he
came to himself after being wounded and the flower of eternal,
unfettered love had instantly unfolded itself in his soul as if
freed from the bondage of life that had restrained it, he no
longer feared death and ceased to think about it.

During the hours of solitude, suffering, and partial delirium he
spent after he was wounded, the more deeply he penetrated into
the new principle of eternal love revealed to him, the more he
unconsciously detached himself from earthly life. To love
everything and everybody and always to sacrifice oneself for love
meant not to love anyone, not to live this earthly life. And the
more imbued he became with that principle of love, the more he
renounced life and the more completely he destroyed that dreadful
barrier which---in the absence of such love---stands between life
and death. When during those first days he remembered that he
would have to die, he said to himself: ``Well, what of it? So
much the better!''

But after the night in Mytishchi when, half delirious, he had
seen her for whom he longed appear before him and, having pressed
her hand to his lips, had shed gentle, happy tears, love for a
particular woman again crept unobserved into his heart and once
more bound him to life. And joyful and agitating thoughts began
to occupy his mind. Recalling the moment at the ambulance station
when he had seen Kuragin, he could not now regain the feeling he
then had, but was tormented by the question whether Kuragin was
alive. And he dared not inquire.

His illness pursued its normal physical course, but what Natasha
referred to when she said: ``This suddenly happened,'' had
occurred two days before Princess Mary arrived. It was the last
spiritual struggle between life and death, in which death gained
the victory. It was the unexpected realization of the fact that
he still valued life as presented to him in the form of his love
for Natasha, and a last, though ultimately vanquished, attack of
terror before the unknown.

It was evening. As usual after dinner he was slightly feverish,
and his thoughts were preternaturally clear. Sonya was sitting by
the table. He began to doze. Suddenly a feeling of happiness
seized him.

``Ah, she has come!'' thought he.

And so it was: in Sonya's place sat Natasha who had just come in
noiselessly.

Since she had begun looking after him, he had always experienced
this physical consciousness of her nearness. She was sitting in
an armchair placed sideways, screening the light of the candle
from him, and was knitting a stocking. She had learned to knit
stockings since Prince Andrew had casually mentioned that no one
nursed the sick so well as old nurses who knit stockings, and
that there is something soothing in the knitting of
stockings. The needles clicked lightly in her slender, rapidly
moving hands, and he could clearly see the thoughtful profile of
her drooping face. She moved, and the ball rolled off her
knees. She started, glanced round at him, and screening the
candle with her hand stooped carefully with a supple and exact
movement, picked up the ball, and regained her former position.

He looked at her without moving and saw that she wanted to draw a
deep breath after stooping, but refrained from doing so and
breathed cautiously.

At the Troitsa monastery they had spoken of the past, and he had
told her that if he lived he would always thank God for his wound
which had brought them together again, but after that they never
spoke of the future.

``Can it or can it not be?'' he now thought as he looked at her
and listened to the light click of the steel needles. ``Can fate
have brought me to her so strangely only for me to die?... Is it
possible that the truth of life has been revealed to me only to
show me that I have spent my life in falsity? I love her more
than anything in the world! But what am I to do if I love her?''
he thought, and he involuntarily groaned, from a habit acquired
during his sufferings.

On hearing that sound Natasha put down the stocking, leaned
nearer to him, and suddenly, noticing his shining eyes, stepped
lightly up to him and bent over him.

``You are not asleep?''

``No, I have been looking at you a long time. I felt you come
in. No one else gives me that sense of soft tranquillity that you
do... that light.  I want to weep for joy.''

Natasha drew closer to him. Her face shone with rapturous joy.

``Natasha, I love you too much! More than anything in the
world.''

``And I!''---She turned away for an instant. ``Why too much?''
she asked.

``Why too much?... Well, what do you, what do you feel in your
soul, your whole soul---shall I live? What do you think?''

``I am sure of it, sure!'' Natasha almost shouted, taking hold of
both his hands with a passionate movement.

He remained silent awhile.

``How good it would be!'' and taking her hand he kissed it.

Natasha felt happy and agitated, but at once remembered that this
would not do and that he had to be quiet.

``But you have not slept,'' she said, repressing her joy. ``Try
to sleep...  please!''

He pressed her hand and released it, and she went back to the
candle and sat down again in her former position. Twice she
turned and looked at him, and her eyes met his beaming at
her. She set herself a task on her stocking and resolved not to
turn round till it was finished.

Soon he really shut his eyes and fell asleep. He did not sleep
long and suddenly awoke with a start and in a cold perspiration.

As he fell asleep he had still been thinking of the subject that
now always occupied his mind---about life and death, and chiefly
about death.  He felt himself nearer to it.

``Love? What is love?'' he thought.

``Love hinders death. Love is life. All, everything that I
understand, I understand only because I love. Everything is,
everything exists, only because I love. Everything is united by
it alone. Love is God, and to die means that I, a particle of
love, shall return to the general and eternal source.'' These
thoughts seemed to him comforting. But they were only
thoughts. Something was lacking in them, they were not clear,
they were too one-sidedly personal and brain-spun. And there was
the former agitation and obscurity. He fell asleep.

He dreamed that he was lying in the room he really was in, but
that he was quite well and unwounded. Many various, indifferent,
and insignificant people appeared before him. He talked to them
and discussed something trivial. They were preparing to go away
somewhere.  Prince Andrew dimly realized that all this was
trivial and that he had more important cares, but he continued to
speak, surprising them by empty witticisms. Gradually, unnoticed,
all these persons began to disappear and a single question, that
of the closed door, superseded all else. He rose and went to the
door to bolt and lock it. Everything depended on whether he was,
or was not, in time to lock it. He went, and tried to hurry, but
his legs refused to move and he knew he would not be in time to
lock the door though he painfully strained all his powers. He was
seized by an agonizing fear. And that fear was the fear of
death. It stood behind the door. But just when he was clumsily
creeping toward the door, that dreadful something on the other
side was already pressing against it and forcing its way
in. Something not human---death---was breaking in through that
door, and had to be kept out. He seized the door, making a final
effort to hold it back---to lock it was no longer possible---but
his efforts were weak and clumsy and the door, pushed from behind
by that terror, opened and closed again.

Once again it pushed from outside. His last superhuman efforts
were vain and both halves of the door noiselessly opened. It
entered, and it was death, and Prince Andrew died.

But at the instant he died, Prince Andrew remembered that he was
asleep, and at the very instant he died, having made an effort,
he awoke.

``Yes, it was death! I died---and woke up. Yes, death is an
awakening!''  And all at once it grew light in his soul and the
veil that had till then concealed the unknown was lifted from his
spiritual vision. He felt as if powers till then confined within
him had been liberated, and that strange lightness did not again
leave him.

When, waking in a cold perspiration, he moved on the divan,
Natasha went up and asked him what was the matter. He did not
answer and looked at her strangely, not understanding.

That was what had happened to him two days before Princess Mary's
arrival. From that day, as the doctor expressed it, the wasting
fever assumed a malignant character, but what the doctor said did
not interest Natasha, she saw the terrible moral symptoms which
to her were more convincing.

From that day an awakening from life came to Prince Andrew
together with his awakening from sleep. And compared to the
duration of life it did not seem to him slower than an awakening
from sleep compared to the duration of a dream.

There was nothing terrible or violent in this comparatively slow
awakening.

His last days and hours passed in an ordinary and simple
way. Both Princess Mary and Natasha, who did not leave him, felt
this. They did not weep or shudder and during these last days
they themselves felt that they were not attending on him (he was
no longer there, he had left them) but on what reminded them most
closely of him---his body. Both felt this so strongly that the
outward and terrible side of death did not affect them and they
did not feel it necessary to foment their grief.  Neither in his
presence nor out of it did they weep, nor did they ever talk to
one another about him. They felt that they could not express in
words what they understood.

They both saw that he was sinking slowly and quietly, deeper and
deeper, away from them, and they both knew that this had to be so
and that it was right.

He confessed, and received communion: everyone came to take leave
of him. When they brought his son to him, he pressed his lips to
the boy's and turned away, not because he felt it hard and sad
(Princess Mary and Natasha understood that) but simply because he
thought it was all that was required of him, but when they told
him to bless the boy, he did what was demanded and looked round
as if asking whether there was anything else he should do.

When the last convulsions of the body, which the spirit was
leaving, occurred, Princess Mary and Natasha were present.

``Is it over?'' said Princess Mary when his body had for a few
minutes lain motionless, growing cold before them. Natasha went
up, looked at the dead eyes, and hastened to close them. She
closed them but did not kiss them, but clung to that which
reminded her most nearly of him---his body.

``Where has he gone? Where is he now?...''

When the body, washed and dressed, lay in the coffin on a table,
everyone came to take leave of him and they all wept.

Little Nicholas cried because his heart was rent by painful
perplexity.  The countess and Sonya cried from pity for Natasha
and because he was no more. The old count cried because he felt
that before long, he, too, must take the same terrible step.

Natasha and Princess Mary also wept now, but not because of their
own personal grief; they wept with a reverent and softening
emotion which had taken possession of their souls at the
consciousness of the simple and solemn mystery of death that had
been accomplished in their presence.