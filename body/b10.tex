\part*{Book Ten: 1812}

% % % % % % % % % % % % % % % % % % % % % % % % % % % % % % % % %
% % % % % % % % % % % % % % % % % % % % % % % % % % % % % % % % %
% % % % % % % % % % % % % % % % % % % % % % % % % % % % % % % % %
% % % % % % % % % % % % % % % % % % % % % % % % % % % % % % % % %
% % % % % % % % % % % % % % % % % % % % % % % % % % % % % % % % %
% % % % % % % % % % % % % % % % % % % % % % % % % % % % % % % % %
% % % % % % % % % % % % % % % % % % % % % % % % % % % % % % % % %
% % % % % % % % % % % % % % % % % % % % % % % % % % % % % % % % %
% % % % % % % % % % % % % % % % % % % % % % % % % % % % % % % % %
% % % % % % % % % % % % % % % % % % % % % % % % % % % % % % % % %
% % % % % % % % % % % % % % % % % % % % % % % % % % % % % % % % %
% % % % % % % % % % % % % % % % % % % % % % % % % % % % % %

\chapter*{Chapter I} \ifaudio \marginpar{
\href{http://ia801407.us.archive.org/32/items/war_and_peace_10_0904_librivox/war_and_peace_10_01_tolstoy_64kb.mp3}{Audio}}
\fi

\initial{N}{apoleon} began the war with Russia because he could not resist
going to Dresden, could not help having his head turned by the
homage he received, could not help donning a Polish uniform and
yielding to the stimulating influence of a June morning, and
could not refrain from bursts of anger in the presence of Kurakin
and then of Balashev.

Alexander refused negotiations because he felt himself to be
personally insulted. Barclay de Tolly tried to command the army
in the best way, because he wished to fulfill his duty and earn
fame as a great commander. Rostov charged the French because he
could not restrain his wish for a gallop across a level field;
and in the same way the innumerable people who took part in the
war acted in accord with their personal characteristics, habits,
circumstances, and aims. They were moved by fear or vanity,
rejoiced or were indignant, reasoned, imagining that they knew
what they were doing and did it of their own free will, but they
all were involuntary tools of history, carrying on a work
concealed from them but comprehensible to us. Such is the
inevitable fate of men of action, and the higher they stand in
the social hierarchy the less are they free.

The actors of 1812 have long since left the stage, their personal
interests have vanished leaving no trace, and nothing remains of
that time but its historic results.

Providence compelled all these men, striving to attain personal
aims, to further the accomplishment of a stupendous result no one
of them at all expected---neither Napoleon, nor Alexander, nor
still less any of those who did the actual fighting.

The cause of the destruction of the French army in 1812 is clear
to us now. No one will deny that that cause was, on the one hand,
its advance into the heart of Russia late in the season without
any preparation for a winter campaign and, on the other, the
character given to the war by the burning of Russian towns and
the hatred of the foe this aroused among the Russian people. But
no one at the time foresaw (what now seems so evident) that this
was the only way an army of eight hundred thousand men---the best
in the world and led by the best general---could be destroyed in
conflict with a raw army of half its numerical strength, and led
by inexperienced commanders as the Russian army was. Not only did
no one see this, but on the Russian side every effort was made to
hinder the only thing that could save Russia, while on the French
side, despite Napoleon's experience and so-called military
genius, every effort was directed to pushing on to Moscow at the
end of the summer, that is, to doing the very thing that was
bound to lead to destruction.

In historical works on the year 1812 French writers are very fond
of saying that Napoleon felt the danger of extending his line,
that he sought a battle and that his marshals advised him to stop
at Smolensk, and of making similar statements to show that the
danger of the campaign was even then understood. Russian authors
are still fonder of telling us that from the commencement of the
campaign a Scythian war plan was adopted to lure Napoleon into
the depths of Russia, and this plan some of them attribute to
Pfuel, others to a certain Frenchman, others to Toll, and others
again to Alexander himself---pointing to notes, projects, and
letters which contain hints of such a line of action. But all
these hints at what happened, both from the French side and the
Russian, are advanced only because they fit in with the
event. Had that event not occurred these hints would have been
forgotten, as we have forgotten the thousands and millions of
hints and expectations to the contrary which were current then
but have now been forgotten because the event falsified
them. There are always so many conjectures as to the issue of any
event that however it may end there will always be people to say:
``I said then that it would be so,'' quite forgetting that amid
their innumerable conjectures many were to quite the contrary
effect.

Conjectures as to Napoleon's awareness of the danger of extending
his line, and (on the Russian side) as to luring the enemy into
the depths of Russia, are evidently of that kind, and only by
much straining can historians attribute such conceptions to
Napoleon and his marshals, or such plans to the Russian
commanders. All the facts are in flat contradiction to such
conjectures. During the whole period of the war not only was
there no wish on the Russian side to draw the French into the
heart of the country, but from their first entry into Russia
everything was done to stop them. And not only was Napoleon not
afraid to extend his line, but he welcomed every step forward as
a triumph and did not seek battle as eagerly as in former
campaigns, but very lazily.

At the very beginning of the war our armies were divided, and our
sole aim was to unite them, though uniting the armies was no
advantage if we meant to retire and lure the enemy into the
depths of the country. Our Emperor joined the army to encourage
it to defend every inch of Russian soil and not to retreat. The
enormous Drissa camp was formed on Pfuel's plan, and there was no
intention of retiring farther. The Emperor reproached the
commanders in chief for every step they retired. He could not
bear the idea of letting the enemy even reach Smolensk, still
less could he contemplate the burning of Moscow, and when our
armies did unite he was displeased that Smolensk was abandoned
and burned without a general engagement having been fought under
its walls.

So thought the Emperor, and the Russian commanders and people
were still more provoked at the thought that our forces were
retreating into the depths of the country.

Napoleon having cut our armies apart advanced far into the
country and missed several chances of forcing an engagement. In
August he was at Smolensk and thought only of how to advance
farther, though as we now see that advance was evidently ruinous
to him.

The facts clearly show that Napoleon did not foresee the danger
of the advance on Moscow, nor did Alexander and the Russian
commanders then think of luring Napoleon on, but quite the
contrary. The luring of Napoleon into the depths of the country
was not the result of any plan, for no one believed it to be
possible; it resulted from a most complex interplay of intrigues,
aims, and wishes among those who took part in the war and had no
perception whatever of the inevitable, or of the one way of
saving Russia. Everything came about fortuitously. The armies
were divided at the commencement of the campaign. We tried to
unite them, with the evident intention of giving battle and
checking the enemy's advance, and by this effort to unite them
while avoiding battle with a much stronger enemy, and necessarily
withdrawing the armies at an acute angle---we led the French on
to Smolensk. But we withdrew at an acute angle not only because
the French advanced between our two armies; the angle became
still more acute and we withdrew still farther, because Barclay
de Tolly was an unpopular foreigner disliked by Bagration (who
would come under his command), and Bagration---being in command
of the second army---tried to postpone joining up and coming
under Barclay's command as long as he could. Bagration was slow
in effecting the junction---though that was the chief aim of all
at headquarters---because, as he alleged, he exposed his army to
danger on this march, and it was best for him to retire more to
the left and more to the south, worrying the enemy from flank and
rear and securing from the Ukraine recruits for his army; and it
looks as if he planned this in order not to come under the
command of the detested foreigner Barclay, whose rank was
inferior to his own.

The Emperor was with the army to encourage it, but his presence
and ignorance of what steps to take, and the enormous number of
advisers and plans, destroyed the first army's energy and it
retired.

The intention was to make a stand at the Drissa camp, but
Paulucci, aiming at becoming commander-in-chief, unexpectedly
employed his energy to influence Alexander, and Pfuel's whole
plan was abandoned and the command entrusted to Barclay. But as
Barclay did not inspire confidence his power was limited. The
armies were divided, there was no unity of command, and Barclay
was unpopular; but from this confusion, division, and the
unpopularity of the foreign commander-in-chief, there resulted on
the one hand indecision and the avoidance of a battle (which we
could not have refrained from had the armies been united and had
someone else, instead of Barclay, been in command) and on the
other an ever-increasing indignation against the foreigners and
an increase in patriotic zeal.

At last the Emperor left the army, and as the most convenient and
indeed the only pretext for his departure it was decided that it
was necessary for him to inspire the people in the capitals and
arouse the nation in general to a patriotic war. And by this
visit of the Emperor to Moscow the strength of the Russian army
was trebled.

He left in order not to obstruct the commander-in-chief's
undivided control of the army, and hoping that more decisive
action would then be taken, but the command of the armies became
still more confused and enfeebled. Bennigsen, the Tsarevich, and
a swarm of adjutants general remained with the army to keep the
commander-in-chief under observation and arouse his energy, and
Barclay, feeling less free than ever under the observation of all
these \emph{eyes of the Emperor}, became still more cautious of
undertaking any decisive action and avoided giving battle.

Barclay stood for caution. The Tsarevich hinted at treachery and
demanded a general engagement. Lubomirski, Bronnitski, Wlocki,
and the others of that group stirred up so much trouble that
Barclay, under pretext of sending papers to the Emperor,
dispatched these Polish adjutants general to Petersburg and
plunged into an open struggle with Bennigsen and the Tsarevich.

At Smolensk the armies at last reunited, much as Bagration
disliked it.

Bagration drove up in a carriage to the house occupied by
Barclay.  Barclay donned his sash and came out to meet and report
to his senior officer Bagration.

Despite his seniority in rank Bagration, in this contest of
magnanimity, took his orders from Barclay, but, having submitted,
agreed with him less than ever. By the Emperor's orders Bagration
reported direct to him. He wrote to Arakcheev, the Emperor's
confidant: ``It must be as my sovereign pleases, but I cannot
work with the Minister (meaning Barclay). For God's sake send me
somewhere else if only in command of a regiment. I cannot stand
it here. Headquarters are so full of Germans that a Russian
cannot exist and there is no sense in anything. I thought I was
really serving my sovereign and the Fatherland, but it turns out
that I am serving Barclay. I confess I do not want to.''

The swarm of Bronnitskis and Wintzingerodes and their like still
further embittered the relations between the commanders in chief,
and even less unity resulted. Preparations were made to fight the
French before Smolensk. A general was sent to survey the
position. This general, hating Barclay, rode to visit a friend of
his own, a corps commander, and, having spent the day with him,
returned to Barclay and condemned, as unsuitable from every point
of view, the battleground he had not seen.

While disputes and intrigues were going on about the future field
of battle, and while we were looking for the French---having lost
touch with them---the French stumbled upon Neverovski's division
and reached the walls of Smolensk.

It was necessary to fight an unexpected battle at Smolensk to
save our lines of communication. The battle was fought and
thousands were killed on both sides.

Smolensk was abandoned contrary to the wishes of the Emperor and
of the whole people. But Smolensk was burned by its own
inhabitants-who had been misled by their governor. And these
ruined inhabitants, setting an example to other Russians, went to
Moscow thinking only of their own losses but kindling hatred of
the foe. Napoleon advanced farther and we retired, thus arriving
at the very result which caused his destruction.

% % % % % % % % % % % % % % % % % % % % % % % % % % % % % % % % %
% % % % % % % % % % % % % % % % % % % % % % % % % % % % % % % % %
% % % % % % % % % % % % % % % % % % % % % % % % % % % % % % % % %
% % % % % % % % % % % % % % % % % % % % % % % % % % % % % % % % %
% % % % % % % % % % % % % % % % % % % % % % % % % % % % % % % % %
% % % % % % % % % % % % % % % % % % % % % % % % % % % % % % % % %
% % % % % % % % % % % % % % % % % % % % % % % % % % % % % % % % %
% % % % % % % % % % % % % % % % % % % % % % % % % % % % % % % % %
% % % % % % % % % % % % % % % % % % % % % % % % % % % % % % % % %
% % % % % % % % % % % % % % % % % % % % % % % % % % % % % % % % %
% % % % % % % % % % % % % % % % % % % % % % % % % % % % % % % % %
% % % % % % % % % % % % % % % % % % % % % % % % % % % % % %

\chapter*{Chapter II} \ifaudio \marginpar{
\href{http://ia801407.us.archive.org/32/items/war_and_peace_10_0904_librivox/war_and_peace_10_02_tolstoy_64kb.mp3}{Audio}}
\fi

\initial{T}{he} day after his son had left, Prince Nicholas sent for Princess
Mary to come to his study.

``Well? Are you satisfied now?'' said he. ``You've made me
quarrel with my son! Satisfied, are you? That's all you wanted!
Satisfied?... It hurts me, it hurts. I'm old and weak and this is
what you wanted. Well then, gloat over it! Gloat over it!''

After that Princess Mary did not see her father for a whole
week. He was ill and did not leave his study.

Princess Mary noticed to her surprise that during this illness
the old prince not only excluded her from his room, but did not
admit Mademoiselle Bourienne either. Tikhon alone attended him.

At the end of the week the prince reappeared and resumed his
former way of life, devoting himself with special activity to
building operations and the arrangement of the gardens and
completely breaking off his relations with Mademoiselle
Bourienne. His looks and cold tone to his daughter seemed to say:
``There, you see? You plotted against me, you lied to Prince
Andrew about my relations with that Frenchwoman and made me
quarrel with him, but you see I need neither her nor you!''

Princess Mary spent half of every day with little Nicholas,
watching his lessons, teaching him Russian and music herself, and
talking to Dessalles; the rest of the day she spent over her
books, with her old nurse, or with \emph{God's folk} who
sometimes came by the back door to see her.

Of the war Princess Mary thought as women do think about
wars. She feared for her brother who was in it, was horrified by
and amazed at the strange cruelty that impels men to kill one
another, but she did not understand the significance of this war,
which seemed to her like all previous wars. She did not realize
the significance of this war, though Dessalles with whom she
constantly conversed was passionately interested in its progress
and tried to explain his own conception of it to her, and though
the \emph{God's folk} who came to see her reported, in their own
way, the rumors current among the people of an invasion by
Antichrist, and though Julie (now Princess Drubetskaya), who had
resumed correspondence with her, wrote patriotic letters from
Moscow.

``I write you in Russian, my good friend,'' wrote Julie in her
Frenchified Russian, ``because I have a detestation for all the
French, and the same for their language which I cannot support to
hear spoken... We in Moscow are elated by enthusiasm for our
adored Emperor.''

``My poor husband is enduring pains and hunger in Jewish taverns,
but the news which I have inspires me yet more.''

``You heard probably of the heroic exploit of Raevski, embracing
his two sons and saying: 'I will perish with them but we will not
be shaken!'  And truly though the enemy was twice stronger than
we, we were unshakable. We pass the time as we can, but in war as
in war! The princesses Aline and Sophie sit whole days with me,
and we, unhappy widows of live men, make beautiful conversations
over our 'charpie', only you, my friend, are missing...'' and so
on.

The chief reason Princess Mary did not realize the full
significance of this war was that the old prince never spoke of
it, did not recognize it, and laughed at Dessalles when he
mentioned it at dinner. The prince's tone was so calm and
confident that Princess Mary unhesitatingly believed him.

All that July the old prince was exceedingly active and even
animated.  He planned another garden and began a new building for
the domestic serfs. The only thing that made Princess Mary
anxious about him was that he slept very little and, instead of
sleeping in his study as usual, changed his sleeping place every
day. One day he would order his camp bed to be set up in the
glass gallery, another day he remained on the couch or on the
lounge chair in the drawing room and dozed there without
undressing, while---instead of Mademoiselle Bourienne---a serf
boy read to him. Then again he would spend a night in the dining
room.

On August 1st, a second letter was received from Prince
Andrew. In his first letter which came soon after he had left
home, Prince Andrew had dutifully asked his father's forgiveness
for what he had allowed himself to say and begged to be restored
to his favor. To this letter the old prince had replied
affectionately, and from that time had kept the Frenchwoman at a
distance. Prince Andrew's second letter, written near Vitebsk
after the French had occupied that town, gave a brief account of
the whole campaign, enclosed for them a plan he had drawn and
forecasts as to the further progress of the war. In this letter
Prince Andrew pointed out to his father the danger of staying at
Bald Hills, so near the theater of war and on the army's direct
line of march, and advised him to move to Moscow.

At dinner that day, on Dessalles' mentioning that the French were
said to have already entered Vitebsk, the old prince remembered
his son's letter.

``There was a letter from Prince Andrew today,''---he said to
Princess Mary---``Haven't you read it?''

``No, Father,'' she replied in a frightened voice.

She could not have read the letter as she did not even know it
had arrived.

``He writes about this war,'' said the prince, with the ironic
smile that had become habitual to him in speaking of the present
war.

``That must be very interesting,'' said Dessalles. ``Prince
Andrew is in a position to know...''

``Oh, very interesting!'' said Mademoiselle Bourienne.

``Go and get it for me,'' said the old prince to Mademoiselle
Bourienne.  ``You know---under the paperweight on the little
table.''

Mademoiselle Bourienne jumped up eagerly.

``No, don't!'' he exclaimed with a frown. ``You go, Michael
Ivanovich.''

Michael Ivanovich rose and went to the study. But as soon as he
had left the room the old prince, looking uneasily round, threw
down his napkin and went himself.

``They can't do anything... always make some muddle,'' he
muttered.

While he was away Princess Mary, Dessalles, Mademoiselle
Bourienne, and even little Nicholas exchanged looks in
silence. The old prince returned with quick steps, accompanied by
Michael Ivanovich, bringing the letter and a plan. These he put
down beside him---not letting anyone read them at dinner.

On moving to the drawing room he handed the letter to Princess
Mary and, spreading out before him the plan of the new building
and fixing his eyes upon it, told her to read the letter
aloud. When she had done so Princess Mary looked inquiringly at
her father. He was examining the plan, evidently engrossed in his
own ideas.

``What do you think of it, Prince?'' Dessalles ventured to ask.

``I? I?...'' said the prince as if unpleasantly awakened, and not
taking his eyes from the plan of the building.

``Very possibly the theater of war will move so near to us
that...''

``Ha ha ha! The theater of war!'' said the prince. ``I have said
and still say that the theater of war is Poland and the enemy
will never get beyond the Niemen.''

Dessalles looked in amazement at the prince, who was talking of
the Niemen when the enemy was already at the Dnieper, but
Princess Mary, forgetting the geographical position of the
Niemen, thought that what her father was saying was correct.

``When the snow melts they'll sink in the Polish swamps. Only
they could fail to see it,'' the prince continued, evidently
thinking of the campaign of 1807 which seemed to him so
recent. ``Bennigsen should have advanced into Prussia sooner,
then things would have taken a different turn...''

``But, Prince,'' Dessalles began timidly, ``the letter mentions
Vitebsk...''

``Ah, the letter? Yes...'' replied the prince
peevishly. ``Yes... yes...''  His face suddenly took on a morose
expression. He paused. ``Yes, he writes that the French were
beaten at... at... what river is it?''

Dessalles dropped his eyes.

``The prince says nothing about that,'' he remarked gently.

``Doesn't he? But I didn't invent it myself.''

No one spoke for a long time.

``Yes... yes... Well, Michael Ivanovich,'' he suddenly went on,
raising his head and pointing to the plan of the building, ``tell
me how you mean to alter it...''

Michael Ivanovich went up to the plan, and the prince after
speaking to him about the building looked angrily at Princess
Mary and Dessalles and went to his own room.

Princess Mary saw Dessalles' embarrassed and astonished look
fixed on her father, noticed his silence, and was struck by the
fact that her father had forgotten his son's letter on the
drawing-room table; but she was not only afraid to speak of it
and ask Dessalles the reason of his confusion and silence, but
was afraid even to think about it.

In the evening Michael Ivanovich, sent by the prince, came to
Princess Mary for Prince Andrew's letter which had been forgotten
in the drawing room. She gave it to him and, unpleasant as it was
to her to do so, ventured to ask him what her father was doing.

``Always busy,'' replied Michael Ivanovich with a respectfully
ironic smile which caused Princess Mary to turn pale. ``He's
worrying very much about the new building. He has been reading a
little, but now''---Michael Ivanovich went on, lowering his
voice---``now he's at his desk, busy with his will, I expect.''
(One of the prince's favorite occupations of late had been the
preparation of some papers he meant to leave at his death and
which he called his \emph{will}.)

``And Alpatych is being sent to Smolensk?'' asked Princess Mary.

``Oh, yes, he has been waiting to start for some time.''

% % % % % % % % % % % % % % % % % % % % % % % % % % % % % % % % %
% % % % % % % % % % % % % % % % % % % % % % % % % % % % % % % % %
% % % % % % % % % % % % % % % % % % % % % % % % % % % % % % % % %
% % % % % % % % % % % % % % % % % % % % % % % % % % % % % % % % %
% % % % % % % % % % % % % % % % % % % % % % % % % % % % % % % % %
% % % % % % % % % % % % % % % % % % % % % % % % % % % % % % % % %
% % % % % % % % % % % % % % % % % % % % % % % % % % % % % % % % %
% % % % % % % % % % % % % % % % % % % % % % % % % % % % % % % % %
% % % % % % % % % % % % % % % % % % % % % % % % % % % % % % % % %
% % % % % % % % % % % % % % % % % % % % % % % % % % % % % % % % %
% % % % % % % % % % % % % % % % % % % % % % % % % % % % % % % % %
% % % % % % % % % % % % % % % % % % % % % % % % % % % % % %

\chapter*{Chapter III} \ifaudio \marginpar{
\href{http://ia801407.us.archive.org/32/items/war_and_peace_10_0904_librivox/war_and_peace_10_03_tolstoy_64kb.mp3}{Audio}}
\fi

\initial{W}{hen} Michael Ivanovich returned to the study with the letter, the
old prince, with spectacles on and a shade over his eyes, was
sitting at his open bureau with screened candles, holding a paper
in his outstretched hand, and in a somewhat dramatic attitude was
reading his manuscript---his \emph{Remarks} as he termed
it---which was to be transmitted to the Emperor after his death.

When Michael Ivanovich went in there were tears in the prince's
eyes evoked by the memory of the time when the paper he was now
reading had been written. He took the letter from Michael
Ivanovich's hand, put it in his pocket, folded up his papers, and
called in Alpatych who had long been waiting.

The prince had a list of things to be bought in Smolensk and,
walking up and down the room past Alpatych who stood by the door,
he gave his instructions.

``First, notepaper---do you hear? Eight quires, like this sample,
gilt-edged... it must be exactly like the sample. Varnish,
sealing wax, as in Michael Ivanovich's list.''

He paced up and down for a while and glanced at his notes.

``Then hand to the governor in person a letter about the deed.''

Next, bolts for the doors of the new building were wanted and had
to be of a special shape the prince had himself designed, and a
leather case had to be ordered to keep the \emph{will} in.

The instructions to Alpatych took over two hours and still the
prince did not let him go. He sat down, sank into thought, closed
his eyes, and dozed off. Alpatych made a slight movement.

``Well, go, go! If anything more is wanted I'll send after you.''

Alpatych went out. The prince again went to his bureau, glanced
into it, fingered his papers, closed the bureau again, and sat
down at the table to write to the governor.

It was already late when he rose after sealing the letter. He
wished to sleep, but he knew he would not be able to and that
most depressing thoughts came to him in bed. So he called Tikhon
and went through the rooms with him to show him where to set up
the bed for that night.

He went about looking at every corner. Every place seemed
unsatisfactory, but worst of all was his customary couch in the
study.  That couch was dreadful to him, probably because of the
oppressive thoughts he had had when lying there. It was
unsatisfactory everywhere, but the corner behind the piano in the
sitting room was better than other places: he had never slept
there yet.

With the help of a footman Tikhon brought in the bedstead and
began putting it up.

``That's not right! That's not right!'' cried the prince, and
himself pushed it a few inches from the corner and then closer in
again.

``Well, at last I've finished, now I'll rest,'' thought the
prince, and let Tikhon undress him.

Frowning with vexation at the effort necessary to divest himself
of his coat and trousers, the prince undressed, sat down heavily
on the bed, and appeared to be meditating as he looked
contemptuously at his withered yellow legs. He was not
meditating, but only deferring the moment of making the effort to
lift those legs up and turn over on the bed. ``Ugh, how hard it
is! Oh, that this toil might end and you would release me!''
thought he. Pressing his lips together he made that effort for
the twenty-thousandth time and lay down. But hardly had he done
so before he felt the bed rocking backwards and forwards beneath
him as if it were breathing heavily and jolting. This happened to
him almost every night. He opened his eyes as they were closing.

``No peace, damn them!'' he muttered, angry he knew not with
whom. ``Ah yes, there was something else important, very
important, that I was keeping till I should be in bed. The bolts?
No, I told him about them.  No, it was something, something in
the drawing room. Princess Mary talked some nonsense. Dessalles,
that fool, said something. Something in my pocket---can't
remember...''

``Tikhon, what did we talk about at dinner?''

``About Prince Michael...''

``Be quiet, quiet!'' The prince slapped his hand on the
table. ``Yes, I know, Prince Andrew's letter! Princess Mary read
it. Dessalles said something about Vitebsk. Now I'll read it.''

He had the letter taken from his pocket and the table---on which
stood a glass of lemonade and a spiral wax candle---moved close
to the bed, and putting on his spectacles he began reading. Only
now in the stillness of the night, reading it by the faint light
under the green shade, did he grasp its meaning for a moment.

``The French at Vitebsk, in four days' march they may be at
Smolensk; perhaps are already there! Tikhon!'' Tikhon jumped
up. ``No, no, I don't want anything!'' he shouted.

He put the letter under the candlestick and closed his eyes. And
there rose before him the Danube at bright noonday: reeds, the
Russian camp, and himself a young general without a wrinkle on
his ruddy face, vigorous and alert, entering Potemkin's gaily
colored tent, and a burning sense of jealousy of \emph{the
favorite} agitated him now as strongly as it had done then. He
recalled all the words spoken at that first meeting with
Potemkin. And he saw before him a plump, rather sallow-faced,
short, stout woman, the Empress Mother, with her smile and her
words at her first gracious reception of him, and then that same
face on the catafalque, and the encounter he had with Zubov over
her coffin about his right to kiss her hand.

``Oh, quicker, quicker! To get back to that time and have done
with all the present! Quicker, quicker---and that they should
leave me in peace!''

% % % % % % % % % % % % % % % % % % % % % % % % % % % % % % % % %
% % % % % % % % % % % % % % % % % % % % % % % % % % % % % % % % %
% % % % % % % % % % % % % % % % % % % % % % % % % % % % % % % % %
% % % % % % % % % % % % % % % % % % % % % % % % % % % % % % % % %
% % % % % % % % % % % % % % % % % % % % % % % % % % % % % % % % %
% % % % % % % % % % % % % % % % % % % % % % % % % % % % % % % % %
% % % % % % % % % % % % % % % % % % % % % % % % % % % % % % % % %
% % % % % % % % % % % % % % % % % % % % % % % % % % % % % % % % %
% % % % % % % % % % % % % % % % % % % % % % % % % % % % % % % % %
% % % % % % % % % % % % % % % % % % % % % % % % % % % % % % % % %
% % % % % % % % % % % % % % % % % % % % % % % % % % % % % % % % %
% % % % % % % % % % % % % % % % % % % % % % % % % % % % % %

\chapter*{Chapter IV} \ifaudio \marginpar{
\href{http://ia801407.us.archive.org/32/items/war_and_peace_10_0904_librivox/war_and_peace_10_04_tolstoy_64kb.mp3}{Audio}}
\fi

\initial{B}{ald} Hills, Prince Nicholas Bolkonski's estate, lay forty miles
east from Smolensk and two miles from the main road to Moscow.

The same evening that the prince gave his instructions to
Alpatych, Dessalles, having asked to see Princess Mary, told her
that, as the prince was not very well and was taking no steps to
secure his safety, though from Prince Andrew's letter it was
evident that to remain at Bald Hills might be dangerous, he
respectfully advised her to send a letter by Alpatych to the
Provincial Governor at Smolensk, asking him to let her know the
state of affairs and the extent of the danger to which Bald Hills
was exposed. Dessalles wrote this letter to the Governor for
Princess Mary, she signed it, and it was given to Alpatych with
instructions to hand it to the Governor and to come back as
quickly as possible if there was danger.

Having received all his orders Alpatych, wearing a white beaver
hat---a present from the prince---and carrying a stick as the
prince did, went out accompanied by his family. Three well-fed
roans stood ready harnessed to a small conveyance with a leather
hood.

The larger bell was muffled and the little bells on the harness
stuffed with paper. The prince allowed no one at Bald Hills to
drive with ringing bells; but on a long journey Alpatych liked to
have them. His satellites---the senior clerk, a countinghouse
clerk, a scullery maid, a cook, two old women, a little pageboy,
the coachman, and various domestic serfs---were seeing him off.

His daughter placed chintz-covered down cushions for him to sit
on and behind his back. His old sister-in-law popped in a small
bundle, and one of the coachmen helped him into the vehicle.

``There! There! Women's fuss! Women, women!'' said Alpatych,
puffing and speaking rapidly just as the prince did, and he
climbed into the trap.

After giving the clerk orders about the work to be done,
Alpatych, not trying to imitate the prince now, lifted the hat
from his bald head and crossed himself three times.

``If there is anything... come back, Yakov Alpatych! For Christ's
sake think of us!'' cried his wife, referring to the rumors of
war and the enemy.

``Women, women! Women's fuss!'' muttered Alpatych to himself and
started on his journey, looking round at the fields of yellow rye
and the still-green, thickly growing oats, and at other quite
black fields just being plowed a second time.

As he went along he looked with pleasure at the year's splendid
crop of corn, scrutinized the strips of ryefield which here and
there were already being reaped, made his calculations as to the
sowing and the harvest, and asked himself whether he had not
forgotten any of the prince's orders.

Having baited the horses twice on the way, he arrived at the town
toward evening on the fourth of August.

Alpatych kept meeting and overtaking baggage trains and troops on
the road. As he approached Smolensk he heard the sounds of
distant firing, but these did not impress him. What struck him
most was the sight of a splendid field of oats in which a camp
had been pitched and which was being mown down by the soldiers,
evidently for fodder. This fact impressed Alpatych, but in
thinking about his own business he soon forgot it.

All the interests of his life for more than thirty years had been
bounded by the will of the prince, and he never went beyond that
limit.  Everything not connected with the execution of the
prince's orders did not interest and did not even exist for
Alpatych.

On reaching Smolensk on the evening of the fourth of August he
put up in the Gachina suburb across the Dnieper, at the inn kept
by Ferapontov, where he had been in the habit of putting up for
the last thirty years.  Some thirty years ago Ferapontov, by
Alpatych's advice, had bought a wood from the prince, had begun
to trade, and now had a house, an inn, and a corn dealer's shop
in that province. He was a stout, dark, red-faced peasant in the
forties, with thick lips, a broad knob of a nose, similar knobs
over his black frowning brows, and a round belly.

Wearing a waistcoat over his cotton shirt, Ferapontov was
standing before his shop which opened onto the street. On seeing
Alpatych he went up to him.

``You're welcome, Yakov Alpatych. Folks are leaving the town, but
you have come to it,'' said he.

``Why are they leaving the town?'' asked Alpatych.

``That's what I say. Folks are foolish! Always afraid of the
French.''

``Women's fuss, women's fuss!'' said Alpatych.

``Just what I think, Yakov Alpatych. What I say is: orders have
been given not to let them in, so that must be right. And the
peasants are asking three rubles for carting---it isn't
Christian!''

Yakov Alpatych heard without heeding. He asked for a samovar and
for hay for his horses, and when he had had his tea he went to
bed.

All night long troops were moving past the inn. Next morning
Alpatych donned a jacket he wore only in town and went out on
business. It was a sunny morning and by eight o'clock it was
already hot. ``A good day for harvesting,'' thought Alpatych.

From beyond the town firing had been heard since early
morning. At eight o'clock the booming of cannon was added to the
sound of musketry. Many people were hurrying through the streets
and there were many soldiers, but cabs were still driving about,
tradesmen stood at their shops, and service was being held in the
churches as usual. Alpatych went to the shops, to government
offices, to the post office, and to the Governor's.  In the
offices and shops and at the post office everyone was talking
about the army and about the enemy who was already attacking the
town, everybody was asking what should be done, and all were
trying to calm one another.

In front of the Governor's house Alpatych found a large number of
people, Cossacks, and a traveling carriage of the Governor's. At
the porch he met two of the landed gentry, one of whom he
knew. This man, an ex-captain of police, was saying angrily:

``It's no joke, you know! It's all very well if you're
single. 'One man though undone is but one,' as the proverb says,
but with thirteen in your family and all the property... They've
brought us to utter ruin!  What sort of governors are they to do
that? They ought to be hanged---the brigands!...''

``Oh come, that's enough!'' said the other.

``What do I care? Let him hear! We're not dogs,'' said the
ex-captain of police, and looking round he noticed Alpatych.

``Oh, Yakov Alpatych! What have you come for?''

``To see the Governor by his excellency's order,'' answered
Alpatych, lifting his head and proudly thrusting his hand into
the bosom of his coat as he always did when he mentioned the
prince... ``He has ordered me to inquire into the position of
affairs,'' he added.

``Yes, go and find out!'' shouted the angry gentleman. ``They've
brought things to such a pass that there are no carts or
anything!... There it is again, do you hear?'' said he, pointing
in the direction whence came the sounds of firing.

``They've brought us all to ruin... the brigands!'' he repeated,
and descended the porch steps.

Alpatych swayed his head and went upstairs. In the waiting room
were tradesmen, women, and officials, looking silently at one
another. The door of the Governor's room opened and they all rose
and moved forward.  An official ran out, said some words to a
merchant, called a stout official with a cross hanging on his
neck to follow him, and vanished again, evidently wishing to
avoid the inquiring looks and questions addressed to
him. Alpatych moved forward and next time the official came out
addressed him, one hand placed in the breast of his buttoned
coat, and handed him two letters.

``To his Honor Baron Asch, from General-in-Chief Prince
Bolkonski,'' he announced with such solemnity and significance
that the official turned to him and took the letters.

A few minutes later the Governor received Alpatych and hurriedly
said to him:

``Inform the prince and princess that I knew nothing: I acted on
the highest instructions---here...'' and he handed a paper to
Alpatych.  ``Still, as the prince is unwell my advice is that
they should go to Moscow. I am just starting myself. Inform
them...''

But the Governor did not finish: a dusty perspiring officer ran
into the room and began to say something in French. The
Governor's face expressed terror.

``Go,'' he said, nodding his head to Alpatych, and began
questioning the officer.

Eager, frightened, helpless glances were turned on Alpatych when
he came out of the Governor's room. Involuntarily listening now
to the firing, which had drawn nearer and was increasing in
strength, Alpatych hurried to his inn. The paper handed to him by
the Governor said this:

``I assure you that the town of Smolensk is not in the slightest
danger as yet and it is unlikely that it will be threatened with
any. I from the one side and Prince Bagration from the other are
marching to unite our forces before Smolensk, which junction will
be effected on the 22nd instant, and both armies with their
united forces will defend our compatriots of the province
entrusted to your care till our efforts shall have beaten back
the enemies of our Fatherland, or till the last warrior in our
valiant ranks has perished. From this you will see that you have
a perfect right to reassure the inhabitants of Smolensk, for
those defended by two such brave armies may feel assured of
victory.''  (Instructions from Barclay de Tolly to Baron Asch,
the civil governor of Smolensk, 1812.)

People were anxiously roaming about the streets.

Carts piled high with household utensils, chairs, and cupboards
kept emerging from the gates of the yards and moving along the
streets.  Loaded carts stood at the house next to Ferapontov's
and women were wailing and lamenting as they said good-by. A
small watchdog ran round barking in front of the harnessed
horses.

Alpatych entered the innyard at a quicker pace than usual and
went straight to the shed where his horses and trap were. The
coachman was asleep. He woke him up, told him to harness, and
went into the passage.  From the host's room came the sounds of a
child crying, the despairing sobs of a woman, and the hoarse
angry shouting of Ferapontov. The cook began running hither and
thither in the passage like a frightened hen, just as Alpatych
entered.

``He's done her to death. Killed the mistress!... Beat
her... dragged her about so!...''

``What for?'' asked Alpatych.

``She kept begging to go away. She's a woman! 'Take me away,'
says she, 'don't let me perish with my little children! Folks,'
she says, 'are all gone, so why,' she says, 'don't we go?' And he
began beating and pulling her about so!''

At these words Alpatych nodded as if in approval, and not wishing
to hear more went to the door of the room opposite the
innkeeper's, where he had left his purchases.

``You brute, you murderer!'' screamed a thin, pale woman who,
with a baby in her arms and her kerchief torn from her head,
burst through the door at that moment and down the steps into the
yard.

Ferapontov came out after her, but on seeing Alpatych adjusted
his waistcoat, smoothed his hair, yawned, and followed Alpatych
into the opposite room.

``Going already?'' said he.

Alpatych, without answering or looking at his host, sorted his
packages and asked how much he owed.

``We'll reckon up! Well, have you been to the Governor's?'' asked
Ferapontov. ``What has been decided?''

Alpatych replied that the Governor had not told him anything
definite.

``With our business, how can we get away?'' said
Ferapontov. ``We'd have to pay seven rubles a cartload to
Dorogobuzh and I tell them they're not Christians to ask it!
Selivanov, now, did a good stroke last Thursday---sold flour to
the army at nine rubles a sack. Will you have some tea?''  he
added.

While the horses were being harnessed Alpatych and Ferapontov
over their tea talked of the price of corn, the crops, and the
good weather for harvesting.

``Well, it seems to be getting quieter,'' remarked Ferapontov,
finishing his third cup of tea and getting up. ``Ours must have
got the best of it.  The orders were not to let them in. So we're
in force, it seems... They say the other day Matthew Ivanych
Platov drove them into the river Marina and drowned some eighteen
thousand in one day.''

Alpatych collected his parcels, handed them to the coachman who
had come in, and settled up with the innkeeper. The noise of
wheels, hoofs, and bells was heard from the gateway as a little
trap passed out.

It was by now late in the afternoon. Half the street was in
shadow, the other half brightly lit by the sun. Alpatych looked
out of the window and went to the door. Suddenly the strange
sound of a far-off whistling and thud was heard, followed by a
boom of cannon blending into a dull roar that set the windows
rattling.

He went out into the street: two men were running past toward the
bridge. From different sides came whistling sounds and the thud
of cannon balls and bursting shells falling on the town. But
these sounds were hardly heard in comparison with the noise of
the firing outside the town and attracted little attention from
the inhabitants. The town was being bombarded by a hundred and
thirty guns which Napoleon had ordered up after four o'clock. The
people did not at once realize the meaning of this bombardment.

At first the noise of the falling bombs and shells only aroused
curiosity. Ferapontov's wife, who till then had not ceased
wailing under the shed, became quiet and with the baby in her
arms went to the gate, listening to the sounds and looking in
silence at the people.

The cook and a shop assistant came to the gate. With lively
curiosity everyone tried to get a glimpse of the projectiles as
they flew over their heads. Several people came round the corner
talking eagerly.

``What force!'' remarked one. ``Knocked the roof and ceiling all
to splinters!''

``Routed up the earth like a pig,'' said another.

``That's grand, it bucks one up!'' laughed the first. ``Lucky you
jumped aside, or it would have wiped you out!''

Others joined those men and stopped and told how cannon balls had
fallen on a house close to them. Meanwhile still more
projectiles, now with the swift sinister whistle of a cannon
ball, now with the agreeable intermittent whistle of a shell,
flew over people's heads incessantly, but not one fell close by,
they all flew over. Alpatych was getting into his trap. The
innkeeper stood at the gate.

``What are you staring at?'' he shouted to the cook, who in her
red skirt, with sleeves rolled up, swinging her bare elbows, had
stepped to the corner to listen to what was being said.

``What marvels!'' she exclaimed, but hearing her master's voice
she turned back, pulling down her tucked-up skirt.

Once more something whistled, but this time quite close, swooping
downwards like a little bird; a flame flashed in the middle of
the street, something exploded, and the street was shrouded in
smoke.

``Scoundrel, what are you doing?'' shouted the innkeeper, rushing
to the cook.

At that moment the pitiful wailing of women was heard from
different sides, the frightened baby began to cry, and people
crowded silently with pale faces round the cook. The loudest
sound in that crowd was her wailing.

``Oh-h-h! Dear souls, dear kind souls! Don't let me die! My good
souls!...''

Five minutes later no one remained in the street. The cook, with
her thigh broken by a shell splinter, had been carried into the
kitchen.  Alpatych, his coachman, Ferapontov's wife and children
and the house porter were all sitting in the cellar,
listening. The roar of guns, the whistling of projectiles, and
the piteous moaning of the cook, which rose above the other
sounds, did not cease for a moment. The mistress rocked and
hushed her baby and when anyone came into the cellar asked in a
pathetic whisper what had become of her husband who had remained
in the street. A shopman who entered told her that her husband
had gone with others to the cathedral, whence they were fetching
the wonder-working icon of Smolensk.

Toward dusk the cannonade began to subside. Alpatych left the
cellar and stopped in the doorway. The evening sky that had been
so clear was clouded with smoke, through which, high up, the
sickle of the new moon shone strangely. Now that the terrible din
of the guns had ceased a hush seemed to reign over the town,
broken only by the rustle of footsteps, the moaning, the distant
cries, and the crackle of fires which seemed widespread
everywhere. The cook's moans had now subsided. On two sides black
curling clouds of smoke rose and spread from the fires. Through
the streets soldiers in various uniforms walked or ran confusedly
in different directions like ants from a ruined ant-hill. Several
of them ran into Ferapontov's yard before Alpatych's
eyes. Alpatych went out to the gate. A retreating regiment,
thronging and hurrying, blocked the street.

Noticing him, an officer said: ``The town is being abandoned. Get
away, get away!'' and then, turning to the soldiers, shouted:

``I'll teach you to run into the yards!''

Alpatych went back to the house, called the coachman, and told
him to set off. Ferapontov's whole household came out too,
following Alpatych and the coachman. The women, who had been
silent till then, suddenly began to wail as they looked at the
fires---the smoke and even the flames of which could be seen in
the failing twilight---and as if in reply the same kind of
lamentation was heard from other parts of the street.  Inside the
shed Alpatych and the coachman arranged the tangled reins and
traces of their horses with trembling hands.

As Alpatych was driving out of the gate he saw some ten soldiers
in Ferapontov's open shop, talking loudly and filling their bags
and knapsacks with flour and sunflower seeds. Just then
Ferapontov returned and entered his shop. On seeing the soldiers
he was about to shout at them, but suddenly stopped and,
clutching at his hair, burst into sobs and laughter:

``Loot everything, lads! Don't let those devils get it!'' he
cried, taking some bags of flour himself and throwing them into
the street.

Some of the soldiers were frightened and ran away, others went on
filling their bags. On seeing Alpatych, Ferapontov turned to him:

``Russia is done for!'' he cried. ``Alpatych, I'll set the place
on fire myself. We're done for!...'' and Ferapontov ran into the
yard.

Soldiers were passing in a constant stream along the street
blocking it completely, so that Alpatych could not pass out and
had to wait.  Ferapontov's wife and children were also sitting in
a cart waiting till it was possible to drive out.

Night had come. There were stars in the sky and the new moon
shone out amid the smoke that screened it. On the sloping descent
to the Dnieper Alpatych's cart and that of the innkeeper's wife,
which were slowly moving amid the rows of soldiers and of other
vehicles, had to stop. In a side street near the crossroads where
the vehicles had stopped, a house and some shops were on
fire. This fire was already burning itself out. The flames now
died down and were lost in the black smoke, now suddenly flared
up again brightly, lighting up with strange distinctness the
faces of the people crowding at the crossroads. Black figures
flitted about before the fire, and through the incessant
crackling of the flames talking and shouting could be
heard. Seeing that his trap would not be able to move on for some
time, Alpatych got down and turned into the side street to look
at the fire. Soldiers were continually rushing backwards and
forwards near it, and he saw two of them and a man in a frieze
coat dragging burning beams into another yard across the street,
while others carried bundles of hay.

Alpatych went up to a large crowd standing before a high barn
which was blazing briskly. The walls were all on fire and the
back wall had fallen in, the wooden roof was collapsing, and the
rafters were alight. The crowd was evidently watching for the
roof to fall in, and Alpatych watched for it too.

``Alpatych!'' a familiar voice suddenly hailed the old man.

``Mercy on us! Your excellency!'' answered Alpatych, immediately
recognizing the voice of his young prince.

Prince Andrew in his riding cloak, mounted on a black horse, was
looking at Alpatych from the back of the crowd.

``Why are you here?'' he asked.

``Your... your excellency,'' stammered Alpatych and broke into
sobs. ``Are we really lost? Master!...''

``Why are you here?'' Prince Andrew repeated.

At that moment the flames flared up and showed his young master's
pale worn face. Alpatych told how he had been sent there and how
difficult it was to get away.

``Are we really quite lost, your excellency?'' he asked again.

Prince Andrew without replying took out a notebook and raising
his knee began writing in pencil on a page he tore out. He wrote
to his sister:

``Smolensk is being abandoned. Bald Hills will be occupied by the
enemy within a week. Set off immediately for Moscow. Let me know
at once when you will start. Send by special messenger to
Usvyazh.''

Having written this and given the paper to Alpatych, he told him
how to arrange for departure of the prince, the princess, his
son, and the boy's tutor, and how and where to let him know
immediately. Before he had had time to finish giving these
instructions, a chief of staff followed by a suite galloped up to
him.

``You are a colonel?'' shouted the chief of staff with a German
accent, in a voice familiar to Prince Andrew. ``Houses are set on
fire in your presence and you stand by! What does this mean? You
will answer for it!''  shouted Berg, who was now assistant to the
chief of staff of the commander of the left flank of the infantry
of the first army, a place, as Berg said, ``very agreeable and
well en evidence.''

Prince Andrew looked at him and without replying went on speaking
to Alpatych.

``So tell them that I shall await a reply till the tenth, and if
by the tenth I don't receive news that they have all got away I
shall have to throw up everything and come myself to Bald
Hills.''

``Prince,'' said Berg, recognizing Prince Andrew, ``I only spoke
because I have to obey orders, because I always do obey
exactly... You must please excuse me,'' he went on
apologetically.

Something cracked in the flames. The fire died down for a moment
and wreaths of black smoke rolled from under the roof. There was
another terrible crash and something huge collapsed.

``Ou-rou-rou!'' yelled the crowd, echoing the crash of the
collapsing roof of the barn, the burning grain in which diffused
a cakelike aroma all around. The flames flared up again, lighting
the animated, delighted, exhausted faces of the spectators.

The man in the frieze coat raised his arms and shouted:

``It's fine, lads! Now it's raging... It's fine!''

``That's the owner himself,'' cried several voices.

``Well then,'' continued Prince Andrew to Alpatych, ``report to
them as I have told you''; and not replying a word to Berg who
was now mute beside him, he touched his horse and rode down the
side street.

% % % % % % % % % % % % % % % % % % % % % % % % % % % % % % % % %
% % % % % % % % % % % % % % % % % % % % % % % % % % % % % % % % %
% % % % % % % % % % % % % % % % % % % % % % % % % % % % % % % % %
% % % % % % % % % % % % % % % % % % % % % % % % % % % % % % % % %
% % % % % % % % % % % % % % % % % % % % % % % % % % % % % % % % %
% % % % % % % % % % % % % % % % % % % % % % % % % % % % % % % % %
% % % % % % % % % % % % % % % % % % % % % % % % % % % % % % % % %
% % % % % % % % % % % % % % % % % % % % % % % % % % % % % % % % %
% % % % % % % % % % % % % % % % % % % % % % % % % % % % % % % % %
% % % % % % % % % % % % % % % % % % % % % % % % % % % % % % % % %
% % % % % % % % % % % % % % % % % % % % % % % % % % % % % % % % %
% % % % % % % % % % % % % % % % % % % % % % % % % % % % % %

\chapter*{Chapter V} \ifaudio \marginpar{
\href{http://ia801407.us.archive.org/32/items/war_and_peace_10_0904_librivox/war_and_peace_10_05_tolstoy_64kb.mp3}{Audio}}
\fi

\initial{F}{rom} Smolensk the troops continued to retreat, followed by the
enemy. On the tenth of August the regiment Prince Andrew
commanded was marching along the highroad past the avenue leading
to Bald Hills. Heat and drought had continued for more than three
weeks. Each day fleecy clouds floated across the sky and
occasionally veiled the sun, but toward evening the sky cleared
again and the sun set in reddish-brown mist.  Heavy night dews
alone refreshed the earth. The unreaped corn was scorched and
shed its grain. The marshes dried up. The cattle lowed from
hunger, finding no food on the sun-parched meadows. Only at night
and in the forests while the dew lasted was there any
freshness. But on the road, the highroad along which the troops
marched, there was no such freshness even at night or when the
road passed through the forest; the dew was imperceptible on the
sandy dust churned up more than six inches deep. As soon as day
dawned the march began. The artillery and baggage wagons moved
noiselessly through the deep dust that rose to the very hubs of
the wheels, and the infantry sank ankle-deep in that soft,
choking, hot dust that never cooled even at night. Some of this
dust was kneaded by the feet and wheels, while the rest rose and
hung like a cloud over the troops, settling in eyes, ears, hair,
and nostrils, and worst of all in the lungs of the men and beasts
as they moved along that road. The higher the sun rose the higher
rose that cloud of dust, and through the screen of its hot fine
particles one could look with naked eye at the sun, which showed
like a huge crimson ball in the unclouded sky. There was no wind,
and the men choked in that motionless atmosphere. They marched
with handkerchiefs tied over their noses and mouths. When they
passed through a village they all rushed to the wells and fought
for the water and drank it down to the mud.

Prince Andrew was in command of a regiment, and the management of
that regiment, the welfare of the men and the necessity of
receiving and giving orders, engrossed him. The burning of
Smolensk and its abandonment made an epoch in his life. A novel
feeling of anger against the foe made him forget his own
sorrow. He was entirely devoted to the affairs of his regiment
and was considerate and kind to his men and officers. In the
regiment they called him \emph{our prince}, were proud of him and
loved him. But he was kind and gentle only to those of his
regiment, to Timokhin and the like---people quite new to him,
belonging to a different world and who could not know and
understand his past. As soon as he came across a former
acquaintance or anyone from the staff, he bristled up immediately
and grew spiteful, ironical, and contemptuous. Everything that
reminded him of his past was repugnant to him, and so in his
relations with that former circle he confined himself to trying
to do his duty and not to be unfair.

In truth everything presented itself in a dark and gloomy light
to Prince Andrew, especially after the abandonment of Smolensk on
the sixth of August (he considered that it could and should have
been defended) and after his sick father had had to flee to
Moscow, abandoning to pillage his dearly beloved Bald Hills which
he had built and peopled.  But despite this, thanks to his
regiment, Prince Andrew had something to think about entirely
apart from general questions. Two days previously he had received
news that his father, son, and sister had left for Moscow; and
though there was nothing for him to do at Bald Hills, Prince
Andrew with a characteristic desire to foment his own grief
decided that he must ride there.

He ordered his horse to be saddled and, leaving his regiment on
the march, rode to his father's estate where he had been born and
spent his childhood. Riding past the pond where there used always
to be dozens of women chattering as they rinsed their linen or
beat it with wooden beetles, Prince Andrew noticed that there was
not a soul about and that the little washing wharf, torn from its
place and half submerged, was floating on its side in the middle
of the pond. He rode to the keeper's lodge. No one at the stone
entrance gates of the drive and the door stood open. Grass had
already begun to grow on the garden paths, and horses and calves
were straying in the English park. Prince Andrew rode up to the
hothouse; some of the glass panes were broken, and of the trees
in tubs some were overturned and others dried up. He called for
Taras the gardener, but no one replied. Having gone round the
corner of the hothouse to the ornamental garden, he saw that the
carved garden fence was broken and branches of the plum trees had
been torn off with the fruit. An old peasant whom Prince Andrew
in his childhood had often seen at the gate was sitting on a
green garden seat, plaiting a bast shoe.

He was deaf and did not hear Prince Andrew ride up. He was
sitting on the seat the old prince used to like to sit on, and
beside him strips of bast were hanging on the broken and withered
branch of a magnolia.

Prince Andrew rode up to the house. Several limes in the old
garden had been cut down and a piebald mare and her foal were
wandering in front of the house among the rosebushes. The
shutters were all closed, except at one window which was open. A
little serf boy, seeing Prince Andrew, ran into the
house. Alpatych, having sent his family away, was alone at Bald
Hills and was sitting indoors reading the Lives of the Saints. On
hearing that Prince Andrew had come, he went out with his
spectacles on his nose, buttoning his coat, and, hastily stepping
up, without a word began weeping and kissing Prince Andrew's
knee.

Then, vexed at his own weakness, he turned away and began to
report on the position of affairs. Everything precious and
valuable had been removed to Bogucharovo. Seventy quarters of
grain had also been carted away. The hay and the spring corn, of
which Alpatych said there had been a remarkable crop that year,
had been commandeered by the troops and mown down while still
green. The peasants were ruined; some of them too had gone to
Bogucharovo, only a few remained.

Without waiting to hear him out, Prince Andrew asked:

``When did my father and sister leave?'' meaning when did they
leave for Moscow.

Alpatych, understanding the question to refer to their departure
for Bogucharovo, replied that they had left on the seventh and
again went into details concerning the estate management, asking
for instructions.

``Am I to let the troops have the oats, and to take a receipt for
them?  We have still six hundred quarters left,'' he inquired.

``What am I to say to him?'' thought Prince Andrew, looking down
on the old man's bald head shining in the sun and seeing by the
expression on his face that the old man himself understood how
untimely such questions were and only asked them to allay his
grief.

``Yes, let them have it,'' replied Prince Andrew.

``If you noticed some disorder in the garden,'' said Alpatych,
``it was impossible to prevent it. Three regiments have been here
and spent the night, dragoons mostly. I took down the name and
rank of their commanding officer, to hand in a complaint about
it.''

``Well, and what are you going to do? Will you stay here if the
enemy occupies the place?'' asked Prince Andrew.

Alpatych turned his face to Prince Andrew, looked at him, and
suddenly with a solemn gesture raised his arm.

``He is my refuge! His will be done!'' he exclaimed.

A group of bareheaded peasants was approaching across the meadow
toward the prince.

``Well, good-by!'' said Prince Andrew, bending over to
Alpatych. ``You must go away too, take away what you can and tell
the serfs to go to the Ryazan estate or to the one near Moscow.''

Alpatych clung to Prince Andrew's leg and burst into sobs. Gently
disengaging himself, the prince spurred his horse and rode down
the avenue at a gallop.

The old man was still sitting in the ornamental garden, like a
fly impassive on the face of a loved one who is dead, tapping the
last on which he was making the bast shoe, and two little girls,
running out from the hot house carrying in their skirts plums
they had plucked from the trees there, came upon Prince
Andrew. On seeing the young master, the elder one with frightened
look clutched her younger companion by the hand and hid with her
behind a birch tree, not stopping to pick up some green plums
they had dropped.

Prince Andrew turned away with startled haste, unwilling to let
them see that they had been observed. He was sorry for the pretty
frightened little girl, was afraid of looking at her, and yet
felt an irresistible desire to do so. A new sensation of comfort
and relief came over him when, seeing these girls, he realized
the existence of other human interests entirely aloof from his
own and just as legitimate as those that occupied him. Evidently
these girls passionately desired one thing---to carry away and
eat those green plums without being caught---and Prince Andrew
shared their wish for the success of their enterprise. He could
not resist looking at them once more. Believing their danger
past, they sprang from their ambush and, chirruping something in
their shrill little voices and holding up their skirts, their
bare little sunburned feet scampered merrily and quickly across
the meadow grass.

Prince Andrew was somewhat refreshed by having ridden off the
dusty highroad along which the troops were moving. But not far
from Bald Hills he again came out on the road and overtook his
regiment at its halting place by the dam of a small pond. It was
past one o'clock. The sun, a red ball through the dust, burned
and scorched his back intolerably through his black coat. The
dust always hung motionless above the buzz of talk that came from
the resting troops. There was no wind. As he crossed the dam
Prince Andrew smelled the ooze and freshness of the pond. He
longed to get into that water, however dirty it might be, and he
glanced round at the pool from whence came sounds of shrieks and
laughter. The small, muddy, green pond had risen visibly more
than a foot, flooding the dam, because it was full of the naked
white bodies of soldiers with brick-red hands, necks, and faces,
who were splashing about in it. All this naked white human flesh,
laughing and shrieking, floundered about in that dirty pool like
carp stuffed into a watering can, and the suggestion of merriment
in that floundering mass rendered it specially pathetic.

One fair-haired young soldier of the third company, whom Prince
Andrew knew and who had a strap round the calf of one leg,
crossed himself, stepped back to get a good run, and plunged into
the water; another, a dark noncommissioned officer who was always
shaggy, stood up to his waist in the water joyfully wriggling his
muscular figure and snorted with satisfaction as he poured the
water over his head with hands blackened to the wrists. There
were sounds of men slapping one another, yelling, and puffing.

Everywhere on the bank, on the dam, and in the pond, there was
healthy, white, muscular flesh. The officer, Timokhin, with his
red little nose, standing on the dam wiping himself with a towel,
felt confused at seeing the prince, but made up his mind to
address him nevertheless.

``It's very nice, your excellency! Wouldn't you like to?'' said
he.

``It's dirty,'' replied Prince Andrew, making a grimace.

``We'll clear it out for you in a minute,'' said Timokhin, and,
still undressed, ran off to clear the men out of the pond.

``The prince wants to bathe.''

``What prince? Ours?'' said many voices, and the men were in such
haste to clear out that the prince could hardly stop them. He
decided that he would rather wash himself with water in the barn.

``Flesh, bodies, cannon fodder!'' he thought, and he looked at
his own naked body and shuddered, not from cold but from a sense
of disgust and horror he did not himself understand, aroused by
the sight of that immense number of bodies splashing about in the
dirty pond.

On the seventh of August Prince Bagration wrote as follows from
his quarters at Mikhaylovna on the Smolensk road:

Dear Count Alexis Andreevich---(He was writing to Arakcheev but
knew that his letter would be read by the Emperor, and therefore
weighed every word in it to the best of his ability.)

I expect the Minister (Barclay de Tolly) has already reported the
abandonment of Smolensk to the enemy. It is pitiable and sad, and
the whole army is in despair that this most important place has
been wantonly abandoned. I, for my part, begged him personally
most urgently and finally wrote him, but nothing would induce him
to consent. I swear to you on my honor that Napoleon was in such
a fix as never before and might have lost half his army but could
not have taken Smolensk. Our troops fought, and are fighting, as
never before. With fifteen thousand men I held the enemy at bay
for thirty-five hours and beat him; but he would not hold out
even for fourteen hours. It is disgraceful, a stain on our army,
and as for him, he ought, it seems to me, not to live. If he
reports that our losses were great, it is not true; perhaps about
four thousand, not more, and not even that; but even were they
ten thousand, that's war! But the enemy has lost masses...

What would it have cost him to hold out for another two days?
They would have had to retire of their own accord, for they had
no water for men or horses. He gave me his word he would not
retreat, but suddenly sent instructions that he was retiring that
night. We cannot fight in this way, or we may soon bring the
enemy to Moscow...

There is a rumor that you are thinking of peace. God forbid that
you should make peace after all our sacrifices and such insane
retreats! You would set all Russia against you and every one of
us would feel ashamed to wear the uniform. If it has come to
this---we must fight as long as Russia can and as long as there
are men able to stand...

One man ought to be in command, and not two. Your Minister may
perhaps be good as a Minister, but as a general he is not merely
bad but execrable, yet to him is entrusted the fate of our whole
country... I am really frantic with vexation; forgive my writing
boldly. It is clear that the man who advocates the conclusion of
a peace, and that the Minister should command the army, does not
love our sovereign and desires the ruin of us all. So I write you
frankly: call out the militia. For the Minister is leading these
visitors after him to Moscow in a most masterly way. The whole
army feels great suspicion of the Imperial aide-de-camp
Wolzogen. He is said to be more Napoleon's man than ours, and he
is always advising the Minister. I am not merely civil to him but
obey him like a corporal, though I am his senior. This is
painful, but, loving my benefactor and sovereign, I submit. Only
I am sorry for the Emperor that he entrusts our fine army to such
as he.  Consider that on our retreat we have lost by fatigue and
left in the hospital more than fifteen thousand men, and had we
attacked this would not have happened. Tell me, for God's sake,
what will Russia, our mother Russia, say to our being so
frightened, and why are we abandoning our good and gallant
Fatherland to such rabble and implanting feelings of hatred and
shame in all our subjects? What are we scared at and of whom are
we afraid? I am not to blame that the Minister is vacillating, a
coward, dense, dilatory, and has all bad qualities. The whole
army bewails it and calls down curses upon him...

% % % % % % % % % % % % % % % % % % % % % % % % % % % % % % % % %
% % % % % % % % % % % % % % % % % % % % % % % % % % % % % % % % %
% % % % % % % % % % % % % % % % % % % % % % % % % % % % % % % % %
% % % % % % % % % % % % % % % % % % % % % % % % % % % % % % % % %
% % % % % % % % % % % % % % % % % % % % % % % % % % % % % % % % %
% % % % % % % % % % % % % % % % % % % % % % % % % % % % % % % % %
% % % % % % % % % % % % % % % % % % % % % % % % % % % % % % % % %
% % % % % % % % % % % % % % % % % % % % % % % % % % % % % % % % %
% % % % % % % % % % % % % % % % % % % % % % % % % % % % % % % % %
% % % % % % % % % % % % % % % % % % % % % % % % % % % % % % % % %
% % % % % % % % % % % % % % % % % % % % % % % % % % % % % % % % %
% % % % % % % % % % % % % % % % % % % % % % % % % % % % % %

\chapter*{Chapter VI} \ifaudio \marginpar{
\href{http://ia801407.us.archive.org/32/items/war_and_peace_10_0904_librivox/war_and_peace_10_06_tolstoy_64kb.mp3}{Audio}}
\fi

\initial{A}{mong} the innumerable categories applicable to the phenomena of
human life one may discriminate between those in which substance
prevails and those in which form prevails. To the latter---as
distinguished from village, country, provincial, or even Moscow
life---we may allot Petersburg life, and especially the life of
its salons. That life of the salons is unchanging. Since the year
1805 we had made peace and had again quarreled with Bonaparte and
had made constitutions and unmade them again, but the salons of
Anna Pavlovna and Helene remained just as they had been---the one
seven and the other five years before. At Anna Pavlovna's they
talked with perplexity of Bonaparte's successes just as before
and saw in them and in the subservience shown to him by the
European sovereigns a malicious conspiracy, the sole object of
which was to cause unpleasantness and anxiety to the court circle
of which Anna Pavlovna was the representative. And in Helene's
salon, which Rumyantsev himself honored with his visits,
regarding Helene as a remarkably intelligent woman, they talked
with the same ecstasy in 1812 as in 1808 of the \emph{great
nation} and the \emph{great man}, and regretted our rupture with
France, a rupture which, according to them, ought to be promptly
terminated by peace.

Of late, since the Emperor's return from the army, there had been
some excitement in these conflicting salon circles and some
demonstrations of hostility to one another, but each camp
retained its own tendency. In Anna Pavlovna's circle only those
Frenchmen were admitted who were deep-rooted legitimists, and
patriotic views were expressed to the effect that one ought not
to go to the French theater and that to maintain the French
troupe was costing the government as much as a whole army corps.
The progress of the war was eagerly followed, and only the
reports most flattering to our army were circulated. In the
French circle of Helene and Rumyantsev the reports of the cruelty
of the enemy and of the war were contradicted and all Napoleon's
attempts at conciliation were discussed. In that circle they
discountenanced those who advised hurried preparations for a
removal to Kazan of the court and the girls' educational
establishments under the patronage of the Dowager Empress.  In
Helene's circle the war in general was regarded as a series of
formal demonstrations which would very soon end in peace, and the
view prevailed expressed by Bilibin---who now in Petersburg was
quite at home in Helene's house, which every clever man was
obliged to visit---that not by gunpowder but by those who
invented it would matters be settled. In that circle the Moscow
enthusiasm---news of which had reached Petersburg simultaneously
with the Emperor's return---was ridiculed sarcastically and very
cleverly, though with much caution.

Anna Pavlovna's circle on the contrary was enraptured by this
enthusiasm and spoke of it as Plutarch speaks of the deeds of the
ancients. Prince Vasili, who still occupied his former important
posts, formed a connecting link between these two circles. He
visited his \emph{good friend Anna Pavlovna} as well as his
daughter's \emph{diplomatic salon}, and often in his constant
comings and goings between the two camps became confused and said
at Helene's what he should have said at Anna Pavlovna's and vice
versa.

Soon after the Emperor's return Prince Vasili in a conversation
about the war at Anna Pavlovna's severely condemned Barclay de
Tolly, but was undecided as to who ought to be appointed
commander-in-chief. One of the visitors, usually spoken of as
\emph{a man of great merit}, having described how he had that day
seen Kutuzov, the newly chosen chief of the Petersburg militia,
presiding over the enrollment of recruits at the Treasury,
cautiously ventured to suggest that Kutuzov would be the man to
satisfy all requirements.

Anna Pavlovna remarked with a melancholy smile that Kutuzov had
done nothing but cause the Emperor annoyance.

``I have talked and talked at the Assembly of the Nobility,''
Prince Vasili interrupted, ``but they did not listen to me. I
told them his election as chief of the militia would not please
the Emperor. They did not listen to me.''

``It's all this mania for opposition,'' he went on. ``And who
for? It is all because we want to ape the foolish enthusiasm of
those Muscovites,'' Prince Vasili continued, forgetting for a
moment that though at Helene's one had to ridicule the Moscow
enthusiasm, at Anna Pavlovna's one had to be ecstatic about
it. But he retrieved his mistake at once. ``Now, is it suitable
that Count Kutuzov, the oldest general in Russia, should preside
at that tribunal? He will get nothing for his pains! How could
they make a man commander-in-chief who cannot mount a horse, who
drops asleep at a council, and has the very worst morals! A good
reputation he made for himself at Bucharest! I don't speak of his
capacity as a general, but at a time like this how they appoint a
decrepit, blind old man, positively blind? A fine idea to have a
blind general! He can't see anything. To play blindman's bluff?
He can't see at all!''

No one replied to his remarks.

This was quite correct on the twenty-fourth of July. But on the
twenty-ninth of July Kutuzov received the title of Prince. This
might indicate a wish to get rid of him, and therefore Prince
Vasili's opinion continued to be correct though he was not now in
any hurry to express it. But on the eighth of August a committee,
consisting of Field Marshal Saltykov, Arakcheev, Vyazmitinov,
Lopukhin, and Kochubey met to consider the progress of the
war. This committee came to the conclusion that our failures were
due to a want of unity in the command and though the members of
the committee were aware of the Emperor's dislike of Kutuzov,
after a short deliberation they agreed to advise his appointment
as commander in chief. That same day Kutuzov was appointed
commander-in-chief with full powers over the armies and over the
whole region occupied by them.

On the ninth of August Prince Vasili at Anna Pavlovna's again met
the \emph{man of great merit}. The latter was very attentive to
Anna Pavlovna because he wanted to be appointed director of one
of the educational establishments for young ladies. Prince Vasili
entered the room with the air of a happy conqueror who has
attained the object of his desires.

``Well, have you heard the great news? Prince Kutuzov is field
marshal!  All dissensions are at an end! I am so glad, so
delighted! At last we have a man!'' said he, glancing sternly and
significantly round at everyone in the drawing room.

The \emph{man of great merit}, despite his desire to obtain the
post of director, could not refrain from reminding Prince Vasili
of his former opinion. Though this was impolite to Prince Vasili
in Anna Pavlovna's drawing room, and also to Anna Pavlovna
herself who had received the news with delight, he could not
resist the temptation.

``But, Prince, they say he is blind!'' said he, reminding Prince
Vasili of his own words.

``Eh? Nonsense! He sees well enough,'' said Prince Vasili
rapidly, in a deep voice and with a slight cough---the voice and
cough with which he was wont to dispose of all difficulties.

``He sees well enough,'' he added. ``And what I am so pleased
about,'' he went on, ``is that our sovereign has given him full
powers over all the armies and the whole region---powers no
commander-in-chief ever had before. He is a second autocrat,'' he
concluded with a victorious smile.

``God grant it! God grant it!'' said Anna Pavlovna.

The \emph{man of great merit}, who was still a novice in court
circles, wishing to flatter Anna Pavlovna by defending her former
position on this question, observed:

``It is said that the Emperor was reluctant to give Kutuzov those
powers.  They say he blushed like a girl to whom Joconde is read,
when he said to Kutuzov: 'Your Emperor and the Fatherland award
you this honor.'{}''

``Perhaps the heart took no part in that speech,'' said Anna
Pavlovna.

``Oh, no, no!'' warmly rejoined Prince Vasili, who would not now
yield Kutuzov to anyone; in his opinion Kutuzov was not only
admirable himself, but was adored by everybody. ``No, that's
impossible,'' said he, ``for our sovereign appreciated him so
highly before.''

``God grant only that Prince Kutuzov assumes real power and does
not allow anyone to put a spoke in his wheel,'' observed Anna
Pavlovna.

Understanding at once to whom she alluded, Prince Vasili said in
a whisper:

``I know for a fact that Kutuzov made it an absolute condition
that the Tsarevich should not be with the army. Do you know what
he said to the Emperor?''

And Prince Vasili repeated the words supposed to have been spoken
by Kutuzov to the Emperor. ``I can neither punish him if he does
wrong nor reward him if he does right.''

``Oh, a very wise man is Prince Kutuzov! I have known him a long
time!''

``They even say,'' remarked the ``man of great merit'' who did
not yet possess courtly tact, ``that his excellency made it an
express condition that the sovereign himself should not be with
the army.''

As soon as he said this both Prince Vasili and Anna Pavlovna
turned away from him and glanced sadly at one another with a sigh
at his naivete.

% % % % % % % % % % % % % % % % % % % % % % % % % % % % % % % % %
% % % % % % % % % % % % % % % % % % % % % % % % % % % % % % % % %
% % % % % % % % % % % % % % % % % % % % % % % % % % % % % % % % %
% % % % % % % % % % % % % % % % % % % % % % % % % % % % % % % % %
% % % % % % % % % % % % % % % % % % % % % % % % % % % % % % % % %
% % % % % % % % % % % % % % % % % % % % % % % % % % % % % % % % %
% % % % % % % % % % % % % % % % % % % % % % % % % % % % % % % % %
% % % % % % % % % % % % % % % % % % % % % % % % % % % % % % % % %
% % % % % % % % % % % % % % % % % % % % % % % % % % % % % % % % %
% % % % % % % % % % % % % % % % % % % % % % % % % % % % % % % % %
% % % % % % % % % % % % % % % % % % % % % % % % % % % % % % % % %
% % % % % % % % % % % % % % % % % % % % % % % % % % % % % %

\chapter*{Chapter VII} \ifaudio \marginpar{
\href{http://ia801407.us.archive.org/32/items/war_and_peace_10_0904_librivox/war_and_peace_10_07_tolstoy_64kb.mp3}{Audio}}
\fi

\initial{W}{hile} this was taking place in Petersburg the French had already
passed Smolensk and were drawing nearer and nearer to
Moscow. Napoleon's historian Thiers, like other of his
historians, trying to justify his hero says that he was drawn to
the walls of Moscow against his will. He is as right as other
historians who look for the explanation of historic events in the
will of one man; he is as right as the Russian historians who
maintain that Napoleon was drawn to Moscow by the skill of the
Russian commanders. Here besides the law of retrospection, which
regards all the past as a preparation for events that
subsequently occur, the law of reciprocity comes in, confusing
the whole matter. A good chessplayer having lost a game is
sincerely convinced that his loss resulted from a mistake he made
and looks for that mistake in the opening, but forgets that at
each stage of the game there were similar mistakes and that none
of his moves were perfect. He only notices the mistake to which
he pays attention, because his opponent took advantage of it. How
much more complex than this is the game of war, which occurs
under certain limits of time, and where it is not one will that
manipulates lifeless objects, but everything results from
innumerable conflicts of various wills!

After Smolensk Napoleon sought a battle beyond Dorogobuzh at
Vyazma, and then at Tsarevo-Zaymishche, but it happened that
owing to a conjunction of innumerable circumstances the Russians
could not give battle till they reached Borodino, seventy miles
from Moscow. From Vyazma Napoleon ordered a direct advance on
Moscow.

Moscou, la capitale asiatique de ce grand empire, la ville sacree
des peuples d'Alexandre, Moscou avec ses innombrables eglises en
forme de pagodes chinoises,\footnote{``Moscow, the Asiatic
capital of this great empire, the sacred city of Alexander's
people, Moscow with its innumerable churches shaped like Chinese
pagodas.''} this Moscow gave Napoleon's imagination no rest. On
the march from Vyazma to Tsarevo-Zaymishche he rode his light bay
bobtailed ambler accompanied by his Guards, his bodyguard, his
pages, and aides-de-camp. Berthier, his chief of staff, dropped
behind to question a Russian prisoner captured by the
cavalry. Followed by Lelorgne d'Ideville, an interpreter, he
overtook Napoleon at a gallop and reined in his horse with an
amused expression.

``Well?'' asked Napoleon.

``One of Platov's Cossacks says that Platov's corps is joining up
with the main army and that Kutuzov has been appointed
commander-in-chief. He is a very shrewd and garrulous fellow.''

Napoleon smiled and told them to give the Cossack a horse and
bring the man to him. He wished to talk to him himself. Several
adjutants galloped off, and an hour later, Lavrushka, the serf
Denisov had handed over to Rostov, rode up to Napoleon in an
orderly's jacket and on a French cavalry saddle, with a merry,
and tipsy face. Napoleon told him to ride by his side and began
questioning him.

``You are a Cossack?''

``Yes, a Cossack, your Honor.''

``The Cossack, not knowing in what company he was, for Napoleon's
plain appearance had nothing about it that would reveal to an
Oriental mind the presence of a monarch, talked with extreme
familiarity of the incidents of the war,'' says Thiers, narrating
this episode. In reality Lavrushka, having got drunk the day
before and left his master dinnerless, had been whipped and sent
to the village in quest of chickens, where he engaged in looting
till the French took him prisoner.  Lavrushka was one of those
coarse, bare-faced lackeys who have seen all sorts of things,
consider it necessary to do everything in a mean and cunning way,
are ready to render any sort of service to their master, and are
keen at guessing their master's baser impulses, especially those
prompted by vanity and pettiness.

Finding himself in the company of Napoleon, whose identity he had
easily and surely recognized, Lavrushka was not in the least
abashed but merely did his utmost to gain his new master's favor.

He knew very well that this was Napoleon, but Napoleon's presence
could no more intimidate him than Rostov's, or a sergeant major's
with the rods, would have done, for he had nothing that either
the sergeant major or Napoleon could deprive him of.

So he rattled on, telling all the gossip he had heard among the
orderlies. Much of it true. But when Napoleon asked him whether
the Russians thought they would beat Bonaparte or not, Lavrushka
screwed up his eyes and considered.

In this question he saw subtle cunning, as men of his type see
cunning in everything, so he frowned and did not answer
immediately.

``It's like this,'' he said thoughtfully, ``if there's a battle
soon, yours will win. That's right. But if three days pass, then
after that, well, then that same battle will not soon be over.''

Lelorgne d'Ideville smilingly interpreted this speech to Napoleon
thus: ``If a battle takes place within the next three days the
French will win, but if later, God knows what will happen.''
Napoleon did not smile, though he was evidently in high good
humor, and he ordered these words to be repeated.

Lavrushka noticed this and to entertain him further, pretending
not to know who Napoleon was, added:

``We know that you have Bonaparte and that he has beaten
everybody in the world, but we are a different
matter...''---without knowing why or how this bit of boastful
patriotism slipped out at the end.

The interpreter translated these words without the last phrase,
and Bonaparte smiled. ``The young Cossack made his mighty
interlocutor smile,'' says Thiers. After riding a few paces in
silence, Napoleon turned to Berthier and said he wished to see
how the news that he was talking to the Emperor himself, to that
very Emperor who had written his immortally victorious name on
the Pyramids, would affect this enfant du Don.\footnote{``Child
of the Don.''}

The fact was accordingly conveyed to Lavrushka.

Lavrushka, understanding that this was done to perplex him and
that Napoleon expected him to be frightened, to gratify his new
masters promptly pretended to be astonished and awe-struck,
opened his eyes wide, and assumed the expression he usually put
on when taken to be whipped. ``As soon as Napoleon's interpreter
had spoken,'' says Thiers, ``the Cossack, seized by amazement,
did not utter another word, but rode on, his eyes fixed on the
conqueror whose fame had reached him across the steppes of the
East. All his loquacity was suddenly arrested and replaced by a
naive and silent feeling of admiration. Napoleon, after making
the Cossack a present, had him set free like a bird restored to
its native fields.''

Napoleon rode on, dreaming of the Moscow that so appealed to his
imagination, and \emph{the bird restored to its native fields}
galloped to our outposts, inventing on the way all that had not
taken place but that he meant to relate to his comrades. What had
really taken place he did not wish to relate because it seemed to
him not worth telling. He found the Cossacks, inquired for the
regiment operating with Platov's detachment and by evening found
his master, Nicholas Rostov, quartered at Yankovo. Rostov was
just mounting to go for a ride round the neighboring villages
with Ilyin; he let Lavrushka have another horse and took him
along with him.

% % % % % % % % % % % % % % % % % % % % % % % % % % % % % % % % %
% % % % % % % % % % % % % % % % % % % % % % % % % % % % % % % % %
% % % % % % % % % % % % % % % % % % % % % % % % % % % % % % % % %
% % % % % % % % % % % % % % % % % % % % % % % % % % % % % % % % %
% % % % % % % % % % % % % % % % % % % % % % % % % % % % % % % % %
% % % % % % % % % % % % % % % % % % % % % % % % % % % % % % % % %
% % % % % % % % % % % % % % % % % % % % % % % % % % % % % % % % %
% % % % % % % % % % % % % % % % % % % % % % % % % % % % % % % % %
% % % % % % % % % % % % % % % % % % % % % % % % % % % % % % % % %
% % % % % % % % % % % % % % % % % % % % % % % % % % % % % % % % %
% % % % % % % % % % % % % % % % % % % % % % % % % % % % % % % % %
% % % % % % % % % % % % % % % % % % % % % % % % % % % % % %

\chapter*{Chapter VIII} \ifaudio \marginpar{
\href{http://ia801407.us.archive.org/32/items/war_and_peace_10_0904_librivox/war_and_peace_10_08_tolstoy_64kb.mp3}{Audio}}
\fi

\initial{P}{rincess} Mary was not in Moscow and out of danger as Prince
Andrew supposed.

After the return of Alpatych from Smolensk the old prince
suddenly seemed to awake as from a dream. He ordered the
militiamen to be called up from the villages and armed, and wrote
a letter to the commander-in-chief informing him that he had
resolved to remain at Bald Hills to the last extremity and to
defend it, leaving to the commander-in-chief's discretion to take
measures or not for the defense of Bald Hills, where one of
Russia's oldest generals would be captured or killed, and he
announced to his household that he would remain at Bald Hills.

But while himself remaining, he gave instructions for the
departure of the princess and Dessalles with the little prince to
Bogucharovo and thence to Moscow. Princess Mary, alarmed by her
father's feverish and sleepless activity after his previous
apathy, could not bring herself to leave him alone and for the
first time in her life ventured to disobey him. She refused to go
away and her father's fury broke over her in a terrible storm. He
repeated every injustice he had ever inflicted on her. Trying to
convict her, he told her she had worn him out, had caused his
quarrel with his son, had harbored nasty suspicions of him,
making it the object of her life to poison his existence, and he
drove her from his study telling her that if she did not go away
it was all the same to him. He declared that he did not wish to
remember her existence and warned her not to dare to let him see
her. The fact that he did not, as she had feared, order her to be
carried away by force but only told her not to let him see her
cheered Princess Mary. She knew it was a proof that in the depth
of his soul he was glad she was remaining at home and had not
gone away.

The morning after little Nicholas had left, the old prince donned
his full uniform and prepared to visit the
commander-in-chief. His caleche was already at the door. Princess
Mary saw him walk out of the house in his uniform wearing all his
orders and go down the garden to review his armed peasants and
domestic serfs. She sat by the window listening to his voice
which reached her from the garden. Suddenly several men came
running up the avenue with frightened faces.

Princess Mary ran out to the porch, down the flower-bordered
path, and into the avenue. A large crowd of militiamen and
domestics were moving toward her, and in their midst several men
were supporting by the armpits and dragging along a little old
man in a uniform and decorations. She ran up to him and, in the
play of the sunlight that fell in small round spots through the
shade of the lime-tree avenue, could not be sure what change
there was in his face. All she could see was that his former
stern and determined expression had altered to one of timidity
and submission. On seeing his daughter he moved his helpless lips
and made a hoarse sound. It was impossible to make out what he
wanted. He was lifted up, carried to his study, and laid on the
very couch he had so feared of late.

The doctor, who was fetched that same night, bled him and said
that the prince had had a seizure paralyzing his right side.

It was becoming more and more dangerous to remain at Bald Hills,
and next day they moved the prince to Bogucharovo, the doctor
accompanying him.

By the time they reached Bogucharovo, Dessalles and the little
prince had already left for Moscow.

For three weeks the old prince lay stricken by paralysis in the
new house Prince Andrew had built at Bogucharovo, ever in the
same state, getting neither better nor worse. He was unconscious
and lay like a distorted corpse. He muttered unceasingly, his
eyebrows and lips twitching, and it was impossible to tell
whether he understood what was going on around him or not. One
thing was certain---that he was suffering and wished to say
something. But what it was, no one could tell: it might be some
caprice of a sick and half-crazy man, or it might relate to
public affairs, or possibly to family concerns.

The doctor said this restlessness did not mean anything and was
due to physical causes; but Princess Mary thought he wished to
tell her something, and the fact that her presence always
increased his restlessness confirmed her opinion.

He was evidently suffering both physically and mentally. There
was no hope of recovery. It was impossible for him to travel, it
would not do to let him die on the road. ``Would it not be better
if the end did come, the very end?'' Princess Mary sometimes
thought. Night and day, hardly sleeping at all, she watched him
and, terrible to say, often watched him not with hope of finding
signs of improvement but wishing to find symptoms of the approach
of the end.

Strange as it was to her to acknowledge this feeling in herself,
yet there it was. And what seemed still more terrible to her was
that since her father's illness began (perhaps even sooner, when
she stayed with him expecting something to happen), all the
personal desires and hopes that had been forgotten or sleeping
within her had awakened. Thoughts that had not entered her mind
for years---thoughts of a life free from the fear of her father,
and even the possibility of love and of family
happiness---floated continually in her imagination like
temptations of the devil. Thrust them aside as she would,
questions continually recurred to her as to how she would order
her life now, after that.  These were temptations of the devil
and Princess Mary knew it. She knew that the sole weapon against
him was prayer, and she tried to pray. She assumed an attitude of
prayer, looked at the icons, repeated the words of a prayer, but
she could not pray. She felt that a different world had now taken
possession of her---the life of a world of strenuous and free
activity, quite opposed to the spiritual world in which till now
she had been confined and in which her greatest comfort had been
prayer. She could not pray, could not weep, and worldly cares
took possession of her.

It was becoming dangerous to remain in Bogucharovo. News of the
approach of the French came from all sides, and in one village,
ten miles from Bogucharovo, a homestead had been looted by French
marauders.

The doctor insisted on the necessity of moving the prince; the
provincial Marshal of the Nobility sent an official to Princess
Mary to persuade her to get away as quickly as possible, and the
head of the rural police having come to Bogucharovo urged the
same thing, saying that the French were only some twenty-five
miles away, that French proclamations were circulating in the
villages, and that if the princess did not take her father away
before the fifteenth, he could not answer for the consequences.

The princess decided to leave on the fifteenth. The cares of
preparation and giving orders, for which everyone came to her,
occupied her all day.  She spent the night of the fourteenth as
usual, without undressing, in the room next to the one where the
prince lay. Several times, waking up, she heard his groans and
muttering, the creak of his bed, and the steps of Tikhon and the
doctor when they turned him over. Several times she listened at
the door, and it seemed to her that his mutterings were louder
than usual and that they turned him over oftener. She could not
sleep and several times went to the door and listened, wishing to
enter but not deciding to do so. Though he did not speak,
Princess Mary saw and knew how unpleasant every sign of anxiety
on his account was to him.  She had noticed with what
dissatisfaction he turned from the look she sometimes
involuntarily fixed on him. She knew that her going in during the
night at an unusual hour would irritate him.

But never had she felt so grieved for him or so much afraid of
losing him. She recalled all her life with him and in every word
and act of his found an expression of his love of
her. Occasionally amid these memories temptations of the devil
would surge into her imagination: thoughts of how things would be
after his death, and how her new, liberated life would be
ordered. But she drove these thoughts away with disgust. Toward
morning he became quiet and she fell asleep.

She woke late. That sincerity which often comes with waking
showed her clearly what chiefly concerned her about her father's
illness. On waking she listened to what was going on behind the
door and, hearing him groan, said to herself with a sigh that
things were still the same.

``But what could have happened? What did I want? I want his
death!'' she cried with a feeling of loathing for herself.

She washed, dressed, said her prayers, and went out to the
porch. In front of it stood carriages without horses and things
were being packed into the vehicles.

It was a warm, gray morning. Princess Mary stopped at the porch,
still horrified by her spiritual baseness and trying to arrange
her thoughts before going to her father. The doctor came
downstairs and went out to her.

``He is a little better today,'' said he. ``I was looking for
you. One can make out something of what he is saying. His head is
clearer. Come in, he is asking for you...''

Princess Mary's heart beat so violently at this news that she
grew pale and leaned against the wall to keep from falling. To
see him, talk to him, feel his eyes on her now that her whole
soul was overflowing with those dreadful, wicked temptations, was
a torment of joy and terror.

``Come,'' said the doctor.

Princess Mary entered her father's room and went up to his
bed. He was lying on his back propped up high, and his small bony
hands with their knotted purple veins were lying on the quilt;
his left eye gazed straight before him, his right eye was awry,
and his brows and lips motionless. He seemed altogether so thin,
small, and pathetic. His face seemed to have shriveled or melted;
his features had grown smaller.  Princess Mary went up and kissed
his hand. His left hand pressed hers so that she understood that
he had long been waiting for her to come. He twitched her hand,
and his brows and lips quivered angrily.

She looked at him in dismay trying to guess what he wanted of
her. When she changed her position so that his left eye could see
her face he calmed down, not taking his eyes off her for some
seconds. Then his lips and tongue moved, sounds came, and he
began to speak, gazing timidly and imploringly at her, evidently
afraid that she might not understand.

Straining all her faculties Princess Mary looked at him. The
comic efforts with which he moved his tongue made her drop her
eyes and with difficulty repress the sobs that rose to her
throat. He said something, repeating the same words several
times. She could not understand them, but tried to guess what he
was saying and inquiringly repeated the words he uttered.

``Mmm...ar...ate...ate...'' he repeated several times.

It was quite impossible to understand these sounds. The doctor
thought he had guessed them, and inquiringly repeated: ``Mary,
are you afraid?''  The prince shook his head, again repeated the
same sounds.

``My mind, my mind aches?'' questioned Princess Mary.

He made a mumbling sound in confirmation of this, took her hand,
and began pressing it to different parts of his breast as if
trying to find the right place for it.

``Always thoughts... about you... thoughts...'' he then uttered
much more clearly than he had done before, now that he was sure
of being understood.

Princess Mary pressed her head against his hand, trying to hide
her sobs and tears.

He moved his hand over her hair.

``I have been calling you all night...'' he brought out.

``If only I had known...'' she said through her tears. ``I was
afraid to come in.''

He pressed her hand.

``Weren't you asleep?''

``No, I did not sleep,'' said Princess Mary, shaking her head.

Unconsciously imitating her father, she now tried to express
herself as he did, as much as possible by signs, and her tongue
too seemed to move with difficulty.

``Dear one... Dearest...'' Princess Mary could not quite make out
what he had said, but from his look it was clear that he had
uttered a tender caressing word such as he had never used to her
before. ``Why didn't you come in?''

``And I was wishing for his death!'' thought Princess Mary.

He was silent awhile.

``Thank you... daughter dear!... for all, for
all... forgive!... thank you!... forgive!... thank you!...'' and
tears began to flow from his eyes. ``Call Andrew!'' he said
suddenly, and a childish, timid expression of doubt showed itself
on his face as he spoke.

He himself seemed aware that his demand was meaningless. So at
least it seemed to Princess Mary.

``I have a letter from him,'' she replied.

He glanced at her with timid surprise.

``Where is he?''

``He's with the army, Father, at Smolensk.''

He closed his eyes and remained silent a long time. Then as if in
answer to his doubts and to confirm the fact that now he
understood and remembered everything, he nodded his head and
reopened his eyes.

``Yes,'' he said, softly and distinctly. ``Russia has
perished. They've destroyed her.''

And he began to sob, and again tears flowed from his
eyes. Princess Mary could no longer restrain herself and wept
while she gazed at his face.

Again he closed his eyes. His sobs ceased, he pointed to his
eyes, and Tikhon, understanding him, wiped away the tears.

Then he again opened his eyes and said something none of them
could understand for a long time, till at last Tikhon understood
and repeated it. Princess Mary had sought the meaning of his
words in the mood in which he had just been speaking. She thought
he was speaking of Russia, or Prince Andrew, of herself, of his
grandson, or of his own death, and so she could not guess his
words.

``Put on your white dress. I like it,'' was what he said.

Having understood this Princess Mary sobbed still louder, and the
doctor taking her arm led her out to the veranda, soothing her
and trying to persuade her to prepare for her journey. When she
had left the room the prince again began speaking about his son,
about the war, and about the Emperor, angrily twitching his brows
and raising his hoarse voice, and then he had a second and final
stroke.

Princess Mary stayed on the veranda. The day had cleared, it was
hot and sunny. She could understand nothing, think of nothing and
feel nothing, except passionate love for her father, love such as
she thought she had never felt till that moment. She ran out
sobbing into the garden and as far as the pond, along the avenues
of young lime trees Prince Andrew had planted.

``Yes... I... I... I wished for his death! Yes, I wanted it to
end quicker... I wished to be at peace... And what will become of
me? What use will peace be when he is no longer here?'' Princess
Mary murmured, pacing the garden with hurried steps and pressing
her hands to her bosom which heaved with convulsive sobs.

When she had completed the tour of the garden, which brought her
again to the house, she saw Mademoiselle Bourienne---who had
remained at Bogucharovo and did not wish to leave it---coming
toward her with a stranger. This was the Marshal of the Nobility
of the district, who had come personally to point out to the
princess the necessity for her prompt departure. Princess Mary
listened without understanding him; she led him to the house,
offered him lunch, and sat down with him. Then, excusing herself,
she went to the door of the old prince's room. The doctor came
out with an agitated face and said she could not enter.

``Go away, Princess! Go away... go away!''

She returned to the garden and sat down on the grass at the foot
of the slope by the pond, where no one could see her. She did not
know how long she had been there when she was aroused by the
sound of a woman's footsteps running along the path. She rose and
saw Dunyasha her maid, who was evidently looking for her, and who
stopped suddenly as if in alarm on seeing her mistress.

``Please come, Princess... The Prince,'' said Dunyasha in a
breaking voice.

``Immediately, I'm coming, I'm coming!'' replied the princess
hurriedly, not giving Dunyasha time to finish what she was
saying, and trying to avoid seeing the girl she ran toward the
house.

``Princess, it's God's will! You must be prepared for
everything,'' said the Marshal, meeting her at the house door.

``Let me alone; it's not true!'' she cried angrily to him.

The doctor tried to stop her. She pushed him aside and ran to her
father's door. ``Why are these people with frightened faces
stopping me?  I don't want any of them! And what are they doing
here?'' she thought.  She opened the door and the bright daylight
in that previously darkened room startled her. In the room were
her nurse and other women. They all drew back from the bed,
making way for her. He was still lying on the bed as before, but
the stern expression of his quiet face made Princess Mary stop
short on the threshold.

``No, he's not dead---it's impossible!'' she told herself and
approached him, and repressing the terror that seized her, she
pressed her lips to his cheek. But she stepped back
immediately. All the force of the tenderness she had been feeling
for him vanished instantly and was replaced by a feeling of
horror at what lay there before her. ``No, he is no more! He is
not, but here where he was is something unfamiliar and hostile,
some dreadful, terrifying, and repellent mystery!'' And hiding
her face in her hands, Princess Mary sank into the arms of the
doctor, who held her up.

In the presence of Tikhon and the doctor the women washed what
had been the prince, tied his head up with a handkerchief that
the mouth should not stiffen while open, and with another
handkerchief tied together the legs that were already spreading
apart. Then they dressed him in uniform with his decorations and
placed his shriveled little body on a table.  Heaven only knows
who arranged all this and when, but it all got done as if of its
own accord. Toward night candles were burning round his coffin, a
pall was spread over it, the floor was strewn with sprays of
juniper, a printed band was tucked in under his shriveled head,
and in a corner of the room sat a chanter reading the psalms.

Just as horses shy and snort and gather about a dead horse, so
the inmates of the house and strangers crowded into the drawing
room round the coffin---the Marshal, the village Elder, peasant
women---and all with fixed and frightened eyes, crossing
themselves, bowed and kissed the old prince's cold and stiffened
hand.

% % % % % % % % % % % % % % % % % % % % % % % % % % % % % % % % %
% % % % % % % % % % % % % % % % % % % % % % % % % % % % % % % % %
% % % % % % % % % % % % % % % % % % % % % % % % % % % % % % % % %
% % % % % % % % % % % % % % % % % % % % % % % % % % % % % % % % %
% % % % % % % % % % % % % % % % % % % % % % % % % % % % % % % % %
% % % % % % % % % % % % % % % % % % % % % % % % % % % % % % % % %
% % % % % % % % % % % % % % % % % % % % % % % % % % % % % % % % %
% % % % % % % % % % % % % % % % % % % % % % % % % % % % % % % % %
% % % % % % % % % % % % % % % % % % % % % % % % % % % % % % % % %
% % % % % % % % % % % % % % % % % % % % % % % % % % % % % % % % %
% % % % % % % % % % % % % % % % % % % % % % % % % % % % % % % % %
% % % % % % % % % % % % % % % % % % % % % % % % % % % % % %

\chapter*{Chapter IX} \ifaudio \marginpar{
\href{http://ia801407.us.archive.org/32/items/war_and_peace_10_0904_librivox/war_and_peace_10_09_tolstoy_64kb.mp3}{Audio}}
\fi

\initial{U}{ntil} Prince Andrew settled in Bogucharovo its owners had always
been absentees, and its peasants were of quite a different
character from those of Bald Hills. They differed from them in
speech, dress, and disposition. They were called steppe
peasants. The old prince used to approve of them for their
endurance at work when they came to Bald Hills to help with the
harvest or to dig ponds, and ditches, but he disliked them for
their boorishness.

Prince Andrew's last stay at Bogucharovo, when he introduced
hospitals and schools and reduced the quitrent the peasants had
to pay, had not softened their disposition but had on the
contrary strengthened in them the traits of character the old
prince called boorishness. Various obscure rumors were always
current among them: at one time a rumor that they would all be
enrolled as Cossacks; at another of a new religion to which they
were all to be converted; then of some proclamation of the Tsar's
and of an oath to the Tsar Paul in 1797 (in connection with which
it was rumored that freedom had been granted them but the
landowners had stopped it), then of Peter Fedorovich's return to
the throne in seven years' time, when everything would be made
free and so \emph{simple} that there would be no
restrictions. Rumors of the war with Bonaparte and his invasion
were connected in their minds with the same sort of vague notions
of Antichrist, the end of the world, and \emph{pure freedom}.

In the vicinity of Bogucharovo were large villages belonging to
the crown or to owners whose serfs paid quitrent and could work
where they pleased. There were very few resident landlords in the
neighborhood and also very few domestic or literate serfs, and in
the lives of the peasantry of those parts the mysterious
undercurrents in the life of the Russian people, the causes and
meaning of which are so baffling to contemporaries, were more
clearly and strongly noticeable than among others. One instance,
which had occurred some twenty years before, was a movement among
the peasants to emigrate to some unknown \emph{warm rivers}.
Hundreds of peasants, among them the Bogucharovo folk, suddenly
began selling their cattle and moving in whole families toward
the southeast.  As birds migrate to somewhere beyond the sea, so
these men with their wives and children streamed to the
southeast, to parts where none of them had ever been. They set
off in caravans, bought their freedom one by one or ran away, and
drove or walked toward the \emph{warm rivers}. Many of them were
punished, some sent to Siberia, many died of cold and hunger on
the road, many returned of their own accord, and the movement
died down of itself just as it had sprung up, without apparent
reason.  But such undercurrents still existed among the people
and gathered new forces ready to manifest themselves just as
strangely, unexpectedly, and at the same time simply, naturally,
and forcibly. Now in 1812, to anyone living in close touch with
these people it was apparent that these undercurrents were acting
strongly and nearing an eruption.

Alpatych, who had reached Bogucharovo shortly before the old
prince's death, noticed an agitation among the peasants, and that
contrary to what was happening in the Bald Hills district, where
over a radius of forty miles all the peasants were moving away
and leaving their villages to be devastated by the Cossacks, the
peasants in the steppe region round Bogucharovo were, it was
rumored, in touch with the French, received leaflets from them
that passed from hand to hand, and did not migrate. He learned
from domestic serfs loyal to him that the peasant Karp, who
possessed great influence in the village commune and had recently
been away driving a government transport, had returned with news
that the Cossacks were destroying deserted villages, but that the
French did not harm them. Alpatych also knew that on the previous
day another peasant had even brought from the village of
Visloukhovo, which was occupied by the French, a proclamation by
a French general that no harm would be done to the inhabitants,
and if they remained they would be paid for anything taken from
them. As proof of this the peasant had brought from Visloukhovo a
hundred rubles in notes (he did not know that they were false)
paid to him in advance for hay.

More important still, Alpatych learned that on the morning of the
very day he gave the village Elder orders to collect carts to
move the princess' luggage from Bogucharovo, there had been a
village meeting at which it had been decided not to move but to
wait. Yet there was no time to waste. On the fifteenth, the day
of the old prince's death, the Marshal had insisted on Princess
Mary's leaving at once, as it was becoming dangerous. He had told
her that after the sixteenth he could not be responsible for what
might happen. On the evening of the day the old prince died the
Marshal went away, promising to return next day for the
funeral. But this he was unable to do, for he received tidings
that the French had unexpectedly advanced, and had barely time to
remove his own family and valuables from his estate.

For some thirty years Bogucharovo had been managed by the village
Elder, Dron, whom the old prince called by the diminutive
'Dronushka.'

Dron was one of those physically and mentally vigorous peasants
who grow big beards as soon as they are of age and go on
unchanged till they are sixty or seventy, without a gray hair or
the loss of a tooth, as straight and strong at sixty as at
thirty.

Soon after the migration to the \emph{warm rivers}, in which he
had taken part like the rest, Dron was made village Elder and
overseer of Bo\-gu\-cha\-ro\-vo, and had since filled that post
irreproachably for twenty-three years. The peasants feared him
more than they did their master.  The masters, both the old
prince and the young, and the steward respected him and jestingly
called him \emph{the Minister}. During the whole time of his
service Dron had never been drunk or ill, never after sleepless
nights or the hardest tasks had he shown the least fatigue, and
though he could not read he had never forgotten a single money
account or the number of quarters of flour in any of the endless
cartloads he sold for the prince, nor a single shock of the whole
corn crop on any single acre of the Bogucharovo fields.

Alpatych, arriving from the devastated Bald Hills estate, sent
for his Dron on the day of the prince's funeral and told him to
have twelve horses got ready for the princess' carriages and
eighteen carts for the things to be removed from
Bogucharovo. Though the peasants paid quitrent, Alpatych thought
no difficulty would be made about complying with this order, for
there were two hundred and thirty households at work in
Bogucharovo and the peasants were well to do. But on hearing the
order Dron lowered his eyes and remained silent. Alpatych named
certain peasants he knew, from whom he told him to take the
carts.

Dron replied that the horses of these peasants were away carting.
Alpatych named others, but they too, according to Dron, had no
horses available: some horses were carting for the government,
others were too weak, and others had died for want of fodder. It
seemed that no horses could be had even for the carriages, much
less for the carting.

Alpatych looked intently at Dron and frowned. Just as Dron was a
model village Elder, so Alpatych had not managed the prince's
estates for twenty years in vain. He was a model steward,
possessing in the highest degree the faculty of divining the
needs and instincts of those he dealt with. Having glanced at
Dron he at once understood that his answers did not express his
personal views but the general mood of the Bogucharovo commune,
by which the Elder had already been carried away. But he also
knew that Dron, who had acquired property and was hated by the
commune, must be hesitating between the two camps: the masters'
and the serfs'.  He noticed this hesitation in Dron's look and
therefore frowned and moved closer up to him.

``Now just listen, Dronushka,'' said he. ``Don't talk nonsense to
me. His excellency Prince Andrew himself gave me orders to move
all the people away and not leave them with the enemy, and there
is an order from the Tsar about it too. Anyone who stays is a
traitor to the Tsar. Do you hear?''

``I hear,'' Dron answered without lifting his eyes.

Alpatych was not satisfied with this reply.

``Eh, Dron, it will turn out badly!'' he said, shaking his head.

``The power is in your hands,'' Dron rejoined sadly.

``Eh, Dron, drop it!'' Alpatych repeated, withdrawing his hand
from his bosom and solemnly pointing to the floor at Dron's
feet. ``I can see through you and three yards into the ground
under you,'' he continued, gazing at the floor in front of Dron.

Dron was disconcerted, glanced furtively at Alpatych and again
lowered his eyes.

``You drop this nonsense and tell the people to get ready to
leave their homes and go to Moscow and to get carts ready for
tomorrow morning for the princess' things. And don't go to any
meeting yourself, do you hear?''

Dron suddenly fell on his knees.

``Yakov Alpatych, discharge me! Take the keys from me and
discharge me, for Christ's sake!''

``Stop that!'' cried Alpatych sternly. ``I see through you and
three yards under you,'' he repeated, knowing that his skill in
beekeeping, his knowledge of the right time to sow the oats, and
the fact that he had been able to retain the old prince's favor
for twenty years had long since gained him the reputation of
being a wizard, and that the power of seeing three yards under a
man is considered an attribute of wizards.

Dron got up and was about to say something, but Alpatych
interrupted him.

``What is it you have got into your heads, eh?... What are you
thinking of, eh?''

``What am I to do with the people?'' said Dron. ``They're quite
beside themselves; I have already told them...''

``'Told them,' I dare say!'' said Alpatych. ``Are they
drinking?'' he asked abruptly.

``Quite beside themselves, Yakov Alpatych; they've fetched
another barrel.''

``Well, then, listen! I'll go to the police officer, and you tell
them so, and that they must stop this and the carts must be got
ready.''

``I understand.''

Alpatych did not insist further. He had managed people for a long
time and knew that the chief way to make them obey is to show no
suspicion that they can possibly disobey. Having wrung a
submissive ``I understand'' from Dron, Alpatych contented himself
with that, though he not only doubted but felt almost certain
that without the help of troops the carts would not be
forthcoming.

And so it was, for when evening came no carts had been
provided. In the village, outside the drink shop, another meeting
was being held, which decided that the horses should be driven
out into the woods and the carts should not be provided. Without
saying anything of this to the princess, Alpatych had his own
belongings taken out of the carts which had arrived from Bald
Hills and had those horses got ready for the princess'
carriages. Meanwhile he went himself to the police authorities.

% % % % % % % % % % % % % % % % % % % % % % % % % % % % % % % % %
% % % % % % % % % % % % % % % % % % % % % % % % % % % % % % % % %
% % % % % % % % % % % % % % % % % % % % % % % % % % % % % % % % %
% % % % % % % % % % % % % % % % % % % % % % % % % % % % % % % % %
% % % % % % % % % % % % % % % % % % % % % % % % % % % % % % % % %
% % % % % % % % % % % % % % % % % % % % % % % % % % % % % % % % %
% % % % % % % % % % % % % % % % % % % % % % % % % % % % % % % % %
% % % % % % % % % % % % % % % % % % % % % % % % % % % % % % % % %
% % % % % % % % % % % % % % % % % % % % % % % % % % % % % % % % %
% % % % % % % % % % % % % % % % % % % % % % % % % % % % % % % % %
% % % % % % % % % % % % % % % % % % % % % % % % % % % % % % % % %
% % % % % % % % % % % % % % % % % % % % % % % % % % % % % %

\chapter*{Chapter X} \ifaudio \marginpar{
\href{http://ia801407.us.archive.org/32/items/war_and_peace_10_0904_librivox/war_and_peace_10_10_tolstoy_64kb.mp3}{Audio}}
\fi

\initial{A}{fter} her father's funeral Princess Mary shut herself up in her
room and did not admit anyone. A maid came to the door to say
that Alpatych was asking for orders about their departure. (This
was before his talk with Dron.) Princess Mary raised herself on
the sofa on which she had been lying and replied through the
closed door that she did not mean to go away and begged to be
left in peace.

The windows of the room in which she was lying looked
westward. She lay on the sofa with her face to the wall,
fingering the buttons of the leather cushion and seeing nothing
but that cushion, and her confused thoughts were centered on one
subject---the irrevocability of death and her own spiritual
baseness, which she had not suspected, but which had shown itself
during her father's illness. She wished to pray but did not dare
to, dared not in her present state of mind address herself to
God.  She lay for a long time in that position.

The sun had reached the other side of the house, and its slanting
rays shone into the open window, lighting up the room and part of
the morocco cushion at which Princess Mary was looking. The flow
of her thoughts suddenly stopped. Unconsciously she sat up,
smoothed her hair, got up, and went to the window, involuntarily
inhaling the freshness of the clear but windy evening.

``Yes, you can well enjoy the evening now! He is gone and no one
will hinder you,'' she said to herself, and sinking into a chair
she let her head fall on the window sill.

Someone spoke her name in a soft and tender voice from the garden
and kissed her head. She looked up. It was Mademoiselle Bourienne
in a black dress and weepers. She softly approached Princess
Mary, sighed, kissed her, and immediately began to cry. The
princess looked up at her. All their former disharmony and her
own jealousy recurred to her mind. But she remembered too how he
had changed of late toward Mademoiselle Bourienne and could not
bear to see her, thereby showing how unjust were the reproaches
Princess Mary had mentally addressed to her. ``Besides, is it for
me, for me who desired his death, to condemn anyone?'' she
thought.

Princess Mary vividly pictured to herself the position of
Mademoiselle Bourienne, whom she had of late kept at a distance,
but who yet was dependent on her and living in her house. She
felt sorry for her and held out her hand with a glance of gentle
inquiry. Mademoiselle Bourienne at once began crying again and
kissed that hand, speaking of the princess' sorrow and making
herself a partner in it. She said her only consolation was the
fact that the princess allowed her to share her sorrow, that all
the old misunderstandings should sink into nothing but this great
grief; that she felt herself blameless in regard to everyone, and
that he, from above, saw her affection and gratitude. The
princess heard her, not heeding her words but occasionally
looking up at her and listening to the sound of her voice.

``Your position is doubly terrible, dear princess,'' said
Mademoiselle Bourienne after a pause. ``I understand that you
could not, and cannot, think of yourself, but with my love for
you I must do so... Has Alpatych been to you? Has he spoken to
you of going away?'' she asked.

Princess Mary did not answer. She did not understand who was to
go or where to. ``Is it possible to plan or think of anything
now? Is it not all the same?'' she thought, and did not reply.

``You know, chere Marie,'' said Mademoiselle Bourienne, ``that we
are in danger---are surrounded by the French. It would be
dangerous to move now.  If we go we are almost sure to be taken
prisoners, and God knows...''

Princess Mary looked at her companion without understanding what
she was talking about.

``Oh, if anyone knew how little anything matters to me now,'' she
said.  ``Of course I would on no account wish to go away from
him... Alpatych did say something about going... Speak to him; I
can do nothing, nothing, and don't want to...''

``I've spoken to him. He hopes we should be in time to get away
tomorrow, but I think it would now be better to stay here,'' said
Mademoiselle Bourienne. ``Because, you will agree, chere Marie,
to fall into the hands of the soldiers or of riotous peasants
would be terrible.''

Mademoiselle Bourienne took from her reticule a proclamation (not
printed on ordinary Russian paper) of General Rameau's, telling
people not to leave their homes and that the French authorities
would afford them proper protection. She handed this to the
princess.

``I think it would be best to appeal to that general,'' she
continued, ``and I am sure that all due respect would be shown
you.''

Princess Mary read the paper, and her face began to quiver with
stifled sobs.

``From whom did you get this?'' she asked.

``They probably recognized that I am French, by my name,''
replied Mademoiselle Bourienne blushing.

Princess Mary, with the paper in her hand, rose from the window
and with a pale face went out of the room and into what had been
Prince Andrew's study.

``Dunyasha, send Alpatych, or Dronushka, or somebody to me!'' she
said, ``and tell Mademoiselle Bourienne not to come to me,'' she
added, hearing Mademoiselle Bourienne's voice. ``We must go at
once, at once!'' she said, appalled at the thought of being left
in the hands of the French.

``If Prince Andrew heard that I was in the power of the French!
That I, the daughter of Prince Nicholas Bolkonski, asked General
Rameau for protection and accepted his favor!'' This idea
horrified her, made her shudder, blush, and feel such a rush of
anger and pride as she had never experienced before. All that was
distressing, and especially all that was humiliating, in her
position rose vividly to her mind. ``They, the French, would
settle in this house: M. le General Rameau would occupy Prince
Andrew's study and amuse himself by looking through and reading
his letters and papers. Mademoiselle Bourienne would do the
honors of Bogucharovo for him. I should be given a small room as
a favor, the soldiers would violate my father's newly dug grave
to steal his crosses and stars, they would tell me of their
victories over the Russians, and would pretend to sympathize with
my sorrow...'' thought Princess Mary, not thinking her own
thoughts but feeling bound to think like her father and her
brother. For herself she did not care where she remained or what
happened to her, but she felt herself the representative of her
dead father and of Prince Andrew. Involuntarily she thought their
thoughts and felt their feelings. What they would have said and
what they would have done she felt bound to say and do. She went
into Prince Andrew's study, trying to enter completely into his
ideas, and considered her position.

The demands of life, which had seemed to her annihilated by her
father's death, all at once rose before her with a new,
previously unknown force and took possession of her.

Agitated and flushed she paced the room, sending now for Michael
Ivanovich and now for Tikhon or Dron. Dunyasha, the nurse, and
the other maids could not say in how far Mademoiselle Bourienne's
statement was correct. Alpatych was not at home, he had gone to
the police. Neither could the architect Michael Ivanovich, who on
being sent for came in with sleepy eyes, tell Princess Mary
anything. With just the same smile of agreement with which for
fifteen years he had been accustomed to answer the old prince
without expressing views of his own, he now replied to Princess
Mary, so that nothing definite could be got from his answers. The
old valet Tikhon, with sunken, emaciated face that bore the stamp
of inconsolable grief, replied: ``Yes, Princess'' to all Princess
Mary's questions and hardly refrained from sobbing as he looked
at her.

At length Dron, the village Elder, entered the room and with a
deep bow to Princess Mary came to a halt by the doorpost.

Princess Mary walked up and down the room and stopped in front of
him.

``Dronushka,'' she said, regarding as a sure friend this
Dronushka who always used to bring a special kind of gingerbread
from his visit to the fair at Vyazma every year and smilingly
offer it to her, ``Dronushka, now since our misfortune...'' she
began, but could not go on.

``We are all in God's hands,'' said he, with a sigh.

They were silent for a while.

``Dronushka, Alpatych has gone off somewhere and I have no one to
turn to. Is it true, as they tell me, that I can't even go
away?''

``Why shouldn't you go away, your excellency? You can go,'' said
Dron.

``I was told it would be dangerous because of the enemy. Dear
friend, I can do nothing. I understand nothing. I have nobody! I
want to go away tonight or early tomorrow morning.''

Dron paused. He looked askance at Princess Mary and said: ``There
are no horses; I told Yakov Alpatych so.''

``Why are there none?'' asked the princess.

``It's all God's scourge,'' said Dron. ``What horses we had have
been taken for the army or have died---this is such a year! It's
not a case of feeding horses---we may die of hunger ourselves! As
it is, some go three days without eating. We've nothing, we've
been ruined.''

Princess Mary listened attentively to what he told her.

``The peasants are ruined? They have no bread?'' she asked.

``They're dying of hunger,'' said Dron. ``It's not a case of
carting.''

``But why didn't you tell me, Dronushka? Isn't it possible to
help them?  I'll do all I can...''

To Princess Mary it was strange that now, at a moment when such
sorrow was filling her soul, there could be rich people and poor,
and the rich could refrain from helping the poor. She had heard
vaguely that there was such a thing as \emph{landlord's corn}
which was sometimes given to the peasants. She also knew that
neither her father nor her brother would refuse to help the
peasants in need, she only feared to make some mistake in
speaking about the distribution of the grain she wished to
give. She was glad such cares presented themselves, enabling her
without scruple to forget her own grief. She began asking Dron
about the peasants' needs and what there was in Bogucharovo that
belonged to the landlord.

``But we have grain belonging to my brother?'' she said.

``The landlord's grain is all safe,'' replied Dron proudly. ``Our
prince did not order it to be sold.''

``Give it to the peasants, let them have all they need; I give
you leave in my brother's name,'' said she.

Dron made no answer but sighed deeply.

``Give them that corn if there is enough of it. Distribute it
all. I give this order in my brother's name; and tell them that
what is ours is theirs. We do not grudge them anything. Tell them
so.''

Dron looked intently at the princess while she was speaking.

``Discharge me, little mother, for God's sake! Order the keys to
be taken from me,'' said he. ``I have served twenty-three years
and have done no wrong. Discharge me, for God's sake!''

Princess Mary did not understand what he wanted of her or why he
was asking to be discharged. She replied that she had never
doubted his devotion and that she was ready to do anything for
him and for the peasants.

% % % % % % % % % % % % % % % % % % % % % % % % % % % % % % % % %
% % % % % % % % % % % % % % % % % % % % % % % % % % % % % % % % %
% % % % % % % % % % % % % % % % % % % % % % % % % % % % % % % % %
% % % % % % % % % % % % % % % % % % % % % % % % % % % % % % % % %
% % % % % % % % % % % % % % % % % % % % % % % % % % % % % % % % %
% % % % % % % % % % % % % % % % % % % % % % % % % % % % % % % % %
% % % % % % % % % % % % % % % % % % % % % % % % % % % % % % % % %
% % % % % % % % % % % % % % % % % % % % % % % % % % % % % % % % %
% % % % % % % % % % % % % % % % % % % % % % % % % % % % % % % % %
% % % % % % % % % % % % % % % % % % % % % % % % % % % % % % % % %
% % % % % % % % % % % % % % % % % % % % % % % % % % % % % % % % %
% % % % % % % % % % % % % % % % % % % % % % % % % % % % % %

\chapter*{Chapter XI} \ifaudio \marginpar{
\href{http://ia801407.us.archive.org/32/items/war_and_peace_10_0904_librivox/war_and_peace_10_11_tolstoy_64kb.mp3}{Audio}}
\fi

\initial{A}{n} hour later Dunyasha came to tell the princess that Dron had
come, and all the peasants had assembled at the barn by the
princess' order and wished to have word with their mistress.

``But I never told them to come,'' said Princess Mary. ``I only
told Dron to let them have the grain.''

``Only, for God's sake, Princess dear, have them sent away and
don't go out to them. It's all a trick,'' said Dunyasha, ``and
when Yakov Alpatych returns let us get away... and please
don't...''

``What is a trick?'' asked Princess Mary in surprise.

``I know it is, only listen to me for God's sake! Ask nurse
too. They say they don't agree to leave Bogucharovo as you
ordered.''

``You're making some mistake. I never ordered them to go away,''
said Princess Mary. ``Call Dronushka.''

Dron came and confirmed Dunyasha's words; the peasants had come
by the princess' order.

``But I never sent for them,'' declared the princess. ``You must
have given my message wrong. I only said that you were to give
them the grain.''

Dron only sighed in reply.

``If you order it they will go away,'' said he.

``No, no. I'll go out to them,'' said Princess Mary, and in spite
of the nurse's and Dunyasha's protests she went out into the
porch; Dron, Dunyasha, the nurse, and Michael Ivanovich following
her.

``They probably think I am offering them the grain to bribe them
to remain here, while I myself go away leaving them to the mercy
of the French,'' thought Princess Mary. ``I will offer them
monthly rations and housing at our Moscow estate. I am sure
Andrew would do even more in my place,'' she thought as she went
out in the twilight toward the crowd standing on the pasture by
the barn.

The men crowded closer together, stirred, and rapidly took off
their hats. Princess Mary lowered her eyes and, tripping over her
skirt, came close up to them. So many different eyes, old and
young, were fixed on her, and there were so many different faces,
that she could not distinguish any of them and, feeling that she
must speak to them all at once, did not know how to do it. But
again the sense that she represented her father and her brother
gave her courage, and she boldly began her speech.

``I am very glad you have come,'' she said without raising her
eyes, and feeling her heart beating quickly and
violently. ``Dronushka tells me that the war has ruined you. That
is our common misfortune, and I shall grudge nothing to help
you. I am myself going away because it is dangerous here... the
enemy is near... because... I am giving you everything, my
friends, and I beg you to take everything, all our grain, so that
you may not suffer want! And if you have been told that I am
giving you the grain to keep you here---that is not true. On the
contrary, I ask you to go with all your belongings to our estate
near Moscow, and I promise you I will see to it that there you
shall want for nothing. You shall be given food and lodging.''

The princess stopped. Sighs were the only sound heard in the
crowd.

``I am not doing this on my own account,'' she continued, ``I do
it in the name of my dead father, who was a good master to you,
and of my brother and his son.''

Again she paused. No one broke the silence.

``Ours is a common misfortune and we will share it together. All
that is mine is yours,'' she concluded, scanning the faces before
her.

All eyes were gazing at her with one and the same expression. She
could not fathom whether it was curiosity, devotion, gratitude,
or apprehension and distrust---but the expression on all the
faces was identical.

``We are all very thankful for your bounty, but it won't do for
us to take the landlord's grain,'' said a voice at the back of
the crowd.

``But why not?'' asked the princess.

No one replied and Princess Mary, looking round at the crowd,
found that every eye she met now was immediately dropped.

``But why don't you want to take it?'' she asked again.

No one answered.

The silence began to oppress the princess and she tried to catch
someone's eye.

``Why don't you speak?'' she inquired of a very old man who stood
just in front of her leaning on his stick. ``If you think
something more is wanted, tell me! I will do anything,'' said
she, catching his eye.

But as if this angered him, he bent his head quite low and
muttered:

``Why should we agree? We don't want the grain.''

``Why should we give up everything? We don't agree. Don't
agree... We are sorry for you, but we're not willing. Go away
yourself, alone...''  came from various sides of the crowd.

And again all the faces in that crowd bore an identical
expression, though now it was certainly not an expression of
curiosity or gratitude, but of angry resolve.

``But you can't have understood me,'' said Princess Mary with a
sad smile.  ``Why don't you want to go? I promise to house and
feed you, while here the enemy would ruin you...''

But her voice was drowned by the voices of the crowd.

``We're not willing. Let them ruin us! We won't take your
grain. We don't agree.''

Again Princess Mary tried to catch someone's eye, but not a
single eye in the crowd was turned to her; evidently they were
all trying to avoid her look. She felt strange and awkward.

``Oh yes, an artful tale! Follow her into slavery! Pull down your
houses and go into bondage! I dare say! 'I'll give you grain,
indeed!' she says,'' voices in the crowd were heard saying.

With drooping head Princess Mary left the crowd and went back to
the house. Having repeated her order to Dron to have horses ready
for her departure next morning, she went to her room and remained
alone with her own thoughts.

% % % % % % % % % % % % % % % % % % % % % % % % % % % % % % % % %
% % % % % % % % % % % % % % % % % % % % % % % % % % % % % % % % %
% % % % % % % % % % % % % % % % % % % % % % % % % % % % % % % % %
% % % % % % % % % % % % % % % % % % % % % % % % % % % % % % % % %
% % % % % % % % % % % % % % % % % % % % % % % % % % % % % % % % %
% % % % % % % % % % % % % % % % % % % % % % % % % % % % % % % % %
% % % % % % % % % % % % % % % % % % % % % % % % % % % % % % % % %
% % % % % % % % % % % % % % % % % % % % % % % % % % % % % % % % %
% % % % % % % % % % % % % % % % % % % % % % % % % % % % % % % % %
% % % % % % % % % % % % % % % % % % % % % % % % % % % % % % % % %
% % % % % % % % % % % % % % % % % % % % % % % % % % % % % % % % %
% % % % % % % % % % % % % % % % % % % % % % % % % % % % % %

\chapter*{Chapter XII} \ifaudio \marginpar{
\href{http://ia801407.us.archive.org/32/items/war_and_peace_10_0904_librivox/war_and_peace_10_12_tolstoy_64kb.mp3}{Audio}}
\fi

\initial{F}{or} a long time that night Princess Mary sat by the open window
of her room hearing the sound of the peasants' voices that
reached her from the village, but it was not of them she was
thinking. She felt that she could not understand them however
much she might think about them. She thought only of one thing,
her sorrow, which, after the break caused by cares for the
present, seemed already to belong to the past. Now she could
remember it and weep or pray.

After sunset the wind had dropped. The night was calm and
fresh. Toward midnight the voices began to subside, a cock
crowed, the full moon began to show from behind the lime trees, a
fresh white dewy mist began to rise, and stillness reigned over
the village and the house.

Pictures of the near past---her father's illness and last
moments---rose one after another to her memory. With mournful
pleasure she now lingered over these images, repelling with
horror only the last one, the picture of his death, which she
felt she could not contemplate even in imagination at this still
and mystic hour of night. And these pictures presented themselves
to her so clearly and in such detail that they seemed now
present, now past, and now future.

She vividly recalled the moment when he had his first stroke and
was being dragged along by his armpits through the garden at Bald
Hills, muttering something with his helpless tongue, twitching
his gray eyebrows and looking uneasily and timidly at her.

``Even then he wanted to tell me what he told me the day he
died,'' she thought. ``He had always thought what he said then.''
And she recalled in all its detail the night at Bald Hills before
he had the last stroke, when with a foreboding of disaster she
had remained at home against his will. She had not slept and had
stolen downstairs on tiptoe, and going to the door of the
conservatory where he slept that night had listened at the
door. In a suffering and weary voice he was saying something to
Tikhon, speaking of the Crimea and its warm nights and of the
Empress.  Evidently he had wanted to talk. ``And why didn't he
call me? Why didn't he let me be there instead of Tikhon?''
Princess Mary had thought and thought again now. ``Now he will
never tell anyone what he had in his soul. Never will that moment
return for him or for me when he might have said all he longed to
say, and not Tikhon but I might have heard and understood
him. Why didn't I enter the room?'' she thought. ``Perhaps he
would then have said to me what he said the day he died. While
talking to Tikhon he asked about me twice. He wanted to see me,
and I was standing close by, outside the door. It was sad and
painful for him to talk to Tikhon who did not understand him. I
remember how he began speaking to him about Lise as if she were
alive---he had forgotten she was dead---and Tikhon reminded him
that she was no more, and he shouted, 'Fool!' He was greatly
depressed. From behind the door I heard how he lay down on his
bed groaning and loudly exclaimed, 'My God!' Why didn't I go in
then? What could he have done to me? What could I have lost? And
perhaps he would then have been comforted and would have said
that word to me.'' And Princess Mary uttered aloud the caressing
word he had said to her on the day of his death. ``Dear-est!''
she repeated, and began sobbing, with tears that relieved her
soul. She now saw his face before her. And not the face she had
known ever since she could remember and had always seen at a
distance, but the timid, feeble face she had seen for the first
time quite closely, with all its wrinkles and details, when she
stooped near to his mouth to catch what he said.

``Dear-est!'' she repeated again.

``What was he thinking when he uttered that word? What is he
thinking now?'' This question suddenly presented itself to her,
and in answer she saw him before her with the expression that was
on his face as he lay in his coffin with his chin bound up with a
white handkerchief. And the horror that had seized her when she
touched him and convinced herself that that was not he, but
something mysterious and horrible, seized her again. She tried to
think of something else and to pray, but could do neither. With
wide-open eyes she gazed at the moonlight and the shadows,
expecting every moment to see his dead face, and she felt that
the silence brooding over the house and within it held her fast.

``Dunyasha,'' she whispered. ``Dunyasha!'' she screamed wildly,
and tearing herself out of this silence she ran to the servants'
quarters to meet her old nurse and the maidservants who came
running toward her.

% % % % % % % % % % % % % % % % % % % % % % % % % % % % % % % % %
% % % % % % % % % % % % % % % % % % % % % % % % % % % % % % % % %
% % % % % % % % % % % % % % % % % % % % % % % % % % % % % % % % %
% % % % % % % % % % % % % % % % % % % % % % % % % % % % % % % % %
% % % % % % % % % % % % % % % % % % % % % % % % % % % % % % % % %
% % % % % % % % % % % % % % % % % % % % % % % % % % % % % % % % %
% % % % % % % % % % % % % % % % % % % % % % % % % % % % % % % % %
% % % % % % % % % % % % % % % % % % % % % % % % % % % % % % % % %
% % % % % % % % % % % % % % % % % % % % % % % % % % % % % % % % %
% % % % % % % % % % % % % % % % % % % % % % % % % % % % % % % % %
% % % % % % % % % % % % % % % % % % % % % % % % % % % % % % % % %
% % % % % % % % % % % % % % % % % % % % % % % % % % % % % %

\chapter*{Chapter XIII} \ifaudio \marginpar{
\href{http://ia801407.us.archive.org/32/items/war_and_peace_10_0904_librivox/war_and_peace_10_13_tolstoy_64kb.mp3}{Audio}}
\fi

\initial{O}{n} the seventeenth of August Rostov and Ilyin, accompanied by
Lavrushka who had just returned from captivity and by an hussar
orderly, left their quarters at Yankovo, ten miles from
Bogucharovo, and went for a ride---to try a new horse Ilyin had
bought and to find out whether there was any hay to be had in the
villages.

For the last three days Bogucharovo had lain between the two
hostile armies, so that it was as easy for the Russian rearguard
to get to it as for the French vanguard; Rostov, as a careful
squadron commander, wished to take such provisions as remained at
Bogucharovo before the French could get them.

Rostov and Ilyin were in the merriest of moods. On the way to
Bogucharovo, a princely estate with a dwelling house and farm
where they hoped to find many domestic serfs and pretty girls,
they questioned Lavrushka about Napoleon and laughed at his
stories, and raced one another to try Ilyin's horse.

Rostov had no idea that the village he was entering was the
property of that very Bolkonski who had been engaged to his
sister.

Rostov and Ilyin gave rein to their horses for a last race along
the incline before reaching Bogucharovo, and Rostov, outstripping
Ilyin, was the first to gallop into the village street.

``You're first!'' cried Ilyin, flushed.

``Yes, always first both on the grassland and here,'' answered
Rostov, stroking his heated Donets horse.

``And I'd have won on my Frenchy, your excellency,'' said
Lavrushka from behind, alluding to his shabby cart horse, ``only
I didn't wish to mortify you.''

They rode at a footpace to the barn, where a large crowd of
peasants was standing.

Some of the men bared their heads, others stared at the new
arrivals without doffing their caps. Two tall old peasants with
wrinkled faces and scanty beards emerged from the tavern,
smiling, staggering, and singing some incoherent song, and
approached the officers.

``Fine fellows!'' said Rostov laughing. ``Is there any hay
here?''

``And how like one another,'' said Ilyin.

``A mo-o-st me-r-r-y co-o-m-pa...!'' sang one of the peasants
with a blissful smile.

One of the men came out of the crowd and went up to Rostov.

``Who do you belong to?'' he asked.

``The French,'' replied Ilyin jestingly, ``and here is Napoleon
himself''---and he pointed to Lavrushka.

``Then you are Russians?'' the peasant asked again.

``And is there a large force of you here?'' said another, a short
man, coming up.

``Very large,'' answered Rostov. ``But why have you collected
here?'' he added. ``Is it a holiday?''

``The old men have met to talk over the business of the
commune,'' replied the peasant, moving away.

At that moment, on the road leading from the big house, two women
and a man in a white hat were seen coming toward the officers.

``The one in pink is mine, so keep off!'' said Ilyin on seeing
Dunyasha running resolutely toward him.

``She'll be ours!'' said Lavrushka to Ilyin, winking.

``What do you want, my pretty?'' said Ilyin with a smile.

``The princess ordered me to ask your regiment and your name.''

``This is Count Rostov, squadron commander, and I am your humble
servant.''

``Co-o-om-pa-ny!'' roared the tipsy peasant with a beatific smile
as he looked at Ilyin talking to the girl. Following Dunyasha,
Alpatych advanced to Rostov, having bared his head while still at
a distance.

``May I make bold to trouble your honor?'' said he respectfully,
but with a shade of contempt for the youthfulness of this officer
and with a hand thrust into his bosom. ``My mistress, daughter of
General in Chief Prince Nicholas Bolkonski who died on the
fifteenth of this month, finding herself in difficulties owing to
the boorishness of these people''---he pointed to the
peasants---``asks you to come up to the house... Won't you,
please, ride on a little farther,'' said Alpatych with a
melancholy smile, ``as it is not convenient in the presence
of...?'' He pointed to the two peasants who kept as close to him
as horseflies to a horse.

``Ah!... Alpatych... Ah, Yakov Alpatych... Grand! Forgive us for
Christ's sake, eh?'' said the peasants, smiling joyfully at him.

Rostov looked at the tipsy peasants and smiled.

``Or perhaps they amuse your honor?'' remarked Alpatych with a
staid air, as he pointed at the old men with his free hand.

``No, there's not much to be amused at here,'' said Rostov, and
rode on a little way. ``What's the matter?'' he asked.

``I make bold to inform your honor that the rude peasants here
don't wish to let the mistress leave the estate, and threaten to
unharness her horses, so that though everything has been packed
up since morning, her excellency cannot get away.''

``Impossible!'' exclaimed Rostov.

``I have the honor to report to you the actual truth,'' said
Alpatych.

Rostov dismounted, gave his horse to the orderly, and followed
Alpatych to the house, questioning him as to the state of
affairs. It appeared that the princess' offer of corn to the
peasants the previous day, and her talk with Dron and at the
meeting, had actually had so bad an effect that Dron had finally
given up the keys and joined the peasants and had not appeared
when Alpatych sent for him; and that in the morning when the
princess gave orders to harness for her journey, the peasants had
come in a large crowd to the barn and sent word that they would
not let her leave the village: that there was an order not to
move, and that they would unharness the horses. Alpatych had gone
out to admonish them, but was told (it was chiefly Karp who did
the talking, Dron not showing himself in the crowd) that they
could not let the princess go, that there was an order to the
contrary, but that if she stayed they would serve her as before
and obey her in everything.

At the moment when Rostov and Ilyin were galloping along the
road, Princess Mary, despite the dissuasions of Alpatych, her
nurse, and the maids, had given orders to harness and intended to
start, but when the cavalrymen were espied they were taken for
Frenchmen, the coachman ran away, and the women in the house
began to wail.

``Father! Benefactor! God has sent you!'' exclaimed deeply moved
voices as Rostov passed through the anteroom.

Princess Mary was sitting helpless and bewildered in the large
sitting room, when Rostov was shown in. She could not grasp who
he was and why he had come, or what was happening to her. When
she saw his Russian face, and by his walk and the first words he
uttered recognized him as a man of her own class, she glanced at
him with her deep radiant look and began speaking in a voice that
faltered and trembled with emotion. This meeting immediately
struck Rostov as a romantic event. ``A helpless girl overwhelmed
with grief, left to the mercy of coarse, rioting peasants!  And
what a strange fate sent me here! What gentleness and nobility
there are in her features and expression!'' thought he as he
looked at her and listened to her timid story.

When she began to tell him that all this had happened the day
after her father's funeral, her voice trembled. She turned away,
and then, as if fearing he might take her words as meant to move
him to pity, looked at him with an apprehensive glance of
inquiry. There were tears in Rostov's eyes. Princess Mary noticed
this and glanced gratefully at him with that radiant look which
caused the plainness of her face to be forgotten.

``I cannot express, Princess, how glad I am that I happened to
ride here and am able to show my readiness to serve you,'' said
Rostov, rising. ``Go when you please, and I give you my word of
honor that no one shall dare to cause you annoyance if only you
will allow me to act as your escort.''  And bowing respectfully,
as if to a lady of royal blood, he moved toward the door.

Rostov's deferential tone seemed to indicate that though he would
consider himself happy to be acquainted with her, he did not wish
to take advantage of her misfortunes to intrude upon her.

Princess Mary understood this and appreciated his delicacy.

``I am very, very grateful to you,'' she said in French, ``but I
hope it was all a misunderstanding and that no one is to blame
for it.'' She suddenly began to cry.

``Excuse me!'' she said.

Rostov, knitting his brows, left the room with another low bow.

% % % % % % % % % % % % % % % % % % % % % % % % % % % % % % % % %
% % % % % % % % % % % % % % % % % % % % % % % % % % % % % % % % %
% % % % % % % % % % % % % % % % % % % % % % % % % % % % % % % % %
% % % % % % % % % % % % % % % % % % % % % % % % % % % % % % % % %
% % % % % % % % % % % % % % % % % % % % % % % % % % % % % % % % %
% % % % % % % % % % % % % % % % % % % % % % % % % % % % % % % % %
% % % % % % % % % % % % % % % % % % % % % % % % % % % % % % % % %
% % % % % % % % % % % % % % % % % % % % % % % % % % % % % % % % %
% % % % % % % % % % % % % % % % % % % % % % % % % % % % % % % % %
% % % % % % % % % % % % % % % % % % % % % % % % % % % % % % % % %
% % % % % % % % % % % % % % % % % % % % % % % % % % % % % % % % %
% % % % % % % % % % % % % % % % % % % % % % % % % % % % % %

\chapter*{Chapter XIV} \ifaudio \marginpar{
\href{http://ia801407.us.archive.org/32/items/war_and_peace_10_0904_librivox/war_and_peace_10_14_tolstoy_64kb.mp3}{Audio}}
\fi

\initial*{W}{ell}, is she pretty? Ah, friend---my pink one is delicious; her
name is Dunyasha...''

But on glancing at Rostov's face Ilyin stopped short. He saw that
his hero and commander was following quite a different train of
thought.

Rostov glanced angrily at Ilyin and without replying strode off
with rapid steps to the village.

``I'll show them; I'll give it to them, the brigands!'' said he
to himself.

Alpatych at a gliding trot, only just managing not to run, kept
up with him with difficulty.

``What decision have you been pleased to come to?'' said he.

Rostov stopped and, clenching his fists, suddenly and sternly
turned on Alpatych.

``Decision? What decision? Old dotard!...'' cried he. ``What have
you been about? Eh? The peasants are rioting, and you can't
manage them? You're a traitor yourself! I know you. I'll flay you
all alive!...'' And as if afraid of wasting his store of anger,
he left Alpatych and went rapidly forward. Alpatych, mastering
his offended feelings, kept pace with Rostov at a gliding gait
and continued to impart his views. He said the peasants were
obdurate and that at the present moment it would be imprudent to
``overresist'' them without an armed force, and would it not be
better first to send for the military?

``I'll give them armed force... I'll 'overresist' them!'' uttered
Rostov meaninglessly, breathless with irrational animal fury and
the need to vent it.

Without considering what he would do he moved unconciously with
quick, resolute steps toward the crowd. And the nearer he drew to
it the more Alpatych felt that this unreasonable action might
produce good results.  The peasants in the crowd were similarly
impressed when they saw Rostov's rapid, firm steps and resolute,
frowning face.

After the hussars had come to the village and Rostov had gone to
see the princess, a certain confusion and dissension had arisen
among the crowd.  Some of the peasants said that these new
arrivals were Russians and might take it amiss that the mistress
was being detained. Dron was of this opinion, but as soon as he
expressed it Karp and others attacked their ex-Elder.

``How many years have you been fattening on the commune?'' Karp
shouted at him. ``It's all one to you! You'll dig up your pot of
money and take it away with you... What does it matter to you
whether our homes are ruined or not?''

``We've been told to keep order, and that no one is to leave
their homes or take away a single grain, and that's all about
it!'' cried another.

``It was your son's turn to be conscripted, but no fear! You
begrudged your lump of a son,'' a little old man suddenly began
attacking Dron---``and so they took my Vanka to be shaved for a
soldier! But we all have to die.''

``To be sure, we all have to die. I'm not against the commune,''
said Dron.

``That's it---not against it! You've filled your belly...''

The two tall peasants had their say. As soon as Rostov, followed
by Ilyin, Lavrushka, and Alpatych, came up to the crowd, Karp,
thrusting his fingers into his belt and smiling a little, walked
to the front.  Dron on the contrary retired to the rear and the
crowd drew closer together.

``Who is your Elder here? Hey?'' shouted Rostov, coming up to the
crowd with quick steps.

``The Elder? What do you want with him?...'' asked Karp.

But before the words were well out of his mouth, his cap flew off
and a fierce blow jerked his head to one side.

``Caps off, traitors!'' shouted Rostov in a wrathful
voice. ``Where's the Elder?'' he cried furiously.

``The Elder... He wants the Elder!... Dron Zakharych, you!'' meek
and flustered voices here and there were heard calling and caps
began to come off their heads.

``We don't riot, we're following the orders,'' declared Karp, and
at that moment several voices began speaking together.

``It's as the old men have decided---there's too many of you
giving orders.''

``Arguing? Mutiny!... Brigands! Traitors!'' cried Rostov
unmeaningly in a voice not his own, gripping Karp by the
collar. ``Bind him, bind him!'' he shouted, though there was no
one to bind him but Lavrushka and Alpatych.

Lavrushka, however, ran up to Karp and seized him by the arms
from behind.

``Shall I call up our men from beyond the hill?'' he called out.

Alpatych turned to the peasants and ordered two of them by name
to come and bind Karp. The men obediently came out of the crowd
and began taking off their belts.

``Where's the Elder?'' demanded Rostov in a loud voice.

With a pale and frowning face Dron stepped out of the crowd.

``Are you the Elder? Bind him, Lavrushka!'' shouted Rostov, as if
that order, too, could not possibly meet with any opposition.

And in fact two more peasants began binding Dron, who took off
his own belt and handed it to them, as if to aid them.

``And you all listen to me!'' said Rostov to the peasants. ``Be
off to your houses at once, and don't let one of your voices be
heard!''

``Why, we've not done any harm! We did it just out of
foolishness. It's all nonsense... I said then that it was not in
order,'' voices were heard bickering with one another.

``There! What did I say?'' said Alpatych, coming into his own
again. ``It's wrong, lads!''

``All our stupidity, Yakov Alpatych,'' came the answers, and the
crowd began at once to disperse through the village.

The two bound men were led off to the master's house. The two
drunken peasants followed them.

``Aye, when I look at you!...'' said one of them to Karp.

``How can one talk to the masters like that? What were you
thinking of, you fool?'' added the other---``A real fool!''

Two hours later the carts were standing in the courtyard of the
Bogucharovo house. The peasants were briskly carrying out the
proprietor's goods and packing them on the carts, and Dron,
liberated at Princess Mary's wish from the cupboard where he had
been confined, was standing in the yard directing the men.

``Don't put it in so carelessly,'' said one of the peasants, a
man with a round smiling face, taking a casket from a
housemaid. ``You know it has cost money! How can you chuck it in
like that or shove it under the cord where it'll get rubbed? I
don't like that way of doing things. Let it all be done properly,
according to rule. Look here, put it under the bast matting and
cover it with hay---that's the way!''

``Eh, books, books!'' said another peasant, bringing out Prince
Andrew's library cupboards. ``Don't catch up against it! It's
heavy, lads---solid books.''

``Yes, they worked all day and didn't play!'' remarked the tall,
round-faced peasant gravely, pointing with a significant wink at
the dictionaries that were on the top.

Unwilling to obtrude himself on the princess, Rostov did not go
back to the house but remained in the village awaiting her
departure. When her carriage drove out of the house, he mounted
and accompanied her eight miles from Bogucharovo to where the
road was occupied by our troops. At the inn at Yankovo he
respectfully took leave of her, for the first time permitting
himself to kiss her hand.

``How can you speak so!'' he blushingly replied to Princess
Mary's expressions of gratitude for her deliverance, as she
termed what had occurred. ``Any police officer would have done as
much! If we had had only peasants to fight, we should not have
let the enemy come so far,'' said he with a sense of shame and
wishing to change the subject. ``I am only happy to have had the
opportunity of making your acquaintance.  Good-bye, Princess. I
wish you happiness and consolation and hope to meet you again in
happier circumstances. If you don't want to make me blush, please
don't thank me!''

But the princess, if she did not again thank him in words,
thanked him with the whole expression of her face, radiant with
gratitude and tenderness. She could not believe that there was
nothing to thank him for. On the contrary, it seemed to her
certain that had he not been there she would have perished at the
hands of the mutineers and of the French, and that he had exposed
himself to terrible and obvious danger to save her, and even more
certain was it that he was a man of lofty and noble soul, able to
understand her position and her sorrow. His kind, honest eyes,
with the tears rising in them when she herself had begun to cry
as she spoke of her loss, did not leave her memory.

When she had taken leave of him and remained alone she suddenly
felt her eyes filling with tears, and then not for the first time
the strange question presented itself to her: did she love him?

On the rest of the way to Moscow, though the princess' position
was not a cheerful one, Dunyasha, who went with her in the
carriage, more than once noticed that her mistress leaned out of
the window and smiled at something with an expression of mingled
joy and sorrow.

``Well, supposing I do love him?'' thought Princess Mary.

Ashamed as she was of acknowledging to herself that she had
fallen in love with a man who would perhaps never love her, she
comforted herself with the thought that no one would ever know it
and that she would not be to blame if, without ever speaking of
it to anyone, she continued to the end of her life to love the
man with whom she had fallen in love for the first and last time
in her life.

Sometimes when she recalled his looks, his sympathy, and his
words, happiness did not appear impossible to her. It was at
those moments that Dunyasha noticed her smiling as she looked out
of the carriage window.

``Was it not fate that brought him to Bogucharovo, and at that
very moment?'' thought Princess Mary. ``And that caused his
sister to refuse my brother?'' And in all this Princess Mary saw
the hand of Providence.

The impression the princess made on Rostov was a very agreeable
one. To remember her gave him pleasure, and when his comrades,
hearing of his adventure at Bogucharovo, rallied him on having
gone to look for hay and having picked up one of the wealthiest
heiresses in Russia, he grew angry. It made him angry just
because the idea of marrying the gentle Princess Mary, who was
attractive to him and had an enormous fortune, had against his
will more than once entered his head. For himself personally
Nicholas could not wish for a better wife: by marrying her he
would make the countess his mother happy, would be able to put
his father's affairs in order, and would even---he felt
it---ensure Princess Mary's happiness.

But Sonya? And his plighted word? That was why Rostov grew angry
when he was rallied about Princess Bolkonskaya.

% % % % % % % % % % % % % % % % % % % % % % % % % % % % % % % % %
% % % % % % % % % % % % % % % % % % % % % % % % % % % % % % % % %
% % % % % % % % % % % % % % % % % % % % % % % % % % % % % % % % %
% % % % % % % % % % % % % % % % % % % % % % % % % % % % % % % % %
% % % % % % % % % % % % % % % % % % % % % % % % % % % % % % % % %
% % % % % % % % % % % % % % % % % % % % % % % % % % % % % % % % %
% % % % % % % % % % % % % % % % % % % % % % % % % % % % % % % % %
% % % % % % % % % % % % % % % % % % % % % % % % % % % % % % % % %
% % % % % % % % % % % % % % % % % % % % % % % % % % % % % % % % %
% % % % % % % % % % % % % % % % % % % % % % % % % % % % % % % % %
% % % % % % % % % % % % % % % % % % % % % % % % % % % % % % % % %
% % % % % % % % % % % % % % % % % % % % % % % % % % % % % %

\chapter*{Chapter XV} \ifaudio \marginpar{
\href{http://ia801407.us.archive.org/32/items/war_and_peace_10_0904_librivox/war_and_peace_10_15_tolstoy_64kb.mp3}{Audio}}
\fi

\initial{O}{n} receiving command of the armies Kutuzov remembered Prince
Andrew and sent an order for him to report at headquarters.

Prince Andrew arrived at Tsarevo-Zaymishche on the very day and
at the very hour that Kutuzov was reviewing the troops for the
first time. He stopped in the village at the priest's house in
front of which stood the commander-in-chief's carriage, and he
sat down on the bench at the gate awaiting his Serene Highness,
as everyone now called Kutuzov. From the field beyond the village
came now sounds of regimental music and now the roar of many
voices shouting ``Hurrah!'' to the new commander-in-chief.  Two
orderlies, a courier and a major-domo, stood near by, some ten
paces from Prince Andrew, availing themselves of Kutuzov's
absence and of the fine weather. A short, swarthy lieutenant
colonel of hussars with thick mustaches and whiskers rode up to
the gate and, glancing at Prince Andrew, inquired whether his
Serene Highness was putting up there and whether he would soon be
back.

Prince Andrew replied that he was not on his Serene Highness'
staff but was himself a new arrival. The lieutenant colonel
turned to a smart orderly, who, with the peculiar contempt with
which a commander-in-chief's orderly speaks to officers, replied:

``What? His Serene Highness? I expect he'll be here soon. What do
you want?''

The lieutenant colonel of hussars smiled beneath his mustache at
the orderly's tone, dismounted, gave his horse to a dispatch
runner, and approached Bolkonski with a slight bow. Bolkonski
made room for him on the bench and the lieutenant colonel sat
down beside him.

``You're also waiting for the commander-in-chief?'' said
he. ``They say he weceives evewyone, thank God!... It's awful
with those sausage eaters!  Ermolov had weason to ask to be
pwomoted to be a German! Now p'waps Wussians will get a look
in. As it was, devil only knows what was happening. We kept
wetweating and wetweating. Did you take part in the campaign?''
he asked.

``I had the pleasure,'' replied Prince Andrew, ``not only of
taking part in the retreat but of losing in that retreat all I
held dear---not to mention the estate and home of my birth---my
father, who died of grief. I belong to the province of
Smolensk.''

``Ah? You're Pwince Bolkonski? Vewy glad to make your
acquaintance! I'm Lieutenant Colonel Denisov, better known as
'Vaska,'{}'' said Denisov, pressing Prince Andrew's hand and
looking into his face with a particularly kindly
attention. ``Yes, I heard,'' said he sympathetically, and after a
short pause added: ``Yes, it's Scythian warfare. It's all vewy
well---only not for those who get it in the neck. So you are
Pwince Andwew Bolkonski?'' He swayed his head. ``Vewy pleased,
Pwince, to make your acquaintance!'' he repeated again, smiling
sadly, and he again pressed Prince Andrew's hand.

Prince Andrew knew Denisov from what Natasha had told him of her
first suitor. This memory carried him sadly and sweetly back to
those painful feelings of which he had not thought lately, but
which still found place in his soul. Of late he had received so
many new and very serious impressions---such as the retreat from
Smolensk, his visit to Bald Hills, and the recent news of his
father's death---and had experienced so many emotions, that for a
long time past those memories had not entered his mind, and now
that they did, they did not act on him with nearly their former
strength. For Denisov, too, the memories awakened by the name of
Bolkonski belonged to a distant, romantic past, when after supper
and after Natasha's singing he had proposed to a little girl of
fifteen without realizing what he was doing. He smiled at the
recollection of that time and of his love for Natasha, and passed
at once to what now interested him passionately and
exclusively. This was a plan of campaign he had devised while
serving at the outposts during the retreat. He had proposed that
plan to Barclay de Tolly and now wished to propose it to
Kutuzov. The plan was based on the fact that the French line of
operation was too extended, and it proposed that instead of, or
concurrently with, action on the front to bar the advance of the
French, we should attack their line of communication. He began
explaining his plan to Prince Andrew.

``They can't hold all that line. It's impossible. I will
undertake to bweak thwough. Give me five hundwed men and I will
bweak the line, that's certain! There's only one way---guewilla
warfare!''

Denisov rose and began gesticulating as he explained his plan to
Bolkonski. In the midst of his explanation shouts were heard from
the army, growing more incoherent and more diffused, mingling
with music and songs and coming from the field where the review
was held. Sounds of hoofs and shouts were nearing the village.

``He's coming! He's coming!'' shouted a Cossack standing at the
gate.

Bolkonski and Denisov moved to the gate, at which a knot of
soldiers (a guard of honor) was standing, and they saw Kutuzov
coming down the street mounted on a rather small sorrel horse. A
huge suite of generals rode behind him. Barclay was riding almost
beside him, and a crowd of officers ran after and around them
shouting, ``Hurrah!''

His adjutants galloped into the yard before him. Kutuzov was
impatiently urging on his horse, which ambled smoothly under his
weight, and he raised his hand to his white Horse Guard's cap
with a red band and no peak, nodding his head continually. When
he came up to the guard of honor, a fine set of Grenadiers mostly
wearing decorations, who were giving him the salute, he looked at
them silently and attentively for nearly a minute with the steady
gaze of a commander and then turned to the crowd of generals and
officers surrounding him. Suddenly his face assumed a subtle
expression, he shrugged his shoulders with an air of perplexity.

``And with such fine fellows to retreat and retreat! Well,
good-by, General,'' he added, and rode into the yard past Prince
Andrew and Denisov.

``Hurrah! hurrah! hurrah!'' shouted those behind him.

Since Prince Andrew had last seen him Kutuzov had grown still
more corpulent, flaccid, and fat. But the bleached eyeball, the
scar, and the familiar weariness of his expression were still the
same. He was wearing the white Horse Guard's cap and a military
overcoat with a whip hanging over his shoulder by a thin
strap. He sat heavily and swayed limply on his brisk little
horse.

``Whew... whew... whew!'' he whistled just audibly as he rode
into the yard. His face expressed the relief of relaxed strain
felt by a man who means to rest after a ceremony. He drew his
left foot out of the stirrup and, lurching with his whole body
and puckering his face with the effort, raised it with difficulty
onto the saddle, leaned on his knee, groaned, and slipped down
into the arms of the Cossacks and adjutants who stood ready to
assist him.

He pulled himself together, looked round, screwing up his eyes,
glanced at Prince Andrew, and, evidently not recognizing him,
moved with his waddling gait to the
porch. ``Whew... whew... whew!'' he whistled, and again glanced
at Prince Andrew. As often occurs with old men, it was only after
some seconds that the impression produced by Prince Andrew's face
linked itself up with Kutuzov's remembrance of his personality.

``Ah, how do you do, my dear prince? How do you do, my dear boy?
Come along...'' said he, glancing wearily round, and he stepped
onto the porch which creaked under his weight.

He unbuttoned his coat and sat down on a bench in the porch.

``And how's your father?''

``I received news of his death, yesterday,'' replied Prince
Andrew abruptly.

Kutuzov looked at him with eyes wide open with dismay and then
took off his cap and crossed himself:

``May the kingdom of Heaven be his! God's will be done to us
all!'' He sighed deeply, his whole chest heaving, and was silent
for a while. ``I loved him and respected him, and sympathize with
you with all my heart.''

He embraced Prince Andrew, pressing him to his fat breast, and
for some time did not let him go. When he released him Prince
Andrew saw that Kutuzov's flabby lips were trembling and that
tears were in his eyes. He sighed and pressed on the bench with
both hands to raise himself.

``Come! Come with me, we'll have a talk,'' said he.

But at that moment Denisov, no more intimidated by his superiors
than by the enemy, came with jingling spurs up the steps of the
porch, despite the angry whispers of the adjutants who tried to
stop him. Kutuzov, his hands still pressed on the seat, glanced
at him glumly. Denisov, having given his name, announced that he
had to communicate to his Serene Highness a matter of great
importance for their country's welfare.  Kutuzov looked wearily
at him and, lifting his hands with a gesture of annoyance, folded
them across his stomach, repeating the words: ``For our country's
welfare? Well, what is it? Speak!'' Denisov blushed like a girl
(it was strange to see the color rise in that shaggy, bibulous,
time-worn face) and boldly began to expound his plan of cutting
the enemy's lines of communication between Smolensk and
Vyazma. Denisov came from those parts and knew the country
well. His plan seemed decidedly a good one, especially from the
strength of conviction with which he spoke.  Kutuzov looked down
at his own legs, occasionally glancing at the door of the
adjoining hut as if expecting something unpleasant to emerge from
it. And from that hut, while Denisov was speaking, a general with
a portfolio under his arm really did appear.

``What?'' said Kutuzov, in the midst of Denisov's explanations,
``are you ready so soon?''

``Ready, your Serene Highness,'' replied the general.

Kutuzov swayed his head, as much as to say: ``How is one man to
deal with it all?'' and again listened to Denisov.

``I give my word of honor as a Wussian officer,'' said Denisov,
``that I can bweak Napoleon's line of communication!''

``What relation are you to Intendant General Kiril Andreevich
Denisov?''  asked Kutuzov, interrupting him.

``He is my uncle, your Sewene Highness.''

``Ah, we were friends,'' said Kutuzov cheerfully. ``All right,
all right, friend, stay here at the staff and tomorrow we'll have
a talk.''

With a nod to Denisov he turned away and put out his hand for the
papers Konovnitsyn had brought him.

``Would not your Serene Highness like to come inside?'' said the
general on duty in a discontented voice, ``the plans must be
examined and several papers have to be signed.''

An adjutant came out and announced that everything was in
readiness within. But Kutuzov evidently did not wish to enter
that room till he was disengaged. He made a grimace...

``No, tell them to bring a small table out here, my dear
boy. I'll look at them here,'' said he. ``Don't go away,'' he
added, turning to Prince Andrew, who remained in the porch and
listened to the general's report.

While this was being given, Prince Andrew heard the whisper of a
woman's voice and the rustle of a silk dress behind the
door. Several times on glancing that way he noticed behind that
door a plump, rosy, handsome woman in a pink dress with a lilac
silk kerchief on her head, holding a dish and evidently awaiting
the entrance of the commander-in-chief.  Kutuzov's adjutant
whispered to Prince Andrew that this was the wife of the priest
whose home it was, and that she intended to offer his Serene
Highness bread and salt. ``Her husband has welcomed his Serene
Highness with the cross at the church, and she intends to welcome
him in the house... She's very pretty,'' added the adjutant with
a smile. At those words Kutuzov looked round. He was listening to
the general's report---which consisted chiefly of a criticism of
the position at Tsarevo-Zaymishche---as he had listened to
Denisov, and seven years previously had listened to the
discussion at the Austerlitz council of war. He evidently
listened only because he had ears which, though there was a piece
of tow in one of them, could not help hearing; but it was evident
that nothing the general could say would surprise or even
interest him, that he knew all that would be said beforehand, and
heard it all only because he had to, as one has to listen to the
chanting of a service of prayer. All that Denisov had said was
clever and to the point. What the general was saying was even
more clever and to the point, but it was evident that Kutuzov
despised knowledge and cleverness, and knew of something else
that would decide the matter---something independent of
cleverness and knowledge. Prince Andrew watched the
commander-in-chief's face attentively, and the only expression he
could see there was one of boredom, curiosity as to the meaning
of the feminine whispering behind the door, and a desire to
observe propriety. It was evident that Kutuzov despised
cleverness and learning and even the patriotic feeling shown by
Denisov, but despised them not because of his own intellect,
feelings, or knowledge---he did not try to display any of
these---but because of something else. He despised them because
of his old age and experience of life. The only instruction
Kutuzov gave of his own accord during that report referred to
looting by the Russian troops. At the end of the report the
general put before him for signature a paper relating to the
recovery of payment from army commanders for green oats mown down
by the soldiers, when landowners lodged petitions for
compensation.

After hearing the matter, Kutuzov smacked his lips together and
shook his head.

``Into the stove... into the fire with it! I tell you once for
all, my dear fellow,'' said he, ``into the fire with all such
things! Let them cut the crops and burn wood to their hearts'
content. I don't order it or allow it, but I don't exact
compensation either. One can't get on without it. 'When wood is
chopped the chips will fly.'{}'' He looked at the paper
again. ``Oh, this German precision!'' he muttered, shaking his
head.

% % % % % % % % % % % % % % % % % % % % % % % % % % % % % % % % %
% % % % % % % % % % % % % % % % % % % % % % % % % % % % % % % % %
% % % % % % % % % % % % % % % % % % % % % % % % % % % % % % % % %
% % % % % % % % % % % % % % % % % % % % % % % % % % % % % % % % %
% % % % % % % % % % % % % % % % % % % % % % % % % % % % % % % % %
% % % % % % % % % % % % % % % % % % % % % % % % % % % % % % % % %
% % % % % % % % % % % % % % % % % % % % % % % % % % % % % % % % %
% % % % % % % % % % % % % % % % % % % % % % % % % % % % % % % % %
% % % % % % % % % % % % % % % % % % % % % % % % % % % % % % % % %
% % % % % % % % % % % % % % % % % % % % % % % % % % % % % % % % %
% % % % % % % % % % % % % % % % % % % % % % % % % % % % % % % % %
% % % % % % % % % % % % % % % % % % % % % % % % % % % % % %

\chapter*{Chapter XVI} \ifaudio \marginpar{
\href{http://ia801407.us.archive.org/32/items/war_and_peace_10_0904_librivox/war_and_peace_10_16_tolstoy_64kb.mp3}{Audio}}
\fi

\initial*{W}{ell}, that's all!'' said Kutuzov as he signed the last of the
documents, and rising heavily and smoothing out the folds in his
fat white neck he moved toward the door with a more cheerful
expression.

The priest's wife, flushing rosy red, caught up the dish she had
after all not managed to present at the right moment, though she
had so long been preparing for it, and with a low bow offered it
to Kutuzov.

He screwed up his eyes, smiled, lifted her chin with his hand,
and said:

``Ah, what a beauty! Thank you, sweetheart!''

He took some gold pieces from his trouser pocket and put them on
the dish for her. ``Well, my dear, and how are we getting on?''
he asked, moving to the door of the room assigned to him. The
priest's wife smiled, and with dimples in her rosy cheeks
followed him into the room.  The adjutant came out to the porch
and asked Prince Andrew to lunch with him. Half an hour later
Prince Andrew was again called to Kutuzov. He found him reclining
in an armchair, still in the same unbuttoned overcoat. He had in
his hand a French book which he closed as Prince Andrew entered,
marking the place with a knife. Prince Andrew saw by the cover
that it was Les Chevaliers du Cygne by Madame de Genlis.

``Well, sit down, sit down here. Let's have a talk,'' said
Kutuzov. ``It's sad, very sad. But remember, my dear fellow, that
I am a father to you, a second father...''

Prince Andrew told Kutuzov all he knew of his father's death, and
what he had seen at Bald Hills when he passed through it.

``What... what they have brought us to!'' Kutuzov suddenly cried
in an agitated voice, evidently picturing vividly to himself from
Prince Andrew's story the condition Russia was in. ``But give me
time, give me time!'' he said with a grim look, evidently not
wishing to continue this agitating conversation, and added: ``I
sent for you to keep you with me.''

``I thank your Serene Highness, but I fear I am no longer fit for
the staff,'' replied Prince Andrew with a smile which Kutuzov
noticed.

Kutuzov glanced inquiringly at him.

``But above all,'' added Prince Andrew, ``I have grown used to my
regiment, am fond of the officers, and I fancy the men also like
me. I should be sorry to leave the regiment. If I decline the
honor of being with you, believe me...''

A shrewd, kindly, yet subtly derisive expression lit up Kutuzov's
podgy face. He cut Bolkonski short.

``I am sorry, for I need you. But you're right, you're right!
It's not here that men are needed. Advisers are always plentiful,
but men are not. The regiments would not be what they are if the
would-be advisers served there as you do. I remember you at
Austerlitz... I remember, yes, I remember you with the
standard!'' said Kutuzov, and a flush of pleasure suffused Prince
Andrew's face at this recollection.

Taking his hand and drawing him downwards, Kutuzov offered his
cheek to be kissed, and again Prince Andrew noticed tears in the
old man's eyes.  Though Prince Andrew knew that Kutuzov's tears
came easily, and that he was particularly tender to and
considerate of him from a wish to show sympathy with his loss,
yet this reminder of Austerlitz was both pleasant and flattering
to him.

``Go your way and God be with you. I know your path is the path
of honor!'' He paused. ``I missed you at Bucharest, but I needed
someone to send.'' And changing the subject, Kutuzov began to
speak of the Turkish war and the peace that had been
concluded. ``Yes, I have been much blamed,'' he said, ``both for
that war and the peace... but everything came at the right
time. Tout vient a point a celui qui sait
attendre.\footnote{``Everything comes in time to him who knows
how to wait.''}  And there were as many advisers there as
here...'' he went on, returning to the subject of \emph{advisers}
which evidently occupied him. ``Ah, those advisers!'' said
he. ``If we had listened to them all we should not have made
peace with Turkey and should not have been through with that war.
Everything in haste, but more haste, less speed. Kamenski would
have been lost if he had not died. He stormed fortresses with
thirty thousand men. It is not difficult to capture a fortress
but it is difficult to win a campaign. For that, not storming and
attacking but patience and time are wanted. Kamenski sent
soldiers to Rustchuk, but I only employed these two things and
took more fortresses than Kamenski and made them Turks eat
horseflesh!'' He swayed his head. ``And the French shall too,
believe me,'' he went on, growing warmer and beating his chest,
``I'll make them eat horseflesh!'' And tears again dimmed his
eyes.

``But shan't we have to accept battle?'' remarked Prince Andrew.

``We shall if everybody wants it; it can't be helped... But
believe me, my dear boy, there is nothing stronger than those
two: patience and time, they will do it all. But the advisers
n'entendent pas de cette oreille, voila le mal.\footnote{``Don't
see it that way, that's the trouble.''} Some want a
thing---others don't. What's one to do?'' he asked, evidently
expecting an answer. ``Well, what do you want us to do?'' he
repeated and his eye shone with a deep, shrewd look. ``I'll tell
you what to do,'' he continued, as Prince Andrew still did not
reply: ``I will tell you what to do, and what I do. Dans le
doute, mon cher,'' he paused, ``abstiens-toi''\footnote{``When in
doubt, my dear fellow, do nothing.''}---he articulated the French
proverb deliberately.

``Well, good-by, my dear fellow; remember that with all my heart
I share your sorrow, and that for you I am not a Serene Highness,
nor a prince, nor a commander-in-chief, but a father! If you want
anything come straight to me. Good-bye, my dear boy.''

Again he embraced and kissed Prince Andrew, but before the latter
had left the room Kutuzov gave a sigh of relief and went on with
his unfinished novel, Les Chevaliers du Cygne by Madame de
Genlis.

Prince Andrew could not have explained how or why it was, but
after that interview with Kutuzov he went back to his regiment
reassured as to the general course of affairs and as to the man
to whom it had been entrusted. The more he realized the absence
of all personal motive in that old man---in whom there seemed to
remain only the habit of passions, and in place of an intellect
(grouping events and drawing conclusions) only the capacity
calmly to contemplate the course of events---the more reassured
he was that everything would be as it should. ``He will not bring
in any plan of his own. He will not devise or undertake
anything,'' thought Prince Andrew, ``but he will hear everything,
remember everything, and put everything in its place. He will not
hinder anything useful nor allow anything harmful. He understands
that there is something stronger and more important than his own
will---the inevitable course of events, and he can see them and
grasp their significance, and seeing that significance can
refrain from meddling and renounce his personal wish directed to
something else. And above all,'' thought Prince Andrew, ``one
believes in him because he's Russian, despite the novel by Genlis
and the French proverbs, and because his voice shook when he
said: 'What they have brought us to!' and had a sob in it when he
said he would 'make them eat horseflesh!'{}''

On such feelings, more or less dimly shared by all, the unanimity
and general approval were founded with which, despite court
influences, the popular choice of Kutuzov as commander-in-chief
was received.

% % % % % % % % % % % % % % % % % % % % % % % % % % % % % % % % %
% % % % % % % % % % % % % % % % % % % % % % % % % % % % % % % % %
% % % % % % % % % % % % % % % % % % % % % % % % % % % % % % % % %
% % % % % % % % % % % % % % % % % % % % % % % % % % % % % % % % %
% % % % % % % % % % % % % % % % % % % % % % % % % % % % % % % % %
% % % % % % % % % % % % % % % % % % % % % % % % % % % % % % % % %
% % % % % % % % % % % % % % % % % % % % % % % % % % % % % % % % %
% % % % % % % % % % % % % % % % % % % % % % % % % % % % % % % % %
% % % % % % % % % % % % % % % % % % % % % % % % % % % % % % % % %
% % % % % % % % % % % % % % % % % % % % % % % % % % % % % % % % %
% % % % % % % % % % % % % % % % % % % % % % % % % % % % % % % % %
% % % % % % % % % % % % % % % % % % % % % % % % % % % % % %

\chapter*{Chapter XVII} \ifaudio \marginpar{
\href{http://ia801407.us.archive.org/32/items/war_and_peace_10_0904_librivox/war_and_peace_10_17_tolstoy_64kb.mp3}{Audio}}
\fi

\initial{A}{fter} the Emperor had left Moscow, life flowed on there in its
usual course, and its course was so very usual that it was
difficult to remember the recent days of patriotic elation and
ardor, hard to believe that Russia was really in danger and that
the members of the English Club were also sons of the Fatherland
ready to sacrifice everything for it. The one thing that recalled
the patriotic fervor everyone had displayed during the Emperor's
stay was the call for contributions of men and money, a necessity
that as soon as the promises had been made assumed a legal,
official form and became unavoidable.

With the enemy's approach to Moscow, the Moscovites' view of
their situation did not grow more serious but on the contrary
became even more frivolous, as always happens with people who see
a great danger approaching. At the approach of danger there are
always two voices that speak with equal power in the human soul:
one very reasonably tells a man to consider the nature of the
danger and the means of escaping it; the other, still more
reasonably, says that it is too depressing and painful to think
of the danger, since it is not in man's power to foresee
everything and avert the general course of events, and it is
therefore better to disregard what is painful till it comes, and
to think about what is pleasant. In solitude a man generally
listens to the first voice, but in society to the second. So it
was now with the inhabitants of Moscow. It was long since people
had been as gay in Moscow as that year.

Rostopchin's broadsheets, headed by woodcuts of a drink shop, a
potman, and a Moscow burgher called Karpushka Chigirin,
``who---having been a militiaman and having had rather too much
at the pub---heard that Napoleon wished to come to Moscow, grew
angry, abused the French in very bad language, came out of the
drink shop, and, under the sign of the eagle, began to address
the assembled people,'' were read and discussed, together with
the latest of Vasili Lvovich Pushkin's bouts rimes.

In the corner room at the club, members gathered to read these
broadsheets, and some liked the way Karpushka jeered at the
French, saying: ``They will swell up with Russian cabbage, burst
with our buckwheat porridge, and choke themselves with cabbage
soup. They are all dwarfs and one peasant woman will toss three
of them with a hayfork.''  Others did not like that tone and said
it was stupid and vulgar. It was said that Rostopchin had
expelled all Frenchmen and even all foreigners from Moscow, and
that there had been some spies and agents of Napoleon among them;
but this was told chiefly to introduce Rostopchin's witty remark
on that occasion. The foreigners were deported to Nizhni by boat,
and Rostopchin had said to them in French: ``Rentrez en
vousmemes; entrez dans la barque, et n'en faites pas une barque
de Charon.''\footnote{``Think it over; get into the barque, and
take care not to make it a barque of Charon.''} There was talk of
all the government offices having been already removed from
Moscow, and to this Shinshin's witticism was added---that for
that alone Moscow ought to be grateful to Napoleon. It was said
that Mamonov's regiment would cost him eight hundred thousand
rubles, and that Bezukhov had spent even more on his, but that
the best thing about Bezukhov's action was that he himself was
going to don a uniform and ride at the head of his regiment
without charging anything for the show.
 

``You don't spare anyone,'' said Julie Drubetskaya as she
collected and pressed together a bunch of raveled lint with her
thin, beringed fingers.

Julie was preparing to leave Moscow next day and was giving a
farewell soiree.

``Bezukhov est ridicule, but he is so kind and good-natured. What
pleasure is there to be so caustique?''

``A forfeit!'' cried a young man in militia uniform whom Julie
called ``mon chevalier,'' and who was going with her to Nizhni.

In Julie's set, as in many other circles in Moscow, it had been
agreed that they would speak nothing but Russian and that those
who made a slip and spoke French should pay fines to the
Committee of Voluntary Contributions.

``Another forfeit for a Gallicism,'' said a Russian writer who
was present. ``'What pleasure is there to be' is not Russian!''

``You spare no one,'' continued Julie to the young man without
heeding the author's remark.

``For caustique---I am guilty and will pay, and I am prepared to
pay again for the pleasure of telling you the truth. For
Gallicisms I won't be responsible,'' she remarked, turning to the
author: ``I have neither the money nor the time, like Prince
Galitsyn, to engage a master to teach me Russian!''

``Ah, here he is!'' she added. ``Quand on... No, no,'' she said
to the militia officer, ``you won't catch me. Speak of the sun
and you see its rays!'' and she smiled amiably at Pierre. ``We
were just talking of you,'' she said with the facility in lying
natural to a society woman. ``We were saying that your regiment
would be sure to be better than Mamonov's.''

``Oh, don't talk to me of my regiment,'' replied Pierre, kissing
his hostess' hand and taking a seat beside her. ``I am so sick of
it.''

``You will, of course, command it yourself?'' said Julie,
directing a sly, sarcastic glance toward the militia officer.

The latter in Pierre's presence had ceased to be caustic, and his
face expressed perplexity as to what Julie's smile might mean. In
spite of his absent-mindedness and good nature, Pierre's
personality immediately checked any attempt to ridicule him to
his face.

``No,'' said Pierre, with a laughing glance at his big, stout
body. ``I should make too good a target for the French, besides I
am afraid I should hardly be able to climb onto a horse.''

Among those whom Julie's guests happened to choose to gossip
about were the Rostovs.

``I hear that their affairs are in a very bad way,'' said
Julie. ``And he is so unreasonable, the count himself I mean. The
Razumovskis wanted to buy his house and his estate near Moscow,
but it drags on and on. He asks too much.''

``No, I think the sale will come off in a few days,'' said
someone.  ``Though it is madness to buy anything in Moscow now.''

``Why?'' asked Julie. ``You don't think Moscow is in danger?''

``Then why are you leaving?''

``I? What a question! I am going because... well, because
everyone is going: and besides---I am not Joan of Arc or an
Amazon.''

``Well, of course, of course! Let me have some more strips of
linen.''

``If he manages the business properly he will be able to pay off
all his debts,'' said the militia officer, speaking of Rostov.

``A kindly old man but not up to much. And why do they stay on so
long in Moscow? They meant to leave for the country long
ago. Natalie is quite well again now, isn't she?'' Julie asked
Pierre with a knowing smile.

``They are waiting for their younger son,'' Pierre replied. ``He
joined Obolenski's Cossacks and went to Belaya Tserkov where the
regiment is being formed. But now they have had him transferred
to my regiment and are expecting him every day. The count wanted
to leave long ago, but the countess won't on any account leave
Moscow till her son returns.''

``I met them the day before yesterday at the Arkharovs'. Natalie
has recovered her looks and is brighter. She sang a song. How
easily some people get over everything!''

``Get over what?'' inquired Pierre, looking displeased.

Julie smiled.

``You know, Count, such knights as you are only found in Madame
de Souza's novels.''

``What knights? What do you mean?'' demanded Pierre, blushing.

``Oh, come, my dear count! C'est la fable de tout Moscou. Je vous
admire, ma parole d'honneur!''\footnote{``It is the talk of all
Moscow. My word, I admire you!''}

``Forfeit, forfeit!'' cried the militia officer.

``All right, one can't talk---how tiresome!''

``What is 'the talk of all Moscow'?'' Pierre asked angrily,
rising to his feet.

``Come now, Count, you know!''

``I don't know anything about it,'' said Pierre.

``I know you were friendly with Natalie, and so... but I was
always more friendly with Vera---that dear Vera.''

``No, madame!'' Pierre continued in a tone of displeasure, ``I
have not taken on myself the role of Natalie Rostova's knight at
all, and have not been to their house for nearly a month. But I
cannot understand the cruelty...''

``Qui s'excuse s'accuse,''\footnote{``Who excuses himself,
accuses himself.''} said Julie, smiling and waving the lint
triumphantly, and to have the last word she promptly changed the
subject. ``Do you know what I heard today? Poor Mary Bolkonskaya
arrived in Moscow yesterday. Do you know that she has lost her
father?''

``Really? Where is she? I should like very much to see her,''
said Pierre.

``I spent the evening with her yesterday. She is going to their
estate near Moscow either today or tomorrow morning, with her
nephew.''

``Well, and how is she?'' asked Pierre.

``She is well, but sad. But do you know who rescued her? It is
quite a romance. Nicholas Rostov! She was surrounded, and they
wanted to kill her and had wounded some of her people. He rushed
in and saved her...''

``Another romance,'' said the militia officer. ``Really, this
general flight has been arranged to get all the old maids married
off. Catiche is one and Princess Bolkonskaya another.''

``Do you know, I really believe she is un petit peu amoureuse du
jeune homme.''\footnote{``A little bit in love with the young
man.''}

``Forfeit, forfeit, forfeit!''

``But how could one say that in Russian?''

% % % % % % % % % % % % % % % % % % % % % % % % % % % % % % % % %
% % % % % % % % % % % % % % % % % % % % % % % % % % % % % % % % %
% % % % % % % % % % % % % % % % % % % % % % % % % % % % % % % % %
% % % % % % % % % % % % % % % % % % % % % % % % % % % % % % % % %
% % % % % % % % % % % % % % % % % % % % % % % % % % % % % % % % %
% % % % % % % % % % % % % % % % % % % % % % % % % % % % % % % % %
% % % % % % % % % % % % % % % % % % % % % % % % % % % % % % % % %
% % % % % % % % % % % % % % % % % % % % % % % % % % % % % % % % %
% % % % % % % % % % % % % % % % % % % % % % % % % % % % % % % % %
% % % % % % % % % % % % % % % % % % % % % % % % % % % % % % % % %
% % % % % % % % % % % % % % % % % % % % % % % % % % % % % % % % %
% % % % % % % % % % % % % % % % % % % % % % % % % % % % % %

\chapter*{Chapter XVIII} \ifaudio \marginpar{
\href{http://ia801407.us.archive.org/32/items/war_and_peace_10_0904_librivox/war_and_peace_10_18_tolstoy_64kb.mp3}{Audio}}
\fi

\initial{W}{hen} Pierre returned home he was handed two of Rostopchin's
broadsheets that had been brought that day.

The first declared that the report that Count Rostopchin had
forbidden people to leave Moscow was false; on the contrary he
was glad that ladies and tradesmen's wives were leaving the
city. ``There will be less panic and less gossip,'' ran the
broadsheet ``but I will stake my life on it that scoundrel will
not enter Moscow.'' These words showed Pierre clearly for the
first time that the French would enter Moscow. The second
broadsheet stated that our headquarters were at Vyazma, that
Count Wittgenstein had defeated the French, but that as many of
the inhabitants of Moscow wished to be armed, weapons were ready
for them at the arsenal: sabers, pistols, and muskets which could
be had at a low price. The tone of the proclamation was not as
jocose as in the former Chigirin talks. Pierre pondered over
these broadsheets. Evidently the terrible stormcloud he had
desired with the whole strength of his soul but which yet aroused
involuntary horror in him was drawing near.

``Shall I join the army and enter the service, or wait?'' he
asked himself for the hundredth time. He took a pack of cards
that lay on the table and began to lay them out for a game of
patience.

``If this patience comes out,'' he said to himself after
shuffling the cards, holding them in his hand, and lifting his
head, ``if it comes out, it means... what does it mean?''

He had not decided what it should mean when he heard the voice of
the eldest princess at the door asking whether she might come in.

``Then it will mean that I must go to the army,'' said Pierre to
himself.  ``Come in, come in!'' he added to the princess.

Only the eldest princess, the one with the stony face and long
waist, was still living in Pierre's house. The two younger ones
had both married.

``Excuse my coming to you, cousin,'' she said in a reproachful
and agitated voice. ``You know some decision must be come
to. What is going to happen? Everyone has left Moscow and the
people are rioting. How is it that we are staying on?''

``On the contrary, things seem satisfactory, ma cousine,'' said
Pierre in the bantering tone he habitually adopted toward her,
always feeling uncomfortable in the role of her benefactor.

``Satisfactory, indeed! Very satisfactory! Barbara Ivanovna told
me today how our troops are distinguishing themselves. It
certainly does them credit! And the people too are quite
mutinous---they no longer obey, even my maid has taken to being
rude. At this rate they will soon begin beating us. One can't
walk in the streets. But, above all, the French will be here any
day now, so what are we waiting for? I ask just one thing of you,
cousin,'' she went on, ``arrange for me to be taken to
Petersburg. Whatever I may be, I can't live under Bonaparte's
rule.''

``Oh, come, ma cousine! Where do you get your information from?
On the contrary...''

``I won't submit to your Napoleon! Others may if they
please... If you don't want to do this...''

``But I will, I'll give the order at once.''

The princess was apparently vexed at not having anyone to be
angry with.  Muttering to herself, she sat down on a chair.

``But you have been misinformed,'' said Pierre. ``Everything is
quiet in the city and there is not the slightest danger. See!
I've just been reading...'' He showed her the broadsheet. ``Count
Rostopchin writes that he will stake his life on it that the
enemy will not enter Moscow.''

``Oh, that count of yours!'' said the princess malevolently. ``He
is a hypocrite, a rascal who has himself roused the people to
riot. Didn't he write in those idiotic broadsheets that anyone,
'whoever it might be, should be dragged to the lockup by his
hair'? (How silly!) 'And honor and glory to whoever captures
him,' he says. This is what his cajolery has brought us to!
Barbara Ivanovna told me the mob near killed her because she said
something in French.''

``Oh, but it's so... You take everything so to heart,'' said
Pierre, and began laying out his cards for patience.

Although that patience did come out, Pierre did not join the
army, but remained in deserted Moscow ever in the same state of
agitation, irresolution, and alarm, yet at the same time joyfully
expecting something terrible.

Next day toward evening the princess set off, and Pierre's head
steward came to inform him that the money needed for the
equipment of his regiment could not be found without selling one
of the estates. In general the head steward made out to Pierre
that his project of raising a regiment would ruin him. Pierre
listened to him, scarcely able to repress a smile.

``Well then, sell it,'' said he. ``What's to be done? I can't
draw back now!''

The worse everything became, especially his own affairs, the
better was Pierre pleased and the more evident was it that the
catastrophe he expected was approaching. Hardly anyone he knew
was left in town. Julie had gone, and so had Princess Mary. Of
his intimate friends only the Rostovs remained, but he did not go
to see them.

To distract his thoughts he drove that day to the village of
Vorontsovo to see the great balloon Leppich was constructing to
destroy the foe, and a trial balloon that was to go up next
day. The balloon was not yet ready, but Pierre learned that it
was being constructed by the Emperor's desire. The Emperor had
written to Count Rostopchin as follows:

As soon as Leppich is ready, get together a crew of reliable and
intelligent men for his car and send a courier to General Kutuzov
to let him know. I have informed him of the matter.

Please impress upon Leppich to be very careful where he descends
for the first time, that he may not make a mistake and fall into
the enemy's hands. It is essential for him to combine his
movements with those of the commander-in-chief.

On his way home from Vorontsovo, as he was passing the Bolotnoe
Place Pierre, seeing a large crowd round the Lobnoe Place,
stopped and got out of his trap. A French cook accused of being a
spy was being flogged. The flogging was only just over, and the
executioner was releasing from the flogging bench a stout man
with red whiskers, in blue stockings and a green jacket, who was
moaning piteously. Another criminal, thin and pale, stood
near. Judging by their faces they were both Frenchmen. With a
frightened and suffering look resembling that on the thin
Frenchman's face, Pierre pushed his way in through the crowd.

``What is it? Who is it? What is it for?'' he kept asking.

But the attention of the crowd---officials, burghers,
shopkeepers, peasants, and women in cloaks and in pelisses---was
so eagerly centered on what was passing in Lobnoe Place that no
one answered him. The stout man rose, frowned, shrugged his
shoulders, and evidently trying to appear firm began to pull on
his jacket without looking about him, but suddenly his lips
trembled and he began to cry, in the way full-blooded grown-up
men cry, though angry with himself for doing so. In the crowd
people began talking loudly, to stifle their feelings of pity as
it seemed to Pierre.

``He's cook to some prince.''

``Eh, mounseer, Russian sauce seems to be sour to a
Frenchman... sets his teeth on edge!'' said a wrinkled clerk who
was standing behind Pierre, when the Frenchman began to cry.

The clerk glanced round, evidently hoping that his joke would be
appreciated. Some people began to laugh, others continued to
watch in dismay the executioner who was undressing the other man.

Pierre choked, his face puckered, and he turned hastily away,
went back to his trap muttering something to himself as he went,
and took his seat. As they drove along he shuddered and exclaimed
several times so audibly that the coachman asked him:

``What is your pleasure?''

``Where are you going?'' shouted Pierre to the man, who was
driving to Lubyanka Street.

``To the Governor's, as you ordered,'' answered the coachman.

``Fool! Idiot!'' shouted Pierre, abusing his coachman---a thing
he rarely did. ``Home, I told you! And drive faster, blockhead!''
``I must get away this very day,'' he murmured to himself.

At the sight of the tortured Frenchman and the crowd surrounding
the Lobnoe Place, Pierre had so definitely made up his mind that
he could no longer remain in Moscow and would leave for the army
that very day that it seemed to him that either he had told the
coachman this or that the man ought to have known it for himself.

On reaching home Pierre gave orders to Evstafey---his head
coachman who knew everything, could do anything, and was known to
all Moscow---that he would leave that night for the army at
Mozhaysk, and that his saddle horses should be sent there. This
could not all be arranged that day, so on Evstafey's
representation Pierre had to put off his departure till next day
to allow time for the relay horses to be sent on in advance.

On the twenty-fourth the weather cleared up after a spell of
rain, and after dinner Pierre left Moscow. When changing horses
that night in Perkhushkovo, he learned that there had been a
great battle that evening. (This was the battle of Shevardino.)
He was told that there in Perkhushkovo the earth trembled from
the firing, but nobody could answer his questions as to who had
won. At dawn next day Pierre was approaching Mozhaysk.

Every house in Mozhaysk had soldiers quartered in it, and at the
hostel where Pierre was met by his groom and coachman there was
no room to be had. It was full of officers.

Everywhere in Mozhaysk and beyond it, troops were stationed or on
the march. Cossacks, foot and horse soldiers, wagons, caissons,
and cannon were everywhere. Pierre pushed forward as fast as he
could, and the farther he left Moscow behind and the deeper he
plunged into that sea of troops the more was he overcome by
restless agitation and a new and joyful feeling he had not
experienced before. It was a feeling akin to what he had felt at
the Sloboda Palace during the Emperor's visit---a sense of the
necessity of undertaking something and sacrificing something. He
now experienced a glad consciousness that everything that
constitutes men's happiness---the comforts of life, wealth, even
life itself---is rubbish it is pleasant to throw away, compared
with something... With what? Pierre could not say, and he did not
try to determine for whom and for what he felt such particular
delight in sacrificing everything. He was not occupied with the
question of what to sacrifice for; the fact of sacrificing in
itself afforded him a new and joyous sensation.

% % % % % % % % % % % % % % % % % % % % % % % % % % % % % % % % %
% % % % % % % % % % % % % % % % % % % % % % % % % % % % % % % % %
% % % % % % % % % % % % % % % % % % % % % % % % % % % % % % % % %
% % % % % % % % % % % % % % % % % % % % % % % % % % % % % % % % %
% % % % % % % % % % % % % % % % % % % % % % % % % % % % % % % % %
% % % % % % % % % % % % % % % % % % % % % % % % % % % % % % % % %
% % % % % % % % % % % % % % % % % % % % % % % % % % % % % % % % %
% % % % % % % % % % % % % % % % % % % % % % % % % % % % % % % % %
% % % % % % % % % % % % % % % % % % % % % % % % % % % % % % % % %
% % % % % % % % % % % % % % % % % % % % % % % % % % % % % % % % %
% % % % % % % % % % % % % % % % % % % % % % % % % % % % % % % % %
% % % % % % % % % % % % % % % % % % % % % % % % % % % % % %

\chapter*{Chapter XIX} \ifaudio \marginpar{
\href{http://ia801407.us.archive.org/32/items/war_and_peace_10_0904_librivox/war_and_peace_10_19_tolstoy_64kb.mp3}{Audio}}
\fi

\initial{O}{n} the twenty-fourth of August the battle of the Shevardino
Redoubt was fought, on the twenty-fifth not a shot was fired by
either side, and on the twenty-sixth the battle of Borodino
itself took place.

Why and how were the battles of Shevardino and Borodino given and
accepted? Why was the battle of Borodino fought? There was not
the least sense in it for either the French or the Russians. Its
immediate result for the Russians was, and was bound to be, that
we were brought nearer to the destruction of Moscow---which we
feared more than anything in the world; and for the French its
immediate result was that they were brought nearer to the
destruction of their whole army---which they feared more than
anything in the world. What the result must be was quite obvious,
and yet Napoleon offered and Kutuzov accepted that battle.

If the commanders had been guided by reason, it would seem that
it must have been obvious to Napoleon that by advancing thirteen
hundred miles and giving battle with a probability of losing a
quarter of his army, he was advancing to certain destruction, and
it must have been equally clear to Kutuzov that by accepting
battle and risking the loss of a quarter of his army he would
certainly lose Moscow. For Kutuzov this was mathematically clear,
as it is that if when playing draughts I have one man less and go
on exchanging, I shall certainly lose, and therefore should not
exchange. When my opponent has sixteen men and I have fourteen, I
am only one eighth weaker than he, but when I have exchanged
thirteen more men he will be three times as strong as I am.

Before the battle of Borodino our strength in proportion to the
French was about as five to six, but after that battle it was
little more than one to two: previously we had a hundred thousand
against a hundred and twenty thousand; afterwards little more
than fifty thousand against a hundred thousand. Yet the shrewd
and experienced Kutuzov accepted the battle, while Napoleon, who
was said to be a commander of genius, gave it, losing a quarter
of his army and lengthening his lines of communication still
more. If it is said that he expected to end the campaign by
occupying Moscow as he had ended a previous campaign by occupying
Vienna, there is much evidence to the contrary. Napoleon's
historians themselves tell us that from Smolensk onwards he
wished to stop, knew the danger of his extended position, and
knew that the occupation of Moscow would not be the end of the
campaign, for he had seen at Smolensk the state in which Russian
towns were left to him, and had not received a single reply to
his repeated announcements of his wish to negotiate.

In giving and accepting battle at Borodino, Kutuzov acted
involuntarily and irrationally. But later on, to fit what had
occurred, the historians provided cunningly devised evidence of
the foresight and genius of the generals who, of all the blind
tools of history were the most enslaved and involuntary.

The ancients have left us model heroic poems in which the heroes
furnish the whole interest of the story, and we are still unable
to accustom ourselves to the fact that for our epoch histories of
that kind are meaningless.

On the other question, how the battle of Borodino and the
preceding battle of Shevardino were fought, there also exists a
definite and well-known, but quite false, conception. All the
historians describe the affair as follows:

The Russian army, they say, in its retreat from Smolensk sought
out for itself the best position for a general engagement and
found such a position at Borodino.

The Russians, they say, fortified this position in advance on the
left of the highroad (from Moscow to Smolensk) and almost at a
right angle to it, from Borodino to Utitsa, at the very place
where the battle was fought.

In front of this position, they say, a fortified outpost was set
up on the Shevardino mound to observe the enemy. On the
twenty-fourth, we are told, Napoleon attacked this advanced post
and took it, and, on the twenty-sixth, attacked the whole Russian
army, which was in position on the field of Borodino.

So the histories say, and it is all quite wrong, as anyone who
cares to look into the matter can easily convince himself.

The Russians did not seek out the best position but, on the
contrary, during the retreat passed many positions better than
Borodino. They did not stop at any one of these positions because
Kutuzov did not wish to occupy a position he had not himself
chosen, because the popular demand for a battle had not yet
expressed itself strongly enough, and because Miloradovich had
not yet arrived with the militia, and for many other reasons. The
fact is that other positions they had passed were stronger, and
that the position at Borodino (the one where the battle was
fought), far from being strong, was no more a position than any
other spot one might find in the Russian Empire by sticking a pin
into the map at hazard.

Not only did the Russians not fortify the position on the field
of Borodino to the left of, and at a right angle to, the highroad
(that is, the position on which the battle took place), but never
till the twenty-fifth of August, 1812, did they think that a
battle might be fought there. This was shown first by the fact
that there were no entrenchments there by the twenty fifth and
that those begun on the twenty-fifth and twenty-sixth were not
completed, and secondly, by the position of the Shevardino
Redoubt. That redoubt was quite senseless in front of the
position where the battle was accepted. Why was it more strongly
fortified than any other post? And why were all efforts exhausted
and six thousand men sacrificed to defend it till late at night
on the twenty-fourth? A Cossack patrol would have sufficed to
observe the enemy. Thirdly, as proof that the position on which
the battle was fought had not been foreseen and that the
Shevardino Redoubt was not an advanced post of that position, we
have the fact that up to the twenty-fifth, Barclay de Tolly and
Bagration were convinced that the Shevardino Redoubt was the left
flank of the position, and that Kutuzov himself in his report,
written in hot haste after the battle, speaks of the Shevardino
Redoubt as the left flank of the position. It was much later,
when reports on the battle of Borodino were written at leisure,
that the incorrect and extraordinary statement was invented
(probably to justify the mistakes of a commander-in-chief who had
to be represented as infallible) that the Shevardino Redoubt was
an advanced post---whereas in reality it was simply a fortified
point on the left flank---and that the battle of Borodino was
fought by us on an entrenched position previously selected, where
as it was fought on a quite unexpected spot which was almost
unentrenched.

The case was evidently this: a position was selected along the
river Kolocha---which crosses the highroad not at a right angle
but at an acute angle---so that the left flank was at Shevardino,
the right flank near the village of Novoe, and the center at
Borodino at the confluence of the rivers Kolocha and Voyna.

To anyone who looks at the field of Borodino without thinking of
how the battle was actually fought, this position, protected by
the river Kolocha, presents itself as obvious for an army whose
object was to prevent an enemy from advancing along the Smolensk
road to Moscow.

Napoleon, riding to Valuevo on the twenty-fourth, did not see (as
the history books say he did) the position of the Russians from
Utitsa to Borodino (he could not have seen that position because
it did not exist), nor did he see an advanced post of the Russian
army, but while pursuing the Russian rearguard he came upon the
left flank of the Russian position---at the Shevardino
Redoubt---and unexpectedly for the Russians moved his army across
the Kolocha. And the Russians, not having time to begin a general
engagement, withdrew their left wing from the position they had
intended to occupy and took up a new position which had not been
foreseen and was not fortified. By crossing to the other side of
the Kolocha to the left of the highroad, Napoleon shifted the
whole forthcoming battle from right to left (looking from the
Russian side) and transferred it to the plain between Utitsa,
Semenovsk, and Borodino---a plain no more advantageous as a
position than any other plain in Russia---and there the whole
battle of the twenty-sixth of August took place.

Had Napoleon not ridden out on the evening of the twenty-fourth
to the Kolocha, and had he not then ordered an immediate attack
on the redoubt but had begun the attack next morning, no one
would have doubted that the Shevardino Redoubt was the left flank
of our position, and the battle would have taken place where we
expected it. In that case we should probably have defended the
Shevardino Redoubt---our left flank---still more obstinately. We
should have attacked Napoleon in the center or on the right, and
the engagement would have taken place on the twenty-fifth, in the
position we intended and had fortified. But as the attack on our
left flank took place in the evening after the retreat of our
rear guard (that is, immediately after the fight at Gridneva),
and as the Russian commanders did not wish, or were not in time,
to begin a general engagement then on the evening of the
twenty-fourth, the first and chief action of the battle of
Borodino was already lost on the twenty-fourth, and obviously led
to the loss of the one fought on the twenty-sixth.

After the loss of the Shevardino Redoubt, we found ourselves on
the morning of the twenty-fifth without a position for our left
flank, and were forced to bend it back and hastily entrench it
where it chanced to be.

Not only was the Russian army on the twenty-sixth defended by
weak, unfinished entrenchments, but the disadvantage of that
position was increased by the fact that the Russian
commanders---not having fully realized what had happened, namely
the loss of our position on the left flank and the shifting of
the whole field of the forthcoming battle from right to
left---maintained their extended position from the village of
Novoe to Utitsa, and consequently had to move their forces from
right to left during the battle. So it happened that throughout
the whole battle the Russians opposed the entire French army
launched against our left flank with but half as many
men. (Poniatowski's action against Utitsa, and Uvarov's on the
right flank against the French, were actions distinct from the
main course of the battle.) So the battle of Borodino did not
take place at all as (in an effort to conceal our commanders'
mistakes even at the cost of diminishing the glory due to the
Russian army and people) it has been described. The battle of
Borodino was not fought on a chosen and entrenched position with
forces only slightly weaker than those of the enemy, but, as a
result of the loss of the Shevardino Redoubt, the Russians fought
the battle of Borodino on an open and almost unentrenched
position, with forces only half as numerous as the French; that
is to say, under conditions in which it was not merely
unthinkable to fight for ten hours and secure an indecisive
result, but unthinkable to keep an army even from complete
disintegration and flight.

% % % % % % % % % % % % % % % % % % % % % % % % % % % % % % % % %
% % % % % % % % % % % % % % % % % % % % % % % % % % % % % % % % %
% % % % % % % % % % % % % % % % % % % % % % % % % % % % % % % % %
% % % % % % % % % % % % % % % % % % % % % % % % % % % % % % % % %
% % % % % % % % % % % % % % % % % % % % % % % % % % % % % % % % %
% % % % % % % % % % % % % % % % % % % % % % % % % % % % % % % % %
% % % % % % % % % % % % % % % % % % % % % % % % % % % % % % % % %
% % % % % % % % % % % % % % % % % % % % % % % % % % % % % % % % %
% % % % % % % % % % % % % % % % % % % % % % % % % % % % % % % % %
% % % % % % % % % % % % % % % % % % % % % % % % % % % % % % % % %
% % % % % % % % % % % % % % % % % % % % % % % % % % % % % % % % %
% % % % % % % % % % % % % % % % % % % % % % % % % % % % % %

\chapter*{Chapter XX} \ifaudio \marginpar{
\href{http://ia801407.us.archive.org/32/items/war_and_peace_10_0904_librivox/war_and_peace_10_20_tolstoy_64kb.mp3}{Audio}}
\fi

\initial{O}{n} the morning of the twenty-fifth Pierre was leaving
Mozhaysk. At the descent of the high steep hill, down which a
winding road led out of the town past the cathedral on the right,
where a service was being held and the bells were ringing, Pierre
got out of his vehicle and proceeded on foot. Behind him a
cavalry regiment was coming down the hill preceded by its
singers. Coming up toward him was a train of carts carrying men
who had been wounded in the engagement the day before. The
peasant drivers, shouting and lashing their horses, kept crossing
from side to side. The carts, in each of which three or four
wounded soldiers were lying or sitting, jolted over the stones
that had been thrown on the steep incline to make it something
like a road. The wounded, bandaged with rags, with pale cheeks,
compressed lips, and knitted brows, held on to the sides of the
carts as they were jolted against one another. Almost all of them
stared with naive, childlike curiosity at Pierre's white hat and
green swallow-tail coat.

Pierre's coachman shouted angrily at the convoy of wounded to
keep to one side of the road. The cavalry regiment, as it
descended the hill with its singers, surrounded Pierre's carriage
and blocked the road.  Pierre stopped, being pressed against the
side of the cutting in which the road ran. The sunshine from
behind the hill did not penetrate into the cutting and there it
was cold and damp, but above Pierre's head was the bright August
sunshine and the bells sounded merrily. One of the carts with
wounded stopped by the side of the road close to Pierre. The
driver in his bast shoes ran panting up to it, placed a stone
under one of its tireless hind wheels, and began arranging the
breech-band on his little horse.

One of the wounded, an old soldier with a bandaged arm who was
following the cart on foot, caught hold of it with his sound hand
and turned to look at Pierre.

``I say, fellow countryman! Will they set us down here or take us
on to Moscow?'' he asked.

Pierre was so deep in thought that he did not hear the
question. He was looking now at the cavalry regiment that had met
the convoy of wounded, now at the cart by which he was standing,
in which two wounded men were sitting and one was lying. One of
those sitting up in the cart had probably been wounded in the
cheek. His whole head was wrapped in rags and one cheek was
swollen to the size of a baby's head. His nose and mouth were
twisted to one side. This soldier was looking at the cathedral
and crossing himself. Another, a young lad, a fair-haired recruit
as white as though there was no blood in his thin face, looked at
Pierre kindly, with a fixed smile. The third lay prone so that
his face was not visible. The cavalry singers were passing close
by:

Ah lost, quite lost... is my head so keen, Living in a foreign
land.

they sang their soldiers' dance song.

As if responding to them but with a different sort of merriment,
the metallic sound of the bells reverberated high above and the
hot rays of the sun bathed the top of the opposite slope with yet
another sort of merriment. But beneath the slope, by the cart
with the wounded near the panting little nag where Pierre stood,
it was damp, somber, and sad.

The soldier with the swollen cheek looked angrily at the cavalry
singers.

``Oh, the coxcombs!'' he muttered reproachfully.

``It's not the soldiers only, but I've seen peasants today,
too... The peasants---even they have to go,'' said the soldier
behind the cart, addressing Pierre with a sad smile. ``No
distinctions made nowadays...  They want the whole nation to fall
on them---in a word, it's Moscow! They want to make an end of
it.''

In spite of the obscurity of the soldier's words Pierre
understood what he wanted to say and nodded approval.

The road was clear again; Pierre descended the hill and drove on.

He kept looking to either side of the road for familiar faces,
but only saw everywhere the unfamiliar faces of various military
men of different branches of the service, who all looked with
astonishment at his white hat and green tail coat.

Having gone nearly three miles he at last met an acquaintance and
eagerly addressed him. This was one of the head army doctors. He
was driving toward Pierre in a covered gig, sitting beside a
young surgeon, and on recognizing Pierre he told the Cossack who
occupied the driver's seat to pull up.

``Count! Your excellency, how come you to be here?'' asked the
doctor.

``Well, you know, I wanted to see...''

``Yes, yes, there will be something to see...''

Pierre got out and talked to the doctor, explaining his intention
of taking part in a battle.

The doctor advised him to apply direct to Kutuzov.

``Why should you be God knows where out of sight, during the
battle?'' he said, exchanging glances with his young
companion. ``Anyhow his Serene Highness knows you and will
receive you graciously. That's what you must do.''

The doctor seemed tired and in a hurry.

``You think so?... Ah, I also wanted to ask you where our
position is exactly?'' said Pierre.

``The position?'' repeated the doctor. ``Well, that's not my
line. Drive past Tatarinova, a lot of digging is going on
there. Go up the hillock and you'll see.''

``Can one see from there?... If you would...''

But the doctor interrupted him and moved toward his gig.

``I would go with you but on my honor I'm up to here''---and he
pointed to his throat. ``I'm galloping to the commander of the
corps. How do matters stand?... You know, Count, there'll be a
battle tomorrow. Out of an army of a hundred thousand we must
expect at least twenty thousand wounded, and we haven't
stretchers, or bunks, or dressers, or doctors enough for six
thousand. We have ten thousand carts, but we need other things as
well---we must manage as best we can!''

The strange thought that of the thousands of men, young and old,
who had stared with merry surprise at his hat (perhaps the very
men he had noticed), twenty thousand were inevitably doomed to
wounds and death amazed Pierre.

``They may die tomorrow; why are they thinking of anything but
death?''  And by some latent sequence of thought the descent of
the Mozhaysk hill, the carts with the wounded, the ringing bells,
the slanting rays of the sun, and the songs of the cavalrymen
vividly recurred to his mind.

``The cavalry ride to battle and meet the wounded and do not for
a moment think of what awaits them, but pass by, winking at the
wounded. Yet from among these men twenty thousand are doomed to
die, and they wonder at my hat! Strange!'' thought Pierre,
continuing his way to Tatarinova.

In front of a landowner's house to the left of the road stood
carriages, wagons, and crowds of orderlies and sentinels. The
commander-in-chief was putting up there, but just when Pierre
arrived he was not in and hardly any of the staff were
there---they had gone to the church service.  Pierre drove on
toward Gorki.

When he had ascended the hill and reached the little village
street, he saw for the first time peasant militiamen in their
white shirts and with crosses on their caps, who, talking and
laughing loudly, animated and perspiring, were at work on a huge
knoll overgrown with grass to the right of the road.

Some of them were digging, others were wheeling barrowloads of
earth along planks, while others stood about doing nothing.

Two officers were standing on the knoll, directing the men. On
seeing these peasants, who were evidently still amused by the
novelty of their position as soldiers, Pierre once more thought
of the wounded men at Mozhaysk and understood what the soldier
had meant when he said: ``They want the whole nation to fall on
them.'' The sight of these bearded peasants at work on the
battlefield, with their queer, clumsy boots and perspiring necks,
and their shirts opening from the left toward the middle,
unfastened, exposing their sunburned collarbones, impressed
Pierre more strongly with the solemnity and importance of the
moment than anything he had yet seen or heard.

% % % % % % % % % % % % % % % % % % % % % % % % % % % % % % % % %
% % % % % % % % % % % % % % % % % % % % % % % % % % % % % % % % %
% % % % % % % % % % % % % % % % % % % % % % % % % % % % % % % % %
% % % % % % % % % % % % % % % % % % % % % % % % % % % % % % % % %
% % % % % % % % % % % % % % % % % % % % % % % % % % % % % % % % %
% % % % % % % % % % % % % % % % % % % % % % % % % % % % % % % % %
% % % % % % % % % % % % % % % % % % % % % % % % % % % % % % % % %
% % % % % % % % % % % % % % % % % % % % % % % % % % % % % % % % %
% % % % % % % % % % % % % % % % % % % % % % % % % % % % % % % % %
% % % % % % % % % % % % % % % % % % % % % % % % % % % % % % % % %
% % % % % % % % % % % % % % % % % % % % % % % % % % % % % % % % %
% % % % % % % % % % % % % % % % % % % % % % % % % % % % % %

\chapter*{Chapter XXI} \ifaudio \marginpar{
\href{http://ia801407.us.archive.org/32/items/war_and_peace_10_0904_librivox/war_and_peace_10_21_tolstoy_64kb.mp3}{Audio}}
\fi

\initial{P}{ierre} stepped out of his carriage and, passing the toiling
militiamen, ascended the knoll from which, according to the
doctor, the battlefield could be seen.

It was about eleven o'clock. The sun shone somewhat to the left
and behind him and brightly lit up the enormous panorama which,
rising like an amphitheater, extended before him in the clear
rarefied atmosphere.

From above on the left, bisecting that amphitheater, wound the
Smolensk highroad, passing through a village with a white church
some five hundred paces in front of the knoll and below it. This
was Borodino.  Below the village the road crossed the river by a
bridge and, winding down and up, rose higher and higher to the
village of Valuevo visible about four miles away, where Napoleon
was then stationed. Beyond Valuevo the road disappeared into a
yellowing forest on the horizon. Far in the distance in that
birch and fir forest to the right of the road, the cross and
belfry of the Kolocha Monastery gleamed in the sun. Here and
there over the whole of that blue expanse, to right and left of
the forest and the road, smoking campfires could be seen and
indefinite masses of troops---ours and the enemy's. The ground to
the right---along the course of the Kolocha and Moskva
rivers---was broken and hilly.  Between the hollows the villages
of Bezubova and Zakharino showed in the distance. On the left the
ground was more level; there were fields of grain, and the
smoking ruins of Semenovsk, which had been burned down, could be
seen.

All that Pierre saw was so indefinite that neither the left nor
the right side of the field fully satisfied his
expectations. Nowhere could he see the battlefield he had
expected to find, but only fields, meadows, troops, woods, the
smoke of campfires, villages, mounds, and streams; and try as he
would he could descry no military \emph{position} in this place
which teemed with life, nor could he even distinguish our troops
from the enemy's.

``I must ask someone who knows,'' he thought, and addressed an
officer who was looking with curiosity at his huge unmilitary
figure.

``May I ask you,'' said Pierre, ``what village that is in
front?''

``Burdino, isn't it?'' said the officer, turning to his
companion.

``Borodino,'' the other corrected him.

The officer, evidently glad of an opportunity for a talk, moved
up to Pierre.

``Are those our men there?'' Pierre inquired.

``Yes, and there, further on, are the French,'' said the
officer. ``There they are, there... you can see them.''

``Where? Where?'' asked Pierre.

``One can see them with the naked eye... Why, there!''

The officer pointed with his hand to the smoke visible on the
left beyond the river, and the same stern and serious expression
that Pierre had noticed on many of the faces he had met came into
his face.

``Ah, those are the French! And over there?...'' Pierre pointed
to a knoll on the left, near which some troops could be seen.

``Those are ours.''

``Ah, ours! And there?...'' Pierre pointed to another knoll in
the distance with a big tree on it, near a village that lay in a
hollow where also some campfires were smoking and something black
was visible.

``That's his again,'' said the officer. (It was the Shevardino
Redoubt.)  ``It was ours yesterday, but now it is his.''

``Then how about our position?''

``Our position?'' replied the officer with a smile of
satisfaction. ``I can tell you quite clearly, because I
constructed nearly all our entrenchments. There, you see? There's
our center, at Borodino, just there,'' and he pointed to the
village in front of them with the white church. ``That's where
one crosses the Kolocha. You see down there where the rows of hay
are lying in the hollow, there's the bridge. That's our
center. Our right flank is over there''---he pointed sharply to
the right, far away in the broken ground---``That's where the
Moskva River is, and we have thrown up three redoubts there, very
strong ones. The left flank...'' here the officer paused. ``Well,
you see, that's difficult to explain... Yesterday our left flank
was there at Shevardino, you see, where the oak is, but now we
have withdrawn our left wing---now it is over there, do you see
that village and the smoke? That's Semenovsk, yes, there,'' he
pointed to Raevski's knoll. ``But the battle will hardly be
there. His having moved his troops there is only a ruse; he will
probably pass round to the right of the Moskva. But wherever it
may be, many a man will be missing tomorrow!'' he remarked.

An elderly sergeant who had approached the officer while he was
giving these explanations had waited in silence for him to finish
speaking, but at this point, evidently not liking the officer's
remark, interrupted him.

``Gabions must be sent for,'' said he sternly.

The officer appeared abashed, as though he understood that one
might think of how many men would be missing tomorrow but ought
not to speak of it.

``Well, send number three company again,'' the officer replied
hurriedly.

``And you, are you one of the doctors?''

``No, I've come on my own,'' answered Pierre, and he went down
the hill again, passing the militiamen.

``Oh, those damned fellows!'' muttered the officer who followed
him, holding his nose as he ran past the men at work.

``There they are... bringing her, coming... There they
are... They'll be here in a minute...'' voices were suddenly
heard saying; and officers, soldiers, and militiamen began
running forward along the road.

A church procession was coming up the hill from Borodino. First
along the dusty road came the infantry in ranks, bareheaded and
with arms reversed. From behind them came the sound of church
singing.

Soldiers and militiamen ran bareheaded past Pierre toward the
procession.

``They are bringing her, our Protectress!... The Iberian Mother
of God!''  someone cried.

``The Smolensk Mother of God,'' another corrected him.

The militiamen, both those who had been in the village and those
who had been at work on the battery, threw down their spades and
ran to meet the church procession. Following the battalion that
marched along the dusty road came priests in their
vestments---one little old man in a hood with attendants and
singers. Behind them soldiers and officers bore a large,
dark-faced icon with an embossed metal cover. This was the icon
that had been brought from Smolensk and had since accompanied the
army. Behind, before, and on both sides, crowds of militiamen
with bared heads walked, ran, and bowed to the ground.

At the summit of the hill they stopped with the icon; the men who
had been holding it up by the linen bands attached to it were
relieved by others, the chanters relit their censers, and service
began. The hot rays of the sun beat down vertically and a fresh
soft wind played with the hair of the bared heads and with the
ribbons decorating the icon.  The singing did not sound loud
under the open sky. An immense crowd of bareheaded officers,
soldiers, and militiamen surrounded the icon.  Behind the priest
and a chanter stood the notabilities on a spot reserved for
them. A bald general with a St. George's Cross on his neck stood
just behind the priest's back, and without crossing himself (he
was evidently a German) patiently awaited the end of the service,
which he considered it necessary to hear to the end, probably to
arouse the patriotism of the Russian people. Another general
stood in a martial pose, crossing himself by shaking his hand in
front of his chest while looking about him. Standing among the
crowd of peasants, Pierre recognized several acquaintances among
these notables, but did not look at them---his whole attention
was absorbed in watching the serious expression on the faces of
the crowd of soldiers and militiamen who were all gazing eagerly
at the icon. As soon as the tired chanters, who were singing the
service for the twentieth time that day, began lazily and
mechanically to sing: ``Save from calamity Thy servants, O Mother
of God,'' and the priest and deacon chimed in: ``For to Thee
under God we all flee as to an inviolable bulwark and
protection,'' there again kindled in all those faces the same
expression of consciousness of the solemnity of the impending
moment that Pierre had seen on the faces at the foot of the hill
at Mozhaysk and momentarily on many and many faces he had met
that morning; and heads were bowed more frequently and hair
tossed back, and sighs and the sound men made as they crossed
themselves were heard.

The crowd round the icon suddenly parted and pressed against
Pierre.  Someone, a very important personage judging by the haste
with which way was made for him, was approaching the icon.

It was Kutuzov, who had been riding round the position and on his
way back to Tatarinova had stopped where the service was being
held. Pierre recognized him at once by his peculiar figure, which
distinguished him from everybody else.

With a long overcoat on his exceedingly stout, round-shouldered
body, with uncovered white head and puffy face showing the white
ball of the eye he had lost, Kutuzov walked with plunging,
swaying gait into the crowd and stopped behind the priest. He
crossed himself with an accustomed movement, bent till he touched
the ground with his hand, and bowed his white head with a deep
sigh. Behind Kutuzov was Bennigsen and the suite. Despite the
presence of the commander-in-chief, who attracted the attention
of all the superior officers, the militiamen and soldiers
continued their prayers without looking at him.

When the service was over, Kutuzov stepped up to the icon, sank
heavily to his knees, bowed to the ground, and for a long time
tried vainly to rise, but could not do so on account of his
weakness and weight. His white head twitched with the effort. At
last he rose, kissed the icon as a child does with naively
pouting lips, and again bowed till he touched the ground with his
hand. The other generals followed his example, then the officers,
and after them with excited faces, pressing on one another,
crowding, panting, and pushing, scrambled the soldiers and
militiamen.

% % % % % % % % % % % % % % % % % % % % % % % % % % % % % % % % %
% % % % % % % % % % % % % % % % % % % % % % % % % % % % % % % % %
% % % % % % % % % % % % % % % % % % % % % % % % % % % % % % % % %
% % % % % % % % % % % % % % % % % % % % % % % % % % % % % % % % %
% % % % % % % % % % % % % % % % % % % % % % % % % % % % % % % % %
% % % % % % % % % % % % % % % % % % % % % % % % % % % % % % % % %
% % % % % % % % % % % % % % % % % % % % % % % % % % % % % % % % %
% % % % % % % % % % % % % % % % % % % % % % % % % % % % % % % % %
% % % % % % % % % % % % % % % % % % % % % % % % % % % % % % % % %
% % % % % % % % % % % % % % % % % % % % % % % % % % % % % % % % %
% % % % % % % % % % % % % % % % % % % % % % % % % % % % % % % % %
% % % % % % % % % % % % % % % % % % % % % % % % % % % % % %

\chapter*{Chapter XXII} \ifaudio \marginpar{
\href{http://ia801407.us.archive.org/32/items/war_and_peace_10_0904_librivox/war_and_peace_10_22_tolstoy_64kb.mp3}{Audio}}
\fi

\initial{S}{taggering} amid the crush, Pierre looked about him.

``Count Peter Kirilovich! How did you get here?'' said a voice.

Pierre looked round. Boris Drubetskoy, brushing his knees with
his hand (he had probably soiled them when he, too, had knelt
before the icon), came up to him smiling. Boris was elegantly
dressed, with a slightly martial touch appropriate to a
campaign. He wore a long coat and like Kutuzov had a whip slung
across his shoulder.

Meanwhile Kutuzov had reached the village and seated himself in
the shade of the nearest house, on a bench which one Cossack had
run to fetch and another had hastily covered with a rug. An
immense and brilliant suite surrounded him.

The icon was carried further, accompanied by the throng. Pierre
stopped some thirty paces from Kutuzov, talking to Boris.

He explained his wish to be present at the battle and to see the
position.

``This is what you must do,'' said Boris. ``I will do the honors
of the camp to you. You will see everything best from where Count
Bennigsen will be. I am in attendance on him, you know; I'll
mention it to him.  But if you want to ride round the position,
come along with us. We are just going to the left flank. Then
when we get back, do spend the night with me and we'll arrange a
game of cards. Of course you know Dmitri Sergeevich? Those are
his quarters,'' and he pointed to the third house in the village
of Gorki.

``But I should like to see the right flank. They say it's very
strong,'' said Pierre. ``I should like to start from the Moskva
River and ride round the whole position.''

``Well, you can do that later, but the chief thing is the left
flank.''

``Yes, yes. But where is Prince Bolkonski's regiment? Can you
point it out to me?''

``Prince Andrew's? We shall pass it and I'll take you to him.''

``What about the left flank?'' asked Pierre

``To tell you the truth, between ourselves, God only knows what
state our left flank is in,'' said Boris confidentially lowering
his voice. ``It is not at all what Count Bennigsen intended. He
meant to fortify that knoll quite differently, but...'' Boris
shrugged his shoulders, ``his Serene Highness would not have it,
or someone persuaded him. You see...'' but Boris did not finish,
for at that moment Kaysarov, Kutuzov's adjutant, came up to
Pierre. ``Ah, Kaysarov!'' said Boris, addressing him with an
unembarrassed smile, ``I was just trying to explain our position
to the count. It is amazing how his Serene Highness could so
foresee the intentions of the French!''

``You mean the left flank?'' asked Kaysarov.

``Yes, exactly; the left flank is now extremely strong.''

Though Kutuzov had dismissed all unnecessary men from the staff,
Boris had contrived to remain at headquarters after the
changes. He had established himself with Count Bennigsen, who,
like all on whom Boris had been in attendance, considered young
Prince Drubetskoy an invaluable man.

In the higher command there were two sharply defined parties:
Kutuzov's party and that of Bennigsen, the chief of staff. Boris
belonged to the latter and no one else, while showing servile
respect to Kutuzov, could so create an impression that the old
fellow was not much good and that Bennigsen managed
everything. Now the decisive moment of battle had come when
Kutuzov would be destroyed and the power pass to Bennigsen, or
even if Kutuzov won the battle it would be felt that everything
was done by Bennigsen. In any case many great rewards would have
to be given for tomorrow's action, and new men would come to the
front. So Boris was full of nervous vivacity all day.

After Kaysarov, others whom Pierre knew came up to him, and he
had not time to reply to all the questions about Moscow that were
showered upon him, or to listen to all that was told him. The
faces all expressed animation and apprehension, but it seemed to
Pierre that the cause of the excitement shown in some of these
faces lay chiefly in questions of personal success; his mind,
however, was occupied by the different expression he saw on other
faces---an expression that spoke not of personal matters but of
the universal questions of life and death.  Kutuzov noticed
Pierre's figure and the group gathered round him.

``Call him to me,'' said Kutuzov.

An adjutant told Pierre of his Serene Highness' wish, and Pierre
went toward Kutuzov's bench. But a militiaman got there before
him. It was Dolokhov.

``How did that fellow get here?'' asked Pierre.

``He's a creature that wriggles in anywhere!'' was the
answer. ``He has been degraded, you know. Now he wants to bob up
again. He's been proposing some scheme or other and has crawled
into the enemy's picket line at night... He's a brave fellow.''

Pierre took off his hat and bowed respectfully to Kutuzov.

``I concluded that if I reported to your Serene Highness you
might send me away or say that you knew what I was reporting, but
then I shouldn't lose anything...'' Dolokhov was saying.

``Yes, yes.''

``But if I were right, I should be rendering a service to my
Fatherland for which I am ready to die.''

``Yes, yes.''

``And should your Serene Highness require a man who will not
spare his skin, please think of me... Perhaps I may prove useful
to your Serene Highness.''

``Yes... Yes...'' Kutuzov repeated, his laughing eye narrowing
more and more as he looked at Pierre.

Just then Boris, with his courtierlike adroitness, stepped up to
Pierre's side near Kutuzov and in a most natural manner, without
raising his voice, said to Pierre, as though continuing an
interrupted conversation:

``The militia have put on clean white shirts to be ready to
die. What heroism, Count!''

Boris evidently said this to Pierre in order to be overheard by
his Serene Highness. He knew Kutuzov's attention would be caught
by those words, and so it was.

``What are you saying about the militia?'' he asked Boris.

``Preparing for tomorrow, your Serene Highness---for death---they
have put on clean shirts.''

``Ah... a wonderful, a matchless people!'' said Kutuzov; and he
closed his eyes and swayed his head. ``A matchless people!'' he
repeated with a sigh.

``So you want to smell gunpowder?'' he said to Pierre. ``Yes,
it's a pleasant smell. I have the honor to be one of your wife's
adorers. Is she well? My quarters are at your service.''

And as often happens with old people, Kutuzov began looking about
absent-mindedly as if forgetting all he wanted to say or do.

Then, evidently remembering what he wanted, he beckoned to Andrew
Kaysarov, his adjutant's brother.

``Those verses... those verses of Marin's... how do they go, eh?
Those he wrote about Gerakov: 'Lectures for the corps
inditing'... Recite them, recite them!'' said he, evidently
preparing to laugh.

Kaysarov recited... Kutuzov smilingly nodded his head to the
rhythm of the verses.

When Pierre had left Kutuzov, Dolokhov came up to him and took
his hand.

``I am very glad to meet you here, Count,'' he said aloud,
regardless of the presence of strangers and in a particularly
resolute and solemn tone. ``On the eve of a day when God alone
knows who of us is fated to survive, I am glad of this
opportunity to tell you that I regret the misunderstandings that
occurred between us and should wish you not to have any ill
feeling for me. I beg you to forgive me.''

Pierre looked at Dolokhov with a smile, not knowing what to say
to him.  With tears in his eyes Dolokhov embraced Pierre and
kissed him.

Boris said a few words to his general, and Count Bennigsen turned
to Pierre and proposed that he should ride with him along the
line.

``It will interest you,'' said he.

``Yes, very much,'' replied Pierre.

Half an hour later Kutuzov left for Tatarinova, and Bennigsen and
his suite, with Pierre among them, set out on their ride along
the line.

% % % % % % % % % % % % % % % % % % % % % % % % % % % % % % % % %
% % % % % % % % % % % % % % % % % % % % % % % % % % % % % % % % %
% % % % % % % % % % % % % % % % % % % % % % % % % % % % % % % % %
% % % % % % % % % % % % % % % % % % % % % % % % % % % % % % % % %
% % % % % % % % % % % % % % % % % % % % % % % % % % % % % % % % %
% % % % % % % % % % % % % % % % % % % % % % % % % % % % % % % % %
% % % % % % % % % % % % % % % % % % % % % % % % % % % % % % % % %
% % % % % % % % % % % % % % % % % % % % % % % % % % % % % % % % %
% % % % % % % % % % % % % % % % % % % % % % % % % % % % % % % % %
% % % % % % % % % % % % % % % % % % % % % % % % % % % % % % % % %
% % % % % % % % % % % % % % % % % % % % % % % % % % % % % % % % %
% % % % % % % % % % % % % % % % % % % % % % % % % % % % % %

\chapter*{Chapter XXIII} \ifaudio \marginpar{
\href{http://ia801407.us.archive.org/32/items/war_and_peace_10_0904_librivox/war_and_peace_10_23_tolstoy_64kb.mp3}{Audio}}
\fi

\initial{F}{rom} Gorki, Bennigsen descended the highroad to the bridge which,
when they had looked at it from the hill, the officer had pointed
out as being the center of our position and where rows of
fragrant new-mown hay lay by the riverside. They rode across that
bridge into the village of Borodino and thence turned to the
left, passing an enormous number of troops and guns, and came to
a high knoll where militiamen were digging.  This was the
redoubt, as yet unnamed, which afterwards became known as the
Raevski Redoubt, or the Knoll Battery, but Pierre paid no special
attention to it. He did not know that it would become more
memorable to him than any other spot on the plain of Borodino.

They then crossed the hollow to Semenovsk, where the soldiers
were dragging away the last logs from the huts and barns. Then
they rode downhill and uphill, across a ryefield trodden and
beaten down as if by hail, following a track freshly made by the
artillery over the furrows of the plowed land, and reached some
fleches\footnote{A kind of entrenchment.} which were still being
dug.

At the fleches Bennigsen stopped and began looking at the
Shevardino Redoubt opposite, which had been ours the day before
and where several horsemen could be descried. The officers said
that either Napoleon or Murat was there, and they all gazed
eagerly at this little group of horsemen. Pierre also looked at
them, trying to guess which of the scarcely discernible figures
was Napoleon. At last those mounted men rode away from the mound
and disappeared.

Bennigsen spoke to a general who approached him, and began
explaining the whole position of our troops. Pierre listened to
him, straining each faculty to understand the essential points of
the impending battle, but was mortified to feel that his mental
capacity was inadequate for the task. He could make nothing of
it. Bennigsen stopped speaking and, noticing that Pierre was
listening, suddenly said to him:

``I don't think this interests you?''

``On the contrary it's very interesting!'' replied Pierre not
quite truthfully.

From the fleches they rode still farther to the left, along a
road winding through a thick, low-growing birch wood. In the
middle of the wood a brown hare with white feet sprang out and,
scared by the tramp of the many horses, grew so confused that it
leaped along the road in front of them for some time, arousing
general attention and laughter, and only when several voices
shouted at it did it dart to one side and disappear in the
thicket. After going through the wood for about a mile and a half
they came out on a glade where troops of Tuchkov's corps were
stationed to defend the left flank.

Here, at the extreme left flank, Bennigsen talked a great deal
and with much heat, and, as it seemed to Pierre, gave orders of
great military importance. In front of Tuchkov's troops was some
high ground not occupied by troops. Bennigsen loudly criticized
this mistake, saying that it was madness to leave a height which
commanded the country around unoccupied and to place troops below
it. Some of the generals expressed the same opinion. One in
particular declared with martial heat that they were put there to
be slaughtered. Bennigsen on his own authority ordered the troops
to occupy the high ground. This disposition on the left flank
increased Pierre's doubt of his own capacity to understand
military matters. Listening to Bennigsen and the generals
criticizing the position of the troops behind the hill, he quite
understood them and shared their opinion, but for that very
reason he could not understand how the man who put them there
behind the hill could have made so gross and palpable a blunder.

Pierre did not know that these troops were not, as Bennigsen
supposed, put there to defend the position, but were in a
concealed position as an ambush, that they should not be seen and
might be able to strike an approaching enemy
unexpectedly. Bennigsen did not know this and moved the troops
forward according to his own ideas without mentioning the matter
to the commander-in-chief.

% % % % % % % % % % % % % % % % % % % % % % % % % % % % % % % % %
% % % % % % % % % % % % % % % % % % % % % % % % % % % % % % % % %
% % % % % % % % % % % % % % % % % % % % % % % % % % % % % % % % %
% % % % % % % % % % % % % % % % % % % % % % % % % % % % % % % % %
% % % % % % % % % % % % % % % % % % % % % % % % % % % % % % % % %
% % % % % % % % % % % % % % % % % % % % % % % % % % % % % % % % %
% % % % % % % % % % % % % % % % % % % % % % % % % % % % % % % % %
% % % % % % % % % % % % % % % % % % % % % % % % % % % % % % % % %
% % % % % % % % % % % % % % % % % % % % % % % % % % % % % % % % %
% % % % % % % % % % % % % % % % % % % % % % % % % % % % % % % % %
% % % % % % % % % % % % % % % % % % % % % % % % % % % % % % % % %
% % % % % % % % % % % % % % % % % % % % % % % % % % % % % %

\chapter*{Chapter XXIV} \ifaudio \marginpar{
\href{http://ia801407.us.archive.org/32/items/war_and_peace_10_0904_librivox/war_and_peace_10_24_tolstoy_64kb.mp3}{Audio}}
\fi

\initial{O}{n} that bright evening of August 25, Prince Andrew lay leaning on
his elbow in a broken-down shed in the village of Knyazkovo at
the further end of his regiment's encampment. Through a gap in
the broken wall he could see, beside the wooden fence, a row of
thirty year-old birches with their lower branches lopped off, a
field on which shocks of oats were standing, and some bushes near
which rose the smoke of campfires---the soldiers' kitchens.

Narrow and burdensome and useless to anyone as his life now
seemed to him, Prince Andrew on the eve of battle felt agitated
and irritable as he had done seven years before at Austerlitz.

He had received and given the orders for next day's battle and
had nothing more to do. But his thoughts---the simplest,
clearest, and therefore most terrible thoughts---would give him
no peace. He knew that tomorrow's battle would be the most
terrible of all he had taken part in, and for the first time in
his life the possibility of death presented itself to him---not
in relation to any worldly matter or with reference to its effect
on others, but simply in relation to himself, to his own
soul---vividly, plainly, terribly, and almost as a certainty. And
from the height of this perception all that had previously
tormented and preoccupied him suddenly became illumined by a cold
white light without shadows, without perspective, without
distinction of outline. All life appeared to him like
magic-lantern pictures at which he had long been gazing by
artificial light through a glass. Now he suddenly saw those badly
daubed pictures in clear daylight and without a glass. ``Yes,
yes!  There they are, those false images that agitated,
enraptured, and tormented me,'' said he to himself, passing in
review the principal pictures of the magic lantern of life and
regarding them now in the cold white daylight of his clear
perception of death. ``There they are, those rudely painted
figures that once seemed splendid and mysterious. Glory, the good
of society, love of a woman, the Fatherland itself---how
important these pictures appeared to me, with what profound
meaning they seemed to be filled! And it is all so simple, pale,
and crude in the cold white light of this morning which I feel is
dawning for me.'' The three great sorrows of his life held his
attention in particular: his love for a woman, his father's
death, and the French invasion which had overrun half
Russia. ``Love... that little girl who seemed to me brimming over
with mystic forces! Yes, indeed, I loved her. I made romantic
plans of love and happiness with her! Oh, what a boy I was!'' he
said aloud bitterly. ``Ah me! I believed in some ideal love which
was to keep her faithful to me for the whole year of my absence!
Like the gentle dove in the fable she was to pine apart from
me... But it was much simpler really... It was all very simple
and horrible.''

``When my father built Bald Hills he thought the place was his:
his land, his air, his peasants. But Napoleon came and swept him
aside, unconscious of his existence, as he might brush a chip
from his path, and his Bald Hills and his whole life fell to
pieces. Princess Mary says it is a trial sent from above. What is
the trial for, when he is not here and will never return? He is
not here! For whom then is the trial intended? The Fatherland,
the destruction of Moscow! And tomorrow I shall be killed,
perhaps not even by a Frenchman but by one of our own men, by a
soldier discharging a musket close to my ear as one of them did
yesterday, and the French will come and take me by head and heels
and fling me into a hole that I may not stink under their noses,
and new conditions of life will arise, which will seem quite
ordinary to others and about which I shall know nothing. I shall
not exist...''

He looked at the row of birches shining in the sunshine, with
their motionless green and yellow foliage and white bark. ``To
die... to be killed tomorrow... That I should not exist... That
all this should still be, but no me...''

And the birches with their light and shade, the curly clouds, the
smoke of the campfires, and all that was around him changed and
seemed terrible and menacing. A cold shiver ran down his
spine. He rose quickly, went out of the shed, and began to walk
about.

After he had returned, voices were heard outside the
shed. ``Who's that?''  he cried.

The red-nosed Captain Timokhin, formerly Dolokhov's squadron
commander, but now from lack of officers a battalion commander,
shyly entered the shed followed by an adjutant and the regimental
paymaster.

Prince Andrew rose hastily, listened to the business they had
come about, gave them some further instructions, and was about to
dismiss them when he heard a familiar, lisping, voice behind the
shed.

``Devil take it!'' said the voice of a man stumbling over
something.

Prince Andrew looked out of the shed and saw Pierre, who had
tripped over a pole on the ground and had nearly fallen, coming
his way. It was unpleasant to Prince Andrew to meet people of his
own set in general, and Pierre especially, for he reminded him of
all the painful moments of his last visit to Moscow.

``You? What a surprise!'' said he. ``What brings you here? This
is unexpected!''

As he said this his eyes and face expressed more than
coldness---they expressed hostility, which Pierre noticed at
once. He had approached the shed full of animation, but on seeing
Prince Andrew's face he felt constrained and ill at ease.

``I have come... simply... you know... come... it interests me,''
said Pierre, who had so often that day senselessly repeated that
word ``interesting.'' ``I wish to see the battle.''

``Oh yes, and what do the masonic brothers say about war? How
would they stop it?'' said Prince Andrew sarcastically. ``Well,
and how's Moscow? And my people? Have they reached Moscow at
last?'' he asked seriously.

``Yes, they have. Julie Drubetskaya told me so. I went to see
them, but missed them. They have gone to your estate near
Moscow.''

% % % % % % % % % % % % % % % % % % % % % % % % % % % % % % % % %
% % % % % % % % % % % % % % % % % % % % % % % % % % % % % % % % %
% % % % % % % % % % % % % % % % % % % % % % % % % % % % % % % % %
% % % % % % % % % % % % % % % % % % % % % % % % % % % % % % % % %
% % % % % % % % % % % % % % % % % % % % % % % % % % % % % % % % %
% % % % % % % % % % % % % % % % % % % % % % % % % % % % % % % % %
% % % % % % % % % % % % % % % % % % % % % % % % % % % % % % % % %
% % % % % % % % % % % % % % % % % % % % % % % % % % % % % % % % %
% % % % % % % % % % % % % % % % % % % % % % % % % % % % % % % % %
% % % % % % % % % % % % % % % % % % % % % % % % % % % % % % % % %
% % % % % % % % % % % % % % % % % % % % % % % % % % % % % % % % %
% % % % % % % % % % % % % % % % % % % % % % % % % % % % % %

\chapter*{Chapter XXV} \ifaudio \marginpar{
\href{http://ia801407.us.archive.org/32/items/war_and_peace_10_0904_librivox/war_and_peace_10_25_tolstoy_64kb.mp3}{Audio}}
\fi

\initial{T}{he} officers were about to take leave, but Prince Andrew,
apparently reluctant to be left alone with his friend, asked them
to stay and have tea. Seats were brought in and so was the
tea. The officers gazed with surprise at Pierre's huge stout
figure and listened to his talk of Moscow and the position of our
army, round which he had ridden. Prince Andrew remained silent,
and his expression was so forbidding that Pierre addressed his
remarks chiefly to the good-natured battalion commander.

``So you understand the whole position of our troops?'' Prince
Andrew interrupted him.

``Yes---that is, how do you mean?'' said Pierre. ``Not being a
military man I can't say I have understood it fully, but I
understand the general position.''

``Well, then, you know more than anyone else, be it who it may,''
said Prince Andrew.

``Oh!'' said Pierre, looking over his spectacles in perplexity at
Prince Andrew. ``Well, and what do you think of Kutuzov's
appointment?'' he asked.

``I was very glad of his appointment, that's all I know,''
replied Prince Andrew.

``And tell me your opinion of Barclay de Tolly. In Moscow they
are saying heaven knows what about him... What do you think of
him?''

``Ask them,'' replied Prince Andrew, indicating the officers.

Pierre looked at Timokhin with the condescendingly interrogative
smile with which everybody involuntarily addressed that officer.

``We see light again, since his Serenity has been appointed, your
excellency,'' said Timokhin timidly, and continually turning to
glance at his colonel.

``Why so?'' asked Pierre.

``Well, to mention only firewood and fodder, let me inform
you. Why, when we were retreating from Sventsyani we dare not
touch a stick or a wisp of hay or anything. You see, we were
going away, so he would get it all; wasn't it so, your
excellency?'' and again Timokhin turned to the prince.  ``But we
daren't. In our regiment two officers were court-martialed for
that kind of thing. But when his Serenity took command everything
became straight forward. Now we see light...''

``Then why was it forbidden?''

Timokhin looked about in confusion, not knowing what or how to
answer such a question. Pierre put the same question to Prince
Andrew.

``Why, so as not to lay waste the country we were abandoning to
the enemy,'' said Prince Andrew with venomous irony. ``It is very
sound: one can't permit the land to be pillaged and accustom the
troops to marauding. At Smolensk too he judged correctly that the
French might outflank us, as they had larger forces. But he could
not understand this,'' cried Prince Andrew in a shrill voice that
seemed to escape him involuntarily: ``he could not understand
that there, for the first time, we were fighting for Russian
soil, and that there was a spirit in the men such as I had never
seen before, that we had held the French for two days, and that
that success had increased our strength tenfold. He ordered us to
retreat, and all our efforts and losses went for nothing.  He had
no thought of betraying us, he tried to do the best he could, he
thought out everything, and that is why he is unsuitable. He is
unsuitable now, just because he plans out everything very
thoroughly and accurately as every German has to. How can I
explain?... Well, say your father has a German valet, and he is a
splendid valet and satisfies your father's requirements better
than you could, then it's all right to let him serve. But if your
father is mortally sick you'll send the valet away and attend to
your father with your own unpracticed, awkward hands, and will
soothe him better than a skilled man who is a stranger could.  So
it has been with Barclay. While Russia was well, a foreigner
could serve her and be a splendid minister; but as soon as she is
in danger she needs one of her own kin. But in your club they
have been making him out a traitor! They slander him as a
traitor, and the only result will be that afterwards, ashamed of
their false accusations, they will make him out a hero or a
genius instead of a traitor, and that will be still more
unjust. He is an honest and very punctilious German.''

``And they say he's a skillful commander,'' rejoined Pierre.

``I don't understand what is meant by 'a skillful commander,'{}''
replied Prince Andrew ironically.

``A skillful commander?'' replied Pierre. ``Why, one who foresees
all contingencies... and foresees the adversary's intentions.''

``But that's impossible,'' said Prince Andrew as if it were a
matter settled long ago.

Pierre looked at him in surprise.

``And yet they say that war is like a game of chess?'' he
remarked.

``Yes,'' replied Prince Andrew, ``but with this little
difference, that in chess you may think over each move as long as
you please and are not limited for time, and with this difference
too, that a knight is always stronger than a pawn, and two pawns
are always stronger than one, while in war a battalion is
sometimes stronger than a division and sometimes weaker than a
company. The relative strength of bodies of troops can never be
known to anyone. Believe me,'' he went on, ``if things depended
on arrangements made by the staff, I should be there making
arrangements, but instead of that I have the honor to serve here
in the regiment with these gentlemen, and I consider that on us
tomorrow's battle will depend and not on those others... Success
never depends, and never will depend, on position, or equipment,
or even on numbers, and least of all on position.''

``But on what then?''

``On the feeling that is in me and in him,'' he pointed to
Timokhin, ``and in each soldier.''

Prince Andrew glanced at Timokhin, who looked at his commander in
alarm and bewilderment. In contrast to his former reticent
taciturnity Prince Andrew now seemed excited. He could apparently
not refrain from expressing the thoughts that had suddenly
occurred to him.

``A battle is won by those who firmly resolve to win it! Why did
we lose the battle at Austerlitz? The French losses were almost
equal to ours, but very early we said to ourselves that we were
losing the battle, and we did lose it. And we said so because we
had nothing to fight for there, we wanted to get away from the
battlefield as soon as we could.  'We've lost, so let us run,'
and we ran. If we had not said that till the evening, heaven
knows what might not have happened. But tomorrow we shan't say
it! You talk about our position, the left flank weak and the
right flank too extended,'' he went on. ``That's all nonsense,
there's nothing of the kind. But what awaits us tomorrow? A
hundred million most diverse chances which will be decided on the
instant by the fact that our men or theirs run or do not run, and
that this man or that man is killed, but all that is being done
at present is only play. The fact is that those men with whom you
have ridden round the position not only do not help matters, but
hinder. They are only concerned with their own petty interests.''

``At such a moment?'' said Pierre reproachfully.

``At such a moment!'' Prince Andrew repeated. ``To them it is
only a moment affording opportunities to undermine a rival and
obtain an extra cross or ribbon. For me tomorrow means this: a
Russian army of a hundred thousand and a French army of a hundred
thousand have met to fight, and the thing is that these two
hundred thousand men will fight and the side that fights more
fiercely and spares itself least will win. And if you like I will
tell you that whatever happens and whatever muddles those at the
top may make, we shall win tomorrow's battle. Tomorrow, happen
what may, we shall win!''

``There now, your excellency! That's the truth, the real truth,''
said Timokhin. ``Who would spare himself now? The soldiers in my
battalion, believe me, wouldn't drink their vodka! 'It's not the
day for that!'  they say.''

All were silent. The officers rose. Prince Andrew went out of the
shed with them, giving final orders to the adjutant. After they
had gone Pierre approached Prince Andrew and was about to start a
conversation when they heard the clatter of three horses' hoofs
on the road not far from the shed, and looking in that direction
Prince Andrew recognized Wolzogen and Clausewitz accompanied by a
Cossack. They rode close by continuing to converse, and Prince
Andrew involuntarily heard these words:

``Der Krieg muss in Raum verlegt werden. Der Ansicht kann ich
nicht genug Preis geben,''\footnote{``The war must be extended
widely. I cannot sufficiently commend that view.''} said one of
them.

``Oh, ja,'' said the other, ``der Zweck ist nur den Feind zu
schw??chen, so kann man gewiss nicht den Verlust der
Privat-Personen in Achtung nehmen.''\footnote{``Oh, yes, the only
aim is to weaken the enemy, so of course one cannot take into
account the loss of private individuals.''}

``Oh, no,'' agreed the other.

``Extend widely!'' said Prince Andrew with an angry snort, when
they had ridden past. ``In that 'extend' were my father, son, and
sister, at Bald Hills. That's all the same to him! That's what I
was saying to you---those German gentlemen won't win the battle
tomorrow but will only make all the mess they can, because they
have nothing in their German heads but theories not worth an
empty eggshell and haven't in their hearts the one thing needed
tomorrow---that which Timokhin has. They have yielded up all
Europe to him, and have now come to teach us. Fine teachers!''
and again his voice grew shrill.

``So you think we shall win tomorrow's battle?'' asked Pierre.

``Yes, yes,'' answered Prince Andrew absently. ``One thing I
would do if I had the power,'' he began again, ``I would not take
prisoners. Why take prisoners? It's chivalry! The French have
destroyed my home and are on their way to destroy Moscow, they
have outraged and are outraging me every moment. They are my
enemies. In my opinion they are all criminals.  And so thinks
Timokhin and the whole army. They should be executed!  Since they
are my foes they cannot be my friends, whatever may have been
said at Tilsit.''

``Yes, yes,'' muttered Pierre, looking with shining eyes at
Prince Andrew.  ``I quite agree with you!''

The question that had perturbed Pierre on the Mozhaysk hill and
all that day now seemed to him quite clear and completely
solved. He now understood the whole meaning and importance of
this war and of the impending battle. All he had seen that day,
all the significant and stern expressions on the faces he had
seen in passing, were lit up for him by a new light. He
understood that latent heat (as they say in physics) of
patriotism which was present in all these men he had seen, and
this explained to him why they all prepared for death calmly, and
as it were lightheartedly.

``Not take prisoners,'' Prince Andrew continued: ``That by itself
would quite change the whole war and make it less cruel. As it is
we have played at war---that's what's vile! We play at
magnanimity and all that stuff. Such magnanimity and sensibility
are like the magnanimity and sensibility of a lady who faints
when she sees a calf being killed: she is so kindhearted that she
can't look at blood, but enjoys eating the calf served up with
sauce. They talk to us of the rules of war, of chivalry, of flags
of truce, of mercy to the unfortunate and so on. It's all
rubbish! I saw chivalry and flags of truce in 1805; they
humbugged us and we humbugged them. They plunder other people's
houses, issue false paper money, and worst of all they kill my
children and my father, and then talk of rules of war and
magnanimity to foes! Take no prisoners, but kill and be killed!
He who has come to this as I have through the same
sufferings...''

Prince Andrew, who had thought it was all the same to him whether
or not Moscow was taken as Smolensk had been, was suddenly
checked in his speech by an unexpected cramp in his throat. He
paced up and down a few times in silence, but his eyes glittered
feverishly and his lips quivered as he began speaking.

``If there was none of this magnanimity in war, we should go to
war only when it was worth while going to certain death, as
now. Then there would not be war because Paul Ivanovich had
offended Michael Ivanovich. And when there was a war, like this
one, it would be war! And then the determination of the troops
would be quite different. Then all these Westphalians and
Hessians whom Napoleon is leading would not follow him into
Russia, and we should not go to fight in Austria and Prussia
without knowing why. War is not courtesy but the most horrible
thing in life; and we ought to understand that and not play at
war. We ought to accept this terrible necessity sternly and
seriously. It all lies in that: get rid of falsehood and let war
be war and not a game. As it is now, war is the favorite pastime
of the idle and frivolous. The military calling is the most
highly honored.

''But what is war? What is needed for success in warfare? What
are the habits of the military? The aim of war is murder; the
methods of war are spying, treachery, and their encouragement,
the ruin of a country's inhabitants, robbing them or stealing to
provision the army, and fraud and falsehood termed military
craft. The habits of the military class are the absence of
freedom, that is, discipline, idleness, ignorance, cruelty,
debauchery, and drunkenness. And in spite of all this it is the
highest class, respected by everyone. All the kings, except the
Chinese, wear military uniforms, and he who kills most people
receives the highest rewards.

``They meet, as we shall meet tomorrow, to murder one another;
they kill and maim tens of thousands, and then have thanksgiving
services for having killed so many people (they even exaggerate
the number), and they announce a victory, supposing that the more
people they have killed the greater their achievement. How does
God above look at them and hear them?'' exclaimed Prince Andrew
in a shrill, piercing voice. ``Ah, my friend, it has of late
become hard for me to live. I see that I have begun to understand
too much. And it doesn't do for man to taste of the tree of
knowledge of good and evil... Ah, well, it's not for long!'' he
added.

``However, you're sleepy, and it's time for me to sleep. Go back
to Gorki!'' said Prince Andrew suddenly.

``Oh no!'' Pierre replied, looking at Prince Andrew with
frightened, compassionate eyes.

``Go, go! Before a battle one must have one's sleep out,''
repeated Prince Andrew.

He came quickly up to Pierre and embraced and kissed
him. ``Good-bye, be off!'' he shouted. ``Whether we meet again or
not...'' and turning away hurriedly he entered the shed.

It was already dark, and Pierre could not make out whether the
expression of Prince Andrew's face was angry or tender.

For some time he stood in silence considering whether he should
follow him or go away. ``No, he does not want it!'' Pierre
concluded. ``And I know that this is our last meeting!'' He
sighed deeply and rode back to Gorki.

On re-entering the shed Prince Andrew lay down on a rug, but he
could not sleep.

He closed his eyes. One picture succeeded another in his
imagination. On one of them he dwelt long and joyfully. He
vividly recalled an evening in Petersburg. Natasha with animated
and excited face was telling him how she had gone to look for
mushrooms the previous summer and had lost her way in the big
forest. She incoherently described the depths of the forest, her
feelings, and a talk with a beekeeper she met, and constantly
interrupted her story to say: ``No, I can't! I'm not telling it
right; no, you don't understand,'' though he encouraged her by
saying that he did understand, and he really had understood all
she wanted to say. But Natasha was not satisfied with her own
words: she felt that they did not convey the passionately poetic
feeling she had experienced that day and wished to convey. ``He
was such a delightful old man, and it was so dark in the
forest... and he had such kind... No, I can't describe it,'' she
had said, flushed and excited. Prince Andrew smiled now the same
happy smile as then when he had looked into her eyes. ``I
understood her,'' he thought. ``I not only understood her, but it
was just that inner, spiritual force, that sincerity, that
frankness of soul---that very soul of hers which seemed to be
fettered by her bodyit was that soul I loved in her... loved so
strongly and happily...'' and suddenly he remembered how his love
had ended. ``He did not need anything of that kind. He neither
saw nor understood anything of the sort. He only saw in her a
pretty and fresh young girl, with whom he did not deign to unite
his fate. And I?... and he is still alive and gay!''

Prince Andrew jumped up as if someone had burned him, and again
began pacing up and down in front of the shed.

% % % % % % % % % % % % % % % % % % % % % % % % % % % % % % % % %
% % % % % % % % % % % % % % % % % % % % % % % % % % % % % % % % %
% % % % % % % % % % % % % % % % % % % % % % % % % % % % % % % % %
% % % % % % % % % % % % % % % % % % % % % % % % % % % % % % % % %
% % % % % % % % % % % % % % % % % % % % % % % % % % % % % % % % %
% % % % % % % % % % % % % % % % % % % % % % % % % % % % % % % % %
% % % % % % % % % % % % % % % % % % % % % % % % % % % % % % % % %
% % % % % % % % % % % % % % % % % % % % % % % % % % % % % % % % %
% % % % % % % % % % % % % % % % % % % % % % % % % % % % % % % % %
% % % % % % % % % % % % % % % % % % % % % % % % % % % % % % % % %
% % % % % % % % % % % % % % % % % % % % % % % % % % % % % % % % %
% % % % % % % % % % % % % % % % % % % % % % % % % % % % % %

\chapter*{Chapter XXVI} \ifaudio \marginpar{
\href{http://ia801407.us.archive.org/32/items/war_and_peace_10_0904_librivox/war_and_peace_10_26_tolstoy_64kb.mp3}{Audio}}
\fi

\initial{O}{n} August 25, the eve of the battle of Borodino, M. de Beausset,
prefect of the French Emperor's palace, arrived at Napoleon's
quarters at Valuevo with Colonel Fabvier, the former from Paris
and the latter from Madrid.

Donning his court uniform, M. de Beausset ordered a box he had
brought for the Emperor to be carried before him and entered the
first compartment of Napoleon's tent, where he began opening the
box while conversing with Napoleon's aides-de-camp who surrounded
him.

Fabvier, not entering the tent, remained at the entrance talking
to some generals of his acquaintance.

The Emperor Napoleon had not yet left his bedroom and was
finishing his toilet. Slightly snorting and grunting, he
presented now his back and now his plump hairy chest to the brush
with which his valet was rubbing him down. Another valet, with
his finger over the mouth of a bottle, was sprinkling Eau de
Cologne on the Emperor's pampered body with an expression which
seemed to say that he alone knew where and how much Eau de
Cologne should be sprinkled. Napoleon's short hair was wet and
matted on the forehead, but his face, though puffy and yellow,
expressed physical satisfaction. ``Go on, harder, go on!'' he
muttered to the valet who was rubbing him, slightly twitching and
grunting. An aide-de-camp, who had entered the bedroom to report
to the Emperor the number of prisoners taken in yesterday's
action, was standing by the door after delivering his message,
awaiting permission to withdraw. Napoleon, frowning, looked at
him from under his brows.

``No prisoners!'' said he, repeating the aide-de-camp's
words. ``They are forcing us to exterminate them. So much the
worse for the Russian army... Go on... harder, harder!'' he
muttered, hunching his back and presenting his fat shoulders.

``All right. Let Monsieur de Beausset enter, and Fabvier too,''
he said, nodding to the aide-de-camp.

``Yes, sire,'' and the aide-de-camp disappeared through the door
of the tent.

Two valets rapidly dressed His Majesty, and wearing the blue
uniform of the Guards he went with firm quick steps to the
reception room.

De Beausset's hands meanwhile were busily engaged arranging the
present he had brought from the Empress, on two chairs directly
in front of the entrance. But Napoleon had dressed and come out
with such unexpected rapidity that he had not time to finish
arranging the surprise.

Napoleon noticed at once what they were about and guessed that
they were not ready. He did not wish to deprive them of the
pleasure of giving him a surprise, so he pretended not to see de
Beausset and called Fabvier to him, listening silently and with a
stern frown to what Fabvier told him of the heroism and devotion
of his troops fighting at Salamanca, at the other end of Europe,
with but one thought---to be worthy of their Emperor---and but
one fear---to fail to please him. The result of that battle had
been deplorable. Napoleon made ironic remarks during Fabvier's
account, as if he had not expected that matters could go
otherwise in his absence.

``I must make up for that in Moscow,'' said Napoleon. ``I'll see
you later,'' he added, and summoned de Beausset, who by that time
had prepared the surprise, having placed something on the chairs
and covered it with a cloth.

De Beausset bowed low, with that courtly French bow which only
the old retainers of the Bourbons knew how to make, and
approached him, presenting an envelope.

Napoleon turned to him gaily and pulled his ear.

``You have hurried here. I am very glad. Well, what is Paris
saying?'' he asked, suddenly changing his former stern expression
for a most cordial tone.

``Sire, all Paris regrets your absence,'' replied de Beausset as
was proper.

But though Napoleon knew that de Beausset had to say something of
this kind, and though in his lucid moments he knew it was untrue,
he was pleased to hear it from him. Again he honored him by
touching his ear.

``I am very sorry to have made you travel so far,'' said he.

``Sire, I expected nothing less than to find you at the gates of
Moscow,'' replied de Beausset.

Napoleon smiled and, lifting his head absent-mindedly, glanced to
the right. An aide-de-camp approached with gliding steps and
offered him a gold snuffbox, which he took.

``Yes, it has happened luckily for you,'' he said, raising the
open snuffbox to his nose. ``You are fond of travel, and in three
days you will see Moscow. You surely did not expect to see that
Asiatic capital.  You will have a pleasant journey.''

De Beausset bowed gratefully at this regard for his taste for
travel (of which he had not till then been aware).

``Ha, what's this?'' asked Napoleon, noticing that all the
courtiers were looking at something concealed under a cloth.

With courtly adroitness de Beausset half turned and without
turning his back to the Emperor retired two steps, twitching off
the cloth at the same time, and said:

``A present to Your Majesty from the Empress.''

It was a portrait, painted in bright colors by Gerard, of the son
borne to Napoleon by the daughter of the Emperor of Austria, the
boy whom for some reason everyone called ``The King of Rome.''

A very pretty curly-headed boy with a look of the Christ in the
Sistine Madonna was depicted playing at stick and ball. The ball
represented the terrestrial globe and the stick in his other hand
a scepter.

Though it was not clear what the artist meant to express by
depicting the so-called King of Rome spiking the earth with a
stick, the allegory apparently seemed to Napoleon, as it had done
to all who had seen it in Paris, quite clear and very pleasing.

``The King of Rome!'' he said, pointing to the portrait with a
graceful gesture. ``Admirable!''

With the natural capacity of an Italian for changing the
expression of his face at will, he drew nearer to the portrait
and assumed a look of pensive tenderness. He felt that what he
now said and did would be historical, and it seemed to him that
it would now be best for him---whose grandeur enabled his son to
play stick and ball with the terrestrial globe---to show, in
contrast to that grandeur, the simplest paternal tenderness. His
eyes grew dim, he moved forward, glanced round at a chair (which
seemed to place itself under him), and sat down on it before the
portrait. At a single gesture from him everyone went out on
tiptoe, leaving the great man to himself and his emotion.

Having sat still for a while he touched---himself not knowing
why---the thick spot of paint representing the highest light in
the portrait, rose, and recalled de Beausset and the officer on
duty. He ordered the portrait to be carried outside his tent,
that the Old Guard, stationed round it, might not be deprived of
the pleasure of seeing the King of Rome, the son and heir of
their adored monarch.

And while he was doing M. de Beausset the honor of breakfasting
with him, they heard, as Napoleon had anticipated, the rapturous
cries of the officers and men of the Old Guard who had run up to
see the portrait.

``Vive l'Empereur! Vive le roi de Rome! Vive l'Empereur!'' came
those ecstatic cries.

After breakfast Napoleon in de Beausset's presence dictated his
order of the day to the army.

``Short and energetic!'' he remarked when he had read over the
proclamation which he had dictated straight off without
corrections. It ran:

Soldiers! This is the battle you have so longed for. Victory
depends on you. It is essential for us; it will give us all we
need: comfortable quarters and a speedy return to our
country. Behave as you did at Austerlitz, Friedland, Vitebsk, and
Smolensk. Let our remotest posterity recall your achievements
this day with pride. Let it be said of each of you: ``He was in
the great battle before Moscow!''

``Before Moscow!'' repeated Napoleon, and inviting M. de
Beausset, who was so fond of travel, to accompany him on his
ride, he went out of the tent to where the horses stood saddled.

``Your Majesty is too kind!'' replied de Beausset to the
invitation to accompany the Emperor; he wanted to sleep, did not
know how to ride and was afraid of doing so.

But Napoleon nodded to the traveler, and de Beausset had to
mount. When Napoleon came out of the tent the shouting of the
Guards before his son's portrait grew still louder. Napoleon
frowned.

``Take him away!'' he said, pointing with a gracefully majestic
gesture to the portrait. ``It is too soon for him to see a field
of battle.''

De Beausset closed his eyes, bowed his head, and sighed deeply,
to indicate how profoundly he valued and comprehended the
Emperor's words.

% % % % % % % % % % % % % % % % % % % % % % % % % % % % % % % % %
% % % % % % % % % % % % % % % % % % % % % % % % % % % % % % % % %
% % % % % % % % % % % % % % % % % % % % % % % % % % % % % % % % %
% % % % % % % % % % % % % % % % % % % % % % % % % % % % % % % % %
% % % % % % % % % % % % % % % % % % % % % % % % % % % % % % % % %
% % % % % % % % % % % % % % % % % % % % % % % % % % % % % % % % %
% % % % % % % % % % % % % % % % % % % % % % % % % % % % % % % % %
% % % % % % % % % % % % % % % % % % % % % % % % % % % % % % % % %
% % % % % % % % % % % % % % % % % % % % % % % % % % % % % % % % %
% % % % % % % % % % % % % % % % % % % % % % % % % % % % % % % % %
% % % % % % % % % % % % % % % % % % % % % % % % % % % % % % % % %
% % % % % % % % % % % % % % % % % % % % % % % % % % % % % %

\chapter*{Chapter XXVII} \ifaudio \marginpar{
\href{http://ia801407.us.archive.org/32/items/war_and_peace_10_0904_librivox/war_and_peace_10_27_tolstoy_64kb.mp3}{Audio}}
\fi

\initial{O}{n} the twenty-fifth of August, so his historians tell us,
Napoleon spent the whole day on horseback inspecting the
locality, considering plans submitted to him by his marshals, and
personally giving commands to his generals.

The original line of the Russian forces along the river Kolocha
had been dislocated by the capture of the Shevardino Redoubt on
the twenty-fourth, and part of the line---the left flank---had
been drawn back. That part of the line was not entrenched and in
front of it the ground was more open and level than elsewhere. It
was evident to anyone, military or not, that it was here the
French should attack. It would seem that not much consideration
was needed to reach this conclusion, nor any particular care or
trouble on the part of the Emperor and his marshals, nor was
there any need of that special and supreme quality called genius
that people are so apt to ascribe to Napoleon; yet the historians
who described the event later and the men who then surrounded
Napoleon, and he himself, thought otherwise.

Napoleon rode over the plain and surveyed the locality with a
profound air and in silence, nodded with approval or shook his
head dubiously, and without communicating to the generals around
him the profound course of ideas which guided his decisions
merely gave them his final conclusions in the form of
commands. Having listened to a suggestion from Davout, who was
now called Prince d'Eckmuhl, to turn the Russian left wing,
Napoleon said it should not be done, without explaining why
not. To a proposal made by General Campan (who was to attack the
fleches) to lead his division through the woods, Napoleon agreed,
though the so-called Duke of Elchingen (Ney) ventured to remark
that a movement through the woods was dangerous and might
disorder the division.

Having inspected the country opposite the Shevardino Redoubt,
Napoleon pondered a little in silence and then indicated the
spots where two batteries should be set up by the morrow to act
against the Russian entrenchments, and the places where, in line
with them, the field artillery should be placed.

After giving these and other commands he returned to his tent,
and the dispositions for the battle were written down from his
dictation.

These dispositions, of which the French historians write with
enthusiasm and other historians with profound respect, were as
follows:

At dawn the two new batteries established during the night on the
plain occupied by the Prince d'Eckmuhl will open fire on the
opposing batteries of the enemy.

At the same time the commander of the artillery of the 1st Corps,
General Pernetti, with thirty cannon of Campan's division and all
the howitzers of Dessaix's and Friant's divisions, will move
forward, open fire, and overwhelm with shellfire the enemy's
battery, against which will operate: 24 guns of the artillery of
the Guards 30 guns of Campan's division and 8 guns of Friant's
and Dessaix's divisions---in all 62 guns.

The commander of the artillery of the 3rd Corps, General Fouche,
will place the howitzers of the 3rd and 8th Corps, sixteen in
all, on the flanks of the battery that is to bombard the
entrenchment on the left, which will have forty guns in all
directed against it.

General Sorbier must be ready at the first order to advance with
all the howitzers of the Guard's artillery against either one or
other of the entrenchments.

During the cannonade Prince Poniatowski is to advance through the
wood on the village and turn the enemy's position.

General Campan will move through the wood to seize the first
fortification.

After the advance has begun in this manner, orders will be given
in accordance with the enemy's movements.

The cannonade on the left flank will begin as soon as the guns of
the right wing are heard. The sharpshooters of Morand's division
and of the vice-King's division will open a heavy fire on seeing
the attack commence on the right wing.

The vice-King will occupy the village and cross by its three
bridges, advancing to the same heights as Morand's and Gibrard's
divisions, which under his leadership will be directed against
the redoubt and come into line with the rest of the forces.

All this must be done in good order (le tout se fera avec ordre
et methode) as far as possible retaining troops in reserve.

The Imperial Camp near Mozhaysk,

September, 6, 1812.

These dispositions, which are very obscure and confused if one
allows oneself to regard the arrangements without religious awe
of his genius, related to Napoleon's orders to deal with four
points---four different orders. Not one of these was, or could
be, carried out.

In the disposition it is said first that the batteries placed on
the spot chosen by Napoleon, with the guns of Pernetti and
Fouche; which were to come in line with them, 102 guns in all,
were to open fire and shower shells on the Russian fleches and
redoubts. This could not be done, as from the spots selected by
Napoleon the projectiles did not carry to the Russian works, and
those 102 guns shot into the air until the nearest commander,
contrary to Napoleon's instructions, moved them forward.

The second order was that Poniatowski, moving to the village
through the wood, should turn the Russian left flank. This could
not be done and was not done, because Poniatowski, advancing on
the village through the wood, met Tuchkov there barring his way,
and could not and did not turn the Russian position.

The third order was: General Campan will move through the wood to
seize the first fortification. General Campan's division did not
seize the first fortification but was driven back, for on
emerging from the wood it had to reform under grapeshot, of which
Napoleon was unaware.

The fourth order was: The vice-King will occupy the village
(Borodino) and cross by its three bridges, advancing to the same
heights as Morand's and Gibrard's divisions (for whose movements
no directions are given), which under his leadership will be
directed against the redoubt and come into line with the rest of
the forces.

As far as one can make out, not so much from this unintelligible
sentence as from the attempts the vice-King made to execute the
orders given him, he was to advance from the left through
Borodino to the redoubt while the divisions of Morand and Gerard
were to advance simultaneously from the front.

All this, like the other parts of the disposition, was not and
could not be executed. After passing through Borodino the
vice-King was driven back to the Kolocha and could get no
farther; while the divisions of Morand and Gerard did not take
the redoubt but were driven back, and the redoubt was only taken
at the end of the battle by the cavalry (a thing probably
unforeseen and not heard of by Napoleon). So not one of the
orders in the disposition was, or could be, executed. But in the
disposition it is said that, after the fight has commenced in
this manner, orders will be given in accordance with the enemy's
movements, and so it might be supposed that all necessary
arrangements would be made by Napoleon during the battle. But
this was not and could not be done, for during the whole battle
Napoleon was so far away that, as appeared later, he could not
know the course of the battle and not one of his orders during
the fight could be executed.

% % % % % % % % % % % % % % % % % % % % % % % % % % % % % % % % %
% % % % % % % % % % % % % % % % % % % % % % % % % % % % % % % % %
% % % % % % % % % % % % % % % % % % % % % % % % % % % % % % % % %
% % % % % % % % % % % % % % % % % % % % % % % % % % % % % % % % %
% % % % % % % % % % % % % % % % % % % % % % % % % % % % % % % % %
% % % % % % % % % % % % % % % % % % % % % % % % % % % % % % % % %
% % % % % % % % % % % % % % % % % % % % % % % % % % % % % % % % %
% % % % % % % % % % % % % % % % % % % % % % % % % % % % % % % % %
% % % % % % % % % % % % % % % % % % % % % % % % % % % % % % % % %
% % % % % % % % % % % % % % % % % % % % % % % % % % % % % % % % %
% % % % % % % % % % % % % % % % % % % % % % % % % % % % % % % % %
% % % % % % % % % % % % % % % % % % % % % % % % % % % % % %

\chapter*{Chapter XXVIII} \ifaudio \marginpar{
\href{http://ia801407.us.archive.org/32/items/war_and_peace_10_0904_librivox/war_and_peace_10_28_tolstoy_64kb.mp3}{Audio}}
\fi

\initial{M}{any} historians say that the French did not win the battle of
Borodino because Napoleon had a cold, and that if he had not had
a cold the orders he gave before and during the battle would have
been still more full of genius and Russia would have been lost
and the face of the world have been changed. To historians who
believe that Russia was shaped by the will of one man---Peter the
Great---and that France from a republic became an empire and
French armies went to Russia at the will of one
man---Napoleon---to say that Russia remained a power because
Napoleon had a bad cold on the twenty-fourth of August may seem
logical and convincing.

If it had depended on Napoleon's will to fight or not to fight
the battle of Borodino, and if this or that other arrangement
depended on his will, then evidently a cold affecting the
manifestation of his will might have saved Russia, and
consequently the valet who omitted to bring Napoleon his
waterproof boots on the twenty-fourth would have been the savior
of Russia. Along that line of thought such a deduction is
indubitable, as indubitable as the deduction Voltaire made in
jest (without knowing what he was jesting at) when he saw that
the Massacre of St. Bartholomew was due to Charles IX's stomach
being deranged. But to men who do not admit that Russia was
formed by the will of one man, Peter I, or that the French Empire
was formed and the war with Russia begun by the will of one man,
Napoleon, that argument seems not merely untrue and irrational,
but contrary to all human reality. To the question of what causes
historic events another answer presents itself, namely, that the
course of human events is predetermined from on high---depends on
the coincidence of the wills of all who take part in the events,
and that a Napoleon's influence on the course of these events is
purely external and fictitious.

Strange as at first glance it may seem to suppose that the
Massacre of St. Bartholomew was not due to Charles IX's will,
though he gave the order for it and thought it was done as a
result of that order; and strange as it may seem to suppose that
the slaughter of eighty thousand men at Borodino was not due to
Napoleon's will, though he ordered the commencement and conduct
of the battle and thought it was done because he ordered it;
strange as these suppositions appear, yet human dignity---which
tells me that each of us is, if not more at least not less a man
than the great Napoleon---demands the acceptance of that solution
of the question, and historic investigation abundantly confirms
it.

At the battle of Borodino Napoleon shot at no one and killed no
one.  That was all done by the soldiers. Therefore it was not he
who killed people.

The French soldiers went to kill and be killed at the battle of
Borodino not because of Napoleon's orders but by their own
volition. The whole army---French, Italian, German, Polish, and
Dutch---hungry, ragged, and weary of the campaign, felt at the
sight of an army blocking their road to Moscow that the wine was
drawn and must be drunk. Had Napoleon then forbidden them to
fight the Russians, they would have killed him and have proceeded
to fight the Russians because it was inevitable.

When they heard Napoleon's proclamation offering them, as
compensation for mutilation and death, the words of posterity
about their having been in the battle before Moscow, they cried
``Vive l'Empereur!'' just as they had cried ``Vive l'Empereur!''
at the sight of the portrait of the boy piercing the terrestrial
globe with a toy stick, and just as they would have cried ``Vive
l'Empereur!'' at any nonsense that might be told them.  There was
nothing left for them to do but cry ``Vive l'Empereur!'' and go
to fight, in order to get food and rest as conquerors in
Moscow. So it was not because of Napoleon's commands that they
killed their fellow men.

And it was not Napoleon who directed the course of the battle,
for none of his orders were executed and during the battle he did
not know what was going on before him. So the way in which these
people killed one another was not decided by Napoleon's will but
occurred independently of him, in accord with the will of
hundreds of thousands of people who took part in the common
action. It only seemed to Napoleon that it all took place by his
will. And so the question whether he had or had not a cold has no
more historic interest than the cold of the least of the
transport soldiers.

Moreover, the assertion made by various writers that his cold was
the cause of his dispositions not being as well-planned as on
former occasions, and of his orders during the battle not being
as good as previously, is quite baseless, which again shows that
Napoleon's cold on the twenty-sixth of August was unimportant.

The dispositions cited above are not at all worse, but are even
better, than previous dispositions by which he had won
victories. His pseudo-orders during the battle were also no worse
than formerly, but much the same as usual. These dispositions and
orders only seem worse than previous ones because the battle of
Borodino was the first Napoleon did not win. The profoundest and
most excellent dispositions and orders seem very bad, and every
learned militarist criticizes them with looks of importance, when
they relate to a battle that has been lost, and the very worst
dispositions and orders seem very good, and serious people fill
whole volumes to demonstrate their merits, when they relate to a
battle that has been won.

The dispositions drawn up by Weyrother for the battle of
Austerlitz were a model of perfection for that kind of
composition, but still they were criticized---criticized for
their very perfection, for their excessive minuteness.

Napoleon at the battle of Borodino fulfilled his office as
representative of authority as well as, and even better than, at
other battles. He did nothing harmful to the progress of the
battle; he inclined to the most reasonable opinions, he made no
confusion, did not contradict himself, did not get frightened or
run away from the field of battle, but with his great tact and
military experience carried out his role of appearing to command,
calmly and with dignity.

% % % % % % % % % % % % % % % % % % % % % % % % % % % % % % % % %
% % % % % % % % % % % % % % % % % % % % % % % % % % % % % % % % %
% % % % % % % % % % % % % % % % % % % % % % % % % % % % % % % % %
% % % % % % % % % % % % % % % % % % % % % % % % % % % % % % % % %
% % % % % % % % % % % % % % % % % % % % % % % % % % % % % % % % %
% % % % % % % % % % % % % % % % % % % % % % % % % % % % % % % % %
% % % % % % % % % % % % % % % % % % % % % % % % % % % % % % % % %
% % % % % % % % % % % % % % % % % % % % % % % % % % % % % % % % %
% % % % % % % % % % % % % % % % % % % % % % % % % % % % % % % % %
% % % % % % % % % % % % % % % % % % % % % % % % % % % % % % % % %
% % % % % % % % % % % % % % % % % % % % % % % % % % % % % % % % %
% % % % % % % % % % % % % % % % % % % % % % % % % % % % % %

\chapter*{Chapter XXIX} \ifaudio \marginpar{
\href{http://ia801407.us.archive.org/32/items/war_and_peace_10_0904_librivox/war_and_peace_10_29_tolstoy_64kb.mp3}{Audio}}
\fi

\initial{O}{n} returning from a second inspection of the lines, Napoleon
remarked:

``The chessmen are set up, the game will begin tomorrow!''

Having ordered punch and summoned de Beausset, he began to talk
to him about Paris and about some changes he meant to make in the
Empress' household, surprising the prefect by his memory of
minute details relating to the court.

He showed an interest in trifles, joked about de Beausset's love
of travel, and chatted carelessly, as a famous, self-confident
surgeon who knows his job does when turning up his sleeves and
putting on his apron while a patient is being strapped to the
operating table. ``The matter is in my hands and is clear and
definite in my head. When the time comes to set to work I shall
do it as no one else could, but now I can jest, and the more I
jest and the calmer I am the more tranquil and confident you
ought to be, and the more amazed at my genius.''

Having finished his second glass of punch, Napoleon went to rest
before the serious business which, he considered, awaited him
next day. He was so much interested in that task that he was
unable to sleep, and in spite of his cold which had grown worse
from the dampness of the evening, he went into the large division
of the tent at three o'clock in the morning, loudly blowing his
nose. He asked whether the Russians had not withdrawn, and was
told that the enemy's fires were still in the same places. He
nodded approval.

The adjutant in attendance came into the tent.

``Well, Rapp, do you think we shall do good business today?''
Napoleon asked him.

``Without doubt, sire,'' replied Rapp.

Napoleon looked at him.

``Do you remember, sire, what you did me the honor to say at
Smolensk?''  continued Rapp. ``The wine is drawn and must be
drunk.''

Napoleon frowned and sat silent for a long time leaning his head
on his hand.

``This poor army!'' he suddenly remarked. ``It has diminished
greatly since Smolensk. Fortune is frankly a courtesan, Rapp. I
have always said so and I am beginning to experience it. But the
Guards, Rapp, the Guards are intact?'' he remarked
interrogatively.

``Yes, sire,'' replied Rapp.

Napoleon took a lozenge, put it in his mouth, and glanced at his
watch.  He was not sleepy and it was still not nearly morning. It
was impossible to give further orders for the sake of killing
time, for the orders had all been given and were now being
executed.

``Have the biscuits and rice been served out to the regiments of
the Guards?'' asked Napoleon sternly.

``Yes, sire.''

``The rice too?''

Rapp replied that he had given the Emperor's order about the
rice, but Napoleon shook his head in dissatisfaction as if not
believing that his order had been executed. An attendant came in
with punch. Napoleon ordered another glass to be brought for
Rapp, and silently sipped his own.

``I have neither taste nor smell,'' he remarked, sniffing at his
glass.  ``This cold is tiresome. They talk about medicine---what
is the good of medicine when it can't cure a cold! Corvisart gave
me these lozenges but they don't help at all. What can doctors
cure? One can't cure anything.  Our body is a machine for
living. It is organized for that, it is its nature. Let life go
on in it unhindered and let it defend itself, it will do more
than if you paralyze it by encumbering it with remedies.  Our
body is like a perfect watch that should go for a certain time;
the watchmaker cannot open it, he can only adjust it by fumbling,
and that blindfold... Yes, our body is just a machine for living,
that is all.''

And having entered on the path of definition, of which he was
fond, Napoleon suddenly and unexpectedly gave a new one.

``Do you know, Rapp, what military art is?'' asked he. ``It is
the art of being stronger than the enemy at a given
moment. That's all.''

Rapp made no reply.

``Tomorrow we shall have to deal with Kutuzov!'' said
Napoleon. ``We shall see! Do you remember at Braunau he commanded
an army for three weeks and did not once mount a horse to inspect
his entrenchments... We shall see!''

He looked at his watch. It was still only four o'clock. He did
not feel sleepy. The punch was finished and there was still
nothing to do. He rose, walked to and fro, put on a warm overcoat
and a hat, and went out of the tent. The night was dark and damp,
a scarcely perceptible moisture was descending from above. Near
by, the campfires were dimly burning among the French Guards, and
in the distance those of the Russian line shone through the
smoke. The weather was calm, and the rustle and tramp of the
French troops already beginning to move to take up their
positions were clearly audible.

Napoleon walked about in front of his tent, looked at the fires
and listened to these sounds, and as he was passing a tall
guardsman in a shaggy cap, who was standing sentinel before his
tent and had drawn himself up like a black pillar at sight of the
Emperor, Napoleon stopped in front of him.

``What year did you enter the service?'' he asked with that
affectation of military bluntness and geniality with which he
always addressed the soldiers.

The man answered the question.

``Ah! One of the old ones! Has your regiment had its rice?''

``It has, Your Majesty.''

Napoleon nodded and walked away.

At half-past five Napoleon rode to the village of Shevardino.

It was growing light, the sky was clearing, only a single cloud
lay in the east. The abandoned campfires were burning themselves
out in the faint morning light.

On the right a single deep report of a cannon resounded and died
away in the prevailing silence. Some minutes passed. A second and
a third report shook the air, then a fourth and a fifth boomed
solemnly near by on the right.

The first shots had not yet ceased to reverberate before others
rang out and yet more were heard mingling with and overtaking one
another.

Napoleon with his suite rode up to the Shevardino Redoubt where
he dismounted. The game had begun.

% % % % % % % % % % % % % % % % % % % % % % % % % % % % % % % % %
% % % % % % % % % % % % % % % % % % % % % % % % % % % % % % % % %
% % % % % % % % % % % % % % % % % % % % % % % % % % % % % % % % %
% % % % % % % % % % % % % % % % % % % % % % % % % % % % % % % % %
% % % % % % % % % % % % % % % % % % % % % % % % % % % % % % % % %
% % % % % % % % % % % % % % % % % % % % % % % % % % % % % % % % %
% % % % % % % % % % % % % % % % % % % % % % % % % % % % % % % % %
% % % % % % % % % % % % % % % % % % % % % % % % % % % % % % % % %
% % % % % % % % % % % % % % % % % % % % % % % % % % % % % % % % %
% % % % % % % % % % % % % % % % % % % % % % % % % % % % % % % % %
% % % % % % % % % % % % % % % % % % % % % % % % % % % % % % % % %
% % % % % % % % % % % % % % % % % % % % % % % % % % % % % %

\chapter*{Chapter XXX} \ifaudio \marginpar{
\href{http://ia801407.us.archive.org/32/items/war_and_peace_10_0904_librivox/war_and_peace_10_30_tolstoy_64kb.mp3}{Audio}}
\fi

\initial{O}{n} returning to Gorki after having seen Prince Andrew, Pierre
ordered his groom to get the horses ready and to call him early
in the morning, and then immediately fell asleep behind a
partition in a corner Boris had given up to him.

Before he was thoroughly awake next morning everybody had already
left the hut. The panes were rattling in the little windows and
his groom was shaking him.

``Your excellency! Your excellency! Your excellency!'' he kept
repeating pertinaciously while he shook Pierre by the shoulder
without looking at him, having apparently lost hope of getting
him to wake up.

``What? Has it begun? Is it time?'' Pierre asked, waking up.

``Hear the firing,'' said the groom, a discharged soldier. ``All
the gentlemen have gone out, and his Serene Highness himself rode
past long ago.''

Pierre dressed hastily and ran out to the porch. Outside all was
bright, fresh, dewy, and cheerful. The sun, just bursting forth
from behind a cloud that had concealed it, was shining, with rays
still half broken by the clouds, over the roofs of the street
opposite, on the dew-besprinkled dust of the road, on the walls
of the houses, on the windows, the fence, and on Pierre's horses
standing before the hut. The roar of guns sounded more distinct
outside. An adjutant accompanied by a Cossack passed by at a
sharp trot.

``It's time, Count; it's time!'' cried the adjutant.

Telling the groom to follow him with the horses, Pierre went down
the street to the knoll from which he had looked at the field of
battle the day before. A crowd of military men was assembled
there, members of the staff could be heard conversing in French,
and Kutuzov's gray head in a white cap with a red band was
visible, his gray nape sunk between his shoulders. He was looking
through a field glass down the highroad before him.

Mounting the steps to the knoll Pierre looked at the scene before
him, spellbound by beauty. It was the same panorama he had
admired from that spot the day before, but now the whole place
was full of troops and covered by smoke clouds from the guns, and
the slanting rays of the bright sun, rising slightly to the left
behind Pierre, cast upon it through the clear morning air
penetrating streaks of rosy, golden-tinted light and long dark
shadows. The forest at the farthest extremity of the panorama
seemed carved in some precious stone of a yellowish-green color;
its undulating outline was silhouetted against the horizon and
was pierced beyond Valuevo by the Smolensk highroad crowded with
troops.  Nearer at hand glittered golden cornfields interspersed
with copses.  There were troops to be seen everywhere, in front
and to the right and left. All this was vivid, majestic, and
unexpected; but what impressed Pierre most of all was the view of
the battlefield itself, of Borodino and the hollows on both sides
of the Kolocha.

Above the Kolocha, in Borodino and on both sides of it,
especially to the left where the Voyna flowing between its marshy
banks falls into the Kolocha, a mist had spread which seemed to
melt, to dissolve, and to become translucent when the brilliant
sun appeared and magically colored and outlined everything. The
smoke of the guns mingled with this mist, and over the whole
expanse and through that mist the rays of the morning sun were
reflected, flashing back like lightning from the water, from the
dew, and from the bayonets of the troops crowded together by the
riverbanks and in Borodino. A white church could be seen through
the mist, and here and there the roofs of huts in Borodino as
well as dense masses of soldiers, or green ammunition chests and
ordnance. And all this moved, or seemed to move, as the smoke and
mist spread out over the whole space. Just as in the
mist-enveloped hollow near Borodino, so along the entire line
outside and above it and especially in the woods and fields to
the left, in the valleys and on the summits of the high ground,
clouds of powder smoke seemed continually to spring up out of
nothing, now singly, now several at a time, some translucent,
others dense, which, swelling, growing, rolling, and blending,
extended over the whole expanse.

These puffs of smoke and (strange to say) the sound of the firing
produced the chief beauty of the spectacle.

``Puff!''---suddenly a round compact cloud of smoke was seen
merging from violet into gray and milky white, and ``boom!'' came
the report a second later.

``Puff! puff!''---and two clouds arose pushing one another and
blending together; and ``boom, boom!'' came the sounds confirming
what the eye had seen.

Pierre glanced round at the first cloud, which he had seen as a
round compact ball, and in its place already were balloons of
smoke floating to one side, and---``puff'' (with a
pause)---``puff, puff!'' three and then four more appeared and
then from each, with the same interval---``boom---boom, boom!''
came the fine, firm, precise sounds in reply. It seemed as if
those smoke clouds sometimes ran and sometimes stood still while
woods, fields, and glittering bayonets ran past them. From the
left, over fields and bushes, those large balls of smoke were
continually appearing followed by their solemn reports, while
nearer still, in the hollows and woods, there burst from the
muskets small cloudlets that had no time to become balls, but had
their little echoes in just the same way. ``Trakh-ta-ta-takh!''
came the frequent crackle of musketry, but it was irregular and
feeble in comparison with the reports of the cannon.

Pierre wished to be there with that smoke, those shining
bayonets, that movement, and those sounds. He turned to look at
Kutuzov and his suite, to compare his impressions with those of
others. They were all looking at the field of battle as he was,
and, as it seemed to him, with the same feelings. All their faces
were now shining with that latent warmth of feeling Pierre had
noticed the day before and had fully understood after his talk
with Prince Andrew.

``Go, my dear fellow, go... and Christ be with you!'' Kutuzov was
saying to a general who stood beside him, not taking his eye from
the battlefield.

Having received this order the general passed by Pierre on his
way down the knoll.

``To the crossing!'' said the general coldly and sternly in reply
to one of the staff who asked where he was going.

``I'll go there too, I too!'' thought Pierre, and followed the
general.

The general mounted a horse a Cossack had brought him. Pierre
went to his groom who was holding his horses and, asking which
was the quietest, clambered onto it, seized it by the mane, and
turning out his toes pressed his heels against its sides and,
feeling that his spectacles were slipping off but unable to let
go of the mane and reins, he galloped after the general, causing
the staff officers to smile as they watched him from the knoll.

% % % % % % % % % % % % % % % % % % % % % % % % % % % % % % % % %
% % % % % % % % % % % % % % % % % % % % % % % % % % % % % % % % %
% % % % % % % % % % % % % % % % % % % % % % % % % % % % % % % % %
% % % % % % % % % % % % % % % % % % % % % % % % % % % % % % % % %
% % % % % % % % % % % % % % % % % % % % % % % % % % % % % % % % %
% % % % % % % % % % % % % % % % % % % % % % % % % % % % % % % % %
% % % % % % % % % % % % % % % % % % % % % % % % % % % % % % % % %
% % % % % % % % % % % % % % % % % % % % % % % % % % % % % % % % %
% % % % % % % % % % % % % % % % % % % % % % % % % % % % % % % % %
% % % % % % % % % % % % % % % % % % % % % % % % % % % % % % % % %
% % % % % % % % % % % % % % % % % % % % % % % % % % % % % % % % %
% % % % % % % % % % % % % % % % % % % % % % % % % % % % % %

\chapter*{Chapter XXXI} \ifaudio \marginpar{
\href{http://ia801407.us.archive.org/32/items/war_and_peace_10_0904_librivox/war_and_peace_10_31_tolstoy_64kb.mp3}{Audio}}
\fi

\initial{H}{aving} descended the hill the general after whom Pierre was
galloping turned sharply to the left, and Pierre, losing sight of
him, galloped in among some ranks of infantry marching ahead of
him. He tried to pass either in front of them or to the right or
left, but there were soldiers everywhere, all with the same
preoccupied expression and busy with some unseen but evidently
important task. They all gazed with the same dissatisfied and
inquiring expression at this stout man in a white hat, who for
some unknown reason threatened to trample them under his horse's
hoofs.

``Why ride into the middle of the battalion?'' one of them
shouted at him.

Another prodded his horse with the butt end of a musket, and
Pierre, bending over his saddlebow and hardly able to control his
shying horse, galloped ahead of the soldiers where there was a
free space.

There was a bridge ahead of him, where other soldiers stood
firing.  Pierre rode up to them. Without being aware of it he had
come to the bridge across the Kolocha between Gorki and Borodino,
which the French (having occupied Borodino) were attacking in the
first phase of the battle. Pierre saw that there was a bridge in
front of him and that soldiers were doing something on both sides
of it and in the meadow, among the rows of new-mown hay which he
had taken no notice of amid the smoke of the campfires the day
before; but despite the incessant firing going on there he had no
idea that this was the field of battle. He did not notice the
sound of the bullets whistling from every side, or the
projectiles that flew over him, did not see the enemy on the
other side of the river, and for a long time did not notice the
killed and wounded, though many fell near him. He looked about
him with a smile which did not leave his face.

``Why's that fellow in front of the line?'' shouted somebody at
him again.

``To the left!... Keep to the right!'' the men shouted to him.

Pierre went to the right, and unexpectedly encountered one of
Raev\-ski's adjutants whom he knew. The adjutant looked angrily at
him, evidently also intending to shout at him, but on recognizing
him he nodded.

``How have you got here?'' he said, and galloped on.

Pierre, feeling out of place there, having nothing to do, and
afraid of getting in someone's way again, galloped after the
adjutant.

``What's happening here? May I come with you?'' he asked.

``One moment, one moment!'' replied the adjutant, and riding up
to a stout colonel who was standing in the meadow, he gave him
some message and then addressed Pierre.

``Why have you come here, Count?'' he asked with a smile. ``Still
inquisitive?''

``Yes, yes,'' assented Pierre.

But the adjutant turned his horse about and rode on.

``Here it's tolerable,'' said he, ``but with Bagration on the
left flank they're getting it frightfully hot.''

``Really?'' said Pierre. ``Where is that?''

``Come along with me to our knoll. We can get a view from there
and in our battery it is still bearable,'' said the
adjutant. ``Will you come?''

``Yes, I'll come with you,'' replied Pierre, looking round for
his groom.

It was only now that he noticed wounded men staggering along or
being carried on stretchers. On that very meadow he had ridden
over the day before, a soldier was lying athwart the rows of
scented hay, with his head thrown awkwardly back and his shako
off.

``Why haven't they carried him away?'' Pierre was about to ask,
but seeing the stern expression of the adjutant who was also
looking that way, he checked himself.

Pierre did not find his groom and rode along the hollow with the
adjutant to Raevski's Redoubt. His horse lagged behind the
adjutant's and jolted him at every step.

``You don't seem to be used to riding, Count?'' remarked the
adjutant.

``No it's not that, but her action seems so jerky,'' said Pierre
in a puzzled tone.

``Why... she's wounded!'' said the adjutant. ``In the off foreleg
above the knee. A bullet, no doubt. I congratulate you, Count, on
your baptism of fire!''

Having ridden in the smoke past the Sixth Corps, behind the
artillery which had been moved forward and was in action,
deafening them with the noise of firing, they came to a small
wood. There it was cool and quiet, with a scent of autumn. Pierre
and the adjutant dismounted and walked up the hill on foot.

``Is the general here?'' asked the adjutant on reaching the
knoll.

``He was here a minute ago but has just gone that way,'' someone
told him, pointing to the right.

The adjutant looked at Pierre as if puzzled what to do with him
now.

``Don't trouble about me,'' said Pierre. ``I'll go up onto the
knoll if I may?''

``Yes, do. You'll see everything from there and it's less
dangerous, and I'll come for you.''

Pierre went to the battery and the adjutant rode on. They did not
meet again, and only much later did Pierre learn that he lost an
arm that day.

The knoll to which Pierre ascended was that famous one afterwards
known to the Russians as the Knoll Battery or Raevski's Redoubt,
and to the French as la grande redoute, la fatale redoute, la
redoute du centre, around which tens of thousands fell, and which
the French regarded as the key to the whole position.

This redoubt consisted of a knoll, on three sides of which
trenches had been dug. Within the entrenchment stood ten guns
that were being fired through openings in the earthwork.

In line with the knoll on both sides stood other guns which also
fired incessantly. A little behind the guns stood infantry. When
ascending that knoll Pierre had no notion that this spot, on
which small trenches had been dug and from which a few guns were
firing, was the most important point of the battle.

On the contrary, just because he happened to be there he thought
it one of the least significant parts of the field.

Having reached the knoll, Pierre sat down at one end of a trench
surrounding the battery and gazed at what was going on around him
with an unconsciously happy smile. Occasionally he rose and
walked about the battery still with that same smile, trying not
to obstruct the soldiers who were loading, hauling the guns, and
continually running past him with bags and charges. The guns of
that battery were being fired continually one after another with
a deafening roar, enveloping the whole neighborhood in powder
smoke.

In contrast with the dread felt by the infantrymen placed in
support, here in the battery where a small number of men busy at
their work were separated from the rest by a trench, everyone
experienced a common and as it were family feeling of animation.

The intrusion of Pierre's nonmilitary figure in a white hat made
an unpleasant impression at first. The soldiers looked askance at
him with surprise and even alarm as they went past him. The
senior artillery officer, a tall, long-legged, pockmarked man,
moved over to Pierre as if to see the action of the farthest gun
and looked at him with curiosity.

A young round-faced officer, quite a boy still and evidently only
just out of the Cadet College, who was zealously commanding the
two guns entrusted to him, addressed Pierre sternly.

``Sir,'' he said, ``permit me to ask you to stand aside. You must
not be here.''

The soldiers shook their heads disapprovingly as they looked at
Pierre.  But when they had convinced themselves that this man in
the white hat was doing no harm, but either sat quietly on the
slope of the trench with a shy smile or, politely making way for
the soldiers, paced up and down the battery under fire as calmly
as if he were on a boulevard, their feeling of hostile distrust
gradually began to change into a kindly and bantering sympathy,
such as soldiers feel for their dogs, cocks, goats, and in
general for the animals that live with the regiment. The men soon
accepted Pierre into their family, adopted him, gave him a
nickname (``our gentleman''), and made kindly fun of him among
themselves.

A shell tore up the earth two paces from Pierre and he looked
around with a smile as he brushed from his clothes some earth it
had thrown up.

``And how's it you're not afraid, sir, really now?'' a red-faced,
broad-shouldered soldier asked Pierre, with a grin that disclosed
a set of sound, white teeth.

``Are you afraid, then?'' said Pierre.

``What else do you expect?'' answered the soldier. ``She has no
mercy, you know! When she comes spluttering down, out go your
innards. One can't help being afraid,'' he said laughing.

Several of the men, with bright kindly faces, stopped beside
Pierre.  They seemed not to have expected him to talk like
anybody else, and the discovery that he did so delighted them.

``It's the business of us soldiers. But in a gentleman it's
wonderful!  There's a gentleman for you!''

``To your places!'' cried the young officer to the men gathered
round Pierre.

The young officer was evidently exercising his duties for the
first or second time and therefore treated both his superiors and
the men with great precision and formality.

The booming cannonade and the fusillade of musketry were growing
more intense over the whole field, especially to the left where
Bagration's fleches were, but where Pierre was the smoke of the
firing made it almost impossible to distinguish
anything. Moreover, his whole attention was engrossed by watching
the family circle---separated from all else---formed by the men
in the battery. His first unconscious feeling of joyful animation
produced by the sights and sounds of the battlefield was now
replaced by another, especially since he had seen that soldier
lying alone in the hayfield. Now, seated on the slope of the
trench, he observed the faces of those around him.

By ten o'clock some twenty men had already been carried away from
the battery; two guns were smashed and cannon balls fell more and
more frequently on the battery and spent bullets buzzed and
whistled around.  But the men in the battery seemed not to notice
this, and merry voices and jokes were heard on all sides.

``A live one!'' shouted a man as a whistling shell approached.

``Not this way! To the infantry!'' added another with loud
laughter, seeing the shell fly past and fall into the ranks of
the supports.

``Are you bowing to a friend, eh?'' remarked another, chaffing a
peasant who ducked low as a cannon ball flew over.

Several soldiers gathered by the wall of the trench, looking out
to see what was happening in front.

``They've withdrawn the front line, it has retired,'' said they,
pointing over the earthwork.

``Mind your own business,'' an old sergeant shouted at them. ``If
they've retired it's because there's work for them to do farther
back.''

And the sergeant, taking one of the men by the shoulders, gave
him a shove with his knee. This was followed by a burst of
laughter.

``To the fifth gun, wheel it up!'' came shouts from one side.

``Now then, all together, like bargees!'' rose the merry voices
of those who were moving the gun.

``Oh, she nearly knocked our gentleman's hat off!'' cried the
red-faced humorist, showing his teeth chaffing Pierre. ``Awkward
baggage!'' he added reproachfully to a cannon ball that struck a
cannon wheel and a man's leg.

``Now then, you foxes!'' said another, laughing at some
militiamen who, stooping low, entered the battery to carry away
the wounded man.

``So this gruel isn't to your taste? Oh, you crows! You're
scared!'' they shouted at the militiamen who stood hesitating
before the man whose leg had been torn off.

``There, lads... oh, oh!'' they mimicked the peasants, ``they
don't like it at all!''

Pierre noticed that after every ball that hit the redoubt, and
after every loss, the liveliness increased more and more.

As the flames of the fire hidden within come more and more
vividly and rapidly from an approaching thundercloud, so, as if
in opposition to what was taking place, the lightning of hidden
fire growing more and more intense glowed in the faces of these
men.

Pierre did not look out at the battlefield and was not concerned
to know what was happening there; he was entirely absorbed in
watching this fire which burned ever more brightly and which he
felt was flaming up in the same way in his own soul.

At ten o'clock the infantry that had been among the bushes in
front of the battery and along the Kamenka streamlet
retreated. From the battery they could be seen running back past
it carrying their wounded on their muskets. A general with his
suite came to the battery, and after speaking to the colonel gave
Pierre an angry look and went away again having ordered the
infantry supports behind the battery to lie down, so as to be
less exposed to fire. After this from amid the ranks of infantry
to the right of the battery came the sound of a drum and shouts
of command, and from the battery one saw how those ranks of
infantry moved forward.

Pierre looked over the wall of the trench and was particularly
struck by a pale young officer who, letting his sword hang down,
was walking backwards and kept glancing uneasily around.

The ranks of the infantry disappeared amid the smoke but their
long-drawn shout and rapid musketry firing could still be
heard. A few minutes later crowds of wounded men and
stretcher-bearers came back from that direction. Projectiles
began to fall still more frequently in the battery. Several men
were lying about who had not been removed. Around the cannon the
men moved still more briskly and busily. No one any longer took
notice of Pierre. Once or twice he was shouted at for being in
the way. The senior officer moved with big, rapid strides from
one gun to another with a frowning face. The young officer, with
his face still more flushed, commanded the men more scrupulously
than ever. The soldiers handed up the charges, turned, loaded,
and did their business with strained smartness. They gave little
jumps as they walked, as though they were on springs.

The stormcloud had come upon them, and in every face the fire
which Pierre had watched kindle burned up brightly. Pierre
standing beside the commanding officer. The young officer, his
hand to his shako, ran up to his superior.

``I have the honor to report, sir, that only eight rounds are
left. Are we to continue firing?'' he asked.

``Grapeshot!'' the senior shouted, without answering the
question, looking over the wall of the trench.

Suddenly something happened: the young officer gave a gasp and
bending double sat down on the ground like a bird shot on the
wing. Everything became strange, confused, and misty in Pierre's
eyes.

One cannon ball after another whistled by and struck the
earthwork, a soldier, or a gun. Pierre, who had not noticed these
sounds before, now heard nothing else. On the right of the
battery soldiers shouting ``Hurrah!'' were running not forwards
but backwards, it seemed to Pierre.

A cannon ball struck the very end of the earth work by which he
was standing, crumbling down the earth; a black ball flashed
before his eyes and at the same instant plumped into
something. Some militiamen who were entering the battery ran
back.

``All with grapeshot!'' shouted the officer.

The sergeant ran up to the officer and in a frightened whisper
informed him (as a butler at dinner informs his master that there
is no more of some wine asked for) that there were no more
charges.

``The scoundrels! What are they doing?'' shouted the officer,
turning to Pierre.

The officer's face was red and perspiring and his eyes glittered
under his frowning brow.

``Run to the reserves and bring up the ammunition boxes!'' he
yelled, angrily avoiding Pierre with his eyes and speaking to his
men.

``I'll go,'' said Pierre.

The officer, without answering him, strode across to the opposite
side.

``Don't fire... Wait!'' he shouted.

The man who had been ordered to go for ammunition stumbled
against Pierre.

``Eh, sir, this is no place for you,'' said he, and ran down the
slope.

Pierre ran after him, avoiding the spot where the young officer
was sitting.

One cannon ball, another, and a third flew over him, falling in
front, beside, and behind him. Pierre ran down the slope. ``Where
am I going?''  he suddenly asked himself when he was already near
the green ammunition wagons. He halted irresolutely, not knowing
whether to return or go on.  Suddenly a terrible concussion threw
him backwards to the ground. At the same instant he was dazzled
by a great flash of flame, and immediately a deafening roar,
crackling, and whistling made his ears tingle.

When he came to himself he was sitting on the ground leaning on
his hands; the ammunition wagons he had been approaching no
longer existed, only charred green boards and rags littered the
scorched grass, and a horse, dangling fragments of its shaft
behind it, galloped past, while another horse lay, like Pierre,
on the ground, uttering prolonged and piercing cries.

% % % % % % % % % % % % % % % % % % % % % % % % % % % % % % % % %
% % % % % % % % % % % % % % % % % % % % % % % % % % % % % % % % %
% % % % % % % % % % % % % % % % % % % % % % % % % % % % % % % % %
% % % % % % % % % % % % % % % % % % % % % % % % % % % % % % % % %
% % % % % % % % % % % % % % % % % % % % % % % % % % % % % % % % %
% % % % % % % % % % % % % % % % % % % % % % % % % % % % % % % % %
% % % % % % % % % % % % % % % % % % % % % % % % % % % % % % % % %
% % % % % % % % % % % % % % % % % % % % % % % % % % % % % % % % %
% % % % % % % % % % % % % % % % % % % % % % % % % % % % % % % % %
% % % % % % % % % % % % % % % % % % % % % % % % % % % % % % % % %
% % % % % % % % % % % % % % % % % % % % % % % % % % % % % % % % %
% % % % % % % % % % % % % % % % % % % % % % % % % % % % % %

\chapter*{Chapter XXXII} \ifaudio \marginpar{
\href{http://ia801407.us.archive.org/32/items/war_and_peace_10_0904_librivox/war_and_peace_10_32_tolstoy_64kb.mp3}{Audio}}
\fi

\initial{B}{eside} himself with terror Pierre jumped up and ran back to the
battery, as to the only refuge from the horrors that surrounded
him.

On entering the earthwork he noticed that there were men doing
something there but that no shots were being fired from the
battery. He had no time to realize who these men were. He saw the
senior officer lying on the earth wall with his back turned as if
he were examining something down below and that one of the
soldiers he had noticed before was struggling forward shouting
``Brothers!'' and trying to free himself from some men who were
holding him by the arm. He also saw something else that was
strange.

But he had not time to realize that the colonel had been killed,
that the soldier shouting ``Brothers!'' was a prisoner, and that
another man had been bayoneted in the back before his eyes, for
hardly had he run into the redoubt before a thin, sallow-faced,
perspiring man in a blue uniform rushed on him sword in hand,
shouting something. Instinctively guarding against the
shock---for they had been running together at full speed before
they saw one another---Pierre put out his hands and seized the
man (a French officer) by the shoulder with one hand and by the
throat with the other. The officer, dropping his sword, seized
Pierre by his collar.

For some seconds they gazed with frightened eyes at one another's
unfamiliar faces and both were perplexed at what they had done
and what they were to do next. ``Am I taken prisoner or have I
taken him prisoner?'' each was thinking. But the French officer
was evidently more inclined to think he had been taken prisoner
because Pierre's strong hand, impelled by instinctive fear,
squeezed his throat ever tighter and tighter. The Frenchman was
about to say something, when just above their heads, terrible and
low, a cannon ball whistled, and it seemed to Pierre that the
French officer's head had been torn off, so swiftly had he ducked
it.

Pierre too bent his head and let his hands fall. Without further
thought as to who had taken whom prisoner, the Frenchman ran back
to the battery and Pierre ran down the slope stumbling over the
dead and wounded who, it seemed to him, caught at his feet. But
before he reached the foot of the knoll he was met by a dense
crowd of Russian soldiers who, stumbling, tripping up, and
shouting, ran merrily and wildly toward the battery. (This was
the attack for which Ermolov claimed the credit, declaring that
only his courage and good luck made such a feat possible: it was
the attack in which he was said to have thrown some St. George's
Crosses he had in his pocket into the battery for the first
soldiers to take who got there.)

The French who had occupied the battery fled, and our troops
shouting ``Hurrah!'' pursued them so far beyond the battery that
it was difficult to call them back.

The prisoners were brought down from the battery and among them
was a wounded French general, whom the officers
surrounded. Crowds of wounded---some known to Pierre and some
unknown---Russians and French, with faces distorted by suffering,
walked, crawled, and were carried on stretchers from the
battery. Pierre again went up onto the knoll where he had spent
over an hour, and of that family circle which had received him as
a member he did not find a single one. There were many dead whom
he did not know, but some he recognized. The young officer still
sat in the same way, bent double, in a pool of blood at the edge
of the earth wall.  The red-faced man was still twitching, but
they did not carry him away.

Pierre ran down the slope once more.

``Now they will stop it, now they will be horrified at what they
have done!'' he thought, aimlessly going toward a crowd of
stretcher bearers moving from the battlefield.

But behind the veil of smoke the sun was still high, and in front
and especially to the left, near Semenovsk, something seemed to
be seething in the smoke, and the roar of cannon and musketry did
not diminish, but even increased to desperation like a man who,
straining himself, shrieks with all his remaining strength.

% % % % % % % % % % % % % % % % % % % % % % % % % % % % % % % % %
% % % % % % % % % % % % % % % % % % % % % % % % % % % % % % % % %
% % % % % % % % % % % % % % % % % % % % % % % % % % % % % % % % %
% % % % % % % % % % % % % % % % % % % % % % % % % % % % % % % % %
% % % % % % % % % % % % % % % % % % % % % % % % % % % % % % % % %
% % % % % % % % % % % % % % % % % % % % % % % % % % % % % % % % %
% % % % % % % % % % % % % % % % % % % % % % % % % % % % % % % % %
% % % % % % % % % % % % % % % % % % % % % % % % % % % % % % % % %
% % % % % % % % % % % % % % % % % % % % % % % % % % % % % % % % %
% % % % % % % % % % % % % % % % % % % % % % % % % % % % % % % % %
% % % % % % % % % % % % % % % % % % % % % % % % % % % % % % % % %
% % % % % % % % % % % % % % % % % % % % % % % % % % % % % %

\chapter*{Chapter XXXIII} \ifaudio \marginpar{
\href{http://ia801407.us.archive.org/32/items/war_and_peace_10_0904_librivox/war_and_peace_10_33_tolstoy_64kb.mp3}{Audio}}
\fi

\initial{T}{he} chief action of the battle of Borodino was fought within the
seven thousand feet between Borodino and Bagration's
fleches. Beyond that space there was, on the one side, a
demonstration made by the Russians with Uvarov's cavalry at
midday, and on the other side, beyond Utitsa, Poniatowski's
collision with Tuchkov; but these two were detached and feeble
actions in comparison with what took place in the center of the
battlefield. On the field between Borodino and the fleches,
beside the wood, the chief action of the day took place on an
open space visible from both sides and was fought in the simplest
and most artless way.

The battle began on both sides with a cannonade from several
hundred guns.

Then when the whole field was covered with smoke, two divisions,
Campan's and Dessaix's, advanced from the French right, while
Murat's troops advanced on Borodino from their left.

From the Shevardino Redoubt where Napoleon was standing the
fleches were two thirds of a mile away, and it was more than a
mile as the crow flies to Borodino, so that Napoleon could not
see what was happening there, especially as the smoke mingling
with the mist hid the whole locality.  The soldiers of Dessaix's
division advancing against the fleches could only be seen till
they had entered the hollow that lay between them and the
fleches. As soon as they had descended into that hollow, the
smoke of the guns and musketry on the fleches grew so dense that
it covered the whole approach on that side of it. Through the
smoke glimpses could be caught of something black---probably
men---and at times the glint of bayonets. But whether they were
moving or stationary, whether they were French or Russian, could
not be discovered from the Shevardino Redoubt.

The sun had risen brightly and its slanting rays struck straight
into Napoleon's face as, shading his eyes with his hand, he
looked at the fleches. The smoke spread out before them, and at
times it looked as if the smoke were moving, at times as if the
troops moved. Sometimes shouts were heard through the firing, but
it was impossible to tell what was being done there.

Napoleon, standing on the knoll, looked through a field glass,
and in its small circlet saw smoke and men, sometimes his own and
sometimes Russians, but when he looked again with the naked eye,
he could not tell where what he had seen was.

He descended the knoll and began walking up and down before it.

Occasionally he stopped, listened to the firing, and gazed
intently at the battlefield.

But not only was it impossible to make out what was happening
from where he was standing down below, or from the knoll above on
which some of his generals had taken their stand, but even from
the fleches themselves---in which by this time there were now
Russian and now French soldiers, alternately or together, dead,
wounded, alive, frightened, or maddened---even at those fleches
themselves it was impossible to make out what was taking
place. There for several hours amid incessant cannon and musketry
fire, now Russians were seen alone, now Frenchmen alone, now
infantry, and now cavalry: they appeared, fired, fell, collided,
not knowing what to do with one another, screamed, and ran back
again.

From the battlefield adjutants he had sent out, and orderlies
from his marshals, kept galloping up to Napoleon with reports of
the progress of the action, but all these reports were false,
both because it was impossible in the heat of battle to say what
was happening at any given moment and because many of the
adjutants did not go to the actual place of conflict but reported
what they had heard from others; and also because while an
adjutant was riding more than a mile to Napoleon circumstances
changed and the news he brought was already becoming false. Thus
an adjutant galloped up from Murat with tidings that Borodino had
been occupied and the bridge over the Kolocha was in the hands of
the French. The adjutant asked whether Napoleon wished the troops
to cross it? Napoleon gave orders that the troops should form up
on the farther side and wait. But before that order was
given---almost as soon in fact as the adjutant had left
Borodino---the bridge had been retaken by the Russians and
burned, in the very skirmish at which Pierre had been present at
the beginning of the battle.

An adjutant galloped up from the fleches with a pale and
frightened face and reported to Napoleon that their attack had
been repulsed, Campan wounded, and Davout killed; yet at the very
time the adjutant had been told that the French had been
repulsed, the fleches had in fact been recaptured by other French
troops, and Davout was alive and only slightly bruised. On the
basis of these necessarily untrustworthy reports Napoleon gave
his orders, which had either been executed before he gave them or
could not be and were not executed.

The marshals and generals, who were nearer to the field of battle
but, like Napoleon, did not take part in the actual fighting and
only occasionally went within musket range, made their own
arrangements without asking Napoleon and issued orders where and
in what direction to fire and where cavalry should gallop and
infantry should run. But even their orders, like Napoleon's, were
seldom carried out, and then but partially. For the most part
things happened contrary to their orders.  Soldiers ordered to
advance ran back on meeting grapeshot; soldiers ordered to remain
where they were, suddenly, seeing Russians unexpectedly before
them, sometimes rushed back and sometimes forward, and the
cavalry dashed without orders in pursuit of the flying Russians.
In this way two cavalry regiments galloped through the Semenovsk
hollow and as soon as they reached the top of the incline turned
round and galloped full speed back again. The infantry moved in
the same way, sometimes running to quite other places than those
they were ordered to go to. All orders as to where and when to
move the guns, when to send infantry to shoot or horsemen to ride
down the Russian infantry---all such orders were given by the
officers on the spot nearest to the units concerned, without
asking either Ney, Davout, or Murat, much less Napoleon. They did
not fear getting into trouble for not fulfilling orders or for
acting on their own initiative, for in battle what is at stake is
what is dearest to man---his own life---and it sometimes seems
that safety lies in running back, sometimes in running forward;
and these men who were right in the heat of the battle acted
according to the mood of the moment. In reality, however, all
these movements forward and backward did not improve or alter the
position of the troops. All their rushing and galloping at one
another did little harm, the harm of disablement and death was
caused by the balls and bullets that flew over the fields on
which these men were floundering about. As soon as they left the
place where the balls and bullets were flying about, their
superiors, located in the background, re-formed them and brought
them under discipline and under the influence of that discipline
led them back to the zone of fire, where under the influence of
fear of death they lost their discipline and rushed about
according to the chance promptings of the throng.

% % % % % % % % % % % % % % % % % % % % % % % % % % % % % % % % %
% % % % % % % % % % % % % % % % % % % % % % % % % % % % % % % % %
% % % % % % % % % % % % % % % % % % % % % % % % % % % % % % % % %
% % % % % % % % % % % % % % % % % % % % % % % % % % % % % % % % %
% % % % % % % % % % % % % % % % % % % % % % % % % % % % % % % % %
% % % % % % % % % % % % % % % % % % % % % % % % % % % % % % % % %
% % % % % % % % % % % % % % % % % % % % % % % % % % % % % % % % %
% % % % % % % % % % % % % % % % % % % % % % % % % % % % % % % % %
% % % % % % % % % % % % % % % % % % % % % % % % % % % % % % % % %
% % % % % % % % % % % % % % % % % % % % % % % % % % % % % % % % %
% % % % % % % % % % % % % % % % % % % % % % % % % % % % % % % % %
% % % % % % % % % % % % % % % % % % % % % % % % % % % % % %

\chapter*{Chapter XXXIV} \ifaudio \marginpar{
\href{http://ia801407.us.archive.org/32/items/war_and_peace_10_0904_librivox/war_and_peace_10_34_tolstoy_64kb.mp3}{Audio}}
\fi

\initial{N}{apoleon}'s generals---Davout, Ney, and Murat, who were near that
region of fire and sometimes even entered it---repeatedly led
into it huge masses of well-ordered troops. But contrary to what
had always happened in their former battles, instead of the news
they expected of the enemy's flight, these orderly masses
returned thence as disorganized and terrified mobs. The generals
re-formed them, but their numbers constantly decreased. In the
middle of the day Murat sent his adjutant to Napoleon to demand
reinforcements.

Napoleon sat at the foot of the knoll, drinking punch, when
Murat's adjutant galloped up with an assurance that the Russians
would be routed if His Majesty would let him have another
division.

``Reinforcements?'' said Napoleon in a tone of stern surprise,
looking at the adjutant---a handsome lad with long black curls
arranged like Murat's own---as though he did not understand his
words.

``Reinforcements!'' thought Napoleon to himself. ``How can they
need reinforcements when they already have half the army directed
against a weak, unentrenched Russian wing?''

``Tell the King of Naples,'' said he sternly, ``that it is not
noon yet, and I don't yet see my chessboard clearly. Go!...''

The handsome boy adjutant with the long hair sighed deeply
without removing his hand from his hat and galloped back to where
men were being slaughtered.

Napoleon rose and having summoned Caulaincourt and Berthier began
talking to them about matters unconnected with the battle.

In the midst of this conversation, which was beginning to
interest Napoleon, Berthier's eyes turned to look at a general
with a suite, who was galloping toward the knoll on a lathering
horse. It was Belliard.  Having dismounted he went up to the
Emperor with rapid strides and in a loud voice began boldly
demonstrating the necessity of sending reinforcements. He swore
on his honor that the Russians were lost if the Emperor would
give another division.

Napoleon shrugged his shoulders and continued to pace up and down
without replying. Belliard began talking loudly and eagerly to
the generals of the suite around him.

``You are very fiery, Belliard,'' said Napoleon, when he again
came up to the general. ``In the heat of a battle it is easy to
make a mistake. Go and have another look and then come back to
me.''

Before Belliard was out of sight, a messenger from another part
of the battlefield galloped up.

``Now then, what do you want?'' asked Napoleon in the tone of a
man irritated at being continually disturbed.

``Sire, the prince...'' began the adjutant.

``Asks for reinforcements?'' said Napoleon with an angry gesture.

The adjutant bent his head affirmatively and began to report, but
the Emperor turned from him, took a couple of steps, stopped,
came back, and called Berthier.

``We must give reserves,'' he said, moving his arms slightly
apart. ``Who do you think should be sent there?'' he asked of
Berthier (whom he subsequently termed ``that gosling I have made
an eagle'').

``Send Claparede's division, sire,'' replied Berthier, who knew
all the division's regiments, and battalions by heart.

Napoleon nodded assent.

The adjutant galloped to Claparede's division and a few minutes
later the Young Guards stationed behind the knoll moved
forward. Napoleon gazed silently in that direction.

``No!'' he suddenly said to Berthier. ``I can't send
Claparede. Send Friant's division.''

Though there was no advantage in sending Friant's division
instead of Claparede's, and even an obvious inconvenience and
delay in stopping Claparede and sending Friant now, the order was
carried out exactly.  Napoleon did not notice that in regard to
his army he was playing the part of a doctor who hinders by his
medicines---a role he so justly understood and condemned.

Friant's division disappeared as the others had done into the
smoke of the battlefield. From all sides adjutants continued to
arrive at a gallop and as if by agreement all said the same
thing. They all asked for reinforcements and all said that the
Russians were holding their positions and maintaining a hellish
fire under which the French army was melting away.

Napoleon sat on a campstool, wrapped in thought.

M. de Beausset, the man so fond of travel, having fasted since
morning, came up to the Emperor and ventured respectfully to
suggest lunch to His Majesty.

``I hope I may now congratulate Your Majesty on a victory?'' said
he.

Napoleon silently shook his head in negation. Assuming the
negation to refer only to the victory and not to the lunch, M. de
Beausset ventured with respectful jocularity to remark that there
is no reason for not having lunch when one can get it.

``Go away...'' exclaimed Napoleon suddenly and morosely, and
turned aside.

A beatific smile of regret, repentance, and ecstasy beamed on
M. de Beausset's face and he glided away to the other generals.

Napoleon was experiencing a feeling of depression like that of an
ever-lucky gambler who, after recklessly flinging money about and
always winning, suddenly just when he has calculated all the
chances of the game, finds that the more he considers his play
the more surely he loses.

His troops were the same, his generals the same, the same
preparations had been made, the same dispositions, and the same
proclamation courte et energique, he himself was still the same:
he knew that and knew that he was now even more experienced and
skillful than before. Even the enemy was the same as at
Austerlitz and Friedland---yet the terrible stroke of his arm had
supernaturally become impotent.

All the old methods that had been unfailingly crowned with
success: the concentration of batteries on one point, an attack
by reserves to break the enemy's line, and a cavalry attack by
``the men of iron,'' all these methods had already been employed,
yet not only was there no victory, but from all sides came the
same news of generals killed and wounded, of reinforcements
needed, of the impossibility of driving back the Russians, and of
disorganization among his own troops.

Formerly, after he had given two or three orders and uttered a
few phrases, marshals and adjutants had come galloping up with
congratulations and happy faces, announcing the trophies taken,
the corps of prisoners, bundles of enemy eagles and standards,
cannon and stores, and Murat had only begged leave to loose the
cavalry to gather in the baggage wagons. So it had been at Lodi,
Marengo, Arcola, Jena, Austerlitz, Wagram, and so on. But now
something strange was happening to his troops.

Despite news of the capture of the fleches, Napoleon saw that
this was not the same, not at all the same, as what had happened
in his former battles. He saw that what he was feeling was felt
by all the men about him experienced in the art of war. All their
faces looked dejected, and they all shunned one another's
eyes---only a de Beausset could fail to grasp the meaning of what
was happening.

But Napoleon with his long experience of war well knew the
meaning of a battle not gained by the attacking side in eight
hours, after all efforts had been expended. He knew that it was a
lost battle and that the least accident might now---with the
fight balanced on such a strained center---destroy him and his
army.

When he ran his mind over the whole of this strange Russian
campaign in which not one battle had been won, and in which not a
flag, or cannon, or army corps had been captured in two months,
when he looked at the concealed depression on the faces around
him and heard reports of the Russians still holding their
ground---a terrible feeling like a nightmare took possession of
him, and all the unlucky accidents that might destroy him
occurred to his mind. The Russians might fall on his left wing,
might break through his center, he himself might be killed by a
stray cannon ball. All this was possible. In former battles he
had only considered the possibilities of success, but now
innumerable unlucky chances presented themselves, and he expected
them all. Yes, it was like a dream in which a man fancies that a
ruffian is coming to attack him, and raises his arm to strike
that ruffian a terrible blow which he knows should annihilate
him, but then feels that his arm drops powerless and limp like a
rag, and the horror of unavoidable destruction seizes him in his
helplessness.

The news that the Russians were attacking the left flank of the
French army aroused that horror in Napoleon. He sat silently on a
campstool below the knoll, with head bowed and elbows on his
knees. Berthier approached and suggested that they should ride
along the line to ascertain the position of affairs.

``What? What do you say?'' asked Napoleon. ``Yes, tell them to
bring me my horse.''

He mounted and rode toward Semenovsk.

Amid the powder smoke, slowly dispersing over the whole space
through which Napoleon rode, horses and men were lying in pools
of blood, singly or in heaps. Neither Napoleon nor any of his
generals had ever before seen such horrors or so many slain in
such a small area. The roar of guns, that had not ceased for ten
hours, wearied the ear and gave a peculiar significance to the
spectacle, as music does to tableaux vivants. Napoleon rode up
the high ground at Semenovsk, and through the smoke saw ranks of
men in uniforms of a color unfamiliar to him. They were Russians.

The Russians stood in serried ranks behind Semenovsk village and
its knoll, and their guns boomed incessantly along their line and
sent forth clouds of smoke. It was no longer a battle: it was a
continuous slaughter which could be of no avail either to the
French or the Russians. Napoleon stopped his horse and again fell
into the reverie from which Berthier had aroused him. He could
not stop what was going on before him and around him and was
supposed to be directed by him and to depend on him, and from its
lack of success this affair, for the first time, seemed to him
unnecessary and horrible.

One of the generals rode up to Napoleon and ventured to offer to
lead the Old Guard into action. Ney and Berthier, standing near
Napoleon, exchanged looks and smiled contemptuously at this
general's senseless offer.

Napoleon bowed his head and remained silent a long time.

``At eight hundred leagues from France, I will not have my Guard
destroyed!'' he said, and turning his horse rode back to
Shevardino.

% % % % % % % % % % % % % % % % % % % % % % % % % % % % % % % % %
% % % % % % % % % % % % % % % % % % % % % % % % % % % % % % % % %
% % % % % % % % % % % % % % % % % % % % % % % % % % % % % % % % %
% % % % % % % % % % % % % % % % % % % % % % % % % % % % % % % % %
% % % % % % % % % % % % % % % % % % % % % % % % % % % % % % % % %
% % % % % % % % % % % % % % % % % % % % % % % % % % % % % % % % %
% % % % % % % % % % % % % % % % % % % % % % % % % % % % % % % % %
% % % % % % % % % % % % % % % % % % % % % % % % % % % % % % % % %
% % % % % % % % % % % % % % % % % % % % % % % % % % % % % % % % %
% % % % % % % % % % % % % % % % % % % % % % % % % % % % % % % % %
% % % % % % % % % % % % % % % % % % % % % % % % % % % % % % % % %
% % % % % % % % % % % % % % % % % % % % % % % % % % % % % %

\chapter*{Chapter XXXV} \ifaudio \marginpar{
\href{http://ia801407.us.archive.org/32/items/war_and_peace_10_0904_librivox/war_and_peace_10_35_tolstoy_64kb.mp3}{Audio}}
\fi

\initial{O}{n} the rug-covered bench where Pierre had seen him in the morning
sat Kutuzov, his gray head hanging, his heavy body relaxed. He
gave no orders, but only assented to or dissented from what
others suggested.

``Yes, yes, do that,'' he replied to various proposals. ``Yes,
yes: go, dear boy, and have a look,'' he would say to one or
another of those about him; or, ``No, don't, we'd better wait!''
He listened to the reports that were brought him and gave
directions when his subordinates demanded that of him; but when
listening to the reports it seemed as if he were not interested
in the import of the words spoken, but rather in something
else---in the expression of face and tone of voice of those who
were reporting. By long years of military experience he knew, and
with the wisdom of age understood, that it is impossible for one
man to direct hundreds of thousands of others struggling with
death, and he knew that the result of a battle is decided not by
the orders of a commander-in-chief, nor the place where the
troops are stationed, nor by the number of cannon or of
slaughtered men, but by that intangible force called the spirit
of the army, and he watched this force and guided it in as far as
that was in his power.

Kutuzov's general expression was one of concentrated quiet
attention, and his face wore a strained look as if he found it
difficult to master the fatigue of his old and feeble body.

At eleven o'clock they brought him news that the fleches captured
by the French had been retaken, but that Prince Bagration was
wounded. Kutuzov groaned and swayed his head.

``Ride over to Prince Peter Ivanovich and find out about it
exactly,'' he said to one of his adjutants, and then turned to
the Duke of Wurttemberg who was standing behind him.

``Will Your Highness please take command of the first army?''

Soon after the duke's departure---before he could possibly have
reached Semenovsk---his adjutant came back from him and told
Kutuzov that the duke asked for more troops.

Kutuzov made a grimace and sent an order to Dokhturov to take
over the command of the first army, and a request to the
duke---whom he said he could not spare at such an important
moment---to return to him. When they brought him news that Murat
had been taken prisoner, and the staff officers congratulated
him, Kutuzov smiled.

``Wait a little, gentlemen,'' said he. ``The battle is won, and
there is nothing extraordinary in the capture of Murat. Still, it
is better to wait before we rejoice.''

But he sent an adjutant to take the news round the army.

When Scherbinin came galloping from the left flank with news that
the French had captured the fleches and the village of Semenovsk,
Kutuzov, guessing by the sounds of the battle and by Scherbinin's
looks that the news was bad, rose as if to stretch his legs and,
taking Scherbinin's arm, led him aside.

``Go, my dear fellow,'' he said to Ermolov, ``and see whether
something can't be done.''

Kutuzov was in Gorki, near the center of the Russian
position. The attack directed by Napoleon against our left flank
had been several times repulsed. In the center the French had not
got beyond Borodino, and on their left flank Uvarov's cavalry had
put the French to flight.

Toward three o'clock the French attacks ceased. On the faces of
all who came from the field of battle, and of those who stood
around him, Kutuzov noticed an expression of extreme tension. He
was satisfied with the day's success---a success exceeding his
expectations, but the old man's strength was failing him. Several
times his head dropped low as if it were falling and he dozed
off. Dinner was brought him.

Adjutant General Wolzogen, the man who when riding past Prince
Andrew had said, ``the war should be extended widely,'' and whom
Bagration so detested, rode up while Kutuzov was at
dinner. Wolzogen had come from Barclay de Tolly to report on the
progress of affairs on the left flank.  The sagacious Barclay de
Tolly, seeing crowds of wounded men running back and the
disordered rear of the army, weighed all the circumstances,
concluded that the battle was lost, and sent his favorite officer
to the commander in chief with that news.

Kutuzov was chewing a piece of roast chicken with difficulty and
glanced at Wolzogen with eyes that brightened under their
puckering lids.

Wolzogen, nonchalantly stretching his legs, approached Kutuzov
with a half-contemptuous smile on his lips, scarcely touching the
peak of his cap.

He treated his Serene Highness with a somewhat affected
nonchalance intended to show that, as a highly trained military
man, he left it to Russians to make an idol of this useless old
man, but that he knew whom he was dealing with. ``Der alte Herr''
(as in their own set the Germans called Kutuzov) ``is making
himself very comfortable,'' thought Wolzogen, and looking
severely at the dishes in front of Kutuzov he began to report to
``the old gentleman'' the position of affairs on the left flank
as Barclay had ordered him to and as he himself had seen and
understood it.

``All the points of our position are in the enemy's hands and we
cannot dislodge them for lack of troops, the men are running away
and it is impossible to stop them,'' he reported.

Kutuzov ceased chewing and fixed an astonished gaze on Wolzogen,
as if not understanding what was said to him. Wolzogen, noticing
``the old gentleman's'' agitation, said with a smile:

``I have not considered it right to conceal from your Serene
Highness what I have seen. The troops are in complete
disorder...''

``You have seen? You have seen?...'' Kutuzov shouted. Frowning
and rising quickly, he went up to Wolzogen.

``How... how dare you!...'' he shouted, choking and making a
threatening gesture with his trembling arms: ``How dare you, sir,
say that to me? You know nothing about it. Tell General Barclay
from me that his information is incorrect and that the real
course of the battle is better known to me, the
commander-in-chief, than to him.''

Wolzogen was about to make a rejoinder, but Kutuzov interrupted
him.

``The enemy has been repulsed on the left and defeated on the
right flank. If you have seen amiss, sir, do not allow yourself
to say what you don't know! Be so good as to ride to General
Barclay and inform him of my firm intention to attack the enemy
tomorrow,'' said Kutuzov sternly.

All were silent, and the only sound audible was the heavy
breathing of the panting old general.

``They are repulsed everywhere, for which I thank God and our
brave army!  The enemy is beaten, and tomorrow we shall drive him
from the sacred soil of Russia,'' said Kutuzov crossing himself,
and he suddenly sobbed as his eyes filled with tears.

Wolzogen, shrugging his shoulders and curling his lips, stepped
silently aside, marveling at ``the old gentleman's'' conceited
stupidity.

``Ah, here he is, my hero!'' said Kutuzov to a portly, handsome,
dark-haired general who was just ascending the knoll.

This was Raevski, who had spent the whole day at the most
important part of the field of Borodino.

Raevski reported that the troops were firmly holding their ground
and that the French no longer ventured to attack.

After hearing him, Kutuzov said in French:

``Then you do not think, like some others, that we must
retreat?''

``On the contrary, your Highness, in indecisive actions it is
always the most stubborn who remain victors,'' replied Raevski,
``and in my opinion...''

``Kaysarov!'' Kutuzov called to his adjutant. ``Sit down and
write out the order of the day for tomorrow. And you,'' he
continued, addressing another, ``ride along the line and announce
that tomorrow we attack.''

While Kutuzov was talking to Raevski and dictating the order of
the day, Wolzogen returned from Barclay and said that General
Barclay wished to have written confirmation of the order the
field marshal had given.

Kutuzov, without looking at Wolzogen, gave directions for the
order to be written out which the former commander-in-chief, to
avoid personal responsibility, very judiciously wished to
receive.

And by means of that mysterious indefinable bond which maintains
throughout an army one and the same temper, known as ``the spirit
of the army,'' and which constitutes the sinew of war, Kutuzov's
words, his order for a battle next day, immediately became known
from one end of the army to the other.

It was far from being the same words or the same order that
reached the farthest links of that chain. The tales passing from
mouth to mouth at different ends of the army did not even
resemble what Kutuzov had said, but the sense of his words spread
everywhere because what he said was not the outcome of cunning
calculations, but of a feeling that lay in the
commander-in-chief's soul as in that of every Russian.

And on learning that tomorrow they were to attack the enemy, and
hearing from the highest quarters a confirmation of what they
wanted to believe, the exhausted, wavering men felt comforted and
inspirited.

% % % % % % % % % % % % % % % % % % % % % % % % % % % % % % % % %
% % % % % % % % % % % % % % % % % % % % % % % % % % % % % % % % %
% % % % % % % % % % % % % % % % % % % % % % % % % % % % % % % % %
% % % % % % % % % % % % % % % % % % % % % % % % % % % % % % % % %
% % % % % % % % % % % % % % % % % % % % % % % % % % % % % % % % %
% % % % % % % % % % % % % % % % % % % % % % % % % % % % % % % % %
% % % % % % % % % % % % % % % % % % % % % % % % % % % % % % % % %
% % % % % % % % % % % % % % % % % % % % % % % % % % % % % % % % %
% % % % % % % % % % % % % % % % % % % % % % % % % % % % % % % % %
% % % % % % % % % % % % % % % % % % % % % % % % % % % % % % % % %
% % % % % % % % % % % % % % % % % % % % % % % % % % % % % % % % %
% % % % % % % % % % % % % % % % % % % % % % % % % % % % % %

\chapter*{Chapter XXXVI} \ifaudio \marginpar{
\href{http://ia801407.us.archive.org/32/items/war_and_peace_10_0904_librivox/war_and_peace_10_36_tolstoy_64kb.mp3}{Audio}}
\fi

\initial{P}{rince} Andrew's regiment was among the reserves which till after
one o'clock were stationed inactive behind Semenovsk, under heavy
artillery fire. Toward two o'clock the regiment, having already
lost more than two hundred men, was moved forward into a trampled
oatfield in the gap between Semenovsk and the Knoll Battery,
where thousands of men perished that day and on which an intense,
concentrated fire from several hundred enemy guns was directed
between one and two o'clock.

Without moving from that spot or firing a single shot the
regiment here lost another third of its men. From in front and
especially from the right, in the unlifting smoke the guns
boomed, and out of the mysterious domain of smoke that overlay
the whole space in front, quick hissing cannon balls and slow
whistling shells flew unceasingly. At times, as if to allow them
a respite, a quarter of an hour passed during which the cannon
balls and shells all flew overhead, but sometimes several men
were torn from the regiment in a minute and the slain were
continually being dragged away and the wounded carried off.

With each fresh blow less and less chance of life remained for
those not yet killed. The regiment stood in columns of battalion,
three hundred paces apart, but nevertheless the men were always
in one and the same mood. All alike were taciturn and
morose. Talk was rarely heard in the ranks, and it ceased
altogether every time the thud of a successful shot and the cry
of ``stretchers!'' was heard. Most of the time, by their
officers' order, the men sat on the ground. One, having taken off
his shako, carefully loosened the gathers of its lining and drew
them tight again; another, rubbing some dry clay between his
palms, polished his bayonet; another fingered the strap and
pulled the buckle of his bandolier, while another smoothed and
refolded his leg bands and put his boots on again. Some built
little houses of the tufts in the plowed ground, or plaited
baskets from the straw in the cornfield. All seemed fully
absorbed in these pursuits. When men were killed or wounded, when
rows of stretchers went past, when some troops retreated, and
when great masses of the enemy came into view through the smoke,
no one paid any attention to these things. But when our artillery
or cavalry advanced or some of our infantry were seen to move
forward, words of approval were heard on all sides. But the
liveliest attention was attracted by occurrences quite apart
from, and unconnected with, the battle. It was as if the minds of
these morally exhausted men found relief in everyday, commonplace
occurrences. A battery of artillery was passing in front of the
regiment. The horse of an ammunition cart put its leg over a
trace.  ``Hey, look at the trace horse!... Get her leg out!
She'll fall... Ah, they don't see it!'' came identical shouts
from the ranks all along the regiment. Another time, general
attention was attracted by a small brown dog, coming heaven knows
whence, which trotted in a preoccupied manner in front of the
ranks with tail stiffly erect till suddenly a shell fell close
by, when it yelped, tucked its tail between its legs, and darted
aside. Yells and shrieks of laughter rose from the whole
regiment. But such distractions lasted only a moment, and for
eight hours the men had been inactive, without food, in constant
fear of death, and their pale and gloomy faces grew ever paler
and gloomier.

Prince Andrew, pale and gloomy like everyone in the regiment,
paced up and down from the border of one patch to another, at the
edge of the meadow beside an oatfield, with head bowed and arms
behind his back.  There was nothing for him to do and no orders
to be given. Everything went on of itself. The killed were
dragged from the front, the wounded carried away, and the ranks
closed up. If any soldiers ran to the rear they returned
immediately and hastily. At first Prince Andrew, considering it
his duty to rouse the courage of the men and to set them an
example, walked about among the ranks, but he soon became
convinced that this was unnecessary and that there was nothing he
could teach them. All the powers of his soul, as of every soldier
there, were unconsciously bent on avoiding the contemplation of
the horrors of their situation. He walked along the meadow,
dragging his feet, rustling the grass, and gazing at the dust
that covered his boots; now he took big strides trying to keep to
the footprints left on the meadow by the mowers, then he counted
his steps, calculating how often he must walk from one strip to
another to walk a mile, then he stripped the flowers from the
wormwood that grew along a boundary rut, rubbed them in his
palms, and smelled their pungent, sweetly bitter scent. Nothing
remained of the previous day's thoughts. He thought of
nothing. He listened with weary ears to the ever-recurring
sounds, distinguishing the whistle of flying projectiles from the
booming of the reports, glanced at the tiresomely familiar faces
of the men of the first battalion, and waited.  ``Here it
comes... this one is coming our way again!'' he thought,
listening to an approaching whistle in the hidden region of
smoke. ``One, another! Again! It has hit...'' He stopped and
looked at the ranks. ``No, it has gone over. But this one has
hit!'' And again he started trying to reach the boundary strip in
sixteen paces. A whizz and a thud! Five paces from him, a cannon
ball tore up the dry earth and disappeared. A chill ran down his
back. Again he glanced at the ranks. Probably many had been
hit---a large crowd had gathered near the second battalion.

``Adjutant!'' he shouted. ``Order them not to crowd together.''

The adjutant, having obeyed this instruction, approached Prince
Andrew.  From the other side a battalion commander rode up.

``Look out!'' came a frightened cry from a soldier and, like a
bird whirring in rapid flight and alighting on the ground, a
shell dropped with little noise within two steps of Prince Andrew
and close to the battalion commander's horse. The horse first,
regardless of whether it was right or wrong to show fear,
snorted, reared almost throwing the major, and galloped
aside. The horse's terror infected the men.

``Lie down!'' cried the adjutant, throwing himself flat on the
ground.

Prince Andrew hesitated. The smoking shell spun like a top
between him and the prostrate adjutant, near a wormwood plant
between the field and the meadow.

``Can this be death?'' thought Prince Andrew, looking with a
quite new, envious glance at the grass, the wormwood, and the
streamlet of smoke that curled up from the rotating black
ball. ``I cannot, I do not wish to die. I love life---I love this
grass, this earth, this air...'' He thought this, and at the same
time remembered that people were looking at him.

``It's shameful, sir!'' he said to the adjutant. ``What...''

He did not finish speaking. At one and the same moment came the
sound of an explosion, a whistle of splinters as from a breaking
window frame, a suffocating smell of powder, and Prince Andrew
started to one side, raising his arm, and fell on his
chest. Several officers ran up to him.  From the right side of
his abdomen, blood was welling out making a large stain on the
grass.

The militiamen with stretchers who were called up stood behind
the officers. Prince Andrew lay on his chest with his face in the
grass, breathing heavily and noisily.

``What are you waiting for? Come along!''

The peasants went up and took him by his shoulders and legs, but
he moaned piteously and, exchanging looks, they set him down
again.

``Pick him up, lift him, it's all the same!'' cried someone.

They again took him by the shoulders and laid him on the
stretcher.

``Ah, God! My God! What is it? The stomach? That means death!  My
God!''---voices among the officers were heard saying.

``It flew a hair's breadth past my ear,'' said the adjutant.

The peasants, adjusting the stretcher to their shoulders, started
hurriedly along the path they had trodden down, to the dressing
station.

``Keep in step! Ah... those peasants!'' shouted an officer,
seizing by their shoulders and checking the peasants, who were
walking unevenly and jolting the stretcher.

``Get into step, Fedor... I say, Fedor!'' said the foremost
peasant.

``Now that's right!'' said the one behind joyfully, when he had
got into step.

``Your excellency! Eh, Prince!'' said the trembling voice of
Timokhin, who had run up and was looking down on the stretcher.

Prince Andrew opened his eyes and looked up at the speaker from
the stretcher into which his head had sunk deep and again his
eyelids drooped.

The militiamen carried Prince Andrew to the dressing station by
the wood, where wagons were stationed. The dressing station
consisted of three tents with flaps turned back, pitched at the
edge of a birch wood.  In the wood, wagons and horses were
standing. The horses were eating oats from their movable troughs
and sparrows flew down and pecked the grains that fell. Some
crows, scenting blood, flew among the birch trees cawing
impatiently. Around the tents, over more than five acres,
bloodstained men in various garbs stood, sat, or lay. Around the
wounded stood crowds of soldier stretcher-bearers with dismal and
attentive faces, whom the officers keeping order tried in vain to
drive from the spot. Disregarding the officers' orders, the
soldiers stood leaning against their stretchers and gazing
intently, as if trying to comprehend the difficult problem of
what was taking place before them. From the tents came now loud
angry cries and now plaintive groans. Occasionally dressers ran
out to fetch water, or to point out those who were to be brought
in next. The wounded men awaiting their turn outside the tents
groaned, sighed, wept, screamed, swore, or asked for vodka. Some
were delirious. Prince Andrew's bearers, stepping over the
wounded who had not yet been bandaged, took him, as a regimental
commander, close up to one of the tents and there stopped,
awaiting instructions. Prince Andrew opened his eyes and for a
long time could not make out what was going on around him. He
remembered the meadow, the wormwood, the field, the whirling
black ball, and his sudden rush of passionate love of life. Two
steps from him, leaning against a branch and talking loudly and
attracting general attention, stood a tall, handsome,
black-haired noncommissioned officer with a bandaged head. He had
been wounded in the head and leg by bullets. Around him, eagerly
listening to his talk, a crowd of wounded and stretcher-bearers
was gathered.

``We kicked him out from there so that he chucked everything, we
grabbed the King himself!'' cried he, looking around him with
eyes that glittered with fever. ``If only reserves had come up
just then, lads, there wouldn't have been nothing left of him! I
tell you surely...''

Like all the others near the speaker, Prince Andrew looked at him
with shining eyes and experienced a sense of comfort. ``But isn't
it all the same now?'' thought he. ``And what will be there, and
what has there been here? Why was I so reluctant to part with
life? There was something in this life I did not and do not
understand.''

% % % % % % % % % % % % % % % % % % % % % % % % % % % % % % % % %
% % % % % % % % % % % % % % % % % % % % % % % % % % % % % % % % %
% % % % % % % % % % % % % % % % % % % % % % % % % % % % % % % % %
% % % % % % % % % % % % % % % % % % % % % % % % % % % % % % % % %
% % % % % % % % % % % % % % % % % % % % % % % % % % % % % % % % %
% % % % % % % % % % % % % % % % % % % % % % % % % % % % % % % % %
% % % % % % % % % % % % % % % % % % % % % % % % % % % % % % % % %
% % % % % % % % % % % % % % % % % % % % % % % % % % % % % % % % %
% % % % % % % % % % % % % % % % % % % % % % % % % % % % % % % % %
% % % % % % % % % % % % % % % % % % % % % % % % % % % % % % % % %
% % % % % % % % % % % % % % % % % % % % % % % % % % % % % % % % %
% % % % % % % % % % % % % % % % % % % % % % % % % % % % % %

\chapter*{Chapter XXXVII} \ifaudio \marginpar{
\href{http://ia801407.us.archive.org/32/items/war_and_peace_10_0904_librivox/war_and_peace_10_37_tolstoy_64kb.mp3}{Audio}}
\fi

\initial{O}{ne} of the doctors came out of the tent in a bloodstained apron,
holding a cigar between the thumb and little finger of one of his
small bloodstained hands, so as not to smear it. He raised his
head and looked about him, but above the level of the wounded
men. He evidently wanted a little respite. After turning his head
from right to left for some time, he sighed and looked down.

``All right, immediately,'' he replied to a dresser who pointed
Prince Andrew out to him, and he told them to carry him into the
tent.

Murmurs arose among the wounded who were waiting.

``It seems that even in the next world only the gentry are to
have a chance!'' remarked one.

Prince Andrew was carried in and laid on a table that had only
just been cleared and which a dresser was washing down. Prince
Andrew could not make out distinctly what was in that tent. The
pitiful groans from all sides and the torturing pain in his
thigh, stomach, and back distracted him. All he saw about him
merged into a general impression of naked, bleeding human bodies
that seemed to fill the whole of the low tent, as a few weeks
previously, on that hot August day, such bodies had filled the
dirty pond beside the Smolensk road. Yes, it was the same flesh,
the same chair a canon, the sight of which had even then filled
him with horror, as by a presentiment.

There were three operating tables in the tent. Two were occupied,
and on the third they placed Prince Andrew. For a little while he
was left alone and involuntarily witnessed what was taking place
on the other two tables. On the nearest one sat a Tartar,
probably a Cossack, judging by the uniform thrown down beside
him. Four soldiers were holding him, and a spectacled doctor was
cutting into his muscular brown back.

``Ooh, ooh, ooh!'' grunted the Tartar, and suddenly lifting up
his swarthy snub-nosed face with its high cheekbones, and baring
his white teeth, he began to wriggle and twitch his body and
utter piercing, ringing, and prolonged yells. On the other table,
round which many people were crowding, a tall well-fed man lay on
his back with his head thrown back.  His curly hair, its color,
and the shape of his head seemed strangely familiar to Prince
Andrew. Several dressers were pressing on his chest to hold him
down. One large, white, plump leg twitched rapidly all the time
with a feverish tremor. The man was sobbing and choking
convulsively. Two doctors---one of whom was pale and
trembling---were silently doing something to this man's other,
gory leg. When he had finished with the Tartar, whom they covered
with an overcoat, the spectacled doctor came up to Prince Andrew,
wiping his hands.

He glanced at Prince Andrew's face and quickly turned away.

``Undress him! What are you waiting for?'' he cried angrily to
the dressers.

His very first, remotest recollections of childhood came back to
Prince Andrew's mind when the dresser with sleeves rolled up
began hastily to undo the buttons of his clothes and undressed
him. The doctor bent down over the wound, felt it, and sighed
deeply. Then he made a sign to someone, and the torturing pain in
his abdomen caused Prince Andrew to lose consciousness. When he
came to himself the splintered portions of his thighbone had been
extracted, the torn flesh cut away, and the wound bandaged. Water
was being sprinkled on his face. As soon as Prince Andrew opened
his eyes, the doctor bent over, kissed him silently on the lips,
and hurried away.

After the sufferings he had been enduring, Prince Andrew enjoyed
a blissful feeling such as he had not experienced for a long
time. All the best and happiest moments of his life---especially
his earliest childhood, when he used to be undressed and put to
bed, and when leaning over him his nurse sang him to sleep and
he, burying his head in the pillow, felt happy in the mere
consciousness of life---returned to his memory, not merely as
something past but as something present.

The doctors were busily engaged with the wounded man the shape of
whose head seemed familiar to Prince Andrew: they were lifting
him up and trying to quiet him.

``Show it to me... Oh, ooh... Oh! Oh, ooh!'' his frightened moans
could be heard, subdued by suffering and broken by sobs.

Hearing those moans Prince Andrew wanted to weep. Whether because
he was dying without glory, or because he was sorry to part with
life, or because of those memories of a childhood that could not
return, or because he was suffering and others were suffering and
that man near him was groaning so piteously---he felt like
weeping childlike, kindly, and almost happy tears.

The wounded man was shown his amputated leg stained with clotted
blood and with the boot still on.

``Oh! Oh, ooh!'' he sobbed, like a woman.

The doctor who had been standing beside him, preventing Prince
Andrew from seeing his face, moved away.

``My God! What is this? Why is he here?'' said Prince Andrew to
himself.

In the miserable, sobbing, enfeebled man whose leg had just been
amputated, he recognized Anatole Kuragin. Men were supporting him
in their arms and offering him a glass of water, but his
trembling, swollen lips could not grasp its rim. Anatole was
sobbing painfully. ``Yes, it is he! Yes, that man is somehow
closely and painfully connected with me,'' thought Prince Andrew,
not yet clearly grasping what he saw before him.  ``What is the
connection of that man with my childhood and life?'' he asked
himself without finding an answer. And suddenly a new unexpected
memory from that realm of pure and loving childhood presented
itself to him. He remembered Natasha as he had seen her for the
first time at the ball in 1810, with her slender neck and arms
and with a frightened happy face ready for rapture, and love and
tenderness for her, stronger and more vivid than ever, awoke in
his soul. He now remembered the connection that existed between
himself and this man who was dimly gazing at him through tears
that filled his swollen eyes. He remembered everything, and
ecstatic pity and love for that man overflowed his happy heart.

Prince Andrew could no longer restrain himself and wept tender
loving tears for his fellow men, for himself, and for his own and
their errors.

``Compassion, love of our brothers, for those who love us and for
those who hate us, love of our enemies; yes, that love which God
preached on earth and which Princess Mary taught me and I did not
understand---that is what made me sorry to part with life, that
is what remained for me had I lived. But now it is too late. I
know it!''

% % % % % % % % % % % % % % % % % % % % % % % % % % % % % % % % %
% % % % % % % % % % % % % % % % % % % % % % % % % % % % % % % % %
% % % % % % % % % % % % % % % % % % % % % % % % % % % % % % % % %
% % % % % % % % % % % % % % % % % % % % % % % % % % % % % % % % %
% % % % % % % % % % % % % % % % % % % % % % % % % % % % % % % % %
% % % % % % % % % % % % % % % % % % % % % % % % % % % % % % % % %
% % % % % % % % % % % % % % % % % % % % % % % % % % % % % % % % %
% % % % % % % % % % % % % % % % % % % % % % % % % % % % % % % % %
% % % % % % % % % % % % % % % % % % % % % % % % % % % % % % % % %
% % % % % % % % % % % % % % % % % % % % % % % % % % % % % % % % %
% % % % % % % % % % % % % % % % % % % % % % % % % % % % % % % % %
% % % % % % % % % % % % % % % % % % % % % % % % % % % % % %

\chapter*{Chapter XXXVIII} \ifaudio \marginpar{
\href{http://ia801407.us.archive.org/32/items/war_and_peace_10_0904_librivox/war_and_peace_10_38_tolstoy_64kb.mp3}{Audio}}
\fi

\initial{T}{he} terrible spectacle of the battlefield covered with dead and
wounded, together with the heaviness of his head and the news
that some twenty generals he knew personally had been killed or
wounded, and the consciousness of the impotence of his once
mighty arm, produced an unexpected impression on Napoleon who
usually liked to look at the killed and wounded, thereby, he
considered, testing his strength of mind. This day the horrible
appearance of the battlefield overcame that strength of mind
which he thought constituted his merit and his greatness. He rode
hurriedly from the battlefield and returned to the Shevardino
knoll, where he sat on his campstool, his sallow face swollen and
heavy, his eyes dim, his nose red, and his voice hoarse,
involuntarily listening, with downcast eyes, to the sounds of
firing.  With painful dejection he awaited the end of this
action, in which he regarded himself as a participant and which
he was unable to arrest. A personal, human feeling for a brief
moment got the better of the artificial phantasm of life he had
served so long. He felt in his own person the sufferings and
death he had witnessed on the battlefield. The heaviness of his
head and chest reminded him of the possibility of suffering and
death for himself. At that moment he did not desire Moscow, or
victory, or glory (what need had he for any more glory?). The one
thing he wished for was rest, tranquillity, and freedom. But when
he had been on the Semenovsk heights the artillery commander had
proposed to him to bring several batteries of artillery up to
those heights to strengthen the fire on the Russian troops
crowded in front of Knyazkovo.  Napoleon had assented and had
given orders that news should be brought to him of the effect
those batteries produced.

An adjutant came now to inform him that the fire of two hundred
guns had been concentrated on the Russians, as he had ordered,
but that they still held their ground.

``Our fire is mowing them down by rows, but still they hold on,''
said the adjutant.

``They want more!...'' said Napoleon in a hoarse voice.

``Sire?'' asked the adjutant who had not heard the remark.

``They want more!'' croaked Napoleon frowning. ``Let them have
it!''

Even before he gave that order the thing he did not desire, and
for which he gave the order only because he thought it was
expected of him, was being done. And he fell back into that
artificial realm of imaginary greatness, and again---as a horse
walking a treadmill thinks it is doing something for itself---he
submissively fulfilled the cruel, sad, gloomy, and inhuman role
predestined for him.

And not for that day and hour alone were the mind and conscience
darkened of this man on whom the responsibility for what was
happening lay more than on all the others who took part in
it. Never to the end of his life could he understand goodness,
beauty, or truth, or the significance of his actions which were
too contrary to goodness and truth, too remote from everything
human, for him ever to be able to grasp their meaning. He could
not disavow his actions, belauded as they were by half the world,
and so he had to repudiate truth, goodness, and all humanity.

Not only on that day, as he rode over the battlefield strewn with
men killed and maimed (by his will as he believed), did he reckon
as he looked at them how many Russians there were for each
Frenchman and, deceiving himself, find reason for rejoicing in
the calculation that there were five Russians for every
Frenchman. Not on that day alone did he write in a letter to
Paris that ``the battle field was superb,'' because fifty
thousand corpses lay there, but even on the island of St.  Helena
in the peaceful solitude where he said he intended to devote his
leisure to an account of the great deeds he had done, he wrote:

The Russian war should have been the most popular war of modern
times: it was a war of good sense, for real interests, for the
tranquillity and security of all; it was purely pacific and
conservative.

It was a war for a great cause, the end of uncertainties and the
beginning of security. A new horizon and new labors were opening
out, full of well-being and prosperity for all. The European
system was already founded; all that remained was to organize it.

Satisfied on these great points and with tranquility everywhere,
I too should have had my Congress and my Holy Alliance. Those
ideas were stolen from me. In that reunion of great sovereigns we
should have discussed our interests like one family, and have
rendered account to the peoples as clerk to master.

Europe would in this way soon have been, in fact, but one people,
and anyone who traveled anywhere would have found himself always
in the common fatherland. I should have demanded the freedom of
all navigable rivers for everybody, that the seas should be
common to all, and that the great standing armies should be
reduced henceforth to mere guards for the sovereigns.

On returning to France, to the bosom of the great, strong,
magnificent, peaceful, and glorious fatherland, I should have
proclaimed her frontiers immutable; all future wars purely
defensive, all aggrandizement antinational. I should have
associated my son in the Empire; my dictatorship would have been
finished, and his constitutional reign would have begun.

Paris would have been the capital of the world, and the French
the envy of the nations!

My leisure then, and my old age, would have been devoted, in
company with the Empress and during the royal apprenticeship of
my son, to leisurely visiting, with our own horses and like a
true country couple, every corner of the Empire, receiving
complaints, redressing wrongs, and scattering public buildings
and benefactions on all sides and everywhere.

Napoleon, predestined by Providence for the gloomy role of
executioner of the peoples, assured himself that the aim of his
actions had been the peoples' welfare and that he could control
the fate of millions and by the employment of power confer
benefactions.

``Of four hundred thousand who crossed the Vistula,'' he wrote
further of the Russian war, ``half were Austrians, Prussians,
Saxons, Poles, Bavarians, Wurttembergers, Mecklenburgers,
Spaniards, Italians, and Neapolitans. The Imperial army, strictly
speaking, was one third composed of Dutch, Belgians, men from the
borders of the Rhine, Piedmontese, Swiss, Genevese, Tuscans,
Romans, inhabitants of the Thirty-second Military Division, of
Bremen, of Hamburg, and so on: it included scarcely a hundred and
forty thousand who spoke French. The Russian expedition actually
cost France less than fifty thousand men; the Russian army in its
retreat from Vilna to Moscow lost in the various battles four
times more men than the French army; the burning of Moscow cost
the lives of a hundred thousand Russians who died of cold and
want in the woods; finally, in its march from Moscow to the Oder
the Russian army also suffered from the severity of the season;
so that by the time it reached Vilna it numbered only fifty
thousand, and at Kalisch less than eighteen thousand.''

He imagined that the war with Russia came about by his will, and
the horrors that occurred did not stagger his soul. He boldly
took the whole responsibility for what happened, and his darkened
mind found justification in the belief that among the hundreds of
thousands who perished there were fewer Frenchmen than Hessians
and Bavarians.

% % % % % % % % % % % % % % % % % % % % % % % % % % % % % % % % %
% % % % % % % % % % % % % % % % % % % % % % % % % % % % % % % % %
% % % % % % % % % % % % % % % % % % % % % % % % % % % % % % % % %
% % % % % % % % % % % % % % % % % % % % % % % % % % % % % % % % %
% % % % % % % % % % % % % % % % % % % % % % % % % % % % % % % % %
% % % % % % % % % % % % % % % % % % % % % % % % % % % % % % % % %
% % % % % % % % % % % % % % % % % % % % % % % % % % % % % % % % %
% % % % % % % % % % % % % % % % % % % % % % % % % % % % % % % % %
% % % % % % % % % % % % % % % % % % % % % % % % % % % % % % % % %
% % % % % % % % % % % % % % % % % % % % % % % % % % % % % % % % %
% % % % % % % % % % % % % % % % % % % % % % % % % % % % % % % % %
% % % % % % % % % % % % % % % % % % % % % % % % % % % % % %

\chapter*{Chapter XXXIX} \ifaudio \marginpar{
\href{http://ia801407.us.archive.org/32/items/war_and_peace_10_0904_librivox/war_and_peace_10_39_tolstoy_64kb.mp3}{Audio}}
\fi

\initial{S}{everal} tens of thousands of the slain lay in diverse postures
and various uniforms on the fields and meadows belonging to the
Davydov family and to the crown serfs---those fields and meadows
where for hundreds of years the peasants of Borodino, Gorki,
Shevardino, and Semenovsk had reaped their harvests and pastured
their cattle. At the dressing stations the grass and earth were
soaked with blood for a space of some three acres around. Crowds
of men of various arms, wounded and unwounded, with frightened
faces, dragged themselves back to Mozhaysk from the one army and
back to Valuevo from the other. Other crowds, exhausted and
hungry, went forward led by their officers. Others held their
ground and continued to fire.

Over the whole field, previously so gaily beautiful with the
glitter of bayonets and cloudlets of smoke in the morning sun,
there now spread a mist of damp and smoke and a strange acid
smell of saltpeter and blood.  Clouds gathered and drops of rain
began to fall on the dead and wounded, on the frightened,
exhausted, and hesitating men, as if to say: ``Enough, men!
Enough! Cease... bethink yourselves! What are you doing?''

To the men of both sides alike, worn out by want of food and
rest, it began equally to appear doubtful whether they should
continue to slaughter one another; all the faces expressed
hesitation, and the question arose in every soul: ``For what, for
whom, must I kill and be killed?... You may go and kill whom you
please, but I don't want to do so anymore!'' By evening this
thought had ripened in every soul. At any moment these men might
have been seized with horror at what they were doing and might
have thrown up everything and run away anywhere.

But though toward the end of the battle the men felt all the
horror of what they were doing, though they would have been glad
to leave off, some incomprehensible, mysterious power continued
to control them, and they still brought up the charges, loaded,
aimed, and applied the match, though only one artilleryman
survived out of every three, and though they stumbled and panted
with fatigue, perspiring and stained with blood and powder. The
cannon balls flew just as swiftly and cruelly from both sides,
crushing human bodies, and that terrible work which was not done
by the will of a man but at the will of Him who governs men and
worlds continued.

Anyone looking at the disorganized rear of the Russian army would
have said that, if only the French made one more slight effort,
it would disappear; and anyone looking at the rear of the French
army would have said that the Russians need only make one more
slight effort and the French would be destroyed. But neither the
French nor the Russians made that effort, and the flame of battle
burned slowly out.

The Russians did not make that effort because they were not
attacking the French. At the beginning of the battle they stood
blocking the way to Moscow and they still did so at the end of
the battle as at the beginning. But even had the aim of the
Russians been to drive the French from their positions, they
could not have made this last effort, for all the Russian troops
had been broken up, there was no part of the Russian army that
had not suffered in the battle, and though still holding their
positions they had lost ONE HALF of their army.

The French, with the memory of all their former victories during
fifteen years, with the assurance of Napoleon's invincibility,
with the consciousness that they had captured part of the
battlefield and had lost only a quarter of their men and still
had their Guards intact, twenty thousand strong, might easily
have made that effort. The French who had attacked the Russian
army in order to drive it from its position ought to have made
that effort, for as long as the Russians continued to block the
road to Moscow as before, the aim of the French had not been
attained and all their efforts and losses were in vain. But the
French did not make that effort. Some historians say that
Napoleon need only have used his Old Guards, who were intact, and
the battle would have been won. To speak of what would have
happened had Napoleon sent his Guards is like talking of what
would happen if autumn became spring. It could not be. Napoleon
did not give his Guards, not because he did not want to, but
because it could not be done. All the generals, officers, and
soldiers of the French army knew it could not be done, because
the flagging spirit of the troops would not permit it.

It was not Napoleon alone who had experienced that nightmare
feeling of the mighty arm being stricken powerless, but all the
generals and soldiers of his army whether they had taken part in
the battle or not, after all their experience of previous
battles---when after one tenth of such efforts the enemy had
fled---experienced a similar feeling of terror before an enemy
who, after losing HALF his men, stood as threateningly at the end
as at the beginning of the battle. The moral force of the
attacking French army was exhausted. Not that sort of victory
which is defined by the capture of pieces of material fastened to
sticks, called standards, and of the ground on which the troops
had stood and were standing, but a moral victory that convinces
the enemy of the moral superiority of his opponent and of his own
impotence was gained by the Russians at Borodino. The French
invaders, like an infuriated animal that has in its onslaught
received a mortal wound, felt that they were perishing, but could
not stop, any more than the Russian army, weaker by one half,
could help swerving. By impetus gained, the French army was still
able to roll forward to Moscow, but there, without further effort
on the part of the Russians, it had to perish, bleeding from the
mortal wound it had received at Borodino. The direct consequence
of the battle of Borodino was Napoleon's senseless flight from
Moscow, his retreat along the old Smolensk road, the destruction
of the invading army of five hundred thousand men, and the
downfall of Napoleonic France, on which at Borodino for the first
time the hand of an opponent of stronger spirit had been laid.