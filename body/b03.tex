\part*{Book Three: 1805}

% % % % % % % % % % % % % % % % % % % % % % % % % % % % % % % % %
% % % % % % % % % % % % % % % % % % % % % % % % % % % % % % % % %
% % % % % % % % % % % % % % % % % % % % % % % % % % % % % % % % %
% % % % % % % % % % % % % % % % % % % % % % % % % % % % % % % % %
% % % % % % % % % % % % % % % % % % % % % % % % % % % % % % % % %
% % % % % % % % % % % % % % % % % % % % % % % % % % % % % % % % %
% % % % % % % % % % % % % % % % % % % % % % % % % % % % % % % % %
% % % % % % % % % % % % % % % % % % % % % % % % % % % % % % % % %
% % % % % % % % % % % % % % % % % % % % % % % % % % % % % % % % %
% % % % % % % % % % % % % % % % % % % % % % % % % % % % % % % % %
% % % % % % % % % % % % % % % % % % % % % % % % % % % % % % % % %
% % % % % % % % % % % % % % % % % % % % % % % % % % % % % %

\chapter*{Chapter I}
\ifaudio
\marginpar{
\href{http://ia800208.us.archive.org/14/items/war_and_peace_03_0712_librivox/war_and_peace_03_01_tolstoy_64kb.mp3}{Audio}} 
\fi

\lettrine[lines=2, loversize=0.3, lraise=0]{\initfamily P}{rince }
Vasili was not a man who deliberately thought out his
plans.  Still less did he think of injuring anyone for his own
advantage. He was merely a man of the world who had got on and to
whom getting on had become a habit. Schemes and devices for which
he never rightly accounted to himself, but which formed the whole
interest of his life, were constantly shaping themselves in his
mind, arising from the circumstances and persons he met. Of these
plans he had not merely one or two in his head but dozens, some
only beginning to form themselves, some approaching achievement,
and some in course of disintegration. He did not, for instance,
say to himself: ``This man now has influence, I must gain his
confidence and friendship and through him obtain a special
grant.'' Nor did he say to himself: ``Pierre is a rich man, I
must entice him to marry my daughter and lend me the forty
thousand rubles I need.''  But when he came across a man of
position his instinct immediately told him that this man could be
useful, and without any premeditation Prince Vasili took the
first opportunity to gain his confidence, flatter him, become
intimate with him, and finally make his request.

He had Pierre at hand in Moscow and procured for him an
appointment as Gentleman of the Bedchamber, which at that time
conferred the status of Councilor of State, and insisted on the
young man accompanying him to Petersburg and staying at his
house. With apparent absent-mindedness, yet with unhesitating
assurance that he was doing the right thing, Prince Vasili did
everything to get Pierre to marry his daughter. Had he thought
out his plans beforehand he could not have been so natural and
shown such unaffected familiarity in intercourse with everybody
both above and below him in social standing. Something always
drew him toward those richer and more powerful than himself and
he had rare skill in seizing the most opportune moment for making
use of people.

Pierre, on unexpectedly becoming Count Bezukhov and a rich man,
felt himself after his recent loneliness and freedom from cares
so beset and preoccupied that only in bed was he able to be by
himself. He had to sign papers, to present himself at government
offices, the purpose of which was not clear to him, to question
his chief steward, to visit his estate near Moscow, and to
receive many people who formerly did not even wish to know of his
existence but would now have been offended and grieved had he
chosen not to see them. These different people---businessmen,
relations, and acquaintances alike---were all disposed to treat
the young heir in the most friendly and flattering manner: they
were all evidently firmly convinced of Pierre's noble
qualities. He was always hearing such words as: ``With your
remarkable kindness,'' or, ``With your excellent heart,'' ``You
are yourself so honorable Count,'' or, ``Were he as clever as
you,'' and so on, till he began sincerely to believe in his own
exceptional kindness and extraordinary intelligence, the more so
as in the depth of his heart it had always seemed to him that he
really was very kind and intelligent. Even people who had
formerly been spiteful toward him and evidently unfriendly now
became gentle and affectionate. The angry eldest princess, with
the long waist and hair plastered down like a doll's, had come
into Pierre's room after the funeral. With drooping eyes and
frequent blushes she told him she was very sorry about their past
misunderstandings and did not now feel she had a right to ask him
for anything, except only for permission, after the blow she had
received, to remain for a few weeks longer in the house she so
loved and where she had sacrificed so much. She could not refrain
from weeping at these words. Touched that this statuesque
princess could so change, Pierre took her hand and begged her
forgiveness, without knowing what for. From that day the eldest
princess quite changed toward Pierre and began knitting a striped
scarf for him.

``Do this for my sake, mon cher; after all, she had to put up
with a great deal from the deceased,'' said Prince Vasili to him,
handing him a deed to sign for the princess' benefit.

Prince Vasili had come to the conclusion that it was necessary to
throw this bone---a bill for thirty thousand rubles---to the poor
princess that it might not occur to her to speak of his share in
the affair of the inlaid portfolio. Pierre signed the deed and
after that the princess grew still kinder. The younger sisters
also became affectionate to him, especially the youngest, the
pretty one with the mole, who often made him feel confused by her
smiles and her own confusion when meeting him.

It seemed so natural to Pierre that everyone should like him, and
it would have seemed so unnatural had anyone disliked him, that
he could not but believe in the sincerity of those around
him. Besides, he had no time to ask himself whether these people
were sincere or not. He was always busy and always felt in a
state of mild and cheerful intoxication. He felt as though he
were the center of some important and general movement; that
something was constantly expected of him, that if he did not do
it he would grieve and disappoint many people, but if he did this
and that, all would be well; and he did what was demanded of him,
but still that happy result always remained in the future.

More than anyone else, Prince Vasili took possession of Pierre's
affairs and of Pierre himself in those early days. From the death
of Count Bezukhov he did not let go his hold of the lad. He had
the air of a man oppressed by business, weary and suffering, who
yet would not, for pity's sake, leave this helpless youth who,
after all, was the son of his old friend and the possessor of
such enormous wealth, to the caprice of fate and the designs of
rogues. During the few days he spent in Moscow after the death of
Count Bezukhov, he would call Pierre, or go to him himself, and
tell him what ought to be done in a tone of weariness and
assurance, as if he were adding every time: ``You know I am
overwhelmed with business and it is purely out of charity that I
trouble myself about you, and you also know quite well that what
I propose is the only thing possible.''

``Well, my dear fellow, tomorrow we are off at last,'' said
Prince Vasili one day, closing his eyes and fingering Pierre's
elbow, speaking as if he were saying something which had long
since been agreed upon and could not now be altered. ``We start
tomorrow and I'm giving you a place in my carriage. I am very
glad. All our important business here is now settled, and I ought
to have been off long ago. Here is something I have received from
the chancellor. I asked him for you, and you have been entered in
the diplomatic corps and made a Gentleman of the Bedchamber.  The
diplomatic career now lies open before you.''

Notwithstanding the tone of wearied assurance with which these
words were pronounced, Pierre, who had so long been considering
his career, wished to make some suggestion. But Prince Vasili
interrupted him in the special deep cooing tone, precluding the
possibility of interrupting his speech, which he used in extreme
cases when special persuasion was needed.

``Mais, mon cher, I did this for my own sake, to satisfy my
conscience, and there is nothing to thank me for. No one has ever
complained yet of being too much loved; and besides, you are
free, you could throw it up tomorrow. But you will see everything
for yourself when you get to Petersburg. It is high time for you
to get away from these terrible recollections.'' Prince Vasili
sighed. ``Yes, yes, my boy. And my valet can go in your
carriage. Ah! I was nearly forgetting,'' he added. ``You know,
mon cher, your father and I had some accounts to settle, so I
have received what was due from the Ryazan estate and will keep
it; you won't require it. We'll go into the accounts later.''

By ``what was due from the Ryazan estate'' Prince Vasili meant
several thousand rubles quitrent received from Pierre's peasants,
which the prince had retained for himself.

In Petersburg, as in Moscow, Pierre found the same atmosphere of
gentleness and affection. He could not refuse the post, or rather
the rank (for he did nothing), that Prince Vasili had procured
for him, and acquaintances, invitations, and social occupations
were so numerous that, even more than in Moscow, he felt a sense
of bewilderment, bustle, and continual expectation of some good,
always in front of him but never attained.

Of his former bachelor acquaintances many were no longer in
Petersburg.  The Guards had gone to the front; Dolokhov had been
reduced to the ranks; Anatole was in the army somewhere in the
provinces; Prince Andrew was abroad; so Pierre had not the
opportunity to spend his nights as he used to like to spend them,
or to open his mind by intimate talks with a friend older than
himself and whom he respected. His whole time was taken up with
dinners and balls and was spent chiefly at Prince Vasili's house
in the company of the stout princess, his wife, and his beautiful
daughter Helene.

Like the others, Anna Pavlovna Scherer showed Pierre the change
of attitude toward him that had taken place in society.

Formerly in Anna Pavlovna's presence, Pierre had always felt that
what he was saying was out of place, tactless and unsuitable,
that remarks which seemed to him clever while they formed in his
mind became foolish as soon as he uttered them, while on the
contrary Hippolyte's stupidest remarks came out clever and
apt. Now everything Pierre said was charmant. Even if Anna
Pavlovna did not say so, he could see that she wished to and only
refrained out of regard for his modesty.

In the beginning of the winter of 1805-6 Pierre received one of
Anna Pavlovna's usual pink notes with an invitation to which was
added: ``You will find the beautiful Helene here, whom it is
always delightful to see.''

When he read that sentence, Pierre felt for the first time that
some link which other people recognized had grown up between
himself and Helene, and that thought both alarmed him, as if some
obligation were being imposed on him which he could not fulfill,
and pleased him as an entertaining supposition.

Anna Pavlovna's \emph{At Home} was like the former one, only the
novelty she offered her guests this time was not Mortemart, but a
diplomatist fresh from Berlin with the very latest details of the
Emperor Alexander's visit to Potsdam, and of how the two august
friends had pledged themselves in an indissoluble alliance to
uphold the cause of justice against the enemy of the human
race. Anna Pavlovna received Pierre with a shade of melancholy,
evidently relating to the young man's recent loss by the death of
Count Bezukhov (everyone constantly considered it a duty to
assure Pierre that he was greatly afflicted by the death of the
father he had hardly known), and her melancholy was just like the
august melancholy she showed at the mention of her most august
Majesty the Empress Marya Fedorovna. Pierre felt flattered by
this. Anna Pavlovna arranged the different groups in her drawing
room with her habitual skill. The large group, in which were
Prince Vasili and the generals, had the benefit of the
diplomat. Another group was at the tea table.  Pierre wished to
join the former, but Anna Pavlovna---who was in the excited
condition of a commander on a battlefield to whom thousands of
new and brilliant ideas occur which there is hardly time to put
in action---seeing Pierre, touched his sleeve with her finger,
saying:

``Wait a bit, I have something in view for you this evening.''
(She glanced at Helene and smiled at her.) ``My dear Helene, be
charitable to my poor aunt who adores you. Go and keep her
company for ten minutes.  And that it will not be too dull, here
is the dear count who will not refuse to accompany you.''

The beauty went to the aunt, but Anna Pavlovna detained Pierre,
looking as if she had to give some final necessary instructions.

``Isn't she exquisite?'' she said to Pierre, pointing to the
stately beauty as she glided away. ``And how she carries herself!
For so young a girl, such tact, such masterly perfection of
manner! It comes from her heart. Happy the man who wins her! With
her the least worldly of men would occupy a most brilliant
position in society. Don't you think so? I only wanted to know
your opinion,'' and Anna Pavlovna let Pierre go.

Pierre, in reply, sincerely agreed with her as to Helene's
perfection of manner. If he ever thought of Helene, it was just
of her beauty and her remarkable skill in appearing silently
dignified in society.

The old aunt received the two young people in her corner, but
seemed desirous of hiding her adoration for Helene and inclined
rather to show her fear of Anna Pavlovna. She looked at her
niece, as if inquiring what she was to do with these people. On
leaving them, Anna Pavlovna again touched Pierre's sleeve,
saying: ``I hope you won't say that it is dull in my house
again,'' and she glanced at Helene.

Helene smiled, with a look implying that she did not admit the
possibility of anyone seeing her without being enchanted. The
aunt coughed, swallowed, and said in French that she was very
pleased to see Helene, then she turned to Pierre with the same
words of welcome and the same look. In the middle of a dull and
halting conversation, Helene turned to Pierre with the beautiful
bright smile that she gave to everyone. Pierre was so used to
that smile, and it had so little meaning for him, that he paid no
attention to it. The aunt was just speaking of a collection of
snuffboxes that had belonged to Pierre's father, Count Bezukhov,
and showed them her own box. Princess Helene asked to see the
portrait of the aunt's husband on the box lid.

``That is probably the work of Vinesse,'' said Pierre, mentioning
a celebrated miniaturist, and he leaned over the table to take
the snuffbox while trying to hear what was being said at the
other table.

He half rose, meaning to go round, but the aunt handed him the
snuffbox, passing it across Helene's back. Helene stooped forward
to make room, and looked round with a smile. She was, as always
at evening parties, wearing a dress such as was then fashionable,
cut very low at front and back. Her bust, which had always seemed
like marble to Pierre, was so close to him that his shortsighted
eyes could not but perceive the living charm of her neck and
shoulders, so near to his lips that he need only have bent his
head a little to have touched them. He was conscious of the
warmth of her body, the scent of perfume, and the creaking of her
corset as she moved. He did not see her marble beauty forming a
complete whole with her dress, but all the charm of her body only
covered by her garments. And having once seen this he could not
help being aware of it, just as we cannot renew an illusion we
have once seen through.

``So you have never noticed before how beautiful I am?'' Helene
seemed to say. ``You had not noticed that I am a woman? Yes, I am
a woman who may belong to anyone---to you too,'' said her
glance. And at that moment Pierre felt that Helene not only
could, but must, be his wife, and that it could not be otherwise.

He knew this at that moment as surely as if he had been standing
at the altar with her. How and when this would be he did not
know, he did not even know if it would be a good thing (he even
felt, he knew not why, that it would be a bad thing), but he knew
it would happen.

Pierre dropped his eyes, lifted them again, and wished once more
to see her as a distant beauty far removed from him, as he had
seen her every day until then, but he could no longer do it. He
could not, any more than a man who has been looking at a tuft of
steppe grass through the mist and taking it for a tree can again
take it for a tree after he has once recognized it to be a tuft
of grass. She was terribly close to him.  She already had power
over him, and between them there was no longer any barrier except
the barrier of his own will.

``Well, I will leave you in your little corner,'' came Anna
Pavlovna's voice, ``I see you are all right there.''

And Pierre, anxiously trying to remember whether he had done
anything reprehensible, looked round with a blush. It seemed to
him that everyone knew what had happened to him as he knew it
himself.

A little later when he went up to the large circle, Anna Pavlovna
said to him: ``I hear you are refitting your Petersburg house?''

This was true. The architect had told him that it was necessary,
and Pierre, without knowing why, was having his enormous
Petersburg house done up.

``That's a good thing, but don't move from Prince Vasili's. It is
good to have a friend like the prince,'' she said, smiling at
Prince Vasili. ``I know something about that. Don't I? And you
are still so young. You need advice. Don't be angry with me for
exercising an old woman's privilege.''

She paused, as women always do, expecting something after they
have mentioned their age. ``If you marry it will be a different
thing,'' she continued, uniting them both in one glance. Pierre
did not look at Helene nor she at him. But she was just as
terribly close to him. He muttered something and colored.

When he got home he could not sleep for a long time for thinking
of what had happened. What had happened? Nothing. He had merely
understood that the woman he had known as a child, of whom when
her beauty was mentioned he had said absent-mindedly: ``Yes,
she's good looking,'' he had understood that this woman might
belong to him.

``But she's stupid. I have myself said she is stupid,'' he
thought. ``There is something nasty, something wrong, in the
feeling she excites in me. I have been told that her brother
Anatole was in love with her and she with him, that there was
quite a scandal and that that's why he was sent away. Hippolyte
is her brother... Prince Vasili is her father... It's bad...'' he
reflected, but while he was thinking this (the reflection was
still incomplete), he caught himself smiling and was conscious
that another line of thought had sprung up, and while thinking of
her worthlessness he was also dreaming of how she would be his
wife, how she would love him become quite different, and how all
he had thought and heard of her might be false. And he again saw
her not as the daughter of Prince Vasili, but visualized her
whole body only veiled by its gray dress. ``But no! Why did this
thought never occur to me before?'' and again he told himself
that it was impossible, that there would be something unnatural,
and as it seemed to him dishonorable, in this marriage. He
recalled her former words and looks and the words and looks of
those who had seen them together. He recalled Anna Pavlovna's
words and looks when she spoke to him about his house, recalled
thousands of such hints from Prince Vasili and others, and was
seized by terror lest he had already, in some way, bound himself
to do something that was evidently wrong and that he ought not to
do. But at the very time he was expressing this conviction to
himself, in another part of his mind her image rose in all its
womanly beauty.

% % % % % % % % % % % % % % % % % % % % % % % % % % % % % % % % %
% % % % % % % % % % % % % % % % % % % % % % % % % % % % % % % % %
% % % % % % % % % % % % % % % % % % % % % % % % % % % % % % % % %
% % % % % % % % % % % % % % % % % % % % % % % % % % % % % % % % %
% % % % % % % % % % % % % % % % % % % % % % % % % % % % % % % % %
% % % % % % % % % % % % % % % % % % % % % % % % % % % % % % % % %
% % % % % % % % % % % % % % % % % % % % % % % % % % % % % % % % %
% % % % % % % % % % % % % % % % % % % % % % % % % % % % % % % % %
% % % % % % % % % % % % % % % % % % % % % % % % % % % % % % % % %
% % % % % % % % % % % % % % % % % % % % % % % % % % % % % % % % %
% % % % % % % % % % % % % % % % % % % % % % % % % % % % % % % % %
% % % % % % % % % % % % % % % % % % % % % % % % % % % % % %

\chapter*{Chapter II}
\ifaudio     
\marginpar{
\href{http://ia800208.us.archive.org/14/items/war_and_peace_03_0712_librivox/war_and_peace_03_02_tolstoy_64kb.mp3}{Audio}} 
\fi

\lettrine[lines=2, loversize=0.3, lraise=0]{\initfamily I}{n}
November, 1805, Prince Vasili had to go on a tour of
inspection in four different provinces. He had arranged this for
himself so as to visit his neglected estates at the same time and
pick up his son Anatole where his regiment was stationed, and
take him to visit Prince Nicholas Bolkonski in order to arrange a
match for him with the daughter of that rich old man. But before
leaving home and undertaking these new affairs, Prince Vasili had
to settle matters with Pierre, who, it is true, had latterly
spent whole days at home, that is, in Prince Vasili's house where
he was staying, and had been absurd, excited, and foolish in
Helene's presence (as a lover should be), but had not yet
proposed to her.

``This is all very fine, but things must be settled,'' said
Prince Vasili to himself, with a sorrowful sigh, one morning,
feeling that Pierre who was under such obligations to him (``But
never mind that'') was not behaving very well in this
matter. ``Youth, frivolity... well, God be with him,'' thought
he, relishing his own goodness of heart, ``but it must be brought
to a head. The day after tomorrow will be Lelya's name day. I
will invite two or three people, and if he does not understand
what he ought to do then it will be my affair---yes, my affair. I
am her father.''

Six weeks after Anna Pavlovna's \emph{At Home} and after the
sleepless night when he had decided that to marry Helene would be
a calamity and that he ought to avoid her and go away, Pierre,
despite that decision, had not left Prince Vasili's and felt with
terror that in people's eyes he was every day more and more
connected with her, that it was impossible for him to return to
his former conception of her, that he could not break away from
her, and that though it would be a terrible thing he would have
to unite his fate with hers. He might perhaps have been able to
free himself but that Prince Vasili (who had rarely before given
receptions) now hardly let a day go by without having an evening
party at which Pierre had to be present unless he wished to spoil
the general pleasure and disappoint everyone's
expectation. Prince Vasili, in the rare moments when he was at
home, would take Pierre's hand in passing and draw it downwards,
or absent-mindedly hold out his wrinkled, clean-shaven cheek for
Pierre to kiss and would say: ``Till tomorrow,'' or, ``Be in to
dinner or I shall not see you,'' or, ``I am staying in for your
sake,'' and so on. And though Prince Vasili, when he stayed in
(as he said) for Pierre's sake, hardly exchanged a couple of
words with him, Pierre felt unable to disappoint him. Every day
he said to himself one and the same thing: ``It is time I
understood her and made up my mind what she really is. Was I
mistaken before, or am I mistaken now? No, she is not stupid, she
is an excellent girl,'' he sometimes said to himself ``she never
makes a mistake, never says anything stupid. She says little, but
what she does say is always clear and simple, so she is not
stupid.  She never was abashed and is not abashed now, so she
cannot be a bad woman!'' He had often begun to make reflections
or think aloud in her company, and she had always answered him
either by a brief but appropriate remark---showing that it did
not interest her---or by a silent look and smile which more
palpably than anything else showed Pierre her superiority. She
was right in regarding all arguments as nonsense in comparison
with that smile.

She always addressed him with a radiantly confiding smile meant
for him alone, in which there was something more significant than
in the general smile that usually brightened her face. Pierre
knew that everyone was waiting for him to say a word and cross a
certain line, and he knew that sooner or later he would step
across it, but an incomprehensible terror seized him at the
thought of that dreadful step. A thousand times during that month
and a half while he felt himself drawn nearer and nearer to that
dreadful abyss, Pierre said to himself: ``What am I doing? I need
resolution. Can it be that I have none?''

He wished to take a decision, but felt with dismay that in this
matter he lacked that strength of will which he had known in
himself and really possessed. Pierre was one of those who are
only strong when they feel themselves quite innocent, and since
that day when he was overpowered by a feeling of desire while
stooping over the snuffbox at Anna Pavlovna's, an unacknowledged
sense of the guilt of that desire paralyzed his will.

On Helene's name day, a small party of just their own people---as
his wife said---met for supper at Prince Vasili's. All these
friends and relations had been given to understand that the fate
of the young girl would be decided that evening. The visitors
were seated at supper.  Princess Kuragina, a portly imposing
woman who had once been handsome, was sitting at the head of the
table. On either side of her sat the more important guests---an
old general and his wife, and Anna Pavlovna Scherer. At the other
end sat the younger and less important guests, and there too sat
the members of the family, and Pierre and Helene, side by
side. Prince Vasili was not having any supper: he went round the
table in a merry mood, sitting down now by one, now by another,
of the guests.  To each of them he made some careless and
agreeable remark except to Pierre and Helene, whose presence he
seemed not to notice. He enlivened the whole party. The wax
candles burned brightly, the silver and crystal gleamed, so did
the ladies' toilets and the gold and silver of the men's
epaulets; servants in scarlet liveries moved round the table, the
clatter of plates, knives, and glasses mingled with the animated
hum of several conversations. At one end of the table, the old
chamberlain was heard assuring an old baroness that he loved her
passionately, at which she laughed; at the other could be heard
the story of the misfortunes of some Mary Viktorovna or other. At
the center of the table, Prince Vasili attracted everybody's
attention. With a facetious smile on his face, he was telling the
ladies about last Wednesday's meeting of the Imperial Council, at
which Sergey Kuzmich Vyazmitinov, the new military governor
general of Petersburg, had received and read the then famous
rescript of the Emperor Alexander from the army to Sergey
Kuzmich, in which the Emperor said that he was receiving from all
sides declarations of the people's loyalty, that the declaration
from Petersburg gave him particular pleasure, and that he was
proud to be at the head of such a nation and would endeavor to be
worthy of it. This rescript began with the words: ``Sergey
Kuzmich, From all sides reports reach me,'' etc.

``Well, and so he never got farther than: 'Sergey Kuzmich'?''
asked one of the ladies.

``Exactly, not a hair's breadth farther,'' answered Prince
Vasili, laughing, ``'Sergey Kuzmich... From all sides... From all
sides... Sergey Kuzmich...' Poor Vyazmitinov could not get any
farther! He began the rescript again and again, but as soon as he
uttered 'Sergey' he sobbed, 'Kuz-mi-ch,' tears, and 'From all
sides' was smothered in sobs and he could get no farther. And
again his handkerchief, and again: 'Sergey Kuzmich, From all
sides,'... and tears, till at last somebody else was asked to
read it.''

``Kuzmich... From all sides... and then tears,'' someone repeated
laughing.

``Don't be unkind,'' cried Anna Pavlovna from her end of the
table holding up a threatening finger. ``He is such a worthy and
excellent man, our dear Vyazmitinov...''

Everybody laughed a great deal. At the head of the table, where
the honored guests sat, everyone seemed to be in high spirits and
under the influence of a variety of exciting sensations. Only
Pierre and Helene sat silently side by side almost at the bottom
of the table, a suppressed smile brightening both their faces, a
smile that had nothing to do with Sergey Kuzmich---a smile of
bashfulness at their own feelings.  But much as all the rest
laughed, talked, and joked, much as they enjoyed their Rhine
wine, saute, and ices, and however they avoided looking at the
young couple, and heedless and unobservant as they seemed of
them, one could feel by the occasional glances they gave that the
story about Sergey Kuzmich, the laughter, and the food were all a
pretense, and that the whole attention of that company was
directed to---Pierre and Helene. Prince Vasili mimicked the
sobbing of Sergey Kuzmich and at the same time his eyes glanced
toward his daughter, and while he laughed the expression on his
face clearly said: ``Yes... it's getting on, it will all be
settled today.'' Anna Pavlovna threatened him on behalf of ``our
dear Vyazmitinov,'' and in her eyes, which, for an instant,
glanced at Pierre, Prince Vasili read a congratulation on his
future son-in-law and on his daughter's happiness. The old
princess sighed sadly as she offered some wine to the old lady
next to her and glanced angrily at her daughter, and her sigh
seemed to say: ``Yes, there's nothing left for you and me but to
sip sweet wine, my dear, now that the time has come for these
young ones to be thus boldly, provocatively happy.'' ``And what
nonsense all this is that I am saying!''  thought a diplomatist,
glancing at the happy faces of the lovers.  ``That's happiness!''

Into the insignificant, trifling, and artificial interests
uniting that society had entered the simple feeling of the
attraction of a healthy and handsome young man and woman for one
another. And this human feeling dominated everything else and
soared above all their affected chatter.  Jests fell flat, news
was not interesting, and the animation was evidently forced. Not
only the guests but even the footmen waiting at table seemed to
feel this, and they forgot their duties as they looked at the
beautiful Helene with her radiant face and at the red, broad, and
happy though uneasy face of Pierre. It seemed as if the very
light of the candles was focused on those two happy faces alone.

Pierre felt that he was the center of it all, and this both
pleased and embarrassed him. He was like a man entirely absorbed
in some occupation.  He did not see, hear, or understand anything
clearly. Only now and then detached ideas and impressions from
the world of reality shot unexpectedly through his mind.

``So it is all finished!'' he thought. ``And how has it all
happened? How quickly! Now I know that not because of her alone,
nor of myself alone, but because of everyone, it must inevitably
come about. They are all expecting it, they are so sure that it
will happen that I cannot, I cannot, disappoint them. But how
will it be? I do not know, but it will certainly happen!''
thought Pierre, glancing at those dazzling shoulders close to his
eyes.

Or he would suddenly feel ashamed of he knew not what. He felt it
awkward to attract everyone's attention and to be considered a
lucky man and, with his plain face, to be looked on as a sort of
Paris possessed of a Helen. ``But no doubt it always is and must
be so!'' he consoled himself. ``And besides, what have I done to
bring it about? How did it begin? I traveled from Moscow with
Prince Vasili. Then there was nothing. So why should I not stay
at his house? Then I played cards with her and picked up her
reticule and drove out with her. How did it begin, when did it
all come about?'' And here he was sitting by her side as her
betrothed, seeing, hearing, feeling her nearness, her breathing,
her movements, her beauty. Then it would suddenly seem to him
that it was not she but he was so unusually beautiful, and that
that was why they all looked so at him, and flattered by this
general admiration he would expand his chest, raise his head, and
rejoice at his good fortune.  Suddenly he heard a familiar voice
repeating something to him a second time. But Pierre was so
absorbed that he did not understand what was said.

``I am asking you when you last heard from Bolkonski,'' repeated
Prince Vasili a third time. ``How absent-minded you are, my dear
fellow.''

Prince Vasili smiled, and Pierre noticed that everyone was
smiling at him and Helene. ``Well, what of it, if you all know
it?'' thought Pierre.  ``What of it? It's the truth!'' and he
himself smiled his gentle childlike smile, and Helene smiled too.

``When did you get the letter? Was it from Olmutz?'' repeated
Prince Vasili, who pretended to want to know this in order to
settle a dispute.

``How can one talk or think of such trifles?'' thought Pierre.

``Yes, from Olmutz,'' he answered, with a sigh.

After supper Pierre with his partner followed the others into the
drawing room. The guests began to disperse, some without taking
leave of Helene. Some, as if unwilling to distract her from an
important occupation, came up to her for a moment and made haste
to go away, refusing to let her see them off. The diplomatist
preserved a mournful silence as he left the drawing room. He
pictured the vanity of his diplomatic career in comparison with
Pierre's happiness. The old general grumbled at his wife when she
asked how his leg was. ``Oh, the old fool,'' he thought. ``That
Princess Helene will be beautiful still when she's fifty.''

``I think I may congratulate you,'' whispered Anna Pavlovna to
the old princess, kissing her soundly. ``If I hadn't this
headache I'd have stayed longer.''

The old princess did not reply, she was tormented by jealousy of
her daughter's happiness.

While the guests were taking their leave Pierre remained for a
long time alone with Helene in the little drawing room where they
were sitting. He had often before, during the last six weeks,
remained alone with her, but had never spoken to her of love. Now
he felt that it was inevitable, but he could not make up his mind
to take the final step. He felt ashamed; he felt that he was
occupying someone else's place here beside Helene. ``This
happiness is not for you,'' some inner voice whispered to
him. ``This happiness is for those who have not in them what
there is in you.''

But, as he had to say something, he began by asking her whether
she was satisfied with the party. She replied in her usual simple
manner that this name day of hers had been one of the pleasantest
she had ever had.

Some of the nearest relatives had not yet left. They were sitting
in the large drawing room. Prince Vasili came up to Pierre with
languid footsteps. Pierre rose and said it was getting
late. Prince Vasili gave him a look of stern inquiry, as though
what Pierre had just said was so strange that one could not take
it in. But then the expression of severity changed, and he drew
Pierre's hand downwards, made him sit down, and smiled
affectionately.

``Well, Lelya?'' he asked, turning instantly to his daughter and
addressing her with the careless tone of habitual tenderness
natural to parents who have petted their children from babyhood,
but which Prince Vasili had only acquired by imitating other
parents.

And he again turned to Pierre.

``Sergey Kuzmich---From all sides-'' he said, unbuttoning the top
button of his waistcoat.

Pierre smiled, but his smile showed that he knew it was not the
story about Sergey Kuzmich that interested Prince Vasili just
then, and Prince Vasili saw that Pierre knew this. He suddenly
muttered something and went away. It seemed to Pierre that even
the prince was disconcerted.  The sight of the discomposure of
that old man of the world touched Pierre: he looked at Helene and
she too seemed disconcerted, and her look seemed to say: ``Well,
it is your own fault.''

``The step must be taken but I cannot, I cannot!'' thought
Pierre, and he again began speaking about indifferent matters,
about Sergey Kuzmich, asking what the point of the story was as
he had not heard it properly.  Helene answered with a smile that
she too had missed it.

When Prince Vasili returned to the drawing room, the princess,
his wife, was talking in low tones to the elderly lady about
Pierre.

``Of course, it is a very brilliant match, but happiness, my
dear...''

``Marriages are made in heaven,'' replied the elderly lady.

Prince Vasili passed by, seeming not to hear the ladies, and sat
down on a sofa in a far corner of the room. He closed his eyes
and seemed to be dozing. His head sank forward and then he roused
himself.

``Aline,'' he said to his wife, ``go and see what they are
about.''

The princess went up to the door, passed by it with a dignified
and indifferent air, and glanced into the little drawing
room. Pierre and Helene still sat talking just as before.

``Still the same,'' she said to her husband.

Prince Vasili frowned, twisting his mouth, his cheeks quivered
and his face assumed the coarse, unpleasant expression peculiar
to him. Shaking himself, he rose, threw back his head, and with
resolute steps went past the ladies into the little drawing
room. With quick steps he went joyfully up to Pierre. His face
was so unusually triumphant that Pierre rose in alarm on seeing
it.

``Thank God!'' said Prince Vasili. ``My wife has told me
everything!'' (He put one arm around Pierre and the other around
his daughter)---``My dear boy... Lelya... I am very pleased.''
(His voice trembled) ``I loved your father... and she will make
you a good wife... God bless you!...''

He embraced his daughter, and then again Pierre, and kissed him
with his malodorous mouth. Tears actually moistened his cheeks.

``Princess, come here!'' he shouted.

The old princess came in and also wept. The elderly lady was
using her handkerchief too. Pierre was kissed, and he kissed the
beautiful Helene's hand several times. After a while they were
left alone again.

``All this had to be and could not be otherwise,'' thought
Pierre, ``so it is useless to ask whether it is good or bad. It
is good because it's definite and one is rid of the old
tormenting doubt.'' Pierre held the hand of his betrothed in
silence, looking at her beautiful bosom as it rose and fell.

``Helene!'' he said aloud and paused.

``Something special is always said in such cases,'' he thought,
but could not remember what it was that people say. He looked at
her face. She drew nearer to him. Her face flushed.

``Oh, take those off... those...'' she said, pointing to his
spectacles.

Pierre took them off, and his eyes, besides the strange look eyes
have from which spectacles have just been removed, had also a
frightened and inquiring look. He was about to stoop over her
hand and kiss it, but with a rapid, almost brutal movement of her
head, she intercepted his lips and met them with her own. Her
face struck Pierre, by its altered, unpleasantly excited
expression.

``It is too late now, it's done; besides I love her,'' thought
Pierre.

``Je vous aime!''\footnote{``I love you.''} he said, remembering
what has to be said at such moments: but his words sounded so
weak that he felt ashamed of himself.

Six weeks later he was married, and settled in Count Bezukhov's
large, newly furnished Petersburg house, the happy possessor, as
people said, of a wife who was a celebrated beauty and of
millions of money.

% % % % % % % % % % % % % % % % % % % % % % % % % % % % % % % % %
% % % % % % % % % % % % % % % % % % % % % % % % % % % % % % % % %
% % % % % % % % % % % % % % % % % % % % % % % % % % % % % % % % %
% % % % % % % % % % % % % % % % % % % % % % % % % % % % % % % % %
% % % % % % % % % % % % % % % % % % % % % % % % % % % % % % % % %
% % % % % % % % % % % % % % % % % % % % % % % % % % % % % % % % %
% % % % % % % % % % % % % % % % % % % % % % % % % % % % % % % % %
% % % % % % % % % % % % % % % % % % % % % % % % % % % % % % % % %
% % % % % % % % % % % % % % % % % % % % % % % % % % % % % % % % %
% % % % % % % % % % % % % % % % % % % % % % % % % % % % % % % % %
% % % % % % % % % % % % % % % % % % % % % % % % % % % % % % % % %
% % % % % % % % % % % % % % % % % % % % % % % % % % % % % %

\chapter*{Chapter III}
\ifaudio     
\marginpar{
\href{http://ia800208.us.archive.org/14/items/war_and_peace_03_0712_librivox/war_and_peace_03_03_tolstoy_64kb.mp3}{Audio}} 
\fi

\lettrine[lines=2, loversize=0.3, lraise=0]{\initfamily O}{ld}
 Prince Nicholas Bolkonski received a letter from Prince
Vasili in November, 1805, announcing that he and his son would be
paying him a visit. ``I am starting on a journey of inspection,
and of course I shall think nothing of an extra seventy miles to
come and see you at the same time, my honored benefactor,'' wrote
Prince Vasili. ``My son Anatole is accompanying me on his way to
the army, so I hope you will allow him personally to express the
deep respect that, emulating his father, he feels for you.''

``It seems that there will be no need to bring Mary out, suitors
are coming to us of their own accord,'' incautiously remarked the
little princess on hearing the news.

Prince Nicholas frowned, but said nothing.

A fortnight after the letter Prince Vasili's servants came one
evening in advance of him, and he and his son arrived next day.

Old Bolkonski had always had a poor opinion of Prince Vasili's
character, but more so recently, since in the new reigns of Paul
and Alexander Prince Vasili had risen to high position and
honors. And now, from the hints contained in his letter and given
by the little princess, he saw which way the wind was blowing,
and his low opinion changed into a feeling of contemptuous ill
will. He snorted whenever he mentioned him. On the day of Prince
Vasili's arrival, Prince Bolkonski was particularly discontented
and out of temper. Whether he was in a bad temper because Prince
Vasili was coming, or whether his being in a bad temper made him
specially annoyed at Prince Vasili's visit, he was in a bad
temper, and in the morning Tikhon had already advised the
architect not to go to the prince with his report.

``Do you hear how he's walking?'' said Tikhon, drawing the
architect's attention to the sound of the prince's
footsteps. ``Stepping flat on his heels---we know what that
means...''

However, at nine o'clock the prince, in his velvet coat with a
sable collar and cap, went out for his usual walk. It had snowed
the day before and the path to the hothouse, along which the
prince was in the habit of walking, had been swept: the marks of
the broom were still visible in the snow and a shovel had been
left sticking in one of the soft snowbanks that bordered both
sides of the path. The prince went through the conservatories,
the serfs' quarters, and the outbuildings, frowning and silent.

``Can a sleigh pass?'' he asked his overseer, a venerable man,
resembling his master in manners and looks, who was accompanying
him back to the house.

``The snow is deep. I am having the avenue swept, your honor.''

The prince bowed his head and went up to the porch. ``God be
thanked,'' thought the overseer, ``the storm has blown over!''

``It would have been hard to drive up, your honor,'' he
added. ``I heard, your honor, that a minister is coming to visit
your honor.''

The prince turned round to the overseer and fixed his eyes on
him, frowning.

``What? A minister? What minister? Who gave orders?'' he said in
his shrill, harsh voice. ``The road is not swept for the princess
my daughter, but for a minister! For me, there are no
ministers!''

``Your honor, I thought...''

``You thought!'' shouted the prince, his words coming more and
more rapidly and indistinctly. ``You thought!... Rascals!
Blackguards!... I'll teach you to think!'' and lifting his stick
he swung it and would have hit Alpatych, the overseer, had not
the latter instinctively avoided the
blow. ``Thought... Blackguards...'' shouted the prince rapidly.

But although Alpatych, frightened at his own temerity in avoiding
the stroke, came up to the prince, bowing his bald head
resignedly before him, or perhaps for that very reason, the
prince, though he continued to shout: ``Blackguards!... Throw the
snow back on the road!'' did not lift his stick again but hurried
into the house.

Before dinner, Princess Mary and Mademoiselle Bourienne, who knew
that the prince was in a bad humor, stood awaiting him;
Mademoiselle Bourienne with a radiant face that said: ``I know
nothing, I am the same as usual,'' and Princess Mary pale,
frightened, and with downcast eyes.  What she found hardest to
bear was to know that on such occasions she ought to behave like
Mademoiselle Bourienne, but could not. She thought: ``If I seem
not to notice he will think that I do not sympathize with him; if
I seem sad and out of spirits myself, he will say (as he has done
before) that I'm in the dumps.''

The prince looked at his daughter's frightened face and snorted.

``Fool... or dummy!'' he muttered.

``And the other one is not here. They've been telling tales,'' he
thought---referring to the little princess who was not in the
dining room.

``Where is the princess?'' he asked. ``Hiding?''

``She is not very well,'' answered Mademoiselle Bourienne with a
bright smile, ``so she won't come down. It is natural in her
state.''

``Hm! Hm!'' muttered the prince, sitting down.

His plate seemed to him not quite clean, and pointing to a spot
he flung it away. Tikhon caught it and handed it to a
footman. The little princess was not unwell, but had such an
overpowering fear of the prince that, hearing he was in a bad
humor, she had decided not to appear.

``I am afraid for the baby,'' she said to Mademoiselle Bourienne:
``Heaven knows what a fright might do.''

In general at Bald Hills the little princess lived in constant
fear, and with a sense of antipathy to the old prince which she
did not realize because the fear was so much the stronger
feeling. The prince reciprocated this antipathy, but it was
overpowered by his contempt for her. When the little princess had
grown accustomed to life at Bald Hills, she took a special fancy
to Mademoiselle Bourienne, spent whole days with her, asked her
to sleep in her room, and often talked with her about the old
prince and criticized him.

``So we are to have visitors, mon prince?'' remarked Mademoiselle
Bourienne, unfolding her white napkin with her rosy
fingers. ``His Excellency Prince Vasili Kuragin and his son, I
understand?'' she said inquiringly.

``Hm!---his excellency is a puppy... I got him his appointment in
the service,'' said the prince disdainfully. ``Why his son is
coming I don't understand. Perhaps Princess Elizabeth and
Princess Mary know. I don't want him.'' (He looked at his
blushing daughter.) ``Are you unwell today?  Eh? Afraid of the
'minister' as that idiot Alpatych called him this morning?''

``No, mon pere.''

Though Mademoiselle Bourienne had been so unsuccessful in her
choice of a subject, she did not stop talking, but chattered
about the conservatories and the beauty of a flower that had just
opened, and after the soup the prince became more genial.

After dinner, he went to see his daughter-in-law. The little
princess was sitting at a small table, chattering with Masha, her
maid. She grew pale on seeing her father-in-law.

She was much altered. She was now plain rather than pretty. Her
cheeks had sunk, her lip was drawn up, and her eyes drawn down.

``Yes, I feel a kind of oppression,'' she said in reply to the
prince's question as to how she felt.

``Do you want anything?''

``No, merci, mon pere.''

``Well, all right, all right.''

He left the room and went to the waiting room where Alpatych
stood with bowed head.

``Has the snow been shoveled back?''

``Yes, your excellency. Forgive me for heaven's sake... It was
only my stupidity.''

``All right, all right,'' interrupted the prince, and laughing
his unnatural way, he stretched out his hand for Alpatych to
kiss, and then proceeded to his study.

Prince Vasili arrived that evening. He was met in the avenue by
coachmen and footmen, who, with loud shouts, dragged his sleighs
up to one of the lodges over the road purposely laden with snow.

Prince Vasili and Anatole had separate rooms assigned to them.

Anatole, having taken off his overcoat, sat with arms akimbo
before a table on a corner of which he smilingly and
absent-mindedly fixed his large and handsome eyes. He regarded
his whole life as a continual round of amusement which someone
for some reason had to provide for him. And he looked on this
visit to a churlish old man and a rich and ugly heiress in the
same way. All this might, he thought, turn out very well and
amusingly. ``And why not marry her if she really has so much
money?  That never does any harm,'' thought Anatole.

He shaved and scented himself with the care and elegance which
had become habitual to him and, his handsome head held high,
entered his father's room with the good-humored and victorious
air natural to him.  Prince Vasili's two valets were busy
dressing him, and he looked round with much animation and
cheerfully nodded to his son as the latter entered, as if to say:
``Yes, that's how I want you to look.''

``I say, Father, joking apart, is she very hideous?'' Anatole
asked, as if continuing a conversation the subject of which had
often been mentioned during the journey.

``Enough! What nonsense! Above all, try to be respectful and
cautious with the old prince.''

``If he starts a row I'll go away,'' said Prince Anatole. ``I
can't bear those old men! Eh?''

``Remember, for you everything depends on this.''

In the meantime, not only was it known in the maidservants' rooms
that the minister and his son had arrived, but the appearance of
both had been minutely described. Princess Mary was sitting alone
in her room, vainly trying to master her agitation.

``Why did they write, why did Lise tell me about it? It can never
happen!'' she said, looking at herself in the glass. ``How shall
I enter the drawing room? Even if I like him I can't now be
myself with him.''  The mere thought of her father's look filled
her with terror. The little princess and Mademoiselle Bourienne
had already received from Masha, the lady's maid, the necessary
report of how handsome the minister's son was, with his rosy
cheeks and dark eyebrows, and with what difficulty the father had
dragged his legs upstairs while the son had followed him like an
eagle, three steps at a time. Having received this information,
the little princess and Mademoiselle Bourienne, whose chattering
voices had reached her from the corridor, went into Princess
Mary's room.

``You know they've come, Marie?'' said the little princess,
waddling in, and sinking heavily into an armchair.

She was no longer in the loose gown she generally wore in the
morning, but had on one of her best dresses. Her hair was
carefully done and her face was animated, which, however, did not
conceal its sunken and faded outlines. Dressed as she used to be
in Petersburg society, it was still more noticeable how much
plainer she had become. Some unobtrusive touch had been added to
Mademoiselle Bourienne's toilet which rendered her fresh and
pretty face yet more attractive.

``What! Are you going to remain as you are, dear princess?'' she
began.  ``They'll be announcing that the gentlemen are in the
drawing room and we shall have to go down, and you have not
smartened yourself up at all!''

The little princess got up, rang for the maid, and hurriedly and
merrily began to devise and carry out a plan of how Princess Mary
should be dressed. Princess Mary's self-esteem was wounded by the
fact that the arrival of a suitor agitated her, and still more so
by both her companions' not having the least conception that it
could be otherwise.  To tell them that she felt ashamed for
herself and for them would be to betray her agitation, while to
decline their offers to dress her would prolong their banter and
insistence. She flushed, her beautiful eyes grew dim, red
blotches came on her face, and it took on the unattractive
martyrlike expression it so often wore, as she submitted herself
to Mademoiselle Bourienne and Lise. Both these women quite
sincerely tried to make her look pretty. She was so plain that
neither of them could think of her as a rival, so they began
dressing her with perfect sincerity, and with the naive and firm
conviction women have that dress can make a face pretty.

``No really, my dear, this dress is not pretty,'' said Lise,
looking sideways at Princess Mary from a little distance. ``You
have a maroon dress, have it fetched. Really! You know the fate
of your whole life may be at stake. But this one is too light,
it's not becoming!''

It was not the dress, but the face and whole figure of Princess
Mary that was not pretty, but neither Mademoiselle Bourienne nor
the little princess felt this; they still thought that if a blue
ribbon were placed in the hair, the hair combed up, and the blue
scarf arranged lower on the best maroon dress, and so on, all
would be well. They forgot that the frightened face and the
figure could not be altered, and that however they might change
the setting and adornment of that face, it would still remain
piteous and plain. After two or three changes to which Princess
Mary meekly submitted, just as her hair had been arranged on the
top of her head (a style that quite altered and spoiled her
looks) and she had put on a maroon dress with a pale-blue scarf,
the little princess walked twice round her, now adjusting a fold
of the dress with her little hand, now arranging the scarf and
looking at her with her head bent first on one side and then on
the other.

``No, it will not do,'' she said decidedly, clasping her
hands. ``No, Mary, really this dress does not suit you. I prefer
you in your little gray everyday dress. Now please, do it for my
sake. Katie,'' she said to the maid, ``bring the princess her
gray dress, and you'll see, Mademoiselle Bourienne, how I shall
arrange it,'' she added, smiling with a foretaste of artistic
pleasure.

But when Katie brought the required dress, Princess Mary remained
sitting motionless before the glass, looking at her face, and saw
in the mirror her eyes full of tears and her mouth quivering,
ready to burst into sobs.

``Come, dear princess,'' said Mademoiselle Bourienne, ``just one
more little effort.''

The little princess, taking the dress from the maid, came up to
Princess Mary.

``Well, now we'll arrange something quite simple and becoming,''
she said.

The three voices, hers, Mademoiselle Bourienne's, and Katie's,
who was laughing at something, mingled in a merry sound, like the
chirping of birds.

``No, leave me alone,'' said Princess Mary.

Her voice sounded so serious and so sad that the chirping of the
birds was silenced at once. They looked at the beautiful, large,
thoughtful eyes full of tears and of thoughts, gazing shiningly
and imploringly at them, and understood that it was useless and
even cruel to insist.

``At least, change your coiffure,'' said the little
princess. ``Didn't I tell you,'' she went on, turning
reproachfully to Mademoiselle Bourienne, ``Mary's is a face which
such a coiffure does not suit in the least. Not in the least!
Please change it.''

``Leave me alone, please leave me alone! It is all quite the same
to me,'' answered a voice struggling with tears.

Mademoiselle Bourienne and the little princess had to own to
themselves that Princess Mary in this guise looked very plain,
worse than usual, but it was too late. She was looking at them
with an expression they both knew, an expression thoughtful and
sad. This expression in Princess Mary did not frighten them (she
never inspired fear in anyone), but they knew that when it
appeared on her face, she became mute and was not to be shaken in
her determination.

``You will change it, won't you?'' said Lise. And as Princess
Mary gave no answer, she left the room.

Princess Mary was left alone. She did not comply with Lise's
request, she not only left her hair as it was, but did not even
look in her glass. Letting her arms fall helplessly, she sat with
downcast eyes and pondered. A husband, a man, a strong dominant
and strangely attractive being rose in her imagination, and
carried her into a totally different happy world of his own. She
fancied a child, her own---such as she had seen the day before in
the arms of her nurse's daughter---at her own breast, the husband
standing by and gazing tenderly at her and the child. ``But no,
it is impossible, I am too ugly,'' she thought.

``Please come to tea. The prince will be out in a moment,'' came
the maid's voice at the door.

She roused herself, and felt appalled at what she had been
thinking, and before going down she went into the room where the
icons hung and, her eyes fixed on the dark face of a large icon
of the Saviour lit by a lamp, she stood before it with folded
hands for a few moments. A painful doubt filled her soul. Could
the joy of love, of earthly love for a man, be for her? In her
thoughts of marriage Princess Mary dreamed of happiness and of
children, but her strongest, most deeply hidden longing was for
earthly love. The more she tried to hide this feeling from others
and even from herself, the stronger it grew. ``O God,'' she said,
``how am I to stifle in my heart these temptations of the devil?
How am I to renounce forever these vile fancies, so as peacefully
to fulfill Thy will?'' And scarcely had she put that question
than God gave her the answer in her own heart. ``Desire nothing
for thyself, seek nothing, be not anxious or envious. Man's
future and thy own fate must remain hidden from thee, but live so
that thou mayest be ready for anything. If it be God's will to
prove thee in the duties of marriage, be ready to fulfill His
will.'' With this consoling thought (but yet with a hope for the
fulfillment of her forbidden earthly longing) Princess Mary
sighed, and having crossed herself went down, thinking neither of
her gown and coiffure nor of how she would go in nor of what she
would say. What could all that matter in comparison with the will
of God, without Whose care not a hair of man's head can fall?

% % % % % % % % % % % % % % % % % % % % % % % % % % % % % % % % %
% % % % % % % % % % % % % % % % % % % % % % % % % % % % % % % % %
% % % % % % % % % % % % % % % % % % % % % % % % % % % % % % % % %
% % % % % % % % % % % % % % % % % % % % % % % % % % % % % % % % %
% % % % % % % % % % % % % % % % % % % % % % % % % % % % % % % % %
% % % % % % % % % % % % % % % % % % % % % % % % % % % % % % % % %
% % % % % % % % % % % % % % % % % % % % % % % % % % % % % % % % %
% % % % % % % % % % % % % % % % % % % % % % % % % % % % % % % % %
% % % % % % % % % % % % % % % % % % % % % % % % % % % % % % % % %
% % % % % % % % % % % % % % % % % % % % % % % % % % % % % % % % %
% % % % % % % % % % % % % % % % % % % % % % % % % % % % % % % % %
% % % % % % % % % % % % % % % % % % % % % % % % % % % % % %

\chapter*{Chapter IV}
\ifaudio     
\marginpar{
\href{http://ia800208.us.archive.org/14/items/war_and_peace_03_0712_librivox/war_and_peace_03_04_tolstoy_64kb.mp3}{Audio}} 
\fi

\lettrine[lines=2, loversize=0.3, lraise=0]{\initfamily W}{hen}
 Princess Mary came down, Prince Vasili and his son were
already in the drawing room, talking to the little princess and
Mademoiselle Bourienne. When she entered with her heavy step,
treading on her heels, the gentlemen and Mademoiselle Bourienne
rose and the little princess, indicating her to the gentlemen,
said: ``Voila Marie!'' Princess Mary saw them all and saw them in
detail. She saw Prince Vasili's face, serious for an instant at
the sight of her, but immediately smiling again, and the little
princess curiously noting the impression ``Marie'' produced on
the visitors. And she saw Mademoiselle Bourienne, with her ribbon
and pretty face, and her unusually animated look which was fixed
on him, but him she could not see, she only saw something large,
brilliant, and handsome moving toward her as she entered the
room. Prince Vasili approached first, and she kissed the bold
forehead that bent over her hand and answered his question by
saying that, on the contrary, she remembered him quite well. Then
Anatole came up to her. She still could not see him. She only
felt a soft hand taking hers firmly, and she touched with her
lips a white forehead, over which was beautiful light-brown hair
smelling of pomade. When she looked up at him she was struck by
his beauty. Anatole stood with his right thumb under a button of
his uniform, his chest expanded and his back drawn in, slightly
swinging one foot, and, with his head a little bent, looked with
beaming face at the princess without speaking and evidently not
thinking about her at all.  Anatole was not quick-witted, nor
ready or eloquent in conversation, but he had the faculty, so
invaluable in society, of composure and imperturbable
self-possession. If a man lacking in self-confidence remains dumb
on a first introduction and betrays a consciousness of the
impropriety of such silence and an anxiety to find something to
say, the effect is bad. But Anatole was dumb, swung his foot, and
smilingly examined the princess' hair. It was evident that he
could be silent in this way for a very long time. ``If anyone
finds this silence inconvenient, let him talk, but I don't want
to,'' he seemed to say.  Besides this, in his behavior to women
Anatole had a manner which particularly inspires in them
curiosity, awe, and even love---a supercilious consciousness of
his own superiority. It was as if he said to them: ``I know you,
I know you, but why should I bother about you?  You'd be only too
glad, of course.'' Perhaps he did not really think this when he
met women---even probably he did not, for in general he thought
very little---but his looks and manner gave that impression. The
princess felt this, and as if wishing to show him that she did
not even dare expect to interest him, she turned to his
father. The conversation was general and animated, thanks to
Princess Lise's voice and little downy lip that lifted over her
white teeth. She met Prince Vasili with that playful manner often
employed by lively chatty people, and consisting in the
assumption that between the person they so address and themselves
there are some semi-private, long-established jokes and amusing
reminiscences, though no such reminiscences really exist---just
as none existed in this case. Prince Vasili readily adopted her
tone and the little princess also drew Anatole, whom she hardly
knew, into these amusing recollections of things that had never
occurred. Mademoiselle Bourienne also shared them and even
Princess Mary felt herself pleasantly made to share in these
merry reminiscences.

``Here at least we shall have the benefit of your company all to
ourselves, dear prince,'' said the little princess (of course, in
French) to Prince Vasili. ``It's not as at
Annette's\footnote{Anna Pavlovna.} receptions where you always
ran away; you remember cette chere Annette!''

``Ah, but you won't talk politics to me like Annette!''

``And our little tea table?''

``Oh, yes!''

``Why is it you were never at Annette's?'' the little princess
asked Anatole. ``Ah, I know, I know,'' she said with a sly
glance, ``your brother Hippolyte told me about your goings
on. Oh!'' and she shook her finger at him, ``I have even heard of
your doings in Paris!''

``And didn't Hippolyte tell you?'' asked Prince Vasili, turning
to his son and seizing the little princess' arm as if she would
have run away and he had just managed to catch her, ``didn't he
tell you how he himself was pining for the dear princess, and how
she showed him the door? Oh, she is a pearl among women,
Princess,'' he added, turning to Princess Mary.

When Paris was mentioned, Mademoiselle Bourienne for her part
seized the opportunity of joining in the general current of
recollections.

She took the liberty of inquiring whether it was long since
Anatole had left Paris and how he had liked that city. Anatole
answered the Frenchwoman very readily and, looking at her with a
smile, talked to her about her native land. When he saw the
pretty little Bourienne, Anatole came to the conclusion that he
would not find Bald Hills dull either.  ``Not at all bad!'' he
thought, examining her, ``not at all bad, that little companion!
I hope she will bring her along with her when we're married, la
petite est gentille.''\footnote{The little one is charming.}

The old prince dressed leisurely in his study, frowning and
considering what he was to do. The coming of these visitors
annoyed him. ``What are Prince Vasili and that son of his to me?
Prince Vasili is a shallow braggart and his son, no doubt, is a
fine specimen,'' he grumbled to himself. What angered him was
that the coming of these visitors revived in his mind an
unsettled question he always tried to stifle, one about which he
always deceived himself. The question was whether he could ever
bring himself to part from his daughter and give her to a
husband. The prince never directly asked himself that question,
knowing beforehand that he would have to answer it justly, and
justice clashed not only with his feelings but with the very
possibility of life. Life without Princess Mary, little as he
seemed to value her, was unthinkable to him.  ``And why should
she marry?'' he thought. ``To be unhappy for certain.  There's
Lise, married to Andrew---a better husband one would think could
hardly be found nowadays---but is she contented with her lot? And
who would marry Marie for love? Plain and awkward! They'll take
her for her connections and wealth. Are there no women living
unmarried, and even the happier for it?'' So thought Prince
Bolkonski while dressing, and yet the question he was always
putting off demanded an immediate answer.  Prince Vasili had
brought his son with the evident intention of proposing, and
today or tomorrow he would probably ask for an answer.  His birth
and position in society were not bad. ``Well, I've nothing
against it,'' the prince said to himself, ``but he must be worthy
of her.  And that is what we shall see.''

``That is what we shall see! That is what we shall see!'' he
added aloud.

He entered the drawing room with his usual alert step, glancing
rapidly round the company. He noticed the change in the little
princess' dress, Mademoiselle Bourienne's ribbon, Princess Mary's
unbecoming coiffure, Mademoiselle Bourienne's and Anatole's
smiles, and the loneliness of his daughter amid the general
conversation. ``Got herself up like a fool!'' he thought, looking
irritably at her. ``She is shameless, and he ignores her!''

He went straight up to Prince Vasili.

``Well! How d'ye do? How d'ye do? Glad to see you!''

``Friendship laughs at distance,'' began Prince Vasili in his
usual rapid, self-confident, familiar tone. ``Here is my second
son; please love and befriend him.''

Prince Bolkonski surveyed Anatole.

``Fine young fellow! Fine young fellow!'' he said. ``Well, come
and kiss me,'' and he offered his cheek.

Anatole kissed the old man, and looked at him with curiosity and
perfect composure, waiting for a display of the eccentricities
his father had told him to expect.

Prince Bolkonski sat down in his usual place in the corner of the
sofa and, drawing up an armchair for Prince Vasili, pointed to it
and began questioning him about political affairs and news. He
seemed to listen attentively to what Prince Vasili said, but kept
glancing at Princess Mary.

``And so they are writing from Potsdam already?'' he said,
repeating Prince Vasili's last words. Then rising, he suddenly
went up to his daughter.

``Is it for visitors you've got yourself up like that, eh?'' said
he.  ``Fine, very fine! You have done up your hair in this new
way for the visitors, and before the visitors I tell you that in
future you are never to dare to change your way of dress without
my consent.''

``It was my fault, mon pere,'' interceded the little princess,
with a blush.

``You must do as you please,'' said Prince Bolkonski, bowing to
his daughter-in-law, ``but she need not make a fool of herself,
she's plain enough as it is.''

And he sat down again, paying no more attention to his daughter,
who was reduced to tears.

``On the contrary, that coiffure suits the princess very well,''
said Prince Vasili.

``Now you, young prince, what's your name?'' said Prince
Bolkonski, turning to Anatole, ``come here, let us talk and get
acquainted.''

``Now the fun begins,'' thought Anatole, sitting down with a
smile beside the old prince.

``Well, my dear boy, I hear you've been educated abroad, not
taught to read and write by the deacon, like your father and
me. Now tell me, my dear boy, are you serving in the Horse
Guards?'' asked the old man, scrutinizing Anatole closely and
intently.

``No, I have been transferred to the line,'' said Anatole, hardly
able to restrain his laughter.

``Ah! That's a good thing. So, my dear boy, you wish to serve the
Tsar and the country? It is wartime. Such a fine fellow must
serve. Well, are you off to the front?''

``No, Prince, our regiment has gone to the front, but I am
attached...  what is it I am attached to, Papa?'' said Anatole,
turning to his father with a laugh.

``A splendid soldier, splendid! 'What am I attached to!' Ha, ha,
ha!''  laughed Prince Bolkonski, and Anatole laughed still
louder. Suddenly Prince Bolkonski frowned.

``You may go,'' he said to Anatole.

Anatole returned smiling to the ladies.

``And so you've had him educated abroad, Prince Vasili, haven't
you?''  said the old prince to Prince Vasili.

``I have done my best for him, and I can assure you the education
there is much better than ours.''

``Yes, everything is different nowadays, everything is
changed. The lad's a fine fellow, a fine fellow! Well, come with
me now.'' He took Prince Vasili's arm and led him to his
study. As soon as they were alone together, Prince Vasili
announced his hopes and wishes to the old prince.

``Well, do you think I shall prevent her, that I can't part from
her?''  said the old prince angrily. ``What an idea! I'm ready
for it tomorrow!  Only let me tell you, I want to know my
son-in-law better. You know my principles---everything
aboveboard? I will ask her tomorrow in your presence; if she is
willing, then he can stay on. He can stay and I'll see.'' The old
prince snorted. ``Let her marry, it's all the same to me!''  he
screamed in the same piercing tone as when parting from his son.

``I will tell you frankly,'' said Prince Vasili in the tone of a
crafty man convinced of the futility of being cunning with so
keen-sighted a companion. ``You know, you see right through
people. Anatole is no genius, but he is an honest, goodhearted
lad; an excellent son or kinsman.''

``All right, all right, we'll see!''

As always happens when women lead lonely lives for any length of
time without male society, on Anatole's appearance all the three
women of Prince Bolkonski's household felt that their life had
not been real till then. Their powers of reasoning, feeling, and
observing immediately increased tenfold, and their life, which
seemed to have been passed in darkness, was suddenly lit up by a
new brightness, full of significance.

Princess Mary grew quite unconscious of her face and
coiffure. The handsome open face of the man who might perhaps be
her husband absorbed all her attention. He seemed to her kind,
brave, determined, manly, and magnanimous. She felt convinced of
that. Thousands of dreams of a future family life continually
rose in her imagination. She drove them away and tried to conceal
them.

``But am I not too cold with him?'' thought the princess. ``I try
to be reserved because in the depth of my soul I feel too near to
him already, but then he cannot know what I think of him and may
imagine that I do not like him.''

And Princess Mary tried, but could not manage, to be cordial to
her new guest. ``Poor girl, she's devilish ugly!'' thought
Anatole.

Mademoiselle Bourienne, also roused to great excitement by
Anatole's arrival, thought in another way. Of course, she, a
handsome young woman without any definite position, without
relations or even a country, did not intend to devote her life to
serving Prince Bolkonski, to reading aloud to him and being
friends with Princess Mary. Mademoiselle Bourienne had long been
waiting for a Russian prince who, able to appreciate at a glance
her superiority to the plain, badly dressed, ungainly Russian
princesses, would fall in love with her and carry her off; and
here at last was a Russian prince. Mademoiselle Bourienne knew a
story, heard from her aunt but finished in her own way, which she
liked to repeat to herself. It was the story of a girl who had
been seduced, and to whom her poor mother (sa pauvre mere)
appeared, and reproached her for yielding to a man without being
married. Mademoiselle Bourienne was often touched to tears as in
imagination she told this story to him, her seducer. And now he,
a real Russian prince, had appeared. He would carry her away and
then sa pauvre mere would appear and he would marry her. So her
future shaped itself in Mademoiselle Bourienne's head at the very
time she was talking to Anatole about Paris. It was not
calculation that guided her (she did not even for a moment
consider what she should do), but all this had long been familiar
to her, and now that Anatole had appeared it just grouped itself
around him and she wished and tried to please him as much as
possible.

The little princess, like an old war horse that hears the
trumpet, unconsciously and quite forgetting her condition,
prepared for the familiar gallop of coquetry, without any
ulterior motive or any struggle, but with naive and lighthearted
gaiety.

Although in female society Anatole usually assumed the role of a
man tired of being run after by women, his vanity was flattered
by the spectacle of his power over these three women. Besides
that, he was beginning to feel for the pretty and provocative
Mademoiselle Bourienne that passionate animal feeling which was
apt to master him with great suddenness and prompt him to the
coarsest and most reckless actions.

After tea, the company went into the sitting room and Princess
Mary was asked to play on the clavichord. Anatole, laughing and
in high spirits, came and leaned on his elbows, facing her and
beside Mademoiselle Bourienne. Princess Mary felt his look with a
painfully joyous emotion.  Her favorite sonata bore her into a
most intimately poetic world and the look she felt upon her made
that world still more poetic. But Anatole's expression, though
his eyes were fixed on her, referred not to her but to the
movements of Mademoiselle Bourienne's little foot, which he was
then touching with his own under the clavichord. Mademoiselle
Bourienne was also looking at Princess Mary, and in her lovely
eyes there was a look of fearful joy and hope that was also new
to the princess.

``How she loves me!'' thought Princess Mary. ``How happy I am
now, and how happy I may be with such a friend and such a
husband! Husband? Can it be possible?'' she thought, not daring
to look at his face, but still feeling his eyes gazing at her.

In the evening, after supper, when all were about to retire,
Anatole kissed Princess Mary's hand. She did not know how she
found the courage, but she looked straight into his handsome face
as it came near to her shortsighted eyes. Turning from Princess
Mary he went up and kissed Mademoiselle Bourienne's hand. (This
was not etiquette, but then he did everything so simply and with
such assurance!) Mademoiselle Bourienne flushed, and gave the
princess a frightened look.

``What delicacy!'' thought the princess. ``Is it possible that
Amelie'' (Mademoiselle Bourienne) ``thinks I could be jealous of
her, and not value her pure affection and devotion to me?'' She
went up to her and kissed her warmly. Anatole went up to kiss the
little princess' hand.

``No! No! No! When your father writes to tell me that you are
behaving well I will give you my hand to kiss. Not till then!''
she said. And smilingly raising a finger at him, she left the
room.

% % % % % % % % % % % % % % % % % % % % % % % % % % % % % % % % %
% % % % % % % % % % % % % % % % % % % % % % % % % % % % % % % % %
% % % % % % % % % % % % % % % % % % % % % % % % % % % % % % % % %
% % % % % % % % % % % % % % % % % % % % % % % % % % % % % % % % %
% % % % % % % % % % % % % % % % % % % % % % % % % % % % % % % % %
% % % % % % % % % % % % % % % % % % % % % % % % % % % % % % % % %
% % % % % % % % % % % % % % % % % % % % % % % % % % % % % % % % %
% % % % % % % % % % % % % % % % % % % % % % % % % % % % % % % % %
% % % % % % % % % % % % % % % % % % % % % % % % % % % % % % % % %
% % % % % % % % % % % % % % % % % % % % % % % % % % % % % % % % %
% % % % % % % % % % % % % % % % % % % % % % % % % % % % % % % % %
% % % % % % % % % % % % % % % % % % % % % % % % % % % % % %

\chapter*{Chapter V}
\ifaudio     
\marginpar{
\href{http://ia800208.us.archive.org/14/items/war_and_peace_03_0712_librivox/war_and_peace_03_05_tolstoy_64kb.mp3}{Audio}} 
\fi

\lettrine[lines=2, loversize=0.3, lraise=0]{\initfamily T}{hey }
all separated, but, except Anatole who fell asleep as soon
as he got into bed, all kept awake a long time that night.

``Is he really to be my husband, this stranger who is so
kind---yes, kind, that is the chief thing,'' thought Princess
Mary; and fear, which she had seldom experienced, came upon
her. She feared to look round, it seemed to her that someone was
there standing behind the screen in the dark corner. And this
someone was he---the devil---and he was also this man with the
white forehead, black eyebrows, and red lips.

She rang for her maid and asked her to sleep in her room.

Mademoiselle Bourienne walked up and down the conservatory for a
long time that evening, vainly expecting someone, now smiling at
someone, now working herself up to tears with the imaginary words
of her pauvre mere rebuking her for her fall.

The little princess grumbled to her maid that her bed was badly
made.  She could not lie either on her face or on her side. Every
position was awkward and uncomfortable, and her burden oppressed
her now more than ever because Anatole's presence had vividly
recalled to her the time when she was not like that and when
everything was light and gay. She sat in an armchair in her
dressing jacket and nightcap and Katie, sleepy and disheveled,
beat and turned the heavy feather bed for the third time,
muttering to herself.

``I told you it was all lumps and holes!'' the little princess
repeated.  ``I should be glad enough to fall asleep, so it's not
my fault!'' and her voice quivered like that of a child about to
cry.

The old prince did not sleep either. Tikhon, half asleep, heard
him pacing angrily about and snorting. The old prince felt as
though he had been insulted through his daughter. The insult was
the more pointed because it concerned not himself but another,
his daughter, whom he loved more than himself. He kept telling
himself that he would consider the whole matter and decide what
was right and how he should act, but instead of that he only
excited himself more and more.

``The first man that turns up---she forgets her father and
everything else, runs upstairs and does up her hair and wags her
tail and is unlike herself! Glad to throw her father over! And
she knew I should notice it.  Fr... fr... fr! And don't I see
that that idiot had eyes only for Bourienne---I shall have to get
rid of her. And how is it she has not pride enough to see it? If
she has no pride for herself she might at least have some for my
sake! She must be shown that the blockhead thinks nothing of her
and looks only at Bourienne. No, she has no pride... but I'll let
her see...''

The old prince knew that if he told his daughter she was making a
mistake and that Anatole meant to flirt with Mademoiselle
Bourienne, Princess Mary's self-esteem would be wounded and his
point (not to be parted from her) would be gained, so pacifying
himself with this thought, he called Tikhon and began to undress.

``What devil brought them here?'' thought he, while Tikhon was
putting the nightshirt over his dried-up old body and gray-haired
chest. ``I never invited them. They came to disturb my life---and
there is not much of it left.''

``Devil take 'em!'' he muttered, while his head was still covered
by the shirt.

Tikhon knew his master's habit of sometimes thinking aloud, and
therefore met with unaltered looks the angrily inquisitive
expression of the face that emerged from the shirt.

``Gone to bed?'' asked the prince.

Tikhon, like all good valets, instinctively knew the direction of
his master's thoughts. He guessed that the question referred to
Prince Vasili and his son.

``They have gone to bed and put out their lights, your
excellency.''

``No good... no good...'' said the prince rapidly, and thrusting
his feet into his slippers and his arms into the sleeves of his
dressing gown, he went to the couch on which he slept.

Though no words had passed between Anatole and Mademoiselle
Bourienne, they quite understood one another as to the first part
of their romance, up to the appearance of the pauvre mere; they
understood that they had much to say to one another in private
and so they had been seeking an opportunity since morning to meet
one another alone. When Princess Mary went to her father's room
at the usual hour, Mademoiselle Bourienne and Anatole met in the
conservatory.

Princess Mary went to the door of the study with special
trepidation. It seemed to her that not only did everybody know
that her fate would be decided that day, but that they also knew
what she thought about it. She read this in Tikhon's face and in
that of Prince Vasili's valet, who made her a low bow when she
met him in the corridor carrying hot water.

The old prince was very affectionate and careful in his treatment
of his daughter that morning. Princess Mary well knew this
painstaking expression of her father's. His face wore that
expression when his dry hands clenched with vexation at her not
understanding a sum in arithmetic, when rising from his chair he
would walk away from her, repeating in a low voice the same words
several times over.

He came to the point at once, treating her ceremoniously.

``I have had a proposition made me concerning you,'' he said with
an unnatural smile. ``I expect you have guessed that Prince
Vasili has not come and brought his pupil with him'' (for some
reason Prince Bolkonski referred to Anatole as a ``pupil'') ``for
the sake of my beautiful eyes.  Last night a proposition was made
me on your account and, as you know my principles, I refer it to
you.''

``How am I to understand you, mon pere?'' said the princess,
growing pale and then blushing.

``How understand me!'' cried her father angrily. ``Prince Vasili
finds you to his taste as a daughter-in-law and makes a proposal
to you on his pupil's behalf. That's how it's to be understood!
'How understand it'!... And I ask you!''

``I do not know what you think, Father,'' whispered the princess.

``I? I? What of me? Leave me out of the question. I'm not going
to get married. What about you? That's what I want to know.''

The princess saw that her father regarded the matter with
disapproval, but at that moment the thought occurred to her that
her fate would be decided now or never. She lowered her eyes so
as not to see the gaze under which she felt that she could not
think, but would only be able to submit from habit, and she said:
``I wish only to do your will, but if I had to express my own
desire...'' She had no time to finish. The old prince interrupted
her.

``That's admirable!'' he shouted. ``He will take you with your
dowry and take Mademoiselle Bourienne into the bargain. She'll be
the wife, while you...''

The prince stopped. He saw the effect these words had produced on
his daughter. She lowered her head and was ready to burst into
tears.

``Now then, now then, I'm only joking!'' he said. ``Remember
this, Princess, I hold to the principle that a maiden has a full
right to choose. I give you freedom. Only remember that your
life's happiness depends on your decision. Never mind me!''

``But I do not know, Father!''

``There's no need to talk! He receives his orders and will marry
you or anybody; but you are free to choose... Go to your room,
think it over, and come back in an hour and tell me in his
presence: yes or no. I know you will pray over it. Well, pray if
you like, but you had better think it over. Go! Yes or no, yes or
no, yes or no!'' he still shouted when the princess, as if lost
in a fog, had already staggered out of the study.

Her fate was decided and happily decided. But what her father had
said about Mademoiselle Bourienne was dreadful. It was untrue to
be sure, but still it was terrible, and she could not help
thinking of it. She was going straight on through the
conservatory, neither seeing nor hearing anything, when suddenly
the well-known whispering of Mademoiselle Bourienne aroused
her. She raised her eyes, and two steps away saw Anatole
embracing the Frenchwoman and whispering something to her. With a
horrified expression on his handsome face, Anatole looked at
Princess Mary, but did not at once take his arm from the waist of
Mademoiselle Bourienne who had not yet seen her.

``Who's that? Why? Wait a moment!'' Anatole's face seemed to
say. Princess Mary looked at them in silence. She could not
understand it. At last Mademoiselle Bourienne gave a scream and
ran away. Anatole bowed to Princess Mary with a gay smile, as if
inviting her to join in a laugh at this strange incident, and
then shrugging his shoulders went to the door that led to his own
apartments.

An hour later, Tikhon came to call Princess Mary to the old
prince; he added that Prince Vasili was also there. When Tikhon
came to her Princess Mary was sitting on the sofa in her room,
holding the weeping Mademoiselle Bourienne in her arms and gently
stroking her hair. The princess' beautiful eyes with all their
former calm radiance were looking with tender affection and pity
at Mademoiselle Bourienne's pretty face.

``No, Princess, I have lost your affection forever!'' said
Mademoiselle Bourienne.

``Why? I love you more than ever,'' said Princess Mary, ``and I
will try to do all I can for your happiness.''

``But you despise me. You who are so pure can never understand
being so carried away by passion. Oh, only my poor mother...''

``I quite understand,'' answered Princess Mary, with a sad
smile. ``Calm yourself, my dear. I will go to my father,'' she
said, and went out.

Prince Vasili, with one leg thrown high over the other and a
snuffbox in his hand, was sitting there with a smile of deep
emotion on his face, as if stirred to his heart's core and
himself regretting and laughing at his own sensibility, when
Princess Mary entered. He hurriedly took a pinch of snuff.

``Ah, my dear, my dear!'' he began, rising and taking her by both
hands.  Then, sighing, he added: ``My son's fate is in your
hands. Decide, my dear, good, gentle Marie, whom I have always
loved as a daughter!''

He drew back and a real tear appeared in his eye.

``Fr... fr...'' snorted Prince Bolkonski. ``The prince is making
a proposition to you in his pupil's---I mean, his
son's---name. Do you wish or not to be Prince Anatole Kuragin's
wife? Reply: yes or no,'' he shouted, ``and then I shall reserve
the right to state my opinion also.  Yes, my opinion, and only my
opinion,'' added Prince Bolkonski, turning to Prince Vasili and
answering his imploring look. ``Yes, or no?''

``My desire is never to leave you, Father, never to separate my
life from yours. I don't wish to marry,`` she answered
positively, glancing at Prince Vasili and at her father with her
beautiful eyes.

``Humbug! Nonsense! Humbug, humbug, humbug!`` cried Prince
Bolkonski, frowning and taking his daughter's hand; he did not
kiss her, but only bending his forehead to hers just touched it,
and pressed her hand so that she winced and uttered a cry.

Prince Vasili rose.

``My dear, I must tell you that this is a moment I shall never,
never forget. But, my dear, will you not give us a little hope of
touching this heart, so kind and generous? Say 'perhaps'... The
future is so long. Say 'perhaps.'{}''

``Prince, what I have said is all there is in my heart. I thank
you for the honor, but I shall never be your son's wife.''

``Well, so that's finished, my dear fellow! I am very glad to
have seen you. Very glad! Go back to your rooms, Princess. Go!''
said the old prince. ``Very, very glad to have seen you,''
repeated he, embracing Prince Vasili.

``My vocation is a different one,'' thought Princess Mary. ``My
vocation is to be happy with another kind of happiness, the
happiness of love and self-sacrifice. And cost what it may, I
will arrange poor Amelie's happiness, she loves him so
passionately, and so passionately repents. I will do all I can to
arrange the match between them. If he is not rich I will give her
the means; I will ask my father and Andrew. I shall be so happy
when she is his wife. She is so unfortunate, a stranger, alone,
helpless! And, oh God, how passionately she must love him if she
could so far forget herself! Perhaps I might have done the
same!...'' thought Princess Mary.

% % % % % % % % % % % % % % % % % % % % % % % % % % % % % % % % %
% % % % % % % % % % % % % % % % % % % % % % % % % % % % % % % % %
% % % % % % % % % % % % % % % % % % % % % % % % % % % % % % % % %
% % % % % % % % % % % % % % % % % % % % % % % % % % % % % % % % %
% % % % % % % % % % % % % % % % % % % % % % % % % % % % % % % % %
% % % % % % % % % % % % % % % % % % % % % % % % % % % % % % % % %
% % % % % % % % % % % % % % % % % % % % % % % % % % % % % % % % %
% % % % % % % % % % % % % % % % % % % % % % % % % % % % % % % % %
% % % % % % % % % % % % % % % % % % % % % % % % % % % % % % % % %
% % % % % % % % % % % % % % % % % % % % % % % % % % % % % % % % %
% % % % % % % % % % % % % % % % % % % % % % % % % % % % % % % % %
% % % % % % % % % % % % % % % % % % % % % % % % % % % % % %

\chapter*{Chapter VI}
\ifaudio     
\marginpar{
\href{http://ia800208.us.archive.org/14/items/war_and_peace_03_0712_librivox/war_and_peace_03_06_tolstoy_64kb.mp3}{Audio}} 
\fi

\lettrine[lines=2, loversize=0.3, lraise=0]{\initfamily I}{t}
was long since the Rostovs had news of Nicholas. Not till
midwinter was the count at last handed a letter addressed in his
son's handwriting. On receiving it, he ran on tiptoe to his study
in alarm and haste, trying to escape notice, closed the door, and
began to read the letter.

Anna Mikhaylovna, who always knew everything that passed in the
house, on hearing of the arrival of the letter went softly into
the room and found the count with it in his hand, sobbing and
laughing at the same time.

Anna Mikhaylovna, though her circumstances had improved, was
still living with the Rostovs.

``My dear friend?'' said she, in a tone of pathetic inquiry,
prepared to sympathize in any way.

The count sobbed yet more.

``Nikolenka... a letter... wa... a... s... wounded... my darling
boy...  the countess... promoted to be an officer... thank
God... How tell the little countess!''

Anna Mikhaylovna sat down beside him, with her own handkerchief
wiped the tears from his eyes and from the letter, then having
dried her own eyes she comforted the count, and decided that at
dinner and till teatime she would prepare the countess, and after
tea, with God's help, would inform her.

At dinner Anna Mikhaylovna talked the whole time about the war
news and about Nikolenka, twice asked when the last letter had
been received from him, though she knew that already, and
remarked that they might very likely be getting a letter from him
that day. Each time that these hints began to make the countess
anxious and she glanced uneasily at the count and at Anna
Mikhaylovna, the latter very adroitly turned the conversation to
insignificant matters. Natasha, who, of the whole family, was the
most gifted with a capacity to feel any shades of intonation,
look, and expression, pricked up her ears from the beginning of
the meal and was certain that there was some secret between her
father and Anna Mikhaylovna, that it had something to do with her
brother, and that Anna Mikhaylovna was preparing them for
it. Bold as she was, Natasha, who knew how sensitive her mother
was to anything relating to Nikolenka, did not venture to ask any
questions at dinner, but she was too excited to eat anything and
kept wriggling about on her chair regardless of her governess'
remarks. After dinner, she rushed head long after Anna
Mikhaylovna and, dashing at her, flung herself on her neck as
soon as she overtook her in the sitting room.

``Auntie, darling, do tell me what it is!''

``Nothing, my dear.''

``No, dearest, sweet one, honey, I won't give up---I know you
know something.''

Anna Mikhaylovna shook her head.

``You are a little slyboots,'' she said.

``A letter from Nikolenka! I'm sure of it!'' exclaimed Natasha,
reading confirmation in Anna Mikhaylovna's face.

``But for God's sake, be careful, you know how it may affect your
mamma.''

``I will, I will, only tell me! You won't? Then I will go and
tell at once.''

Anna Mikhaylovna, in a few words, told her the contents of the
letter, on condition that she should tell no one.

``No, on my true word of honor,'' said Natasha, crossing herself,
``I won't tell anyone!'' and she ran off at once to Sonya.

``Nikolenka... wounded... a letter,'' she announced in gleeful
triumph.

``Nicholas!'' was all Sonya said, instantly turning white.

Natasha, seeing the impression the news of her brother's wound
produced on Sonya, felt for the first time the sorrowful side of
the news.

She rushed to Sonya, hugged her, and began to cry.

``A little wound, but he has been made an officer; he is well
now, he wrote himself,'' said she through her tears.

``There now! It's true that all you women are crybabies,''
remarked Petya, pacing the room with large, resolute
strides. ``Now I'm very glad, very glad indeed, that my brother
has distinguished himself so. You are all blubberers and
understand nothing.''

Natasha smiled through her tears.

``You haven't read the letter?'' asked Sonya.

``No, but she said that it was all over and that he's now an
officer.''

``Thank God!'' said Sonya, crossing herself. ``But perhaps she
deceived you. Let us go to Mamma.''

Petya paced the room in silence for a time.

``If I'd been in Nikolenka's place I would have killed even more
of those Frenchmen,'' he said. ``What nasty brutes they are! I'd
have killed so many that there'd have been a heap of them.''

``Hold your tongue, Petya, what a goose you are!''

``I'm not a goose, but they are who cry about trifles,'' said
Petya.

``Do you remember him?'' Natasha suddenly asked, after a moment's
silence.

Sonya smiled.

``Do I remember Nicholas?''

``No, Sonya, but do you remember so that you remember him
perfectly, remember everything?'' said Natasha, with an
expressive gesture, evidently wishing to give her words a very
definite meaning. ``I remember Nikolenka too, I remember him
well,'' she said. ``But I don't remember Boris. I don't remember
him a bit.''

``What! You don't remember Boris?'' asked Sonya in surprise.

``It's not that I don't remember---I know what he is like, but
not as I remember Nikolenka. Him---I just shut my eyes and
remember, but Boris...  No!'' (She shut her eyes.) ``No! there's
nothing at all.''

``Oh, Natasha!'' said Sonya, looking ecstatically and earnestly
at her friend as if she did not consider her worthy to hear what
she meant to say and as if she were saying it to someone else,
with whom joking was out of the question, ``I am in love with
your brother once for all and, whatever may happen to him or to
me, shall never cease to love him as long as I live.''

Natasha looked at Sonya with wondering and inquisitive eyes, and
said nothing. She felt that Sonya was speaking the truth, that
there was such love as Sonya was speaking of. But Natasha had not
yet felt anything like it. She believed it could be, but did not
understand it.

``Shall you write to him?'' she asked.

Sonya became thoughtful. The question of how to write to
Nicholas, and whether she ought to write, tormented her. Now that
he was already an officer and a wounded hero, would it be right
to remind him of herself and, as it might seem, of the
obligations to her he had taken on himself?

``I don't know. I think if he writes, I will write too,'' she
said, blushing.

``And you won't feel ashamed to write to him?''

Sonya smiled.

``No.''

``And I should be ashamed to write to Boris. I'm not going to.''

``Why should you be ashamed?''

``Well, I don't know. It's awkward and would make me ashamed.''

``And I know why she'd be ashamed,'' said Petya, offended by
Natasha's previous remark. ``It's because she was in love with
that fat one in spectacles'' (that was how Petya described his
namesake, the new Count Bezukhov) ``and now she's in love with
that singer'' (he meant Natasha's Italian singing master),
``that's why she's ashamed!''

``Petya, you're a stupid!'' said Natasha.

``Not more stupid than you, madam,'' said the nine-year-old
Petya, with the air of an old brigadier.

The countess had been prepared by Anna Mikhaylovna's hints at
dinner. On retiring to her own room, she sat in an armchair, her
eyes fixed on a miniature portrait of her son on the lid of a
snuffbox, while the tears kept coming into her eyes. Anna
Mikhaylovna, with the letter, came on tiptoe to the countess'
door and paused.

``Don't come in,'' she said to the old count who was following
her. ``Come later.'' And she went in, closing the door behind
her.

The count put his ear to the keyhole and listened.

At first he heard the sound of indifferent voices, then Anna
Mikhaylovna's voice alone in a long speech, then a cry, then
silence, then both voices together with glad intonations, and
then footsteps.  Anna Mikhaylovna opened the door. Her face wore
the proud expression of a surgeon who has just performed a
difficult operation and admits the public to appreciate his
skill.

``It is done!'' she said to the count, pointing triumphantly to
the countess, who sat holding in one hand the snuffbox with its
portrait and in the other the letter, and pressing them
alternately to her lips.

When she saw the count, she stretched out her arms to him,
embraced his bald head, over which she again looked at the letter
and the portrait, and in order to press them again to her lips,
she slightly pushed away the bald head. Vera, Natasha, Sonya, and
Petya now entered the room, and the reading of the letter
began. After a brief description of the campaign and the two
battles in which he had taken part, and his promotion, Nicholas
said that he kissed his father's and mother's hands asking for
their blessing, and that he kissed Vera, Natasha, and Petya.
Besides that, he sent greetings to Monsieur Schelling, Madame
Schoss, and his old nurse, and asked them to kiss for him ``dear
Sonya, whom he loved and thought of just the same as ever.'' When
she heard this Sonya blushed so that tears came into her eyes
and, unable to bear the looks turned upon her, ran away into the
dancing hall, whirled round it at full speed with her dress
puffed out like a balloon, and, flushed and smiling, plumped down
on the floor. The countess was crying.

``Why are you crying, Mamma?'' asked Vera. ``From all he says one
should be glad and not cry.''

This was quite true, but the count, the countess, and Natasha
looked at her reproachfully. ``And who is it she takes after?''
thought the countess.

Nicholas' letter was read over hundreds of times, and those who
were considered worthy to hear it had to come to the countess,
for she did not let it out of her hands. The tutors came, and the
nurses, and Dmitri, and several acquaintances, and the countess
reread the letter each time with fresh pleasure and each time
discovered in it fresh proofs of Nikolenka's virtues. How
strange, how extraordinary, how joyful it seemed, that her son,
the scarcely perceptible motion of whose tiny limbs she had felt
twenty years ago within her, that son about whom she used to have
quarrels with the too indulgent count, that son who had first
learned to say ``pear'' and then ``granny,'' that this son should
now be away in a foreign land amid strange surroundings, a manly
warrior doing some kind of man's work of his own, without help or
guidance. The universal experience of ages, showing that children
do grow imperceptibly from the cradle to manhood, did not exist
for the countess. Her son's growth toward manhood, at each of its
stages, had seemed as extraordinary to her as if there had never
existed the millions of human beings who grew up in the same
way. As twenty years before, it seemed impossible that the little
creature who lived somewhere under her heart would ever cry, suck
her breast, and begin to speak, so now she could not believe that
that little creature could be this strong, brave man, this model
son and officer that, judging by this letter, he now was.

``What a style! How charmingly he describes!'' said she, reading
the descriptive part of the letter. ``And what a soul! Not a word
about himself... Not a word! About some Denisov or other, though
he himself, I dare say, is braver than any of them. He says
nothing about his sufferings. What a heart! How like him it is!
And how he has remembered everybody! Not forgetting anyone. I
always said when he was only so high---I always said...''

For more than a week preparations were being made, rough drafts
of letters to Nicholas from all the household were written and
copied out, while under the supervision of the countess and the
solicitude of the count, money and all things necessary for the
uniform and equipment of the newly commissioned officer were
collected. Anna Mikhaylovna, practical woman that she was, had
even managed by favor with army authorities to secure
advantageous means of communication for herself and her son. She
had opportunities of sending her letters to the Grand Duke
Constantine Pavlovich, who commanded the Guards. The Rostovs
supposed that The Russian Guards, Abroad, was quite a definite
address, and that if a letter reached the Grand Duke in command
of the Guards there was no reason why it should not reach the
Pavlograd regiment, which was presumably somewhere in the same
neighborhood. And so it was decided to send the letters and money
by the Grand Duke's courier to Boris and Boris was to forward
them to Nicholas. The letters were from the old count, the
countess, Petya, Vera, Natasha, and Sonya, and finally there were
six thousand rubles for his outfit and various other things the
old count sent to his son.

% % % % % % % % % % % % % % % % % % % % % % % % % % % % % % % % %
% % % % % % % % % % % % % % % % % % % % % % % % % % % % % % % % %
% % % % % % % % % % % % % % % % % % % % % % % % % % % % % % % % %
% % % % % % % % % % % % % % % % % % % % % % % % % % % % % % % % %
% % % % % % % % % % % % % % % % % % % % % % % % % % % % % % % % %
% % % % % % % % % % % % % % % % % % % % % % % % % % % % % % % % %
% % % % % % % % % % % % % % % % % % % % % % % % % % % % % % % % %
% % % % % % % % % % % % % % % % % % % % % % % % % % % % % % % % %
% % % % % % % % % % % % % % % % % % % % % % % % % % % % % % % % %
% % % % % % % % % % % % % % % % % % % % % % % % % % % % % % % % %
% % % % % % % % % % % % % % % % % % % % % % % % % % % % % % % % %
% % % % % % % % % % % % % % % % % % % % % % % % % % % % % %

\chapter*{Chapter VII}
\ifaudio     
\marginpar{
\href{http://ia800208.us.archive.org/14/items/war_and_peace_03_0712_librivox/war_and_peace_03_07_tolstoy_64kb.mp3}{Audio}} 
\fi

\lettrine[lines=2, loversize=0.3, lraise=0]{\initfamily O}{n}
the twelfth of November, Kutuzov's active army, in camp before
Olmutz, was preparing to be reviewed next day by the two
Emperors---the Russian and the Austrian. The Guards, just arrived
from Russia, spent the night ten miles from Olmutz and next
morning were to come straight to the review, reaching the field
at Olmutz by ten o'clock.

That day Nicholas Rostov received a letter from Boris, telling
him that the Ismaylov regiment was quartered for the night ten
miles from Olmutz and that he wanted to see him as he had a
letter and money for him.  Rostov was particularly in need of
money now that the troops, after their active service, were
stationed near Olmutz and the camp swarmed with well-provisioned
sutlers and Austrian Jews offering all sorts of tempting
wares. The Pavlograds held feast after feast, celebrating awards
they had received for the campaign, and made expeditions to
Olmutz to visit a certain Caroline the Hungarian, who had
recently opened a restaurant there with girls as
waitresses. Rostov, who had just celebrated his promotion to a
cornetcy and bought Denisov's horse, Bedouin, was in debt all
round, to his comrades and the sutlers. On receiving Boris'
letter he rode with a fellow officer to Olmutz, dined there,
drank a bottle of wine, and then set off alone to the Guards'
camp to find his old playmate. Rostov had not yet had time to get
his uniform. He had on a shabby cadet jacket, decorated with a
soldier's cross, equally shabby cadet's riding breeches lined
with worn leather, and an officer's saber with a sword knot. The
Don horse he was riding was one he had bought from a Cossack
during the campaign, and he wore a crumpled hussar cap stuck
jauntily back on one side of his head. As he rode up to the camp
he thought how he would impress Boris and all his comrades of the
Guards by his appearance---that of a fighting hussar who had been
under fire.

The Guards had made their whole march as if on a pleasure trip,
parading their cleanliness and discipline. They had come by easy
stages, their knapsacks conveyed on carts, and the Austrian
authorities had provided excellent dinners for the officers at
every halting place. The regiments had entered and left the town
with their bands playing, and by the Grand Duke's orders the men
had marched all the way in step (a practice on which the Guards
prided themselves), the officers on foot and at their proper
posts. Boris had been quartered, and had marched all the way,
with Berg who was already in command of a company. Berg, who had
obtained his captaincy during the campaign, had gained the
confidence of his superiors by his promptitude and accuracy and
had arranged his money matters very satisfactorily. Boris, during
the campaign, had made the acquaintance of many persons who might
prove useful to him, and by a letter of recommendation he had
brought from Pierre had become acquainted with Prince Andrew
Bolkonski, through whom he hoped to obtain a post on the
commander-in-chief's staff. Berg and Boris, having rested after
yesterday's march, were sitting, clean and neatly dressed, at a
round table in the clean quarters allotted to them, playing
chess. Berg held a smoking pipe between his knees. Boris, in the
accurate way characteristic of him, was building a little pyramid
of chessmen with his delicate white fingers while awaiting Berg's
move, and watched his opponent's face, evidently thinking about
the game as he always thought only of whatever he was engaged on.

``Well, how are you going to get out of that?'' he remarked.

``We'll try to,'' replied Berg, touching a pawn and then removing
his hand.

At that moment the door opened.

``Here he is at last!'' shouted Rostov. ``And Berg too! Oh, you
petisenfans, allay cushay dormir!'' he exclaimed, imitating his
Russian nurse's French, at which he and Boris used to laugh long
ago.

``Dear me, how you have changed!''

Boris rose to meet Rostov, but in doing so did not omit to steady
and replace some chessmen that were falling. He was about to
embrace his friend, but Nicholas avoided him. With that peculiar
feeling of youth, that dread of beaten tracks, and wish to
express itself in a manner different from that of its elders
which is often insincere, Nicholas wished to do something special
on meeting his friend. He wanted to pinch him, push him, do
anything but kiss him---a thing everybody did. But
notwithstanding this, Boris embraced him in a quiet, friendly way
and kissed him three times.

They had not met for nearly half a year and, being at the age
when young men take their first steps on life's road, each saw
immense changes in the other, quite a new reflection of the
society in which they had taken those first steps. Both had
changed greatly since they last met and both were in a hurry to
show the changes that had taken place in them.

``Oh, you damned dandies! Clean and fresh as if you'd been to a
fete, not like us sinners of the line,'' cried Rostov, with
martial swagger and with baritone notes in his voice, new to
Boris, pointing to his own mud-bespattered breeches. The German
landlady, hearing Rostov's loud voice, popped her head in at the
door.

``Eh, is she pretty?'' he asked with a wink.

``Why do you shout so? You'll frighten them!'' said Boris. ``I
did not expect you today,'' he added. ``I only sent you the note
yesterday by Bolkonski---an adjutant of Kutuzov's, who's a friend
of mine. I did not think he would get it to you so
quickly... Well, how are you? Been under fire already?'' asked
Boris.

Without answering, Rostov shook the soldier's Cross of St. George
fastened to the cording of his uniform and, indicating a bandaged
arm, glanced at Berg with a smile.

``As you see,'' he said.

``Indeed? Yes, yes!'' said Boris, with a smile. ``And we too have
had a splendid march. You know, of course, that His Imperial
Highness rode with our regiment all the time, so that we had
every comfort and every advantage. What receptions we had in
Poland! What dinners and balls! I can't tell you. And the
Tsarevich was very gracious to all our officers.''

And the two friends told each other of their doings, the one of
his hussar revels and life in the fighting line, the other of the
pleasures and advantages of service under members of the Imperial
family.

``Oh, you Guards!'' said Rostov. ``I say, send for some wine.''

Boris made a grimace.

``If you really want it,'' said he.

He went to his bed, drew a purse from under the clean pillow, and
sent for wine.

``Yes, and I have some money and a letter to give you,'' he
added.

Rostov took the letter and, throwing the money on the sofa, put
both arms on the table and began to read. After reading a few
lines, he glanced angrily at Berg, then, meeting his eyes, hid
his face behind the letter.

``Well, they've sent you a tidy sum,'' said Berg, eying the heavy
purse that sank into the sofa. ``As for us, Count, we get along
on our pay. I can tell you for myself...''

``I say, Berg, my dear fellow,'' said Rostov, ``when you get a
letter from home and meet one of your own people whom you want to
talk everything over with, and I happen to be there, I'll go at
once, to be out of your way! Do go somewhere, anywhere... to the
devil!'' he exclaimed, and immediately seizing him by the
shoulder and looking amiably into his face, evidently wishing to
soften the rudeness of his words, he added, ``Don't be hurt, my
dear fellow; you know I speak from my heart as to an old
acquaintance.''

``Oh, don't mention it, Count! I quite understand,'' said Berg,
getting up and speaking in a muffled and guttural voice.

``Go across to our hosts: they invited you,'' added Boris.

Berg put on the cleanest of coats, without a spot or speck of
dust, stood before a looking glass and brushed the hair on his
temples upwards, in the way affected by the Emperor Alexander,
and, having assured himself from the way Rostov looked at it that
his coat had been noticed, left the room with a pleasant smile.

``Oh dear, what a beast I am!'' muttered Rostov, as he read the
letter.

``Why?''

``Oh, what a pig I am, not to have written and to have given them
such a fright! Oh, what a pig I am!'' he repeated, flushing
suddenly. ``Well, have you sent Gabriel for some wine? All right
let's have some!''

In the letter from his parents was enclosed a letter of
recommendation to Bagration which the old countess at Anna
Mikhaylovna's advice had obtained through an acquaintance and
sent to her son, asking him to take it to its destination and
make use of it.

``What nonsense! Much I need it!'' said Rostov, throwing the
letter under the table.

``Why have you thrown that away?'' asked Boris.

``It is some letter of recommendation... what the devil do I want
it for!''

``Why 'What the devil'?'' said Boris, picking it up and reading
the address. ``This letter would be of great use to you.''

``I want nothing, and I won't be anyone's adjutant.''

``Why not?'' inquired Boris.

``It's a lackey's job!''

``You are still the same dreamer, I see,'' remarked Boris,
shaking his head.

``And you're still the same diplomatist! But that's not the
point...  Come, how are you?'' asked Rostov.

``Well, as you see. So far everything's all right, but I confess
I should much like to be an adjutant and not remain at the
front.''

``Why?''

``Because when once a man starts on military service, he should
try to make as successful a career of it as possible.''

``Oh, that's it!'' said Rostov, evidently thinking of something
else.

He looked intently and inquiringly into his friend's eyes,
evidently trying in vain to find the answer to some question.

Old Gabriel brought in the wine.

``Shouldn't we now send for Berg?'' asked Boris. ``He would drink
with you.  I can't.''

``Well, send for him... and how do you get on with that German?''
asked Rostov, with a contemptuous smile.

``He is a very, very nice, honest, and pleasant fellow,''
answered Boris.

Again Rostov looked intently into Boris' eyes and sighed. Berg
returned, and over the bottle of wine conversation between the
three officers became animated. The Guardsmen told Rostov of
their march and how they had been made much of in Russia, Poland,
and abroad. They spoke of the sayings and doings of their
commander, the Grand Duke, and told stories of his kindness and
irascibility. Berg, as usual, kept silent when the subject did
not relate to himself, but in connection with the stories of the
Grand Duke's quick temper he related with gusto how in Galicia he
had managed to deal with the Grand Duke when the latter made a
tour of the regiments and was annoyed at the irregularity of a
movement. With a pleasant smile Berg related how the Grand Duke
had ridden up to him in a violent passion, shouting: ``Arnauts!''
(``Arnauts'' was the Tsarevich's favorite expression when he was
in a rage) and called for the company commander.

``Would you believe it, Count, I was not at all alarmed, because
I knew I was right. Without boasting, you know, I may say that I
know the Army Orders by heart and know the Regulations as well as
I do the Lord's Prayer. So, Count, there never is any negligence
in my company, and so my conscience was at ease. I came
forward...'' (Berg stood up and showed how he presented himself,
with his hand to his cap, and really it would have been difficult
for a face to express greater respect and self-complacency than
his did.) ``Well, he stormed at me, as the saying is, stormed and
stormed and stormed! It was not a matter of life but rather of
death, as the saying is. 'Albanians!' and 'devils!' and 'To
Siberia!'{}'' said Berg with a sagacious smile. ``I knew I was in
the right so I kept silent; was not that best, Count?... 'Hey,
are you dumb?' he shouted. Still I remained silent. And what do
you think, Count? The next day it was not even mentioned in the
Orders of the Day. That's what keeping one's head means. That's
the way, Count,'' said Berg, lighting his pipe and emitting rings
of smoke.

``Yes, that was fine,'' said Rostov, smiling.

But Boris noticed that he was preparing to make fun of Berg, and
skillfully changed the subject. He asked him to tell them how and
where he got his wound. This pleased Rostov and he began talking
about it, and as he went on became more and more animated. He
told them of his Schon Grabern affair, just as those who have
taken part in a battle generally do describe it, that is, as they
would like it to have been, as they have heard it described by
others, and as sounds well, but not at all as it really
was. Rostov was a truthful young man and would on no account have
told a deliberate lie. He began his story meaning to tell
everything just as it happened, but imperceptibly, involuntarily,
and inevitably he lapsed into falsehood. If he had told the truth
to his hearers---who like himself had often heard stories of
attacks and had formed a definite idea of what an attack was and
were expecting to hear just such a story---they would either not
have believed him or, still worse, would have thought that Rostov
was himself to blame since what generally happens to the
narrators of cavalry attacks had not happened to him. He could
not tell them simply that everyone went at a trot and that he
fell off his horse and sprained his arm and then ran as hard as
he could from a Frenchman into the wood. Besides, to tell
everything as it really happened, it would have been necessary to
make an effort of will to tell only what happened. It is very
difficult to tell the truth, and young people are rarely capable
of it. His hearers expected a story of how beside himself and all
aflame with excitement, he had flown like a storm at the square,
cut his way in, slashed right and left, how his saber had tasted
flesh and he had fallen exhausted, and so on. And so he told them
all that.

In the middle of his story, just as he was saying: ``You cannot
imagine what a strange frenzy one experiences during an attack,''
Prince Andrew, whom Boris was expecting, entered the room. Prince
Andrew, who liked to help young men, was flattered by being asked
for his assistance and being well disposed toward Boris, who had
managed to please him the day before, he wished to do what the
young man wanted. Having been sent with papers from Kutuzov to
the Tsarevich, he looked in on Boris, hoping to find him
alone. When he came in and saw an hussar of the line recounting
his military exploits (Prince Andrew could not endure that sort
of man), he gave Boris a pleasant smile, frowned as with
half-closed eyes he looked at Rostov, bowed slightly and wearily,
and sat down languidly on the sofa: he felt it unpleasant to have
dropped in on bad company.  Rostov flushed up on noticing this,
but he did not care, this was a mere stranger. Glancing, however,
at Boris, he saw that he too seemed ashamed of the hussar of the
line.

In spite of Prince Andrew's disagreeable, ironical tone, in spite
of the contempt with which Rostov, from his fighting army point
of view, regarded all these little adjutants on the staff of whom
the newcomer was evidently one, Rostov felt confused, blushed,
and became silent.  Boris inquired what news there might be on
the staff, and what, without indiscretion, one might ask about
our plans.

``We shall probably advance,'' replied Bolkonski, evidently
reluctant to say more in the presence of a stranger.

Berg took the opportunity to ask, with great politeness, whether,
as was rumored, the allowance of forage money to captains of
companies would be doubled. To this Prince Andrew answered with a
smile that he could give no opinion on such an important
government order, and Berg laughed gaily.

``As to your business,'' Prince Andrew continued, addressing
Boris, ``we will talk of it later'' (and he looked round at
Rostov). ``Come to me after the review and we will do what is
possible.''

And, having glanced round the room, Prince Andrew turned to
Rostov, whose state of unconquerable childish embarrassment now
changing to anger he did not condescend to notice, and said: ``I
think you were talking of the Schon Grabern affair? Were you
there?''

``I was there,'' said Rostov angrily, as if intending to insult
the aide-de-camp.

Bolkonski noticed the hussar's state of mind, and it amused
him. With a slightly contemptuous smile, he said: ``Yes, there
are many stories now told about that affair!''

``Yes, stories!'' repeated Rostov loudly, looking with eyes
suddenly grown furious, now at Boris, now at Bolkonski. ``Yes,
many stories! But our stories are the stories of men who have
been under the enemy's fire! Our stories have some weight, not
like the stories of those fellows on the staff who get rewards
without doing anything!''

``Of whom you imagine me to be one?'' said Prince Andrew, with a
quiet and particularly amiable smile.

A strange feeling of exasperation and yet of respect for this
man's self-possession mingled at that moment in Rostov's soul.

``I am not talking about you,'' he said, ``I don't know you and,
frankly, I don't want to. I am speaking of the staff in
general.''

``And I will tell you this,'' Prince Andrew interrupted in a tone
of quiet authority, ``you wish to insult me, and I am ready to
agree with you that it would be very easy to do so if you haven't
sufficient self-respect, but admit that the time and place are
very badly chosen. In a day or two we shall all have to take part
in a greater and more serious duel, and besides, Drubetskoy, who
says he is an old friend of yours, is not at all to blame that my
face has the misfortune to displease you. However,'' he added
rising, ``you know my name and where to find me, but don't forget
that I do not regard either myself or you as having been at all
insulted, and as a man older than you, my advice is to let the
matter drop. Well then, on Friday after the review I shall expect
you, Drubetskoy. Au revoir!'' exclaimed Prince Andrew, and with a
bow to them both he went out.

Only when Prince Andrew was gone did Rostov think of what he
ought to have said. And he was still more angry at having omitted
to say it. He ordered his horse at once and, coldly taking leave
of Boris, rode home.  Should he go to headquarters next day and
challenge that affected adjutant, or really let the matter drop,
was the question that worried him all the way. He thought angrily
of the pleasure he would have at seeing the fright of that small
and frail but proud man when covered by his pistol, and then he
felt with surprise that of all the men he knew there was none he
would so much like to have for a friend as that very adjutant
whom he so hated.

% % % % % % % % % % % % % % % % % % % % % % % % % % % % % % % % %
% % % % % % % % % % % % % % % % % % % % % % % % % % % % % % % % %
% % % % % % % % % % % % % % % % % % % % % % % % % % % % % % % % %
% % % % % % % % % % % % % % % % % % % % % % % % % % % % % % % % %
% % % % % % % % % % % % % % % % % % % % % % % % % % % % % % % % %
% % % % % % % % % % % % % % % % % % % % % % % % % % % % % % % % %
% % % % % % % % % % % % % % % % % % % % % % % % % % % % % % % % %
% % % % % % % % % % % % % % % % % % % % % % % % % % % % % % % % %
% % % % % % % % % % % % % % % % % % % % % % % % % % % % % % % % %
% % % % % % % % % % % % % % % % % % % % % % % % % % % % % % % % %
% % % % % % % % % % % % % % % % % % % % % % % % % % % % % % % % %
% % % % % % % % % % % % % % % % % % % % % % % % % % % % % %

\chapter*{Chapter VIII}
\ifaudio     
\marginpar{
\href{http://ia800208.us.archive.org/14/items/war_and_peace_03_0712_librivox/war_and_peace_03_08_tolstoy_64kb.mp3}{Audio}} 
\fi

\lettrine[lines=2, loversize=0.3, lraise=0]{\initfamily T}{he}
day after Rostov had been to see Boris, a review was held of
the Austrian and Russian troops, both those freshly arrived from
Russia and those who had been campaigning under Kutuzov. The two
Emperors, the Russian with his heir the Tsarevich, and the
Austrian with the Archduke, inspected the allied army of eighty
thousand men.

From early morning the smart clean troops were on the move,
forming up on the field before the fortress. Now thousands of
feet and bayonets moved and halted at the officers' command,
turned with banners flying, formed up at intervals, and wheeled
round other similar masses of infantry in different uniforms; now
was heard the rhythmic beat of hoofs and the jingling of showy
cavalry in blue, red, and green braided uniforms, with smartly
dressed bandsmen in front mounted on black, roan, or gray horses;
then again, spreading out with the brazen clatter of the polished
shining cannon that quivered on the gun carriages and with the
smell of linstocks, came the artillery which crawled between the
infantry and cavalry and took up its appointed position. Not only
the generals in full parade uniforms, with their thin or thick
waists drawn in to the utmost, their red necks squeezed into
their stiff collars, and wearing scarves and all their
decorations, not only the elegant, pomaded officers, but every
soldier with his freshly washed and shaven face and his weapons
clean and polished to the utmost, and every horse groomed till
its coat shone like satin and every hair of its wetted mane lay
smooth---felt that no small matter was happening, but an
important and solemn affair. Every general and every soldier was
conscious of his own insignificance, aware of being but a drop in
that ocean of men, and yet at the same time was conscious of his
strength as a part of that enormous whole.

From early morning strenuous activities and efforts had begun and
by ten o'clock all had been brought into due order. The ranks
were drawn up on the vast field. The whole army was extended in
three lines: the cavalry in front, behind it the artillery, and
behind that again the infantry.

A space like a street was left between each two lines of
troops. The three parts of that army were sharply distinguished:
Kutuzov's fighting army (with the Pavlograds on the right flank
of the front); those recently arrived from Russia, both Guards
and regiments of the line; and the Austrian troops. But they all
stood in the same lines, under one command, and in a like order.

Like wind over leaves ran an excited whisper: ``They're coming!
They're coming!'' Alarmed voices were heard, and a stir of final
preparation swept over all the troops.

From the direction of Olmutz in front of them, a group was seen
approaching. And at that moment, though the day was still, a
light gust of wind blowing over the army slightly stirred the
streamers on the lances and the unfolded standards fluttered
against their staffs. It looked as if by that slight motion the
army itself was expressing its joy at the approach of the
Emperors. One voice was heard shouting: ``Eyes front!'' Then,
like the crowing of cocks at sunrise, this was repeated by others
from various sides and all became silent.

In the deathlike stillness only the tramp of horses was
heard. This was the Emperors' suites. The Emperors rode up to the
flank, and the trumpets of the first cavalry regiment played the
general march. It seemed as though not the trumpeters were
playing, but as if the army itself, rejoicing at the Emperors'
approach, had naturally burst into music. Amid these sounds, only
the youthful kindly voice of the Emperor Alexander was clearly
heard. He gave the words of greeting, and the first regiment
roared ``Hurrah!'' so deafeningly, continuously, and joyfully
that the men themselves were awed by their multitude and the
immensity of the power they constituted.

Rostov, standing in the front lines of Kutuzov's army which the
Tsar approached first, experienced the same feeling as every
other man in that army: a feeling of self-forgetfulness, a proud
consciousness of might, and a passionate attraction to him who
was the cause of this triumph.

He felt that at a single word from that man all this vast mass
(and he himself an insignificant atom in it) would go through
fire and water, commit crime, die, or perform deeds of highest
heroism, and so he could not but tremble and his heart stand
still at the imminence of that word.

``Hurrah! Hurrah! Hurrah!'' thundered from all sides, one
regiment after another greeting the Tsar with the strains of the
march, and then ``Hurrah!''... Then the general march, and again
``Hurrah! Hurrah!'' growing ever stronger and fuller and merging
into a deafening roar.

Till the Tsar reached it, each regiment in its silence and
immobility seemed like a lifeless body, but as soon as he came up
it became alive, its thunder joining the roar of the whole line
along which he had already passed. Through the terrible and
deafening roar of those voices, amid the square masses of troops
standing motionless as if turned to stone, hundreds of riders
composing the suites moved carelessly but symmetrically and above
all freely, and in front of them two men---the Emperors. Upon
them the undivided, tensely passionate attention of that whole
mass of men was concentrated.

The handsome young Emperor Alexander, in the uniform of the Horse
Guards, wearing a cocked hat with its peaks front and back, with
his pleasant face and resonant though not loud voice, attracted
everyone's attention.

Rostov was not far from the trumpeters, and with his keen sight
had recognized the Tsar and watched his approach. When he was
within twenty paces, and Nicholas could clearly distinguish every
detail of his handsome, happy young face, he experienced a
feeling of tenderness and ecstasy such as he had never before
known. Every trait and every movement of the Tsar's seemed to him
enchanting.

Stopping in front of the Pavlograds, the Tsar said something in
French to the Austrian Emperor and smiled.

Seeing that smile, Rostov involuntarily smiled himself and felt a
still stronger flow of love for his sovereign. He longed to show
that love in some way and knowing that this was impossible was
ready to cry. The Tsar called the colonel of the regiment and
said a few words to him.

``Oh God, what would happen to me if the Emperor spoke to me?''
thought Rostov. ``I should die of happiness!''

The Tsar addressed the officers also: ``I thank you all,
gentlemen, I thank you with my whole heart.'' To Rostov every
word sounded like a voice from heaven. How gladly would he have
died at once for his Tsar!

``You have earned the St. George's standards and will be worthy
of them.''

``Oh, to die, to die for him,'' thought Rostov.

The Tsar said something more which Rostov did not hear, and the
soldiers, straining their lungs, shouted ``Hurrah!''

Rostov too, bending over his saddle, shouted ``Hurrah!'' with all
his might, feeling that he would like to injure himself by that
shout, if only to express his rapture fully.

The Tsar stopped a few minutes in front of the hussars as if
undecided.

``How can the Emperor be undecided?'' thought Rostov, but then
even this indecision appeared to him majestic and enchanting,
like everything else the Tsar did.

That hesitation lasted only an instant. The Tsar's foot, in the
narrow pointed boot then fashionable, touched the groin of the
bobtailed bay mare he rode, his hand in a white glove gathered up
the reins, and he moved off accompanied by an irregularly swaying
sea of aides-de-camp.  Farther and farther he rode away, stopping
at other regiments, till at last only his white plumes were
visible to Rostov from amid the suites that surrounded the
Emperors.

Among the gentlemen of the suite, Rostov noticed Bolkonski,
sitting his horse indolently and carelessly. Rostov recalled
their quarrel of yesterday and the question presented itself
whether he ought or ought not to challenge Bolkonski. ``Of course
not!'' he now thought. ``Is it worth thinking or speaking of it
at such a moment? At a time of such love, such rapture, and such
self-sacrifice, what do any of our quarrels and affronts matter?
I love and forgive everybody now.''

When the Emperor had passed nearly all the regiments, the troops
began a ceremonial march past him, and Rostov on Bedouin,
recently purchased from Denisov, rode past too, at the rear of
his squadron---that is, alone and in full view of the Emperor.

Before he reached him, Rostov, who was a splendid horseman,
spurred Bedouin twice and successfully put him to the showy trot
in which the animal went when excited. Bending his foaming muzzle
to his chest, his tail extended, Bedouin, as if also conscious of
the Emperor's eye upon him, passed splendidly, lifting his feet
with a high and graceful action, as if flying through the air
without touching the ground.

Rostov himself, his legs well back and his stomach drawn in and
feeling himself one with his horse, rode past the Emperor with a
frowning but blissful face ``like a vewy devil,'' as Denisov
expressed it.

``Fine fellows, the Pavlograds!'' remarked the Emperor.

``My God, how happy I should be if he ordered me to leap into the
fire this instant!'' thought Rostov.

When the review was over, the newly arrived officers, and also
Kutuzov's, collected in groups and began to talk about the
awards, about the Austrians and their uniforms, about their
lines, about Bonaparte, and how badly the latter would fare now,
especially if the Essen corps arrived and Prussia took our side.

But the talk in every group was chiefly about the Emperor
Alexander. His every word and movement was described with
ecstasy.

They all had but one wish: to advance as soon as possible against
the enemy under the Emperor's command. Commanded by the Emperor
himself they could not fail to vanquish anyone, be it whom it
might: so thought Rostov and most of the officers after the
review.

All were then more confident of victory than the winning of two
battles would have made them.

% % % % % % % % % % % % % % % % % % % % % % % % % % % % % % % % %
% % % % % % % % % % % % % % % % % % % % % % % % % % % % % % % % %
% % % % % % % % % % % % % % % % % % % % % % % % % % % % % % % % %
% % % % % % % % % % % % % % % % % % % % % % % % % % % % % % % % %
% % % % % % % % % % % % % % % % % % % % % % % % % % % % % % % % %
% % % % % % % % % % % % % % % % % % % % % % % % % % % % % % % % %
% % % % % % % % % % % % % % % % % % % % % % % % % % % % % % % % %
% % % % % % % % % % % % % % % % % % % % % % % % % % % % % % % % %
% % % % % % % % % % % % % % % % % % % % % % % % % % % % % % % % %
% % % % % % % % % % % % % % % % % % % % % % % % % % % % % % % % %
% % % % % % % % % % % % % % % % % % % % % % % % % % % % % % % % %
% % % % % % % % % % % % % % % % % % % % % % % % % % % % % %

\chapter*{Chapter IX}
\ifaudio     
\marginpar{
\href{http://ia800208.us.archive.org/14/items/war_and_peace_03_0712_librivox/war_and_peace_03_09_tolstoy_64kb.mp3}{Audio}} 
\fi

\lettrine[lines=2, loversize=0.3, lraise=0]{\initfamily T}{he}
 day after the review, Boris, in his best uniform and with his
comrade Berg's best wishes for success, rode to Olmutz to see
Bolkonski, wishing to profit by his friendliness and obtain for
himself the best post he could---preferably that of adjutant to
some important personage, a position in the army which seemed to
him most attractive. ``It is all very well for Rostov, whose
father sends him ten thousand rubles at a time, to talk about not
wishing to cringe to anybody and not be anyone's lackey, but I
who have nothing but my brains have to make a career and must not
miss opportunities, but must avail myself of them!'' he
reflected.

He did not find Prince Andrew in Olmutz that day, but the
appearance of the town where the headquarters and the diplomatic
corps were stationed and the two Emperors were living with their
suites, households, and courts only strengthened his desire to
belong to that higher world.

He knew no one, and despite his smart Guardsman's uniform, all
these exalted personages passing in the streets in their elegant
carriages with their plumes, ribbons, and medals, both courtiers
and military men, seemed so immeasurably above him, an
insignificant officer of the Guards, that they not only did not
wish to, but simply could not, be aware of his existence. At the
quarters of the commander-in-chief, Kutuzov, where he inquired
for Bolkonski, all the adjutants and even the orderlies looked at
him as if they wished to impress on him that a great many
officers like him were always coming there and that everybody was
heartily sick of them. In spite of this, or rather because of it,
next day, November 15, after dinner he again went to Olmutz and,
entering the house occupied by Kutuzov, asked for
Bolkonski. Prince Andrew was in and Boris was shown into a large
hall probably formerly used for dancing, but in which five beds
now stood, and furniture of various kinds: a table, chairs, and a
clavichord. One adjutant, nearest the door, was sitting at the
table in a Persian dressing gown, writing. Another, the red,
stout Nesvitski, lay on a bed with his arms under his head,
laughing with an officer who had sat down beside him. A third was
playing a Viennese waltz on the clavichord, while a fourth, lying
on the clavichord, sang the tune. Bolkonski was not there. None
of these gentlemen changed his position on seeing Boris. The one
who was writing and whom Boris addressed turned round crossly and
told him Bolkonski was on duty and that he should go through the
door on the left into the reception room if he wished to see
him. Boris thanked him and went to the reception room, where he
found some ten officers and generals.

When he entered, Prince Andrew, his eyes drooping contemptuously
(with that peculiar expression of polite weariness which plainly
says, ``If it were not my duty I would not talk to you for a
moment''), was listening to an old Russian general with
decorations, who stood very erect, almost on tiptoe, with a
soldier's obsequious expression on his purple face, reporting
something.

``Very well, then, be so good as to wait,'' said Prince Andrew to
the general, in Russian, speaking with the French intonation he
affected when he wished to speak contemptuously, and noticing
Boris, Prince Andrew, paying no more heed to the general who ran
after him imploring him to hear something more, nodded and turned
to him with a cheerful smile.

At that moment Boris clearly realized what he had before
surmised, that in the army, besides the subordination and
discipline prescribed in the military code, which he and the
others knew in the regiment, there was another, more important,
subordination, which made this tight-laced, purple-faced general
wait respectfully while Captain Prince Andrew, for his own
pleasure, chose to chat with Lieutenant Drubetskoy. More than
ever was Boris resolved to serve in future not according to the
written code, but under this unwritten law. He felt now that
merely by having been recommended to Prince Andrew he had already
risen above the general who at the front had the power to
annihilate him, a lieutenant of the Guards. Prince Andrew came up
to him and took his hand.

``I am very sorry you did not find me in yesterday. I was fussing
about with Germans all day. We went with Weyrother to survey the
dispositions.  When Germans start being accurate, there's no end
to it!''

Boris smiled, as if he understood what Prince Andrew was alluding
to as something generally known. But it was the first time he had
heard Weyrother's name, or even the term ``dispositions.''

``Well, my dear fellow, so you still want to be an adjutant? I
have been thinking about you.''

``Yes, I was thinking''---for some reason Boris could not help
blushing---``of asking the commander-in-chief. He has had a
letter from Prince Kuragin about me. I only wanted to ask because
I fear the Guards won't be in action,'' he added as if in
apology.

``All right, all right. We'll talk it over,'' replied Prince
Andrew. ``Only let me report this gentleman's business, and I
shall be at your disposal.''

While Prince Andrew went to report about the purple-faced
general, that gentleman---evidently not sharing Boris' conception
of the advantages of the unwritten code of subordination---looked
so fixedly at the presumptuous lieutenant who had prevented his
finishing what he had to say to the adjutant that Boris felt
uncomfortable. He turned away and waited impatiently for Prince
Andrew's return from the commander-in-chief's room.

``You see, my dear fellow, I have been thinking about you,'' said
Prince Andrew when they had gone into the large room where the
clavichord was.  ``It's no use your going to the
commander-in-chief. He would say a lot of pleasant things, ask
you to dinner'' (``That would not be bad as regards the unwritten
code,'' thought Boris), ``but nothing more would come of it.
There will soon be a battalion of us aides-de-camp and adjutants!
But this is what we'll do: I have a good friend, an adjutant
general and an excellent fellow, Prince Dolgorukov; and though
you may not know it, the fact is that now Kutuzov with his staff
and all of us count for nothing.  Everything is now centered
round the Emperor. So we will go to Dolgorukov; I have to go
there anyhow and I have already spoken to him about you. We shall
see whether he cannot attach you to himself or find a place for
you somewhere nearer the sun.''

Prince Andrew always became specially keen when he had to guide a
young man and help him to worldly success. Under cover of
obtaining help of this kind for another, which from pride he
would never accept for himself, he kept in touch with the circle
which confers success and which attracted him. He very readily
took up Boris' cause and went with him to Dolgorukov.

It was late in the evening when they entered the palace at Olmutz
occupied by the Emperors and their retinues.

That same day a council of war had been held in which all the
members of the Hofkriegsrath and both Emperors took part. At that
council, contrary to the views of the old generals Kutuzov and
Prince Schwartzenberg, it had been decided to advance immediately
and give battle to Bonaparte.  The council of war was just over
when Prince Andrew accompanied by Boris arrived at the palace to
find Dolgorukov. Everyone at headquarters was still under the
spell of the day's council, at which the party of the young had
triumphed. The voices of those who counseled delay and advised
waiting for something else before advancing had been so
completely silenced and their arguments confuted by such
conclusive evidence of the advantages of attacking that what had
been discussed at the council---the coming battle and the victory
that would certainly result from it---no longer seemed to be in
the future but in the past. All the advantages were on our
side. Our enormous forces, undoubtedly superior to Napoleon's,
were concentrated in one place, the troops inspired by the
Emperors' presence were eager for action. The strategic position
where the operations would take place was familiar in all its
details to the Austrian General Weyrother: a lucky accident had
ordained that the Austrian army should maneuver the previous year
on the very fields where the French had now to be fought; the
adjacent locality was known and shown in every detail on the
maps, and Bonaparte, evidently weakened, was undertaking nothing.

Dolgorukov, one of the warmest advocates of an attack, had just
returned from the council, tired and exhausted but eager and
proud of the victory that had been gained. Prince Andrew
introduced his protege, but Prince Dolgorukov politely and firmly
pressing his hand said nothing to Boris and, evidently unable to
suppress the thoughts which were uppermost in his mind at that
moment, addressed Prince Andrew in French.

``Ah, my dear fellow, what a battle we have gained! God grant
that the one that will result from it will be as victorious!
However, dear fellow,'' he said abruptly and eagerly, ``I must
confess to having been unjust to the Austrians and especially to
Weyrother. What exactitude, what minuteness, what knowledge of
the locality, what foresight for every eventuality, every
possibility even to the smallest detail! No, my dear fellow, no
conditions better than our present ones could have been
devised. This combination of Austrian precision with Russian
valor---what more could be wished for?''

``So the attack is definitely resolved on?'' asked Bolkonski.

``And do you know, my dear fellow, it seems to me that Bonaparte
has decidedly lost bearings, you know that a letter was received
from him today for the Emperor.'' Dolgorukov smiled
significantly.

``Is that so? And what did he say?'' inquired Bolkonski.

``What can he say? Tra-di-ri-di-ra and so on... merely to gain
time. I tell you he is in our hands, that's certain! But what was
most amusing,'' he continued, with a sudden, good-natured laugh,
``was that we could not think how to address the reply! If not as
'Consul' and of course not as 'Emperor,' it seemed to me it
should be to 'General Bonaparte.'{}''

``But between not recognizing him as Emperor and calling him
General Bonaparte, there is a difference,'' remarked Bolkonski.

``That's just it,'' interrupted Dolgorukov quickly,
laughing. ``You know Bilibin---he's a very clever fellow. He
suggested addressing him as 'Usurper and Enemy of Mankind.'{}''

Dolgorukov laughed merrily.

``Only that?'' said Bolkonski.

``All the same, it was Bilibin who found a suitable form for the
address.  He is a wise and clever fellow.''

``What was it?''

``To the Head of the French Government... Au chef du gouvernement
francais,'' said Dolgorukov, with grave satisfaction. ``Good,
wasn't it?''

``Yes, but he will dislike it extremely,'' said Bolkonski.

``Oh yes, very much! My brother knows him, he's dined with
him---the present Emperor---more than once in Paris, and tells me
he never met a more cunning or subtle diplomatist---you know, a
combination of French adroitness and Italian play-acting! Do you
know the tale about him and Count Markov? Count Markov was the
only man who knew how to handle him.  You know the story of the
handkerchief? It is delightful!''

And the talkative Dolgorukov, turning now to Boris, now to Prince
Andrew, told how Bonaparte wishing to test Markov, our
ambassador, purposely dropped a handkerchief in front of him and
stood looking at Markov, probably expecting Markov to pick it up
for him, and how Markov immediately dropped his own beside it and
picked it up without touching Bonaparte's.

``Delightful!'' said Bolkonski. ``But I have come to you, Prince,
as a petitioner on behalf of this young man. You see...'' but
before Prince Andrew could finish, an aide-de-camp came in to
summon Dolgorukov to the Emperor.

``Oh, what a nuisance,'' said Dolgorukov, getting up hurriedly
and pressing the hands of Prince Andrew and Boris. ``You know I
should be very glad to do all in my power both for you and for
this dear young man.'' Again he pressed the hand of the latter
with an expression of good-natured, sincere, and animated
levity. ``But you see... another time!''

Boris was excited by the thought of being so close to the higher
powers as he felt himself to be at that moment. He was conscious
that here he was in contact with the springs that set in motion
the enormous movements of the mass of which in his regiment he
felt himself a tiny, obedient, and insignificant atom. They
followed Prince Dolgorukov out into the corridor and met---coming
out of the door of the Emperor's room by which Dolgorukov had
entered---a short man in civilian clothes with a clever face and
sharply projecting jaw which, without spoiling his face, gave him
a peculiar vivacity and shiftiness of expression. This short man
nodded to Dolgorukov as to an intimate friend and stared at
Prince Andrew with cool intensity, walking straight toward him
and evidently expecting him to bow or to step out of his
way. Prince Andrew did neither: a look of animosity appeared on
his face and the other turned away and went down the side of the
corridor.

``Who was that?'' asked Boris.

``He is one of the most remarkable, but to me most unpleasant of
men---the Minister of Foreign Affairs, Prince Adam
Czartoryski... It is such men as he who decide the fate of
nations,'' added Bolkonski with a sigh he could not suppress, as
they passed out of the palace.

Next day, the army began its campaign, and up to the very battle
of Austerlitz, Boris was unable to see either Prince Andrew or
Dolgorukov again and remained for a while with the Ismaylov
regiment.

% % % % % % % % % % % % % % % % % % % % % % % % % % % % % % % % %
% % % % % % % % % % % % % % % % % % % % % % % % % % % % % % % % %
% % % % % % % % % % % % % % % % % % % % % % % % % % % % % % % % %
% % % % % % % % % % % % % % % % % % % % % % % % % % % % % % % % %
% % % % % % % % % % % % % % % % % % % % % % % % % % % % % % % % %
% % % % % % % % % % % % % % % % % % % % % % % % % % % % % % % % %
% % % % % % % % % % % % % % % % % % % % % % % % % % % % % % % % %
% % % % % % % % % % % % % % % % % % % % % % % % % % % % % % % % %
% % % % % % % % % % % % % % % % % % % % % % % % % % % % % % % % %
% % % % % % % % % % % % % % % % % % % % % % % % % % % % % % % % %
% % % % % % % % % % % % % % % % % % % % % % % % % % % % % % % % %
% % % % % % % % % % % % % % % % % % % % % % % % % % % % % %

\chapter*{Chapter X}
\ifaudio     
\marginpar{
\href{http://ia800208.us.archive.org/14/items/war_and_peace_03_0712_librivox/war_and_peace_03_10_tolstoy_64kb.mp3}{Audio}} 
\fi

\lettrine[lines=2, loversize=0.3, lraise=0]{\initfamily A}{t}
dawn on the sixteenth of November, Denisov's squadron, in
which Nicholas Rostov served and which was in Prince Bagration's
detachment, moved from the place where it had spent the night,
advancing into action as arranged, and after going behind other
columns for about two thirds of a mile was stopped on the
highroad. Rostov saw the Cossacks and then the first and second
squadrons of hussars and infantry battalions and artillery pass
by and go forward and then Generals Bagration and Dolgorukov ride
past with their adjutants. All the fear before action which he
had experienced as previously, all the inner struggle to conquer
that fear, all his dreams of distinguishing himself as a true
hussar in this battle, had been wasted. Their squadron remained
in reserve and Nicholas Rostov spent that day in a dull and
wretched mood.  At nine in the morning, he heard firing in front
and shouts of hurrah, and saw wounded being brought back (there
were not many of them), and at last he saw how a whole detachment
of French cavalry was brought in, convoyed by a sotnya of
Cossacks. Evidently the affair was over and, though not big, had
been a successful engagement. The men and officers returning
spoke of a brilliant victory, of the occupation of the town of
Wischau and the capture of a whole French squadron. The day was
bright and sunny after a sharp night frost, and the cheerful
glitter of that autumn day was in keeping with the news of
victory which was conveyed, not only by the tales of those who
had taken part in it, but also by the joyful expression on the
faces of soldiers, officers, generals, and adjutants, as they
passed Rostov going or coming. And Nicholas, who had vainly
suffered all the dread that precedes a battle and had spent that
happy day in inactivity, was all the more depressed.

``Come here, Wostov. Let's dwink to dwown our gwief!'' shouted
Denisov, who had settled down by the roadside with a flask and
some food.

The officers gathered round Denisov's canteen, eating and
talking.

``There! They are bringing another!'' cried one of the officers,
indicating a captive French dragoon who was being brought in on
foot by two Cossacks.

One of them was leading by the bridle a fine large French horse
he had taken from the prisoner.

``Sell us that horse!'' Denisov called out to the Cossacks.

``If you like, your honor!''

The officers got up and stood round the Cossacks and their
prisoner. The French dragoon was a young Alsatian who spoke
French with a German accent. He was breathless with agitation,
his face was red, and when he heard some French spoken he at once
began speaking to the officers, addressing first one, then
another. He said he would not have been taken, it was not his
fault but the corporal's who had sent him to seize some
horsecloths, though he had told him the Russians were there. And
at every word he added: ``But don't hurt my little horse!'' and
stroked the animal. It was plain that he did not quite grasp
where he was. Now he excused himself for having been taken
prisoner and now, imagining himself before his own officers,
insisted on his soldierly discipline and zeal in the service. He
brought with him into our rearguard all the freshness of
atmosphere of the French army, which was so alien to us.

The Cossacks sold the horse for two gold pieces, and Rostov,
being the richest of the officers now that he had received his
money, bought it.

``But don't hurt my little horse!'' said the Alsatian
good-naturedly to Rostov when the animal was handed over to the
hussar.

Rostov smilingly reassured the dragoon and gave him money.

``Alley! Alley!'' said the Cossack, touching the prisoner's arm
to make him go on.

``The Emperor! The Emperor!'' was suddenly heard among the
hussars.

All began to run and bustle, and Rostov saw coming up the road
behind him several riders with white plumes in their hats. In a
moment everyone was in his place, waiting.

Rostov did not know or remember how he ran to his place and
mounted.  Instantly his regret at not having been in action and
his dejected mood amid people of whom he was weary had gone,
instantly every thought of himself had vanished. He was filled
with happiness at his nearness to the Emperor. He felt that this
nearness by itself made up to him for the day he had lost. He was
happy as a lover when the longed-for moment of meeting
arrives. Not daring to look round and without looking round, he
was ecstatically conscious of his approach. He felt it not only
from the sound of the hoofs of the approaching cavalcade, but
because as he drew near everything grew brighter, more joyful,
more significant, and more festive around him. Nearer and nearer
to Rostov came that sun shedding beams of mild and majestic light
around, and already he felt himself enveloped in those beams, he
heard his voice, that kindly, calm, and majestic voice that was
yet so simple! And as if in accord with Rostov's feeling, there
was a deathly stillness amid which was heard the Emperor's voice.

``The Pavlograd hussars?'' he inquired.

``The reserves, sire!'' replied a voice, a very human one
compared to that which had said: ``The Pavlograd hussars?''

The Emperor drew level with Rostov and halted. Alexander's face
was even more beautiful than it had been three days before at the
review. It shone with such gaiety and youth, such innocent youth,
that it suggested the liveliness of a fourteen-year-old boy, and
yet it was the face of the majestic Emperor. Casually, while
surveying the squadron, the Emperor's eyes met Rostov's and
rested on them for not more than two seconds. Whether or no the
Emperor understood what was going on in Rostov's soul (it seemed
to Rostov that he understood everything), at any rate his
light-blue eyes gazed for about two seconds into Rostov's face. A
gentle, mild light poured from them. Then all at once he raised
his eyebrows, abruptly touched his horse with his left foot, and
galloped on.

The younger Emperor could not restrain his wish to be present at
the battle and, in spite of the remonstrances of his courtiers,
at twelve o'clock left the third column with which he had been
and galloped toward the vanguard. Before he came up with the
hussars, several adjutants met him with news of the successful
result of the action.

This battle, which consisted in the capture of a French squadron,
was represented as a brilliant victory over the French, and so
the Emperor and the whole army, especially while the smoke hung
over the battlefield, believed that the French had been defeated
and were retreating against their will. A few minutes after the
Emperor had passed, the Pavlograd division was ordered to
advance. In Wischau itself, a petty German town, Rostov saw the
Emperor again. In the market place, where there had been some
rather heavy firing before the Emperor's arrival, lay several
killed and wounded soldiers whom there had not been time to
move. The Emperor, surrounded by his suite of officers and
courtiers, was riding a bobtailed chestnut mare, a different one
from that which he had ridden at the review, and bending to one
side he gracefully held a gold lorgnette to his eyes and looked
at a soldier who lay prone, with blood on his uncovered head. The
wounded soldier was so dirty, coarse, and revolting that his
proximity to the Emperor shocked Rostov. Rostov saw how the
Emperor's rather round shoulders shuddered as if a cold shiver
had run down them, how his left foot began convulsively tapping
the horse's side with the spur, and how the well-trained horse
looked round unconcerned and did not stir. An adjutant,
dismounting, lifted the soldier under the arms to place him on a
stretcher that had been brought. The soldier groaned.

``Gently, gently! Can't you do it more gently?'' said the Emperor
apparently suffering more than the dying soldier, and he rode
away.

Rostov saw tears filling the Emperor's eyes and heard him, as he
was riding away, say to Czartoryski: ``What a terrible thing war
is: what a terrible thing! Quelle terrible chose que la guerre!''

The troops of the vanguard were stationed before Wischau, within
sight of the enemy's lines, which all day long had yielded ground
to us at the least firing. The Emperor's gratitude was announced
to the vanguard, rewards were promised, and the men received a
double ration of vodka.  The campfires crackled and the soldiers'
songs resounded even more merrily than on the previous
night. Denisov celebrated his promotion to the rank of major, and
Rostov, who had already drunk enough, at the end of the feast
proposed the Emperor's health. ``Not 'our Sovereign, the
Emperor,' as they say at official dinners,'' said he, ``but the
health of our Sovereign, that good, enchanting, and great man!
Let us drink to his health and to the certain defeat of the
French!''

``If we fought before,'' he said, ``not letting the French pass,
as at Schon Grabern, what shall we not do now when he is at the
front? We will all die for him gladly! Is it not so, gentlemen?
Perhaps I am not saying it right, I have drunk a good deal---but
that is how I feel, and so do you too! To the health of Alexander
the First! Hurrah!''

``Hurrah!'' rang the enthusiastic voices of the officers.

And the old cavalry captain, Kirsten, shouted enthusiastically
and no less sincerely than the twenty-year-old Rostov.

When the officers had emptied and smashed their glasses, Kirsten
filled others and, in shirt sleeves and breeches, went glass in
hand to the soldiers' bonfires and with his long gray mustache,
his white chest showing under his open shirt, he stood in a
majestic pose in the light of the campfire, waving his uplifted
arm.

``Lads! here's to our Sovereign, the Emperor, and victory over
our enemies! Hurrah!'' he exclaimed in his dashing, old, hussar's
baritone.

The hussars crowded round and responded heartily with loud
shouts.

Late that night, when all had separated, Denisov with his short
hand patted his favorite, Rostov, on the shoulder.

``As there's no one to fall in love with on campaign, he's fallen
in love with the Tsar,'' he said.

``Denisov, don't make fun of it!'' cried Rostov. ``It is such a
lofty, beautiful feeling, such a...''

``I believe it, I believe it, fwiend, and I share and
appwove...''

``No, you don't understand!''

And Rostov got up and went wandering among the campfires,
dreaming of what happiness it would be to die---not in saving the
Emperor's life (he did not even dare to dream of that), but
simply to die before his eyes.  He really was in love with the
Tsar and the glory of the Russian arms and the hope of future
triumph. And he was not the only man to experience that feeling
during those memorable days preceding the battle of Austerlitz:
nine tenths of the men in the Russian army were then in love,
though less ecstatically, with their Tsar and the glory of the
Russian arms.

% % % % % % % % % % % % % % % % % % % % % % % % % % % % % % % % %
% % % % % % % % % % % % % % % % % % % % % % % % % % % % % % % % %
% % % % % % % % % % % % % % % % % % % % % % % % % % % % % % % % %
% % % % % % % % % % % % % % % % % % % % % % % % % % % % % % % % %
% % % % % % % % % % % % % % % % % % % % % % % % % % % % % % % % %
% % % % % % % % % % % % % % % % % % % % % % % % % % % % % % % % %
% % % % % % % % % % % % % % % % % % % % % % % % % % % % % % % % %
% % % % % % % % % % % % % % % % % % % % % % % % % % % % % % % % %
% % % % % % % % % % % % % % % % % % % % % % % % % % % % % % % % %
% % % % % % % % % % % % % % % % % % % % % % % % % % % % % % % % %
% % % % % % % % % % % % % % % % % % % % % % % % % % % % % % % % %
% % % % % % % % % % % % % % % % % % % % % % % % % % % % % %

\chapter*{Chapter XI}
\ifaudio     
\marginpar{
\href{http://ia800208.us.archive.org/14/items/war_and_peace_03_0712_librivox/war_and_peace_03_11_tolstoy_64kb.mp3}{Audio}} 
\fi

\lettrine[lines=2, loversize=0.3, lraise=0]{\initfamily T}{he}
next day the Emperor stopped at Wischau, and Villier, his
physician, was repeatedly summoned to see him. At headquarters
and among the troops near by the news spread that the Emperor was
unwell. He ate nothing and had slept badly that night, those
around him reported. The cause of this indisposition was the
strong impression made on his sensitive mind by the sight of the
killed and wounded.

At daybreak on the seventeenth, a French officer who had come
with a flag of truce, demanding an audience with the Russian
Emperor, was brought into Wischau from our outposts. This officer
was Savary. The Emperor had only just fallen asleep and so Savary
had to wait. At midday he was admitted to the Emperor, and an
hour later he rode off with Prince Dolgorukov to the advanced
post of the French army.

It was rumored that Savary had been sent to propose to Alexander
a meeting with Napoleon. To the joy and pride of the whole army,
a personal interview was refused, and instead of the Sovereign,
Prince Dolgorukov, the victor at Wischau, was sent with Savary to
negotiate with Napoleon if, contrary to expectations, these
negotiations were actuated by a real desire for peace.

Toward evening Dolgorukov came back, went straight to the Tsar,
and remained alone with him for a long time.

On the eighteenth and nineteenth of November, the army advanced
two days' march and the enemy's outposts after a brief
interchange of shots retreated. In the highest army circles from
midday on the nineteenth, a great, excitedly bustling activity
began which lasted till the morning of the twentieth, when the
memorable battle of Austerlitz was fought.

Till midday on the nineteenth, the activity---the eager talk,
running to and fro, and dispatching of adjutants---was confined
to the Emperor's headquarters. But on the afternoon of that day,
this activity reached Kutuzov's headquarters and the staffs of
the commanders of columns. By evening, the adjutants had spread
it to all ends and parts of the army, and in the night from the
nineteenth to the twentieth, the whole eighty thousand allied
troops rose from their bivouacs to the hum of voices, and the
army swayed and started in one enormous mass six miles long.

The concentrated activity which had begun at the Emperor's
headquarters in the morning and had started the whole movement
that followed was like the first movement of the main wheel of a
large tower clock. One wheel slowly moved, another was set in
motion, and a third, and wheels began to revolve faster and
faster, levers and cogwheels to work, chimes to play, figures to
pop out, and the hands to advance with regular motion as a result
of all that activity.

Just as in the mechanism of a clock, so in the mechanism of the
military machine, an impulse once given leads to the final
result; and just as indifferently quiescent till the moment when
motion is transmitted to them are the parts of the mechanism
which the impulse has not yet reached. Wheels creak on their
axles as the cogs engage one another and the revolving pulleys
whirr with the rapidity of their movement, but a neighboring
wheel is as quiet and motionless as though it were prepared to
remain so for a hundred years; but the moment comes when the
lever catches it and obeying the impulse that wheel begins to
creak and joins in the common motion the result and aim of which
are beyond its ken.

Just as in a clock, the result of the complicated motion of
innumerable wheels and pulleys is merely a slow and regular
movement of the hands which show the time, so the result of all
the complicated human activities of 160,000 Russians and
French---all their passions, desires, remorse, humiliations,
sufferings, outbursts of pride, fear, and enthusiasm---was only
the loss of the battle of Austerlitz, the so-called battle of the
three Emperors---that is to say, a slow movement of the hand on
the dial of human history.

Prince Andrew was on duty that day and in constant attendance on
the commander-in-chief.

At six in the evening, Kutuzov went to the Emperor's headquarters
and after staying but a short time with the Tsar went to see the
grand marshal of the court, Count Tolstoy.

Bolkonski took the opportunity to go in to get some details of
the coming action from Dolgorukov. He felt that Kutuzov was upset
and dissatisfied about something and that at headquarters they
were dissatisfied with him, and also that at the Emperor's
headquarters everyone adopted toward him the tone of men who know
something others do not know: he therefore wished to speak to
Dolgorukov.

``Well, how d'you do, my dear fellow?'' said Dolgorukov, who was
sitting at tea with Bilibin. ``The fete is for tomorrow. How is
your old fellow?  Out of sorts?''

``I won't say he is out of sorts, but I fancy he would like to be
heard.''

``But they heard him at the council of war and will hear him when
he talks sense, but to temporize and wait for something now when
Bonaparte fears nothing so much as a general battle is
impossible.''

``Yes, you have seen him?'' said Prince Andrew. ``Well, what is
Bonaparte like? How did he impress you?''

``Yes, I saw him, and am convinced that he fears nothing so much
as a general engagement,'' repeated Dolgorukov, evidently prizing
this general conclusion which he had arrived at from his
interview with Napoleon. ``If he weren't afraid of a battle why
did he ask for that interview? Why negotiate, and above all why
retreat, when to retreat is so contrary to his method of
conducting war? Believe me, he is afraid, afraid of a general
battle. His hour has come! Mark my words!''

``But tell me, what is he like, eh?'' said Prince Andrew again.

``He is a man in a gray overcoat, very anxious that I should call
him 'Your Majesty,' but who, to his chagrin, got no title from
me! That's the sort of man he is, and nothing more,'' replied
Dolgorukov, looking round at Bilibin with a smile.

``Despite my great respect for old Kutuzov,'' he continued, ``we
should be a nice set of fellows if we were to wait about and so
give him a chance to escape, or to trick us, now that we
certainly have him in our hands!  No, we mustn't forget Suvorov
and his rule---not to put yourself in a position to be attacked,
but yourself to attack. Believe me in war the energy of young men
often shows the way better than all the experience of old
Cunctators.''

``But in what position are we going to attack him? I have been at
the outposts today and it is impossible to say where his chief
forces are situated,'' said Prince Andrew.

He wished to explain to Dolgorukov a plan of attack he had
himself formed.

``Oh, that is all the same,'' Dolgorukov said quickly, and
getting up he spread a map on the table. ``All eventualities have
been foreseen. If he is standing before Brunn...''

And Prince Dolgorukov rapidly but indistinctly explained
Weyrother's plan of a flanking movement.

Prince Andrew began to reply and to state his own plan, which
might have been as good as Weyrother's, but for the disadvantage
that Weyrother's had already been approved. As soon as Prince
Andrew began to demonstrate the defects of the latter and the
merits of his own plan, Prince Dolgorukov ceased to listen to him
and gazed absent-mindedly not at the map, but at Prince Andrew's
face.

``There will be a council of war at Kutuzov's tonight, though;
you can say all this there,'' remarked Dolgorukov.

``I will do so,'' said Prince Andrew, moving away from the map.

``Whatever are you bothering about, gentlemen?'' said Bilibin,
who, till then, had listened with an amused smile to their
conversation and now was evidently ready with a joke. ``Whether
tomorrow brings victory or defeat, the glory of our Russian arms
is secure. Except your Kutuzov, there is not a single Russian in
command of a column! The commanders are: Herr General Wimpfen, le
Comte de Langeron, le Prince de Lichtenstein, le Prince, de
Hohenlohe, and finally Prishprish, and so on like all those
Polish names.''

``Be quiet, backbiter!'' said Dolgorukov. ``It is not true; there
are now two Russians, Miloradovich, and Dokhturov, and there
would be a third, Count Arakcheev, if his nerves were not too
weak.''

``However, I think General Kutuzov has come out,'' said Prince
Andrew. ``I wish you good luck and success, gentlemen!'' he added
and went out after shaking hands with Dolgorukov and Bilibin.

On the way home, Prince Andrew could not refrain from asking
Kutuzov, who was sitting silently beside him, what he thought of
tomorrow's battle.

Kutuzov looked sternly at his adjutant and, after a pause,
replied: ``I think the battle will be lost, and so I told Count
Tolstoy and asked him to tell the Emperor. What do you think he
replied? 'But, my dear general, I am engaged with rice and
cutlets, look after military matters yourself!' Yes... That was
the answer I got!''

% % % % % % % % % % % % % % % % % % % % % % % % % % % % % % % % %
% % % % % % % % % % % % % % % % % % % % % % % % % % % % % % % % %
% % % % % % % % % % % % % % % % % % % % % % % % % % % % % % % % %
% % % % % % % % % % % % % % % % % % % % % % % % % % % % % % % % %
% % % % % % % % % % % % % % % % % % % % % % % % % % % % % % % % %
% % % % % % % % % % % % % % % % % % % % % % % % % % % % % % % % %
% % % % % % % % % % % % % % % % % % % % % % % % % % % % % % % % %
% % % % % % % % % % % % % % % % % % % % % % % % % % % % % % % % %
% % % % % % % % % % % % % % % % % % % % % % % % % % % % % % % % %
% % % % % % % % % % % % % % % % % % % % % % % % % % % % % % % % %
% % % % % % % % % % % % % % % % % % % % % % % % % % % % % % % % %
% % % % % % % % % % % % % % % % % % % % % % % % % % % % % %

\chapter*{Chapter XII}
\ifaudio     
\marginpar{
\href{http://ia800208.us.archive.org/14/items/war_and_peace_03_0712_librivox/war_and_peace_03_12_tolstoy_64kb.mp3}{Audio}} 
\fi

\lettrine[lines=2, loversize=0.3, lraise=0]{\initfamily S}{hortly }
after nine o'clock that evening, Weyrother drove with his
plans to Kutuzov's quarters where the council of war was to be
held. All the commanders of columns were summoned to the
commander-in-chief's and with the exception of Prince Bagration,
who declined to come, were all there at the appointed time.

Weyrother, who was in full control of the proposed battle, by his
eagerness and briskness presented a marked contrast to the
dissatisfied and drowsy Kutuzov, who reluctantly played the part
of chairman and president of the council of war. Weyrother
evidently felt himself to be at the head of a movement that had
already become unrestrainable. He was like a horse running
downhill harnessed to a heavy cart. Whether he was pulling it or
being pushed by it he did not know, but rushed along at headlong
speed with no time to consider what this movement might lead
to. Weyrother had been twice that evening to the enemy's picket
line to reconnoiter personally, and twice to the Emperors,
Russian and Austrian, to report and explain, and to his
headquarters where he had dictated the dispositions in German,
and now, much exhausted, he arrived at Kutuzov's.

He was evidently so busy that he even forgot to be polite to the
commander in chief. He interrupted him, talked rapidly and
indistinctly, without looking at the man he was addressing, and
did not reply to questions put to him. He was bespattered with
mud and had a pitiful, weary, and distracted air, though at the
same time he was haughty and self-confident.

Kutuzov was occupying a nobleman's castle of modest dimensions
near Ostralitz. In the large drawing room which had become the
commander in chief's office were gathered Kutuzov himself,
Weyrother, and the members of the council of war. They were
drinking tea, and only awaited Prince Bagration to begin the
council. At last Bagration's orderly came with the news that the
prince could not attend. Prince Andrew came in to inform the
commander-in-chief of this and, availing himself of permission
previously given him by Kutuzov to be present at the council, he
remained in the room.

``Since Prince Bagration is not coming, we may begin,'' said
Weyrother, hurriedly rising from his seat and going up to the
table on which an enormous map of the environs of Brunn was
spread out.

Kutuzov, with his uniform unbuttoned so that his fat neck bulged
over his collar as if escaping, was sitting almost asleep in a
low chair, with his podgy old hands resting symmetrically on its
arms. At the sound of Weyrother's voice, he opened his one eye
with an effort.

``Yes, yes, if you please! It is already late,'' said he, and
nodding his head he let it droop and again closed his eye.

If at first the members of the council thought that Kutuzov was
pretending to sleep, the sounds his nose emitted during the
reading that followed proved that the commander-in-chief at that
moment was absorbed by a far more serious matter than a desire to
show his contempt for the dispositions or anything else---he was
engaged in satisfying the irresistible human need for sleep. He
really was asleep. Weyrother, with the gesture of a man too busy
to lose a moment, glanced at Kutuzov and, having convinced
himself that he was asleep, took up a paper and in a loud,
monotonous voice began to read out the dispositions for the
impending battle, under a heading which he also read out:

``Dispositions for an attack on the enemy position behind
Kobelnitz and Sokolnitz, November 30, 1805.''

The dispositions were very complicated and difficult. They began
as follows:

``As the enemy's left wing rests on wooded hills and his right
extends along Kobelnitz and Sokolnitz behind the ponds that are
there, while we, on the other hand, with our left wing by far
outflank his right, it is advantageous to attack the enemy's
latter wing especially if we occupy the villages of Sokolnitz and
Kobelnitz, whereby we can both fall on his flank and pursue him
over the plain between Schlappanitz and the Thuerassa forest,
avoiding the defiles of Schlappanitz and Bellowitz which cover
the enemy's front. For this object it is necessary that...  The
first column marches... The second column marches... The third
column marches...'' and so on, read Weyrother.

The generals seemed to listen reluctantly to the difficult
dispositions.  The tall, fair-haired General Buxhowden stood,
leaning his back against the wall, his eyes fixed on a burning
candle, and seemed not to listen or even to wish to be thought to
listen. Exactly opposite Weyrother, with his glistening wide-open
eyes fixed upon him and his mustache twisted upwards, sat the
ruddy Miloradovich in a military pose, his elbows turned
outwards, his hands on his knees, and his shoulders raised. He
remained stubbornly silent, gazing at Weyrother's face, and only
turned away his eyes when the Austrian chief of staff finished
reading. Then Miloradovich looked round significantly at the
other generals. But one could not tell from that significant look
whether he agreed or disagreed and was satisfied or not with the
arrangements. Next to Weyrother sat Count Langeron who, with a
subtle smile that never left his typically southern French face
during the whole time of the reading, gazed at his delicate
fingers which rapidly twirled by its corners a gold snuffbox on
which was a portrait. In the middle of one of the longest
sentences, he stopped the rotary motion of the snuffbox, raised
his head, and with inimical politeness lurking in the corners of
his thin lips interrupted Weyrother, wishing to say
something. But the Austrian general, continuing to read, frowned
angrily and jerked his elbows, as if to say: ``You can tell me
your views later, but now be so good as to look at the map and
listen.'' Langeron lifted his eyes with an expression of
perplexity, turned round to Miloradovich as if seeking an
explanation, but meeting the latter's impressive but meaningless
gaze drooped his eyes sadly and again took to twirling his
snuffbox.

``A geography lesson!'' he muttered as if to himself, but loud
enough to be heard.

Przebyszewski, with respectful but dignified politeness, held his
hand to his ear toward Weyrother, with the air of a man absorbed
in attention. Dohkturov, a little man, sat opposite Weyrother,
with an assiduous and modest mien, and stooping over the
outspread map conscientiously studied the dispositions and the
unfamiliar locality. He asked Weyrother several times to repeat
words he had not clearly heard and the difficult names of
villages. Weyrother complied and Dohkturov noted them down.

When the reading which lasted more than an hour was over,
Langeron again brought his snuffbox to rest and, without looking
at Weyrother or at anyone in particular, began to say how
difficult it was to carry out such a plan in which the enemy's
position was assumed to be known, whereas it was perhaps not
known, since the enemy was in movement.  Langeron's objections
were valid but it was obvious that their chief aim was to show
General Weyrother---who had read his dispositions with as much
self-confidence as if he were addressing school children---that
he had to do, not with fools, but with men who could teach him
something in military matters.

When the monotonous sound of Weyrother's voice ceased, Kutuzov
opened his eye as a miller wakes up when the soporific drone of
the mill wheel is interrupted. He listened to what Langeron said,
as if remarking, ``So you are still at that silly business!''
quickly closed his eye again, and let his head sink still lower.

Langeron, trying as virulently as possible to sting Weyrother's
vanity as author of the military plan, argued that Bonaparte
might easily attack instead of being attacked, and so render the
whole of this plan perfectly worthless. Weyrother met all
objections with a firm and contemptuous smile, evidently prepared
beforehand to meet all objections be they what they might.

``If he could attack us, he would have done so today,'' said he.

``So you think he is powerless?'' said Langeron.

``He has forty thousand men at most,'' replied Weyrother, with
the smile of a doctor to whom an old wife wishes to explain the
treatment of a case.

``In that case he is inviting his doom by awaiting our attack,''
said Langeron, with a subtly ironical smile, again glancing round
for support to Miloradovich who was near him.

But Miloradovich was at that moment evidently thinking of
anything rather than of what the generals were disputing about.

``Ma foi!'' said he, ``tomorrow we shall see all that on the
battlefield.''

Weyrother again gave that smile which seemed to say that to him
it was strange and ridiculous to meet objections from Russian
generals and to have to prove to them what he had not merely
convinced himself of, but had also convinced the sovereign
Emperors of.

``The enemy has quenched his fires and a continual noise is heard
from his camp,'' said he. ``What does that mean? Either he is
retreating, which is the only thing we need fear, or he is
changing his position.'' (He smiled ironically.) ``But even if he
also took up a position in the Thuerassa, he merely saves us a
great deal of trouble and all our arrangements to the minutest
detail remain the same.''

``How is that?...'' began Prince Andrew, who had for long been
waiting an opportunity to express his doubts.

Kutuzov here woke up, coughed heavily, and looked round at the
generals.

``Gentlemen, the dispositions for tomorrow---or rather for today,
for it is past midnight---cannot now be altered,'' said he. ``You
have heard them, and we shall all do our duty. But before a
battle, there is nothing more important...'' he paused, ``than to
have a good sleep.''

He moved as if to rise. The generals bowed and retired. It was
past midnight. Prince Andrew went out.

The council of war, at which Prince Andrew had not been able to
express his opinion as he had hoped to, left on him a vague and
uneasy impression. Whether Dolgorukov and Weyrother, or Kutuzov,
Langeron, and the others who did not approve of the plan of
attack, were right---he did not know. ``But was it really not
possible for Kutuzov to state his views plainly to the Emperor?
Is it possible that on account of court and personal
considerations tens of thousands of lives, and my life, my
life,'' he thought, ``must be risked?''

``Yes, it is very likely that I shall be killed tomorrow,'' he
thought.  And suddenly, at this thought of death, a whole series
of most distant, most intimate, memories rose in his imagination:
he remembered his last parting from his father and his wife; he
remembered the days when he first loved her. He thought of her
pregnancy and felt sorry for her and for himself, and in a
nervously emotional and softened mood he went out of the hut in
which he was billeted with Nesvitski and began to walk up and
down before it.

The night was foggy and through the fog the moonlight gleamed
mysteriously. ``Yes, tomorrow, tomorrow!'' he thought. ``Tomorrow
everything may be over for me! All these memories will be no
more, none of them will have any meaning for me. Tomorrow
perhaps, even certainly, I have a presentiment that for the first
time I shall have to show all I can do.'' And his fancy pictured
the battle, its loss, the concentration of fighting at one point,
and the hesitation of all the commanders. And then that happy
moment, that Toulon for which he had so long waited, presents
itself to him at last. He firmly and clearly expresses his
opinion to Kutuzov, to Weyrother, and to the Emperors. All are
struck by the justness of his views, but no one undertakes to
carry them out, so he takes a regiment, a division-stipulates
that no one is to interfere with his arrangements---leads his
division to the decisive point, and gains the victory
alone. ``But death and suffering?'' suggested another
voice. Prince Andrew, however, did not answer that voice and went
on dreaming of his triumphs. The dispositions for the next battle
are planned by him alone. Nominally he is only an adjutant on
Kutuzov's staff, but he does everything alone. The next battle is
won by him alone. Kutuzov is removed and he is
appointed... ``Well and then?'' asked the other voice. ``If
before that you are not ten times wounded, killed, or betrayed,
well... what then?...'' ``Well then,'' Prince Andrew answered
himself, ``I don't know what will happen and don't want to know,
and can't, but if I want this---want glory, want to be known to
men, want to be loved by them, it is not my fault that I want it
and want nothing but that and live only for that. Yes, for that
alone! I shall never tell anyone, but, oh God! what am I to do if
I love nothing but fame and men's esteem? Death, wounds, the loss
of family---I fear nothing. And precious and dear as many persons
are to me---father, sister, wife---those dearest to me---yet
dreadful and unnatural as it seems, I would give them all at once
for a moment of glory, of triumph over men, of love from men I
don't know and never shall know, for the love of these men
here,'' he thought, as he listened to voices in Kutuzov's
courtyard. The voices were those of the orderlies who were
packing up; one voice, probably a coachman's, was teasing
Kutuzov's old cook whom Prince Andrew knew, and who was called
Tit. He was saying, ``Tit, I say, Tit!''

``Well?'' returned the old man.

``Go, Tit, thresh a bit!'' said the wag.

``Oh, go to the devil!'' called out a voice, drowned by the
laughter of the orderlies and servants.

``All the same, I love and value nothing but triumph over them
all, I value this mystic power and glory that is floating here
above me in this mist!''

% % % % % % % % % % % % % % % % % % % % % % % % % % % % % % % % %
% % % % % % % % % % % % % % % % % % % % % % % % % % % % % % % % %
% % % % % % % % % % % % % % % % % % % % % % % % % % % % % % % % %
% % % % % % % % % % % % % % % % % % % % % % % % % % % % % % % % %
% % % % % % % % % % % % % % % % % % % % % % % % % % % % % % % % %
% % % % % % % % % % % % % % % % % % % % % % % % % % % % % % % % %
% % % % % % % % % % % % % % % % % % % % % % % % % % % % % % % % %
% % % % % % % % % % % % % % % % % % % % % % % % % % % % % % % % %
% % % % % % % % % % % % % % % % % % % % % % % % % % % % % % % % %
% % % % % % % % % % % % % % % % % % % % % % % % % % % % % % % % %
% % % % % % % % % % % % % % % % % % % % % % % % % % % % % % % % %
% % % % % % % % % % % % % % % % % % % % % % % % % % % % % %

\chapter*{Chapter XIII}
\ifaudio     
\marginpar{
\href{http://ia800208.us.archive.org/14/items/war_and_peace_03_0712_librivox/war_and_peace_03_13_tolstoy_64kb.mp3}{Audio}} 
\fi

\lettrine[lines=2, loversize=0.3, lraise=0]{\initfamily T}{hat}
same night, Rostov was with a platoon on skirmishing duty in
front of Bagration's detachment. His hussars were placed along
the line in couples and he himself rode along the line trying to
master the sleepiness that kept coming over him. An enormous
space, with our army's campfires dimly glowing in the fog, could
be seen behind him; in front of him was misty darkness. Rostov
could see nothing, peer as he would into that foggy distance: now
something gleamed gray, now there was something black, now little
lights seemed to glimmer where the enemy ought to be, now he
fancied it was only something in his own eyes. His eyes kept
closing, and in his fancy appeared---now the Emperor, now
Denisov, and now Moscow memories---and he again hurriedly opened
his eyes and saw close before him the head and ears of the horse
he was riding, and sometimes, when he came within six paces of
them, the black figures of hussars, but in the distance was still
the same misty darkness. ``Why not?... It might easily happen,''
thought Rostov, ``that the Emperor will meet me and give me an
order as he would to any other officer; he'll say: 'Go and find
out what's there.' There are many stories of his getting to know
an officer in just such a chance way and attaching him to
himself! What if he gave me a place near him? Oh, how I would
guard him, how I would tell him the truth, how I would unmask his
deceivers!''  And in order to realize vividly his love devotion
to the sovereign, Rostov pictured to himself an enemy or a
deceitful German, whom he would not only kill with pleasure but
whom he would slap in the face before the Emperor. Suddenly a
distant shout aroused him. He started and opened his eyes.

``Where am I? Oh yes, in the skirmishing line... pass and
watchword---shaft, Olmutz. What a nuisance that our squadron will
be in reserve tomorrow,'' he thought. ``I'll ask leave to go to
the front, this may be my only chance of seeing the Emperor. It
won't be long now before I am off duty. I'll take another turn
and when I get back I'll go to the general and ask him.'' He
readjusted himself in the saddle and touched up his horse to ride
once more round his hussars. It seemed to him that it was getting
lighter. To the left he saw a sloping descent lit up, and facing
it a black knoll that seemed as steep as a wall. On this knoll
there was a white patch that Rostov could not at all make out:
was it a glade in the wood lit up by the moon, or some unmelted
snow, or some white houses? He even thought something moved on
that white spot. ``I expect it's snow... that spot... a
spot---une tache,'' he thought. ``There now... it's not a
tache... Natasha... sister, black eyes... Na...  tasha... (Won't
she be surprised when I tell her how I've seen the Emperor?)
Natasha... take my sabretache...''---``Keep to the right, your
honor, there are bushes here,'' came the voice of an hussar, past
whom Rostov was riding in the act of falling asleep. Rostov
lifted his head that had sunk almost to his horse's mane and
pulled up beside the hussar. He was succumbing to irresistible,
youthful, childish drowsiness. ``But what was I thinking? I
mustn't forget. How shall I speak to the Emperor? No, that's not
it---that's tomorrow. Oh yes!  Natasha... sabretache... saber
them... Whom? The hussars... Ah, the hussars with
mustaches. Along the Tverskaya Street rode the hussar with
mustaches... I thought about him too, just opposite Guryev's
house...  Old Guryev... Oh, but Denisov's a fine fellow. But
that's all nonsense.  The chief thing is that the Emperor is
here. How he looked at me and wished to say something, but dared
not... No, it was I who dared not.  But that's nonsense, the
chief thing is not to forget the important thing I was thinking
of. Yes, Na-tasha, sabretache, oh, yes, yes! That's right!'' And
his head once more sank to his horse's neck. All at once it
seemed to him that he was being fired at. ``What? What?
What?... Cut them down! What?...'' said Rostov, waking up. At the
moment he opened his eyes he heard in front of him, where the
enemy was, the long-drawn shouts of thousands of voices. His
horse and the horse of the hussar near him pricked their ears at
these shouts. Over there, where the shouting came from, a fire
flared up and went out again, then another, and all along the
French line on the hill fires flared up and the shouting grew
louder and louder. Rostov could hear the sound of French words
but could not distinguish them. The din of many voices was too
great; all he could hear was: ``ahahah!'' and ``rrrr!''

``What's that? What do you make of it?'' said Rostov to the
hussar beside him. ``That must be the enemy's camp!''

The hussar did not reply.

``Why, don't you hear it?'' Rostov asked again, after waiting for
a reply.

``Who can tell, your honor?'' replied the hussar reluctantly.

``From the direction, it must be the enemy,'' repeated Rostov.

``It may be he or it may be nothing,'' muttered the
hussar. ``It's dark...  Steady!'' he cried to his fidgeting
horse.

Rostov's horse was also getting restive: it pawed the frozen
ground, pricking its ears at the noise and looking at the
lights. The shouting grew still louder and merged into a general
roar that only an army of several thousand men could produce. The
lights spread farther and farther, probably along the line of the
French camp. Rostov no longer wanted to sleep. The gay triumphant
shouting of the enemy army had a stimulating effect on
him. ``Vive l'Empereur! L'Empereur!'' he now heard distinctly.

``They can't be far off, probably just beyond the stream,'' he
said to the hussar beside him.

The hussar only sighed without replying and coughed angrily. The
sound of horse's hoofs approaching at a trot along the line of
hussars was heard, and out of the foggy darkness the figure of a
sergeant of hussars suddenly appeared, looming huge as an
elephant.

``Your honor, the generals!'' said the sergeant, riding up to
Rostov.

Rostov, still looking round toward the fires and the shouts, rode
with the sergeant to meet some mounted men who were riding along
the line.  One was on a white horse. Prince Bagration and Prince
Dolgorukov with their adjutants had come to witness the curious
phenomenon of the lights and shouts in the enemy's camp. Rostov
rode up to Bagration, reported to him, and then joined the
adjutants listening to what the generals were saying.

``Believe me,'' said Prince Dolgorukov, addressing Bagration,
``it is nothing but a trick! He has retreated and ordered the
rearguard to kindle fires and make a noise to deceive us.''

``Hardly,'' said Bagration. ``I saw them this evening on that
knoll; if they had retreated they would have withdrawn from that
too... Officer!''  said Bagration to Rostov, ``are the enemy's
skirmishers still there?''

``They were there this evening, but now I don't know, your
excellency.  Shall I go with some of my hussars to see?'' replied
Rostov.

Bagration stopped and, before replying, tried to see Rostov's
face in the mist.

``Well, go and see,'' he said, after a pause.

``Yes, sir.''

Rostov spurred his horse, called to Sergeant Fedchenko and two
other hussars, told them to follow him, and trotted downhill in
the direction from which the shouting came. He felt both
frightened and pleased to be riding alone with three hussars into
that mysterious and dangerous misty distance where no one had
been before him. Bagration called to him from the hill not to go
beyond the stream, but Rostov pretended not to hear him and did
not stop but rode on and on, continually mistaking bushes for
trees and gullies for men and continually discovering his
mistakes.  Having descended the hill at a trot, he no longer saw
either our own or the enemy's fires, but heard the shouting of
the French more loudly and distinctly. In the valley he saw
before him something like a river, but when he reached it he
found it was a road. Having come out onto the road he reined in
his horse, hesitating whether to ride along it or cross it and
ride over the black field up the hillside. To keep to the road
which gleamed white in the mist would have been safer because it
would be easier to see people coming along it. ``Follow me!''
said he, crossed the road, and began riding up the hill at a
gallop toward the point where the French pickets had been
standing that evening.

``Your honor, there he is!'' cried one of the hussars behind
him. And before Rostov had time to make out what the black thing
was that had suddenly appeared in the fog, there was a flash,
followed by a report, and a bullet whizzing high up in the mist
with a plaintive sound passed out of hearing. Another musket
missed fire but flashed in the pan.  Rostov turned his horse and
galloped back. Four more reports followed at intervals, and the
bullets passed somewhere in the fog singing in different
tones. Rostov reined in his horse, whose spirits had risen, like
his own, at the firing, and went back at a footpace. ``Well, some
more! Some more!'' a merry voice was saying in his soul. But no
more shots came.

Only when approaching Bagration did Rostov let his horse gallop
again, and with his hand at the salute rode up to the general.

Dolgorukov was still insisting that the French had retreated and
had only lit fires to deceive us.

``What does that prove?'' he was saying as Rostov rode up. ``They
might retreat and leave the pickets.''

``It's plain that they have not all gone yet, Prince,'' said
Bagration.  ``Wait till tomorrow morning, we'll find out
everything tomorrow.''

``The picket is still on the hill, your excellency, just where it
was in the evening,'' reported Rostov, stooping forward with his
hand at the salute and unable to repress the smile of delight
induced by his ride and especially by the sound of the bullets.

``Very good, very good,'' said Bagration. ``Thank you, officer.''

``Your excellency,'' said Rostov, ``may I ask a favor?''

``What is it?''

``Tomorrow our squadron is to be in reserve. May I ask to be
attached to the first squadron?''

``What's your name?''

``Count Rostov.''

``Oh, very well, you may stay in attendance on me.''

``Count Ilya Rostov's son?'' asked Dolgorukov.

But Rostov did not reply.

``Then I may reckon on it, your excellency?''

``I will give the order.''

``Tomorrow very likely I may be sent with some message to the
Emperor,'' thought Rostov.

``Thank God!''

The fires and shouting in the enemy's army were occasioned by the
fact that while Napoleon's proclamation was being read to the
troops the Emperor himself rode round his bivouacs. The soldiers,
on seeing him, lit wisps of straw and ran after him, shouting,
``Vive l'Empereur!''  Napoleon's proclamation was as follows:

\begin{quote} \calli
Soldiers! The Russian army is advancing against you
to avenge the Austrian army of Ulm. They are the same battalions
you broke at Hollabrunn and have pursued ever since to this
place. The position we occupy is a strong one, and while they are
marching to go round me on the right they will expose a flank to
me. Soldiers! I will myself direct your battalions. I will keep
out of fire if you with your habitual valor carry disorder and
confusion into the enemy's ranks, but should victory be in doubt,
even for a moment, you will see your Emperor exposing himself to
the first blows of the enemy, for there must be no doubt of
victory, especially on this day when what is at stake is the
honor of the French infantry, so necessary to the honor of our
nation.

Do not break your ranks on the plea of removing the wounded! Let
every man be fully imbued with the thought that we must defeat
these hirelings of England, inspired by such hatred of our
nation! This victory will conclude our campaign and we can return
to winter quarters, where fresh French troops who are being
raised in France will join us, and the peace I shall conclude
will be worthy of my people, of you, and of myself.

\textsc{Napoleon}
\end{quote}

% % % % % % % % % % % % % % % % % % % % % % % % % % % % % % % % %
% % % % % % % % % % % % % % % % % % % % % % % % % % % % % % % % %
% % % % % % % % % % % % % % % % % % % % % % % % % % % % % % % % %
% % % % % % % % % % % % % % % % % % % % % % % % % % % % % % % % %
% % % % % % % % % % % % % % % % % % % % % % % % % % % % % % % % %
% % % % % % % % % % % % % % % % % % % % % % % % % % % % % % % % %
% % % % % % % % % % % % % % % % % % % % % % % % % % % % % % % % %
% % % % % % % % % % % % % % % % % % % % % % % % % % % % % % % % %
% % % % % % % % % % % % % % % % % % % % % % % % % % % % % % % % %
% % % % % % % % % % % % % % % % % % % % % % % % % % % % % % % % %
% % % % % % % % % % % % % % % % % % % % % % % % % % % % % % % % %
% % % % % % % % % % % % % % % % % % % % % % % % % % % % % %

\chapter*{Chapter XIV}
\ifaudio     
\marginpar{
\href{http://ia800208.us.archive.org/14/items/war_and_peace_03_0712_librivox/war_and_peace_03_14_tolstoy_64kb.mp3}{Audio}} 
\fi

\lettrine[lines=2, loversize=0.3, lraise=0]{\initfamily A}{t}
five in the morning it was still quite dark. The troops of the
center, the reserves, and Bagration's right flank had not yet
moved, but on the left flank the columns of infantry, cavalry,
and artillery, which were to be the first to descend the heights
to attack the French right flank and drive it into the Bohemian
mountains according to plan, were already up and astir. The smoke
of the campfires, into which they were throwing everything
superfluous, made the eyes smart. It was cold and dark. The
officers were hurriedly drinking tea and breakfasting, the
soldiers, munching biscuit and beating a tattoo with their feet
to warm themselves, gathering round the fires throwing into the
flames the remains of sheds, chairs, tables, wheels, tubs, and
everything that they did not want or could not carry away with
them. Austrian column guides were moving in and out among the
Russian troops and served as heralds of the advance. As soon as
an Austrian officer showed himself near a commanding officer's
quarters, the regiment began to move: the soldiers ran from the
fires, thrust their pipes into their boots, their bags into the
carts, got their muskets ready, and formed rank. The officers
buttoned up their coats, buckled on their swords and pouches, and
moved along the ranks shouting. The train drivers and orderlies
harnessed and packed the wagons and tied on the loads. The
adjutants and battalion and regimental commanders mounted,
crossed themselves, gave final instructions, orders, and
commissions to the baggage men who remained behind, and the
monotonous tramp of thousands of feet resounded. The column moved
forward without knowing where and unable, from the masses around
them, the smoke and the increasing fog, to see either the place
they were leaving or that to which they were going.

A soldier on the march is hemmed in and borne along by his
regiment as much as a sailor is by his ship. However far he has
walked, whatever strange, unknown, and dangerous places he
reaches, just as a sailor is always surrounded by the same decks,
masts, and rigging of his ship, so the soldier always has around
him the same comrades, the same ranks, the same sergeant major
Ivan Mitrich, the same company dog Jack, and the same
commanders. The sailor rarely cares to know the latitude in which
his ship is sailing, but on the day of battle---heaven knows how
and whence---a stern note of which all are conscious sounds in
the moral atmosphere of an army, announcing the approach of
something decisive and solemn, and awakening in the men an
unusual curiosity. On the day of battle the soldiers excitedly
try to get beyond the interests of their regiment, they listen
intently, look about, and eagerly ask concerning what is going on
around them.

The fog had grown so dense that though it was growing light they
could not see ten paces ahead. Bushes looked like gigantic trees
and level ground like cliffs and slopes. Anywhere, on any side,
one might encounter an enemy invisible ten paces off. But the
columns advanced for a long time, always in the same fog,
descending and ascending hills, avoiding gardens and enclosures,
going over new and unknown ground, and nowhere encountering the
enemy. On the contrary, the soldiers became aware that in front,
behind, and on all sides, other Russian columns were moving in
the same direction. Every soldier felt glad to know that to the
unknown place where he was going, many more of our men were going
too.

``There now, the Kurskies have also gone past,'' was being said
in the ranks.

``It's wonderful what a lot of our troops have gathered, lads!
Last night I looked at the campfires and there was no end of
them. A regular Moscow!''

Though none of the column commanders rode up to the ranks or
talked to the men (the commanders, as we saw at the council of
war, were out of humor and dissatisfied with the affair, and so
did not exert themselves to cheer the men but merely carried out
the orders), yet the troops marched gaily, as they always do when
going into action, especially to an attack. But when they had
marched for about an hour in the dense fog, the greater part of
the men had to halt and an unpleasant consciousness of some
dislocation and blunder spread through the ranks. How such a
consciousness is communicated is very difficult to define, but it
certainly is communicated very surely, and flows rapidly,
imperceptibly, and irrepressibly, as water does in a creek. Had
the Russian army been alone without any allies, it might perhaps
have been a long time before this consciousness of mismanagement
became a general conviction, but as it was, the disorder was
readily and naturally attributed to the stupid Germans, and
everyone was convinced that a dangerous muddle had been
occasioned by the sausage eaters.

``Why have we stopped? Is the way blocked? Or have we already
come up against the French?''

``No, one can't hear them. They'd be firing if we had.''

``They were in a hurry enough to start us, and now here we stand
in the middle of a field without rhyme or reason. It's all those
damned Germans' muddling! What stupid devils!''

``Yes, I'd send them on in front, but no fear, they're crowding
up behind. And now here we stand hungry.''

``I say, shall we soon be clear? They say the cavalry are
blocking the way,'' said an officer.

``Ah, those damned Germans! They don't know their own country!''
said another.

``What division are you?'' shouted an adjutant, riding up.

``The Eighteenth.''

``Then why are you here? You should have gone on long ago, now
you won't get there till evening.''

``What stupid orders! They don't themselves know what they are
doing!''  said the officer and rode off.

Then a general rode past shouting something angrily, not in
Russian.

``Tafa-lafa! But what he's jabbering no one can make out,'' said
a soldier, mimicking the general who had ridden away. ``I'd shoot
them, the scoundrels!''

``We were ordered to be at the place before nine, but we haven't
got halfway. Fine orders!'' was being repeated on different
sides.

And the feeling of energy with which the troops had started began
to turn into vexation and anger at the stupid arrangements and at
the Germans.

The cause of the confusion was that while the Austrian cavalry
was moving toward our left flank, the higher command found that
our center was too far separated from our right flank and the
cavalry were all ordered to turn back to the right. Several
thousand cavalry crossed in front of the infantry, who had to
wait.

At the front an altercation occurred between an Austrian guide
and a Russian general. The general shouted a demand that the
cavalry should be halted, the Austrian argued that not he, but
the higher command, was to blame. The troops meanwhile stood
growing listless and dispirited. After an hour's delay they at
last moved on, descending the hill. The fog that was dispersing
on the hill lay still more densely below, where they were
descending. In front in the fog a shot was heard and then
another, at first irregularly at varying
intervals---trata... tat---and then more and more regularly and
rapidly, and the action at the Goldbach Stream began.

Not expecting to come on the enemy down by the stream, and having
stumbled on him in the fog, hearing no encouraging word from
their commanders, and with a consciousness of being too late
spreading through the ranks, and above all being unable to see
anything in front or around them in the thick fog, the Russians
exchanged shots with the enemy lazily and advanced and again
halted, receiving no timely orders from the officers or adjutants
who wandered about in the fog in those unknown surroundings
unable to find their own regiments. In this way the action began
for the first, second, and third columns, which had gone down
into the valley. The fourth column, with which Kutuzov was, stood
on the Pratzen Heights.

Below, where the fight was beginning, there was still thick fog;
on the higher ground it was clearing, but nothing could be seen
of what was going on in front. Whether all the enemy forces were,
as we supposed, six miles away, or whether they were near by in
that sea of mist, no one knew till after eight o'clock.

It was nine o'clock in the morning. The fog lay unbroken like a
sea down below, but higher up at the village of Schlappanitz
where Napoleon stood with his marshals around him, it was quite
light. Above him was a clear blue sky, and the sun's vast orb
quivered like a huge hollow, crimson float on the surface of that
milky sea of mist. The whole French army, and even Napoleon
himself with his staff, were not on the far side of the streams
and hollows of Sokolnitz and Schlappanitz beyond which we
intended to take up our position and begin the action, but were
on this side, so close to our own forces that Napoleon with the
naked eye could distinguish a mounted man from one on
foot. Napoleon, in the blue cloak which he had worn on his
Italian campaign, sat on his small gray Arab horse a little in
front of his marshals. He gazed silently at the hills which
seemed to rise out of the sea of mist and on which the Russian
troops were moving in the distance, and he listened to the sounds
of firing in the valley. Not a single muscle of his face---which
in those days was still thin---moved. His gleaming eyes were
fixed intently on one spot. His predictions were being
justified. Part of the Russian force had already descended into
the valley toward the ponds and lakes and part were leaving these
Pratzen Heights which he intended to attack and regarded as the
key to the position. He saw over the mist that in a hollow
between two hills near the village of Pratzen, the Russian
columns, their bayonets glittering, were moving continuously in
one direction toward the valley and disappearing one after
another into the mist. From information he had received the
evening before, from the sound of wheels and footsteps heard by
the outposts during the night, by the disorderly movement of the
Russian columns, and from all indications, he saw clearly that
the allies believed him to be far away in front of them, and that
the columns moving near Pratzen constituted the center of the
Russian army, and that that center was already sufficiently
weakened to be successfully attacked. But still he did not begin
the engagement.

Today was a great day for him---the anniversary of his
coronation. Before dawn he had slept for a few hours, and
refreshed, vigorous, and in good spirits, he mounted his horse
and rode out into the field in that happy mood in which
everything seems possible and everything succeeds. He sat
motionless, looking at the heights visible above the mist, and
his cold face wore that special look of confident,
self-complacent happiness that one sees on the face of a boy
happily in love. The marshals stood behind him not venturing to
distract his attention. He looked now at the Pratzen Heights, now
at the sun floating up out of the mist.

When the sun had entirely emerged from the fog, and fields and
mist were aglow with dazzling light---as if he had only awaited
this to begin the action---he drew the glove from his shapely
white hand, made a sign with it to the marshals, and ordered the
action to begin. The marshals, accompanied by adjutants, galloped
off in different directions, and a few minutes later the chief
forces of the French army moved rapidly toward those Pratzen
Heights which were being more and more denuded by Russian troops
moving down the valley to their left.

% % % % % % % % % % % % % % % % % % % % % % % % % % % % % % % % %
% % % % % % % % % % % % % % % % % % % % % % % % % % % % % % % % %
% % % % % % % % % % % % % % % % % % % % % % % % % % % % % % % % %
% % % % % % % % % % % % % % % % % % % % % % % % % % % % % % % % %
% % % % % % % % % % % % % % % % % % % % % % % % % % % % % % % % %
% % % % % % % % % % % % % % % % % % % % % % % % % % % % % % % % %
% % % % % % % % % % % % % % % % % % % % % % % % % % % % % % % % %
% % % % % % % % % % % % % % % % % % % % % % % % % % % % % % % % %
% % % % % % % % % % % % % % % % % % % % % % % % % % % % % % % % %
% % % % % % % % % % % % % % % % % % % % % % % % % % % % % % % % %
% % % % % % % % % % % % % % % % % % % % % % % % % % % % % % % % %
% % % % % % % % % % % % % % % % % % % % % % % % % % % % % %

\chapter*{Chapter XV}
\ifaudio
\marginpar{
\href{http://ia800208.us.archive.org/14/items/war_and_peace_03_0712_librivox/war_and_peace_03_15_tolstoy_64kb.mp3}{Audio}} 
\fi

\lettrine[lines=2, loversize=0.3, lraise=0]{\initfamily A}{t}
eight o'clock Kutuzov rode to Pratzen at the head of the
fourth column, Miloradovich's, the one that was to take the place
of Przebyszewski's and Langeron's columns which had already gone
down into the valley. He greeted the men of the foremost regiment
and gave them the order to march, thereby indicating that he
intended to lead that column himself. When he had reached the
village of Pratzen he halted.  Prince Andrew was behind, among
the immense number forming the commander-in-chief's suite. He was
in a state of suppressed excitement and irritation, though
controlledly calm as a man is at the approach of a long-awaited
moment. He was firmly convinced that this was the day of his
Toulon, or his bridge of Arcola. How it would come about he did
not know, but he felt sure it would do so. The locality and the
position of our troops were known to him as far as they could be
known to anyone in our army. His own strategic plan, which
obviously could not now be carried out, was forgotten. Now,
entering into Weyrother's plan, Prince Andrew considered possible
contingencies and formed new projects such as might call for his
rapidity of perception and decision.

To the left down below in the mist, the musketry fire of unseen
forces could be heard. It was there Prince Andrew thought the
fight would concentrate. ``There we shall encounter difficulties,
and there,'' thought he, ``I shall be sent with a brigade or
division, and there, standard in hand, I shall go forward and
break whatever is in front of me.''

He could not look calmly at the standards of the passing
battalions.  Seeing them he kept thinking, ``That may be the very
standard with which I shall lead the army.''

In the morning all that was left of the night mist on the heights
was a hoar frost now turning to dew, but in the valleys it still
lay like a milk-white sea. Nothing was visible in the valley to
the left into which our troops had descended and from whence came
the sounds of firing.  Above the heights was the dark clear sky,
and to the right the vast orb of the sun. In front, far off on
the farther shore of that sea of mist, some wooded hills were
discernible, and it was there the enemy probably was, for
something could be descried. On the right the Guards were
entering the misty region with a sound of hoofs and wheels and
now and then a gleam of bayonets; to the left beyond the village
similar masses of cavalry came up and disappeared in the sea of
mist. In front and behind moved infantry. The commander-in-chief
was standing at the end of the village letting the troops pass by
him. That morning Kutuzov seemed worn and irritable. The infantry
passing before him came to a halt without any command being
given, apparently obstructed by something in front.

``Do order them to form into battalion columns and go round the
village!''  he said angrily to a general who had ridden
up. ``Don't you understand, your excellency, my dear sir, that
you must not defile through narrow village streets when we are
marching against the enemy?''

``I intended to re-form them beyond the village, your
excellency,'' answered the general.

Kutuzov laughed bitterly.

``You'll make a fine thing of it, deploying in sight of the
enemy! Very fine!''

``The enemy is still far away, your excellency. According to the
dispositions...''

``The dispositions!'' exclaimed Kutuzov bitterly. ``Who told you
that?...  Kindly do as you are ordered.''

``Yes, sir.''

``My dear fellow,'' Nesvitski whispered to Prince Andrew, ``the
old man is as surly as a dog.''

An Austrian officer in a white uniform with green plumes in his
hat galloped up to Kutuzov and asked in the Emperor's name had
the fourth column advanced into action.

Kutuzov turned round without answering and his eye happened to
fall upon Prince Andrew, who was beside him. Seeing him,
Kutuzov's malevolent and caustic expression softened, as if
admitting that what was being done was not his adjutant's fault,
and still not answering the Austrian adjutant, he addressed
Bolkonski.

``Go, my dear fellow, and see whether the third division has
passed the village. Tell it to stop and await my orders.''

Hardly had Prince Andrew started than he stopped him.

``And ask whether sharpshooters have been posted,'' he
added. ``What are they doing? What are they doing?'' he murmured
to himself, still not replying to the Austrian.

Prince Andrew galloped off to execute the order.

Overtaking the battalions that continued to advance, he stopped
the third division and convinced himself that there really were
no sharpshooters in front of our columns. The colonel at the head
of the regiment was much surprised at the commander-in-chief's
order to throw out skirmishers. He had felt perfectly sure that
there were other troops in front of him and that the enemy must
be at least six miles away.  There was really nothing to be seen
in front except a barren descent hidden by dense mist. Having
given orders in the commander-in-chief's name to rectify this
omission, Prince Andrew galloped back. Kutuzov still in the same
place, his stout body resting heavily in the saddle with the
lassitude of age, sat yawning wearily with closed eyes. The
troops were no longer moving, but stood with the butts of their
muskets on the ground.

``All right, all right!'' he said to Prince Andrew, and turned to
a general who, watch in hand, was saying it was time they started
as all the left-flank columns had already descended.

``Plenty of time, your excellency,'' muttered Kutuzov in the
midst of a yawn. ``Plenty of time,'' he repeated.

Just then at a distance behind Kutuzov was heard the sound of
regiments saluting, and this sound rapidly came nearer along the
whole extended line of the advancing Russian columns. Evidently
the person they were greeting was riding quickly. When the
soldiers of the regiment in front of which Kutuzov was standing
began to shout, he rode a little to one side and looked round
with a frown. Along the road from Pratzen galloped what looked
like a squadron of horsemen in various uniforms. Two of them rode
side by side in front, at full gallop. One in a black uniform
with white plumes in his hat rode a bobtailed chestnut horse, the
other who was in a white uniform rode a black one. These were the
two Emperors followed by their suites. Kutuzov, affecting the
manners of an old soldier at the front, gave the command
``Attention!'' and rode up to the Emperors with a salute. His
whole appearance and manner were suddenly transformed. He put on
the air of a subordinate who obeys without reasoning. With an
affectation of respect which evidently struck Alexander
unpleasantly, he rode up and saluted.

This unpleasant impression merely flitted over the young and
happy face of the Emperor like a cloud of haze across a clear sky
and vanished.  After his illness he looked rather thinner that
day than on the field of Olmutz where Bolkonski had seen him for
the first time abroad, but there was still the same bewitching
combination of majesty and mildness in his fine gray eyes, and on
his delicate lips the same capacity for varying expression and
the same prevalent appearance of goodhearted innocent youth.

At the Olmutz review he had seemed more majestic; here he seemed
brighter and more energetic. He was slightly flushed after
galloping two miles, and reining in his horse he sighed restfully
and looked round at the faces of his suite, young and animated as
his own. Czartoryski, Novosiltsev, Prince Volkonsky, Strogonov,
and the others, all richly dressed gay young men on splendid,
well-groomed, fresh, only slightly heated horses, exchanging
remarks and smiling, had stopped behind the Emperor. The Emperor
Francis, a rosy, long faced young man, sat very erect on his
handsome black horse, looking about him in a leisurely and
preoccupied manner. He beckoned to one of his white adjutants and
asked some question---``Most likely he is asking at what o'clock
they started,'' thought Prince Andrew, watching his old
acquaintance with a smile he could not repress as he recalled his
reception at Brunn. In the Emperors' suite were the picked young
orderly officers of the Guard and line regiments, Russian and
Austrian. Among them were grooms leading the Tsar's beautiful
relay horses covered with embroidered cloths.

As when a window is opened a whiff of fresh air from the fields
enters a stuffy room, so a whiff of youthfulness, energy, and
confidence of success reached Kutuzov's cheerless staff with the
galloping advent of all these brilliant young men.

``Why aren't you beginning, Michael Ilarionovich?'' said the
Emperor Alexander hurriedly to Kutuzov, glancing courteously at
the same time at the Emperor Francis.

``I am waiting, Your Majesty,'' answered Kutuzov, bending forward
respectfully.

The Emperor, frowning slightly, bent his ear forward as if he had
not quite heard.

``Waiting, Your Majesty,'' repeated Kutuzov. (Prince Andrew noted
that Kutuzov's upper lip twitched unnaturally as he said the word
``waiting.'')  ``Not all the columns have formed up yet, Your
Majesty.''

The Tsar heard but obviously did not like the reply; he shrugged
his rather round shoulders and glanced at Novosiltsev who was
near him, as if complaining of Kutuzov.

``You know, Michael Ilarionovich, we are not on the Empress'
Field where a parade does not begin till all the troops are
assembled,'' said the Tsar with another glance at the Emperor
Francis, as if inviting him if not to join in at least to listen
to what he was saying. But the Emperor Francis continued to look
about him and did not listen.

``That is just why I do not begin, sire,'' said Kutuzov in a
resounding voice, apparently to preclude the possibility of not
being heard, and again something in his face twitched---``That is
just why I do not begin, sire, because we are not on parade and
not on the Empress' Field,'' said clearly and distinctly.

In the Emperor's suite all exchanged rapid looks that expressed
dissatisfaction and reproach. ``Old though he may be, he should
not, he certainly should not, speak like that,'' their glances
seemed to say.

The Tsar looked intently and observantly into Kutuzov's eye
waiting to hear whether he would say anything more. But Kutuzov,
with respectfully bowed head, seemed also to be waiting. The
silence lasted for about a minute.

``However, if you command it, Your Majesty,'' said Kutuzov,
lifting his head and again assuming his former tone of a dull,
unreasoning, but submissive general.

He touched his horse and having called Miloradovich, the
commander of the column, gave him the order to advance.

The troops again began to move, and two battalions of the
Novgorod and one of the Apsheron regiment went forward past the
Emperor.

As this Apsheron battalion marched by, the red-faced
Miloradovich, without his greatcoat, with his Orders on his
breast and an enormous tuft of plumes in his cocked hat worn on
one side with its corners front and back, galloped strenuously
forward, and with a dashing salute reined in his horse before the
Emperor.

``God be with you, general!'' said the Emperor.

``Ma foi, sire, nous ferons ce qui sera dans notre possibilite,
sire,''\footnote{``Indeed, Sire, we shall do everything it is
possible to do, Sire.''}  he answered gaily, raising nevertheless
ironic smiles among the gentlemen of the Tsar's suite by his poor
French.

Miloradovich wheeled his horse sharply and stationed himself a
little behind the Emperor. The Apsheron men, excited by the
Tsar's presence, passed in step before the Emperors and their
suites at a bold, brisk pace.

``Lads!'' shouted Miloradovich in a loud, self-confident, and
cheery voice, obviously so elated by the sound of firing, by the
prospect of battle, and by the sight of the gallant Apsherons,
his comrades in Suvorov's time, now passing so gallantly before
the Emperors, that he forgot the sovereigns' presence. ``Lads,
it's not the first village you've had to take,'' cried he.

``Glad to do our best!'' shouted the soldiers.

The Emperor's horse started at the sudden cry. This horse that
had carried the sovereign at reviews in Russia bore him also here
on the field of Austerlitz, enduring the heedless blows of his
left foot and pricking its ears at the sound of shots just as it
had done on the Empress' Field, not understanding the
significance of the firing, nor of the nearness of the Emperor
Francis' black cob, nor of all that was being said, thought, and
felt that day by its rider.

The Emperor turned with a smile to one of his followers and made
a remark to him, pointing to the gallant Apsherons.

% % % % % % % % % % % % % % % % % % % % % % % % % % % % % % % % %
% % % % % % % % % % % % % % % % % % % % % % % % % % % % % % % % %
% % % % % % % % % % % % % % % % % % % % % % % % % % % % % % % % %
% % % % % % % % % % % % % % % % % % % % % % % % % % % % % % % % %
% % % % % % % % % % % % % % % % % % % % % % % % % % % % % % % % %
% % % % % % % % % % % % % % % % % % % % % % % % % % % % % % % % %
% % % % % % % % % % % % % % % % % % % % % % % % % % % % % % % % %
% % % % % % % % % % % % % % % % % % % % % % % % % % % % % % % % %
% % % % % % % % % % % % % % % % % % % % % % % % % % % % % % % % %
% % % % % % % % % % % % % % % % % % % % % % % % % % % % % % % % %
% % % % % % % % % % % % % % % % % % % % % % % % % % % % % % % % %
% % % % % % % % % % % % % % % % % % % % % % % % % % % % % %

\chapter*{Chapter XVI}
\ifaudio     
\marginpar{
\href{http://ia800208.us.archive.org/14/items/war_and_peace_03_0712_librivox/war_and_peace_03_16_tolstoy_64kb.mp3}{Audio}} 
\fi

\lettrine[lines=2, loversize=0.3, lraise=0]{\initfamily K}{utuzov}
 accompanied by his adjutants rode at a walking pace
behind the carabineers.

When he had gone less than half a mile in the rear of the column
he stopped at a solitary, deserted house that had probably once
been an inn, where two roads parted. Both of them led downhill
and troops were marching along both.

The fog had begun to clear and enemy troops were already dimly
visible about a mile and a half off on the opposite heights. Down
below, on the left, the firing became more distinct. Kutuzov had
stopped and was speaking to an Austrian general. Prince Andrew,
who was a little behind looking at them, turned to an adjutant to
ask him for a field glass.

``Look, look!'' said this adjutant, looking not at the troops in
the distance, but down the hill before him. ``It's the French!''

The two generals and the adjutant took hold of the field glass,
trying to snatch it from one another. The expression on all their
faces suddenly changed to one of horror. The French were supposed
to be a mile and a half away, but had suddenly and unexpectedly
appeared just in front of us.

``It's the enemy?... No!... Yes, see it is!... for certain... But
how is that?'' said different voices.

With the naked eye Prince Andrew saw below them to the right, not
more than five hundred paces from where Kutuzov was standing, a
dense French column coming up to meet the Apsherons.

``Here it is! The decisive moment has arrived. My turn has
come,'' thought Prince Andrew, and striking his horse he rode up
to Kutuzov.

``The Apsherons must be stopped, your excellency,'' cried he. But
at that very instant a cloud of smoke spread all round, firing
was heard quite close at hand, and a voice of naive terror barely
two steps from Prince Andrew shouted, ``Brothers! All's lost!''
And at this as if at a command, everyone began to run.

Confused and ever-increasing crowds were running back to where
five minutes before the troops had passed the Emperors. Not only
would it have been difficult to stop that crowd, it was even
impossible not to be carried back with it oneself. Bolkonski only
tried not to lose touch with it, and looked around bewildered and
unable to grasp what was happening in front of him. Nesvitski
with an angry face, red and unlike himself, was shouting to
Kutuzov that if he did not ride away at once he would certainly
be taken prisoner. Kutuzov remained in the same place and without
answering drew out a handkerchief. Blood was flowing from his
cheek. Prince Andrew forced his way to him.

``You are wounded?'' he asked, hardly able to master the
trembling of his lower jaw.

``The wound is not here, it is there!'' said Kutuzov, pressing
the handkerchief to his wounded cheek and pointing to the fleeing
soldiers.  ``Stop them!'' he shouted, and at the same moment,
probably realizing that it was impossible to stop them, spurred
his horse and rode to the right.

A fresh wave of the flying mob caught him and bore him back with
it.

The troops were running in such a dense mass that once surrounded
by them it was difficult to get out again. One was shouting,
``Get on! Why are you hindering us?'' Another in the same place
turned round and fired in the air; a third was striking the horse
Kutuzov himself rode. Having by a great effort got away to the
left from that flood of men, Kutuzov, with his suite diminished
by more than half, rode toward a sound of artillery fire near
by. Having forced his way out of the crowd of fugitives, Prince
Andrew, trying to keep near Kutuzov, saw on the slope of the hill
amid the smoke a Russian battery that was still firing and
Frenchmen running toward it. Higher up stood some Russian
infantry, neither moving forward to protect the battery nor
backward with the fleeing crowd. A mounted general separated
himself from the infantry and approached Kutuzov. Of Kutuzov's
suite only four remained. They were all pale and exchanged looks
in silence.

``Stop those wretches!'' gasped Kutuzov to the regimental
commander, pointing to the flying soldiers; but at that instant,
as if to punish him for those words, bullets flew hissing across
the regiment and across Kutuzov's suite like a flock of little
birds.

The French had attacked the battery and, seeing Kutuzov, were
firing at him. After this volley the regimental commander
clutched at his leg; several soldiers fell, and a second
lieutenant who was holding the flag let it fall from his
hands. It swayed and fell, but caught on the muskets of the
nearest soldiers. The soldiers started firing without orders.

``Oh! Oh! Oh!'' groaned Kutuzov despairingly and looked around...
``Bolkonski!'' he whispered, his voice trembling from a
consciousness of the feebleness of age, ``Bolkonski!'' he
whispered, pointing to the disordered battalion and at the enemy,
``what's that?''

But before he had finished speaking, Prince Andrew, feeling tears
of shame and anger choking him, had already leapt from his horse
and run to the standard.

``Forward, lads!'' he shouted in a voice piercing as a child's.

``Here it is!'' thought he, seizing the staff of the standard and
hearing with pleasure the whistle of bullets evidently aimed at
him. Several soldiers fell.

``Hurrah!'' shouted Prince Andrew, and, scarcely able to hold up
the heavy standard, he ran forward with full confidence that the
whole battalion would follow him.

And really he only ran a few steps alone. One soldier moved and
then another and soon the whole battalion ran forward shouting
``Hurrah!'' and overtook him. A sergeant of the battalion ran up
and took the flag that was swaying from its weight in Prince
Andrew's hands, but he was immediately killed. Prince Andrew
again seized the standard and, dragging it by the staff, ran on
with the battalion. In front he saw our artillerymen, some of
whom were fighting, while others, having abandoned their guns,
were running toward him. He also saw French infantry soldiers who
were seizing the artillery horses and turning the guns
round. Prince Andrew and the battalion were already within twenty
paces of the cannon. He heard the whistle of bullets above him
unceasingly and to right and left of him soldiers continually
groaned and dropped. But he did not look at them: he looked only
at what was going on in front of him---at the battery. He now saw
clearly the figure of a red-haired gunner with his shako knocked
awry, pulling one end of a mop while a French soldier tugged at
the other. He could distinctly see the distraught yet angry
expression on the faces of these two men, who evidently did not
realize what they were doing.

``What are they about?'' thought Prince Andrew as he gazed at
them. ``Why doesn't the red-haired gunner run away as he is
unarmed? Why doesn't the Frenchman stab him? He will not get away
before the Frenchman remembers his bayonet and stabs him...''

And really another French soldier, trailing his musket, ran up to
the struggling men, and the fate of the red-haired gunner, who
had triumphantly secured the mop and still did not realize what
awaited him, was about to be decided. But Prince Andrew did not
see how it ended. It seemed to him as though one of the soldiers
near him hit him on the head with the full swing of a
bludgeon. It hurt a little, but the worst of it was that the pain
distracted him and prevented his seeing what he had been looking
at.

``What's this? Am I falling? My legs are giving way,'' thought
he, and fell on his back. He opened his eyes, hoping to see how
the struggle of the Frenchmen with the gunners ended, whether the
red-haired gunner had been killed or not and whether the cannon
had been captured or saved.  But he saw nothing. Above him there
was now nothing but the sky---the lofty sky, not clear yet still
immeasurably lofty, with gray clouds gliding slowly across
it. ``How quiet, peaceful, and solemn; not at all as I ran,''
thought Prince Andrew---``not as we ran, shouting and fighting,
not at all as the gunner and the Frenchman with frightened and
angry faces struggled for the mop: how differently do those
clouds glide across that lofty infinite sky! How was it I did not
see that lofty sky before? And how happy I am to have found it at
last! Yes! All is vanity, all falsehood, except that infinite
sky. There is nothing, nothing, but that. But even it does not
exist, there is nothing but quiet and peace.  Thank God!...''

% % % % % % % % % % % % % % % % % % % % % % % % % % % % % % % % %
% % % % % % % % % % % % % % % % % % % % % % % % % % % % % % % % %
% % % % % % % % % % % % % % % % % % % % % % % % % % % % % % % % %
% % % % % % % % % % % % % % % % % % % % % % % % % % % % % % % % %
% % % % % % % % % % % % % % % % % % % % % % % % % % % % % % % % %
% % % % % % % % % % % % % % % % % % % % % % % % % % % % % % % % %
% % % % % % % % % % % % % % % % % % % % % % % % % % % % % % % % %
% % % % % % % % % % % % % % % % % % % % % % % % % % % % % % % % %
% % % % % % % % % % % % % % % % % % % % % % % % % % % % % % % % %
% % % % % % % % % % % % % % % % % % % % % % % % % % % % % % % % %
% % % % % % % % % % % % % % % % % % % % % % % % % % % % % % % % %
% % % % % % % % % % % % % % % % % % % % % % % % % % % % % %

\chapter*{Chapter XVII}
\ifaudio     
\marginpar{
\href{http://ia800208.us.archive.org/14/items/war_and_peace_03_0712_librivox/war_and_peace_03_17_tolstoy_64kb.mp3}{Audio}} 
\fi

\lettrine[lines=2, loversize=0.3, lraise=0]{\initfamily O}{n}
our right flank commanded by Bagration, at nine o'clock the
battle had not yet begun. Not wishing to agree to Dolgorukov's
demand to commence the action, and wishing to avert
responsibility from himself, Prince Bagration proposed to
Dolgorukov to send to inquire of the
commander-in-chief. Bagration knew that as the distance between
the two flanks was more than six miles, even if the messenger
were not killed (which he very likely would be), and found the
commander-in-chief (which would be very difficult), he would not
be able to get back before evening.

Bagration cast his large, expressionless, sleepy eyes round his
suite, and the boyish face Rostov, breathless with excitement and
hope, was the first to catch his eye. He sent him.

``And if I should meet His Majesty before I meet the
commander-in-chief, your excellency?'' said Rostov, with his hand
to his cap.

``You can give the message to His Majesty,'' said Dolgorukov,
hurriedly interrupting Bagration.

On being relieved from picket duty Rostov had managed to get a
few hours' sleep before morning and felt cheerful, bold, and
resolute, with elasticity of movement, faith in his good fortune,
and generally in that state of mind which makes everything seem
possible, pleasant, and easy.

All his wishes were being fulfilled that morning: there was to be
a general engagement in which he was taking part, more than that,
he was orderly to the bravest general, and still more, he was
going with a message to Kutuzov, perhaps even to the sovereign
himself. The morning was bright, he had a good horse under him,
and his heart was full of joy and happiness. On receiving the
order he gave his horse the rein and galloped along the line. At
first he rode along the line of Bagration's troops, which had not
yet advanced into action but were standing motionless; then he
came to the region occupied by Uvarov's cavalry and here he
noticed a stir and signs of preparation for battle; having passed
Uvarov's cavalry he clearly heard the sound of cannon and
musketry ahead of him. The firing grew louder and louder.

In the fresh morning air were now heard, not two or three musket
shots at irregular intervals as before, followed by one or two
cannon shots, but a roll of volleys of musketry from the slopes
of the hill before Pratzen, interrupted by such frequent reports
of cannon that sometimes several of them were not separated from
one another but merged into a general roar.

He could see puffs of musketry smoke that seemed to chase one
another down the hillsides, and clouds of cannon smoke rolling,
spreading, and mingling with one another. He could also, by the
gleam of bayonets visible through the smoke, make out moving
masses of infantry and narrow lines of artillery with green
caissons.

Rostov stopped his horse for a moment on a hillock to see what
was going on, but strain his attention as he would he could not
understand or make out anything of what was happening: there in
the smoke men of some sort were moving about, in front and behind
moved lines of troops; but why, whither, and who they were, it
was impossible to make out. These sights and sounds had no
depressing or intimidating effect on him; on the contrary, they
stimulated his energy and determination.

``Go on! Go on! Give it them!'' he mentally exclaimed at these
sounds, and again proceeded to gallop along the line, penetrating
farther and farther into the region where the army was already in
action.

``How it will be there I don't know, but all will be well!''
thought Rostov.

After passing some Austrian troops he noticed that the next part
of the line (the Guards) was already in action.

``So much the better! I shall see it close,'' he thought.

He was riding almost along the front line. A handful of men came
galloping toward him. They were our uhlans who with disordered
ranks were returning from the attack. Rostov got out of their
way, involuntarily noticed that one of them was bleeding, and
galloped on.

``That is no business of mine,'' he thought. He had not ridden
many hundred yards after that before he saw to his left, across
the whole width of the field, an enormous mass of cavalry in
brilliant white uniforms, mounted on black horses, trotting
straight toward him and across his path. Rostov put his horse to
full gallop to get out of the way of these men, and he would have
got clear had they continued at the same speed, but they kept
increasing their pace, so that some of the horses were already
galloping. Rostov heard the thud of their hoofs and the jingle of
their weapons and saw their horses, their figures, and even their
faces, more and more distinctly. They were our Horse Guards,
advancing to attack the French cavalry that was coming to meet
them.

The Horse Guards were galloping, but still holding in their
horses.  Rostov could already see their faces and heard the
command: ``Charge!''  shouted by an officer who was urging his
thoroughbred to full speed.  Rostov, fearing to be crushed or
swept into the attack on the French, galloped along the front as
hard as his horse could go, but still was not in time to avoid
them.

The last of the Horse Guards, a huge pockmarked fellow, frowned
angrily on seeing Rostov before him, with whom he would
inevitably collide. This Guardsman would certainly have bowled
Rostov and his Bedouin over (Rostov felt himself quite tiny and
weak compared to these gigantic men and horses) had it not
occurred to Rostov to flourish his whip before the eyes of the
Guardsman's horse. The heavy black horse, sixteen hands high,
shied, throwing back its ears; but the pockmarked Guardsman drove
his huge spurs in violently, and the horse, flourishing its tail
and extending its neck, galloped on yet faster. Hardly had the
Horse Guards passed Rostov before he heard them shout,
``Hurrah!'' and looking back saw that their foremost ranks were
mixed up with some foreign cavalry with red epaulets, probably
French. He could see nothing more, for immediately afterwards
cannon began firing from somewhere and smoke enveloped
everything.

At that moment, as the Horse Guards, having passed him,
disappeared in the smoke, Rostov hesitated whether to gallop
after them or to go where he was sent. This was the brilliant
charge of the Horse Guards that amazed the French
themselves. Rostov was horrified to hear later that of all that
mass of huge and handsome men, of all those brilliant, rich
youths, officers and cadets, who had galloped past him on their
thousand-ruble horses, only eighteen were left after the charge.

``Why should I envy them? My chance is not lost, and maybe I
shall see the Emperor immediately!'' thought Rostov and galloped
on.

When he came level with the Foot Guards he noticed that about
them and around them cannon balls were flying, of which he was
aware not so much because he heard their sound as because he saw
uneasiness on the soldiers' faces and unnatural warlike solemnity
on those of the officers.

Passing behind one of the lines of a regiment of Foot Guards he
heard a voice calling him by name.

``Rostov!''

``What?'' he answered, not recognizing Boris.

``I say, we've been in the front line! Our regiment attacked!''
said Boris with the happy smile seen on the faces of young men
who have been under fire for the first time.

Rostov stopped.

``Have you?'' he said. ``Well, how did it go?''

``We drove them back!'' said Boris with animation, growing
talkative. ``Can you imagine it?'' and he began describing how
the Guards, having taken up their position and seeing troops
before them, thought they were Austrians, and all at once
discovered from the cannon balls discharged by those troops that
they were themselves in the front line and had unexpectedly to go
into action. Rostov without hearing Boris to the end spurred his
horse.

``Where are you off to?'' asked Boris.

``With a message to His Majesty.''

``There he is!'' said Boris, thinking Rostov had said ``His
Highness,'' and pointing to the Grand Duke who with his high
shoulders and frowning brows stood a hundred paces away from them
in his helmet and Horse Guards' jacket, shouting something to a
pale, white uniformed Austrian officer.

``But that's the Grand Duke, and I want the commander-in-chief or
the Emperor,'' said Rostov, and was about to spur his horse.

``Count! Count!'' shouted Berg who ran up from the other side as
eager as Boris. ``Count! I am wounded in my right hand'' (and he
showed his bleeding hand with a handkerchief tied round it) ``and
I remained at the front. I held my sword in my left hand,
Count. All our family---the von Bergs---have been knights!''

He said something more, but Rostov did not wait to hear it and
rode away.

Having passed the Guards and traversed an empty space, Rostov, to
avoid again getting in front of the first line as he had done
when the Horse Guards charged, followed the line of reserves,
going far round the place where the hottest musket fire and
cannonade were heard. Suddenly he heard musket fire quite close
in front of him and behind our troops, where he could never have
expected the enemy to be.

``What can it be?'' he thought. ``The enemy in the rear of our
army?  Impossible!'' And suddenly he was seized by a panic of
fear for himself and for the issue of the whole battle. ``But be
that what it may,'' he reflected, ``there is no riding round it
now. I must look for the commander in chief here, and if all is
lost it is for me to perish with the rest.''

The foreboding of evil that had suddenly come over Rostov was
more and more confirmed the farther he rode into the region
behind the village of Pratzen, which was full of troops of all
kinds.

``What does it mean? What is it? Whom are they firing at? Who is
firing?''  Rostov kept asking as he came up to Russian and
Austrian soldiers running in confused crowds across his path.

``The devil knows! They've killed everybody! It's all up now!''
he was told in Russian, German, and Czech by the crowd of
fugitives who understood what was happening as little as he did.

``Kill the Germans!'' shouted one.

``May the devil take them---the traitors!''

``Zum Henker diese Russen!''\footnote{``Damned Russians''
lit.:``For executioner these Russians!''} muttered a German.

Several wounded men passed along the road, and words of abuse,
screams, and groans mingled in a general hubbub, then the firing
died down.  Rostov learned later that Russian and Austrian
soldiers had been firing at one another.

``My God! What does it all mean?'' thought he. ``And here, where
at any moment the Emperor may see them... But no, these must be
only a handful of scoundrels. It will soon be over, it can't be
that, it can't be! Only to get past them quicker, quicker!''

The idea of defeat and flight could not enter Rostov's
head. Though he saw French cannon and French troops on the
Pratzen Heights just where he had been ordered to look for the
commander-in-chief, he could not, did not wish to, believe that.

% % % % % % % % % % % % % % % % % % % % % % % % % % % % % % % % %
% % % % % % % % % % % % % % % % % % % % % % % % % % % % % % % % %
% % % % % % % % % % % % % % % % % % % % % % % % % % % % % % % % %
% % % % % % % % % % % % % % % % % % % % % % % % % % % % % % % % %
% % % % % % % % % % % % % % % % % % % % % % % % % % % % % % % % %
% % % % % % % % % % % % % % % % % % % % % % % % % % % % % % % % %
% % % % % % % % % % % % % % % % % % % % % % % % % % % % % % % % %
% % % % % % % % % % % % % % % % % % % % % % % % % % % % % % % % %
% % % % % % % % % % % % % % % % % % % % % % % % % % % % % % % % %
% % % % % % % % % % % % % % % % % % % % % % % % % % % % % % % % %
% % % % % % % % % % % % % % % % % % % % % % % % % % % % % % % % %
% % % % % % % % % % % % % % % % % % % % % % % % % % % % % %

\chapter*{Chapter XVIII}
\ifaudio     
\marginpar{
\href{http://ia800208.us.archive.org/14/items/war_and_peace_03_0712_librivox/war_and_peace_03_18_tolstoy_64kb.mp3}{Audio}} 
\fi

\lettrine[lines=2, loversize=0.3, lraise=0]{\initfamily R}{ostov}
 had been ordered to look for Kutuzov and the Emperor near
the village of Pratzen. But neither they nor a single commanding
officer were there, only disorganized crowds of troops of various
kinds. He urged on his already weary horse to get quickly past
these crowds, but the farther he went the more disorganized they
were. The highroad on which he had come out was thronged with
caleches, carriages of all sorts, and Russian and Austrian
soldiers of all arms, some wounded and some not. This whole mass
droned and jostled in confusion under the dismal influence of
cannon balls flying from the French batteries stationed on the
Pratzen Heights.

``Where is the Emperor? Where is Kutuzov?'' Rostov kept asking
everyone he could stop, but got no answer from anyone.

At last seizing a soldier by his collar he forced him to answer.

``Eh, brother! They've all bolted long ago!'' said the soldier,
laughing for some reason and shaking himself free.

Having left that soldier who was evidently drunk, Rostov stopped
the horse of a batman or groom of some important personage and
began to question him. The man announced that the Tsar had been
driven in a carriage at full speed about an hour before along
that very road and that he was dangerously wounded.

``It can't be!'' said Rostov. ``It must have been someone else.''

``I saw him myself,'' replied the man with a self-confident smile
of derision. ``I ought to know the Emperor by now, after the
times I've seen him in Petersburg. I saw him just as I see
you... There he sat in the carriage as pale as anything. How they
made the four black horses fly!  Gracious me, they did rattle
past! It's time I knew the Imperial horses and Ilya Ivanych. I
don't think Ilya drives anyone except the Tsar!''

Rostov let go of the horse and was about to ride on, when a
wounded officer passing by addressed him:

``Who is it you want?'' he asked. ``The commander-in-chief? He
was killed by a cannon ball---struck in the breast before our
regiment.''

``Not killed---wounded!'' another officer corrected him.

``Who? Kutuzov?'' asked Rostov.

``Not Kutuzov, but what's his name---well, never mind... there
are not many left alive. Go that way, to that village, all the
commanders are there,'' said the officer, pointing to the village
of Hosjeradek, and he walked on.

Rostov rode on at a footpace not knowing why or to whom he was
now going. The Emperor was wounded, the battle lost. It was
impossible to doubt it now. Rostov rode in the direction pointed
out to him, in which he saw turrets and a church. What need to
hurry? What was he now to say to the Tsar or to Kutuzov, even if
they were alive and unwounded?

``Take this road, your honor, that way you will be killed at
once!'' a soldier shouted to him. ``They'd kill you there!''

``Oh, what are you talking about?'' said another. ``Where is he
to go? That way is nearer.''

Rostov considered, and then went in the direction where they said
he would be killed.

``It's all the same now. If the Emperor is wounded, am I to try
to save myself?'' he thought. He rode on to the region where the
greatest number of men had perished in fleeing from Pratzen. The
French had not yet occupied that region, and the Russians---the
uninjured and slightly wounded---had left it long ago. All about
the field, like heaps of manure on well-kept plowland, lay from
ten to fifteen dead and wounded to each couple of acres. The
wounded crept together in twos and threes and one could hear
their distressing screams and groans, sometimes feigned---or so
it seemed to Rostov. He put his horse to a trot to avoid seeing
all these suffering men, and he felt afraid---afraid not for his
life, but for the courage he needed and which he knew would not
stand the sight of these unfortunates.

The French, who had ceased firing at this field strewn with dead
and wounded where there was no one left to fire at, on seeing an
adjutant riding over it trained a gun on him and fired several
shots. The sensation of those terrible whistling sounds and of
the corpses around him merged in Rostov's mind into a single
feeling of terror and pity for himself. He remembered his
mother's last letter. ``What would she feel,'' thought he, ``if
she saw me here now on this field with the cannon aimed at me?''

In the village of Hosjeradek there were Russian troops retiring
from the field of battle, who though still in some confusion were
less disordered. The French cannon did not reach there and the
musketry fire sounded far away. Here everyone clearly saw and
said that the battle was lost. No one whom Rostov asked could
tell him where the Emperor or Kutuzov was. Some said the report
that the Emperor was wounded was correct, others that it was not,
and explained the false rumor that had spread by the fact that
the Emperor's carriage had really galloped from the field of
battle with the pale and terrified Ober-Hofmarschal Count
Tolstoy, who had ridden out to the battlefield with others in the
Emperor's suite. One officer told Rostov that he had seen someone
from headquarters behind the village to the left, and thither
Rostov rode, not hoping to find anyone but merely to ease his
conscience. When he had ridden about two miles and had passed the
last of the Russian troops, he saw, near a kitchen garden with a
ditch round it, two men on horseback facing the ditch. One with a
white plume in his hat seemed familiar to Rostov; the other on a
beautiful chestnut horse (which Rostov fancied he had seen
before) rode up to the ditch, struck his horse with his spurs,
and giving it the rein leaped lightly over. Only a little earth
crumbled from the bank under the horse's hind hoofs. Turning the
horse sharply, he again jumped the ditch, and deferentially
addressed the horseman with the white plumes, evidently
suggesting that he should do the same. The rider, whose figure
seemed familiar to Rostov and involuntarily riveted his
attention, made a gesture of refusal with his head and hand and
by that gesture Rostov instantly recognized his lamented and
adored monarch.

``But it can't be he, alone in the midst of this empty field!''
thought Rostov. At that moment Alexander turned his head and
Rostov saw the beloved features that were so deeply engraved on
his memory. The Emperor was pale, his cheeks sunken and his eyes
hollow, but the charm, the mildness of his features, was all the
greater. Rostov was happy in the assurance that the rumors about
the Emperor being wounded were false. He was happy to be seeing
him. He knew that he might and even ought to go straight to him
and give the message Dolgorukov had ordered him to deliver.

But as a youth in love trembles, is unnerved, and dares not utter
the thoughts he has dreamed of for nights, but looks around for
help or a chance of delay and flight when the longed-for moment
comes and he is alone with her, so Rostov, now that he had
attained what he had longed for more than anything else in the
world, did not know how to approach the Emperor, and a thousand
reasons occurred to him why it would be inconvenient, unseemly,
and impossible to do so.

``What! It is as if I were glad of a chance to take advantage of
his being alone and despondent! A strange face may seem
unpleasant or painful to him at this moment of sorrow; besides,
what can I say to him now, when my heart fails me and my mouth
feels dry at the mere sight of him?'' Not one of the innumerable
speeches addressed to the Emperor that he had composed in his
imagination could he now recall. Those speeches were intended for
quite other conditions, they were for the most part to be spoken
at a moment of victory and triumph, generally when he was dying
of wounds and the sovereign had thanked him for heroic deeds, and
while dying he expressed the love his actions had proved.

``Besides how can I ask the Emperor for his instructions for the
right flank now that it is nearly four o'clock and the battle is
lost? No, certainly I must not approach him, I must not intrude
on his reflections. Better die a thousand times than risk
receiving an unkind look or bad opinion from him,'' Rostov
decided; and sorrowfully and with a heart full despair he rode
away, continually looking back at the Tsar, who still remained in
the same attitude of indecision.

While Rostov was thus arguing with himself and riding sadly away,
Captain von Toll chanced to ride to the same spot, and seeing the
Emperor at once rode up to him, offered his services, and
assisted him to cross the ditch on foot. The Emperor, wishing to
rest and feeling unwell, sat down under an apple tree and von
Toll remained beside him.  Rostov from a distance saw with envy
and remorse how von Toll spoke long and warmly to the Emperor and
how the Emperor, evidently weeping, covered his eyes with his
hand and pressed von Toll's hand.

``And I might have been in his place!'' thought Rostov, and
hardly restraining his tears of pity for the Emperor, he rode on
in utter despair, not knowing where to or why he was now riding.

His despair was all the greater from feeling that his own
weakness was the cause of his grief.

He might... not only might but should, have gone up to the
sovereign. It was a unique chance to show his devotion to the
Emperor and he had not made use of it... ``What have I done?''
thought he. And he turned round and galloped back to the place
where he had seen the Emperor, but there was no one beyond the
ditch now. Only some carts and carriages were passing by. From
one of the drivers he learned that Kutuzov's staff were not far
off, in the village the vehicles were going to. Rostov followed
them. In front of him walked Kutuzov's groom leading horses in
horsecloths. Then came a cart, and behind that walked an old,
bandy-legged domestic serf in a peaked cap and sheepskin coat.

``Tit! I say, Tit!'' said the groom.

``What?'' answered the old man absent-mindedly.

``Go, Tit! Thresh a bit!''

``Oh, you fool!'' said the old man, spitting angrily. Some time
passed in silence, and then the same joke was repeated.

Before five in the evening the battle had been lost at all
points. More than a hundred cannon were already in the hands of
the French.

Przebyszewski and his corps had laid down their arms. Other
columns after losing half their men were retreating in disorderly
confused masses.

The remains of Langeron's and Dokhturov's mingled forces were
crowding around the dams and banks of the ponds near the village
of Augesd.

After five o'clock it was only at the Augesd Dam that a hot
cannonade (delivered by the French alone) was still to be heard
from numerous batteries ranged on the slopes of the Pratzen
Heights, directed at our retreating forces.

In the rearguard, Dokhturov and others rallying some battalions
kept up a musketry fire at the French cavalry that was pursuing
our troops. It was growing dusk. On the narrow Augesd Dam where
for so many years the old miller had been accustomed to sit in
his tasseled cap peacefully angling, while his grandson, with
shirt sleeves rolled up, handled the floundering silvery fish in
the watering can, on that dam over which for so many years
Moravians in shaggy caps and blue jackets had peacefully driven
their two-horse carts loaded with wheat and had returned dusty
with flour whitening their carts---on that narrow dam amid the
wagons and the cannon, under the horses' hoofs and between the
wagon wheels, men disfigured by fear of death now crowded
together, crushing one another, dying, stepping over the dying
and killing one another, only to move on a few steps and be
killed themselves in the same way.

Every ten seconds a cannon ball flew compressing the air around,
or a shell burst in the midst of that dense throng, killing some
and splashing with blood those near them.

Dolokhov---now an officer---wounded in the arm, and on foot, with
the regimental commander on horseback and some ten men of his
company, represented all that was left of that whole
regiment. Impelled by the crowd, they had got wedged in at the
approach to the dam and, jammed in on all sides, had stopped
because a horse in front had fallen under a cannon and the crowd
were dragging it out. A cannon ball killed someone behind them,
another fell in front and splashed Dolokhov with blood. The
crowd, pushing forward desperately, squeezed together, moved a
few steps, and again stopped.

``Move on a hundred yards and we are certainly saved, remain here
another two minutes and it is certain death,'' thought each one.

Dolokhov who was in the midst of the crowd forced his way to the
edge of the dam, throwing two soldiers off their feet, and ran
onto the slippery ice that covered the millpool.

``Turn this way!'' he shouted, jumping over the ice which creaked
under him; ``turn this way!'' he shouted to those with the
gun. ``It bears!...''

The ice bore him but it swayed and creaked, and it was plain that
it would give way not only under a cannon or a crowd, but very
soon even under his weight alone. The men looked at him and
pressed to the bank, hesitating to step onto the ice. The general
on horseback at the entrance to the dam raised his hand and
opened his mouth to address Dolokhov. Suddenly a cannon ball
hissed so low above the crowd that everyone ducked. It flopped
into something moist, and the general fell from his horse in a
pool of blood. Nobody gave him a look or thought of raising him.

``Get onto the ice, over the ice! Go on! Turn! Don't you hear? Go
on!''  innumerable voices suddenly shouted after the ball had
struck the general, the men themselves not knowing what, or why,
they were shouting.

One of the hindmost guns that was going onto the dam turned off
onto the ice. Crowds of soldiers from the dam began running onto
the frozen pond.  The ice gave way under one of the foremost
soldiers, and one leg slipped into the water. He tried to right
himself but fell in up to his waist.  The nearest soldiers shrank
back, the gun driver stopped his horse, but from behind still
came the shouts: ``Onto the ice, why do you stop? Go on! Go on!''
And cries of horror were heard in the crowd. The soldiers near
the gun waved their arms and beat the horses to make them turn
and move on. The horses moved off the bank. The ice, that had
held under those on foot, collapsed in a great mass, and some
forty men who were on it dashed, some forward and some back,
drowning one another.

Still the cannon balls continued regularly to whistle and flop
onto the ice and into the water and oftenest of all among the
crowd that covered the dam, the pond, and the bank.

% % % % % % % % % % % % % % % % % % % % % % % % % % % % % % % % %
% % % % % % % % % % % % % % % % % % % % % % % % % % % % % % % % %
% % % % % % % % % % % % % % % % % % % % % % % % % % % % % % % % %
% % % % % % % % % % % % % % % % % % % % % % % % % % % % % % % % %
% % % % % % % % % % % % % % % % % % % % % % % % % % % % % % % % %
% % % % % % % % % % % % % % % % % % % % % % % % % % % % % % % % %
% % % % % % % % % % % % % % % % % % % % % % % % % % % % % % % % %
% % % % % % % % % % % % % % % % % % % % % % % % % % % % % % % % %
% % % % % % % % % % % % % % % % % % % % % % % % % % % % % % % % %
% % % % % % % % % % % % % % % % % % % % % % % % % % % % % % % % %
% % % % % % % % % % % % % % % % % % % % % % % % % % % % % % % % %
% % % % % % % % % % % % % % % % % % % % % % % % % % % % % %

\chapter*{Chapter XIX}
\ifaudio     
\marginpar{
\href{http://ia800208.us.archive.org/14/items/war_and_peace_03_0712_librivox/war_and_peace_03_19_tolstoy_64kb.mp3}{Audio}} 
\fi

\lettrine[lines=2, loversize=0.3, lraise=0]{\initfamily O}{n}
the Pratzen Heights, where he had fallen with the flagstaff in
his hand, lay Prince Andrew Bolkonski bleeding profusely and
unconsciously uttering a gentle, piteous, and childlike moan.

Toward evening he ceased moaning and became quite still. He did
not know how long his unconsciousness lasted. Suddenly he again
felt that he was alive and suffering from a burning, lacerating
pain in his head.

``Where is it, that lofty sky that I did not know till now, but
saw today?'' was his first thought. ``And I did not know this
suffering either,'' he thought. ``Yes, I did not know anything,
anything at all till now. But where am I?''

He listened and heard the sound of approaching horses, and voices
speaking French. He opened his eyes. Above him again was the same
lofty sky with clouds that had risen and were floating still
higher, and between them gleamed blue infinity. He did not turn
his head and did not see those who, judging by the sound of hoofs
and voices, had ridden up and stopped near him.

It was Napoleon accompanied by two aides-de-camp. Bonaparte
riding over the battlefield had given final orders to strengthen
the batteries firing at the Augesd Dam and was looking at the
killed and wounded left on the field.

``Fine men!'' remarked Napoleon, looking at a dead Russian
grenadier, who, with his face buried in the ground and a
blackened nape, lay on his stomach with an already stiffened arm
flung wide.

``The ammunition for the guns in position is exhausted, Your
Majesty,'' said an adjutant who had come from the batteries that
were firing at Augesd.

``Have some brought from the reserve,'' said Napoleon, and having
gone on a few steps he stopped before Prince Andrew, who lay on
his back with the flagstaff that had been dropped beside
him. (The flag had already been taken by the French as a trophy.)

``That's a fine death!'' said Napoleon as he gazed at Bolkonski.

Prince Andrew understood that this was said of him and that it
was Napoleon who said it. He heard the speaker addressed as
Sire. But he heard the words as he might have heard the buzzing
of a fly. Not only did they not interest him, but he took no
notice of them and at once forgot them. His head was burning, he
felt himself bleeding to death, and he saw above him the remote,
lofty, and everlasting sky. He knew it was Napoleon---his
hero---but at that moment Napoleon seemed to him such a small,
insignificant creature compared with what was passing now between
himself and that lofty infinite sky with the clouds flying over
it. At that moment it meant nothing to him who might be standing
over him, or what was said of him; he was only glad that people
were standing near him and only wished that they would help him
and bring him back to life, which seemed to him so beautiful now
that he had today learned to understand it so differently. He
collected all his strength, to stir and utter a sound. He feebly
moved his leg and uttered a weak, sickly groan which aroused his
own pity.

``Ah! He is alive,'' said Napoleon. ``Lift this young man up and
carry him to the dressing station.''

Having said this, Napoleon rode on to meet Marshal Lannes, who,
hat in hand, rode up smiling to the Emperor to congratulate him
on the victory.

Prince Andrew remembered nothing more: he lost consciousness from
the terrible pain of being lifted onto the stretcher, the jolting
while being moved, and the probing of his wound at the dressing
station. He did not regain consciousness till late in the day,
when with other wounded and captured Russian officers he was
carried to the hospital.  During this transfer he felt a little
stronger and was able to look about him and even speak.

The first words he heard on coming to his senses were those of a
French convoy officer, who said rapidly: ``We must halt here: the
Emperor will pass here immediately; it will please him to see
these gentlemen prisoners.''

``There are so many prisoners today, nearly the whole Russian
army, that he is probably tired of them,'' said another officer.

``All the same! They say this one is the commander of all the
Emperor Alexander's Guards,'' said the first one, indicating a
Russian officer in the white uniform of the Horse Guards.

Bolkonski recognized Prince Repnin whom he had met in Petersburg
society. Beside him stood a lad of nineteen, also a wounded
officer of the Horse Guards.

Bonaparte, having come up at a gallop, stopped his horse.

``Which is the senior?'' he asked, on seeing the prisoners.

They named the colonel, Prince Repnin.

``You are the commander of the Emperor Alexander's regiment of
Horse Guards?'' asked Napoleon.

``I commanded a squadron,'' replied Repnin.

``Your regiment fulfilled its duty honorably,'' said Napoleon.

``The praise of a great commander is a soldier's highest
reward,'' said Repnin.

``I bestow it with pleasure,'' said Napoleon. ``And who is that
young man beside you?''

Prince Repnin named Lieutenant Sukhtelen.

After looking at him Napoleon smiled.

``He's very young to come to meddle with us.''

``Youth is no hindrance to courage,'' muttered Sukhtelen in a
failing voice.

``A splendid reply!'' said Napoleon. ``Young man, you will go
far!''

Prince Andrew, who had also been brought forward before the
Emperor's eyes to complete the show of prisoners, could not fail
to attract his attention. Napoleon apparently remembered seeing
him on the battlefield and, addressing him, again used the
epithet ``young man'' that was connected in his memory with
Prince Andrew.

``Well, and you, young man,'' said he. ``How do you feel, mon
brave?''

Though five minutes before, Prince Andrew had been able to say a
few words to the soldiers who were carrying him, now with his
eyes fixed straight on Napoleon, he was silent... So
insignificant at that moment seemed to him all the interests that
engrossed Napoleon, so mean did his hero himself with his paltry
vanity and joy in victory appear, compared to the lofty,
equitable, and kindly sky which he had seen and understood, that
he could not answer him.

Everything seemed so futile and insignificant in comparison with
the stern and solemn train of thought that weakness from loss of
blood, suffering, and the nearness of death aroused in
him. Looking into Napoleon's eyes Prince Andrew thought of the
insignificance of greatness, the unimportance of life which no
one could understand, and the still greater unimportance of
death, the meaning of which no one alive could understand or
explain.

The Emperor without waiting for an answer turned away and said to
one of the officers as he went: ``Have these gentlemen attended
to and taken to my bivouac; let my doctor, Larrey, examine their
wounds. Au revoir, Prince Repnin!'' and he spurred his horse and
galloped away.

His face shone with self-satisfaction and pleasure.

The soldiers who had carried Prince Andrew had noticed and taken
the little gold icon Princess Mary had hung round her brother's
neck, but seeing the favor the Emperor showed the prisoners, they
now hastened to return the holy image.

Prince Andrew did not see how and by whom it was replaced, but
the little icon with its thin gold chain suddenly appeared upon
his chest outside his uniform.

``It would be good,'' thought Prince Andrew, glancing at the icon
his sister had hung round his neck with such emotion and
reverence, ``it would be good if everything were as clear and
simple as it seems to Mary. How good it would be to know where to
seek for help in this life, and what to expect after it beyond
the grave! How happy and calm I should be if I could now say:
'Lord, have mercy on me!'... But to whom should I say that?
Either to a Power indefinable, incomprehensible, which I not only
cannot address but which I cannot even express in words---the
Great All or Nothing---'' said he to himself, ``or to that God who
has been sewn into this amulet by Mary! There is nothing certain,
nothing at all except the unimportance of everything I
understand, and the greatness of something incomprehensible but
all-important.''

The stretchers moved on. At every jolt he again felt unendurable
pain; his feverishness increased and he grew delirious. Visions
of his father, wife, sister, and future son, and the tenderness
he had felt the night before the battle, the figure of the
insignificant little Napoleon, and above all this the lofty sky,
formed the chief subjects of his delirious fancies.

The quiet home life and peaceful happiness of Bald Hills
presented itself to him. He was already enjoying that happiness
when that little Napoleon had suddenly appeared with his
unsympathizing look of shortsighted delight at the misery of
others, and doubts and torments had followed, and only the
heavens promised peace. Toward morning all these dreams melted
and merged into the chaos and darkness of unconciousness and
oblivion which in the opinion of Napoleon's doctor, Larrey, was
much more likely to end in death than in convalescence.

``He is a nervous, bilious subject,'' said Larrey, ``and will not
recover.''

And Prince Andrew, with others fatally wounded, was left to the
care of the inhabitants of the district.

