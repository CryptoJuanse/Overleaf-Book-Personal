\part*{Book Seven: 1810 - 11}

% % % % % % % % % % % % % % % % % % % % % % % % % % % % % % % % %
% % % % % % % % % % % % % % % % % % % % % % % % % % % % % % % % %
% % % % % % % % % % % % % % % % % % % % % % % % % % % % % % % % %
% % % % % % % % % % % % % % % % % % % % % % % % % % % % % % % % %
% % % % % % % % % % % % % % % % % % % % % % % % % % % % % % % % %
% % % % % % % % % % % % % % % % % % % % % % % % % % % % % % % % %
% % % % % % % % % % % % % % % % % % % % % % % % % % % % % % % % %
% % % % % % % % % % % % % % % % % % % % % % % % % % % % % % % % %
% % % % % % % % % % % % % % % % % % % % % % % % % % % % % % % % %
% % % % % % % % % % % % % % % % % % % % % % % % % % % % % % % % %
% % % % % % % % % % % % % % % % % % % % % % % % % % % % % % % % %
% % % % % % % % % % % % % % % % % % % % % % % % % % % % % %

\chapter*{Chapter I}
\ifaudio
\marginpar{
\href{http://ia802705.us.archive.org/22/items/war_and_peace_07_0808_librivox/war_and_peace_07_01_tolstoy_64kb.mp3}{Audio}} 
\fi

\lettrine[lines=2, loversize=0.3, lraise=0]{\initfamily T}{he}
Bible legend tells us that the absence of
labor---idleness---was a condition of the first man's blessedness
before the Fall. Fallen man has retained a love of idleness, but
the curse weighs on the race not only because we have to seek our
bread in the sweat of our brows, but because our moral nature is
such that we cannot be both idle and at ease. An inner voice
tells us we are in the wrong if we are idle. If man could find a
state in which he felt that though idle he was fulfilling his
duty, he would have found one of the conditions of man's
primitive blessedness. And such a state of obligatory and
irreproachable idleness is the lot of a whole class---the
military. The chief attraction of military service has consisted
and will consist in this compulsory and irreproachable idleness.

Nicholas Rostov experienced this blissful condition to the full
when, after 1807, he continued to serve in the Pavlograd
regiment, in which he already commanded the squadron he had taken
over from Denisov.

Rostov had become a bluff, good-natured fellow, whom his Moscow
acquaintances would have considered rather bad form, but who was
liked and respected by his comrades, subordinates, and superiors,
and was well contented with his life. Of late, in 1809, he found
in letters from home more frequent complaints from his mother
that their affairs were falling into greater and greater
disorder, and that it was time for him to come back to gladden
and comfort his old parents.

Reading these letters, Nicholas felt a dread of their wanting to
take him away from surroundings in which, protected from all the
entanglements of life, he was living so calmly and quietly. He
felt that sooner or later he would have to re-enter that
whirlpool of life, with its embarrassments and affairs to be
straightened out, its accounts with stewards, quarrels, and
intrigues, its ties, society, and with Sonya's love and his
promise to her. It was all dreadfully difficult and complicated;
and he replied to his mother in cold, formal letters in French,
beginning: ``My dear Mamma,'' and ending: ``Your obedient son,''
which said nothing of when he would return. In 1810 he received
letters from his parents, in which they told him of Natasha's
engagement to Bolkonski, and that the wedding would be in a
year's time because the old prince made difficulties. This letter
grieved and mortified Nicholas. In the first place he was sorry
that Natasha, for whom he cared more than for anyone else in the
family, should be lost to the home; and secondly, from his hussar
point of view, he regretted not to have been there to show that
fellow Bolkonski that connection with him was no such great honor
after all, and that if he loved Natasha he might dispense with
permission from his dotard father. For a moment he hesitated
whether he should not apply for leave in order to see Natasha
before she was married, but then came the maneuvers, and
considerations about Sonya and about the confusion of their
affairs, and Nicholas again put it off. But in the spring of that
year, he received a letter from his mother, written without his
father's knowledge, and that letter persuaded him to return. She
wrote that if he did not come and take matters in hand, their
whole property would be sold by auction and they would all have
to go begging. The count was so weak, and trusted Mitenka so
much, and was so good-natured, that everybody took advantage of
him and things were going from bad to worse. ``For God's sake, I
implore you, come at once if you do not wish to make me and the
whole family wretched,'' wrote the countess.

This letter touched Nicholas. He had that common sense of a
matter-of-fact man which showed him what he ought to do.

The right thing now was, if not to retire from the service, at
any rate to go home on leave. Why he had to go he did not know;
but after his after-dinner nap he gave orders to saddle Mars, an
extremely vicious gray stallion that had not been ridden for a
long time, and when he returned with the horse all in a lather,
he informed Lavrushka (Denisov's servant who had remained with
him) and his comrades who turned up in the evening that he was
applying for leave and was going home. Difficult and strange as
it was for him to reflect that he would go away without having
heard from the staff---and this interested him
extremely---whether he was promoted to a captaincy or would
receive the Order of St. Anne for the last maneuvers; strange as
it was to think that he would go away without having sold his
three roans to the Polish Count Golukhovski, who was bargaining
for the horses Rostov had betted he would sell for two thousand
rubles; incomprehensible as it seemed that the ball the hussars
were giving in honor of the Polish Mademoiselle Przazdziecka (out
of rivalry to the uhlans who had given one in honor of their
Polish Mademoiselle Borzozowska) would take place without
him---he knew he must go away from this good, bright world to
somewhere where everything was stupid and confused. A week later
he obtained his leave. His hussar comrades---not only those of
his own regiment, but the whole brigade---gave Rostov a dinner to
which the subscription was fifteen rubles a head, and at which
there were two bands and two choirs of singers. Rostov danced the
Trepak with Major Basov; the tipsy officers tossed, embraced, and
dropped Rostov; the soldiers of the third squadron tossed him
too, and shouted ``hurrah!'' and then they put him in his sleigh
and escorted him as far as the first post station.

During the first half of the journey---from Kremenchug to
Kiev---all Rostov's thoughts, as is usual in such cases, were
behind him, with the squadron; but when he had gone more than
halfway he began to forget his three roans and Dozhoyveyko, his
quartermaster, and to wonder anxiously how things would be at
Otradnoe and what he would find there. Thoughts of home grew
stronger the nearer he approached it---far stronger, as though
this feeling of his was subject to the law by which the force of
attraction is in inverse proportion to the square of the
distance. At the last post station before Otradnoe he gave the
driver a three-ruble tip, and on arriving he ran breathlessly,
like a boy, up the steps of his home.

After the rapture of meeting, and after that odd feeling of
unsatisfied expectation---the feeling that ``everything is just
the same, so why did I hurry?''---Nicholas began to settle down
in his old home world. His father and mother were much the same,
only a little older. What was new in them was a certain
uneasiness and occasional discord, which there used not to be,
and which, as Nicholas soon found out, was due to the bad state
of their affairs. Sonya was nearly twenty; she had stopped
growing prettier and promised nothing more than she was already,
but that was enough. She exhaled happiness and love from the time
Nicholas returned, and the faithful, unalterable love of this
girl had a gladdening effect on him.  Petya and Natasha surprised
Nicholas most. Petya was a big handsome boy of thirteen, merry,
witty, and mischievous, with a voice that was already
breaking. As for Natasha, for a long while Nicholas wondered and
laughed whenever he looked at her.

``You're not the same at all,'' he said.

``How? Am I uglier?''

``On the contrary, but what dignity? A princess!'' he whispered
to her.

``Yes, yes, yes!'' cried Natasha, joyfully.

She told him about her romance with Prince Andrew and of his
visit to Otradnoe and showed him his last letter.

``Well, are you glad?'' Natasha asked. ``I am so tranquil and
happy now.''

``Very glad,'' answered Nicholas. ``He is an excellent
fellow... And are you very much in love?''

``How shall I put it?'' replied Natasha. ``I was in love with
Boris, with my teacher, and with Denisov, but this is quite
different. I feel at peace and settled. I know that no better man
than he exists, and I am calm and contented now. Not at all as
before.''

Nicholas expressed his disapproval of the postponement of the
marriage for a year; but Natasha attacked her brother with
exasperation, proving to him that it could not be otherwise, and
that it would be a bad thing to enter a family against the
father's will, and that she herself wished it so.

``You don't at all understand,'' she said.

Nicholas was silent and agreed with her.

Her brother often wondered as he looked at her. She did not seem
at all like a girl in love and parted from her affianced
husband. She was even-tempered and calm and quite as cheerful as
of old. This amazed Nicholas and even made him regard Bolkonski's
courtship skeptically. He could not believe that her fate was
sealed, especially as he had not seen her with Prince Andrew. It
always seemed to him that there was something not quite right
about this intended marriage.

``Why this delay? Why no betrothal?'' he thought. Once, when he
had touched on this topic with his mother, he discovered, to his
surprise and somewhat to his satisfaction, that in the depth of
her soul she too had doubts about this marriage.

``You see he writes,'' said she, showing her son a letter of
Prince Andrew's, with that latent grudge a mother always has in
regard to a daughter's future married happiness, ``he writes that
he won't come before December. What can be keeping him? Illness,
probably! His health is very delicate. Don't tell Natasha. And
don't attach importance to her being so bright: that's because
she's living through the last days of her girlhood, but I know
what she is like every time we receive a letter from him!
However, God grant that everything turns out well!'' (She always
ended with these words.) ``He is an excellent man!''

% % % % % % % % % % % % % % % % % % % % % % % % % % % % % % % % %
% % % % % % % % % % % % % % % % % % % % % % % % % % % % % % % % %
% % % % % % % % % % % % % % % % % % % % % % % % % % % % % % % % %
% % % % % % % % % % % % % % % % % % % % % % % % % % % % % % % % %
% % % % % % % % % % % % % % % % % % % % % % % % % % % % % % % % %
% % % % % % % % % % % % % % % % % % % % % % % % % % % % % % % % %
% % % % % % % % % % % % % % % % % % % % % % % % % % % % % % % % %
% % % % % % % % % % % % % % % % % % % % % % % % % % % % % % % % %
% % % % % % % % % % % % % % % % % % % % % % % % % % % % % % % % %
% % % % % % % % % % % % % % % % % % % % % % % % % % % % % % % % %
% % % % % % % % % % % % % % % % % % % % % % % % % % % % % % % % %
% % % % % % % % % % % % % % % % % % % % % % % % % % % % % %

\chapter*{Chapter II}
\ifaudio     
\marginpar{
\href{http://ia802705.us.archive.org/22/items/war_and_peace_07_0808_librivox/war_and_peace_07_02_tolstoy_64kb.mp3}{Audio}} 
\fi

\lettrine[lines=2, loversize=0.3, lraise=0]{\initfamily A}{fter}
reaching home Nicholas was at first serious and even
dull. He was worried by the impending necessity of interfering in
the stupid business matters for which his mother had called him
home. To throw off this burden as quickly as possible, on the
third day after his arrival he went, angry and scowling and
without answering questions as to where he was going, to
Mitenka's lodge and demanded an account of everything. But what
an account of everything might be Nicholas knew even less than
the frightened and bewildered Mitenka. The conversation and the
examination of the accounts with Mitenka did not last long. The
village elder, a peasant delegate, and the village clerk, who
were waiting in the passage, heard with fear and delight first
the young count's voice roaring and snapping and rising louder
and louder, and then words of abuse, dreadful words, ejaculated
one after the other.

``Robber!... Ungrateful wretch!... I'll hack the dog to pieces!
I'm not my father!... Robbing us!...'' and so on.

Then with no less fear and delight they saw how the young count,
red in the face and with bloodshot eyes, dragged Mitenka out by
the scruff of the neck and applied his foot and knee to his
behind with great agility at convenient moments between the
words, shouting, ``Be off! Never let me see your face here again,
you villain!''

Mitenka flew headlong down the six steps and ran away into the
shrubbery. (This shrubbery was a well-known haven of refuge for
culprits at Otradnoe. Mitenka himself, returning tipsy from the
town, used to hide there, and many of the residents at Otradnoe,
hiding from Mitenka, knew of its protective qualities.)

Mitenka's wife and sisters-in-law thrust their heads and
frightened faces out of the door of a room where a bright samovar
was boiling and where the steward's high bedstead stood with its
patchwork quilt.

The young count paid no heed to them, but, breathing hard, passed
by with resolute strides and went into the house.

The countess, who heard at once from the maids what had happened
at the lodge, was calmed by the thought that now their affairs
would certainly improve, but on the other hand felt anxious as to
the effect this excitement might have on her son. She went
several times to his door on tiptoe and listened, as he lighted
one pipe after another.

Next day the old count called his son aside and, with an
embarrassed smile, said to him:

``But you know, my dear boy, it's a pity you got excited! Mitenka
has told me all about it.''

``I knew,'' thought Nicholas, ``that I should never understand
anything in this crazy world.''

``You were angry that he had not entered those 700 rubles. But
they were carried forward---and you did not look at the other
page.''

``Papa, he is a blackguard and a thief! I know he is! And what I
have done, I have done; but, if you like, I won't speak to him
again.''

``No, my dear boy'' (the count, too, felt embarrassed. He knew he
had mismanaged his wife's property and was to blame toward his
children, but he did not know how to remedy it). ``No, I beg you
to attend to the business. I am old. I...''

``No, Papa. Forgive me if I have caused you unpleasantness. I
understand it all less than you do.''

``Devil take all these peasants, and money matters, and carryings
forward from page to page,'' he thought. ``I used to understand
what a 'corner' and the stakes at cards meant, but carrying
forward to another page I don't understand at all,'' said he to
himself, and after that he did not meddle in business
affairs. But once the countess called her son and informed him
that she had a promissory note from Anna Mikhaylovna for two
thousand rubles, and asked him what he thought of doing with it.

``This,'' answered Nicholas. ``You say it rests with me. Well, I
don't like Anna Mikhaylovna and I don't like Boris, but they were
our friends and poor. Well then, this!'' and he tore up the note,
and by so doing caused the old countess to weep tears of
joy. After that, young Rostov took no further part in any
business affairs, but devoted himself with passionate enthusiasm
to what was to him a new pursuit---the chase---for which his
father kept a large establishment.

% % % % % % % % % % % % % % % % % % % % % % % % % % % % % % % % %
% % % % % % % % % % % % % % % % % % % % % % % % % % % % % % % % %
% % % % % % % % % % % % % % % % % % % % % % % % % % % % % % % % %
% % % % % % % % % % % % % % % % % % % % % % % % % % % % % % % % %
% % % % % % % % % % % % % % % % % % % % % % % % % % % % % % % % %
% % % % % % % % % % % % % % % % % % % % % % % % % % % % % % % % %
% % % % % % % % % % % % % % % % % % % % % % % % % % % % % % % % %
% % % % % % % % % % % % % % % % % % % % % % % % % % % % % % % % %
% % % % % % % % % % % % % % % % % % % % % % % % % % % % % % % % %
% % % % % % % % % % % % % % % % % % % % % % % % % % % % % % % % %
% % % % % % % % % % % % % % % % % % % % % % % % % % % % % % % % %
% % % % % % % % % % % % % % % % % % % % % % % % % % % % % %

\chapter*{Chapter III}
\ifaudio     
\marginpar{
\href{http://ia802705.us.archive.org/22/items/war_and_peace_07_0808_librivox/war_and_peace_07_03_tolstoy_64kb.mp3}{Audio}} 
\fi

\lettrine[lines=2, loversize=0.3, lraise=0]{\initfamily T}{he}
weather was already growing wintry and morning frosts
congealed an earth saturated by autumn rains. The verdure had
thickened and its bright green stood out sharply against the
brownish strips of winter rye trodden down by the cattle, and
against the pale-yellow stubble of the spring buckwheat. The
wooded ravines and the copses, which at the end of August had
still been green islands amid black fields and stubble, had
become golden and bright-red islands amid the green winter
rye. The hares had already half changed their summer coats, the
fox cubs were beginning to scatter, and the young wolves were
bigger than dogs. It was the best time of the year for the
chase. The hounds of that ardent young sportsman Rostov had not
merely reached hard winter condition, but were so jaded that at a
meeting of the huntsmen it was decided to give them a three days'
rest and then, on the sixteenth of September, to go on a distant
expedition, starting from the oak grove where there was an
undisturbed litter of wolf cubs.

All that day the hounds remained at home. It was frosty and the
air was sharp, but toward evening the sky became overcast and it
began to thaw.  On the fifteenth, when young Rostov, in his
dressing gown, looked out of the window, he saw it was an
unsurpassable morning for hunting: it was as if the sky were
melting and sinking to the earth without any wind.  The only
motion in the air was that of the dripping, microscopic particles
of drizzling mist. The bare twigs in the garden were hung with
transparent drops which fell on the freshly fallen leaves. The
earth in the kitchen garden looked wet and black and glistened
like poppy seed and at a short distance merged into the dull,
moist veil of mist.  Nicholas went out into the wet and muddy
porch. There was a smell of decaying leaves and of dog. Milka, a
black-spotted, broad-haunched bitch with prominent black eyes,
got up on seeing her master, stretched her hind legs, lay down
like a hare, and then suddenly jumped up and licked him right on
his nose and mustache. Another borzoi, a dog, catching sight of
his master from the garden path, arched his back and, rushing
headlong toward the porch with lifted tail, began rubbing himself
against his legs.

``O-hoy!'' came at that moment, that inimitable huntsman's call
which unites the deepest bass with the shrillest tenor, and round
the corner came Daniel the head huntsman and head kennelman, a
gray, wrinkled old man with hair cut straight over his forehead,
Ukrainian fashion, a long bent whip in his hand, and that look of
independence and scorn of everything that is only seen in
huntsmen. He doffed his Circassian cap to his master and looked
at him scornfully. This scorn was not offensive to his
master. Nicholas knew that this Daniel, disdainful of everybody
and who considered himself above them, was all the same his serf
and huntsman.

``Daniel!'' Nicholas said timidly, conscious at the sight of the
weather, the hounds, and the huntsman that he was being carried
away by that irresistible passion for sport which makes a man
forget all his previous resolutions, as a lover forgets in the
presence of his mistress.

``What orders, your excellency?'' said the huntsman in his deep
bass, deep as a proto-deacon's and hoarse with hallooing---and
two flashing black eyes gazed from under his brows at his master,
who was silent. ``Can you resist it?'' those eyes seemed to be
asking.

``It's a good day, eh? For a hunt and a gallop, eh?'' asked
Nicholas, scratching Milka behind the ears.

Daniel did not answer, but winked instead.

``I sent Uvarka at dawn to listen,'' his bass boomed out after a
minute's pause. ``He says she's moved them into the Otradnoe
enclosure. They were howling there.'' (This meant that the
she-wolf, about whom they both knew, had moved with her cubs to
the Otradnoe copse, a small place a mile and a half from the
house.)

``We ought to go, don't you think so?'' said Nicholas. ``Come to
me with Uvarka.''

``As you please.''

``Then put off feeding them.''

``Yes, sir.''

Five minutes later Daniel and Uvarka were standing in Nicholas'
big study. Though Daniel was not a big man, to see him in a room
was like seeing a horse or a bear on the floor among the
furniture and surroundings of human life. Daniel himself felt
this, and as usual stood just inside the door, trying to speak
softly and not move, for fear of breaking something in the
master's apartment, and he hastened to say all that was necessary
so as to get from under that ceiling, out into the open under the
sky once more.

Having finished his inquiries and extorted from Daniel an opinion
that the hounds were fit (Daniel himself wished to go hunting),
Nicholas ordered the horses to be saddled. But just as Daniel was
about to go Natasha came in with rapid steps, not having done up
her hair or finished dressing and with her old nurse's big shawl
wrapped round her.  Petya ran in at the same time.

``You are going?'' asked Natasha. ``I knew you would! Sonya said
you wouldn't go, but I knew that today is the sort of day when
you couldn't help going.''

``Yes, we are going,'' replied Nicholas reluctantly, for today,
as he intended to hunt seriously, he did not want to take Natasha
and Petya.  ``We are going, but only wolf hunting: it would be
dull for you.''

``You know it is my greatest pleasure,'' said Natasha. ``It's not
fair; you are going by yourself, are having the horses saddled
and said nothing to us about it.''

``'No barrier bars a Russian's path'---we'll go!'' shouted Petya.

``But you can't. Mamma said you mustn't,'' said Nicholas to
Natasha.

``Yes, I'll go. I shall certainly go,'' said Natasha
decisively. ``Daniel, tell them to saddle for us, and Michael
must come with my dogs,'' she added to the huntsman.

It seemed to Daniel irksome and improper to be in a room at all,
but to have anything to do with a young lady seemed to him
impossible. He cast down his eyes and hurried out as if it were
none of his business, careful as he went not to inflict any
accidental injury on the young lady.

% % % % % % % % % % % % % % % % % % % % % % % % % % % % % % % % %
% % % % % % % % % % % % % % % % % % % % % % % % % % % % % % % % %
% % % % % % % % % % % % % % % % % % % % % % % % % % % % % % % % %
% % % % % % % % % % % % % % % % % % % % % % % % % % % % % % % % %
% % % % % % % % % % % % % % % % % % % % % % % % % % % % % % % % %
% % % % % % % % % % % % % % % % % % % % % % % % % % % % % % % % %
% % % % % % % % % % % % % % % % % % % % % % % % % % % % % % % % %
% % % % % % % % % % % % % % % % % % % % % % % % % % % % % % % % %
% % % % % % % % % % % % % % % % % % % % % % % % % % % % % % % % %
% % % % % % % % % % % % % % % % % % % % % % % % % % % % % % % % %
% % % % % % % % % % % % % % % % % % % % % % % % % % % % % % % % %
% % % % % % % % % % % % % % % % % % % % % % % % % % % % % %

\chapter*{Chapter IV}
\ifaudio     
\marginpar{
\href{http://ia802705.us.archive.org/22/items/war_and_peace_07_0808_librivox/war_and_peace_07_04_tolstoy_64kb.mp3}{Audio}} 
\fi

\lettrine[lines=2, loversize=0.3, lraise=0]{\initfamily T}{he}
old count, who had always kept up an enormous hunting
establishment but had now handed it all completely over to his
son's care, being in very good spirits on this fifteenth of
September, prepared to go out with the others.

In an hour's time the whole hunting party was at the
porch. Nicholas, with a stern and serious air which showed that
now was no time for attending to trifles, went past Natasha and
Petya who were trying to tell him something. He had a look at all
the details of the hunt, sent a pack of hounds and huntsmen on
ahead to find the quarry, mounted his chestnut Donets, and
whistling to his own leash of borzois, set off across the
threshing ground to a field leading to the Otradnoe wood. The old
count's horse, a sorrel gelding called Viflyanka, was led by the
groom in attendance on him, while the count himself was to drive
in a small trap straight to a spot reserved for him.

They were taking fifty-four hounds, with six hunt attendants and
whippers-in. Besides the family, there were eight borzoi
kennelmen and more than forty borzois, so that, with the borzois
on the leash belonging to members of the family, there were about
a hundred and thirty dogs and twenty horsemen.

Each dog knew its master and its call. Each man in the hunt knew
his business, his place, what he had to do. As soon as they had
passed the fence they all spread out evenly and quietly, without
noise or talk, along the road and field leading to the Otradnoe
covert.

The horses stepped over the field as over a thick carpet, now and
then splashing into puddles as they crossed a road. The misty sky
still seemed to descend evenly and imperceptibly toward the
earth, the air was still, warm, and silent. Occasionally the
whistle of a huntsman, the snort of a horse, the crack of a whip,
or the whine of a straggling hound could be heard.

When they had gone a little less than a mile, five more riders
with dogs appeared out of the mist, approaching the Rostovs. In
front rode a fresh-looking, handsome old man with a large gray
mustache.

``Good morning, Uncle!'' said Nicholas, when the old man drew
near.

``That's it. Come on!... I was sure of it,'' began ``Uncle.'' (He
was a distant relative of the Rostovs', a man of small means, and
their neighbor.) ``I knew you wouldn't be able to resist it and
it's a good thing you're going. That's it! Come on!'' (This was
``Uncle's'' favorite expression.) ``Take the covert at once, for
my Girchik says the Ilagins are at Korniki with their
hounds. That's it. Come on!... They'll take the cubs from under
your very nose.''

``That's where I'm going. Shall we join up our packs?'' asked
Nicholas.

The hounds were joined into one pack, and \emph{Uncle} and
Nicholas rode on side by side. Natasha, muffled up in shawls
which did not hide her eager face and shining eyes, galloped up
to them. She was followed by Petya who always kept close to her,
by Michael, a huntsman, and by a groom appointed to look after
her. Petya, who was laughing, whipped and pulled at his
horse. Natasha sat easily and confidently on her black Arabchik
and reined him in without effort with a firm hand.

\emph{Uncle} looked round disapprovingly at Petya and Natasha. He
did not like to combine frivolity with the serious business of
hunting.

``Good morning, Uncle! We are going too!'' shouted Petya.

``Good morning, good morning! But don't go overriding the
hounds,'' said \emph{Uncle} sternly.

``Nicholas, what a fine dog Trunila is! He knew me,'' said
Natasha, referring to her favorite hound.

``In the first place, Trunila is not a 'dog,' but a harrier,''
thought Nicholas, and looked sternly at his sister, trying to
make her feel the distance that ought to separate them at that
moment. Natasha understood it.

``You mustn't think we'll be in anyone's way, Uncle,'' she
said. ``We'll go to our places and won't budge.''

``A good thing too, little countess,'' said \emph{Uncle}, ``only
mind you don't fall off your horse,'' he added,
``because---that's it, come on!---you've nothing to hold on to.''

The oasis of the Otradnoe covert came in sight a few hundred
yards off, the huntsmen were already nearing it. Rostov, having
finally settled with \emph{Uncle} where they should set on the
hounds, and having shown Natasha where she was to stand---a spot
where nothing could possibly run out---went round above the
ravine.

``Well, nephew, you're going for a big wolf,'' said \emph{Uncle}.
``Mind and don't let her slip!''

``That's as may happen,'' answered Rostov. ``Karay, here!'' he
shouted, answering \emph{Uncle's} remark by this call to his
borzoi. Karay was a shaggy old dog with a hanging jowl, famous
for having tackled a big wolf unaided. They all took up their
places.

The old count, knowing his son's ardor in the hunt, hurried so as
not to be late, and the huntsmen had not yet reached their places
when Count Ilya Rostov, cheerful, flushed, and with quivering
cheeks, drove up with his black horses over the winter rye to the
place reserved for him, where a wolf might come out. Having
straightened his coat and fastened on his hunting knives and
horn, he mounted his good, sleek, well-fed, and comfortable
horse, Viflyanka, which was turning gray, like himself.  His
horses and trap were sent home. Count Ilya Rostov, though not at
heart a keen sportsman, knew the rules of the hunt well, and rode
to the bushy edge of the road where he was to stand, arranged his
reins, settled himself in the saddle, and, feeling that he was
ready, looked about with a smile.

Beside him was Simon Chekmar, his personal attendant, an old
horseman now somewhat stiff in the saddle. Chekmar held in leash
three formidable wolfhounds, who had, however, grown fat like
their master and his horse.  Two wise old dogs lay down
unleashed. Some hundred paces farther along the edge of the wood
stood Mitka, the count's other groom, a daring horseman and keen
rider to hounds. Before the hunt, by old custom, the count had
drunk a silver cupful of mulled brandy, taken a snack, and washed
it down with half a bottle of his favorite Bordeaux.

He was somewhat flushed with the wine and the drive. His eyes
were rather moist and glittered more than usual, and as he sat in
his saddle, wrapped up in his fur coat, he looked like a child
taken out for an outing.

The thin, hollow-cheeked Chekmar, having got everything ready,
kept glancing at his master with whom he had lived on the best of
terms for thirty years, and understanding the mood he was in
expected a pleasant chat. A third person rode up circumspectly
through the wood (it was plain that he had had a lesson) and
stopped behind the count. This person was a gray-bearded old man
in a woman's cloak, with a tall peaked cap on his head. He was
the buffoon, who went by a woman's name, Nastasya Ivanovna.

``Well, Nastasya Ivanovna!'' whispered the count, winking at
him. ``If you scare away the beast, Daniel'll give it you!''

``I know a thing or two myself!'' said Nastasya Ivanovna.

``Hush!'' whispered the count and turned to Simon. ``Have you
seen the young countess?'' he asked. ``Where is she?''

``With young Count Peter, by the Zharov rank grass,'' answered
Simon, smiling. ``Though she's a lady, she's very fond of
hunting.''

``And you're surprised at the way she rides, Simon, eh?'' said
the count.  ``She's as good as many a man!''

``Of course! It's marvelous. So bold, so easy!''

``And Nicholas? Where is he? By the Lyadov upland, isn't he?''

``Yes, sir. He knows where to stand. He understands the matter so
well that Daniel and I are often quite astounded,'' said Simon,
well knowing what would please his master.

``Rides well, eh? And how well he looks on his horse, eh?''

``A perfect picture! How he chased a fox out of the rank grass by
the Zavarzinsk thicket the other day! Leaped a fearful place;
what a sight when they rushed from the covert... the horse worth
a thousand rubles and the rider beyond all price! Yes, one would
have to search far to find another as smart.''

``To search far...'' repeated the count, evidently sorry Simon
had not said more. ``To search far,'' he said, turning back the
skirt of his coat to get at his snuffbox.

``The other day when he came out from Mass in full uniform,
Michael Sidorych...'' Simon did not finish, for on the still air
he had distinctly caught the music of the hunt with only two or
three hounds giving tongue. He bent down his head and listened,
shaking a warning finger at his master. ``They are on the scent
of the cubs...'' he whispered, ``straight to the Lyadov
uplands.''

The count, forgetting to smooth out the smile on his face, looked
into the distance straight before him, down the narrow open
space, holding the snuffbox in his hand but not taking any. After
the cry of the hounds came the deep tones of the wolf call from
Daniel's hunting horn; the pack joined the first three hounds and
they could be heard in full cry, with that peculiar lift in the
note that indicates that they are after a wolf. The whippers-in
no longer set on the hounds, but changed to the cry of ulyulyu,
and above the others rose Daniel's voice, now a deep bass, now
piercingly shrill. His voice seemed to fill the whole wood and
carried far beyond out into the open field.

After listening a few moments in silence, the count and his
attendant convinced themselves that the hounds had separated into
two packs: the sound of the larger pack, eagerly giving tongue,
began to die away in the distance, the other pack rushed by the
wood past the count, and it was with this that Daniel's voice was
heard calling ulyulyu. The sounds of both packs mingled and broke
apart again, but both were becoming more distant.

Simon sighed and stooped to straighten the leash a young borzoi
had entangled; the count too sighed and, noticing the snuffbox in
his hand, opened it and took a pinch. ``Back!'' cried Simon to a
borzoi that was pushing forward out of the wood. The count
started and dropped the snuffbox. Nastasya Ivanovna dismounted to
pick it up. The count and Simon were looking at him.

Then, unexpectedly, as often happens, the sound of the hunt
suddenly approached, as if the hounds in full cry and Daniel
ulyulyuing were just in front of them.

The count turned and saw on his right Mitka staring at him with
eyes starting out of his head, raising his cap and pointing
before him to the other side.

``Look out!'' he shouted, in a voice plainly showing that he had
long fretted to utter that word, and letting the borzois slip he
galloped toward the count.

The count and Simon galloped out of the wood and saw on their
left a wolf which, softly swaying from side to side, was coming
at a quiet lope farther to the left to the very place where they
were standing. The angry borzois whined and getting free of the
leash rushed past the horses' feet at the wolf.

The wolf paused, turned its heavy forehead toward the dogs
awkwardly, like a man suffering from the quinsy, and, still
slightly swaying from side to side, gave a couple of leaps and
with a swish of its tail disappeared into the skirt of the
wood. At the same instant, with a cry like a wail, first one
hound, then another, and then another, sprang helter-skelter from
the wood opposite and the whole pack rushed across the field
toward the very spot where the wolf had disappeared. The hazel
bushes parted behind the hounds and Daniel's chestnut horse
appeared, dark with sweat. On its long back sat Daniel, hunched
forward, capless, his disheveled gray hair hanging over his
flushed, perspiring face.

``Ulyulyulyu! ulyulyu!...'' he cried. When he caught sight of the
count his eyes flashed lightning.

``Blast you!'' he shouted, holding up his whip threateningly at
the count.

``You've let the wolf go!... What sportsmen!'' and as if scorning
to say more to the frightened and shamefaced count, he lashed the
heaving flanks of his sweating chestnut gelding with all the
anger the count had aroused and flew off after the hounds. The
count, like a punished schoolboy, looked round, trying by a smile
to win Simon's sympathy for his plight. But Simon was no longer
there. He was galloping round by the bushes while the field was
coming up on both sides, all trying to head the wolf, but it
vanished into the wood before they could do so.

% % % % % % % % % % % % % % % % % % % % % % % % % % % % % % % % %
% % % % % % % % % % % % % % % % % % % % % % % % % % % % % % % % %
% % % % % % % % % % % % % % % % % % % % % % % % % % % % % % % % %
% % % % % % % % % % % % % % % % % % % % % % % % % % % % % % % % %
% % % % % % % % % % % % % % % % % % % % % % % % % % % % % % % % %
% % % % % % % % % % % % % % % % % % % % % % % % % % % % % % % % %
% % % % % % % % % % % % % % % % % % % % % % % % % % % % % % % % %
% % % % % % % % % % % % % % % % % % % % % % % % % % % % % % % % %
% % % % % % % % % % % % % % % % % % % % % % % % % % % % % % % % %
% % % % % % % % % % % % % % % % % % % % % % % % % % % % % % % % %
% % % % % % % % % % % % % % % % % % % % % % % % % % % % % % % % %
% % % % % % % % % % % % % % % % % % % % % % % % % % % % % %

\chapter*{Chapter V}
\ifaudio     
\marginpar{
\href{http://ia802705.us.archive.org/22/items/war_and_peace_07_0808_librivox/war_and_peace_07_05_tolstoy_64kb.mp3}{Audio}} 
\fi

\lettrine[lines=2, loversize=0.3, lraise=0]{\initfamily N}{icholas}
Rostov meanwhile remained at his post, waiting for the
wolf. By the way the hunt approached and receded, by the cries of
the dogs whose notes were familiar to him, by the way the voices
of the huntsmen approached, receded, and rose, he realized what
was happening at the copse. He knew that young and old wolves
were there, that the hounds had separated into two packs, that
somewhere a wolf was being chased, and that something had gone
wrong. He expected the wolf to come his way any moment. He made
thousands of different conjectures as to where and from what side
the beast would come and how he would set upon it. Hope
alternated with despair. Several times he addressed a prayer to
God that the wolf should come his way. He prayed with that
passionate and shamefaced feeling with which men pray at moments
of great excitement arising from trivial causes. ``What would it
be to Thee to do this for me?'' he said to God. ``I know Thou art
great, and that it is a sin to ask this of Thee, but for God's
sake do let the old wolf come my way and let Karay spring at
it---in sight of \emph{Uncle} who is watching from over
there---and seize it by the throat in a death grip!'' A thousand
times during that half-hour Rostov cast eager and restless
glances over the edge of the wood, with the two scraggy oaks
rising above the aspen undergrowth and the gully with its
water-worn side and \emph{Uncle's} cap just visible above the
bush on his right.

``No, I shan't have such luck,'' thought Rostov, ``yet what
wouldn't it be worth! It is not to be! Everywhere, at cards and
in war, I am always unlucky.'' Memories of Austerlitz and of
Dolokhov flashed rapidly and clearly through his mind. ``Only
once in my life to get an old wolf, I want only that!'' thought
he, straining eyes and ears and looking to the left and then to
the right and listening to the slightest variation of note in the
cries of the dogs.

Again he looked to the right and saw something running toward him
across the deserted field. ``No, it can't be!'' thought Rostov,
taking a deep breath, as a man does at the coming of something
long hoped for. The height of happiness was reached---and so
simply, without warning, or noise, or display, that Rostov could
not believe his eyes and remained in doubt for over a second. The
wolf ran forward and jumped heavily over a gully that lay in her
path. She was an old animal with a gray back and big reddish
belly. She ran without hurry, evidently feeling sure that no one
saw her. Rostov, holding his breath, looked round at the borzois.
They stood or lay not seeing the wolf or understanding the
situation.  Old Karay had turned his head and was angrily
searching for fleas, baring his yellow teeth and snapping at his
hind legs.

``Ulyulyulyu!'' whispered Rostov, pouting his lips. The borzois
jumped up, jerking the rings of the leashes and pricking their
ears. Karay finished scratching his hindquarters and, cocking his
ears, got up with quivering tail from which tufts of matted hair
hung down.

``Shall I loose them or not?'' Nicholas asked himself as the wolf
approached him coming from the copse. Suddenly the wolf's whole
physiognomy changed: she shuddered, seeing what she had probably
never seen before---human eyes fixed upon her---and turning her
head a little toward Rostov, she paused.

``Back or forward? Eh, no matter, forward...'' the wolf seemed to
say to herself, and she moved forward without again looking round
and with a quiet, long, easy yet resolute lope.

``Ulyulyu!'' cried Nicholas, in a voice not his own, and of its
own accord his good horse darted headlong downhill, leaping over
gullies to head off the wolf, and the borzois passed it, running
faster still. Nicholas did not hear his own cry nor feel that he
was galloping, nor see the borzois, nor the ground over which he
went: he saw only the wolf, who, increasing her speed, bounded on
in the same direction along the hollow.  The first to come into
view was Milka, with her black markings and powerful quarters,
gaining upon the wolf. Nearer and nearer... now she was ahead of
it; but the wolf turned its head to face her, and instead of
putting on speed as she usually did Milka suddenly raised her
tail and stiffened her forelegs.

``Ulyulyulyulyu!'' shouted Nicholas.

The reddish Lyubim rushed forward from behind Milka, sprang
impetuously at the wolf, and seized it by its hindquarters, but
immediately jumped aside in terror. The wolf crouched, gnashed
her teeth, and again rose and bounded forward, followed at the
distance of a couple of feet by all the borzois, who did not get
any closer to her.

``She'll get away! No, it's impossible!'' thought Nicholas, still
shouting with a hoarse voice.

``Karay, ulyulyu!...'' he shouted, looking round for the old
borzoi who was now his only hope. Karay, with all the strength
age had left him, stretched himself to the utmost and, watching
the wolf, galloped heavily aside to intercept it. But the
quickness of the wolf's lope and the borzoi's slower pace made it
plain that Karay had miscalculated.  Nicholas could already see
not far in front of him the wood where the wolf would certainly
escape should she reach it. But, coming toward him, he saw hounds
and a huntsman galloping almost straight at the wolf.  There was
still hope. A long, yellowish young borzoi, one Nicholas did not
know, from another leash, rushed impetuously at the wolf from in
front and almost knocked her over. But the wolf jumped up more
quickly than anyone could have expected and, gnashing her teeth,
flew at the yellowish borzoi, which, with a piercing yelp, fell
with its head on the ground, bleeding from a gash in its side.

``Karay? Old fellow!...'' wailed Nicholas.

Thanks to the delay caused by this crossing of the wolf's path,
the old dog with its felted hair hanging from its thigh was
within five paces of it. As if aware of her danger, the wolf
turned her eyes on Karay, tucked her tail yet further between her
legs, and increased her speed. But here Nicholas only saw that
something happened to Karay---the borzoi was suddenly on the
wolf, and they rolled together down into a gully just in front of
them.

That instant, when Nicholas saw the wolf struggling in the gully
with the dogs, while from under them could be seen her gray hair
and outstretched hind leg and her frightened choking head, with
her ears laid back (Karay was pinning her by the throat), was the
happiest moment of his life. With his hand on his saddlebow, he
was ready to dismount and stab the wolf, when she suddenly thrust
her head up from among that mass of dogs, and then her forepaws
were on the edge of the gully. She clicked her teeth (Karay no
longer had her by the throat), leaped with a movement of her hind
legs out of the gully, and having disengaged herself from the
dogs, with tail tucked in again, went forward. Karay, his hair
bristling, and probably bruised or wounded, climbed with
difficulty out of the gully.

``Oh my God! Why?'' Nicholas cried in despair.

\emph{Uncle's} huntsman was galloping from the other side across
the wolf's path and his borzois once more stopped the animal's
advance. She was again hemmed in.

Nicholas and his attendant, with \emph{Uncle} and his huntsman,
were all riding round the wolf, crying ``ulyulyu!'' shouting and
preparing to dismount each moment that the wolf crouched back,
and starting forward again every time she shook herself and moved
toward the wood where she would be safe.

Already, at the beginning of this chase, Daniel, hearing the
ulyulyuing, had rushed out from the wood. He saw Karay seize the
wolf, and checked his horse, supposing the affair to be over. But
when he saw that the horsemen did not dismount and that the wolf
shook herself and ran for safety, Daniel set his chestnut
galloping, not at the wolf but straight toward the wood, just as
Karay had run to cut the animal off. As a result of this, he
galloped up to the wolf just when she had been stopped a second
time by \emph{Uncle's} borzois.

Daniel galloped up silently, holding a naked dagger in his left
hand and thrashing the laboring sides of his chestnut horse with
his whip as if it were a flail.

Nicholas neither saw nor heard Daniel until the chestnut,
breathing heavily, panted past him, and he heard the fall of a
body and saw Daniel lying on the wolf's back among the dogs,
trying to seize her by the ears. It was evident to the dogs, the
hunters, and to the wolf herself that all was now over. The
terrified wolf pressed back her ears and tried to rise, but the
borzois stuck to her. Daniel rose a little, took a step, and with
his whole weight, as if lying down to rest, fell on the wolf,
seizing her by the ears. Nicholas was about to stab her, but
Daniel whispered, ``Don't! We'll gag her!'' and, changing his
position, set his foot on the wolf's neck. A stick was thrust
between her jaws and she was fastened with a leash, as if
bridled, her legs were bound together, and Daniel rolled her over
once or twice from side to side.

With happy, exhausted faces, they laid the old wolf, alive, on a
shying and snorting horse and, accompanied by the dogs yelping at
her, took her to the place where they were all to meet. The
hounds had killed two of the cubs and the borzois three. The
huntsmen assembled with their booty and their stories, and all
came to look at the wolf, which, with her broad-browed head
hanging down and the bitten stick between her jaws, gazed with
great glassy eyes at this crowd of dogs and men surrounding
her. When she was touched, she jerked her bound legs and looked
wildly yet simply at everybody. Old Count Rostov also rode up and
touched the wolf.

``Oh, what a formidable one!'' said he. ``A formidable one, eh?''
he asked Daniel, who was standing near.

``Yes, your excellency,'' answered Daniel, quickly doffing his
cap.

The count remembered the wolf he had let slip and his encounter
with Daniel.

``Ah, but you are a crusty fellow, friend!'' said the count.

For sole reply Daniel gave him a shy, childlike, meek, and
amiable smile.

% % % % % % % % % % % % % % % % % % % % % % % % % % % % % % % % %
% % % % % % % % % % % % % % % % % % % % % % % % % % % % % % % % %
% % % % % % % % % % % % % % % % % % % % % % % % % % % % % % % % %
% % % % % % % % % % % % % % % % % % % % % % % % % % % % % % % % %
% % % % % % % % % % % % % % % % % % % % % % % % % % % % % % % % %
% % % % % % % % % % % % % % % % % % % % % % % % % % % % % % % % %
% % % % % % % % % % % % % % % % % % % % % % % % % % % % % % % % %
% % % % % % % % % % % % % % % % % % % % % % % % % % % % % % % % %
% % % % % % % % % % % % % % % % % % % % % % % % % % % % % % % % %
% % % % % % % % % % % % % % % % % % % % % % % % % % % % % % % % %
% % % % % % % % % % % % % % % % % % % % % % % % % % % % % % % % %
% % % % % % % % % % % % % % % % % % % % % % % % % % % % % %

\chapter*{Chapter VI}
\ifaudio     
\marginpar{
\href{http://ia802705.us.archive.org/22/items/war_and_peace_07_0808_librivox/war_and_peace_07_06_tolstoy_64kb.mp3}{Audio}} 
\fi

\lettrine[lines=2, loversize=0.3, lraise=0]{\initfamily T}{he}
old count went home, and Natasha and Petya promised to return
very soon, but as it was still early the hunt went farther. At
midday they put the hounds into a ravine thickly overgrown with
young trees.  Nicholas standing in a fallow field could see all
his whips.

Facing him lay a field of winter rye, there his own huntsman
stood alone in a hollow behind a hazel bush. The hounds had
scarcely been loosed before Nicholas heard one he knew, Voltorn,
giving tongue at intervals; other hounds joined in, now pausing
and now again giving tongue. A moment later he heard a cry from
the wooded ravine that a fox had been found, and the whole pack,
joining together, rushed along the ravine toward the ryefield and
away from Nicholas.

He saw the whips in their red caps galloping along the edge of
the ravine, he even saw the hounds, and was expecting a fox to
show itself at any moment on the ryefield opposite.

The huntsman standing in the hollow moved and loosed his borzois,
and Nicholas saw a queer, short-legged red fox with a fine brush
going hard across the field. The borzois bore down on it... Now
they drew close to the fox which began to dodge between the field
in sharper and sharper curves, trailing its brush, when suddenly
a strange white borzoi dashed in followed by a black one, and
everything was in confusion; the borzois formed a star-shaped
figure, scarcely swaying their bodies and with tails turned away
from the center of the group. Two huntsmen galloped up to the
dogs; one in a red cap, the other, a stranger, in a green coat.

``What's this?'' thought Nicholas. ``Where's that huntsman from?
He is not \emph{Uncle's} man.''

The huntsmen got the fox, but stayed there a long time without
strapping it to the saddle. Their horses, bridled and with high
saddles, stood near them and there too the dogs were lying. The
huntsmen waved their arms and did something to the fox. Then from
that spot came the sound of a horn, with the signal agreed on in
case of a fight.

``That's Ilagin's huntsman having a row with our Ivan,'' said
Nicholas' groom.

Nicholas sent the man to call Natasha and Petya to him, and rode
at a footpace to the place where the whips were getting the
hounds together.  Several of the field galloped to the spot where
the fight was going on.

Nicholas dismounted, and with Natasha and Petya, who had ridden
up, stopped near the hounds, waiting to see how the matter would
end. Out of the bushes came the huntsman who had been fighting
and rode toward his young master, with the fox tied to his
crupper. While still at a distance he took off his cap and tried
to speak respectfully, but he was pale and breathless and his
face was angry. One of his eyes was black, but he probably was
not even aware of it.

``What has happened?'' asked Nicholas.

``A likely thing, killing a fox our dogs had hunted! And it was
my gray bitch that caught it! Go to law, indeed!... He snatches
at the fox! I gave him one with the fox. Here it is on my saddle!
Do you want a taste of this?...'' said the huntsman, pointing to
his dagger and probably imagining himself still speaking to his
foe.

Nicholas, not stopping to talk to the man, asked his sister and
Petya to wait for him and rode to the spot where the enemy's,
Ilagin's, hunting party was.

The victorious huntsman rode off to join the field, and there,
surrounded by inquiring sympathizers, recounted his exploits.

The facts were that Ilagin, with whom the Rostovs had a quarrel
and were at law, hunted over places that belonged by custom to
the Rostovs, and had now, as if purposely, sent his men to the
very woods the Rostovs were hunting and let his man snatch a fox
their dogs had chased.

Nicholas, though he had never seen Ilagin, with his usual absence
of moderation in judgment, hated him cordially from reports of
his arbitrariness and violence, and regarded him as his bitterest
foe. He rode in angry agitation toward him, firmly grasping his
whip and fully prepared to take the most resolute and desperate
steps to punish his enemy.

Hardly had he passed an angle of the wood before a stout
gentleman in a beaver cap came riding toward him on a handsome
raven-black horse, accompanied by two hunt servants.

Instead of an enemy, Nicholas found in Ilagin a stately and
courteous gentleman who was particularly anxious to make the
young count's acquaintance. Having ridden up to Nicholas, Ilagin
raised his beaver cap and said he much regretted what had
occurred and would have the man punished who had allowed himself
to seize a fox hunted by someone else's borzois. He hoped to
become better acquainted with the count and invited him to draw
his covert.

Natasha, afraid that her brother would do something dreadful, had
followed him in some excitement. Seeing the enemies exchanging
friendly greetings, she rode up to them. Ilagin lifted his beaver
cap still higher to Natasha and said, with a pleasant smile, that
the young countess resembled Diana in her passion for the chase
as well as in her beauty, of which he had heard much.

To expiate his huntsman's offense, Ilagin pressed the Rostovs to
come to an upland of his about a mile away which he usually kept
for himself and which, he said, swarmed with hares. Nicholas
agreed, and the hunt, now doubled, moved on.

The way to Iligin's upland was across the fields. The hunt
servants fell into line. The masters rode together. \emph{Uncle},
Rostov, and Ilagin kept stealthily glancing at one another's
dogs, trying not to be observed by their companions and searching
uneasily for rivals to their own borzois.

Rostov was particularly struck by the beauty of a small,
pure-bred, red-spotted bitch on Ilagin's leash, slender but with
muscles like steel, a delicate muzzle, and prominent black
eyes. He had heard of the swiftness of Ilagin's borzois, and in
that beautiful bitch saw a rival to his own Milka.

In the middle of a sober conversation begun by Ilagin about the
year's harvest, Nicholas pointed to the red-spotted bitch.

``A fine little bitch, that!'' said he in a careless tone. ``Is
she swift?''

``That one? Yes, she's a good dog, gets what she's after,''
answered Ilagin indifferently, of the red-spotted bitch Erza, for
which, a year before, he had given a neighbor three families of
house serfs. ``So in your parts, too, the harvest is nothing to
boast of, Count?'' he went on, continuing the conversation they
had begun. And considering it polite to return the young count's
compliment, Ilagin looked at his borzois and picked out Milka who
attracted his attention by her breadth. ``That black-spotted one
of yours is fine---well shaped!'' said he.

``Yes, she's fast enough,'' replied Nicholas, and thought: ``If
only a full-grown hare would cross the field now I'd show you
what sort of borzoi she is,'' and turning to his groom, he said
he would give a ruble to anyone who found a hare.

``I don't understand,'' continued Ilagin, ``how some sportsmen
can be so jealous about game and dogs. For myself, I can tell
you, Count, I enjoy riding in company such as this... what could
be better?'' (he again raised his cap to Natasha) ``but as for
counting skins and what one takes, I don't care about that.''

``Of course not!''

``Or being upset because someone else's borzoi and not mine
catches something. All I care about is to enjoy seeing the chase,
is it not so, Count? For I consider that...''

``A-tu!'' came the long-drawn cry of one of the borzoi
whippers-in, who had halted. He stood on a knoll in the stubble,
holding his whip aloft, and again repeated his long-drawn cry,
``A-tu!'' (This call and the uplifted whip meant that he saw a
sitting hare.)

``Ah, he has found one, I think,'' said Ilagin carelessly. ``Yes,
we must ride up... Shall we both course it?'' answered Nicholas,
seeing in Erza and \emph{Uncle's} red Rugay two rivals he had
never yet had a chance of pitting against his own borzois. ``And
suppose they outdo my Milka at once!'' he thought as he rode with
\emph{Uncle} and Ilagin toward the hare.

``A full-grown one?'' asked Ilagin as he approached the whip who
had sighted the hare---and not without agitation he looked round
and whistled to Erza.

``And you, Michael Nikanorovich?'' he said, addressing
\emph{Uncle}.

The latter was riding with a sullen expression on his face.

``How can I join in? Why, you've given a village for each of your
borzois! That's it, come on! Yours are worth thousands. Try yours
against one another, you two, and I'll look on!''

``Rugay, hey, hey!'' he shouted. ``Rugayushka!'' he added,
involuntarily by this diminutive expressing his affection and the
hopes he placed on this red borzoi. Natasha saw and felt the
agitation the two elderly men and her brother were trying to
conceal, and was herself excited by it.

The huntsman stood halfway up the knoll holding up his whip and
the gentlefolk rode up to him at a footpace; the hounds that were
far off on the horizon turned away from the hare, and the whips,
but not the gentlefolk, also moved away. All were moving slowly
and sedately.

``How is it pointing?'' asked Nicholas, riding a hundred paces
toward the whip who had sighted the hare.

But before the whip could reply, the hare, scenting the frost
coming next morning, was unable to rest and leaped up. The pack
on leash rushed downhill in full cry after the hare, and from all
sides the borzois that were not on leash darted after the hounds
and the hare. All the hunt, who had been moving slowly, shouted,
``Stop!'' calling in the hounds, while the borzoi whips, with a
cry of ``A-tu!'' galloped across the field setting the borzois on
the hare. The tranquil Ilagin, Nicholas, Natasha, and
\emph{Uncle} flew, reckless of where and how they went, seeing
only the borzois and the hare and fearing only to lose sight even
for an instant of the chase. The hare they had started was a
strong and swift one. When he jumped up he did not run at once,
but pricked his ears listening to the shouting and trampling that
resounded from all sides at once. He took a dozen bounds, not
very quickly, letting the borzois gain on him, and, finally
having chosen his direction and realized his danger, laid back
his ears and rushed off headlong. He had been lying in the
stubble, but in front of him was the autumn sowing where the
ground was soft. The two borzois of the huntsman who had sighted
him, having been the nearest, were the first to see and pursue
him, but they had not gone far before Ilagin's red-spotted Erza
passed them, got within a length, flew at the hare with terrible
swiftness aiming at his scut, and, thinking she had seized him,
rolled over like a ball. The hare arched his back and bounded off
yet more swiftly. From behind Erza rushed the broad-haunched,
black-spotted Milka and began rapidly gaining on the hare.

``Milashka, dear!'' rose Nicholas' triumphant cry. It looked as
if Milka would immediately pounce on the hare, but she overtook
him and flew past. The hare had squatted. Again the beautiful
Erza reached him, but when close to the hare's scut paused as if
measuring the distance, so as not to make a mistake this time but
seize his hind leg.

``Erza, darling!'' Ilagin wailed in a voice unlike his own. Erza
did not hearken to his appeal. At the very moment when she would
have seized her prey, the hare moved and darted along the balk
between the winter rye and the stubble. Again Erza and Milka were
abreast, running like a pair of carriage horses, and began to
overtake the hare, but it was easier for the hare to run on the
balk and the borzois did not overtake him so quickly.

``Rugay, Rugayushka! That's it, come on!'' came a third voice
just then, and \emph{Uncle's} red borzoi, straining and curving
its back, caught up with the two foremost borzois, pushed ahead
of them regardless of the terrible strain, put on speed close to
the hare, knocked it off the balk onto the ryefield, again put on
speed still more viciously, sinking to his knees in the muddy
field, and all one could see was how, muddying his back, he
rolled over with the hare. A ring of borzois surrounded him. A
moment later everyone had drawn up round the crowd of dogs. Only
the delighted \emph{Uncle} dismounted, and cut off a pad, shaking
the hare for the blood to drip off, and anxiously glancing round
with restless eyes while his arms and legs twitched. He spoke
without himself knowing whom to or what about. ``That's it, come
on! That's a dog!... There, it has beaten them all, the
thousand-ruble as well as the one-ruble borzois. That's it, come
on!'' said he, panting and looking wrathfully around as if he
were abusing someone, as if they were all his enemies and had
insulted him, and only now had he at last succeeded in justifying
himself. ``There are your thousand-ruble ones... That's it, come
on!...''

``Rugay, here's a pad for you!'' he said, throwing down the
hare's muddy pad. ``You've deserved it, that's it, come on!''

``She'd tired herself out, she'd run it down three times by
herself,'' said Nicholas, also not listening to anyone and
regardless of whether he were heard or not.

``But what is there in running across it like that?'' said
Ilagin's groom.

``Once she had missed it and turned it away, any mongrel could
take it,'' Ilagin was saying at the same time, breathless from
his gallop and his excitement. At the same moment Natasha,
without drawing breath, screamed joyously, ecstatically, and so
piercingly that it set everyone's ear tingling. By that shriek
she expressed what the others expressed by all talking at once,
and it was so strange that she must herself have been ashamed of
so wild a cry and everyone else would have been amazed at it at
any other time. \emph{Uncle} himself twisted up the hare, threw
it neatly and smartly across his horse's back as if by that
gesture he meant to rebuke everybody, and, with an air of not
wishing to speak to anyone, mounted his bay and rode off. The
others all followed, dispirited and shamefaced, and only much
later were they able to regain their former affectation of
indifference. For a long time they continued to look at red Rugay
who, his arched back spattered with mud and clanking the ring of
his leash, walked along just behind \emph{Uncle's} horse with the
serene air of a conqueror.

``Well, I am like any other dog as long as it's not a question of
coursing. But when it is, then look out!'' his appearance seemed
to Nicholas to be saying.

When, much later, \emph{Uncle} rode up to Nicholas and began
talking to him, he felt flattered that, after what had happened,
\emph{Uncle} deigned to speak to him.

% % % % % % % % % % % % % % % % % % % % % % % % % % % % % % % % %
% % % % % % % % % % % % % % % % % % % % % % % % % % % % % % % % %
% % % % % % % % % % % % % % % % % % % % % % % % % % % % % % % % %
% % % % % % % % % % % % % % % % % % % % % % % % % % % % % % % % %
% % % % % % % % % % % % % % % % % % % % % % % % % % % % % % % % %
% % % % % % % % % % % % % % % % % % % % % % % % % % % % % % % % %
% % % % % % % % % % % % % % % % % % % % % % % % % % % % % % % % %
% % % % % % % % % % % % % % % % % % % % % % % % % % % % % % % % %
% % % % % % % % % % % % % % % % % % % % % % % % % % % % % % % % %
% % % % % % % % % % % % % % % % % % % % % % % % % % % % % % % % %
% % % % % % % % % % % % % % % % % % % % % % % % % % % % % % % % %
% % % % % % % % % % % % % % % % % % % % % % % % % % % % % %

\chapter*{Chapter VII}
\ifaudio     
\marginpar{
\href{http://ia802705.us.archive.org/22/items/war_and_peace_07_0808_librivox/war_and_peace_07_07_tolstoy_64kb.mp3}{Audio}} 
\fi

\lettrine[lines=2, loversize=0.3, lraise=0]{\initfamily T}{oward}
evening Ilagin took leave of Nicholas, who found that they
were so far from home that he accepted \emph{Uncle's} offer that
the hunting party should spend the night in his little village of
Mikhaylovna.

``And if you put up at my house that will be better still. That's
it, come on!'' said \emph{Uncle}. ``You see it's damp weather,
and you could rest, and the little countess could be driven home
in a trap.''

\emph{Uncle's} offer was accepted. A huntsman was sent to
Otradnoe for a trap, while Nicholas rode with Natasha and Petya
to \emph{Uncle's} house.

Some five male domestic serfs, big and little, rushed out to the
front porch to meet their master. A score of women serfs, old and
young, as well as children, popped out from the back entrance to
have a look at the hunters who were arriving. The presence of
Natasha---a woman, a lady, and on horseback---raised the
curiosity of the serfs to such a degree that many of them came up
to her, stared her in the face, and unabashed by her presence
made remarks about her as though she were some prodigy on show
and not a human being able to hear or understand what was said
about her.

``Arinka! Look, she sits sideways! There she sits and her skirt
dangles... See, she's got a little hunting horn!''

``Goodness gracious! See her knife?...''

``Isn't she a Tartar!''

``How is it you didn't go head over heels?'' asked the boldest of
all, addressing Natasha directly.

\emph{Uncle} dismounted at the porch of his little wooden house
which stood in the midst of an overgrown garden and, after a
glance at his retainers, shouted authoritatively that the
superfluous ones should take themselves off and that all
necessary preparations should be made to receive the guests and
the visitors.

The serfs all dispersed. \emph{Uncle} lifted Natasha off her
horse and taking her hand led her up the rickety wooden steps of
the porch. The house, with its bare, unplastered log walls, was
not overclean---it did not seem that those living in it aimed at
keeping it spotless---but neither was it noticeably neglected. In
the entry there was a smell of fresh apples, and wolf and fox
skins hung about.

\emph{Uncle} led the visitors through the anteroom into a small
hall with a folding table and red chairs, then into the drawing
room with a round birchwood table and a sofa, and finally into
his private room where there was a tattered sofa, a worn carpet,
and portraits of Suvorov, of the host's father and mother, and of
himself in military uniform. The study smelt strongly of tobacco
and dogs. \emph{Uncle} asked his visitors to sit down and make
themselves at home, and then went out of the room.  Rugay, his
back still muddy, came into the room and lay down on the sofa,
cleaning himself with his tongue and teeth. Leading from the
study was a passage in which a partition with ragged curtains
could be seen.  From behind this came women's laughter and
whispers. Natasha, Nicholas, and Petya took off their wraps and
sat down on the sofa. Petya, leaning on his elbow, fell asleep at
once. Natasha and Nicholas were silent.  Their faces glowed, they
were hungry and very cheerful. They looked at one another (now
that the hunt was over and they were in the house, Nicholas no
longer considered it necessary to show his manly superiority over
his sister), Natasha gave him a wink, and neither refrained long
from bursting into a peal of ringing laughter even before they
had a pretext ready to account for it.

After a while \emph{Uncle} came in, in a Cossack coat, blue
trousers, and small top boots. And Natasha felt that this
costume, the very one she had regarded with surprise and
amusement at Otradnoe, was just the right thing and not at all
worse than a swallow-tail or frock coat. \emph{Uncle} too was in
high spirits and far from being offended by the brother's and
sister's laughter (it could never enter his head that they might
be laughing at his way of life) he himself joined in the
merriment.

``That's right, young countess, that's it, come on! I never saw
anyone like her!'' said he, offering Nicholas a pipe with a long
stem and, with a practiced motion of three fingers, taking down
another that had been cut short. ``She's ridden all day like a
man, and is as fresh as ever!''

Soon after \emph{Uncle's} reappearance the door was opened,
evidently from the sound by a barefooted girl, and a stout, rosy,
good-looking woman of about forty, with a double chin and full
red lips, entered carrying a large loaded tray. With hospitable
dignity and cordiality in her glance and in every motion, she
looked at the visitors and, with a pleasant smile, bowed
respectfully. In spite of her exceptional stoutness, which caused
her to protrude her chest and stomach and throw back her head,
this woman (who was \emph{Uncle's} housekeeper) trod very
lightly. She went to the table, set down the tray, and with her
plump white hands deftly took from it the bottles and various
hors d'oeuvres and dishes and arranged them on the table. When
she had finished, she stepped aside and stopped at the door with
a smile on her face. ``Here I am. I am she! Now do you understand
\emph{Uncle}?'' her expression said to Rostov. How could one help
understanding? Not only Nicholas, but even Natasha understood the
meaning of his puckered brow and the happy complacent smile that
slightly puckered his lips when Anisya Fedorovna entered. On the
tray was a bottle of herb wine, different kinds of vodka, pickled
mushrooms, rye cakes made with buttermilk, honey in the comb,
still mead and sparkling mead, apples, nuts (raw and roasted),
and nut-and-honey sweets. Afterwards she brought a freshly
roasted chicken, ham, preserves made with honey, and preserves
made with sugar.

All this was the fruit of Anisya Fedorovna's housekeeping,
gathered and prepared by her. The smell and taste of it all had a
smack of Anisya Fedorovna herself: a savor of juiciness,
cleanliness, whiteness, and pleasant smiles.

``Take this, little Lady-Countess!'' she kept saying, as she
offered Natasha first one thing and then another.

Natasha ate of everything and thought she had never seen or eaten
such buttermilk cakes, such aromatic jam, such honey-and-nut
sweets, or such a chicken anywhere. Anisya Fedorovna left the
room.

After supper, over their cherry brandy, Rostov and \emph{Uncle}
talked of past and future hunts, of Rugay and Ilagin's dogs,
while Natasha sat upright on the sofa and listened with sparkling
eyes. She tried several times to wake Petya that he might eat
something, but he only muttered incoherent words without waking
up. Natasha felt so lighthearted and happy in these novel
surroundings that she only feared the trap would come for her too
soon. After a casual pause, such as often occurs when receiving
friends for the first time in one's own house, \emph{Uncle},
answering a thought that was in his visitors' minds, said:

``This, you see, is how I am finishing my days... Death will
come. That's it, come on! Nothing will remain. Then why harm
anyone?''

\emph{Uncle's} face was very significant and even handsome as he
said this.  Involuntarily Rostov recalled all the good he had
heard about him from his father and the neighbors. Throughout the
whole province \emph{Uncle} had the reputation of being the most
honorable and disinterested of cranks.  They called him in to
decide family disputes, chose him as executor, confided secrets
to him, elected him to be a justice and to other posts; but he
always persistently refused public appointments, passing the
autumn and spring in the fields on his bay gelding, sitting at
home in winter, and lying in his overgrown garden in summer.

``Why don't you enter the service, Uncle?''

``I did once, but gave it up. I am not fit for it. That's it,
come on! I can't make head or tail of it. That's for you---I
haven't brains enough.  Now, hunting is another matter---that's
it, come on! Open the door, there!'' he shouted. ``Why have you
shut it?''

The door at the end of the passage led to the huntsmen's room, as
they called the room for the hunt servants.

There was a rapid patter of bare feet, and an unseen hand opened
the door into the huntsmen's room, from which came the clear
sounds of a balalayka on which someone, who was evidently a
master of the art, was playing. Natasha had been listening to
those strains for some time and now went out into the passage to
hear better.

``That's Mitka, my coachman... I have got him a good
balalayka. I'm fond of it,'' said \emph{Uncle}.

It was the custom for Mitka to play the balalayka in the
huntsmen's room when \emph{Uncle} returned from the
chase. \emph{Uncle} was fond of such music.

``How good! Really very good!'' said Nicholas with some
unintentional superciliousness, as if ashamed to confess that the
sounds pleased him very much.

``Very good?'' said Natasha reproachfully, noticing her brother's
tone.  ``Not 'very good' it's simply delicious!''

Just as \emph{Uncle's} pickled mushrooms, honey, and cherry
brandy had seemed to her the best in the world, so also that
song, at that moment, seemed to her the acme of musical delight.

``More, please, more!'' cried Natasha at the door as soon as the
balalayka ceased. Mitka tuned up afresh, and recommenced
thrumming the balalayka to the air of My Lady, with trills and
variations. \emph{Uncle} sat listening, slightly smiling, with
his head on one side. The air was repeated a hundred times. The
balalayka was retuned several times and the same notes were
thrummed again, but the listeners did not grow weary of it and
wished to hear it again and again. Anisya Fedorovna came in and
leaned her portly person against the doorpost.

``You like listening?'' she said to Natasha, with a smile
extremely like \emph{Uncle's}. ``That's a good player of ours,''
she added.

``He doesn't play that part right!'' said \emph{Uncle} suddenly,
with an energetic gesture. ``Here he ought to burst out---that's
it, come on!---ought to burst out.''

``Do you play then?'' asked Natasha.

\emph{Uncle} did not answer, but smiled.

``Anisya, go and see if the strings of my guitar are all right. I
haven't touched it for a long time. That's it---come on! I've
given it up.''

Anisya Fedorovna, with her light step, willingly went to fulfill
her errand and brought back the guitar.

Without looking at anyone, \emph{Uncle} blew the dust off it and,
tapping the case with his bony fingers, tuned the guitar and
settled himself in his armchair. He took the guitar a little
above the fingerboard, arching his left elbow with a somewhat
theatrical gesture, and, with a wink at Anisya Fedorovna, struck
a single chord, pure and sonorous, and then quietly, smoothly,
and confidently began playing in very slow time, not My Lady, but
the well-known song: Came a maiden down the street. The tune,
played with precision and in exact time, began to thrill in the
hearts of Nicholas and Natasha, arousing in them the same kind of
sober mirth as radiated from Anisya Fedorovna's whole
being. Anisya Fedorovna flushed, and drawing her kerchief over
her face went laughing out of the room. \emph{Uncle} continued to
play correctly, carefully, with energetic firmness, looking with
a changed and inspired expression at the spot where Anisya
Fedorovna had just stood. Something seemed to be laughing a
little on one side of his face under his gray mustaches,
especially as the song grew brisker and the time quicker and
when, here and there, as he ran his fingers over the strings,
something seemed to snap.

``Lovely, lovely! Go on, Uncle, go on!'' shouted Natasha as soon
as he had finished. She jumped up and hugged and kissed
him. ``Nicholas, Nicholas!''  she said, turning to her brother,
as if asking him: ``What is it moves me so?''

Nicholas too was greatly pleased by \emph{Uncle's} playing, and
\emph{Uncle} played the piece over again. Anisya Fedorovna's
smiling face reappeared in the doorway and behind hers other
faces...

Fetching water clear and sweet, Stop, dear maiden, I
entreat---played \emph{Uncle} once more, running his fingers
skillfully over the strings, and then he stopped short and jerked
his shoulders.

``Go on, Uncle dear,'' Natasha wailed in an imploring tone as if
her life depended on it.

\emph{Uncle} rose, and it was as if there were two men in him:
one of them smiled seriously at the merry fellow, while the merry
fellow struck a naive and precise attitude preparatory to a folk
dance.

``Now then, niece!'' he exclaimed, waving to Natasha the hand
that had just struck a chord.

Natasha threw off the shawl from her shoulders, ran forward to
face \emph{Uncle}, and setting her arms akimbo also made a motion
with her shoulders and struck an attitude.

Where, how, and when had this young countess, educated by an
emigree French governess, imbibed from the Russian air she
breathed that spirit and obtained that manner which the pas de
chale\footnote{The French shawl dance.} would, one would have
supposed, long ago have effaced? But the spirit and the movements
were those inimitable and unteachable Russian ones that
\emph{Uncle} had expected of her. As soon as she had struck her
pose, and smiled triumphantly, proudly, and with sly merriment,
the fear that had at first seized Nicholas and the others that
she might not do the right thing was at an end, and they were
already admiring her.

She did the right thing with such precision, such complete
precision, that Anisya Fedorovna, who had at once handed her the
handkerchief she needed for the dance, had tears in her eyes,
though she laughed as she watched this slim, graceful countess,
reared in silks and velvets and so different from herself, who
yet was able to understand all that was in Anisya and in Anisya's
father and mother and aunt, and in every Russian man and woman.

``Well, little countess; that's it---come on!'' cried
\emph{Uncle}, with a joyous laugh, having finished the
dance. ``Well done, niece! Now a fine young fellow must be found
as husband for you. That's it---come on!''

``He's chosen already,'' said Nicholas smiling.

``Oh?'' said \emph{Uncle} in surprise, looking inquiringly at
Natasha, who nodded her head with a happy smile.

``And such a one!'' she said. But as soon as she had said it a
new train of thoughts and feelings arose in her. ``What did
Nicholas' smile mean when he said 'chosen already'? Is he glad of
it or not? It is as if he thought my Bolkonski would not approve
of or understand our gaiety. But he would understand it
all. Where is he now?'' she thought, and her face suddenly became
serious. But this lasted only a second. ``Don't dare to think
about it,'' she said to herself, and sat down again smilingly
beside \emph{Uncle}, begging him to play something more.

\emph{Uncle} played another song and a valse; then after a pause
he cleared his throat and sang his favorite hunting song:

As 'twas growing dark last night Fell the snow so soft and
light...

\emph{Uncle} sang as peasants sing, with full and naive
conviction that the whole meaning of a song lies in the words and
that the tune comes of itself, and that apart from the words
there is no tune, which exists only to give measure to the
words. As a result of this the unconsidered tune, like the song
of a bird, was extraordinarily good. Natasha was in ecstasies
over \emph{Uncle's} singing. She resolved to give up learning the
harp and to play only the guitar. She asked \emph{Uncle} for his
guitar and at once found the chords of the song.

After nine o'clock two traps and three mounted men, who had been
sent to look for them, arrived to fetch Natasha and Petya. The
count and countess did not know where they were and were very
anxious, said one of the men.

Petya was carried out like a log and laid in the larger of the
two traps. Natasha and Nicholas got into the other. \emph{Uncle}
wrapped Natasha up warmly and took leave of her with quite a new
tenderness. He accompanied them on foot as far as the bridge that
could not be crossed, so that they had to go round by the ford,
and he sent huntsmen to ride in front with lanterns.

``Good-bye, dear niece,'' his voice called out of the
darkness---not the voice Natasha had known previously, but the
one that had sung As 'twas growing dark last night.

In the village through which they passed there were red lights
and a cheerful smell of smoke.

``What a darling Uncle is!'' said Natasha, when they had come out
onto the highroad.

``Yes,'' returned Nicholas. ``You're not cold?''

``No. I'm quite, quite all right. I feel so comfortable!''
answered Natasha, almost perplexed by her feelings. They remained
silent a long while. The night was dark and damp. They could not
see the horses, but only heard them splashing through the unseen
mud.

What was passing in that receptive childlike soul that so eagerly
caught and assimilated all the diverse impressions of life? How
did they all find place in her? But she was very happy. As they
were nearing home she suddenly struck up the air of As 'twas
growing dark last night---the tune of which she had all the way
been trying to get and had at last caught.

``Got it?'' said Nicholas.

``What were you thinking about just now, Nicholas?'' inquired
Natasha.

They were fond of asking one another that question.

``I?'' said Nicholas, trying to remember. ``Well, you see, first
I thought that Rugay, the red hound, was like Uncle, and that if
he were a man he would always keep Uncle near him, if not for his
riding, then for his manner. What a good fellow Uncle is! Don't
you think so?... Well, and you?''

``I? Wait a bit, wait... Yes, first I thought that we are driving
along and imagining that we are going home, but that heaven knows
where we are really going in the darkness, and that we shall
arrive and suddenly find that we are not in Otradnoe, but in
Fairyland. And then I thought... No, nothing else.''

``I know, I expect you thought of him,'' said Nicholas, smiling
as Natasha knew by the sound of his voice.

``No,'' said Natasha, though she had in reality been thinking
about Prince Andrew at the same time as of the rest, and of how
he would have liked \emph{Uncle}. ``And then I was saying to
myself all the way, 'How well Anisya carried herself, how
well!'{}'' And Nicholas heard her spontaneous, happy, ringing
laughter. ``And do you know,'' she suddenly said, ``I know that I
shall never again be as happy and tranquil as I am now.''

``Rubbish, nonsense, humbug!'' exclaimed Nicholas, and he
thought: ``How charming this Natasha of mine is! I have no other
friend like her and never shall have. Why should she marry? We
might always drive about together!''

``What a darling this Nicholas of mine is!'' thought Natasha.

``Ah, there are still lights in the drawing-room!'' she said,
pointing to the windows of the house that gleamed invitingly in
the moist velvety darkness of the night.

% % % % % % % % % % % % % % % % % % % % % % % % % % % % % % % % %
% % % % % % % % % % % % % % % % % % % % % % % % % % % % % % % % %
% % % % % % % % % % % % % % % % % % % % % % % % % % % % % % % % %
% % % % % % % % % % % % % % % % % % % % % % % % % % % % % % % % %
% % % % % % % % % % % % % % % % % % % % % % % % % % % % % % % % %
% % % % % % % % % % % % % % % % % % % % % % % % % % % % % % % % %
% % % % % % % % % % % % % % % % % % % % % % % % % % % % % % % % %
% % % % % % % % % % % % % % % % % % % % % % % % % % % % % % % % %
% % % % % % % % % % % % % % % % % % % % % % % % % % % % % % % % %
% % % % % % % % % % % % % % % % % % % % % % % % % % % % % % % % %
% % % % % % % % % % % % % % % % % % % % % % % % % % % % % % % % %
% % % % % % % % % % % % % % % % % % % % % % % % % % % % % %

\chapter*{Chapter VIII}
\ifaudio     
\marginpar{
\href{http://ia802705.us.archive.org/22/items/war_and_peace_07_0808_librivox/war_and_peace_07_08_tolstoy_64kb.mp3}{Audio}} 
\fi

\lettrine[lines=2, loversize=0.3, lraise=0]{\initfamily C}{ount}
Ilya Rostov had resigned the position of Marshal of the
Nobility because it involved him in too much expense, but still
his affairs did not improve. Natasha and Nicholas often noticed
their parents conferring together anxiously and privately and
heard suggestions of selling the fine ancestral Rostov house and
estate near Moscow. It was not necessary to entertain so freely
as when the count had been Marshal, and life at Otradnoe was
quieter than in former years, but still the enormous house and
its lodges were full of people and more than twenty sat down to
table every day. These were all their own people who had settled
down in the house almost as members of the family, or persons who
were, it seemed, obliged to live in the count's house. Such were
Dimmler the musician and his wife, Vogel the dancing master and
his family, Belova, an old maiden lady, an inmate of the house,
and many others such as Petya's tutors, the girls' former
governess, and other people who simply found it preferable and
more advantageous to live in the count's house than at home. They
had not as many visitors as before, but the old habits of life
without which the count and countess could not conceive of
existence remained unchanged. There was still the hunting
establishment which Nicholas had even enlarged, the same fifty
horses and fifteen grooms in the stables, the same expensive
presents and dinner parties to the whole district on name days;
there were still the count's games of whist and boston, at
which---spreading out his cards so that everybody could see
them---he let himself be plundered of hundreds of rubles every
day by his neighbors, who looked upon an opportunity to play a
rubber with Count Rostov as a most profitable source of income.

The count moved in his affairs as in a huge net, trying not to
believe that he was entangled but becoming more and more so at
every step, and feeling too feeble to break the meshes or to set
to work carefully and patiently to disentangle them. The
countess, with her loving heart, felt that her children were
being ruined, that it was not the count's fault for he could not
help being what he was---that (though he tried to hide it) he
himself suffered from the consciousness of his own and his
children's ruin, and she tried to find means of remedying the
position.  From her feminine point of view she could see only one
solution, namely, for Nicholas to marry a rich heiress. She felt
this to be their last hope and that if Nicholas refused the match
she had found for him, she would have to abandon the hope of ever
getting matters right. This match was with Julie Karagina, the
daughter of excellent and virtuous parents, a girl the Rostovs
had known from childhood, and who had now become a wealthy
heiress through the death of the last of her brothers.

The countess had written direct to Julie's mother in Moscow
suggesting a marriage between their children and had received a
favorable answer from her. Karagina had replied that for her part
she was agreeable, and everything depend on her daughter's
inclination. She invited Nicholas to come to Moscow.

Several times the countess, with tears in her eyes, told her son
that now both her daughters were settled, her only wish was to
see him married. She said she could lie down in her grave
peacefully if that were accomplished. Then she told him that she
knew of a splendid girl and tried to discover what he thought
about marriage.

At other times she praised Julie to him and advised him to go to
Moscow during the holidays to amuse himself. Nicholas guessed
what his mother's remarks were leading to and during one of these
conversations induced her to speak quite frankly. She told him
that her only hope of getting their affairs disentangled now lay
in his marrying Julie Karagina.

``But, Mamma, suppose I loved a girl who has no fortune, would
you expect me to sacrifice my feelings and my honor for the sake
of money?'' he asked his mother, not realizing the cruelty of his
question and only wishing to show his noble-mindedness.

``No, you have not understood me,'' said his mother, not knowing
how to justify herself. ``You have not understood me,
Nikolenka. It is your happiness I wish for,'' she added, feeling
that she was telling an untruth and was becoming entangled. She
began to cry.

``Mamma, don't cry! Only tell me that you wish it, and you know I
will give my life, anything, to put you at ease,'' said
Nicholas. ``I would sacrifice anything for you---even my
feelings.''

But the countess did not want the question put like that: she did
not want a sacrifice from her son, she herself wished to make a
sacrifice for him.

``No, you have not understood me, don't let us talk about it,''
she replied, wiping away her tears.

``Maybe I do love a poor girl,'' said Nicholas to himself. ``Am I
to sacrifice my feelings and my honor for money? I wonder how
Mamma could speak so to me. Because Sonya is poor I must not love
her,'' he thought, ``must not respond to her faithful, devoted
love? Yet I should certainly be happier with her than with some
doll-like Julie. I can always sacrifice my feelings for my
family's welfare,'' he said to himself, ``but I can't coerce my
feelings. If I love Sonya, that feeling is for me stronger and
higher than all else.''

Nicholas did not go to Moscow, and the countess did not renew the
conversation with him about marriage. She saw with sorrow, and
sometimes with exasperation, symptoms of a growing attachment
between her son and the portionless Sonya. Though she blamed
herself for it, she could not refrain from grumbling at and
worrying Sonya, often pulling her up without reason, addressing
her stiffly as \emph{my dear}, and using the formal \emph{you}
instead of the intimate \emph{thou} in speaking to her. The
kindhearted countess was the more vexed with Sonya because that
poor, dark-eyed niece of hers was so meek, so kind, so devotedly
grateful to her benefactors, and so faithfully, unchangingly, and
unselfishly in love with Nicholas, that there were no grounds for
finding fault with her.

Nicholas was spending the last of his leave at home. A fourth
letter had come from Prince Andrew, from Rome, in which he wrote
that he would have been on his way back to Russia long ago had
not his wound unexpectedly reopened in the warm climate, which
obliged him to defer his return till the beginning of the new
year. Natasha was still as much in love with her betrothed, found
the same comfort in that love, and was still as ready to throw
herself into all the pleasures of life as before; but at the end
of the fourth month of their separation she began to have fits of
depression which she could not master. She felt sorry for
herself: sorry that she was being wasted all this time and of no
use to anyone---while she felt herself so capable of loving and
being loved.

Things were not cheerful in the Rostovs' home.

% % % % % % % % % % % % % % % % % % % % % % % % % % % % % % % % %
% % % % % % % % % % % % % % % % % % % % % % % % % % % % % % % % %
% % % % % % % % % % % % % % % % % % % % % % % % % % % % % % % % %
% % % % % % % % % % % % % % % % % % % % % % % % % % % % % % % % %
% % % % % % % % % % % % % % % % % % % % % % % % % % % % % % % % %
% % % % % % % % % % % % % % % % % % % % % % % % % % % % % % % % %
% % % % % % % % % % % % % % % % % % % % % % % % % % % % % % % % %
% % % % % % % % % % % % % % % % % % % % % % % % % % % % % % % % %
% % % % % % % % % % % % % % % % % % % % % % % % % % % % % % % % %
% % % % % % % % % % % % % % % % % % % % % % % % % % % % % % % % %
% % % % % % % % % % % % % % % % % % % % % % % % % % % % % % % % %
% % % % % % % % % % % % % % % % % % % % % % % % % % % % % %

\chapter*{Chapter IX}
\ifaudio     
\marginpar{
\href{http://ia802705.us.archive.org/22/items/war_and_peace_07_0808_librivox/war_and_peace_07_09_tolstoy_64kb.mp3}{Audio}} 
\fi

\lettrine[lines=2, loversize=0.3, lraise=0]{\initfamily C}{hristmas}
came and except for the ceremonial Mass, the solemn and
wearisome Christmas congratulations from neighbors and servants,
and the new dresses everyone put on, there were no special
festivities, though the calm frost of twenty degrees Reaumur, the
dazzling sunshine by day, and the starlight of the winter nights
seemed to call for some special celebration of the season.

On the third day of Christmas week, after the midday dinner, all
the inmates of the house dispersed to various rooms. It was the
dullest time of the day. Nicholas, who had been visiting some
neighbors that morning, was asleep on the sitting-room sofa. The
old count was resting in his study. Sonya sat in the drawing room
at the round table, copying a design for embroidery. The countess
was playing patience. Nastasya Ivanovna the buffoon sat with a
sad face at the window with two old ladies. Natasha came into the
room, went up to Sonya, glanced at what she was doing, and then
went up to her mother and stood without speaking.

``Why are you wandering about like an outcast?'' asked her
mother. ``What do you want?''

``Him... I want him... now, this minute! I want him!'' said
Natasha, with glittering eyes and no sign of a smile.

The countess lifted her head and looked attentively at her
daughter.

``Don't look at me, Mamma! Don't look; I shall cry directly.''

``Sit down with me a little,'' said the countess.

``Mamma, I want him. Why should I be wasted like this, Mamma?''

Her voice broke, tears gushed from her eyes, and she turned
quickly to hide them and left the room.

She passed into the sitting room, stood there thinking awhile,
and then went into the maids' room. There an old maidservant was
grumbling at a young girl who stood panting, having just run in
through the cold from the serfs' quarters.

``Stop playing---there's a time for everything,'' said the old
woman.

``Let her alone, Kondratevna,'' said Natasha. ``Go, Mavrushka,
go.''

Having released Mavrushka, Natasha crossed the dancing hall and
went to the vestibule. There an old footman and two young ones
were playing cards. They broke off and rose as she entered.

``What can I do with them?'' thought Natasha.

``Oh, Nikita, please go... where can I send him?... Yes, go to
the yard and fetch a fowl, please, a cock, and you, Misha, bring
me some oats.''

``Just a few oats?'' said Misha, cheerfully and readily.

``Go, go quickly,'' the old man urged him.

``And you, Theodore, get me a piece of chalk.''

On her way past the butler's pantry she told them to set a
samovar, though it was not at all the time for tea.

Foka, the butler, was the most ill-tempered person in the
house. Natasha liked to test her power over him. He distrusted
the order and asked whether the samovar was really wanted.

``Oh dear, what a young lady!'' said Foka, pretending to frown at
Natasha.

No one in the house sent people about or gave them as much
trouble as Natasha did. She could not see people unconcernedly,
but had to send them on some errand. She seemed to be trying
whether any of them would get angry or sulky with her; but the
serfs fulfilled no one's orders so readily as they did
hers. ``What can I do, where can I go?'' thought she, as she went
slowly along the passage.

``Nastasya Ivanovna, what sort of children shall I have?'' she
asked the buffoon, who was coming toward her in a woman's jacket.

``Why, fleas, crickets, grasshoppers,'' answered the buffoon.

``O Lord, O Lord, it's always the same! Oh, where am I to go?
What am I to do with myself?'' And tapping with her heels, she
ran quickly upstairs to see Vogel and his wife who lived on the
upper story.

Two governesses were sitting with the Vogels at a table, on which
were plates of raisins, walnuts, and almonds. The governesses
were discussing whether it was cheaper to live in Moscow or
Odessa. Natasha sat down, listened to their talk with a serious
and thoughtful air, and then got up again.

``The island of Madagascar,'' she said, ``Ma-da-gas-car,'' she
repeated, articulating each syllable distinctly, and, not
replying to Madame Schoss who asked her what she was saying, she
went out of the room.

Her brother Petya was upstairs too; with the man in attendance on
him he was preparing fireworks to let off that night.

``Petya! Petya!'' she called to him. ``Carry me downstairs.''

Petya ran up and offered her his back. She jumped on it, putting
her arms round his neck, and he pranced along with her.

``No, don't... the island of Madagascar!'' she said, and jumping
off his back she went downstairs.

Having as it were reviewed her kingdom, tested her power, and
made sure that everyone was submissive, but that all the same it
was dull, Natasha betook herself to the ballroom, picked up her
guitar, sat down in a dark corner behind a bookcase, and began to
run her fingers over the strings in the bass, picking out a
passage she recalled from an opera she had heard in Petersburg
with Prince Andrew. What she drew from the guitar would have had
no meaning for other listeners, but in her imagination a whole
series of reminiscences arose from those sounds. She sat behind
the bookcase with her eyes fixed on a streak of light escaping
from the pantry door and listened to herself and pondered. She
was in a mood for brooding on the past.

Sonya passed to the pantry with a glass in her hand. Natasha
glanced at her and at the crack in the pantry door, and it seemed
to her that she remembered the light falling through that crack
once before and Sonya passing with a glass in her hand. ``Yes it
was exactly the same,'' thought Natasha.

``Sonya, what is this?'' she cried, twanging a thick string.

``Oh, you are there!'' said Sonya with a start, and came near and
listened. ``I don't know. A storm?'' she ventured timidly, afraid
of being wrong.

``There! That's just how she started and just how she came up
smiling timidly when all this happened before,'' thought Natasha,
``and in just the same way I thought there was something lacking
in her.''

``No, it's the chorus from The Water-Carrier, listen!'' and
Natasha sang the air of the chorus so that Sonya should catch
it. ``Where were you going?'' she asked.

``To change the water in this glass. I am just finishing the
design.''

``You always find something to do, but I can't,'' said
Natasha. ``And where's Nicholas?''

``Asleep, I think.''

``Sonya, go and wake him,'' said Natasha. ``Tell him I want him
to come and sing.''

She sat awhile, wondering what the meaning of it all having
happened before could be, and without solving this problem, or at
all regretting not having done so, she again passed in fancy to
the time when she was with him and he was looking at her with a
lover's eyes.

``Oh, if only he would come quicker! I am so afraid it will never
be!  And, worst of all, I am growing old---that's the thing!
There won't then be in me what there is now. But perhaps he'll
come today, will come immediately. Perhaps he has come and is
sitting in the drawing room.  Perhaps he came yesterday and I
have forgotten it.'' She rose, put down the guitar, and went to
the drawing room.

All the domestic circle, tutors, governesses, and guests, were
already at the tea table. The servants stood round the
table---but Prince Andrew was not there and life was going on as
before.

``Ah, here she is!'' said the old count, when he saw Natasha
enter. ``Well, sit down by me.'' But Natasha stayed by her mother
and glanced round as if looking for something.

``Mamma!'' she muttered, ``give him to me, give him, Mamma,
quickly, quickly!'' and she again had difficulty in repressing
her sobs.

She sat down at the table and listened to the conversation
between the elders and Nicholas, who had also come to the
table. ``My God, my God!  The same faces, the same talk, Papa
holding his cup and blowing in the same way!'' thought Natasha,
feeling with horror a sense of repulsion rising up in her for the
whole household, because they were always the same.

After tea, Nicholas, Sonya, and Natasha went to the sitting room,
to their favorite corner where their most intimate talks always
began.

% % % % % % % % % % % % % % % % % % % % % % % % % % % % % % % % %
% % % % % % % % % % % % % % % % % % % % % % % % % % % % % % % % %
% % % % % % % % % % % % % % % % % % % % % % % % % % % % % % % % %
% % % % % % % % % % % % % % % % % % % % % % % % % % % % % % % % %
% % % % % % % % % % % % % % % % % % % % % % % % % % % % % % % % %
% % % % % % % % % % % % % % % % % % % % % % % % % % % % % % % % %
% % % % % % % % % % % % % % % % % % % % % % % % % % % % % % % % %
% % % % % % % % % % % % % % % % % % % % % % % % % % % % % % % % %
% % % % % % % % % % % % % % % % % % % % % % % % % % % % % % % % %
% % % % % % % % % % % % % % % % % % % % % % % % % % % % % % % % %
% % % % % % % % % % % % % % % % % % % % % % % % % % % % % % % % %
% % % % % % % % % % % % % % % % % % % % % % % % % % % % % %

\chapter*{Chapter X}
\ifaudio     
\marginpar{
\href{http://ia802705.us.archive.org/22/items/war_and_peace_07_0808_librivox/war_and_peace_07_10_tolstoy_64kb.mp3}{Audio}} 
\fi

\lettrine[lines=2, loversize=0.3, lraise=0]{`` \initfamily D}{oes}
it ever happen to you,'' said Natasha to her brother, when
they settled down in the sitting room, ``does it ever happen to
you to feel as if there were nothing more to come---nothing; that
everything good is past? And to feel not exactly dull, but sad?''

``I should think so!'' he replied. ``I have felt like that when
everything was all right and everyone was cheerful. The thought
has come into my mind that I was already tired of it all, and
that we must all die. Once in the regiment I had not gone to some
merrymaking where there was music... and suddenly I felt so
depressed...''

``Oh yes, I know, I know, I know!'' Natasha interrupted
him. ``When I was quite little that used to be so with me. Do you
remember when I was punished once about some plums? You were all
dancing, and I sat sobbing in the schoolroom? I shall never
forget it: I felt sad and sorry for everyone, for myself, and for
everyone. And I was innocent---that was the chief thing,'' said
Natasha. ``Do you remember?''

``I remember,'' answered Nicholas. ``I remember that I came to
you afterwards and wanted to comfort you, but do you know, I felt
ashamed to. We were terribly absurd. I had a funny doll then and
wanted to give it to you. Do you remember?''

``And do you remember,'' Natasha asked with a pensive smile,
``how once, long, long ago, when we were quite little, Uncle
called us into the study---that was in the old house---and it was
dark---we went in and suddenly there stood...''

``A Negro,'' chimed in Nicholas with a smile of delight. ``Of
course I remember. Even now I don't know whether there really was
a Negro, or if we only dreamed it or were told about him.''

``He was gray, you remember, and had white teeth, and stood and
looked at us...''

``Sonya, do you remember?'' asked Nicholas.

``Yes, yes, I do remember something too,'' Sonya answered
timidly.

``You know I have asked Papa and Mamma about that Negro,'' said
Natasha, ``and they say there was no Negro at all. But you see,
you remember!''

``Of course I do, I remember his teeth as if I had just seen
them.''

``How strange it is! It's as if it were a dream! I like that.''

``And do you remember how we rolled hard-boiled eggs in the
ballroom, and suddenly two old women began spinning round on the
carpet? Was that real or not? Do you remember what fun it was?''

``Yes, and you remember how Papa in his blue overcoat fired a gun
in the porch?''

So they went through their memories, smiling with pleasure: not
the sad memories of old age, but poetic, youthful ones---those
impressions of one's most distant past in which dreams and
realities blend---and they laughed with quiet enjoyment.

Sonya, as always, did not quite keep pace with them, though they
shared the same reminiscences.

Much that they remembered had slipped from her mind, and what she
recalled did not arouse the same poetic feeling as they
experienced. She simply enjoyed their pleasure and tried to fit
in with it.

She only really took part when they recalled Sonya's first
arrival. She told them how afraid she had been of Nicholas
because he had on a corded jacket and her nurse had told her that
she, too, would be sewn up with cords.

``And I remember their telling me that you had been born under a
cabbage,'' said Natasha, ``and I remember that I dared not
disbelieve it then, but knew that it was not true, and I felt so
uncomfortable.''

While they were talking a maid thrust her head in at the other
door of the sitting room.

``They have brought the cock, Miss,'' she said in a whisper.

``It isn't wanted, Petya. Tell them to take it away,'' replied
Natasha.

In the middle of their talk in the sitting room, Dimmler came in
and went up to the harp that stood there in a corner. He took off
its cloth covering, and the harp gave out a jarring sound.

``Mr. Dimmler, please play my favorite nocturne by Field,'' came
the old countess' voice from the drawing room.

Dimmler struck a chord and, turning to Natasha, Nicholas, and
Sonya, remarked: ``How quiet you young people are!''

``Yes, we're philosophizing,'' said Natasha, glancing round for a
moment and then continuing the conversation. They were now
discussing dreams.

Dimmler began to play; Natasha went on tiptoe noiselessly to the
table, took up a candle, carried it out, and returned, seating
herself quietly in her former place. It was dark in the room
especially where they were sitting on the sofa, but through the
big windows the silvery light of the full moon fell on the
floor. Dimmler had finished the piece but still sat softly
running his fingers over the strings, evidently uncertain whether
to stop or to play something else.

``Do you know,'' said Natasha in a whisper, moving closer to
Nicholas and Sonya, ``that when one goes on and on recalling
memories, one at last begins to remember what happened before one
was in the world...''

``That is metempsychosis,'' said Sonya, who had always learned
well, and remembered everything. ``The Egyptians believed that
our souls have lived in animals, and will go back into animals
again.''

``No, I don't believe we ever were in animals,'' said Natasha,
still in a whisper though the music had ceased. ``But I am
certain that we were angels somewhere there, and have been here,
and that is why we remember...''

``May I join you?'' said Dimmler who had come up quietly, and he
sat down by them.

``If we have been angels, why have we fallen lower?'' said
Nicholas. ``No, that can't be!''

``Not lower, who said we were lower?... How do I know what I was
before?''  Natasha rejoined with conviction. ``The soul is
immortal---well then, if I shall always live I must have lived
before, lived for a whole eternity.''

``Yes, but it is hard for us to imagine eternity,'' remarked
Dimmler, who had joined the young folk with a mildly
condescending smile but now spoke as quietly and seriously as
they.

``Why is it hard to imagine eternity?'' said Natasha. ``It is now
today, and it will be tomorrow, and always; and there was
yesterday, and the day before...''

``Natasha! Now it's your turn. Sing me something,'' they heard
the countess say. ``Why are you sitting there like
conspirators?''

``Mamma, I don't at all want to,'' replied Natasha, but all the
same she rose.

None of them, not even the middle-aged Dimmler, wanted to break
off their conversation and quit that corner in the sitting room,
but Natasha got up and Nicholas sat down at the
clavichord. Standing as usual in the middle of the hall and
choosing the place where the resonance was best, Natasha began to
sing her mother's favorite song.

She had said she did not want to sing, but it was long since she
had sung, and long before she again sang, as she did that
evening. The count, from his study where he was talking to
Mitenka, heard her and, like a schoolboy in a hurry to run out to
play, blundered in his talk while giving orders to the steward,
and at last stopped, while Mitenka stood in front of him also
listening and smiling. Nicholas did not take his eyes off his
sister and drew breath in time with her. Sonya, as she listened,
thought of the immense difference there was between herself and
her friend, and how impossible it was for her to be anything like
as bewitching as her cousin. The old countess sat with a blissful
yet sad smile and with tears in her eyes, occasionally shaking
her head. She thought of Natasha and of her own youth, and of how
there was something unnatural and dreadful in this impending
marriage of Natasha and Prince Andrew.

Dimmler, who had seated himself beside the countess, listened
with closed eyes.

``Ah, Countess,'' he said at last, ``that's a European talent,
she has nothing to learn---what softness, tenderness, and
strength...''

``Ah, how afraid I am for her, how afraid I am!'' said the
countess, not realizing to whom she was speaking. Her maternal
instinct told her that Natasha had too much of something, and
that because of this she would not be happy. Before Natasha had
finished singing, fourteen-year-old Petya rushed in delightedly,
to say that some mummers had arrived.

Natasha stopped abruptly.

``Idiot!'' she screamed at her brother and, running to a chair,
threw herself on it, sobbing so violently that she could not stop
for a long time.

``It's nothing, Mamma, really it's nothing; only Petya startled
me,'' she said, trying to smile, but her tears still flowed and
sobs still choked her.

The mummers (some of the house serfs) dressed up as bears, Turks,
innkeepers, and ladies---frightening and funny---bringing in with
them the cold from outside and a feeling of gaiety, crowded, at
first timidly, into the anteroom, then hiding behind one another
they pushed into the ballroom where, shyly at first and then more
and more merrily and heartily, they started singing, dancing, and
playing Christmas games.  The countess, when she had identified
them and laughed at their costumes, went into the drawing
room. The count sat in the ballroom, smiling radiantly and
applauding the players. The young people had disappeared.

Half an hour later there appeared among the other mummers in the
ballroom an old lady in a hooped skirt---this was Nicholas. A
Turkish girl was Petya. A clown was Dimmler. An hussar was
Natasha, and a Circassian was Sonya with burnt-cork mustache and
eyebrows.

After the condescending surprise, nonrecognition, and praise,
from those who were not themselves dressed up, the young people
decided that their costumes were so good that they ought to be
shown elsewhere.

Nicholas, who, as the roads were in splendid condition, wanted to
take them all for a drive in his troyka, proposed to take with
them about a dozen of the serf mummers and drive to
\emph{Uncle's}.

``No, why disturb the old fellow?'' said the countess. ``Besides,
you wouldn't have room to turn round there. If you must go, go to
the Melyukovs'.''

Melyukova was a widow, who, with her family and their tutors and
governesses, lived three miles from the Rostovs.

``That's right, my dear,'' chimed in the old count, thoroughly
aroused.  ``I'll dress up at once and go with them. I'll make
Pashette open her eyes.''

But the countess would not agree to his going; he had had a bad
leg all these last days. It was decided that the count must not
go, but that if Louisa Ivanovna (Madame Schoss) would go with
them, the young ladies might go to the Melyukovs', Sonya,
generally so timid and shy, more urgently than anyone begging
Louisa Ivanovna not to refuse.

Sonya's costume was the best of all. Her mustache and eyebrows
were extraordinarily becoming. Everyone told her she looked very
handsome, and she was in a spirited and energetic mood unusual
with her. Some inner voice told her that now or never her fate
would be decided, and in her male attire she seemed quite a
different person. Louisa Ivanovna consented to go, and in half an
hour four troyka sleighs with large and small bells, their
runners squeaking and whistling over the frozen snow, drove up to
the porch.

Natasha was foremost in setting a merry holiday tone, which,
passing from one to another, grew stronger and reached its climax
when they all came out into the frost and got into the sleighs,
talking, calling to one another, laughing, and shouting.

Two of the troykas were the usual household sleighs, the third
was the old count's with a trotter from the Orlov stud as shaft
horse, the fourth was Nicholas' own with a short shaggy black
shaft horse.  Nicholas, in his old lady's dress over which he had
belted his hussar overcoat, stood in the middle of the sleigh,
reins in hand.

It was so light that he could see the moonlight reflected from
the metal harness disks and from the eyes of the horses, who
looked round in alarm at the noisy party under the shadow of the
porch roof.

Natasha, Sonya, Madame Schoss, and two maids got into Nicholas'
sleigh; Dimmler, his wife, and Petya, into the old count's, and
the rest of the mummers seated themselves in the other two
sleighs.

``You go ahead, Zakhar!'' shouted Nicholas to his father's
coachman, wishing for a chance to race past him.

The old count's troyka, with Dimmler and his party, started
forward, squeaking on its runners as though freezing to the snow,
its deep-toned bell clanging. The side horses, pressing against
the shafts of the middle horse, sank in the snow, which was dry
and glittered like sugar, and threw it up.

Nicholas set off, following the first sleigh; behind him the
others moved noisily, their runners squeaking. At first they
drove at a steady trot along the narrow road. While they drove
past the garden the shadows of the bare trees often fell across
the road and hid the brilliant moonlight, but as soon as they
were past the fence, the snowy plain bathed in moonlight and
motionless spread out before them glittering like diamonds and
dappled with bluish shadows. Bang, bang! went the first sleigh
over a cradle hole in the snow of the road, and each of the other
sleighs jolted in the same way, and rudely breaking the
frost-bound stillness, the troykas began to speed along the road,
one after the other.

``A hare's track, a lot of tracks!'' rang out Natasha's voice
through the frost-bound air.

``How light it is, Nicholas!'' came Sonya's voice.

Nicholas glanced round at Sonya, and bent down to see her face
closer.  Quite a new, sweet face with black eyebrows and
mustaches peeped up at him from her sable furs---so close and yet
so distant---in the moonlight.

``That used to be Sonya,'' thought he, and looked at her closer
and smiled.

``What is it, Nicholas?''

``Nothing,'' said he and turned again to the horses.

When they came out onto the beaten highroad---polished by sleigh
runners and cut up by rough-shod hoofs, the marks of which were
visible in the moonlight---the horses began to tug at the reins
of their own accord and increased their pace. The near side
horse, arching his head and breaking into a short canter, tugged
at his traces. The shaft horse swayed from side to side, moving
his ears as if asking: ``Isn't it time to begin now?'' In front,
already far ahead the deep bell of the sleigh ringing farther and
farther off, the black horses driven by Zakhar could be clearly
seen against the white snow. From that sleigh one could hear the
shouts, laughter, and voices of the mummers.

``Gee up, my darlings!'' shouted Nicholas, pulling the reins to
one side and flourishing the whip.

It was only by the keener wind that met them and the jerks given
by the side horses who pulled harder---ever increasing their
gallop---that one noticed how fast the troyka was
flying. Nicholas looked back. With screams squeals, and waving of
whips that caused even the shaft horses to gallop---the other
sleighs followed. The shaft horse swung steadily beneath the bow
over its head, with no thought of slackening pace and ready to
put on speed when required.

Nicholas overtook the first sleigh. They were driving downhill
and coming out upon a broad trodden track across a meadow, near a
river.

``Where are we?'' thought he. ``It's the Kosoy meadow, I
suppose. But no---this is something new I've never seen
before. This isn't the Kosoy meadow nor the Demkin hill, and
heaven only knows what it is! It is something new and
enchanted. Well, whatever it may be...'' And shouting to his
horses, he began to pass the first sleigh.

Zakhar held back his horses and turned his face, which was
already covered with hoarfrost to his eyebrows.

Nicholas gave the horses the rein, and Zakhar, stretching out his
arms, clucked his tongue and let his horses go.

``Now, look out, master!'' he cried.

Faster still the two troykas flew side by side, and faster moved
the feet of the galloping side horses. Nicholas began to draw
ahead. Zakhar, while still keeping his arms extended, raised one
hand with the reins.

``No you won't, master!'' he shouted.

Nicholas put all his horses to a gallop and passed Zakhar. The
horses showered the fine dry snow on the faces of those in the
sleigh---beside them sounded quick ringing bells and they caught
confused glimpses of swiftly moving legs and the shadows of the
troyka they were passing. The whistling sound of the runners on
the snow and the voices of girls shrieking were heard from
different sides.

Again checking his horses, Nicholas looked around him. They were
still surrounded by the magic plain bathed in moonlight and
spangled with stars.

``Zakhar is shouting that I should turn to the left, but why to
the left?'' thought Nicholas. ``Are we getting to the Melyukovs'?
Is this Melyukovka? Heaven only knows where we are going, and
heaven knows what is happening to us---but it is very strange and
pleasant whatever it is.''  And he looked round in the sleigh.

``Look, his mustache and eyelashes are all white!'' said one of
the strange, pretty, unfamiliar people---the one with fine
eyebrows and mustache.

``I think this used to be Natasha,'' thought Nicholas, ``and that
was Madame Schoss, but perhaps it's not, and this Circassian with
the mustache I don't know, but I love her.''

``Aren't you cold?'' he asked.

They did not answer but began to laugh. Dimmler from the sleigh
behind shouted something---probably something funny---but they
could not make out what he said.

``Yes, yes!'' some voices answered, laughing.

``But here was a fairy forest with black moving shadows, and a
glitter of diamonds and a flight of marble steps and the silver
roofs of fairy buildings and the shrill yells of some
animals. And if this is really Melyukovka, it is still stranger
that we drove heaven knows where and have come to Melyukovka,''
thought Nicholas.

It really was Melyukovka, and maids and footmen with merry faces
came running, out to the porch carrying candles.

``Who is it?'' asked someone in the porch.

``The mummers from the count's. I know by the horses,'' replied
some voices.

% % % % % % % % % % % % % % % % % % % % % % % % % % % % % % % % %
% % % % % % % % % % % % % % % % % % % % % % % % % % % % % % % % %
% % % % % % % % % % % % % % % % % % % % % % % % % % % % % % % % %
% % % % % % % % % % % % % % % % % % % % % % % % % % % % % % % % %
% % % % % % % % % % % % % % % % % % % % % % % % % % % % % % % % %
% % % % % % % % % % % % % % % % % % % % % % % % % % % % % % % % %
% % % % % % % % % % % % % % % % % % % % % % % % % % % % % % % % %
% % % % % % % % % % % % % % % % % % % % % % % % % % % % % % % % %
% % % % % % % % % % % % % % % % % % % % % % % % % % % % % % % % %
% % % % % % % % % % % % % % % % % % % % % % % % % % % % % % % % %
% % % % % % % % % % % % % % % % % % % % % % % % % % % % % % % % %
% % % % % % % % % % % % % % % % % % % % % % % % % % % % % %

\chapter*{Chapter XI}
\ifaudio
\marginpar{
\href{http://ia802705.us.archive.org/22/items/war_and_peace_07_0808_librivox/war_and_peace_07_11_tolstoy_64kb.mp3}{Audio}} 
\fi

\lettrine[lines=2, loversize=0.3, lraise=0]{\initfamily P}{elageya}
Danilovna Melyukova, a broadly built, energetic woman
wearing spectacles, sat in the drawing room in a loose dress,
surrounded by her daughters whom she was trying to keep from
feeling dull. They were quietly dropping melted wax into snow and
looking at the shadows the wax figures would throw on the wall,
when they heard the steps and voices of new arrivals in the
vestibule.

Hussars, ladies, witches, clowns, and bears, after clearing their
throats and wiping the hoarfrost from their faces in the
vestibule, came into the ballroom where candles were hurriedly
lighted. The clown---Dimmler---and the lady---Nicholas---started
a dance. Surrounded by the screaming children the mummers,
covering their faces and disguising their voices, bowed to their
hostess and arranged themselves about the room.

``Dear me! there's no recognizing them! And Natasha! See whom she
looks like! She really reminds me of somebody. But Herr
Dimmler---isn't he good! I didn't know him! And how he
dances. Dear me, there's a Circassian. Really, how becoming it is
to dear Sonya. And who is that?  Well, you have cheered us up!
Nikita and Vanya---clear away the tables!  And we were sitting so
quietly. Ha, ha, ha!... The hussar, the hussar!  Just like a boy!
And the legs!... I can't look at him...'' different voices were
saying.

Natasha, the young Melyukovs' favorite, disappeared with them
into the back rooms where a cork and various dressing gowns and
male garments were called for and received from the footman by
bare girlish arms from behind the door. Ten minutes later, all
the young Melyukovs joined the mummers.

Pelageya Danilovna, having given orders to clear the rooms for
the visitors and arranged about refreshments for the gentry and
the serfs, went about among the mummers without removing her
spectacles, peering into their faces with a suppressed smile and
failing to recognize any of them. It was not merely Dimmler and
the Rostovs she failed to recognize, she did not even recognize
her own daughters, or her late husband's, dressing gowns and
uniforms, which they had put on.

``And who is this?'' she asked her governess, peering into the
face of her own daughter dressed up as a Kazan-Tartar. ``I
suppose it is one of the Rostovs! Well, Mr. Hussar, and what
regiment do you serve in?'' she asked Natasha. ``Here, hand some
fruit jelly to the Turk!'' she ordered the butler who was handing
things round. ``That's not forbidden by his law.''

Sometimes, as she looked at the strange but amusing capers cut by
the dancers, who---having decided once for all that being
disguised, no one would recognize them---were not at all shy,
Pelageya Danilovna hid her face in her handkerchief, and her
whole stout body shook with irrepressible, kindly, elderly
laughter.

``My little Sasha! Look at Sasha!'' she said.

After Russian country dances and chorus dances, Pelageya
Danilovna made the serfs and gentry join in one large circle: a
ring, a string, and a silver ruble were fetched and they all
played games together.

In an hour, all the costumes were crumpled and disordered. The
corked eyebrows and mustaches were smeared over the perspiring,
flushed, and merry faces. Pelageya Danilovna began to recognize
the mummers, admired their cleverly contrived costumes, and
particularly how they suited the young ladies, and she thanked
them all for having entertained her so well. The visitors were
invited to supper in the drawing room, and the serfs had
something served to them in the ballroom.

``Now to tell one's fortune in the empty bathhouse is
frightening!'' said an old maid who lived with the Melyukovs,
during supper.

``Why?'' said the eldest Melyukov girl.

``You wouldn't go, it takes courage...''

``I'll go,'' said Sonya.

``Tell what happened to the young lady!'' said the second
Melyukov girl.

``Well,'' began the old maid, ``a young lady once went out, took
a cock, laid the table for two, all properly, and sat down. After
sitting a while, she suddenly hears someone coming... a sleigh
drives up with harness bells; she hears him coming! He comes in,
just in the shape of a man, like an officer---comes in and sits
down to table with her.''

``Ah! ah!'' screamed Natasha, rolling her eyes with horror.

``Yes? And how... did he speak?''

``Yes, like a man. Everything quite all right, and he began
persuading her; and she should have kept him talking till
cockcrow, but she got frightened, just got frightened and hid her
face in her hands. Then he caught her up. It was lucky the maids
ran in just then...''

``Now, why frighten them?'' said Pelageya Danilovna.

``Mamma, you used to try your fate yourself...'' said her
daughter.

``And how does one do it in a barn?'' inquired Sonya.

``Well, say you went to the barn now, and listened. It depends on
what you hear; hammering and knocking---that's bad; but a sound
of shifting grain is good and one sometimes hears that, too.''

``Mamma, tell us what happened to you in the barn.''

Pelageya Danilovna smiled.

``Oh, I've forgotten...'' she replied. ``But none of you would
go?''

``Yes, I will; Pelageya Danilovna, let me! I'll go,'' said Sonya.

``Well, why not, if you're not afraid?''

``Louisa Ivanovna, may I?'' asked Sonya.

Whether they were playing the ring and string game or the ruble
game or talking as now, Nicholas did not leave Sonya's side, and
gazed at her with quite new eyes. It seemed to him that it was
only today, thanks to that burnt-cork mustache, that he had fully
learned to know her. And really, that evening, Sonya was
brighter, more animated, and prettier than Nicholas had ever seen
her before.

``So that's what she is like; what a fool I have been!'' he
thought gazing at her sparkling eyes, and under the mustache a
happy rapturous smile dimpled her cheeks, a smile he had never
seen before.

``I'm not afraid of anything,'' said Sonya. ``May I go at once?''
She got up.

They told her where the barn was and how she should stand and
listen, and they handed her a fur cloak. She threw this over her
head and shoulders and glanced at Nicholas.

``What a darling that girl is!'' thought he. ``And what have I
been thinking of till now?''

Sonya went out into the passage to go to the barn. Nicholas went
hastily to the front porch, saying he felt too hot. The crowd of
people really had made the house stuffy.

Outside, there was the same cold stillness and the same moon, but
even brighter than before. The light was so strong and the snow
sparkled with so many stars that one did not wish to look up at
the sky and the real stars were unnoticed. The sky was black and
dreary, while the earth was gay.

``I am a fool, a fool! what have I been waiting for?'' thought
Nicholas, and running out from the porch he went round the corner
of the house and along the path that led to the back porch. He
knew Sonya would pass that way. Halfway lay some snow-covered
piles of firewood and across and along them a network of shadows
from the bare old lime trees fell on the snow and on the
path. This path led to the barn. The log walls of the barn and
its snow-covered roof, that looked as if hewn out of some
precious stone, sparkled in the moonlight. A tree in the garden
snapped with the frost, and then all was again perfectly
silent. His bosom seemed to inhale not air but the strength of
eternal youth and gladness.

From the back porch came the sound of feet descending the steps,
the bottom step upon which snow had fallen gave a ringing creak
and he heard the voice of an old maidservant saying, ``Straight,
straight, along the path, Miss. Only, don't look back.''

``I am not afraid,'' answered Sonya's voice, and along the path
toward Nicholas came the crunching, whistling sound of Sonya's
feet in her thin shoes.

Sonya came along, wrapped in her cloak. She was only a couple of
paces away when she saw him, and to her too he was not the
Nicholas she had known and always slightly feared. He was in a
woman's dress, with tousled hair and a happy smile new to
Sonya. She ran rapidly toward him.

``Quite different and yet the same,'' thought Nicholas, looking
at her face all lit up by the moonlight. He slipped his arms
under the cloak that covered her head, embraced her, pressed her
to him, and kissed her on the lips that wore a mustache and had a
smell of burnt cork. Sonya kissed him full on the lips, and
disengaging her little hands pressed them to his cheeks.

``Sonya!... Nicholas!''... was all they said. They ran to the
barn and then back again, re-entering, he by the front and she by
the back porch.

% % % % % % % % % % % % % % % % % % % % % % % % % % % % % % % % %
% % % % % % % % % % % % % % % % % % % % % % % % % % % % % % % % %
% % % % % % % % % % % % % % % % % % % % % % % % % % % % % % % % %
% % % % % % % % % % % % % % % % % % % % % % % % % % % % % % % % %
% % % % % % % % % % % % % % % % % % % % % % % % % % % % % % % % %
% % % % % % % % % % % % % % % % % % % % % % % % % % % % % % % % %
% % % % % % % % % % % % % % % % % % % % % % % % % % % % % % % % %
% % % % % % % % % % % % % % % % % % % % % % % % % % % % % % % % %
% % % % % % % % % % % % % % % % % % % % % % % % % % % % % % % % %
% % % % % % % % % % % % % % % % % % % % % % % % % % % % % % % % %
% % % % % % % % % % % % % % % % % % % % % % % % % % % % % % % % %
% % % % % % % % % % % % % % % % % % % % % % % % % % % % % %

\chapter*{Chapter XII}
\ifaudio
\marginpar{
\href{http://ia802705.us.archive.org/22/items/war_and_peace_07_0808_librivox/war_and_peace_07_12_tolstoy_64kb.mp3}{Audio}} 
\fi

\lettrine[lines=2, loversize=0.3, lraise=0]{\initfamily W}{hen}
they all drove back from Pelageya Danilovna's, Natasha, who
always saw and noticed everything, arranged that she and Madame
Schoss should go back in the sleigh with Dimmler, and Sonya with
Nicholas and the maids.

On the way back Nicholas drove at a steady pace instead of racing
and kept peering by that fantastic all-transforming light into
Sonya's face and searching beneath the eyebrows and mustache for
his former and his present Sonya from whom he had resolved never
to be parted again. He looked and recognizing in her both the old
and the new Sonya, and being reminded by the smell of burnt cork
of the sensation of her kiss, inhaled the frosty air with a full
breast and, looking at the ground flying beneath him and at the
sparkling sky, felt himself again in fairyland.

``Sonya, is it well with thee?'' he asked from time to time.

``Yes!'' she replied. ``And with thee?''

When halfway home Nicholas handed the reins to the coachman and
ran for a moment to Natasha's sleigh and stood on its wing.

``Natasha!'' he whispered in French, ``do you know I have made up
my mind about Sonya?''

``Have you told her?'' asked Natasha, suddenly beaming all over
with joy.

``Oh, how strange you are with that mustache and those
eyebrows!...  Natasha---are you glad?''

``I am so glad, so glad! I was beginning to be vexed with you. I
did not tell you, but you have been treating her badly. What a
heart she has, Nicholas! I am horrid sometimes, but I was ashamed
to be happy while Sonya was not,'' continued Natasha. ``Now I am
so glad! Well, run back to her.''

``No, wait a bit... Oh, how funny you look!'' cried Nicholas,
peering into her face and finding in his sister too something
new, unusual, and bewitchingly tender that he had not seen in her
before. ``Natasha, it's magical, isn't it?''

``Yes,'' she replied. ``You have done splendidly.''

``Had I seen her before as she is now,'' thought Nicholas, ``I
should long ago have asked her what to do and have done whatever
she told me, and all would have been well.''

``So you are glad and I have done right?''

``Oh, quite right! I had a quarrel with Mamma some time ago about
it.  Mamma said she was angling for you. How could she say such a
thing! I nearly stormed at Mamma. I will never let anyone say
anything bad of Sonya, for there is nothing but good in her.''

``Then it's all right?'' said Nicholas, again scrutinizing the
expression of his sister's face to see if she was in
earnest. Then he jumped down and, his boots scrunching the snow,
ran back to his sleigh. The same happy, smiling Circassian, with
mustache and beaming eyes looking up from under a sable hood, was
still sitting there, and that Circassian was Sonya, and that
Sonya was certainly his future happy and loving wife.

When they reached home and had told their mother how they had
spent the evening at the Melyukovs', the girls went to their
bedroom. When they had undressed, but without washing off the
cork mustaches, they sat a long time talking of their
happiness. They talked of how they would live when they were
married, how their husbands would be friends, and how happy they
would be. On Natasha's table stood two looking glasses which
Dunyasha had prepared beforehand.

``Only when will all that be? I am afraid never... It would be
too good!'' said Natasha, rising and going to the looking
glasses.

``Sit down, Natasha; perhaps you'll see him,'' said Sonya.

Natasha lit the candles, one on each side of one of the looking
glasses, and sat down.

``I see someone with a mustache,'' said Natasha, seeing her own
face.

``You mustn't laugh, Miss,'' said Dunyasha.

With Sonya's help and the maid's, Natasha got the glass she held
into the right position opposite the other; her face assumed a
serious expression and she sat silent. She sat a long time
looking at the receding line of candles reflected in the glasses
and expecting (from tales she had heard) to see a coffin, or him,
Prince Andrew, in that last dim, indistinctly outlined
square. But ready as she was to take the smallest speck for the
image of a man or of a coffin, she saw nothing.  She began
blinking rapidly and moved away from the looking glasses.

``Why is it others see things and I don't?'' she said. ``You sit
down now, Sonya. You absolutely must, tonight! Do it for
me... Today I feel so frightened!''

Sonya sat down before the glasses, got the right position, and
began looking.

``Now, Miss Sonya is sure to see something,'' whispered Dunyasha;
``while you do nothing but laugh.''

Sonya heard this and Natasha's whisper:

``I know she will. She saw something last year.''

For about three minutes all were silent.

``Of course she will!'' whispered Natasha, but did not
finish... suddenly Sonya pushed away the glass she was holding
and covered her eyes with her hand.

``Oh, Natasha!'' she cried.

``Did you see? Did you? What was it?'' exclaimed Natasha, holding
up the looking glass.

Sonya had not seen anything, she was just wanting to blink and to
get up when she heard Natasha say, ``Of course she will!'' She
did not wish to disappoint either Dunyasha or Natasha, but it was
hard to sit still. She did not herself know how or why the
exclamation escaped her when she covered her eyes.

``You saw him?'' urged Natasha, seizing her hand.

``Yes. Wait a bit... I... saw him,'' Sonya could not help saying,
not yet knowing whom Natasha meant by him, Nicholas or Prince
Andrew.

``But why shouldn't I say I saw something? Others do see! Besides
who can tell whether I saw anything or not?'' flashed through
Sonya's mind.

``Yes, I saw him,'' she said.

``How? Standing or lying?''

``No, I saw... At first there was nothing, then I saw him lying
down.''

``Andrew lying? Is he ill?'' asked Natasha, her frightened eyes
fixed on her friend.

``No, on the contrary, on the contrary! His face was cheerful,
and he turned to me.'' And when saying this she herself fancied
she had really seen what she described.

``Well, and then, Sonya?...''

``After that, I could not make out what there was; something blue
and red...''

``Sonya! When will he come back? When shall I see him! O, God,
how afraid I am for him and for myself and about everything!...''
Natasha began, and without replying to Sonya's words of comfort
she got into bed, and long after her candle was out lay open-eyed
and motionless, gazing at the moonlight through the frosty
windowpanes.

% % % % % % % % % % % % % % % % % % % % % % % % % % % % % % % % %
% % % % % % % % % % % % % % % % % % % % % % % % % % % % % % % % %
% % % % % % % % % % % % % % % % % % % % % % % % % % % % % % % % %
% % % % % % % % % % % % % % % % % % % % % % % % % % % % % % % % %
% % % % % % % % % % % % % % % % % % % % % % % % % % % % % % % % %
% % % % % % % % % % % % % % % % % % % % % % % % % % % % % % % % %
% % % % % % % % % % % % % % % % % % % % % % % % % % % % % % % % %
% % % % % % % % % % % % % % % % % % % % % % % % % % % % % % % % %
% % % % % % % % % % % % % % % % % % % % % % % % % % % % % % % % %
% % % % % % % % % % % % % % % % % % % % % % % % % % % % % % % % %
% % % % % % % % % % % % % % % % % % % % % % % % % % % % % % % % %
% % % % % % % % % % % % % % % % % % % % % % % % % % % % % %

\chapter*{Chapter XIII}
\ifaudio     
\marginpar{
\href{http://ia802705.us.archive.org/22/items/war_and_peace_07_0808_librivox/war_and_peace_07_13_tolstoy_64kb.mp3}{Audio}} 
\fi

\lettrine[lines=2, loversize=0.3, lraise=0]{\initfamily S}{oon}
after the Christmas holidays Nicholas told his mother of his
love for Sonya and of his firm resolve to marry her. The
countess, who had long noticed what was going on between them and
was expecting this declaration, listened to him in silence and
then told her son that he might marry whom he pleased, but that
neither she nor his father would give their blessing to such a
marriage. Nicholas, for the first time, felt that his mother was
displeased with him and that, despite her love for him, she would
not give way. Coldly, without looking at her son, she sent for
her husband and, when he came, tried briefly and coldly to inform
him of the facts, in her son's presence, but unable to restrain
herself she burst into tears of vexation and left the room. The
old count began irresolutely to admonish Nicholas and beg him to
abandon his purpose. Nicholas replied that he could not go back
on his word, and his father, sighing and evidently disconcerted,
very soon became silent and went in to the countess. In all his
encounters with his son, the count was always conscious of his
own guilt toward him for having wasted the family fortune, and so
he could not be angry with him for refusing to marry an heiress
and choosing the dowerless Sonya. On this occasion, he was only
more vividly conscious of the fact that if his affairs had not
been in disorder, no better wife for Nicholas than Sonya could
have been wished for, and that no one but himself with his
Mitenka and his uncomfortable habits was to blame for the
condition of the family finances.

The father and mother did not speak of the matter to their son
again, but a few days later the countess sent for Sonya and, with
a cruelty neither of them expected, reproached her niece for
trying to catch Nicholas and for ingratitude. Sonya listened
silently with downcast eyes to the countess' cruel words, without
understanding what was required of her. She was ready to
sacrifice everything for her benefactors. Self-sacrifice was her
most cherished idea but in this case she could not see what she
ought to sacrifice, or for whom. She could not help loving the
countess and the whole Rostov family, but neither could she help
loving Nicholas and knowing that his happiness depended on that
love. She was silent and sad and did not reply. Nicholas felt the
situation to be intolerable and went to have an explanation with
his mother. He first implored her to forgive him and Sonya and
consent to their marriage, then he threatened that if she
molested Sonya he would at once marry her secretly.

The countess, with a coldness her son had never seen in her
before, replied that he was of age, that Prince Andrew was
marrying without his father's consent, and he could do the same,
but that she would never receive that intriguer as her daughter.

Exploding at the word intriguer, Nicholas, raising his voice,
told his mother he had never expected her to try to force him to
sell his feelings, but if that were so, he would say for the last
time... But he had no time to utter the decisive word which the
expression of his face caused his mother to await with terror,
and which would perhaps have forever remained a cruel memory to
them both. He had not time to say it, for Natasha, with a pale
and set face, entered the room from the door at which she had
been listening.

``Nicholas, you are talking nonsense! Be quiet, be quiet, be
quiet, I tell you!...'' she almost screamed, so as to drown his
voice.

``Mamma darling, it's not at all so... my poor, sweet darling,''
she said to her mother, who conscious that they had been on the
brink of a rupture gazed at her son with terror, but in the
obstinacy and excitement of the conflict could not and would not
give way.

``Nicholas, I'll explain to you. Go away! Listen, Mamma
darling,'' said Natasha.

Her words were incoherent, but they attained the purpose at which
she was aiming.

The countess, sobbing heavily, hid her face on her daughter's
breast, while Nicholas rose, clutching his head, and left the
room.

Natasha set to work to effect a reconciliation, and so far
succeeded that Nicholas received a promise from his mother that
Sonya should not be troubled, while he on his side promised not
to undertake anything without his parents' knowledge.

Firmly resolved, after putting his affairs in order in the
regiment, to retire from the army and return and marry Sonya,
Nicholas, serious, sorrowful, and at variance with his parents,
but, as it seemed to him, passionately in love, left at the
beginning of January to rejoin his regiment.

After Nicholas had gone things in the Rostov household were more
depressing than ever, and the countess fell ill from mental
agitation.

Sonya was unhappy at the separation from Nicholas and still more
so on account of the hostile tone the countess could not help
adopting toward her. The count was more perturbed than ever by
the condition of his affairs, which called for some decisive
action. Their town house and estate near Moscow had inevitably to
be sold, and for this they had to go to Moscow. But the countess'
health obliged them to delay their departure from day to day.

Natasha, who had borne the first period of separation from her
betrothed lightly and even cheerfully, now grew more agitated and
impatient every day. The thought that her best days, which she
would have employed in loving him, were being vainly wasted, with
no advantage to anyone, tormented her incessantly. His letters
for the most part irritated her.  It hurt her to think that while
she lived only in the thought of him, he was living a real life,
seeing new places and new people that interested him. The more
interesting his letters were the more vexed she felt. Her letters
to him, far from giving her any comfort, seemed to her a
wearisome and artificial obligation. She could not write, because
she could not conceive the possibility of expressing sincerely in
a letter even a thousandth part of what she expressed by voice,
smile, and glance. She wrote to him formal, monotonous, and dry
letters, to which she attached no importance herself, and in the
rough copies of which the countess corrected her mistakes in
spelling.

There was still no improvement in the countess' health, but it
was impossible to defer the journey to Moscow any
longer. Natasha's trousseau had to be ordered and the house
sold. Moreover, Prince Andrew was expected in Moscow, where old
Prince Bolkonski was spending the winter, and Natasha felt sure
he had already arrived.

So the countess remained in the country, and the count, taking
Sonya and Natasha with him, went to Moscow at the end of January.