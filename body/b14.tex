\part*{Book Fourteen: 1812}

% % % % % % % % % % % % % % % % % % % % % % % % % % % % % % % % %
% % % % % % % % % % % % % % % % % % % % % % % % % % % % % % % % %
% % % % % % % % % % % % % % % % % % % % % % % % % % % % % % % % %
% % % % % % % % % % % % % % % % % % % % % % % % % % % % % % % % %
% % % % % % % % % % % % % % % % % % % % % % % % % % % % % % % % %
% % % % % % % % % % % % % % % % % % % % % % % % % % % % % % % % %
% % % % % % % % % % % % % % % % % % % % % % % % % % % % % % % % %
% % % % % % % % % % % % % % % % % % % % % % % % % % % % % % % % %
% % % % % % % % % % % % % % % % % % % % % % % % % % % % % % % % %
% % % % % % % % % % % % % % % % % % % % % % % % % % % % % % % % %
% % % % % % % % % % % % % % % % % % % % % % % % % % % % % % % % %
% % % % % % % % % % % % % % % % % % % % % % % % % % % % % %

\chapter*{Chapter I} \ifaudio \marginpar{
\href{http://ia801404.us.archive.org/17/items/war_and_peace_14_1003_librivox/war_and_peace_14_01_tolstoy_64kb.mp3}{Audio}}
\fi

\initial{T}{he} Battle of Borodino, with the occupation of Moscow that
followed it and the flight of the French without further
conflicts, is one of the most instructive phenomena in history.

All historians agree that the external activity of states and
nations in their conflicts with one another is expressed in wars,
and that as a direct result of greater or less success in war the
political strength of states and nations increases or decreases.

Strange as may be the historical account of how some king or
emperor, having quarreled with another, collects an army, fights
his enemy's army, gains a victory by killing three, five, or ten
thousand men, and subjugates a kingdom and an entire nation of
several millions, all the facts of history (as far as we know it)
confirm the truth of the statement that the greater or lesser
success of one army against another is the cause, or at least an
essential indication, of an increase or decrease in the strength
of the nation---even though it is unintelligible why the defeat
of an army---a hundredth part of a nation---should oblige that
whole nation to submit. An army gains a victory, and at once the
rights of the conquering nation have increased to the detriment
of the defeated. An army has suffered defeat, and at once a
people loses its rights in proportion to the severity of the
reverse, and if its army suffers a complete defeat the nation is
quite subjugated.

So according to history it has been found from the most ancient
times, and so it is to our own day. All Napoleon's wars serve to
confirm this rule. In proportion to the defeat of the Austrian
army Austria loses its rights, and the rights and the strength of
France increase. The victories of the French at Jena and
Auerstadt destroy the independent existence of Prussia.

But then, in 1812, the French gain a victory near Moscow. Moscow
is taken and after that, with no further battles, it is not
Russia that ceases to exist, but the French army of six hundred
thousand, and then Napoleonic France itself. To strain the facts
to fit the rules of history: to say that the field of battle at
Borodino remained in the hands of the Russians, or that after
Moscow there were other battles that destroyed Napoleon's army,
is impossible.

After the French victory at Borodino there was no general
engagement nor any that were at all serious, yet the French army
ceased to exist. What does this mean? If it were an example taken
from the history of China, we might say that it was not an
historic phenomenon (which is the historians' usual expedient
when anything does not fit their standards); if the matter
concerned some brief conflict in which only a small number of
troops took part, we might treat it as an exception; but this
event occurred before our fathers' eyes, and for them it was a
question of the life or death of their fatherland, and it
happened in the greatest of all known wars.

The period of the campaign of 1812 from the battle of Borodino to
the expulsion of the French proved that the winning of a battle
does not produce a conquest and is not even an invariable
indication of conquest; it proved that the force which decides
the fate of peoples lies not in the conquerors, nor even in
armies and battles, but in something else.

The French historians, describing the condition of the French
army before it left Moscow, affirm that all was in order in the
Grand Army, except the cavalry, the artillery, and the
transport---there was no forage for the horses or the
cattle. That was a misfortune no one could remedy, for the
peasants of the district burned their hay rather than let the
French have it.

The victory gained did not bring the usual results because the
peasants Karp and Vlas (who after the French had evacuated Moscow
drove in their carts to pillage the town, and in general
personally failed to manifest any heroic feelings), and the whole
innumerable multitude of such peasants, did not bring their hay
to Moscow for the high price offered them, but burned it instead.

Let us imagine two men who have come out to fight a duel with
rapiers according to all the rules of the art of fencing. The
fencing has gone on for some time; suddenly one of the
combatants, feeling himself wounded and understanding that the
matter is no joke but concerns his life, throws down his rapier,
and seizing the first cudgel that comes to hand begins to
brandish it. Then let us imagine that the combatant who so
sensibly employed the best and simplest means to attain his end
was at the same time influenced by traditions of chivalry and,
desiring to conceal the facts of the case, insisted that he had
gained his victory with the rapier according to all the rules of
art. One can imagine what confusion and obscurity would result
from such an account of the duel.

The fencer who demanded a contest according to the rules of
fencing was the French army; his opponent who threw away the
rapier and snatched up the cudgel was the Russian people; those
who try to explain the matter according to the rules of fencing
are the historians who have described the event.

After the burning of Smolensk a war began which did not follow
any previous traditions of war. The burning of towns and
villages, the retreats after battles, the blow dealt at Borodino
and the renewed retreat, the burning of Moscow, the capture of
marauders, the seizure of transports, and the guerrilla war were
all departures from the rules.

Napoleon felt this, and from the time he took up the correct
fencing attitude in Moscow and instead of his opponent's rapier
saw a cudgel raised above his head, he did not cease to complain
to Kutuzov and to the Emperor Alexander that the war was being
carried on contrary to all the rules---as if there were any rules
for killing people. In spite of the complaints of the French as
to the nonobservance of the rules, in spite of the fact that to
some highly placed Russians it seemed rather disgraceful to fight
with a cudgel and they wanted to assume a pose en quarte or en
tierce according to all the rules, and to make an adroit thrust
en prime, and so on---the cudgel of the people's war was lifted
with all its menacing and majestic strength, and without
consulting anyone's tastes or rules and regardless of anything
else, it rose and fell with stupid simplicity, but consistently,
and belabored the French till the whole invasion had perished.

And it is well for a people who do not---as the French did in
1813---salute according to all the rules of art, and, presenting
the hilt of their rapier gracefully and politely, hand it to
their magnanimous conqueror, but at the moment of trial, without
asking what rules others have adopted in similar cases, simply
and easily pick up the first cudgel that comes to hand and strike
with it till the feeling of resentment and revenge in their soul
yields to a feeling of contempt and compassion.

% % % % % % % % % % % % % % % % % % % % % % % % % % % % % % % % %
% % % % % % % % % % % % % % % % % % % % % % % % % % % % % % % % %
% % % % % % % % % % % % % % % % % % % % % % % % % % % % % % % % %
% % % % % % % % % % % % % % % % % % % % % % % % % % % % % % % % %
% % % % % % % % % % % % % % % % % % % % % % % % % % % % % % % % %
% % % % % % % % % % % % % % % % % % % % % % % % % % % % % % % % %
% % % % % % % % % % % % % % % % % % % % % % % % % % % % % % % % %
% % % % % % % % % % % % % % % % % % % % % % % % % % % % % % % % %
% % % % % % % % % % % % % % % % % % % % % % % % % % % % % % % % %
% % % % % % % % % % % % % % % % % % % % % % % % % % % % % % % % %
% % % % % % % % % % % % % % % % % % % % % % % % % % % % % % % % %
% % % % % % % % % % % % % % % % % % % % % % % % % % % % % %

\chapter*{Chapter II} \ifaudio \marginpar{
\href{http://ia801404.us.archive.org/17/items/war_and_peace_14_1003_librivox/war_and_peace_14_02_tolstoy_64kb.mp3}{Audio}}
\fi

\initial{O}{ne} of the most obvious and advantageous departures from the
so-called laws of war is the action of scattered groups against
men pressed together in a mass. Such action always occurs in wars
that take on a national character. In such actions, instead of
two crowds opposing each other, the men disperse, attack singly,
run away when attacked by stronger forces, but again attack when
opportunity offers. This was done by the guerrillas in Spain, by
the mountain tribes in the Caucasus, and by the Russians in 1812.

People have called this kind of war \emph{guerrilla warfare} and
assume that by so calling it they have explained its meaning. But
such a war does not fit in under any rule and is directly opposed
to a well-known rule of tactics which is accepted as
infallible. That rule says that an attacker should concentrate
his forces in order to be stronger than his opponent at the
moment of conflict.

Guerrilla war (always successful, as history shows) directly
infringes that rule.

This contradiction arises from the fact that military science
assumes the strength of an army to be identical with its
numbers. Military science says that the more troops the greater
the strength. Les gros bataillons ont toujours
raison.\footnote{Large battalions are always victorious.}

For military science to say this is like defining momentum in
mechanics by reference to the mass only: stating that momenta are
equal or unequal to each other simply because the masses involved
are equal or unequal.

Momentum (quantity of motion) is the product of mass and
velocity.

In military affairs the strength of an army is the product of its
mass and some unknown $x$.

Military science, seeing in history innumerable instances of the
fact that the size of any army does not coincide with its
strength and that small detachments defeat larger ones, obscurely
admits the existence of this unknown factor and tries to discover
it---now in a geometric formation, now in the equipment employed,
now, and most usually, in the genius of the commanders. But the
assignment of these various meanings to the factor does not yield
results which accord with the historic facts.

Yet it is only necessary to abandon the false view (adopted to
gratify the \emph{heroes}) of the efficacy of the directions
issued in wartime by commanders, in order to find this unknown
quantity.

That unknown quantity is the spirit of the army, that is to say,
the greater or lesser readiness to fight and face danger felt by
all the men composing an army, quite independently of whether
they are, or are not, fighting under the command of a genius, in
two---or three-line formation, with cudgels or with rifles that
repeat thirty times a minute. Men who want to fight will always
put themselves in the most advantageous conditions for fighting.

The spirit of an army is the factor which multiplied by the mass
gives the resulting force. To define and express the significance
of this unknown factor---the spirit of an army---is a problem for
science.

This problem is only solvable if we cease arbitrarily to
substitute for the unknown $x$ itself the conditions under which
that force becomes apparent---such as the commands of the
general, the equipment employed, and so on---mistaking these for
the real significance of the factor, and if we recognize this
unknown quantity in its entirety as being the greater or lesser
desire to fight and to face danger. Only then, expressing known
historic facts by equations and comparing the relative
significance of this factor, can we hope to define the unknown.

Ten men, battalions, or divisions, fighting fifteen men,
battalions, or divisions, conquer---that is, kill or take
captive---all the others, while themselves losing four, so that
on the one side four and on the other fifteen were
lost. Consequently the four were equal to the fifteen, and
therefore $4x = 15y$. Consequently $x/y = 15/4$. This equation
does not give us the value of the unknown factor but gives us a
ratio between two unknowns. And by bringing variously selected
historic units (battles, campaigns, periods of war) into such
equations, a series of numbers could be obtained in which certain
laws should exist and might be discovered.

The tactical rule that an army should act in masses when
attacking, and in smaller groups in retreat, unconsciously
confirms the truth that the strength of an army depends on its
spirit. To lead men forward under fire more discipline
(obtainable only by movement in masses) is needed than is needed
to resist attacks. But this rule which leaves out of account the
spirit of the army continually proves incorrect and is in
particularly striking contrast to the facts when some strong rise
or fall in the spirit of the troops occurs, as in all national
wars.

The French, retreating in 1812---though according to tactics they
should have separated into detachments to defend
themselves---cong\-re\-ga\-ted into a mass because the spirit of the
army had so fallen that only the mass held the army together. The
Russians, on the contrary, ought according to tactics to have
attacked in mass, but in fact they split up into small units,
because their spirit had so risen that separate individuals,
without orders, dealt blows at the French without needing any
compulsion to induce them to expose themselves to hardships and
dangers.

% % % % % % % % % % % % % % % % % % % % % % % % % % % % % % % % %
% % % % % % % % % % % % % % % % % % % % % % % % % % % % % % % % %
% % % % % % % % % % % % % % % % % % % % % % % % % % % % % % % % %
% % % % % % % % % % % % % % % % % % % % % % % % % % % % % % % % %
% % % % % % % % % % % % % % % % % % % % % % % % % % % % % % % % %
% % % % % % % % % % % % % % % % % % % % % % % % % % % % % % % % %
% % % % % % % % % % % % % % % % % % % % % % % % % % % % % % % % %
% % % % % % % % % % % % % % % % % % % % % % % % % % % % % % % % %
% % % % % % % % % % % % % % % % % % % % % % % % % % % % % % % % %
% % % % % % % % % % % % % % % % % % % % % % % % % % % % % % % % %
% % % % % % % % % % % % % % % % % % % % % % % % % % % % % % % % %
% % % % % % % % % % % % % % % % % % % % % % % % % % % % % %

\chapter*{Chapter III} \ifaudio \marginpar{
\href{http://ia801404.us.archive.org/17/items/war_and_peace_14_1003_librivox/war_and_peace_14_03_tolstoy_64kb.mp3}{Audio}}
\fi

\initial{T}{he} so-called partisan war began with the entry of the French
into Smolensk.

Before partisan warfare had been officially recognized by the
government, thousands of enemy stragglers, marauders, and
foragers had been destroyed by the Cossacks and the peasants, who
killed them off as instinctively as dogs worry a stray mad dog to
death. Denis Davydov, with his Russian instinct, was the first to
recognize the value of this terrible cudgel which regardless of
the rules of military science destroyed the French, and to him
belongs the credit for taking the first step toward regularizing
this method of warfare.

On August 24th Davydov's first partisan detachment was formed and
then others were recognized. The further the campaign progressed
the more numerous these detachments became.

The irregulars destroyed the great army piecemeal. They gathered
the fallen leaves that dropped of themselves from that withered
tree---the French army---and sometimes shook that tree itself. By
October, when the French were fleeing toward Smolensk, there were
hundreds of such companies, of various sizes and
characters. There were some that adopted all the army methods and
had infantry, artillery, staffs, and the comforts of life. Others
consisted solely of Cossack cavalry. There were also small
scratch groups of foot and horse, and groups of peasants and
landowners that remained unknown. A sacristan commanded one party
which captured several hundred prisoners in the course of a
month; and there was Vasilisa, the wife of a village elder, who
slew hundreds of the French.

The partisan warfare flamed up most fiercely in the latter days
of October. Its first period had passed: when the partisans
themselves, amazed at their own boldness, feared every minute to
be surrounded and captured by the French, and hid in the forests
without unsaddling, hardly daring to dismount and always
expecting to be pursued. By the end of October this kind of
warfare had taken definite shape: it had become clear to all what
could be ventured against the French and what could not. Now only
the commanders of detachments with staffs, and moving according
to rules at a distance from the French, still regarded many
things as impossible. The small bands that had started their
activities long before and had already observed the French
closely considered things possible which the commanders of the
big detachments did not dare to contemplate. The Cossacks and
peasants who crept in among the French now considered everything
possible.

On October 22nd, Denisov (who was one of the irregulars) was with
his group at the height of the guerrilla enthusiasm. Since early
morning he and his party had been on the move. All day long he
had been watching from the forest that skirted the highroad a
large French convoy of cavalry baggage and Russian prisoners
separated from the rest of the army, which---as was learned from
spies and prisoners---was moving under a strong escort to
Smolensk. Besides Denisov and Dolokhov (who also led a small
party and moved in Denisov's vicinity), the commanders of some
large divisions with staffs also knew of this convoy and, as
Denisov expressed it, were sharpening their teeth for it. Two of
the commanders of large parties---one a Pole and the other a
German---sent invitations to Denisov almost simultaneously,
requesting him to join up with their divisions to attack the
convoy.

``No, bwother, I have gwown mustaches myself,'' said Denisov on
reading these documents, and he wrote to the German that, despite
his heartfelt desire to serve under so valiant and renowned a
general, he had to forgo that pleasure because he was already
under the command of the Polish general. To the Polish general he
replied to the same effect, informing him that he was already
under the command of the German.

Having arranged matters thus, Denisov and Dolokhov intended,
without reporting matters to the higher command, to attack and
seize that convoy with their own small forces. On October 22 it
was moving from the village of Mikulino to that of Shamshevo. To
the left of the road between Mikulino and Shamshevo there were
large forests, extending in some places up to the road itself
though in others a mile or more back from it. Through these
forests Denisov and his party rode all day, sometimes keeping
well back in them and sometimes coming to the very edge, but
never losing sight of the moving French. That morning, Cossacks
of Denisov's party had seized and carried off into the forest two
wagons loaded with cavalry saddles, which had stuck in the mud
not far from Mikulino where the forest ran close to the
road. Since then, and until evening, the party had watched the
movements of the French without attacking. It was necessary to
let the French reach Shamshevo quietly without alarming them and
then, after joining Dolokhov who was to come that evening to a
consultation at a watchman's hut in the forest less than a mile
from Shamshevo, to surprise the French at dawn, falling like an
avalanche on their heads from two sides, and rout and capture
them all at one blow.

In their rear, more than a mile from Mikulino where the forest
came right up to the road, six Cossacks were posted to report if
any fresh columns of French should show themselves.

Beyond Shamshevo, Dolokhov was to observe the road in the same
way, to find out at what distance there were other French
troops. They reckoned that the convoy had fifteen hundred
men. Denisov had two hundred, and Dolokhov might have as many
more, but the disparity of numbers did not deter Denisov. All
that he now wanted to know was what troops these were and to
learn that he had to capture a \emph{tongue}---that is, a man
from the enemy column. That morning's attack on the wagons had
been made so hastily that the Frenchmen with the wagons had all
been killed; only a little drummer boy had been taken alive, and
as he was a straggler he could tell them nothing definite about
the troops in that column.

Denisov considered it dangerous to make a second attack for fear
of putting the whole column on the alert, so he sent Tikhon
Shcherbaty, a peasant of his party, to Shamshevo to try and seize
at least one of the French quartermasters who had been sent on in
advance.

% % % % % % % % % % % % % % % % % % % % % % % % % % % % % % % % %
% % % % % % % % % % % % % % % % % % % % % % % % % % % % % % % % %
% % % % % % % % % % % % % % % % % % % % % % % % % % % % % % % % %
% % % % % % % % % % % % % % % % % % % % % % % % % % % % % % % % %
% % % % % % % % % % % % % % % % % % % % % % % % % % % % % % % % %
% % % % % % % % % % % % % % % % % % % % % % % % % % % % % % % % %
% % % % % % % % % % % % % % % % % % % % % % % % % % % % % % % % %
% % % % % % % % % % % % % % % % % % % % % % % % % % % % % % % % %
% % % % % % % % % % % % % % % % % % % % % % % % % % % % % % % % %
% % % % % % % % % % % % % % % % % % % % % % % % % % % % % % % % %
% % % % % % % % % % % % % % % % % % % % % % % % % % % % % % % % %
% % % % % % % % % % % % % % % % % % % % % % % % % % % % % %

\chapter*{Chapter IV} \ifaudio \marginpar{
\href{http://ia801404.us.archive.org/17/items/war_and_peace_14_1003_librivox/war_and_peace_14_04_tolstoy_64kb.mp3}{Audio}}
\fi

\initial{I}{t} was a warm rainy autumn day. The sky and the horizon were both
the color of muddy water. At times a sort of mist descended, and
then suddenly heavy slanting rain came down.

Denisov in a felt cloak and a sheepskin cap from which the rain
ran down was riding a thin thoroughbred horse with sunken
sides. Like his horse, which turned its head and laid its ears
back, he shrank from the driving rain and gazed anxiously before
him. His thin face with its short, thick black beard looked
angry.

Beside Denisov rode an esaul,\footnote{A captain of Cossacks.}
Denisov's fellow worker, also in felt cloak and sheepskin cap,
and riding a large sleek Don horse.

Esaul Lovayski the Third was a tall man as straight as an arrow,
pale-faced, fair-haired, with narrow light eyes and with calm
self- satisfaction in his face and bearing. Though it was
impossible to say in what the peculiarity of the horse and rider
lay, yet at first glance at the esaul and Denisov one saw that
the latter was wet and uncomfortable and was a man mounted on a
horse, while looking at the esaul one saw that he was as
comfortable and as much at ease as always and that he was not a
man who had mounted a horse, but a man who was one with his
horse, a being consequently possessed of twofold strength.

A little ahead of them walked a peasant guide, wet to the skin
and wearing a gray peasant coat and a white knitted cap.

A little behind, on a poor, small, lean Kirghiz mount with an
enormous tail and mane and a bleeding mouth, rode a young officer
in a blue French overcoat.

Beside him rode an hussar, with a boy in a tattered French
uniform and blue cap behind him on the crupper of his horse. The
boy held on to the hussar with cold, red hands, and raising his
eyebrows gazed about him with surprise. This was the French
drummer boy captured that morning.

Behind them along the narrow, sodden, cutup forest road came
hussars in threes and fours, and then Cossacks: some in felt
cloaks, some in French greatcoats, and some with horsecloths over
their heads. The horses, being drenched by the rain, all looked
black whether chestnut or bay.  Their necks, with their wet,
close-clinging manes, looked strangely thin. Steam rose from
them. Clothes, saddles, reins, were all wet, slippery, and
sodden, like the ground and the fallen leaves that strewed the
road. The men sat huddled up trying not to stir, so as to warm
the water that had trickled to their bodies and not admit the
fresh cold water that was leaking in under their seats, their
knees, and at the back of their necks. In the midst of the
outspread line of Cossacks two wagons, drawn by French horses and
by saddled Cossack horses that had been hitched on in front,
rumbled over the tree stumps and branches and splashed through
the water that lay in the ruts.

Denisov's horse swerved aside to avoid a pool in the track and
bumped his rider's knee against a tree.

``Oh, the devil!'' exclaimed Denisov angrily, and showing his
teeth he struck his horse three times with his whip, splashing
himself and his comrades with mud.

Denisov was out of sorts both because of the rain and also from
hunger (none of them had eaten anything since morning), and yet
more because he still had no news from Dolokhov and the man sent
to capture a \emph{tongue} had not returned.

``There'll hardly be another such chance to fall on a transport
as today.  It's too risky to attack them by oneself, and if we
put it off till another day one of the big guerrilla detachments
will snatch the prey from under our noses,'' thought Denisov,
continually peering forward, hoping to see a messenger from
Dolokhov.

On coming to a path in the forest along which he could see far to
the right, Denisov stopped.

``There's someone coming,'' said he.

The esaul looked in the direction Denisov indicated.

``There are two, an officer and a Cossack. But it is not
presupposable that it is the lieutenant colonel himself,'' said
the esaul, who was fond of using words the Cossacks did not know.

The approaching riders having descended a decline were no longer
visible, but they reappeared a few minutes later. In front, at a
weary gallop and using his leather whip, rode an officer,
disheveled and drenched, whose trousers had worked up to above
his knees. Behind him, standing in the stirrups, trotted a
Cossack. The officer, a very young lad with a broad rosy face and
keen merry eyes, galloped up to Denisov and handed him a sodden
envelope.

``From the general,'' said the officer. ``Please excuse its not
being quite dry.''

Denisov, frowning, took the envelope and opened it.

``There, they kept telling us: 'It's dangerous, it's dangerous,'\
'' said the officer, addressing the esaul while Denisov was
reading the dispatch. ``But Komarov and I''---he pointed to the
Cossack---``were prepared. We have each of us two pistols... But
what's this?'' he asked, noticing the French drummer boy. ``A
prisoner? You've already been in action? May I speak to him?''

``Wostov! Petya!'' exclaimed Denisov, having run through the
dispatch.  ``Why didn't you say who you were?'' and turning with
a smile he held out his hand to the lad.

The officer was Petya Rostov.

All the way Petya had been preparing himself to behave with
Denisov as befitted a grownup man and an officer---without
hinting at their previous acquaintance. But as soon as Denisov
smiled at him Petya brightened up, blushed with pleasure, forgot
the official manner he had been rehearsing, and began telling him
how he had already been in a battle near Vyazma and how a certain
hussar had distinguished himself there.

``Well, I am glad to see you,'' Denisov interrupted him, and his
face again assumed its anxious expression.

``Michael Feoklitych,'' said he to the esaul, ``this is again
fwom that German, you know. He''---he indicated Petya---``is
serving under him.''

And Denisov told the esaul that the dispatch just delivered was a
repetition of the German general's demand that he should join
forces with him for an attack on the transport.

``If we don't take it tomowwow, he'll snatch it fwom under our
noses,'' he added.

While Denisov was talking to the esaul, Petya---abashed by
Denisov's cold tone and supposing that it was due to the
condition of his trousers---furtively tried to pull them down
under his greatcoat so that no one should notice it, while
maintaining as martial an air as possible.

``Will there be any orders, your honor?'' he asked Denisov,
holding his hand at the salute and resuming the game of adjutant
and general for which he had prepared himself, ``or shall I
remain with your honor?''

``Orders?'' Denisov repeated thoughtfully. ``But can you stay
till tomowwow?''

``Oh, please... May I stay with you?'' cried Petya.

``But, just what did the genewal tell you? To weturn at once?''
asked Denisov.

Petya blushed.

``He gave me no instructions. I think I could?'' he returned,
inquiringly.

``Well, all wight,'' said Denisov.

And turning to his men he directed a party to go on to the
halting place arranged near the watchman's hut in the forest, and
told the officer on the Kirghiz horse (who performed the duties
of an adjutant) to go and find out where Dolokhov was and whether
he would come that evening.  Denisov himself intended going with
the esaul and Petya to the edge of the forest where it reached
out to Shamshevo, to have a look at the part of the French
bivouac they were to attack next day.

``Well, old fellow,'' said he to the peasant guide, ``lead us to
Shamshevo.''

Denisov, Petya, and the esaul, accompanied by some Cossacks and
the hussar who had the prisoner, rode to the left across a ravine
to the edge of the forest.

% % % % % % % % % % % % % % % % % % % % % % % % % % % % % % % % %
% % % % % % % % % % % % % % % % % % % % % % % % % % % % % % % % %
% % % % % % % % % % % % % % % % % % % % % % % % % % % % % % % % %
% % % % % % % % % % % % % % % % % % % % % % % % % % % % % % % % %
% % % % % % % % % % % % % % % % % % % % % % % % % % % % % % % % %
% % % % % % % % % % % % % % % % % % % % % % % % % % % % % % % % %
% % % % % % % % % % % % % % % % % % % % % % % % % % % % % % % % %
% % % % % % % % % % % % % % % % % % % % % % % % % % % % % % % % %
% % % % % % % % % % % % % % % % % % % % % % % % % % % % % % % % %
% % % % % % % % % % % % % % % % % % % % % % % % % % % % % % % % %
% % % % % % % % % % % % % % % % % % % % % % % % % % % % % % % % %
% % % % % % % % % % % % % % % % % % % % % % % % % % % % % %

\chapter*{Chapter V} \ifaudio \marginpar{
\href{http://ia801404.us.archive.org/17/items/war_and_peace_14_1003_librivox/war_and_peace_14_05_tolstoy_64kb.mp3}{Audio}}
\fi

\initial{T}{he} rain had stopped, and only the mist was falling and drops
from the trees. Denisov, the esaul, and Petya rode silently,
following the peasant in the knitted cap who, stepping lightly
with outturned toes and moving noiselessly in his bast shoes over
the roots and wet leaves, silently led them to the edge of the
forest.

He ascended an incline, stopped, looked about him, and advanced
to where the screen of trees was less dense. On reaching a large
oak tree that had not yet shed its leaves, he stopped and
beckoned mysteriously to them with his hand.

Denisov and Petya rode up to him. From the spot where the peasant
was standing they could see the French. Immediately beyond the
forest, on a downward slope, lay a field of spring rye. To the
right, beyond a steep ravine, was a small village and a
landowner's house with a broken roof.  In the village, in the
house, in the garden, by the well, by the pond, over all the
rising ground, and all along the road uphill from the bridge
leading to the village, not more than five hundred yards away,
crowds of men could be seen through the shimmering mist. Their
un-Russian shouting at their horses which were straining uphill
with the carts, and their calls to one another, could be clearly
heard.

``Bwing the prisoner here,'' said Denisov in a low voice, not
taking his eyes off the French.

A Cossack dismounted, lifted the boy down, and took him to
Denisov.  Pointing to the French troops, Denisov asked him what
these and those of them were. The boy, thrusting his cold hands
into his pockets and lifting his eyebrows, looked at Denisov in
affright, but in spite of an evident desire to say all he knew
gave confused answers, merely assenting to everything Denisov
asked him. Denisov turned away from him frowning and addressed
the esaul, conveying his own conjectures to him.

Petya, rapidly turning his head, looked now at the drummer boy,
now at Denisov, now at the esaul, and now at the French in the
village and along the road, trying not to miss anything of
importance.

``Whether Dolokhov comes or not, we must seize it, eh?'' said
Denisov with a merry sparkle in his eyes.

``It is a very suitable spot,'' said the esaul.

``We'll send the infantwy down by the swamps,'' Denisov
continued.  ``They'll cweep up to the garden; you'll wide up fwom
there with the Cossacks''---he pointed to a spot in the forest
beyond the village---``and I with my hussars fwom here. And at
the signal shot...''

``The hollow is impassable---there's a swamp there,'' said the
esaul. ``The horses would sink. We must ride round more to the
left...''

While they were talking in undertones the crack of a shot sounded
from the low ground by the pond, a puff of white smoke appeared,
then another, and the sound of hundreds of seemingly merry French
voices shouting together came up from the slope. For a moment
Denisov and the esaul drew back. They were so near that they
thought they were the cause of the firing and shouting. But the
firing and shouting did not relate to them. Down below, a man
wearing something red was running through the marsh. The French
were evidently firing and shouting at him.

``Why, that's our Tikhon,'' said the esaul.

``So it is! It is!''

``The wascal!'' said Denisov.

``He'll get away!'' said the esaul, screwing up his eyes.

The man whom they called Tikhon, having run to the stream,
plunged in so that the water splashed in the air, and, having
disappeared for an instant, scrambled out on all fours, all black
with the wet, and ran on.  The French who had been pursuing him
stopped.

``Smart, that!'' said the esaul.

``What a beast!'' said Denisov with his former look of
vexation. ``What has he been doing all this time?''

``Who is he?'' asked Petya.

``He's our plastun. I sent him to capture a 'tongue.'{}''

``Oh, yes,'' said Petya, nodding at the first words Denisov
uttered as if he understood it all, though he really did not
understand anything of it.

Tikhon Shcherbaty was one of the most indispensable men in their
band.  He was a peasant from Pokrovsk, near the river Gzhat. When
Denisov had come to Pokrovsk at the beginning of his operations
and had as usual summoned the village elder and asked him what he
knew about the French, the elder, as though shielding himself,
had replied, as all village elders did, that he had neither seen
nor heard anything of them. But when Denisov explained that his
purpose was to kill the French, and asked if no French had
strayed that way, the elder replied that some
\emph{more-orderers} had really been at their village, but that
Tikhon Shcherbaty was the only man who dealt with such
matters. Denisov had Tikhon called and, having praised him for
his activity, said a few words in the elder's presence about
loyalty to the Tsar and the country and the hatred of the French
that all sons of the fatherland should cherish.

``We don't do the French any harm,'' said Tikhon, evidently
frightened by Denisov's words. ``We only fooled about with the
lads for fun, you know!  We killed a score or so of
'more-orderers,' but we did no harm else...''

Next day when Denisov had left Pokrovsk, having quite forgotten
about this peasant, it was reported to him that Tikhon had
attached himself to their party and asked to be allowed to remain
with it. Denisov gave orders to let him do so.

Tikhon, who at first did rough work, laying campfires, fetching
water, flaying dead horses, and so on, soon showed a great liking
and aptitude for partisan warfare. At night he would go out for
booty and always brought back French clothing and weapons, and
when told to would bring in French captives also. Denisov then
relieved him from drudgery and began taking him with him when he
went out on expeditions and had him enrolled among the Cossacks.

Tikhon did not like riding, and always went on foot, never
lagging behind the cavalry. He was armed with a musketoon (which
he carried rather as a joke), a pike and an ax, which latter he
used as a wolf uses its teeth, with equal ease picking fleas out
of its fur or crunching thick bones. Tikhon with equal accuracy
would split logs with blows at arm's length, or holding the head
of the ax would cut thin little pegs or carve spoons. In
Denisov's party he held a peculiar and exceptional position. When
anything particularly difficult or nasty had to be done---to push
a cart out of the mud with one's shoulders, pull a horse out of a
swamp by its tail, skin it, slink in among the French, or walk
more than thirty miles in a day---everybody pointed laughingly at
Tikhon.

``It won't hurt that devil---he's as strong as a horse!'' they
said of him.

Once a Frenchman Tikhon was trying to capture fired a pistol at
him and shot him in the fleshy part of the back. That wound
(which Tikhon treated only with internal and external
applications of vodka) was the subject of the liveliest jokes by
the whole detachment---jokes in which Tikhon readily joined.

``Hallo, mate! Never again? Gave you a twist?'' the Cossacks
would banter him. And Tikhon, purposely writhing and making
faces, pretended to be angry and swore at the French with the
funniest curses. The only effect of this incident on Tikhon was
that after being wounded he seldom brought in prisoners.

He was the bravest and most useful man in the party. No one found
more opportunities for attacking, no one captured or killed more
Frenchmen, and consequently he was made the buffoon of all the
Cossacks and hussars and willingly accepted that role. Now he had
been sent by Denisov overnight to Shamshevo to capture a
\emph{tongue}. But whether because he had not been content to
take only one Frenchman or because he had slept through the
night, he had crept by day into some bushes right among the
French and, as Denisov had witnessed from above, had been
detected by them.

% % % % % % % % % % % % % % % % % % % % % % % % % % % % % % % % %
% % % % % % % % % % % % % % % % % % % % % % % % % % % % % % % % %
% % % % % % % % % % % % % % % % % % % % % % % % % % % % % % % % %
% % % % % % % % % % % % % % % % % % % % % % % % % % % % % % % % %
% % % % % % % % % % % % % % % % % % % % % % % % % % % % % % % % %
% % % % % % % % % % % % % % % % % % % % % % % % % % % % % % % % %
% % % % % % % % % % % % % % % % % % % % % % % % % % % % % % % % %
% % % % % % % % % % % % % % % % % % % % % % % % % % % % % % % % %
% % % % % % % % % % % % % % % % % % % % % % % % % % % % % % % % %
% % % % % % % % % % % % % % % % % % % % % % % % % % % % % % % % %
% % % % % % % % % % % % % % % % % % % % % % % % % % % % % % % % %
% % % % % % % % % % % % % % % % % % % % % % % % % % % % % %

\chapter*{Chapter VI} \ifaudio \marginpar{
\href{http://ia801404.us.archive.org/17/items/war_and_peace_14_1003_librivox/war_and_peace_14_07_tolstoy_64kb.mp3}{Audio}}
\fi

\initial{A}{fter} talking for some time with the esaul about next day's
attack, which now, seeing how near they were to the French, he
seemed to have definitely decided on, Denisov turned his horse
and rode back.

``Now, my lad, we'll go and get dwy,'' he said to Petya.

As they approached the watchhouse Denisov stopped, peering into
the forest. Among the trees a man with long legs and long,
swinging arms, wearing a short jacket, bast shoes, and a Kazan
hat, was approaching with long, light steps. He had a musketoon
over his shoulder and an ax stuck in his girdle. When he espied
Denisov he hastily threw something into the bushes, removed his
sodden hat by its floppy brim, and approached his commander. It
was Tikhon. His wrinkled and pockmarked face and narrow little
eyes beamed with self-satisfied merriment. He lifted his head
high and gazed at Denisov as if repressing a laugh.

``Well, where did you disappear to?'' inquired Denisov.

``Where did I disappear to? I went to get Frenchmen,'' answered
Tikhon boldly and hurriedly, in a husky but melodious bass voice.

``Why did you push yourself in there by daylight? You ass! Well,
why haven't you taken one?''

``Oh, I took one all right,'' said Tikhon.

``Where is he?''

``You see, I took him first thing at dawn,'' Tikhon continued,
spreading out his flat feet with outturned toes in their bast
shoes. ``I took him into the forest. Then I see he's no good and
think I'll go and fetch a likelier one.''

``You see?... What a wogue---it's just as I thought,'' said
Denisov to the esaul. ``Why didn't you bwing that one?''

``What was the good of bringing him?'' Tikhon interrupted hastily
and angrily---``that one wouldn't have done for you. As if I
don't know what sort you want!''

``What a bwute you are!... Well?''

``I went for another one,'' Tikhon continued, ``and I crept like
this through the wood and lay down.'' (He suddenly lay down on
his stomach with a supple movement to show how he had done it.)
``One turned up and I grabbed him, like this.'' (He jumped up
quickly and lightly.) ``'Come along to the colonel,' I said. He
starts yelling, and suddenly there were four of them. They rushed
at me with their little swords. So I went for them with my ax,
this way: 'What are you up to?' says I. 'Christ be with you!'{}''
shouted Tikhon, waving his arms with an angry scowl and throwing
out his chest.

``Yes, we saw from the hill how you took to your heels through
the puddles!'' said the esaul, screwing up his glittering eyes.

Petya badly wanted to laugh, but noticed that they all refrained
from laughing. He turned his eyes rapidly from Tikhon's face to
the esaul's and Denisov's, unable to make out what it all meant.

``Don't play the fool!'' said Denisov, coughing angrily. ``Why
didn't you bwing the first one?''

Tikhon scratched his back with one hand and his head with the
other, then suddenly his whole face expanded into a beaming,
foolish grin, disclosing a gap where he had lost a tooth (that
was why he was called Shcherbaty---the gap-toothed). Denisov
smiled, and Petya burst into a peal of merry laughter in which
Tikhon himself joined.

``Oh, but he was a regular good-for-nothing,'' said Tikhon. ``The
clothes on him---poor stuff! How could I bring him? And so rude,
your honor! Why, he says: 'I'm a general's son myself, I won't
go!' he says.''

``You are a bwute!'' said Denisov. ``I wanted to question...''

``But I questioned him,'' said Tikhon. ``He said he didn't know
much.  'There are a lot of us,' he says, 'but all poor
stuff---only soldiers in name,' he says. 'Shout loud at them,' he
says, 'and you'll take them all,'{}'' Tikhon concluded, looking
cheerfully and resolutely into Denisov's eyes.

``I'll give you a hundwed sharp lashes---that'll teach you to
play the fool!'' said Denisov severely.

``But why are you angry?'' remonstrated Tikhon, ``just as if I'd
never seen your Frenchmen! Only wait till it gets dark and I'll
fetch you any of them you want---three if you like.''

``Well, let's go,'' said Denisov, and rode all the way to the
watchhouse in silence and frowning angrily.

Tikhon followed behind and Petya heard the Cossacks laughing with
him and at him, about some pair of boots he had thrown into the
bushes.

When the fit of laughter that had seized him at Tikhon's words
and smile had passed and Petya realized for a moment that this
Tikhon had killed a man, he felt uneasy. He looked round at the
captive drummer boy and felt a pang in his heart. But this
uneasiness lasted only a moment. He felt it necessary to hold his
head higher, to brace himself, and to question the esaul with an
air of importance about tomorrow's undertaking, that he might not
be unworthy of the company in which he found himself.

The officer who had been sent to inquire met Denisov on the way
with the news that Dolokhov was soon coming and that all was well
with him.

Denisov at once cheered up and, calling Petya to him, said:
``Well, tell me about yourself.''

% % % % % % % % % % % % % % % % % % % % % % % % % % % % % % % % %
% % % % % % % % % % % % % % % % % % % % % % % % % % % % % % % % %
% % % % % % % % % % % % % % % % % % % % % % % % % % % % % % % % %
% % % % % % % % % % % % % % % % % % % % % % % % % % % % % % % % %
% % % % % % % % % % % % % % % % % % % % % % % % % % % % % % % % %
% % % % % % % % % % % % % % % % % % % % % % % % % % % % % % % % %
% % % % % % % % % % % % % % % % % % % % % % % % % % % % % % % % %
% % % % % % % % % % % % % % % % % % % % % % % % % % % % % % % % %
% % % % % % % % % % % % % % % % % % % % % % % % % % % % % % % % %
% % % % % % % % % % % % % % % % % % % % % % % % % % % % % % % % %
% % % % % % % % % % % % % % % % % % % % % % % % % % % % % % % % %
% % % % % % % % % % % % % % % % % % % % % % % % % % % % % %

\chapter*{Chapter VII} \ifaudio \marginpar{
\href{http://ia801404.us.archive.org/17/items/war_and_peace_14_1003_librivox/war_and_peace_14_07_tolstoy_64kb.mp3}{Audio}}
\fi

\initial{P}{etya}, having left his people after their departure from Moscow,
joined his regiment and was soon taken as orderly by a general
commanding a large guerrilla detachment. From the time he
received his commission, and especially since he had joined the
active army and taken part in the battle of Vyazma, Petya had
been in a constant state of blissful excitement at being grown-up
and in a perpetual ecstatic hurry not to miss any chance to do
something really heroic. He was highly delighted with what he saw
and experienced in the army, but at the same time it always
seemed to him that the really heroic exploits were being
performed just where he did not happen to be. And he was always
in a hurry to get where he was not.

When on the twenty-first of October his general expressed a wish
to send somebody to Denisov's detachment, Petya begged so
piteously to be sent that the general could not refuse. But when
dispatching him he recalled Petya's mad action at the battle of
Vyazma, where instead of riding by the road to the place to which
he had been sent, he had galloped to the advanced line under the
fire of the French and had there twice fired his pistol. So now
the general explicitly forbade his taking part in any action
whatever of Denisov's. That was why Petya had blushed and grown
confused when Denisov asked him whether he could stay. Before
they had ridden to the outskirts of the forest Petya had
considered he must carry out his instructions strictly and return
at once. But when he saw the French and saw Tikhon and learned
that there would certainly be an attack that night, he decided,
with the rapidity with which young people change their views,
that the general, whom he had greatly respected till then, was a
rubbishy German, that Denisov was a hero, the esaul a hero, and
Tikhon a hero too, and that it would be shameful for him to leave
them at a moment of difficulty.

It was already growing dusk when Denisov, Petya, and the esaul
rode up to the watchhouse. In the twilight saddled horses could
be seen, and Cossacks and hussars who had rigged up rough
shelters in the glade and were kindling glowing fires in a hollow
of the forest where the French could not see the smoke. In the
passage of the small watchhouse a Cossack with sleeves rolled up
was chopping some mutton. In the room three officers of Denisov's
band were converting a door into a tabletop.  Petya took off his
wet clothes, gave them to be dried, and at once began helping the
officers to fix up the dinner table.

In ten minutes the table was ready and a napkin spread on it. On
the table were vodka, a flask of rum, white bread, roast mutton,
and salt.

Sitting at table with the officers and tearing the fat savory
mutton with his hands, down which the grease trickled, Petya was
in an ecstatic childish state of love for all men, and
consequently of confidence that others loved him in the same way.

``So then what do you think, Vasili Dmitrich?'' said he to
Denisov. ``It's all right my staying a day with you?'' And not
waiting for a reply he answered his own question: ``You see I was
told to find out---well, I am finding out... Only do let me into
the very... into the chief... I don't want a reward... But I
want...''

Petya clenched his teeth and looked around, throwing back his
head and flourishing his arms.

``Into the vewy chief...'' Denisov repeated with a smile.

``Only, please let me command something, so that I may really
command...''  Petya went on. ``What would it be to you?... Oh,
you want a knife?'' he said, turning to an officer who wished to
cut himself a piece of mutton.

And he handed him his clasp knife. The officer admired it.

``Please keep it. I have several like it,'' said Petya, blushing.
``Heavens! I was quite forgetting!'' he suddenly cried. ``I have
some raisins, fine ones; you know, seedless ones. We have a new
sutler and he has such capital things. I bought ten pounds. I am
used to something sweet. Would you like some?...'' and Petya ran
out into the passage to his Cossack and brought back some bags
which contained about five pounds of raisins. ``Have some,
gentlemen, have some!''

``You want a coffeepot, don't you?'' he asked the esaul. ``I
bought a capital one from our sutler! He has splendid things. And
he's very honest, that's the chief thing. I'll be sure to send it
to you. Or perhaps your flints are giving out, or are worn
out---that happens sometimes, you know. I have brought some with
me, here they are''---and he showed a bag---``a hundred flints. I
bought them very cheap. Please take as many as you want, or all
if you like...''

Then suddenly, dismayed lest he had said too much, Petya stopped
and blushed.

He tried to remember whether he had not done anything else that
was foolish. And running over the events of the day he remembered
the French drummer boy. ``It's capital for us here, but what of
him? Where have they put him? Have they fed him? Haven't they
hurt his feelings?'' he thought.  But having caught himself
saying too much about the flints, he was now afraid to speak out.

``I might ask,'' he thought, ``but they'll say: 'He's a boy
himself and so he pities the boy.' I'll show them tomorrow
whether I'm a boy. Will it seem odd if I ask?'' Petya
thought. ``Well, never mind!'' and immediately, blushing and
looking anxiously at the officers to see if they appeared
ironical, he said:

``May I call in that boy who was taken prisoner and give him
something to eat?... Perhaps...''

``Yes, he's a poor little fellow,'' said Denisov, who evidently
saw nothing shameful in this reminder. ``Call him in. His name is
Vincent Bosse. Have him fetched.''

``I'll call him,'' said Petya.

``Yes, yes, call him. A poor little fellow,'' Denisov repeated.

Petya was standing at the door when Denisov said this. He slipped
in between the officers, came close to Denisov, and said:

``Let me kiss you, dear old fellow! Oh, how fine, how splendid!''

And having kissed Denisov he ran out of the hut.

``Bosse! Vincent!'' Petya cried, stopping outside the door.

``Who do you want, sir?'' asked a voice in the darkness.

Petya replied that he wanted the French lad who had been captured
that day.

``Ah, Vesenny?'' said a Cossack.

Vincent, the boy's name, had already been changed by the Cossacks
into Vesenny (vernal) and into Vesenya by the peasants and
soldiers. In both these adaptations the reference to spring
(vesna) matched the impression made by the young lad.

``He is warming himself there by the bonfire. Ho, Vesenya!
Vesenya!---Vesenny!'' laughing voices were heard calling to one
another in the darkness.

``He's a smart lad,'' said an hussar standing near Petya. ``We
gave him something to eat a while ago. He was awfully hungry!''

The sound of bare feet splashing through the mud was heard in the
darkness, and the drummer boy came to the door.

``Ah, c'est vous!'' said Petya. ``Voulez-vous manger? N'ayez pas
peur, on ne vous fera pas de mal,''\footnote{``Ah, it's you! Do
you want something to eat? Don't be afraid, they won't hurt
you.''} he added shyly and affectionately, touching the boy's
hand. ``Entrez, entrez.''\footnote{``Come in, come in.''}

``Merci, monsieur,''\footnote{``Thank you, sir.''} said the
drummer boy in a trembling almost childish voice, and he began
scraping his dirty feet on the threshold.

There were many things Petya wanted to say to the drummer boy,
but did not dare to. He stood irresolutely beside him in the
passage. Then in the darkness he took the boy's hand and pressed
it.

``Come in, come in!'' he repeated in a gentle whisper. ``Oh, what
can I do for him?'' he thought, and opening the door he let the
boy pass in first.

When the boy had entered the hut, Petya sat down at a distance
from him, considering it beneath his dignity to pay attention to
him. But he fingered the money in his pocket and wondered whether
it would seem ridiculous to give some to the drummer boy.

% % % % % % % % % % % % % % % % % % % % % % % % % % % % % % % % %
% % % % % % % % % % % % % % % % % % % % % % % % % % % % % % % % %
% % % % % % % % % % % % % % % % % % % % % % % % % % % % % % % % %
% % % % % % % % % % % % % % % % % % % % % % % % % % % % % % % % %
% % % % % % % % % % % % % % % % % % % % % % % % % % % % % % % % %
% % % % % % % % % % % % % % % % % % % % % % % % % % % % % % % % %
% % % % % % % % % % % % % % % % % % % % % % % % % % % % % % % % %
% % % % % % % % % % % % % % % % % % % % % % % % % % % % % % % % %
% % % % % % % % % % % % % % % % % % % % % % % % % % % % % % % % %
% % % % % % % % % % % % % % % % % % % % % % % % % % % % % % % % %
% % % % % % % % % % % % % % % % % % % % % % % % % % % % % % % % %
% % % % % % % % % % % % % % % % % % % % % % % % % % % % % %

\chapter*{Chapter VIII} \ifaudio \marginpar{
\href{http://ia801404.us.archive.org/17/items/war_and_peace_14_1003_librivox/war_and_peace_14_08_tolstoy_64kb.mp3}{Audio}}
\fi

\initial{T}{he} arrival of Dolokhov diverted Petya's attention from the
drummer boy, to whom Denisov had had some mutton and vodka given,
and whom he had had dressed in a Russian coat so that he might be
kept with their band and not sent away with the other
prisoners. Petya had heard in the army many stories of Dolokhov's
extraordinary bravery and of his cruelty to the French, so from
the moment he entered the hut Petya did not take his eyes from
him, but braced himself up more and more and held his head high,
that he might not be unworthy even of such company.

Dolokhov's appearance amazed Petya by its simplicity.

Denisov wore a Cossack coat, had a beard, had an icon of Nicholas
the Wonder-Worker on his breast, and his way of speaking and
everything he did indicated his unusual position. But Dolokhov,
who in Moscow had worn a Persian costume, had now the appearance
of a most correct officer of the Guards. He was clean-shaven and
wore a Guardsman's padded coat with an Order of St. George at his
buttonhole and a plain forage cap set straight on his head. He
took off his wet felt cloak in a corner of the room, and without
greeting anyone went up to Denisov and began questioning him
about the matter in hand. Denisov told him of the designs the
large detachments had on the transport, of the message Petya had
brought, and his own replies to both generals. Then he told him
all he knew of the French detachment.

``That's so. But we must know what troops they are and their
numbers,'' said Dolokhov. ``It will be necessary to go there. We
can't start the affair without knowing for certain how many there
are. I like to work accurately. Here now---wouldn't one of these
gentlemen like to ride over to the French camp with me? I have
brought a spare uniform.''

``I, I... I'll go with you!'' cried Petya.

``There's no need for you to go at all,'' said Denisov,
addressing Dolokhov, ``and as for him, I won't let him go on any
account.''

``I like that!'' exclaimed Petya. ``Why shouldn't I go?''

``Because it's useless.''

``Well, you must excuse me, because... because... I shall go, and
that's all. You'll take me, won't you?'' he said, turning to
Dolokhov.

``Why not?'' Dolokhov answered absently, scrutinizing the face of
the French drummer boy. ``Have you had that youngster with you
long?'' he asked Denisov.

``He was taken today but he knows nothing. I'm keeping him with
me.''

``Yes, and where do you put the others?'' inquired Dolokhov.

``Where? I send them away and take a weceipt for them,'' shouted
Denisov, suddenly flushing. ``And I say boldly that I have not a
single man's life on my conscience. Would it be difficult for you
to send thirty or thwee hundwed men to town under escort, instead
of staining---I speak bluntly---staining the honor of a
soldier?''

``That kind of amiable talk would be suitable from this young
count of sixteen,'' said Dolokhov with cold irony, ``but it's
time for you to drop it.''

``Why, I've not said anything! I only say that I'll certainly go
with you,'' said Petya shyly.

``But for you and me, old fellow, it's time to drop these
amenities,'' continued Dolokhov, as if he found particular
pleasure in speaking of this subject which irritated
Denisov. ``Now, why have you kept this lad?''  he went on,
swaying his head. ``Because you are sorry for him! Don't we know
those 'receipts' of yours? You send a hundred men away, and
thirty get there. The rest either starve or get killed. So isn't
it all the same not to send them?''

The esaul, screwing up his light-colored eyes, nodded
approvingly.

``That's not the point. I'm not going to discuss the matter. I do
not wish to take it on my conscience. You say they'll die. All
wight. Only not by my fault!''

Dolokhov began laughing.

``Who has told them not to capture me these twenty times over?
But if they did catch me they'd string me up to an aspen tree,
and with all your chivalry just the same.'' He paused. ``However,
we must get to work.  Tell the Cossack to fetch my kit. I have
two French uniforms in it.  Well, are you coming with me?'' he
asked Petya.

``I? Yes, yes, certainly!'' cried Petya, blushing almost to tears
and glancing at Denisov.

While Dolokhov had been disputing with Denisov what should be
done with prisoners, Petya had once more felt awkward and
restless; but again he had no time to grasp fully what they were
talking about. ``If grown-up, distinguished men think so, it must
be necessary and right,'' thought he.  ``But above all Denisov
must not dare to imagine that I'll obey him and that he can order
me about. I will certainly go to the French camp with
Dolokhov. If he can, so can I!''

And to all Denisov's persuasions, Petya replied that he too was
accustomed to do everything accurately and not just anyhow, and
that he never considered personal danger.

``For you'll admit that if we don't know for sure how many of
them there are... hundreds of lives may depend on it, while there
are only two of us. Besides, I want to go very much and certainly
will go, so don't hinder me,'' said he. ``It will only make
things worse...''

% % % % % % % % % % % % % % % % % % % % % % % % % % % % % % % % %
% % % % % % % % % % % % % % % % % % % % % % % % % % % % % % % % %
% % % % % % % % % % % % % % % % % % % % % % % % % % % % % % % % %
% % % % % % % % % % % % % % % % % % % % % % % % % % % % % % % % %
% % % % % % % % % % % % % % % % % % % % % % % % % % % % % % % % %
% % % % % % % % % % % % % % % % % % % % % % % % % % % % % % % % %
% % % % % % % % % % % % % % % % % % % % % % % % % % % % % % % % %
% % % % % % % % % % % % % % % % % % % % % % % % % % % % % % % % %
% % % % % % % % % % % % % % % % % % % % % % % % % % % % % % % % %
% % % % % % % % % % % % % % % % % % % % % % % % % % % % % % % % %
% % % % % % % % % % % % % % % % % % % % % % % % % % % % % % % % %
% % % % % % % % % % % % % % % % % % % % % % % % % % % % % %

\chapter*{Chapter IX} \ifaudio \marginpar{
\href{http://ia801404.us.archive.org/17/items/war_and_peace_14_1003_librivox/war_and_peace_14_09_tolstoy_64kb.mp3}{Audio}}
\fi

\initial{H}{aving} put on French greatcoats and shakos, Petya and Dolokhov
rode to the clearing from which Denisov had reconnoitered the
French camp, and emerging from the forest in pitch darkness they
descended into the hollow. On reaching the bottom, Dolokhov told
the Cossacks accompanying him to await him there and rode on at a
quick trot along the road to the bridge. Petya, his heart in his
mouth with excitement, rode by his side.

``If we're caught, I won't be taken alive! I have a pistol,''
whispered he.

``Don't talk Russian,'' said Dolokhov in a hurried whisper, and
at that very moment they heard through the darkness the
challenge: ``Qui vive?''\footnote{``Who goes there?''} and the
click of a musket.

The blood rushed to Petya's face and he grasped his pistol.

``Lanciers du 6-me,''\footnote{``Lancers of the 6th Regiment.''}
replied Dolokhov, neither hastening nor slackening his horse's
pace.

The black figure of a sentinel stood on the bridge.

``Mot d'ordre.''\footnote{``Password.''}

Dolokhov reined in his horse and advanced at a walk.

``Dites donc, le colonel Gerard est ici?''\footnote{``Tell me, is
  Colonel Gerard here?''} he asked.

``Mot d'ordre,'' repeated the sentinel, barring the way and not
replying.

``Quand un officier fait sa ronde, les sentinelles ne demandent
pas le mot d'ordre...'' cried Dolokhov suddenly flaring up and
riding straight at the sentinel. ``Je vous demande si le colonel
est ici.''\footnote{``When an officer is making his round,
sentinels don't ask him for the password... I am asking you if
the colonel is here.''}

And without waiting for an answer from the sentinel, who had
stepped aside, Dolokhov rode up the incline at a walk.

Noticing the black outline of a man crossing the road, Dolokhov
stopped him and inquired where the commander and officers
were. The man, a soldier with a sack over his shoulder, stopped,
came close up to Dolokhov's horse, touched it with his hand, and
explained simply and in a friendly way that the commander and the
officers were higher up the hill to the right in the courtyard of
the farm, as he called the landowner's house.

Having ridden up the road, on both sides of which French talk
could be heard around the campfires, Dolokhov turned into the
courtyard of the landowner's house. Having ridden in, he
dismounted and approached a big blazing campfire, around which
sat several men talking noisily.  Something was boiling in a
small cauldron at the edge of the fire and a soldier in a peaked
cap and blue overcoat, lit up by the fire, was kneeling beside it
stirring its contents with a ramrod.

``Oh, he's a hard nut to crack,'' said one of the officers who
was sitting in the shadow at the other side of the fire.

``He'll make them get a move on, those fellows!'' said another,
laughing.

Both fell silent, peering out through the darkness at the sound
of Dolokhov's and Petya's steps as they advanced to the fire
leading their horses.

``Bonjour, messieurs!''\footnote{``Good day, gentlemen.''} said
Dolokhov loudly and clearly.

There was a stir among the officers in the shadow beyond the
fire, and one tall, long-necked officer, walking round the fire,
came up to Dolokhov.

``Is that you, Clement?'' he asked. ``Where the devil...?'' But,
noticing his mistake, he broke off short and, with a frown,
greeted Dolokhov as a stranger, asking what he could do for him.

Dolokhov said that he and his companion were trying to overtake
their regiment, and addressing the company in general asked
whether they knew anything of the 6th Regiment. None of them knew
anything, and Petya thought the officers were beginning to look
at him and Dolokhov with hostility and suspicion. For some
seconds all were silent.

``If you were counting on the evening soup, you have come too
late,'' said a voice from behind the fire with a repressed laugh.

Dolokhov replied that they were not hungry and must push on
farther that night.

He handed the horses over to the soldier who was stirring the pot
and squatted down on his heels by the fire beside the officer
with the long neck. That officer did not take his eyes from
Dolokhov and again asked to what regiment he belonged. Dolokhov,
as if he had not heard the question, did not reply, but lighting
a short French pipe which he took from his pocket began asking
the officer in how far the road before them was safe from
Cossacks.

``Those brigands are everywhere,'' replied an officer from behind
the fire.

Dolokhov remarked that the Cossacks were a danger only to
stragglers such as his companion and himself, ``but probably they
would not dare to attack large detachments?'' he added
inquiringly. No one replied.

``Well, now he'll come away,'' Petya thought every moment as he
stood by the campfire listening to the talk.

But Dolokhov restarted the conversation which had dropped and
began putting direct questions as to how many men there were in
the battalion, how many battalions, and how many
prisoners. Asking about the Russian prisoners with that
detachment, Dolokhov said:

``A horrid business dragging these corpses about with one! It
would be better to shoot such rabble,'' and burst into loud
laughter, so strange that Petya thought the French would
immediately detect their disguise, and involuntarily took a step
back from the campfire.

No one replied a word to Dolokhov's laughter, and a French
officer whom they could not see (he lay wrapped in a greatcoat)
rose and whispered something to a companion. Dolokhov got up and
called to the soldier who was holding their horses.

``Will they bring our horses or not?'' thought Petya,
instinctively drawing nearer to Dolokhov.

The horses were brought.

``Good evening, gentlemen,'' said Dolokhov.

Petya wished to say ``Good night'' but could not utter a
word. The officers were whispering together. Dolokhov was a long
time mounting his horse which would not stand still, then he rode
out of the yard at a footpace. Petya rode beside him, longing to
look round to see whether or not the French were running after
them, but not daring to.

Coming out onto the road Dolokhov did not ride back across the
open country, but through the village. At one spot he stopped and
listened.  ``Do you hear?'' he asked. Petya recognized the sound
of Russian voices and saw the dark figures of Russian prisoners
round their campfires.  When they had descended to the bridge
Petya and Dolokhov rode past the sentinel, who without saying a
word paced morosely up and down it, then they descended into the
hollow where the Cossacks awaited them.

``Well now, good-by. Tell Denisov, \emph{at the first shot at
daybreak},'' said Dolokhov and was about to ride away, but Petya
seized hold of him.

``Really!'' he cried, ``you are such a hero! Oh, how fine, how
splendid!  How I love you!''

``All right, all right!'' said Dolokhov. But Petya did not let go
of him and Dolokhov saw through the gloom that Petya was bending
toward him and wanted to kiss him. Dolokhov kissed him, laughed,
turned his horse, and vanished into the darkness.

% % % % % % % % % % % % % % % % % % % % % % % % % % % % % % % % %
% % % % % % % % % % % % % % % % % % % % % % % % % % % % % % % % %
% % % % % % % % % % % % % % % % % % % % % % % % % % % % % % % % %
% % % % % % % % % % % % % % % % % % % % % % % % % % % % % % % % %
% % % % % % % % % % % % % % % % % % % % % % % % % % % % % % % % %
% % % % % % % % % % % % % % % % % % % % % % % % % % % % % % % % %
% % % % % % % % % % % % % % % % % % % % % % % % % % % % % % % % %
% % % % % % % % % % % % % % % % % % % % % % % % % % % % % % % % %
% % % % % % % % % % % % % % % % % % % % % % % % % % % % % % % % %
% % % % % % % % % % % % % % % % % % % % % % % % % % % % % % % % %
% % % % % % % % % % % % % % % % % % % % % % % % % % % % % % % % %
% % % % % % % % % % % % % % % % % % % % % % % % % % % % % %

\chapter*{Chapter X} \ifaudio \marginpar{
\href{http://ia801404.us.archive.org/17/items/war_and_peace_14_1003_librivox/war_and_peace_14_10_tolstoy_64kb.mp3}{Audio}}
\fi

\initial{H}{aving} returned to the watchman's hut, Petya found Denisov in the
passage. He was awaiting Petya's return in a state of agitation,
anxiety, and self-reproach for having let him go.

``Thank God!'' he exclaimed. ``Yes, thank God!'' he repeated,
listening to Petya's rapturous account. ``But, devil take you, I
haven't slept because of you! Well, thank God. Now lie down. We
can still get a nap before morning.''

``But... no,'' said Petya, ``I don't want to sleep yet. Besides I
know myself, if I fall asleep it's finished. And then I am used
to not sleeping before a battle.''

He sat awhile in the hut joyfully recalling the details of his
expedition and vividly picturing to himself what would happen
next day.

Then, noticing that Denisov was asleep, he rose and went out of
doors.

It was still quite dark outside. The rain was over, but drops
were still falling from the trees. Near the watchman's hut the
black shapes of the Cossacks' shanties and of horses tethered
together could be seen. Behind the hut the dark shapes of the two
wagons with their horses beside them were discernible, and in the
hollow the dying campfire gleamed red. Not all the Cossacks and
hussars were asleep; here and there, amid the sounds of falling
drops and the munching of the horses near by, could be heard low
voices which seemed to be whispering.

Petya came out, peered into the darkness, and went up to the
wagons.  Someone was snoring under them, and around them stood
saddled horses munching their oats. In the dark Petya recognized
his own horse, which he called ``Karabakh'' though it was of
Ukranian breed, and went up to it.

``Well, Karabakh! We'll do some service tomorrow,'' said he,
sniffing its nostrils and kissing it.

``Why aren't you asleep, sir?'' said a Cossack who was sitting
under a wagon.

``No, ah... Likhachev---isn't that your name? Do you know I have
only just come back! We've been into the French camp.''

And Petya gave the Cossack a detailed account not only of his
ride but also of his object, and why he considered it better to
risk his life than to act ``just anyhow.''

``Well, you should get some sleep now,'' said the Cossack.

``No, I am used to this,'' said Petya. ``I say, aren't the flints
in your pistols worn out? I brought some with me. Don't you want
any? You can have some.''

The Cossack bent forward from under the wagon to get a closer
look at Petya.

``Because I am accustomed to doing everything accurately,'' said
Petya.  ``Some fellows do things just anyhow, without
preparation, and then they're sorry for it afterwards. I don't
like that.''

``Just so,'' said the Cossack.

``Oh yes, another thing! Please, my dear fellow, will you sharpen
my saber for me? It's got bl...'' (Petya feared to tell a lie,
and the saber never had been sharpened.) ``Can you do it?''

``Of course I can.''

Likhachev got up, rummaged in his pack, and soon Petya heard the
warlike sound of steel on whetstone. He climbed onto the wagon
and sat on its edge. The Cossack was sharpening the saber under
the wagon.

``I say! Are the lads asleep?'' asked Petya.

``Some are, and some aren't---like us.''

``Well, and that boy?''

``Vesenny? Oh, he's thrown himself down there in the
passage. Fast asleep after his fright. He was that glad!''

After that Petya remained silent for a long time, listening to
the sounds. He heard footsteps in the darkness and a black figure
appeared.

``What are you sharpening?'' asked a man coming up to the wagon.

``Why, this gentleman's saber.''

``That's right,'' said the man, whom Petya took to be an
hussar. ``Was the cup left here?''

``There, by the wheel!''

The hussar took the cup.

``It must be daylight soon,'' said he, yawning, and went away.

Petya ought to have known that he was in a forest with Denisov's
guerrilla band, less than a mile from the road, sitting on a
wagon captured from the French beside which horses were tethered,
that under it Likhachev was sitting sharpening a saber for him,
that the big dark blotch to the right was the watchman's hut, and
the red blotch below to the left was the dying embers of a
campfire, that the man who had come for the cup was an hussar who
wanted a drink; but he neither knew nor waited to know anything
of all this. He was in a fairy kingdom where nothing resembled
reality. The big dark blotch might really be the watchman's hut
or it might be a cavern leading to the very depths of the
earth. Perhaps the red spot was a fire, or it might be the eye of
an enormous monster. Perhaps he was really sitting on a wagon,
but it might very well be that he was not sitting on a wagon but
on a terribly high tower from which, if he fell, he would have to
fall for a whole day or a whole month, or go on falling and never
reach the bottom. Perhaps it was just the Cossack, Likhachev, who
was sitting under the wagon, but it might be the kindest,
bravest, most wonderful, most splendid man in the world, whom no
one knew of. It might really have been that the hussar came for
water and went back into the hollow, but perhaps he had simply
vanished---disappeared altogether and dissolved into nothingness.

Nothing Petya could have seen now would have surprised him. He
was in a fairy kingdom where everything was possible.

He looked up at the sky. And the sky was a fairy realm like the
earth.  It was clearing, and over the tops of the trees clouds
were swiftly sailing as if unveiling the stars. Sometimes it
looked as if the clouds were passing, and a clear black sky
appeared. Sometimes it seemed as if the black spaces were
clouds. Sometimes the sky seemed to be rising high, high
overhead, and then it seemed to sink so low that one could touch
it with one's hand.

Petya's eyes began to close and he swayed a little.

The trees were dripping. Quiet talking was heard. The horses
neighed and jostled one another. Someone snored.

``Ozheg-zheg, Ozheg-zheg...'' hissed the saber against the
whetstone, and suddenly Petya heard an harmonious orchestra
playing some unknown, sweetly solemn hymn. Petya was as musical
as Natasha and more so than Nicholas, but had never learned music
or thought about it, and so the melody that unexpectedly came to
his mind seemed to him particularly fresh and attractive. The
music became more and more audible. The melody grew and passed
from one instrument to another. And what was played was a
fugue---though Petya had not the least conception of what a fugue
is.  Each instrument---now resembling a violin and now a horn,
but better and clearer than violin or horn---played its own part,
and before it had finished the melody merged with another
instrument that began almost the same air, and then with a third
and a fourth; and they all blended into one and again became
separate and again blended, now into solemn church music, now
into something dazzlingly brilliant and triumphant.

``Oh---why, that was in a dream!'' Petya said to himself, as he
lurched forward. ``It's in my ears. But perhaps it's music of my
own. Well, go on, my music! Now!...''

He closed his eyes, and, from all sides as if from a distance,
sounds fluttered, grew into harmonies, separated, blended, and
again all mingled into the same sweet and solemn hymn. ``Oh, this
is delightful! As much as I like and as I like!'' said Petya to
himself. He tried to conduct that enormous orchestra.

``Now softly, softly die away!'' and the sounds obeyed him. ``Now
fuller, more joyful. Still more and more joyful!'' And from an
unknown depth rose increasingly triumphant sounds. ``Now voices
join in!'' ordered Petya. And at first from afar he heard men's
voices and then women's. The voices grew in harmonious triumphant
strength, and Petya listened to their surpassing beauty in awe
and joy.

With a solemn triumphal march there mingled a song, the drip from
the trees, and the hissing of the saber, ``Ozheg-zheg-zheg...''
and again the horses jostled one another and neighed, not
disturbing the choir but joining in it.

Petya did not know how long this lasted: he enjoyed himself all
the time, wondered at his enjoyment and regretted that there was
no one to share it. He was awakened by Likhachev's kindly voice.

``It's ready, your honor; you can split a Frenchman in half with
it!''

Petya woke up.

``It's getting light, it's really getting light!'' he exclaimed.

The horses that had previously been invisible could now be seen
to their very tails, and a watery light showed itself through the
bare branches.  Petya shook himself, jumped up, took a ruble from
his pocket and gave it to Likhachev; then he flourished the
saber, tested it, and sheathed it.  The Cossacks were untying
their horses and tightening their saddle girths.

``And here's the commander,'' said Likhachev.

Denisov came out of the watchman's hut and, having called Petya,
gave orders to get ready.

% % % % % % % % % % % % % % % % % % % % % % % % % % % % % % % % %
% % % % % % % % % % % % % % % % % % % % % % % % % % % % % % % % %
% % % % % % % % % % % % % % % % % % % % % % % % % % % % % % % % %
% % % % % % % % % % % % % % % % % % % % % % % % % % % % % % % % %
% % % % % % % % % % % % % % % % % % % % % % % % % % % % % % % % %
% % % % % % % % % % % % % % % % % % % % % % % % % % % % % % % % %
% % % % % % % % % % % % % % % % % % % % % % % % % % % % % % % % %
% % % % % % % % % % % % % % % % % % % % % % % % % % % % % % % % %
% % % % % % % % % % % % % % % % % % % % % % % % % % % % % % % % %
% % % % % % % % % % % % % % % % % % % % % % % % % % % % % % % % %
% % % % % % % % % % % % % % % % % % % % % % % % % % % % % % % % %
% % % % % % % % % % % % % % % % % % % % % % % % % % % % % %

\chapter*{Chapter XI} \ifaudio \marginpar{
\href{http://ia801404.us.archive.org/17/items/war_and_peace_14_1003_librivox/war_and_peace_14_11_tolstoy_64kb.mp3}{Audio}}
\fi

\initial{T}{he} men rapidly picked out their horses in the semidarkness,
tightened their saddle girths, and formed companies. Denisov
stood by the watchman's hut giving final orders. The infantry of
the detachment passed along the road and quickly disappeared amid
the trees in the mist of early dawn, hundreds of feet splashing
through the mud. The esaul gave some orders to his men. Petya
held his horse by the bridle, impatiently awaiting the order to
mount. His face, having been bathed in cold water, was all aglow,
and his eyes were particularly brilliant.  Cold shivers ran down
his spine and his whole body pulsed rhythmically.

``Well, is ev'wything weady?'' asked Denisov. ``Bwing the
horses.''

The horses were brought. Denisov was angry with the Cossack
because the saddle girths were too slack, reproved him, and
mounted. Petya put his foot in the stirrup. His horse by habit
made as if to nip his leg, but Petya leaped quickly into the
saddle unconscious of his own weight and, turning to look at the
hussars starting in the darkness behind him, rode up to Denisov.

``Vasili Dmitrich, entrust me with some commission! Please... for
God's sake...!'' said he.

Denisov seemed to have forgotten Petya's very existence. He
turned to glance at him.

``I ask one thing of you,'' he said sternly, ``to obey me and not
shove yourself forward anywhere.''

He did not say another word to Petya but rode in silence all the
way.  When they had come to the edge of the forest it was
noticeably growing light over the field. Denisov talked in
whispers with the esaul and the Cossacks rode past Petya and
Denisov. When they had all ridden by, Denisov touched his horse
and rode down the hill. Slipping onto their haunches and sliding,
the horses descended with their riders into the ravine. Petya
rode beside Denisov, the pulsation of his body constantly
increasing. It was getting lighter and lighter, but the mist
still hid distant objects. Having reached the valley, Denisov
looked back and nodded to a Cossack beside him.

``The signal!'' said he.

The Cossack raised his arm and a shot rang out. In an instant the
tramp of horses galloping forward was heard, shouts came from
various sides, and then more shots.

At the first sound of trampling hoofs and shouting, Petya lashed
his horse and loosening his rein galloped forward, not heeding
Denisov who shouted at him. It seemed to Petya that at the moment
the shot was fired it suddenly became as bright as noon. He
galloped to the bridge.  Cossacks were galloping along the road
in front of him. On the bridge he collided with a Cossack who had
fallen behind, but he galloped on. In front of him soldiers,
probably Frenchmen, were running from right to left across the
road. One of them fell in the mud under his horse's feet.

Cossacks were crowding about a hut, busy with something. From the
midst of that crowd terrible screams arose. Petya galloped up,
and the first thing he saw was the pale face and trembling jaw of
a Frenchman, clutching the handle of a lance that had been aimed
at him.

``Hurrah!... Lads!... ours!'' shouted Petya, and giving rein to
his excited horse he galloped forward along the village street.

He could hear shooting ahead of him. Cossacks, hussars, and
ragged Russian prisoners, who had come running from both sides of
the road, were shouting something loudly and incoherently. A
gallant-looking Frenchman, in a blue overcoat, capless, and with
a frowning red face, had been defending himself against the
hussars. When Petya galloped up the Frenchman had already
fallen. ``Too late again!'' flashed through Petya's mind and he
galloped on to the place from which the rapid firing could be
heard. The shots came from the yard of the landowner's house he
had visited the night before with Dolokhov. The French were
making a stand there behind a wattle fence in a garden thickly
overgrown with bushes and were firing at the Cossacks who crowded
at the gateway.  Through the smoke, as he approached the gate,
Petya saw Dolokhov, whose face was of a pale-greenish tint,
shouting to his men. ``Go round! Wait for the infantry!'' he
exclaimed as Petya rode up to him.

``Wait?... Hurrah-ah-ah!'' shouted Petya, and without pausing a
moment galloped to the place whence came the sounds of firing and
where the smoke was thickest.

A volley was heard, and some bullets whistled past, while others
plashed against something. The Cossacks and Dolokhov galloped
after Petya into the gateway of the courtyard. In the dense
wavering smoke some of the French threw down their arms and ran
out of the bushes to meet the Cossacks, while others ran down the
hill toward the pond. Petya was galloping along the courtyard,
but instead of holding the reins he waved both his arms about
rapidly and strangely, slipping farther and farther to one side
in his saddle. His horse, having galloped up to a campfire that
was smoldering in the morning light, stopped suddenly, and Petya
fell heavily on to the wet ground. The Cossacks saw that his arms
and legs jerked rapidly though his head was quite motionless. A
bullet had pierced his skull.

After speaking to the senior French officer, who came out of the
house with a white handkerchief tied to his sword and announced
that they surrendered, Dolokhov dismounted and went up to Petya,
who lay motionless with outstretched arms.

``Done for!'' he said with a frown, and went to the gate to meet
Denisov who was riding toward him.

``Killed?'' cried Denisov, recognizing from a distance the
unmistakably lifeless attitude---very familiar to him---in which
Petya's body was lying.

``Done for!'' repeated Dolokhov as if the utterance of these
words afforded him pleasure, and he went quickly up to the
prisoners, who were surrounded by Cossacks who had hurried
up. ``We won't take them!'' he called out to Denisov.

Denisov did not reply; he rode up to Petya, dismounted, and with
trembling hands turned toward himself the bloodstained,
mud-be\-spat\-tered face which had already gone white.

``I am used to something sweet. Raisins, fine ones... take them
all!'' he recalled Petya's words. And the Cossacks looked round
in surprise at the sound, like the yelp of a dog, with which
Denisov turned away, walked to the wattle fence, and seized hold
of it.

Among the Russian prisoners rescued by Denisov and Dolokhov was
Pierre Bezukhov.

% % % % % % % % % % % % % % % % % % % % % % % % % % % % % % % % %
% % % % % % % % % % % % % % % % % % % % % % % % % % % % % % % % %
% % % % % % % % % % % % % % % % % % % % % % % % % % % % % % % % %
% % % % % % % % % % % % % % % % % % % % % % % % % % % % % % % % %
% % % % % % % % % % % % % % % % % % % % % % % % % % % % % % % % %
% % % % % % % % % % % % % % % % % % % % % % % % % % % % % % % % %
% % % % % % % % % % % % % % % % % % % % % % % % % % % % % % % % %
% % % % % % % % % % % % % % % % % % % % % % % % % % % % % % % % %
% % % % % % % % % % % % % % % % % % % % % % % % % % % % % % % % %
% % % % % % % % % % % % % % % % % % % % % % % % % % % % % % % % %
% % % % % % % % % % % % % % % % % % % % % % % % % % % % % % % % %
% % % % % % % % % % % % % % % % % % % % % % % % % % % % % %

\chapter*{Chapter XII} \ifaudio \marginpar{
\href{http://ia801404.us.archive.org/17/items/war_and_peace_14_1003_librivox/war_and_peace_14_12_tolstoy_64kb.mp3}{Audio}}
\fi

\initial{D}{uring} the whole of their march from Moscow no fresh orders had
been issued by the French authorities concerning the party of
prisoners among whom was Pierre. On the twenty-second of October
that party was no longer with the same troops and baggage trains
with which it had left Moscow. Half the wagons laden with
hardtack that had traveled the first stages with them had been
captured by Cossacks, the other half had gone on ahead. Not one
of those dismounted cavalrymen who had marched in front of the
prisoners was left; they had all disappeared. The artillery the
prisoners had seen in front of them during the first days was now
replaced by Marshal Junot's enormous baggage train, convoyed by
Westphalians. Behind the prisoners came a cavalry baggage train.

From Vyazma onwards the French army, which had till then moved in
three columns, went on as a single group. The symptoms of
disorder that Pierre had noticed at their first halting place
after leaving Moscow had now reached the utmost limit.

The road along which they moved was bordered on both sides by
dead horses; ragged men who had fallen behind from various
regiments continually changed about, now joining the moving
column, now again lagging behind it.

Several times during the march false alarms had been given and
the soldiers of the escort had raised their muskets, fired, and
run headlong, crushing one another, but had afterwards
reassembled and abused each other for their causeless panic.

These three groups traveling together---the cavalry stores, the
convoy of prisoners, and Junot's baggage train---still
constituted a separate and united whole, though each of the
groups was rapidly melting away.

Of the artillery baggage train which had consisted of a hundred
and twenty wagons, not more than sixty now remained; the rest had
been captured or left behind. Some of Junot's wagons also had
been captured or abandoned. Three wagons had been raided and
robbed by stragglers from Davout's corps. From the talk of the
Germans Pierre learned that a larger guard had been allotted to
that baggage train than to the prisoners, and that one of their
comrades, a German soldier, had been shot by the marshal's own
order because a silver spoon belonging to the marshal had been
found in his possession.

The group of prisoners had melted away most of all. Of the three
hundred and thirty men who had set out from Moscow fewer than a
hundred now remained. The prisoners were more burdensome to the
escort than even the cavalry saddles or Junot's baggage. They
understood that the saddles and Junot's spoon might be of some
use, but that cold and hungry soldiers should have to stand and
guard equally cold and hungry Russians who froze and lagged
behind on the road (in which case the order was to shoot them)
was not merely incomprehensible but revolting. And the escort, as
if afraid, in the grievous condition they themselves were in, of
giving way to the pity they felt for the prisoners and so
rendering their own plight still worse, treated them with
particular moroseness and severity.

At Dorogobuzh while the soldiers of the convoy, after locking the
prisoners in a stable, had gone off to pillage their own stores,
several of the soldier prisoners tunneled under the wall and ran
away, but were recaptured by the French and shot.

The arrangement adopted when they started, that the officer
prisoners should be kept separate from the rest, had long since
been abandoned.  All who could walk went together, and after the
third stage Pierre had rejoined Karataev and the gray-blue
bandy-legged dog that had chosen Karataev for its master.

On the third day after leaving Moscow Karataev again fell ill
with the fever he had suffered from in the hospital in Moscow,
and as he grew gradually weaker Pierre kept away from him. Pierre
did not know why, but since Karataev had begun to grow weaker it
had cost him an effort to go near him. When he did so and heard
the subdued moaning with which Karataev generally lay down at the
halting places, and when he smelled the odor emanating from him
which was now stronger than before, Pierre moved farther away and
did not think about him.

While imprisoned in the shed Pierre had learned not with his
intellect but with his whole being, by life itself, that man is
created for happiness, that happiness is within him, in the
satisfaction of simple human needs, and that all unhappiness
arises not from privation but from superfluity. And now during
these last three weeks of the march he had learned still another
new, consolatory truth---that nothing in this world is
terrible. He had learned that as there is no condition in which
man can be happy and entirely free, so there is no condition in
which he need be unhappy and lack freedom. He learned that
suffering and freedom have their limits and that those limits are
very near together; that the person in a bed of roses with one
crumpled petal suffered as keenly as he now, sleeping on the bare
damp earth with one side growing chilled while the other was
warming; and that when he had put on tight dancing shoes he had
suffered just as he did now when he walked with bare feet that
were covered with sores---his footgear having long since fallen
to pieces. He discovered that when he had married his wife---of
his own free will as it had seemed to him---he had been no more
free than now when they locked him up at night in a stable. Of
all that he himself subsequently termed his sufferings, but which
at the time he scarcely felt, the worst was the state of his
bare, raw, and scab-covered feet.  (The horseflesh was appetizing
and nourishing, the saltpeter flavor of the gunpowder they used
instead of salt was even pleasant; there was no great cold, it
was always warm walking in the daytime, and at night there were
the campfires; the lice that devoured him warmed his body.)  The
one thing that was at first hard to bear was his feet.

After the second day's march Pierre, having examined his feet by
the campfire, thought it would be impossible to walk on them; but
when everybody got up he went along, limping, and, when he had
warmed up, walked without feeling the pain, though at night his
feet were more terrible to look at than before. However, he did
not look at them now, but thought of other things.

Only now did Pierre realize the full strength of life in man and
the saving power he has of transferring his attention from one
thing to another, which is like the safety valve of a boiler that
allows superfluous steam to blow off when the pressure exceeds a
certain limit.

He did not see and did not hear how they shot the prisoners who
lagged behind, though more than a hundred perished in that
way. He did not think of Karataev who grew weaker every day and
evidently would soon have to share that fate. Still less did
Pierre think about himself. The harder his position became and
the more terrible the future, the more independent of that
position in which he found himself were the joyful and comforting
thoughts, memories, and imaginings that came to him.

% % % % % % % % % % % % % % % % % % % % % % % % % % % % % % % % %
% % % % % % % % % % % % % % % % % % % % % % % % % % % % % % % % %
% % % % % % % % % % % % % % % % % % % % % % % % % % % % % % % % %
% % % % % % % % % % % % % % % % % % % % % % % % % % % % % % % % %
% % % % % % % % % % % % % % % % % % % % % % % % % % % % % % % % %
% % % % % % % % % % % % % % % % % % % % % % % % % % % % % % % % %
% % % % % % % % % % % % % % % % % % % % % % % % % % % % % % % % %
% % % % % % % % % % % % % % % % % % % % % % % % % % % % % % % % %
% % % % % % % % % % % % % % % % % % % % % % % % % % % % % % % % %
% % % % % % % % % % % % % % % % % % % % % % % % % % % % % % % % %
% % % % % % % % % % % % % % % % % % % % % % % % % % % % % % % % %
% % % % % % % % % % % % % % % % % % % % % % % % % % % % % %

\chapter*{Chapter XIII} \ifaudio \marginpar{
\href{http://ia801404.us.archive.org/17/items/war_and_peace_14_1003_librivox/war_and_peace_14_13_tolstoy_64kb.mp3}{Audio}}
\fi

\initial{A}{t} midday on the twenty-second of October Pierre was going uphill
along the muddy, slippery road, looking at his feet and at the
roughness of the way. Occasionally he glanced at the familiar
crowd around him and then again at his feet. The former and the
latter were alike familiar and his own. The blue-gray bandy
legged dog ran merrily along the side of the road, sometimes in
proof of its agility and self-satisfaction lifting one hind leg
and hopping along on three, and then again going on all four and
rushing to bark at the crows that sat on the carrion. The dog was
merrier and sleeker than it had been in Moscow. All around lay
the flesh of different animals---from men to horses---in various
stages of decomposition; and as the wolves were kept off by the
passing men the dog could eat all it wanted.

It had been raining since morning and had seemed as if at any
moment it might cease and the sky clear, but after a short break
it began raining harder than before. The saturated road no longer
absorbed the water, which ran along the ruts in streams.

Pierre walked along, looking from side to side, counting his
steps in threes, and reckoning them off on his fingers. Mentally
addressing the rain, he repeated: ``Now then, now then, go on!
Pelt harder!''

It seemed to him that he was thinking of nothing, but far down
and deep within him his soul was occupied with something
important and comforting. This something was a most subtle
spiritual deduction from a conversation with Karataev the day
before.

At their yesterday's halting place, feeling chilly by a dying
campfire, Pierre had got up and gone to the next one, which was
burning better.  There Platon Karataev was sitting covered
up---head and all---with his greatcoat as if it were a vestment,
telling the soldiers in his effective and pleasant though now
feeble voice a story Pierre knew. It was already past midnight,
the hour when Karataev was usually free of his fever and
particularly lively. When Pierre reached the fire and heard
Platon's voice enfeebled by illness, and saw his pathetic face
brightly lit up by the blaze, he felt a painful prick at his
heart. His feeling of pity for this man frightened him and he
wished to go away, but there was no other fire, and Pierre sat
down, trying not to look at Platon.

``Well, how are you?'' he asked.

``How am I? If we grumble at sickness, God won't grant us
death,'' replied Platon, and at once resumed the story he had
begun.

``And so, brother,'' he continued, with a smile on his pale
emaciated face and a particularly happy light in his eyes, ``you
see, brother...''

Pierre had long been familiar with that story. Karataev had told
it to him alone some half-dozen times and always with a specially
joyful emotion. But well as he knew it, Pierre now listened to
that tale as to something new, and the quiet rapture Karataev
evidently felt as he told it communicated itself also to
Pierre. The story was of an old merchant who lived a good and
God-fearing life with his family, and who went once to the Nizhni
fair with a companion---a rich merchant.

Having put up at an inn they both went to sleep, and next morning
his companion was found robbed and with his throat cut. A
bloodstained knife was found under the old merchant's pillow. He
was tried, knouted, and his nostrils having been torn off, ``all
in due form'' as Karataev put it, he was sent to hard labor in
Siberia.

``And so, brother'' (it was at this point that Pierre came up),
``ten years or more passed by. The old man was living as a
convict, submitting as he should and doing no wrong. Only he
prayed to God for death. Well, one night the convicts were
gathered just as we are, with the old man among them. And they
began telling what each was suffering for, and how they had
sinned against God. One told how he had taken a life, another had
taken two, a third had set a house on fire, while another had
simply been a vagrant and had done nothing. So they asked the old
man: 'What are you being punished for, Daddy?'---'I, my dear
brothers,' said he, 'am being punished for my own and other men's
sins. But I have not killed anyone or taken anything that was not
mine, but have only helped my poorer brothers. I was a merchant,
my dear brothers, and had much property. 'And he went on to tell
them all about it in due order. 'I don't grieve for myself,' he
says, 'God, it seems, has chastened me.  Only I am sorry for my
old wife and the children,' and the old man began to weep. Now it
happened that in the group was the very man who had killed the
other merchant. 'Where did it happen, Daddy?' he said. 'When, and
in what month?' He asked all about it and his heart began to
ache.  So he comes up to the old man like this, and falls down at
his feet!  'You are perishing because of me, Daddy,' he
says. 'It's quite true, lads, that this man,' he says, 'is being
tortured innocently and for nothing! I,' he says, 'did that deed,
and I put the knife under your head while you were
asleep. Forgive me, Daddy,' he says, 'for Christ's sake!'{}''

Karataev paused, smiling joyously as he gazed into the fire, and
he drew the logs together.

``And the old man said, 'God will forgive you, we are all sinners
in His sight. I suffer for my own sins,' and he wept bitter
tears. Well, and what do you think, dear friends?'' Karataev
continued, his face brightening more and more with a rapturous
smile as if what he now had to tell contained the chief charm and
the whole meaning of his story: ``What do you think, dear
fellows? That murderer confessed to the authorities. 'I have
taken six lives,' he says (he was a great sinner), 'but what I am
most sorry for is this old man. Don't let him suffer because of
me.' So he confessed and it was all written down and the papers
sent off in due form. The place was a long way off, and while
they were judging, what with one thing and another, filling in
the papers all in due form---the authorities I mean---time
passed. The affair reached the Tsar. After a while the Tsar's
decree came: to set the merchant free and give him a compensation
that had been awarded. The paper arrived and they began to look
for the old man. 'Where is the old man who has been suffering
innocently and in vain? A paper has come from the Tsar!' so they
began looking for him,'' here Karataev's lower jaw trembled,
``but God had already forgiven him---he was dead! That's how it
was, dear fellows!'' Karataev concluded and sat for a long time
silent, gazing before him with a smile.

And Pierre's soul was dimly but joyfully filled not by the story
itself but by its mysterious significance: by the rapturous joy
that lit up Karataev's face as he told it, and the mystic
significance of that joy.

% % % % % % % % % % % % % % % % % % % % % % % % % % % % % % % % %
% % % % % % % % % % % % % % % % % % % % % % % % % % % % % % % % %
% % % % % % % % % % % % % % % % % % % % % % % % % % % % % % % % %
% % % % % % % % % % % % % % % % % % % % % % % % % % % % % % % % %
% % % % % % % % % % % % % % % % % % % % % % % % % % % % % % % % %
% % % % % % % % % % % % % % % % % % % % % % % % % % % % % % % % %
% % % % % % % % % % % % % % % % % % % % % % % % % % % % % % % % %
% % % % % % % % % % % % % % % % % % % % % % % % % % % % % % % % %
% % % % % % % % % % % % % % % % % % % % % % % % % % % % % % % % %
% % % % % % % % % % % % % % % % % % % % % % % % % % % % % % % % %
% % % % % % % % % % % % % % % % % % % % % % % % % % % % % % % % %
% % % % % % % % % % % % % % % % % % % % % % % % % % % % % %

\chapter*{Chapter XIV} \ifaudio \marginpar{
\href{http://ia801404.us.archive.org/17/items/war_and_peace_14_1003_librivox/war_and_peace_14_14_tolstoy_64kb.mp3}{Audio}}
\fi

\initial*{A}{} vos places!''\footnote{``To your places.''} suddenly cried a
voice.

A pleasant feeling of excitement and an expectation of something
joyful and solemn was aroused among the soldiers of the convoy
and the prisoners. From all sides came shouts of command, and
from the left came smartly dressed cavalrymen on good horses,
passing the prisoners at a trot. The expression on all faces
showed the tension people feel at the approach of those in
authority. The prisoners thronged together and were pushed off
the road. The convoy formed up.

``The Emperor! The Emperor! The Marshal! The Duke!'' and hardly
had the sleek cavalry passed, before a carriage drawn by six gray
horses rattled by. Pierre caught a glimpse of a man in a
three-cornered hat with a tranquil look on his handsome, plump,
white face. It was one of the marshals. His eye fell on Pierre's
large and striking figure, and in the expression with which he
frowned and looked away Pierre thought he detected sympathy and a
desire to conceal that sympathy.

The general in charge of the stores galloped after the carriage
with a red and frightened face, whipping up his skinny
horse. Several officers formed a group and some soldiers crowded
round them. Their faces all looked excited and worried.

``What did he say? What did he say?'' Pierre heard them ask.

While the marshal was passing, the prisoners had huddled together
in a crowd, and Pierre saw Karataev whom he had not yet seen that
morning. He sat in his short overcoat leaning against a birch
tree. On his face, besides the look of joyful emotion it had worn
yesterday while telling the tale of the merchant who suffered
innocently, there was now an expression of quiet solemnity.

Karataev looked at Pierre with his kindly round eyes now filled
with tears, evidently wishing him to come near that he might say
something to him. But Pierre was not sufficiently sure of
himself. He made as if he did not notice that look and moved
hastily away.

When the prisoners again went forward Pierre looked
round. Karataev was still sitting at the side of the road under
the birch tree and two Frenchmen were talking over his
head. Pierre did not look round again but went limping up the
hill.

From behind, where Karataev had been sitting, came the sound of a
shot.  Pierre heard it plainly, but at that moment he remembered
that he had not yet finished reckoning up how many stages still
remained to Smolensk---a calculation he had begun before the
marshal went by. And he again started reckoning. Two French
soldiers ran past Pierre, one of whom carried a lowered and
smoking gun. They both looked pale, and in the expression on
their faces---one of them glanced timidly at Pierre---there was
something resembling what he had seen on the face of the young
soldier at the execution. Pierre looked at the soldier and
remembered that, two days before, that man had burned his shirt
while drying it at the fire and how they had laughed at him.

Behind him, where Karataev had been sitting, the dog began to
howl.  ``What a stupid beast! Why is it howling?'' thought
Pierre.

His comrades, the prisoner soldiers walking beside him, avoided
looking back at the place where the shot had been fired and the
dog was howling, just as Pierre did, but there was a set look on
all their faces.

% % % % % % % % % % % % % % % % % % % % % % % % % % % % % % % % %
% % % % % % % % % % % % % % % % % % % % % % % % % % % % % % % % %
% % % % % % % % % % % % % % % % % % % % % % % % % % % % % % % % %
% % % % % % % % % % % % % % % % % % % % % % % % % % % % % % % % %
% % % % % % % % % % % % % % % % % % % % % % % % % % % % % % % % %
% % % % % % % % % % % % % % % % % % % % % % % % % % % % % % % % %
% % % % % % % % % % % % % % % % % % % % % % % % % % % % % % % % %
% % % % % % % % % % % % % % % % % % % % % % % % % % % % % % % % %
% % % % % % % % % % % % % % % % % % % % % % % % % % % % % % % % %
% % % % % % % % % % % % % % % % % % % % % % % % % % % % % % % % %
% % % % % % % % % % % % % % % % % % % % % % % % % % % % % % % % %
% % % % % % % % % % % % % % % % % % % % % % % % % % % % % %

\chapter*{Chapter XV} \ifaudio \marginpar{
\href{http://ia801404.us.archive.org/17/items/war_and_peace_14_1003_librivox/war_and_peace_14_15_tolstoy_64kb.mp3}{Audio}}
\fi

\initial{T}{he} stores, the prisoners, and the marshal's baggage train
stopped at the village of Shamshevo. The men crowded together
round the campfires.  Pierre went up to the fire, ate some roast
horseflesh, lay down with his back to the fire, and immediately
fell asleep. He again slept as he had done at Mozhaysk after the
battle of Borodino.

Again real events mingled with dreams and again someone, he or
another, gave expression to his thoughts, and even to the same
thoughts that had been expressed in his dream at Mozhaysk.

``Life is everything. Life is God. Everything changes and moves
and that movement is God. And while there is life there is joy in
consciousness of the divine. To love life is to love God. Harder
and more blessed than all else is to love this life in one's
sufferings, in innocent sufferings.''

``Karataev!'' came to Pierre's mind.

And suddenly he saw vividly before him a long-forgotten, kindly
old man who had given him geography lessons in
Switzerland. ``Wait a bit,'' said the old man, and showed Pierre
a globe. This globe was alive---a vibrating ball without fixed
dimensions. Its whole surface consisted of drops closely pressed
together, and all these drops moved and changed places, sometimes
several of them merging into one, sometimes one dividing into
many. Each drop tried to spread out and occupy as much space as
possible, but others striving to do the same compressed it,
sometimes destroyed it, and sometimes merged with it.

``That is life,'' said the old teacher.

``How simple and clear it is,'' thought Pierre. ``How is it I did
not know it before?''

``God is in the midst, and each drop tries to expand so as to
reflect Him to the greatest extent. And it grows, merges,
disappears from the surface, sinks to the depths, and again
emerges. There now, Karataev has spread out and disappeared. Do
you understand, my child?'' said the teacher.

``Do you understand, damn you?'' shouted a voice, and Pierre woke
up.

He lifted himself and sat up. A Frenchman who had just pushed a
Russian soldier away was squatting by the fire, engaged in
roasting a piece of meat stuck on a ramrod. His sleeves were
rolled up and his sinewy, hairy, red hands with their short
fingers deftly turned the ramrod. His brown morose face with
frowning brows was clearly visible by the glow of the charcoal.

``It's all the same to him,'' he muttered, turning quickly to a
soldier who stood behind him. ``Brigand! Get away!''

And twisting the ramrod he looked gloomily at Pierre, who turned
away and gazed into the darkness. A prisoner, the Russian soldier
the Frenchman had pushed away, was sitting near the fire patting
something with his hand. Looking more closely Pierre recognized
the blue-gray dog, sitting beside the soldier, wagging its tail.

``Ah, he's come?'' said Pierre. ``And Plat-'' he began, but did
not finish.

Suddenly and simultaneously a crowd of memories awoke in his
fancy---of the look Platon had given him as he sat under the
tree, of the shot heard from that spot, of the dog's howl, of the
guilty faces of the two Frenchmen as they ran past him, of the
lowered and smoking gun, and of Karataev's absence at this
halt---and he was on the point of realizing that Karataev had
been killed, but just at that instant, he knew not why, the
recollection came to his mind of a summer evening he had spent
with a beautiful Polish lady on the veranda of his house in
Kiev. And without linking up the events of the day or drawing a
conclusion from them, Pierre closed his eyes, seeing a vision of
the country in summertime mingled with memories of bathing and of
the liquid, vibrating globe, and he sank into water so that it
closed over his head.

Before sunrise he was awakened by shouts and loud and rapid
firing.  French soldiers were running past him.

``The Cossacks!'' one of them shouted, and a moment later a crowd
of Russians surrounded Pierre.

For a long time he could not understand what was happening to
him. All around he heard his comrades sobbing with joy.

``Brothers! Dear fellows! Darlings!'' old soldiers exclaimed,
weeping, as they embraced Cossacks and hussars.

The hussars and Cossacks crowded round the prisoners; one offered
them clothes, another boots, and a third bread. Pierre sobbed as
he sat among them and could not utter a word. He hugged the first
soldier who approached him, and kissed him, weeping.

Dolokhov stood at the gate of the ruined house, letting a crowd
of disarmed Frenchmen pass by. The French, excited by all that
had happened, were talking loudly among themselves, but as they
passed Dolokhov who gently switched his boots with his whip and
watched them with cold glassy eyes that boded no good, they
became silent. On the opposite side stood Dolokhov's Cossack,
counting the prisoners and marking off each hundred with a chalk
line on the gate.

``How many?'' Dolokhov asked the Cossack.

``The second hundred,'' replied the Cossack.

``Filez, filez!''\footnote{``Get along, get along!''} Dolokhov
kept saying, having adopted this expression from the French, and
when his eyes met those of the prisoners they flashed with a
cruel light.

Denisov, bareheaded and with a gloomy face, walked behind some
Cossacks who were carrying the body of Petya Rostov to a hole
that had been dug in the garden.

% % % % % % % % % % % % % % % % % % % % % % % % % % % % % % % % %
% % % % % % % % % % % % % % % % % % % % % % % % % % % % % % % % %
% % % % % % % % % % % % % % % % % % % % % % % % % % % % % % % % %
% % % % % % % % % % % % % % % % % % % % % % % % % % % % % % % % %
% % % % % % % % % % % % % % % % % % % % % % % % % % % % % % % % %
% % % % % % % % % % % % % % % % % % % % % % % % % % % % % % % % %
% % % % % % % % % % % % % % % % % % % % % % % % % % % % % % % % %
% % % % % % % % % % % % % % % % % % % % % % % % % % % % % % % % %
% % % % % % % % % % % % % % % % % % % % % % % % % % % % % % % % %
% % % % % % % % % % % % % % % % % % % % % % % % % % % % % % % % %
% % % % % % % % % % % % % % % % % % % % % % % % % % % % % % % % %
% % % % % % % % % % % % % % % % % % % % % % % % % % % % % %

\chapter*{Chapter XVI} \ifaudio \marginpar{
\href{http://ia801404.us.archive.org/17/items/war_and_peace_14_1003_librivox/war_and_peace_14_16_tolstoy_64kb.mp3}{Audio}}
\fi

\initial{A}{fter} the twenty-eighth of October when the frosts began, the
flight of the French assumed a still more tragic character, with
men freezing, or roasting themselves to death at the campfires,
while carriages with people dressed in furs continued to drive
past, carrying away the property that had been stolen by the
Emperor, kings, and dukes; but the process of the flight and
disintegration of the French army went on essentially as before.

From Moscow to Vyazma the French army of seventy-three thousand
men not reckoning the Guards (who did nothing during the whole
war but pillage) was reduced to thirty-six thousand, though not
more than five thousand had fallen in battle. From this beginning
the succeeding terms of the progression could be determined
mathematically. The French army melted away and perished at the
same rate from Moscow to Vyazma, from Vyazma to Smolensk, from
Smolensk to the Berezina, and from the Berezina to
Vilna---independently of the greater or lesser intensity of the
cold, the pursuit, the barring of the way, or any other
particular conditions.  Beyond Vyazma the French army instead of
moving in three columns huddled together into one mass, and so
went on to the end. Berthier wrote to his Emperor (we know how
far commanding officers allow themselves to diverge from the
truth in describing the condition of an army) and this is what he
said: \begin{quote}\calli I deem it my duty to report to Your
Majesty the condition of the various corps I have had occasion to
observe during different stages of the last two or three days'
march. They are almost disbanded. Scarcely a quarter of the
soldiers remain with the standards of their regiments, the others
go off by themselves in different directions hoping to find food
and escape discipline. In general they regard Smolensk as the
place where they hope to recover. During the last few days many
of the men have been seen to throw away their cartridges and
their arms. In such a state of affairs, whatever your ultimate
plans may be, the interest of Your Majesty's service demands that
the army should be rallied at Smolensk and should first of all be
freed from ineffectives, such as dismounted cavalry, unnecessary
baggage, and artillery material that is no longer in proportion
to the present forces. The soldiers, who are worn out with hunger
and fatigue, need these supplies as well as a few days' rest.
Many have died these last days on the road or at the
bivouacs. This state of things is continually becoming worse and
makes one fear that unless a prompt remedy is applied the troops
will no longer be under control in case of an engagement.

  November 9th, twenty miles from Smolensk.  \end{quote}

After staggering into Smolensk which seemed to them a promised
land, the French, searching for food, killed one another, sacked
their own stores, and when everything had been plundered fled
farther.

They all went without knowing whither or why they were
going. Still less did that genius, Napoleon, know it, for no one
issued any orders to him.  But still he and those about him
retained their old habits: wrote commands, letters, reports, and
orders of the day; called one another sire, mon cousin, prince
d'Eckmuhl, roi de Naples, and so on. But these orders and reports
were only on paper, nothing in them was acted upon for they could
not be carried out, and though they entitled one another
Majesties, Highnesses, or Cousins, they all felt that they were
miserable wretches who had done much evil for which they had now
to pay.  And though they pretended to be concerned about the
army, each was thinking only of himself and of how to get away
quickly and save himself.

% % % % % % % % % % % % % % % % % % % % % % % % % % % % % % % % %
% % % % % % % % % % % % % % % % % % % % % % % % % % % % % % % % %
% % % % % % % % % % % % % % % % % % % % % % % % % % % % % % % % %
% % % % % % % % % % % % % % % % % % % % % % % % % % % % % % % % %
% % % % % % % % % % % % % % % % % % % % % % % % % % % % % % % % %
% % % % % % % % % % % % % % % % % % % % % % % % % % % % % % % % %
% % % % % % % % % % % % % % % % % % % % % % % % % % % % % % % % %
% % % % % % % % % % % % % % % % % % % % % % % % % % % % % % % % %
% % % % % % % % % % % % % % % % % % % % % % % % % % % % % % % % %
% % % % % % % % % % % % % % % % % % % % % % % % % % % % % % % % %
% % % % % % % % % % % % % % % % % % % % % % % % % % % % % % % % %
% % % % % % % % % % % % % % % % % % % % % % % % % % % % % %

\chapter*{Chapter XVII} \ifaudio \marginpar{
\href{http://ia801404.us.archive.org/17/items/war_and_peace_14_1003_librivox/war_and_peace_14_17_tolstoy_64kb.mp3}{Audio}}
\fi

\initial{T}{he} movements of the Russian and French armies during the
campaign from Moscow back to the Niemen were like those in a game
of Russian blindman's bluff, in which two players are blindfolded
and one of them occasionally rings a little bell to inform the
catcher of his whereabouts. First he rings his bell fearlessly,
but when he gets into a tight place he runs away as quietly as he
can, and often thinking to escape runs straight into his
opponent's arms.

At first while they were still moving along the Kaluga road,
Napoleon's armies made their presence known, but later when they
reached the Smolensk road they ran holding the clapper of their
bell tight---and often thinking they were escaping ran right into
the Russians.

Owing to the rapidity of the French flight and the Russian
pursuit and the consequent exhaustion of the horses, the chief
means of approximately ascertaining the enemy's position---by
cavalry scouting---was not available. Besides, as a result of the
frequent and rapid change of position by each army, even what
information was obtained could not be delivered in time. If news
was received one day that the enemy had been in a certain
position the day before, by the third day when something could
have been done, that army was already two days' march farther on
and in quite another position.

One army fled and the other pursued. Beyond Smolensk there were
several different roads available for the French, and one would
have thought that during their stay of four days they might have
learned where the enemy was, might have arranged some more
advantageous plan and undertaken something new. But after a four
days' halt the mob, with no maneuvers or plans, again began
running along the beaten track, neither to the right nor to the
left but along the old---the worst---road, through Krasnoe and
Orsha.

Expecting the enemy from behind and not in front, the French
separated in their flight and spread out over a distance of
twenty-four hours. In front of them all fled the Emperor, then
the kings, then the dukes. The Russian army, expecting Napoleon
to take the road to the right beyond the Dnieper---which was the
only reasonable thing for him to do---themselves turned to the
right and came out onto the highroad at Krasnoe. And here as in a
game of blindman's buff the French ran into our vanguard. Seeing
their enemy unexpectedly the French fell into confusion and
stopped short from the sudden fright, but then they resumed their
flight, abandoning their comrades who were farther behind.  Then
for three days separate portions of the French army---first
Murat's (the vice-king's), then Davout's, and then Ney's---ran,
as it were, the gauntlet of the Russian army. They abandoned one
another, abandoned all their heavy baggage, their artillery, and
half their men, and fled, getting past the Russians by night by
making semicircles to the right.

Ney, who came last, had been busying himself blowing up the walls
of Smolensk which were in nobody's way, because despite the
unfortunate plight of the French or because of it, they wished to
punish the floor against which they had hurt themselves. Ney, who
had had a corps of ten thousand men, reached Napoleon at Orsha
with only one thousand men left, having abandoned all the rest
and all his cannon, and having crossed the Dnieper at night by
stealth at a wooded spot.

From Orsha they fled farther along the road to Vilna, still
playing at blindman's buff with the pursuing army. At the
Berezina they again became disorganized, many were drowned and
many surrendered, but those who got across the river fled
farther. Their supreme chief donned a fur coat and, having seated
himself in a sleigh, galloped on alone, abandoning his
companions. The others who could do so drove away too, leaving
those who could not to surrender or die.

% % % % % % % % % % % % % % % % % % % % % % % % % % % % % % % % %
% % % % % % % % % % % % % % % % % % % % % % % % % % % % % % % % %
% % % % % % % % % % % % % % % % % % % % % % % % % % % % % % % % %
% % % % % % % % % % % % % % % % % % % % % % % % % % % % % % % % %
% % % % % % % % % % % % % % % % % % % % % % % % % % % % % % % % %
% % % % % % % % % % % % % % % % % % % % % % % % % % % % % % % % %
% % % % % % % % % % % % % % % % % % % % % % % % % % % % % % % % %
% % % % % % % % % % % % % % % % % % % % % % % % % % % % % % % % %
% % % % % % % % % % % % % % % % % % % % % % % % % % % % % % % % %
% % % % % % % % % % % % % % % % % % % % % % % % % % % % % % % % %
% % % % % % % % % % % % % % % % % % % % % % % % % % % % % % % % %
% % % % % % % % % % % % % % % % % % % % % % % % % % % % % %

\chapter*{Chapter XVIII} \ifaudio \marginpar{
\href{http://ia801404.us.archive.org/17/items/war_and_peace_14_1003_librivox/war_and_peace_14_18_tolstoy_64kb.mp3}{Audio}}
\fi

\initial{T}{his} campaign consisted in a flight of the French during which
they did all they could to destroy themselves. From the time they
turned onto the Kaluga road to the day their leader fled from the
army, none of the movements of the crowd had any sense. So one
might have thought that regarding this period of the campaign the
historians, who attributed the actions of the mass to the will of
one man, would have found it impossible to make the story of the
retreat fit their theory. But no!  Mountains of books have been
written by the historians about this campaign, and everywhere are
described Napoleon's arrangements, the maneuvers, and his
profound plans which guided the army, as well as the military
genius shown by his marshals.

The retreat from Malo-Yaroslavets when he had a free road into a
well-supplied district and the parallel road was open to him
along which Kutuzov afterwards pursued him---this unnecessary
retreat along a devastated road---is explained to us as being due
to profound considerations. Similarly profound considerations are
given for his retreat from Smolensk to Orsha. Then his heroism at
Krasnoe is described, where he is reported to have been prepared
to accept battle and take personal command, and to have walked
about with a birch stick and said:

``J'ai assez fait l'empereur; il est temps de faire le
general,''\footnote{``I have acted the Emperor long enough; it is
time to act the general.''} but nevertheless immediately ran away
again, abandoning to its fate the scattered fragments of the army
he left behind.

Then we are told of the greatness of soul of the marshals,
especially of Ney---a greatness of soul consisting in this: that
he made his way by night around through the forest and across the
Dnieper and escaped to Orsha, abandoning standards, artillery,
and nine tenths of his men.

And lastly, the final departure of the great Emperor from his
heroic army is presented to us by the historians as something
great and characteristic of genius. Even that final running away,
described in ordinary language as the lowest depth of baseness
which every child is taught to be ashamed of---even that act
finds justification in the historians' language.

When it is impossible to stretch the very elastic threads of
historical ratiocination any farther, when actions are clearly
contrary to all that humanity calls right or even just, the
historians produce a saving conception of \emph{greatness}.
\emph{Greatness}, it seems, excludes the standards of right and
wrong. For the \emph{great} man nothing is wrong, there is no
atrocity for which a \emph{great} man can be blamed.

``C'est grand!''\footnote{``It is great.''} say the historians,
and there no longer exists either good or evil but only
\emph{grand} and \emph{not grand}. Grand is good, not grand is
bad. Grand is the characteristic, in their conception, of some
special animals called \emph{heroes}. And Napoleon, escaping home
in a warm fur coat and leaving to perish those who were not
merely his comrades but were (in his opinion) men he had brought
there, feels que c'est grand,\footnote{That it is great.} and his
soul is tranquil.

``Du sublime (he saw something sublime in himself) au ridicule il
n'y a qu'un pas,''\footnote{``From the sublime to the ridiculous
is but a step.''} said he. And the whole world for fifty years
has been repeating: ``Sublime! Grand! Napoleon le Grand!'' Du
sublime au ridicule il n'y a qu'un pas.

And it occurs to no one that to admit a greatness not
commensurable with the standard of right and wrong is merely to
admit one's own nothingness and immeasurable meanness.

For us with the standard of good and evil given us by Christ, no
human actions are incommensurable. And there is no greatness
where simplicity, goodness, and truth are absent.

% % % % % % % % % % % % % % % % % % % % % % % % % % % % % % % % %
% % % % % % % % % % % % % % % % % % % % % % % % % % % % % % % % %
% % % % % % % % % % % % % % % % % % % % % % % % % % % % % % % % %
% % % % % % % % % % % % % % % % % % % % % % % % % % % % % % % % %
% % % % % % % % % % % % % % % % % % % % % % % % % % % % % % % % %
% % % % % % % % % % % % % % % % % % % % % % % % % % % % % % % % %
% % % % % % % % % % % % % % % % % % % % % % % % % % % % % % % % %
% % % % % % % % % % % % % % % % % % % % % % % % % % % % % % % % %
% % % % % % % % % % % % % % % % % % % % % % % % % % % % % % % % %
% % % % % % % % % % % % % % % % % % % % % % % % % % % % % % % % %
% % % % % % % % % % % % % % % % % % % % % % % % % % % % % % % % %
% % % % % % % % % % % % % % % % % % % % % % % % % % % % % %

\chapter*{Chapter XIX} \ifaudio \marginpar{
\href{http://ia801404.us.archive.org/17/items/war_and_peace_14_1003_librivox/war_and_peace_14_19_tolstoy_64kb.mp3}{Audio}}
\fi

\initial{W}{hat} Russian, reading the account of the last part of the
campaign of 1812, has not experienced an uncomfortable feeling of
regret, dissatisfaction, and perplexity? Who has not asked
himself how it is that the French were not all captured or
destroyed when our three armies surrounded them in superior
numbers, when the disordered French, hungry and freezing,
surrendered in crowds, and when (as the historians relate) the
aim of the Russians was to stop the French, to cut them off, and
capture them all?

How was it that the Russian army, which when numerically weaker
than the French had given battle at Borodino, did not achieve its
purpose when it had surrounded the French on three sides and when
its aim was to capture them? Can the French be so enormously
superior to us that when we had surrounded them with superior
forces we could not beat them? How could that happen?

History (or what is called by that name) replying to these
questions says that this occurred because Kutuzov and Tormasov
and Chichagov, and this man and that man, did not execute such
and such maneuvers...

But why did they not execute those maneuvers? And why if they
were guilty of not carrying out a prearranged plan were they not
tried and punished? But even if we admitted that Kutuzov,
Chichagov, and others were the cause of the Russian failures, it
is still incomprehensible why, the position of the Russian army
being what it was at Krasnoe and at the Berezina (in both cases
we had superior forces), the French army with its marshals,
kings, and Emperor was not captured, if that was what the
Russians aimed at.

The explanation of this strange fact given by Russian military
historians (to the effect that Kutuzov hindered an attack) is
unfounded, for we know that he could not restrain the troops from
attacking at Vyazma and Tarutino.

Why was the Russian army---which with inferior forces had
withstood the enemy in full strength at Borodino---defeated at
Krasnoe and the Berezina by the disorganized crowds of the French
when it was numerically superior?

If the aim of the Russians consisted in cutting off and capturing
Napoleon and his marshals---and that aim was not merely
frustrated but all attempts to attain it were most shamefully
baffled---then this last period of the campaign is quite rightly
considered by the French to be a series of victories, and quite
wrongly considered victorious by Russian historians.

The Russian military historians in so far as they submit to
claims of logic must admit that conclusion, and in spite of their
lyrical rhapsodies about valor, devotion, and so forth, must
reluctantly admit that the French retreat from Moscow was a
series of victories for Napoleon and defeats for Kutuzov.

But putting national vanity entirely aside one feels that such a
conclusion involves a contradiction, since the series of French
victories brought the French complete destruction, while the
series of Russian defeats led to the total destruction of their
enemy and the liberation of their country.

The source of this contradiction lies in the fact that the
historians studying the events from the letters of the sovereigns
and the generals, from memoirs, reports, projects, and so forth,
have attributed to this last period of the war of 1812 an aim
that never existed, namely that of cutting off and capturing
Napoleon with his marshals and his army.

There never was or could have been such an aim, for it would have
been senseless and its attainment quite impossible.

It would have been senseless, first because Napoleon's
disorganized army was flying from Russia with all possible speed,
that is to say, was doing just what every Russian desired. So
what was the use of performing various operations on the French
who were running away as fast as they possibly could?

Secondly, it would have been senseless to block the passage of
men whose whole energy was directed to flight.

Thirdly, it would have been senseless to sacrifice one's own
troops in order to destroy the French army, which without
external interference was destroying itself at such a rate that,
though its path was not blocked, it could not carry across the
frontier more than it actually did in December, namely a
hundredth part of the original army.

Fourthly, it would have been senseless to wish to take captive
the Emperor, kings, and dukes---whose capture would have been in
the highest degree embarrassing for the Russians, as the most
adroit diplomatists of the time (Joseph de Maistre and others)
recognized. Still more senseless would have been the wish to
capture army corps of the French, when our own army had melted
away to half before reaching Krasnoe and a whole division would
have been needed to convoy the corps of prisoners, and when our
men were not always getting full rations and the prisoners
already taken were perishing of hunger.

All the profound plans about cutting off and capturing Napoleon
and his army were like the plan of a market gardener who, when
driving out of his garden a cow that had trampled down the beds
he had planted, should run to the gate and hit the cow on the
head. The only thing to be said in excuse of that gardener would
be that he was very angry. But not even that could be said for
those who drew up this project, for it was not they who had
suffered from the trampled beds.

But besides the fact that cutting off Napoleon with his army
would have been senseless, it was impossible.

It was impossible first because---as experience shows that a
three-mile movement of columns on a battlefield never coincides
with the plans---the probability of Chichagov, Kutuzov, and
Wittgenstein effecting a junction on time at an appointed place
was so remote as to be tantamount to impossibility, as in fact
thought Kutuzov, who when he received the plan remarked that
diversions planned over great distances do not yield the desired
results.

Secondly it was impossible, because to paralyze the momentum with
which Napoleon's army was retiring, incomparably greater forces
than the Russians possessed would have been required.

Thirdly it was impossible, because the military term ``to cut
off'' has no meaning. One can cut off a slice of bread, but not
an army. To cut off an army---to bar its road---is quite
impossible, for there is always plenty of room to avoid capture
and there is the night when nothing can be seen, as the military
scientists might convince themselves by the example of Krasnoe
and of the Berezina. It is only possible to capture prisoners if
they agree to be captured, just as it is only possible to catch a
swallow if it settles on one's hand. Men can only be taken
prisoners if they surrender according to the rules of strategy
and tactics, as the Germans did. But the French troops quite
rightly did not consider that this suited them, since death by
hunger and cold awaited them in flight or captivity alike.

Fourthly and chiefly it was impossible, because never since the
world began has a war been fought under such conditions as those
that obtained in 1812, and the Russian army in its pursuit of the
French strained its strength to the utmost and could not have
done more without destroying itself.

During the movement of the Russian army from Tarutino to Krasnoe
it lost fifty thousand sick or stragglers, that is a number equal
to the population of a large provincial town. Half the men fell
out of the army without a battle.

And it is of this period of the campaign---when the army lacked
boots and sheepskin coats, was short of provisions and without
vodka, and was camping out at night for months in the snow with
fifteen degrees of frost, when there were only seven or eight
hours of daylight and the rest was night in which the influence
of discipline cannot be maintained, when men were taken into that
region of death where discipline fails, not for a few hours only
as in a battle, but for months, where they were every moment
fighting death from hunger and cold, when half the army perished
in a single month---it is of this period of the campaign that the
historians tell us how Miloradovich should have made a flank
march to such and such a place, Tormasov to another place, and
Chichagov should have crossed (more than knee-deep in snow) to
somewhere else, and how so-and-so \emph{routed} and \emph{cut
off} the French and so on and so on.

The Russians, half of whom died, did all that could and should
have been done to attain an end worthy of the nation, and they
are not to blame because other Russians, sitting in warm rooms,
proposed that they should do what was impossible.

All that strange contradiction now difficult to understand
between the facts and the historical accounts only arises because
the historians dealing with the matter have written the history
of the beautiful words and sentiments of various generals, and
not the history of the events.

To them the words of Miloradovich seem very interesting, and so
do their surmises and the rewards this or that general received;
but the question of those fifty thousand men who were left in
hospitals and in graves does not even interest them, for it does
not come within the range of their investigation.

Yet one need only discard the study of the reports and general
plans and consider the movement of those hundreds of thousands of
men who took a direct part in the events, and all the questions
that seemed insoluble easily and simply receive an immediate and
certain solution.

The aim of cutting off Napoleon and his army never existed except
in the imaginations of a dozen people. It could not exist because
it was senseless and unattainable.

The people had a single aim: to free their land from
invasion. That aim was attained in the first place of itself, as
the French ran away, and so it was only necessary not to stop
their flight. Secondly it was attained by the guerrilla warfare
which was destroying the French, and thirdly by the fact that a
large Russian army was following the French, ready to use its
strength in case their movement stopped.

The Russian army had to act like a whip to a running animal. And
the experienced driver knew it was better to hold the whip raised
as a menace than to strike the running animal on the head.

