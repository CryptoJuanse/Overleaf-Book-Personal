\part*{Book Fifteen: 1812 - 13}

% % % % % % % % % % % % % % % % % % % % % % % % % % % % % % % % %
% % % % % % % % % % % % % % % % % % % % % % % % % % % % % % % % %
% % % % % % % % % % % % % % % % % % % % % % % % % % % % % % % % %
% % % % % % % % % % % % % % % % % % % % % % % % % % % % % % % % %
% % % % % % % % % % % % % % % % % % % % % % % % % % % % % % % % %
% % % % % % % % % % % % % % % % % % % % % % % % % % % % % % % % %
% % % % % % % % % % % % % % % % % % % % % % % % % % % % % % % % %
% % % % % % % % % % % % % % % % % % % % % % % % % % % % % % % % %
% % % % % % % % % % % % % % % % % % % % % % % % % % % % % % % % %
% % % % % % % % % % % % % % % % % % % % % % % % % % % % % % % % %
% % % % % % % % % % % % % % % % % % % % % % % % % % % % % % % % %
% % % % % % % % % % % % % % % % % % % % % % % % % % % % % %

\chapter*{Chapter I} 
\ifaudio 
\marginpar{
\href{http://ia800504.us.archive.org/9/items/war_and_peace_15_1105_librivox/war_and_peace_15_01_tolstoy_64kb.mp3}{Audio}}
\fi

\initial{W}{hen} seeing a dying animal a man feels a sense of horror:
substance similar to his own is perishing before his eyes. But
when it is a beloved and intimate human being that is dying,
besides this horror at the extinction of life there is a
severance, a spiritual wound, which like a physical wound is
sometimes fatal and sometimes heals, but always aches and shrinks
at any external irritating touch.

After Prince Andrew's death Natasha and Princess Mary alike felt
this.  Drooping in spirit and closing their eyes before the
menacing cloud of death that overhung them, they dared not look
life in the face. They carefully guarded their open wounds from
any rough and painful contact.  Everything: a carriage passing
rapidly in the street, a summons to dinner, the maid's inquiry
what dress to prepare, or worse still any word of insincere or
feeble sympathy, seemed an insult, painfully irritated the wound,
interrupting that necessary quiet in which they both tried to
listen to the stern and dreadful choir that still resounded in
their imagination, and hindered their gazing into those
mysterious limitless vistas that for an instant had opened out
before them.

Only when alone together were they free from such outrage and
pain. They spoke little even to one another, and when they did it
was of very unimportant matters.

Both avoided any allusion to the future. To admit the possibility
of a future seemed to them to insult his memory. Still more
carefully did they avoid anything relating to him who was
dead. It seemed to them that what they had lived through and
experienced could not be expressed in words, and that any
reference to the details of his life infringed the majesty and
sacredness of the mystery that had been accomplished before their
eyes.

Continued abstention from speech, and constant avoidance of
everything that might lead up to the subject---this halting on
all sides at the boundary of what they might not
mention---brought before their minds with still greater purity
and clearness what they were both feeling.

But pure and complete sorrow is as impossible as pure and
complete joy.  Princess Mary, in her position as absolute and
independent arbiter of her own fate and guardian and instructor
of her nephew, was the first to be called back to life from that
realm of sorrow in which she had dwelt for the first
fortnight. She received letters from her relations to which she
had to reply; the room in which little Nicholas had been put was
damp and he began to cough; Alpatych came to Yaroslavl with
reports on the state of their affairs and with advice and
suggestions that they should return to Moscow to the house on the
Vozdvizhenka Street, which had remained uninjured and needed only
slight repairs. Life did not stand still and it was necessary to
live. Hard as it was for Princess Mary to emerge from the realm
of secluded contemplation in which she had lived till then, and
sorry and almost ashamed as she felt to leave Natasha alone, yet
the cares of life demanded her attention and she involuntarily
yielded to them. She went through the accounts with Alpatych,
conferred with Dessalles about her nephew, and gave orders and
made preparations for the journey to Moscow.

Natasha remained alone and, from the time Princess Mary began
making preparations for departure, held aloof from her too.

Princess Mary asked the countess to let Natasha go with her to
Moscow, and both parents gladly accepted this offer, for they saw
their daughter losing strength every day and thought that a
change of scene and the advice of Moscow doctors would be good
for her.

``I am not going anywhere,'' Natasha replied when this was
proposed to her. ``Do please just leave me alone!'' And she ran
out of the room, with difficulty refraining from tears of
vexation and irritation rather than of sorrow.

After she felt herself deserted by Princes Mary and alone in her
grief, Natasha spent most of the time in her room by herself,
sitting huddled up feet and all in the corner of the sofa,
tearing and twisting something with her slender nervous fingers
and gazing intently and fixedly at whatever her eyes chanced to
fall on. This solitude exhausted and tormented her but she was in
absolute need of it. As soon as anyone entered she got up
quickly, changed her position and expression, and picked up a
book or some sewing, evidently waiting impatiently for the
intruder to go.

She felt all the time as if she might at any moment penetrate
that on which---with a terrible questioning too great for her
strength---her spiritual gaze was fixed.

One day toward the end of December Natasha, pale and thin,
dressed in a black woolen gown, her plaited hair negligently
twisted into a knot, was crouched feet and all in the corner of
her sofa, nervously crumpling and smoothing out the end of her
sash while she looked at a corner of the door.

She was gazing in the direction in which he had gone---to the
other side of life. And that other side of life, of which she had
never before thought and which had formerly seemed to her so far
away and improbable, was now nearer and more akin and more
comprehensible than this side of life, where everything was
either emptiness and desolation or suffering and indignity.

She was gazing where she knew him to be; but she could not
imagine him otherwise than as he had been here. She now saw him
again as he had been at Mytishchi, at Troitsa, and at Yaroslavl.

She saw his face, heard his voice, repeated his words and her
own, and sometimes devised other words they might have spoken.

There he is lying back in an armchair in his velvet cloak,
leaning his head on his thin pale hand. His chest is dreadfully
hollow and his shoulders raised. His lips are firmly closed, his
eyes glitter, and a wrinkle comes and goes on his pale
forehead. One of his legs twitches just perceptibly, but
rapidly. Natasha knows that he is struggling with terrible
pain. ``What is that pain like? Why does he have that pain? What
does he feel? How does it hurt him?'' thought Natasha. He noticed
her watching him, raised his eyes, and began to speak seriously:

``One thing would be terrible,'' said he: ``to bind oneself
forever to a suffering man. It would be continual torture.'' And
he looked searchingly at her. Natasha as usual answered before
she had time to think what she would say. She said: ``This can't
go on---it won't. You will get well---quite well.''

She now saw him from the commencement of that scene and relived
what she had then felt. She recalled his long sad and severe look
at those words and understood the meaning of the rebuke and
despair in that protracted gaze.

``I agreed,'' Natasha now said to herself, ``that it would be
dreadful if he always continued to suffer. I said it then only
because it would have been dreadful for him, but he understood it
differently. He thought it would be dreadful for me. He then
still wished to live and feared death.  And I said it so
awkwardly and stupidly! I did not say what I meant. I thought
quite differently. Had I said what I thought, I should have said:
even if he had to go on dying, to die continually before my eyes,
I should have been happy compared with what I am now. Now there
is nothing... nobody. Did he know that? No, he did not and never
will know it. And now it will never, never be possible to put it
right.'' And now he again seemed to be saying the same words to
her, only in her imagination Natasha this time gave him a
different answer. She stopped him and said: ``Terrible for you,
but not for me! You know that for me there is nothing in life but
you, and to suffer with you is the greatest happiness for me,''
and he took her hand and pressed it as he had pressed it that
terrible evening four days before his death. And in her
imagination she said other tender and loving words which she
might have said then but only spoke now: ``I love thee!... thee!
I love, love...''  she said, convulsively pressing her hands and
setting her teeth with a desperate effort...

She was overcome by sweet sorrow and tears were already rising in
her eyes; then she suddenly asked herself to whom she was saying
this. Again everything was shrouded in hard, dry perplexity, and
again with a strained frown she peered toward the world where he
was. And now, now it seemed to her she was penetrating the
mystery... But at the instant when it seemed that the
incomprehensible was revealing itself to her a loud rattle of the
door handle struck painfully on her ears. Dunyasha, her maid,
entered the room quickly and abruptly with a frightened look on
her face and showing no concern for her mistress.

``Come to your Papa at once, please!'' said she with a strange,
excited look. ``A misfortune... about Peter Ilynich... a
letter,'' she finished with a sob.

% % % % % % % % % % % % % % % % % % % % % % % % % % % % % % % % %
% % % % % % % % % % % % % % % % % % % % % % % % % % % % % % % % %
% % % % % % % % % % % % % % % % % % % % % % % % % % % % % % % % %
% % % % % % % % % % % % % % % % % % % % % % % % % % % % % % % % %
% % % % % % % % % % % % % % % % % % % % % % % % % % % % % % % % %
% % % % % % % % % % % % % % % % % % % % % % % % % % % % % % % % %
% % % % % % % % % % % % % % % % % % % % % % % % % % % % % % % % %
% % % % % % % % % % % % % % % % % % % % % % % % % % % % % % % % %
% % % % % % % % % % % % % % % % % % % % % % % % % % % % % % % % %
% % % % % % % % % % % % % % % % % % % % % % % % % % % % % % % % %
% % % % % % % % % % % % % % % % % % % % % % % % % % % % % % % % %
% % % % % % % % % % % % % % % % % % % % % % % % % % % % % %

\chapter*{Chapter II}
\ifaudio 
\marginpar{
\href{http://ia800504.us.archive.org/9/items/war_and_peace_15_1105_librivox/war_and_peace_15_02_tolstoy_64kb.mp3}{Audio}}
\fi

\initial{B}{esides} a feeling of aloofness from everybody Natasha was feeling
a special estrangement from the members of her own family. All of
them---her father, mother, and Sonya---were so near to her, so
familiar, so commonplace, that all their words and feelings
seemed an insult to the world in which she had been living of
late, and she felt not merely indifferent to them but regarded
them with hostility. She heard Dunyasha's words about Peter
Ilynich and a misfortune, but did not grasp them.

``What misfortune? What misfortune can happen to them? They just
live their own old, quiet, and commonplace life,'' thought
Natasha.

As she entered the ballroom her father was hurriedly coming out
of her mother's room. His face was puckered up and wet with
tears. He had evidently run out of that room to give vent to the
sobs that were choking him. When he saw Natasha he waved his arms
despairingly and burst into convulsively painful sobs that
distorted his soft round face.

``Pe... Petya... Go, go, she... is calling...'' and weeping like
a child and quickly shuffling on his feeble legs to a chair, he
almost fell into it, covering his face with his hands.

Suddenly an electric shock seemed to run through Natasha's whole
being.  Terrible anguish struck her heart, she felt a dreadful
ache as if something was being torn inside her and she were
dying. But the pain was immediately followed by a feeling of
release from the oppressive constraint that had prevented her
taking part in life. The sight of her father, the terribly wild
cries of her mother that she heard through the door, made her
immediately forget herself and her own grief.

She ran to her father, but he feebly waved his arm, pointing to
her mother's door. Princess Mary, pale and with quivering chin,
came out from that room and taking Natasha by the arm said
something to her.  Natasha neither saw nor heard her. She went in
with rapid steps, pausing at the door for an instant as if
struggling with herself, and then ran to her mother.

The countess was lying in an armchair in a strange and awkward
position, stretching out and beating her head against the
wall. Sonya and the maids were holding her arms.

``Natasha! Natasha!...'' cried the countess. ``It's not
true... it's not true... He's lying... Natasha!'' she shrieked,
pushing those around her away. ``Go away, all of you; it's not
true! Killed!... ha, ha, ha!...  It's not true!''

Natasha put one knee on the armchair, stooped over her mother,
embraced her, and with unexpected strength raised her, turned her
face toward herself, and clung to her.

``Mummy!... darling!... I am here, my dearest Mummy,'' she kept
on whispering, not pausing an instant.

She did not let go of her mother but struggled tenderly with her,
demanded a pillow and hot water, and unfastened and tore open her
mother's dress.

``My dearest darling... Mummy, my precious!...'' she whispered
incessantly, kissing her head, her hands, her face, and feeling
her own irrepressible and streaming tears tickling her nose and
cheeks.

The countess pressed her daughter's hand, closed her eyes, and
became quiet for a moment. Suddenly she sat up with unaccustomed
swiftness, glanced vacantly around her, and seeing Natasha began
to press her daughter's head with all her strength. Then she
turned toward her daughter's face which was wincing with pain and
gazed long at it.

``Natasha, you love me?'' she said in a soft trustful
whisper. ``Natasha, you would not deceive me? You'll tell me the
whole truth?''

Natasha looked at her with eyes full of tears and in her look
there was nothing but love and an entreaty for forgiveness.

``My darling Mummy!'' she repeated, straining all the power of
her love to find some way of taking on herself the excess of
grief that crushed her mother.

And again in a futile struggle with reality her mother, refusing
to believe that she could live when her beloved boy was killed in
the bloom of life, escaped from reality into a world of delirium.

Natasha did not remember how that day passed nor that night, nor
the next day and night. She did not sleep and did not leave her
mother. Her persevering and patient love seemed completely to
surround the countess every moment, not explaining or consoling,
but recalling her to life.

During the third night the countess kept very quiet for a few
minutes, and Natasha rested her head on the arm of her chair and
closed her eyes, but opened them again on hearing the bedstead
creak. The countess was sitting up in bed and speaking softly.

``How glad I am you have come. You are tired. Won't you have some
tea?''  Natasha went up to her. ``You have improved in looks and
grown more manly,'' continued the countess, taking her daughter's
hand.

``Mamma! What are you saying...''

``Natasha, he is no more, no more!''

And embracing her daughter, the countess began to weep for the
first time.

% % % % % % % % % % % % % % % % % % % % % % % % % % % % % % % % %
% % % % % % % % % % % % % % % % % % % % % % % % % % % % % % % % %
% % % % % % % % % % % % % % % % % % % % % % % % % % % % % % % % %
% % % % % % % % % % % % % % % % % % % % % % % % % % % % % % % % %
% % % % % % % % % % % % % % % % % % % % % % % % % % % % % % % % %
% % % % % % % % % % % % % % % % % % % % % % % % % % % % % % % % %
% % % % % % % % % % % % % % % % % % % % % % % % % % % % % % % % %
% % % % % % % % % % % % % % % % % % % % % % % % % % % % % % % % %
% % % % % % % % % % % % % % % % % % % % % % % % % % % % % % % % %
% % % % % % % % % % % % % % % % % % % % % % % % % % % % % % % % %
% % % % % % % % % % % % % % % % % % % % % % % % % % % % % % % % %
% % % % % % % % % % % % % % % % % % % % % % % % % % % % % %

\chapter*{Chapter III}
\ifaudio 
\marginpar{
\href{http://ia800504.us.archive.org/9/items/war_and_peace_15_1105_librivox/war_and_peace_15_03_tolstoy_64kb.mp3}{Audio}}
\fi

\initial{P}{rincess} Mary postponed her departure. Sonya and the count tried
to replace Natasha but could not. They saw that she alone was
able to restrain her mother from unreasoning despair. For three
weeks Natasha remained constantly at her mother's side, sleeping
on a lounge chair in her room, making her eat and drink, and
talking to her incessantly because the mere sound of her tender,
caressing tones soothed her mother.

The mother's wounded spirit could not heal. Petya's death had
torn from her half her life. When the news of Petya's death had
come she had been a fresh and vigorous woman of fifty, but a
month later she left her room a listless old woman taking no
interest in life. But the same blow that almost killed the
countess, this second blow, restored Natasha to life.

A spiritual wound produced by a rending of the spiritual body is
like a physical wound and, strange as it may seem, just as a deep
wound may heal and its edges join, physical and spiritual wounds
alike can yet heal completely only as the result of a vital force
from within.

Natasha's wound healed in that way. She thought her life was
ended, but her love for her mother unexpectedly showed her that
the essence of life---love---was still active within her. Love
awoke and so did life.

Prince Andrew's last days had bound Princess Mary and Natasha
together; this new sorrow brought them still closer to one
another. Princess Mary put off her departure, and for three weeks
looked after Natasha as if she had been a sick child. The last
weeks passed in her mother's bedroom had strained Natasha's
physical strength.

One afternoon noticing Natasha shivering with fever, Princess
Mary took her to her own room and made her lie down on the
bed. Natasha lay down, but when Princess Mary had drawn the
blinds and was going away she called her back.

``I don't want to sleep, Mary, sit by me a little.''

``You are tired---try to sleep.''

``No, no. Why did you bring me away? She will be asking for me.''

``She is much better. She spoke so well today,'' said Princess
Mary.

Natasha lay on the bed and in the semidarkness of the room
scanned Princess Mary's face.

``Is she like him?'' thought Natasha. ``Yes, like and yet not
like. But she is quite original, strange, new, and unknown. And
she loves me. What is in her heart? All that is good. But how?
What is her mind like? What does she think about me? Yes, she is
splendid!''

``Mary,'' she said timidly, drawing Princess Mary's hand to
herself, ``Mary, you mustn't think me wicked. No? Mary darling,
how I love you!  Let us be quite, quite friends.''

And Natasha, embracing her, began kissing her face and hands,
making Princess Mary feel shy but happy by this demonstration of
her feelings.

From that day a tender and passionate friendship such as exists
only between women was established between Princess Mary and
Natasha. They were continually kissing and saying tender things
to one another and spent most of their time together. When one
went out the other became restless and hastened to rejoin
her. Together they felt more in harmony with one another than
either of them felt with herself when alone. A feeling stronger
than friendship sprang up between them; an exclusive feeling of
life being possible only in each other's presence.

Sometimes they were silent for hours; sometimes after they were
already in bed they would begin talking and go on till
morning. They spoke most of what was long past. Princess Mary
spoke of her childhood, of her mother, her father, and her
daydreams; and Natasha, who with a passive lack of understanding
had formerly turned away from that life of devotion, submission,
and the poetry of Christian self-sacrifice, now feeling herself
bound to Princess Mary by affection, learned to love her past too
and to understand a side of life previously incomprehensible to
her. She did not think of applying submission and self-abnegation
to her own life, for she was accustomed to seek other joys, but
she understood and loved in another those previously
incomprehensible virtues. For Princess Mary, listening to
Natasha's tales of childhood and early youth, there also opened
out a new and hitherto uncomprehended side of life: belief in
life and its enjoyment.

Just as before, they never mentioned him so as not to lower (as
they thought) their exalted feelings by words; but this silence
about him had the effect of making them gradually begin to forget
him without being conscious of it.

Natasha had grown thin and pale and physically so weak that they
all talked about her health, and this pleased her. But sometimes
she was suddenly overcome by fear not only of death but of
sickness, weakness, and loss of good looks, and involuntarily she
examined her bare arm carefully, surprised at its thinness, and
in the morning noticed her drawn and, as it seemed to her,
piteous face in her glass. It seemed to her that things must be
so, and yet it was dreadfully sad.

One day she went quickly upstairs and found herself out of
breath.  Unconsciously she immediately invented a reason for
going down, and then, testing her strength, ran upstairs again,
observing the result.

Another time when she called Dunyasha her voice trembled, so she
called again---though she could hear Dunyasha coming---called her
in the deep chest tones in which she had been wont to sing, and
listened attentively to herself.

She did not know and would not have believed it, but beneath the
layer of slime that covered her soul and seemed to her
impenetrable, delicate young shoots of grass were already
sprouting, which taking root would so cover with their living
verdure the grief that weighed her down that it would soon no
longer be seen or noticed. The wound had begun to heal from
within.

At the end of January Princess Mary left for Moscow, and the
count insisted on Natasha's going with her to consult the
doctors.

% % % % % % % % % % % % % % % % % % % % % % % % % % % % % % % % %
% % % % % % % % % % % % % % % % % % % % % % % % % % % % % % % % %
% % % % % % % % % % % % % % % % % % % % % % % % % % % % % % % % %
% % % % % % % % % % % % % % % % % % % % % % % % % % % % % % % % %
% % % % % % % % % % % % % % % % % % % % % % % % % % % % % % % % %
% % % % % % % % % % % % % % % % % % % % % % % % % % % % % % % % %
% % % % % % % % % % % % % % % % % % % % % % % % % % % % % % % % %
% % % % % % % % % % % % % % % % % % % % % % % % % % % % % % % % %
% % % % % % % % % % % % % % % % % % % % % % % % % % % % % % % % %
% % % % % % % % % % % % % % % % % % % % % % % % % % % % % % % % %
% % % % % % % % % % % % % % % % % % % % % % % % % % % % % % % % %
% % % % % % % % % % % % % % % % % % % % % % % % % % % % % %

\chapter*{Chapter IV}
\ifaudio 
\marginpar{
\href{http://ia800504.us.archive.org/9/items/war_and_peace_15_1105_librivox/war_and_peace_15_04_tolstoy_64kb.mp3}{Audio}}
\fi

\initial{A}{fter} the encounter at Vyazma, where Kutuzov had been unable to
hold back his troops in their anxiety to overwhelm and cut off
the enemy and so on, the farther movement of the fleeing French,
and of the Russians who pursued them, continued as far as Krasnoe
without a battle. The flight was so rapid that the Russian army
pursuing the French could not keep up with them; cavalry and
artillery horses broke down, and the information received of the
movements of the French was never reliable.

The men in the Russian army were so worn out by this continuous
marching at the rate of twenty-seven miles a day that they could
not go any faster.

To realize the degree of exhaustion of the Russian army it is
only necessary to grasp clearly the meaning of the fact that,
while not losing more than five thousand killed and wounded after
Tarutino and less than a hundred prisoners, the Russian army
which left that place a hundred thousand strong reached Krasnoe
with only fifty thousand.

The rapidity of the Russian pursuit was just as destructive to
our army as the flight of the French was to theirs. The only
difference was that the Russian army moved voluntarily, with no
such threat of destruction as hung over the French, and that the
sick Frenchmen were left behind in enemy hands while the sick
Russians left behind were among their own people. The chief cause
of the wastage of Napoleon's army was the rapidity of its
movement, and a convincing proof of this is the corresponding
decrease of the Russian army.

Kutuzov as far as was in his power, instead of trying to check
the movement of the French as was desired in Petersburg and by
the Russian army generals, directed his whole activity here, as
he had done at Tarutino and Vyazma, to hastening it on while
easing the movement of our army.

But besides this, since the exhaustion and enormous diminution of
the army caused by the rapidity of the advance had become
evident, another reason for slackening the pace and delaying
presented itself to Kutuzov.  The aim of the Russian army was to
pursue the French. The road the French would take was unknown,
and so the closer our troops trod on their heels the greater
distance they had to cover. Only by following at some distance
could one cut across the zigzag path of the French. All the
artful maneuvers suggested by our generals meant fresh movements
of the army and a lengthening of its marches, whereas the only
reasonable aim was to shorten those marches. To that end
Kutuzov's activity was directed during the whole campaign from
Moscow to Vilna---not casually or intermittently but so
consistently that he never once deviated from it.

Kutuzov felt and knew---not by reasoning or science but with the
whole of his Russian being---what every Russian soldier felt:
that the French were beaten, that the enemy was flying and must
be driven out; but at the same time he like the soldiers realized
all the hardship of this march, the rapidity of which was
unparalleled for such a time of the year.

But to the generals, especially the foreign ones in the Russian
army, who wished to distinguish themselves, to astonish somebody,
and for some reason to capture a king or a duke---it seemed that
now---when any battle must be horrible and senseless---was the
very time to fight and conquer somebody. Kutuzov merely shrugged
his shoulders when one after another they presented projects of
maneuvers to be made with those soldiers---ill-shod,
insufficiently clad, and half starved---who within a month and
without fighting a battle had dwindled to half their number, and
who at the best if the flight continued would have to go a
greater distance than they had already traversed, before they
reached the frontier.

This longing to distinguish themselves, to maneuver, to
overthrow, and to cut off showed itself particularly whenever the
Russians stumbled on the French army.

So it was at Krasnoe, where they expected to find one of the
three French columns and stumbled instead on Napoleon himself
with sixteen thousand men. Despite all Kutuzov's efforts to avoid
that ruinous encounter and to preserve his troops, the massacre
of the broken mob of French soldiers by worn-out Russians
continued at Krasnoe for three days.

Toll wrote a disposition: ``The first column will march to so and
so,'' etc. And as usual nothing happened in accord with the
disposition.  Prince Eugene of Wurttemberg fired from a hill over
the French crowds that were running past, and demanded
reinforcements which did not arrive. The French, avoiding the
Russians, dispersed and hid themselves in the forest by night,
making their way round as best they could, and continued their
flight.

Miloradovich, who said he did not want to know anything about the
commissariat affairs of his detachment, and could never be found
when he was wanted---that chevalier sans peur et sans
reproche\footnote{Knight without fear and without reproach.}  as
he styled himself---who was fond of parleys with the French, sent
envoys demanding their surrender, wasted time, and did not do
what he was ordered to do.

``I give you that column, lads,'' he said, riding up to the
troops and pointing out the French to the cavalry.

And the cavalry, with spurs and sabers urging on horses that
could scarcely move, trotted with much effort to the column
presented to them---that is to say, to a crowd of Frenchmen stark
with cold, frost-bitten, and starving---and the column that had
been presented to them threw down its arms and surrendered as it
had long been anxious to do.

At Krasnoe they took twenty-six thousand prisoners, several
hundred cannon, and a stick called a \emph{marshal's staff}, and
disputed as to who had distinguished himself and were pleased
with their achievement---though they much regretted not having
taken Napoleon, or at least a marshal or a hero of some sort, and
reproached one another and especially Kutuzov for having failed
to do so.

These men, carried away by their passions, were but blind tools
of the most melancholy law of necessity, but considered
themselves heroes and imagined that they were accomplishing a
most noble and honorable deed.  They blamed Kutuzov and said that
from the very beginning of the campaign he had prevented their
vanquishing Napoleon, that he thought of nothing but satisfying
his passions and would not advance from the Linen Factories
because he was comfortable there, that at Krasnoe he checked the
advance because on learning that Napoleon was there he had quite
lost his head, and that it was probable that he had an
understanding with Napoleon and had been bribed by him, and so
on, and so on.

Not only did his contemporaries, carried away by their passions,
talk in this way, but posterity and history have acclaimed
Napoleon as grand, while Kutuzov is described by foreigners as a
crafty, dissolute, weak old courtier, and by Russians as
something indefinite---a sort of puppet useful only because he
had a Russian name.

% % % % % % % % % % % % % % % % % % % % % % % % % % % % % % % % %
% % % % % % % % % % % % % % % % % % % % % % % % % % % % % % % % %
% % % % % % % % % % % % % % % % % % % % % % % % % % % % % % % % %
% % % % % % % % % % % % % % % % % % % % % % % % % % % % % % % % %
% % % % % % % % % % % % % % % % % % % % % % % % % % % % % % % % %
% % % % % % % % % % % % % % % % % % % % % % % % % % % % % % % % %
% % % % % % % % % % % % % % % % % % % % % % % % % % % % % % % % %
% % % % % % % % % % % % % % % % % % % % % % % % % % % % % % % % %
% % % % % % % % % % % % % % % % % % % % % % % % % % % % % % % % %
% % % % % % % % % % % % % % % % % % % % % % % % % % % % % % % % %
% % % % % % % % % % % % % % % % % % % % % % % % % % % % % % % % %
% % % % % % % % % % % % % % % % % % % % % % % % % % % % % %

\chapter*{Chapter V}
\ifaudio 
\marginpar{
\href{http://ia800504.us.archive.org/9/items/war_and_peace_15_1105_librivox/war_and_peace_15_05_tolstoy_64kb.mp3}{Audio}}
\fi

\initial{I}{n} 1812 and 1813 Kutuzov was openly accused of blundering. The
Emperor was dissatisfied with him. And in a history recently
written by order of the Highest Authorities it is said that
Kutuzov was a cunning court liar, frightened of the name of
Napoleon, and that by his blunders at Krasnoe and the Berezina he
deprived the Russian army of the glory of complete victory over
the French.\footnote{History of the year 1812. The character of
Kutuzov and reflections on the unsatisfactory results of the
battles at Krasnoe, by Bogdanovich.}

Such is the fate not of great men (grands hommes) whom the
Russian mind does not acknowledge, but of those rare and always
solitary individuals who, discerning the will of Providence,
submit their personal will to it. The hatred and contempt of the
crowd punish such men for discerning the higher laws.

For Russian historians, strange and terrible to say,
Napoleon---that most insignificant tool of history who never
anywhere, even in exile, showed human dignity---Napoleon is the
object of adulation and enthusiasm; he is grand. But
Kutuzov---the man who from the beginning to the end of his
activity in 1812, never once swerving by word or deed from
Borodino to Vilna, presented an example exceptional in history of
self-sacrifice and a present consciousness of the future
importance of what was happening---Kutuzov seems to them
something indefinite and pitiful, and when speaking of him and of
the year 1812 they always seem a little ashamed.

And yet it is difficult to imagine an historical character whose
activity was so unswervingly directed to a single aim; and it
would be difficult to imagine any aim more worthy or more
consonant with the will of the whole people. Still more difficult
would it be to find an instance in history of the aim of an
historical personage being so completely accomplished as that to
which all Kutuzov's efforts were directed in 1812.

Kutuzov never talked of ``forty centuries looking down from the
Pyramids,'' of the sacrifices he offered for the fatherland, or
of what he intended to accomplish or had accomplished; in general
he said nothing about himself, adopted no pose, always appeared
to be the simplest and most ordinary of men, and said the
simplest and most ordinary things. He wrote letters to his
daughters and to Madame de Stael, read novels, liked the society
of pretty women, jested with generals, officers, and soldiers,
and never contradicted those who tried to prove anything to
him. When Count Rostopchin at the Yauza bridge galloped up to
Kutuzov with personal reproaches for having caused the
destruction of Moscow, and said: ``How was it you promised not to
abandon Moscow without a battle?'' Kutuzov replied: ``And I shall
not abandon Moscow without a battle,'' though Moscow was then
already abandoned. When Arakcheev, coming to him from the
Emperor, said that Ermolov ought to be appointed chief of the
artillery, Kutuzov replied: ``Yes, I was just saying so myself,''
though a moment before he had said quite the contrary. What did
it matter to him---who then alone amid a senseless crowd
understood the whole tremendous significance of what was
happening---what did it matter to him whether Rostopchin
attributed the calamities of Moscow to him or to himself? Still
less could it matter to him who was appointed chief of the
artillery.

Not merely in these cases but continually did that old man---who
by experience of life had reached the conviction that thoughts
and the words serving as their expression are not what move
people---use quite meaningless words that happened to enter his
head.

But that man, so heedless of his words, did not once during the
whole time of his activity utter one word inconsistent with the
single aim toward which he moved throughout the whole
war. Obviously in spite of himself, in very diverse
circumstances, he repeatedly expressed his real thoughts with the
bitter conviction that he would not be understood.  Beginning
with the battle of Borodino, from which time his disagreement
with those about him began, he alone said that the battle of
Borodino was a victory, and repeated this both verbally and in
his dispatches and reports up to the time of his death. He alone
said that the loss of Moscow is not the loss of Russia. In reply
to Lauriston's proposal of peace, he said: There can be no peace,
for such is the people's will. He alone during the retreat of the
French said that all our maneuvers are useless, everything is
being accomplished of itself better than we could desire; that
the enemy must be offered ``a golden bridge''; that neither the
Tarutino, the Vyazma, nor the Krasnoe battles were necessary;
that we must keep some force to reach the frontier with, and that
he would not sacrifice a single Russian for ten Frenchmen.

And this courtier, as he is described to us, who lies to
Arakcheev to please the Emperor, he alone---incurring thereby the
Emperor's dis\-plea\-sure---said in Vilna that to carry the war 
beyond the frontier is useless and harmful.

Nor do words alone prove that only he understood the meaning of
the events. His actions---without the smallest deviation---were
all directed to one and the same threefold end:
\begin{enumerate}
\item to brace all his strength for conflict with the French,
\item to defeat them, and
\item to drive them out of Russia, minimizing as far as possible
the sufferings of our people and of our army.
\end{enumerate}

This procrastinator Kutuzov, whose motto was \emph{Patience and
Time}, this enemy of decisive action, gave battle at Borodino,
investing the preparations for it with unparalleled
solemnity. This Kutuzov who before the battle of Austerlitz began
said that it would be lost, he alone, in contradiction to
everyone else, declared till his death that Borodino was a
victory, despite the assurance of generals that the battle was
lost and despite the fact that for an army to have to retire
after winning a battle was unprecedented. He alone during the
whole retreat insisted that battles, which were useless then,
should not be fought, and that a new war should not be begun nor
the frontiers of Russia crossed.

It is easy now to understand the significance of these
events---if only we abstain from attributing to the activity of
the mass aims that existed only in the heads of a dozen
individuals---for the events and results now lie before us.

But how did that old man, alone, in opposition to the general
opinion, so truly discern the importance of the people's view of
the events that in all his activity he was never once untrue to
it?

The source of that extraordinary power of penetrating the meaning
of the events then occuring lay in the national feeling which he
possessed in full purity and strength.

Only the recognition of the fact that he possessed this feeling
caused the people in so strange a manner, contrary to the Tsar's
wish, to select him---an old man in disfavor---to be their
representative in the national war. And only that feeling placed
him on that highest human pedestal from which he, the
commander-in-chief, devoted all his powers not to slaying and
destroying men but to saving and showing pity on them.

That simple, modest, and therefore truly great, figure could not
be cast in the false mold of a European hero---the supposed ruler
of men---that history has invented.

To a lackey no man can be great, for a lackey has his own
conception of greatness.

% % % % % % % % % % % % % % % % % % % % % % % % % % % % % % % % %
% % % % % % % % % % % % % % % % % % % % % % % % % % % % % % % % %
% % % % % % % % % % % % % % % % % % % % % % % % % % % % % % % % %
% % % % % % % % % % % % % % % % % % % % % % % % % % % % % % % % %
% % % % % % % % % % % % % % % % % % % % % % % % % % % % % % % % %
% % % % % % % % % % % % % % % % % % % % % % % % % % % % % % % % %
% % % % % % % % % % % % % % % % % % % % % % % % % % % % % % % % %
% % % % % % % % % % % % % % % % % % % % % % % % % % % % % % % % %
% % % % % % % % % % % % % % % % % % % % % % % % % % % % % % % % %
% % % % % % % % % % % % % % % % % % % % % % % % % % % % % % % % %
% % % % % % % % % % % % % % % % % % % % % % % % % % % % % % % % %
% % % % % % % % % % % % % % % % % % % % % % % % % % % % % %

\chapter*{Chapter VI}
\ifaudio 
\marginpar{
\href{http://ia800504.us.archive.org/9/items/war_and_peace_15_1105_librivox/war_and_peace_15_06_tolstoy_64kb.mp3}{Audio}}
\fi

\initial{T}{he} fifth of November was the first day of what is called the
battle of Krasnoe. Toward evening---after much disputing and many
mistakes made by generals who did not go to their proper places,
and after adjutants had been sent about with counterorders---when
it had become plain that the enemy was everywhere in flight and
that there could and would be no battle, Kutuzov left Krasnoe and
went to Dobroe whither his headquarters had that day been
transferred.

The day was clear and frosty. Kutuzov rode to Dobroe on his plump
little white horse, followed by an enormous suite of discontented
generals who whispered among themselves behind his back. All
along the road groups of French prisoners captured that day
(there were seven thousand of them) were crowding to warm
themselves at campfires. Near Dobroe an immense crowd of tattered
prisoners, buzzing with talk and wrapped and bandaged in anything
they had been able to get hold of, were standing in the road
beside a long row of unharnessed French guns. At the approach of
the commander-in-chief the buzz of talk ceased and all eyes were
fixed on Kutuzov who, wearing a white cap with a red band and a
padded overcoat that bulged on his round shoulders, moved slowly
along the road on his white horse. One of the generals was
reporting to him where the guns and prisoners had been captured.

Kutuzov seemed preoccupied and did not listen to what the general
was saying. He screwed up his eyes with a dissatisfied look as he
gazed attentively and fixedly at these prisoners, who presented a
specially wretched appearance. Most of them were disfigured by
frost-bitten noses and cheeks, and nearly all had red, swollen
and festering eyes.

One group of the French stood close to the road, and two of them,
one of whom had his face covered with sores, were tearing a piece
of raw flesh with their hands. There was something horrible and
bestial in the fleeting glance they threw at the riders and in
the malevolent expression with which, after a glance at Kutuzov,
the soldier with the sores immediately turned away and went on
with what he was doing.

Kutuzov looked long and intently at these two soldiers. He
puckered his face, screwed up his eyes, and pensively swayed his
head. At another spot he noticed a Russian soldier laughingly
patting a Frenchman on the shoulder, saying something to him in a
friendly manner, and Kutuzov with the same expression on his face
again swayed his head.

``What were you saying?'' he asked the general, who continuing
his report directed the commander-in-chief's attention to some
standards captured from the French and standing in front of the
Preobrazhensk regiment.

``Ah, the standards!'' said Kutuzov, evidently detaching himself
with difficulty from the thoughts that preoccupied him.

He looked about him absently. Thousands of eyes were looking at
him from all sides awaiting a word from him.

He stopped in front of the Preobrazhensk regiment, sighed deeply,
and closed his eyes. One of his suite beckoned to the soldiers
carrying the standards to advance and surround the
commander-in-chief with them.  Kutuzov was silent for a few
seconds and then, submitting with evident reluctance to the duty
imposed by his position, raised his head and began to speak. A
throng of officers surrounded him. He looked attentively around
at the circle of officers, recognizing several of them.

``I thank you all!'' he said, addressing the soldiers and then
again the officers. In the stillness around him his slowly
uttered words were distinctly heard. ``I thank you all for your
hard and faithful service.  The victory is complete and Russia
will not forget you! Honor to you forever.''

He paused and looked around.

``Lower its head, lower it!'' he said to a soldier who had
accidentally lowered the French eagle he was holding before the
Preobrazhensk standards. ``Lower, lower, that's it. Hurrah
lads!''  he added, addressing the men with a rapid movement of
his chin.

``Hur-r-rah!'' roared thousands of voices.

While the soldiers were shouting Kutuzov leaned forward in his
saddle and bowed his head, and his eye lit up with a mild and
apparently ironic gleam.

``You see, brothers...'' said he when the shouts had
ceased... and all at once his voice and the expression of his
face changed. It was no longer the commander-in-chief speaking
but an ordinary old man who wanted to tell his comrades something
very important.

There was a stir among the throng of officers and in the ranks of
the soldiers, who moved that they might hear better what he was
going to say.

``You see, brothers, I know it's hard for you, but it can't be
helped!  Bear up; it won't be for long now! We'll see our
visitors off and then we'll rest. The Tsar won't forget your
service. It is hard for you, but still you are at home while
they---you see what they have come to,'' said he, pointing to the
prisoners. ``Worse off than our poorest beggars.  While they were
strong we didn't spare ourselves, but now we may even pity
them. They are human beings too. Isn't it so, lads?''

He looked around, and in the direct, respectful, wondering gaze
fixed upon him he read sympathy with what he had said. His face
grew brighter and brighter with an old man's mild smile, which
drew the corners of his lips and eyes into a cluster of
wrinkles. He ceased speaking and bowed his head as if in
perplexity.

``But after all who asked them here? Serves them right, the
bloody bastards!'' he cried, suddenly lifting his head.

And flourishing his whip he rode off at a gallop for the first
time during the whole campaign, and left the broken ranks of the
soldiers laughing joyfully and shouting ``Hurrah!''

Kutuzov's words were hardly understood by the troops. No one
could have repeated the field marshal's address, begun solemnly
and then changing into an old man's simplehearted talk; but the
hearty sincerity of that speech, the feeling of majestic triumph
combined with pity for the foe and consciousness of the justice
of our cause, exactly expressed by that old man's good-natured
expletives, was not merely understood but lay in the soul of
every soldier and found expression in their joyous and
long-sustained shouts. Afterwards when one of the generals
addressed Kutuzov asking whether he wished his caleche to be sent
for, Kutuzov in answering unexpectedly gave a sob, being
evidently greatly moved.

% % % % % % % % % % % % % % % % % % % % % % % % % % % % % % % % %
% % % % % % % % % % % % % % % % % % % % % % % % % % % % % % % % %
% % % % % % % % % % % % % % % % % % % % % % % % % % % % % % % % %
% % % % % % % % % % % % % % % % % % % % % % % % % % % % % % % % %
% % % % % % % % % % % % % % % % % % % % % % % % % % % % % % % % %
% % % % % % % % % % % % % % % % % % % % % % % % % % % % % % % % %
% % % % % % % % % % % % % % % % % % % % % % % % % % % % % % % % %
% % % % % % % % % % % % % % % % % % % % % % % % % % % % % % % % %
% % % % % % % % % % % % % % % % % % % % % % % % % % % % % % % % %
% % % % % % % % % % % % % % % % % % % % % % % % % % % % % % % % %
% % % % % % % % % % % % % % % % % % % % % % % % % % % % % % % % %
% % % % % % % % % % % % % % % % % % % % % % % % % % % % % %

\chapter*{Chapter VII}
\ifaudio 
\marginpar{
\href{http://ia800504.us.archive.org/9/items/war_and_peace_15_1105_librivox/war_and_peace_15_07_tolstoy_64kb.mp3}{Audio}}
\fi

\initial{W}{hen} the troops reached their night's halting place on the eighth
of November, the last day of the Krasnoe battles, it was already
growing dusk. All day it had been calm and frosty with occasional
lightly falling snow and toward evening it began to
clear. Through the falling snow a purple-black and starry sky
showed itself and the frost grew keener.

An infantry regiment which had left Tarutino three thousand
strong but now numbered only nine hundred was one of the first to
arrive that night at its halting place---a village on the
highroad. The quartermasters who met the regiment announced that
all the huts were full of sick and dead Frenchmen, cavalrymen,
and members of the staff. There was only one hut available for
the regimental commander.

The commander rode up to his hut. The regiment passed through the
village and stacked its arms in front of the last huts.

Like some huge many-limbed animal, the regiment began to prepare
its lair and its food. One part of it dispersed and waded
knee-deep through the snow into a birch forest to the right of
the village, and immediately the sound of axes and swords, the
crashing of branches, and merry voices could be heard from
there. Another section amid the regimental wagons and horses
which were standing in a group was busy getting out caldrons and
rye biscuit, and feeding the horses. A third section scattered
through the village arranging quarters for the staff officers,
carrying out the French corpses that were in the huts, and
dragging away boards, dry wood, and thatch from the roofs, for
the campfires, or wattle fences to serve for shelter.

Some fifteen men with merry shouts were shaking down the high
wattle wall of a shed, the roof of which had already been
removed.

``Now then, all together---shove!'' cried the voices, and the
huge surface of the wall, sprinkled with snow and creaking with
frost, was seen swaying in the gloom of the night. The lower
stakes cracked more and more and at last the wall fell, and with
it the men who had been pushing it. Loud, coarse laughter and
joyous shouts ensued.

``Now then, catch hold in twos! Hand up the lever! That's
it... Where are you shoving to?''

``Now, all together! But wait a moment, boys... With a song!''

All stood silent, and a soft, pleasant velvety voice began to
sing. At the end of the third verse as the last note died away,
twenty voices roared out at once: ``Oo-oo-oo-oo! That's it. All
together! Heave away, boys!...'' but despite their united efforts
the wattle hardly moved, and in the silence that followed the
heavy breathing of the men was audible.

``Here, you of the Sixth Company! Devils that you are! Lend a
hand...  will you? You may want us one of these days.''

Some twenty men of the Sixth Company who were on their way into
the village joined the haulers, and the wattle wall, which was
about thirty-five feet long and seven feet high, moved forward
along the village street, swaying, pressing upon and cutting the
shoulders of the gasping men.

``Get along... Falling? What are you stopping for? There now...''

Merry senseless words of abuse flowed freely.

``What are you up to?'' suddenly came the authoritative voice of
a sergeant major who came upon the men who were hauling their
burden.  ``There are gentry here; the general himself is in that
hut, and you foul-mouthed devils, you brutes, I'll give it to
you!'' shouted he, hitting the first man who came in his way a
swinging blow on the back.  ``Can't you make less noise?''

The men became silent. The soldier who had been struck groaned
and wiped his face, which had been scratched till it bled by his
falling against the wattle.

``There, how that devil hits out! He's made my face all bloody,''
said he in a frightened whisper when the sergeant major had
passed on.

``Don't you like it?'' said a laughing voice, and moderating
their tones the men moved forward.

When they were out of the village they began talking again as
loud as before, interlarding their talk with the same aimless
expletives.

In the hut which the men had passed, the chief officers had
gathered and were in animated talk over their tea about the
events of the day and the maneuvers suggested for tomorrow. It
was proposed to make a flank march to the left, cut off the
Vice-King (Murat) and capture him.

By the time the soldiers had dragged the wattle fence to its
place the campfires were blazing on all sides ready for cooking,
the wood crackled, the snow was melting, and black shadows of
soldiers flitted to and fro all over the occupied space where the
snow had been trodden down.

Axes and choppers were plied all around. Everything was done
without any orders being given. Stores of wood were brought for
the night, shelters were rigged up for the officers, caldrons
were being boiled, and muskets and accouterments put in order.

The wattle wall the men had brought was set up in a semicircle by
the Eighth Company as a shelter from the north, propped up by
musket rests, and a campfire was built before it. They beat the
tattoo, called the roll, had supper, and settled down round the
fires for the night---some repairing their footgear, some smoking
pipes, and some stripping themselves naked to steam the lice out
of their shirts.

% % % % % % % % % % % % % % % % % % % % % % % % % % % % % % % % %
% % % % % % % % % % % % % % % % % % % % % % % % % % % % % % % % %
% % % % % % % % % % % % % % % % % % % % % % % % % % % % % % % % %
% % % % % % % % % % % % % % % % % % % % % % % % % % % % % % % % %
% % % % % % % % % % % % % % % % % % % % % % % % % % % % % % % % %
% % % % % % % % % % % % % % % % % % % % % % % % % % % % % % % % %
% % % % % % % % % % % % % % % % % % % % % % % % % % % % % % % % %
% % % % % % % % % % % % % % % % % % % % % % % % % % % % % % % % %
% % % % % % % % % % % % % % % % % % % % % % % % % % % % % % % % %
% % % % % % % % % % % % % % % % % % % % % % % % % % % % % % % % %
% % % % % % % % % % % % % % % % % % % % % % % % % % % % % % % % %
% % % % % % % % % % % % % % % % % % % % % % % % % % % % % %

\chapter*{Chapter VIII}
\ifaudio 
\marginpar{
\href{http://ia800504.us.archive.org/9/items/war_and_peace_15_1105_librivox/war_and_peace_15_08_tolstoy_64kb.mp3}{Audio}}
\fi

\initial{O}{ne} would have thought that under the almost incredibly wretched
conditions the Russian soldiers were in at that time---lacking
warm boots and sheepskin coats, without a roof over their heads,
in the snow with eighteen degrees of frost, and without even full
rations (the commissariat did not always keep up with the
troops)---they would have presented a very sad and depressing
spectacle.

On the contrary, the army had never under the best material
conditions presented a more cheerful and animated aspect. This
was because all who began to grow depressed or who lost strength
were sifted out of the army day by day. All the physically or
morally weak had long since been left behind and only the flower
of the army---physically and mentally---remained.

More men collected behind the wattle fence of the Eighth Company
than anywhere else. Two sergeants major were sitting with them
and their campfire blazed brighter than others. For leave to sit
by their wattle they demanded contributions of fuel.

``Eh, Makeev! What has become of you, you son of a bitch? Are you
lost or have the wolves eaten you? Fetch some more wood!''
shouted a red-haired and red-faced man, screwing up his eyes and
blinking because of the smoke but not moving back from the
fire. ``And you, Jackdaw, go and fetch some wood!'' said he to
another soldier.

This red-haired man was neither a sergeant nor a corporal, but
being robust he ordered about those weaker than himself. The
soldier they called ``Jackdaw,'' a thin little fellow with a
sharp nose, rose obediently and was about to go but at that
instant there came into the light of the fire the slender,
handsome figure of a young soldier carrying a load of wood.

``Bring it here---that's fine!''

They split up the wood, pressed it down on the fire, blew at it
with their mouths, and fanned it with the skirts of their
greatcoats, making the flames hiss and crackle. The men drew
nearer and lit their pipes.  The handsome young soldier who had
brought the wood, setting his arms akimbo, began stamping his
cold feet rapidly and deftly on the spot where he stood.

``Mother! The dew is cold but clear... It's well that I'm a
musketeer...'' he sang, pretending to hiccough after each
syllable.

``Look out, your soles will fly off!'' shouted the red-haired
man, noticing that the sole of the dancer's boot was hanging
loose. ``What a fellow you are for dancing!''

The dancer stopped, pulled off the loose piece of leather, and
threw it on the fire.

``Right enough, friend,'' said he, and, having sat down, took out
of his knapsack a scrap of blue French cloth, and wrapped it
round his foot.  ``It's the steam that spoils them,'' he added,
stretching out his feet toward the fire.

``They'll soon be issuing us new ones. They say that when we've
finished hammering them, we're to receive double kits!''

``And that son of a bitch Petrov has lagged behind after all, it
seems,'' said one sergeant major.

``I've had an eye on him this long while,'' said the other.

``Well, he's a poor sort of soldier...''

``But in the Third Company they say nine men were missing
yesterday.''

``Yes, it's all very well, but when a man's feet are frozen how
can he walk?''

``Eh? Don't talk nonsense!'' said a sergeant major.

``Do you want to be doing the same?'' said an old soldier,
turning reproachfully to the man who had spoken of frozen feet.

``Well, you know,'' said the sharp-nosed man they called Jackdaw
in a squeaky and unsteady voice, raising himself at the other
side of the fire, ``a plump man gets thin, but for a thin one
it's death. Take me, now! I've got no strength left,'' he added,
with sudden resolution turning to the sergeant major. ``Tell them
to send me to hospital; I'm aching all over; anyway I shan't be
able to keep up.''

``That'll do, that'll do!'' replied the sergeant major quietly.

The soldier said no more and the talk went on.

``What a lot of those Frenchies were taken today, and the fact is
that not one of them had what you might call real boots on,''
said a soldier, starting a new theme. ``They were no more than
make-believes.''

``The Cossacks have taken their boots. They were clearing the hut
for the colonel and carried them out. It was pitiful to see them,
boys,'' put in the dancer. ``As they turned them over one seemed
still alive and, would you believe it, he jabbered something in
their lingo.''

``But they're a clean folk, lads,'' the first man went on; ``he
was white---as white as birchbark---and some of them are such
fine fellows, you might think they were nobles.''

``Well, what do you think? They make soldiers of all classes
there.''

``But they don't understand our talk at all,'' said the dancer
with a puzzled smile. ``I asked him whose subject he was, and he
jabbered in his own way. A queer lot!''

``But it's strange, friends,'' continued the man who had wondered
at their whiteness, ``the peasants at Mozhaysk were saying that
when they began burying the dead---where the battle was you
know---well, those dead had been lying there for nearly a month,
and says the peasant, 'they lie as white as paper, clean, and not
as much smell as a puff of powder smoke.'{}''

``Was it from the cold?'' asked someone.

``You're a clever fellow! From the cold indeed! Why, it was
hot. If it had been from the cold, ours would not have rotted
either. 'But,' he says, 'go up to ours and they are all rotten
and maggoty. So,' he says, 'we tie our faces up with kerchiefs
and turn our heads away as we drag them off: we can hardly do
it. But theirs,' he says, 'are white as paper and not so much
smell as a whiff of gunpowder.'{}''

All were silent.

``It must be from their food,'' said the sergeant major. ``They
used to gobble the same food as the gentry.''

No one contradicted him.

``That peasant near Mozhaysk where the battle was said the men
were all called up from ten villages around and they carted for
twenty days and still didn't finish carting the dead away. And as
for the wolves, he says...''

``That was a real battle,'' said an old soldier. ``It's the only
one worth remembering; but since that... it's only been
tormenting folk.''

``And do you know, Daddy, the day before yesterday we ran at them
and, my word, they didn't let us get near before they just threw
down their muskets and went on their knees. 'Pardon!' they
say. That's only one case. They say Platov took 'Poleon himself
twice. But he didn't know the right charm. He catches him and
catches him---no good! He turns into a bird in his hands and
flies away. And there's no way of killing him either.''

``You're a first-class liar, Kiselev, when I come to look at
you!''

``Liar, indeed! It's the real truth.''

``If he fell into my hands, when I'd caught him I'd bury him in
the ground with an aspen stake to fix him down. What a lot of men
he's ruined!''

``Well, anyhow we're going to end it. He won't come here again,''
remarked the old soldier, yawning.

The conversation flagged, and the soldiers began settling down to
sleep.

``Look at the stars. It's wonderful how they shine! You would
think the women had spread out their linen,'' said one of the
men, gazing with admiration at the Milky Way.

``That's a sign of a good harvest next year.''

``We shall want some more wood.''

``You warm your back and your belly gets frozen. That's queer.''

``O Lord!''

``What are you pushing for? Is the fire only for you? Look how
he's sprawling!''

In the silence that ensued, the snoring of those who had fallen
asleep could be heard. Others turned over and warmed themselves,
now and again exchanging a few words. From a campfire a hundred
paces off came a sound of general, merry laughter.

``Hark at them roaring there in the Fifth Company!'' said one of
the soldiers, ``and what a lot of them there are!''

One of the men got up and went over to the Fifth Company.

``They're having such fun,'' said he, coming back. ``Two
Frenchies have turned up. One's quite frozen and the other's an
awful swaggerer. He's singing songs...''

``Oh, I'll go across and have a look...''

And several of the men went over to the Fifth Company.

% % % % % % % % % % % % % % % % % % % % % % % % % % % % % % % % %
% % % % % % % % % % % % % % % % % % % % % % % % % % % % % % % % %
% % % % % % % % % % % % % % % % % % % % % % % % % % % % % % % % %
% % % % % % % % % % % % % % % % % % % % % % % % % % % % % % % % %
% % % % % % % % % % % % % % % % % % % % % % % % % % % % % % % % %
% % % % % % % % % % % % % % % % % % % % % % % % % % % % % % % % %
% % % % % % % % % % % % % % % % % % % % % % % % % % % % % % % % %
% % % % % % % % % % % % % % % % % % % % % % % % % % % % % % % % %
% % % % % % % % % % % % % % % % % % % % % % % % % % % % % % % % %
% % % % % % % % % % % % % % % % % % % % % % % % % % % % % % % % %
% % % % % % % % % % % % % % % % % % % % % % % % % % % % % % % % %
% % % % % % % % % % % % % % % % % % % % % % % % % % % % % %

\chapter*{Chapter IX}
\ifaudio 
\marginpar{
\href{http://ia800504.us.archive.org/9/items/war_and_peace_15_1105_librivox/war_and_peace_15_09_tolstoy_64kb.mp3}{Audio}}
\fi

\initial{T}{he} fifth company was bivouacking at the very edge of the
forest. A huge campfire was blazing brightly in the midst of the
snow, lighting up the branches of trees heavy with hoarfrost.

About midnight they heard the sound of steps in the snow of the
forest, and the crackling of dry branches.

``A bear, lads,'' said one of the men.

They all raised their heads to listen, and out of the forest into
the bright firelight stepped two strangely clad human figures
clinging to one another.

These were two Frenchmen who had been hiding in the forest. They
came up to the fire, hoarsely uttering something in a language
our soldiers did not understand. One was taller than the other;
he wore an officer's hat and seemed quite exhausted. On
approaching the fire he had been going to sit down, but fell. The
other, a short sturdy soldier with a shawl tied round his head,
was stronger. He raised his companion and said something,
pointing to his mouth. The soldiers surrounded the Frenchmen,
spread a greatcoat on the ground for the sick man, and brought
some buckwheat porridge and vodka for both of them.

The exhausted French officer was Ramballe and the man with his
head wrapped in the shawl was Morel, his orderly.

When Morel had drunk some vodka and finished his bowl of porridge
he suddenly became unnaturally merry and chattered incessantly to
the soldiers, who could not understand him. Ramballe refused food
and resting his head on his elbow lay silent beside the campfire,
looking at the Russian soldiers with red and vacant
eyes. Occasionally he emitted a long-drawn groan and then again
became silent. Morel, pointing to his shoulders, tried to impress
on the soldiers the fact that Ramballe was an officer and ought
to be warmed. A Russian officer who had come up to the fire sent
to ask his colonel whether he would not take a French officer
into his hut to warm him, and when the messenger returned and
said that the colonel wished the officer to be brought to him,
Ramballe was told to go. He rose and tried to walk, but staggered
and would have fallen had not a soldier standing by held him up.

``You won't do it again, eh?'' said one of the soldiers, winking
and turning mockingly to Ramballe.

``Oh, you fool! Why talk rubbish, lout that you are---a real
peasant!''  came rebukes from all sides addressed to the jesting
soldier.

They surrounded Ramballe, lifted him on the crossed arms of two
soldiers, and carried him to the hut. Ramballe put his arms
around their necks while they carried him and began wailing
plaintively:

``Oh, you fine fellows, my kind, kind friends! These are men! Oh,
my brave, kind friends,'' and he leaned his head against the
shoulder of one of the men like a child.

Meanwhile Morel was sitting in the best place by the fire,
surrounded by the soldiers.

Morel, a short sturdy Frenchman with inflamed and streaming eyes,
was wearing a woman's cloak and had a shawl tied woman fashion
round his head over his cap. He was evidently tipsy, and was
singing a French song in a hoarse broken voice, with an arm
thrown round the nearest soldier.  The soldiers simply held their
sides as they watched him.

``Now then, now then, teach us how it goes! I'll soon pick it
up. How is it?'' said the man---a singer and a wag---whom Morel
was embracing.

``Vive Henri Quatre! Vive ce roi valiant!'' sang Morel,
winking. ``Ce diable a quatre...''\footnote{``Long live Henry the
Fourth, that valiant king! That rowdy devil.''}

``Vivarika! Vif-seruvaru! Sedyablyaka!'' repeated the soldier,
flourishing his arm and really catching the tune.

``Bravo! Ha, ha, ha!'' rose their rough, joyous laughter from all
sides.

Morel, wrinkling up his face, laughed too.

``Well, go on, go on!''

``Qui eut le triple talent, De boire, de battre, Et d'etre un
vert galant.''\footnote{Who had a triple talent For drinking, for
fighting, And for being a gallant old boy...}

``It goes smoothly, too. Well, now, Zaletaev!''

``Ke...'' Zaletaev, brought out with effort: ``ke-e-e-e,'' he
drawled, laboriously pursing his lips,
``le-trip-ta-la-de-bu-de-ba, e de-tra-va-ga-la'' he sang.

``Fine! Just like the Frenchie! Oh, ho ho! Do you want some more
to eat?''

``Give him some porridge: it takes a long time to get filled up
after starving.''

They gave him some more porridge and Morel with a laugh set to
work on his third bowl. All the young soldiers smiled gaily as
they watched him.  The older men, who thought it undignified to
amuse themselves with such nonsense, continued to lie at the
opposite side of the fire, but one would occasionally raise
himself on an elbow and glance at Morel with a smile.

``They are men too,'' said one of them as he wrapped himself up
in his coat. ``Even wormwood grows on its own root.''

``O Lord, O Lord! How starry it is! Tremendous! That means a hard
frost...''

They all grew silent. The stars, as if knowing that no one was
looking at them, began to disport themselves in the dark sky: now
flaring up, now vanishing, now trembling, they were busy
whispering something gladsome and mysterious to one another.

% % % % % % % % % % % % % % % % % % % % % % % % % % % % % % % % %
% % % % % % % % % % % % % % % % % % % % % % % % % % % % % % % % %
% % % % % % % % % % % % % % % % % % % % % % % % % % % % % % % % %
% % % % % % % % % % % % % % % % % % % % % % % % % % % % % % % % %
% % % % % % % % % % % % % % % % % % % % % % % % % % % % % % % % %
% % % % % % % % % % % % % % % % % % % % % % % % % % % % % % % % %
% % % % % % % % % % % % % % % % % % % % % % % % % % % % % % % % %
% % % % % % % % % % % % % % % % % % % % % % % % % % % % % % % % %
% % % % % % % % % % % % % % % % % % % % % % % % % % % % % % % % %
% % % % % % % % % % % % % % % % % % % % % % % % % % % % % % % % %
% % % % % % % % % % % % % % % % % % % % % % % % % % % % % % % % %
% % % % % % % % % % % % % % % % % % % % % % % % % % % % % %

\chapter*{Chapter X}
\ifaudio 
\marginpar{
\href{http://ia800504.us.archive.org/9/items/war_and_peace_15_1105_librivox/war_and_peace_15_10_tolstoy_64kb.mp3}{Audio}}
\fi

\initial{T}{he} French army melted away at the uniform rate of a mathematical
progression; and that crossing of the Berezina about which so
much has been written was only one intermediate stage in its
destruction, and not at all the decisive episode of the
campaign. If so much has been and still is written about the
Berezina, on the French side this is only because at the broken
bridge across that river the calamities their army had been
previously enduring were suddenly concentrated at one moment into
a tragic spectacle that remained in every memory, and on the
Russian side merely because in Petersburg---far from the seat of
war---a plan (again one of Pfuel's) had been devised to catch
Napoleon in a strategic trap at the Berezina River. Everyone
assured himself that all would happen according to plan, and
therefore insisted that it was just the crossing of the Berezina
that destroyed the French army. In reality the results of the
crossing were much less disastrous to the French---in guns and
men lost---than Krasnoe had been, as the figures show.

The sole importance of the crossing of the Berezina lies in the
fact that it plainly and indubitably proved the fallacy of all
the plans for cutting off the enemy's retreat and the soundness
of the only possible line of action---the one Kutuzov and the
general mass of the army demanded---namely, simply to follow the
enemy up. The French crowd fled at a continually increasing speed
and all its energy was directed to reaching its goal. It fled
like a wounded animal and it was impossible to block its
path. This was shown not so much by the arrangements it made for
crossing as by what took place at the bridges. When the bridges
broke down, unarmed soldiers, people from Moscow and women with
children who were with the French transport, all---carried on by
vis inertiae---pressed forward into boats and into the
ice-covered water and did not, surrender.

That impulse was reasonable. The condition of fugitives and of
pursuers was equally bad. As long as they remained with their own
people each might hope for help from his fellows and the definite
place he held among them. But those who surrendered, while
remaining in the same pitiful plight, would be on a lower level
to claim a share in the necessities of life. The French did not
need to be informed of the fact that half the prisoners---with
whom the Russians did not know what to do---perished of cold and
hunger despite their captors' desire to save them; they felt that
it could not be otherwise. The most compassionate Russian
commanders, those favorable to the French---and even the
Frenchmen in the Russian service---could do nothing for the
prisoners. The French perished from the conditions to which the
Russian army was itself exposed. It was impossible to take bread
and clothes from our hungry and indispensable soldiers to give to
the French who, though not harmful, or hated, or guilty, were
simply unnecessary. Some Russians even did that, but they were
exceptions.

Certain destruction lay behind the French but in front there was
hope.  Their ships had been burned, there was no salvation save
in collective flight, and on that the whole strength of the
French was concentrated.

The farther they fled the more wretched became the plight of the
remnant, especially after the Berezina, on which (in consequence
of the Petersburg plan) special hopes had been placed by the
Russians, and the keener grew the passions of the Russian
commanders, who blamed one another and Kutuzov most of
all. Anticipation that the failure of the Petersburg Berezina
plan would be attributed to Kutuzov led to dissatisfaction,
contempt, and ridicule, more and more strongly expressed. The
ridicule and contempt were of course expressed in a respectful
form, making it impossible for him to ask wherein he was to
blame. They did not talk seriously to him; when reporting to him
or asking for his sanction they appeared to be fulfilling a
regrettable formality, but they winked behind his back and tried
to mislead him at every turn.

Because they could not understand him all these people assumed
that it was useless to talk to the old man; that he would never
grasp the profundity of their plans, that he would answer with
his phrases (which they thought were mere phrases) about a
\emph{golden bridge}, about the impossibility of crossing the
frontier with a crowd of tatterdemalions, and so forth. They had
heard all that before. And all he said---that it was necessary to
await provisions, or that the men had no boots---was so simple,
while what they proposed was so complicated and clever, that it
was evident that he was old and stupid and that they, though not
in power, were commanders of genius.

After the junction with the army of the brilliant admiral and
Petersburg hero Wittgenstein, this mood and the gossip of the
staff reached their maximum. Kutuzov saw this and merely sighed
and shrugged his shoulders.  Only once, after the affair of the
Berezina, did he get angry and write to Bennigsen (who reported
separately to the Emperor) the following letter:

``On account of your spells of ill health, will your excellency
please be so good as to set off for Kaluga on receipt of this,
and there await further commands and appointments from His
Imperial Majesty.''

But after Bennigsen's departure, the Grand Duke Tsarevich
Constantine Pavlovich joined the army. He had taken part in the
beginning of the campaign but had subsequently been removed from
the army by Kutuzov. Now having come to the army, he informed
Kutuzov of the Emperor's displeasure at the poor success of our
forces and the slowness of their advance. The Emperor intended to
join the army personally in a few days' time.

The old man, experienced in court as well as in military
affairs---this same Kutuzov who in August had been chosen
commander-in-chief against the sovereign's wishes and who had
removed the Grand Duke and heir---apparent from the army---who on
his own authority and contrary to the Emperor's will had decided
on the abandonment of Moscow, now realized at once that his day
was over, that his part was played, and that the power he was
supposed to hold was no longer his. And he understood this not
merely from the attitude of the court. He saw on the one hand
that the military business in which he had played his part was
ended and felt that his mission was accomplished; and at the same
time he began to be conscious of the physical weariness of his
aged body and of the necessity of physical rest.

On the twenty-ninth of November Kutuzov entered Vilna---his
\emph{dear Vilna} as he called it. Twice during his career
Kutuzov had been governor of Vilna. In that wealthy town, which
had not been injured, he found old friends and associations,
besides the comforts of life of which he had so long been
deprived. And he suddenly turned from the cares of army and state
and, as far as the passions that seethed around him allowed,
immersed himself in the quiet life to which he had formerly been
accustomed, as if all that was taking place and all that had
still to be done in the realm of history did not concern him at
all.

Chichagov, one of the most zealous \emph{cutters-off} and
\emph{breakers-up}, who had first wanted to effect a diversion in
Greece and then in Warsaw but never wished to go where he was
sent: Chichagov, noted for the boldness with which he spoke to
the Emperor, and who considered Kutuzov to be under an obligation
to him because when he was sent to make peace with Turkey in 1811
independently of Kutuzov, and found that peace had already been
concluded, he admitted to the Emperor that the merit of securing
that peace was really Kutuzov's; this Chichagov was the first to
meet Kutuzov at the castle where the latter was to stay. In
undress naval uniform, with a dirk, and holding his cap under his
arm, he handed Kutuzov a garrison report and the keys of the
town. The contemptuously respectful attitude of the younger men
to the old man in his dotage was expressed in the highest degree
by the behavior of Chichagov, who knew of the accusations that
were being directed against Kutuzov.

When speaking to Chichagov, Kutuzov incidentally mentioned that
the vehicles packed with china that had been captured from him at
Borisov had been recovered and would be restored to him.

``You mean to imply that I have nothing to eat out of... On the
contrary, I can supply you with everything even if you want to
give dinner parties,'' warmly replied Chichagov, who tried by
every word he spoke to prove his own rectitude and therefore
imagined Kutuzov to be animated by the same desire.

Kutuzov, shrugging his shoulders, replied with his subtle
penetrating smile: ``I meant merely to say what I said.''

Contrary to the Emperor's wish Kutuzov detained the greater part
of the army at Vilna. Those about him said that he became
extraordinarily slack and physically feeble during his stay in
that town. He attended to army affairs reluctantly, left
everything to his generals, and while awaiting the Emperor's
arrival led a dissipated life.

Having left Petersburg on the seventh of December with his
suite---Count Tolstoy, Prince Volkonski, Arakcheev, and
others---the Emperor reached Vilna on the eleventh, and in his
traveling sleigh drove straight to the castle. In spite of the
severe frost some hundred generals and staff officers in full
parade uniform stood in front of the castle, as well as a guard
of honor of the Semenov regiment.

A courier who galloped to the castle in advance, in a troyka with
three foam-flecked horses, shouted ``Coming!'' and Konovnitsyn
rushed into the vestibule to inform Kutuzov, who was waiting in
the hall porter's little lodge.

A minute later the old man's large stout figure in full-dress
uniform, his chest covered with orders and a scarf drawn round
his stomach, waddled out into the porch. He put on his hat with
its peaks to the sides and, holding his gloves in his hand and
walking with an effort sideways down the steps to the level of
the street, took in his hand the report he had prepared for the
Emperor.

There was running to and fro and whispering; another troyka flew
furiously up, and then all eyes were turned on an approaching
sleigh in which the figures of the Emperor and Volkonski could
already be descried.

From the habit of fifty years all this had a physically agitating
effect on the old general. He carefully and hastily felt himself
all over, readjusted his hat, and pulling himself together drew
himself up and, at the very moment when the Emperor, having
alighted from the sleigh, lifted his eyes to him, handed him the
report and began speaking in his smooth, ingratiating voice.

The Emperor with a rapid glance scanned Kutuzov from head to
foot, frowned for an instant, but immediately mastering himself
went up to the old man, extended his arms and embraced him. And
this embrace too, owing to a long-standing impression related to
his innermost feelings, had its usual effect on Kutuzov and he
gave a sob.

The Emperor greeted the officers and the Semenov guard, and again
pressing the old man's hand went with him into the castle.

When alone with the field marshal the Emperor expressed his
dissatisfaction at the slowness of the pursuit and at the
mistakes made at Krasnoe and the Berezina, and informed him of
his intentions for a future campaign abroad. Kutuzov made no
rejoinder or remark. The same submissive, expressionless look
with which he had listened to the Emperor's commands on the field
of Austerlitz seven years before settled on his face now.

When Kutuzov came out of the study and with lowered head was
crossing the ballroom with his heavy waddling gait, he was
arrested by someone's voice saying:

``Your Serene Highness!''

Kutuzov raised his head and looked for a long while into the eyes
of Count Tolstoy, who stood before him holding a silver salver on
which lay a small object. Kutuzov seemed not to understand what
was expected of him.

Suddenly he seemed to remember; a scarcely perceptible smile
flashed across his puffy face, and bowing low and respectfully he
took the object that lay on the salver. It was the Order of
St. George of the First Class.

% % % % % % % % % % % % % % % % % % % % % % % % % % % % % % % % %
% % % % % % % % % % % % % % % % % % % % % % % % % % % % % % % % %
% % % % % % % % % % % % % % % % % % % % % % % % % % % % % % % % %
% % % % % % % % % % % % % % % % % % % % % % % % % % % % % % % % %
% % % % % % % % % % % % % % % % % % % % % % % % % % % % % % % % %
% % % % % % % % % % % % % % % % % % % % % % % % % % % % % % % % %
% % % % % % % % % % % % % % % % % % % % % % % % % % % % % % % % %
% % % % % % % % % % % % % % % % % % % % % % % % % % % % % % % % %
% % % % % % % % % % % % % % % % % % % % % % % % % % % % % % % % %
% % % % % % % % % % % % % % % % % % % % % % % % % % % % % % % % %
% % % % % % % % % % % % % % % % % % % % % % % % % % % % % % % % %
% % % % % % % % % % % % % % % % % % % % % % % % % % % % % %

\chapter*{Chapter XI}
\ifaudio 
\marginpar{
\href{http://ia800504.us.archive.org/9/items/war_and_peace_15_1105_librivox/war_and_peace_15_11_tolstoy_64kb.mp3}{Audio}}
\fi

\initial{N}{ext} day the field marshal gave a dinner and ball which the
Emperor honored by his presence. Kutuzov had received the Order
of St. George of the First Class and the Emperor showed him the
highest honors, but everyone knew of the imperial dissatisfaction
with him. The proprieties were observed and the Emperor was the
first to set that example, but everybody understood that the old
man was blameworthy and good-for-nothing. When Kutuzov,
conforming to a custom of Catherine's day, ordered the standards
that had been captured to be lowered at the Emperor's feet on his
entering the ballroom, the Emperor made a wry face and muttered
something in which some people caught the words, ``the old
comedian.''

The Emperor's displeasure with Kutuzov was specially increased at
Vilna by the fact that Kutuzov evidently could not or would not
understand the importance of the coming campaign.

When on the following morning the Emperor said to the officers
assembled about him: ``You have not only saved Russia, you have
saved Europe!'' they all understood that the war was not ended.

Kutuzov alone would not see this and openly expressed his opinion
that no fresh war could improve the position or add to the glory
of Russia, but could only spoil and lower the glorious position
that Russia had gained. He tried to prove to the Emperor the
impossibility of levying fresh troops, spoke of the hardships
already endured by the people, of the possibility of failure and
so forth.

This being the field marshal's frame of mind he was naturally
regarded as merely a hindrance and obstacle to the impending war.

To avoid unpleasant encounters with the old man, the natural
method was to do what had been done with him at Austerlitz and
with Barclay at the beginning of the Russian campaign---to
transfer the authority to the Emperor himself, thus cutting the
ground from under the commander in chief's feet without upsetting
the old man by informing him of the change.

With this object his staff was gradually reconstructed and its
real strength removed and transferred to the Emperor. Toll,
Konovnitsyn, and Ermolov received fresh appointments. Everyone
spoke loudly of the field marshal's great weakness and failing
health.

His health had to be bad for his place to be taken away and given
to another. And in fact his health was poor.

So naturally, simply, and gradually---just as he had come from
Turkey to the Treasury in Petersburg to recruit the militia, and
then to the army when he was needed there---now when his part was
played out, Kutuzov's place was taken by a new and necessary
performer.

The war of 1812, besides its national significance dear to every
Russian heart, was now to assume another, a European,
significance.

The movement of peoples from west to east was to be succeeded by
a movement of peoples from east to west, and for this fresh war
another leader was necessary, having qualities and views
differing from Kutuzov's and animated by different motives.

Alexander I was as necessary for the movement of the peoples from
east to west and for the refixing of national frontiers as
Kutuzov had been for the salvation and glory of Russia.

Kutuzov did not understand what Europe, the balance of power, or
Napoleon meant. He could not understand it. For the
representative of the Russian people, after the enemy had been
destroyed and Russia had been liberated and raised to the summit
of her glory, there was nothing left to do as a Russian. Nothing
remained for the representative of the national war but to die,
and Kutuzov died.

% % % % % % % % % % % % % % % % % % % % % % % % % % % % % % % % %
% % % % % % % % % % % % % % % % % % % % % % % % % % % % % % % % %
% % % % % % % % % % % % % % % % % % % % % % % % % % % % % % % % %
% % % % % % % % % % % % % % % % % % % % % % % % % % % % % % % % %
% % % % % % % % % % % % % % % % % % % % % % % % % % % % % % % % %
% % % % % % % % % % % % % % % % % % % % % % % % % % % % % % % % %
% % % % % % % % % % % % % % % % % % % % % % % % % % % % % % % % %
% % % % % % % % % % % % % % % % % % % % % % % % % % % % % % % % %
% % % % % % % % % % % % % % % % % % % % % % % % % % % % % % % % %
% % % % % % % % % % % % % % % % % % % % % % % % % % % % % % % % %
% % % % % % % % % % % % % % % % % % % % % % % % % % % % % % % % %
% % % % % % % % % % % % % % % % % % % % % % % % % % % % % %

\chapter*{Chapter XII}
\ifaudio 
\marginpar{
\href{http://ia800504.us.archive.org/9/items/war_and_peace_15_1105_librivox/war_and_peace_15_12_tolstoy_64kb.mp3}{Audio}}
\fi

\initial{A}{s} generally happens, Pierre did not feel the full effects of the
physical privation and strain he had suffered as prisoner until
after they were over. After his liberation he reached Orel, and
on the third day there, when preparing to go to Kiev, he fell ill
and was laid up for three months. He had what the doctors termed
\emph{bilious fever}. But despite the fact that the doctors
treated him, bled him, and gave him medicines to drink, he
recovered.

Scarcely any impression was left on Pierre's mind by all that
happened to him from the time of his rescue till his illness. He
remembered only the dull gray weather now rainy and now snowy,
internal physical distress, and pains in his feet and side. He
remembered a general impression of the misfortunes and sufferings
of people and of being worried by the curiosity of officers and
generals who questioned him, he also remembered his difficulty in
procuring a conveyance and horses, and above all he remembered
his incapacity to think and feel all that time.  On the day of
his rescue he had seen the body of Petya Rostov. That same day he
had learned that Prince Andrew, after surviving the battle of
Borodino for more than a month had recently died in the Rostovs'
house at Yaroslavl, and Denisov who told him this news also
mentioned Helene's death, supposing that Pierre had heard of it
long before. All this at the time seemed merely strange to
Pierre: he felt he could not grasp its significance. Just then he
was only anxious to get away as quickly as possible from places
where people were killing one another, to some peaceful refuge
where he could recover himself, rest, and think over all the
strange new facts he had learned; but on reaching Orel he
immediately fell ill. When he came to himself after his illness
he saw in attendance on him two of his servants, Terenty and
Vaska, who had come from Moscow; and also his cousin the eldest
princess, who had been living on his estate at Elets and hearing
of his rescue and illness had come to look after him.

It was only gradually during his convalescence that Pierre lost
the impressions he had become accustomed to during the last few
months and got used to the idea that no one would oblige him to
go anywhere tomorrow, that no one would deprive him of his warm
bed, and that he would be sure to get his dinner, tea, and
supper. But for a long time in his dreams he still saw himself in
the conditions of captivity. In the same way little by little he
came to understand the news he had been told after his rescue,
about the death of Prince Andrew, the death of his wife, and the
destruction of the French.

A joyous feeling of freedom---that complete inalienable freedom
natural to man which he had first experienced at the first halt
outside Moscow---filled Pierre's soul during his
convalescence. He was surprised to find that this inner freedom,
which was independent of external conditions, now had as it were
an additional setting of external liberty. He was alone in a
strange town, without acquaintances. No one demanded anything of
him or sent him anywhere. He had all he wanted: the thought of
his wife which had been a continual torment to him was no longer
there, since she was no more.

``Oh, how good! How splendid!'' said he to himself when a cleanly
laid table was moved up to him with savory beef tea, or when he
lay down for the night on a soft clean bed, or when he remembered
that the French had gone and that his wife was no more. ``Oh, how
good, how splendid!''

And by old habit he asked himself the question: ``Well, and what
then?  What am I going to do?'' And he immediately gave himself
the answer: ``Well, I shall live. Ah, how splendid!''

The very question that had formerly tormented him, the thing he
had continually sought to find---the aim of life---no longer
existed for him now. That search for the aim of life had not
merely disappeared temporarily---he felt that it no longer
existed for him and could not present itself again. And this very
absence of an aim gave him the complete, joyous sense of freedom
which constituted his happiness at this time.

He could not see an aim, for he now had faith---not faith in any
kind of rule, or words, or ideas, but faith in an ever-living,
ever-manifest God. Formerly he had sought Him in aims he set
himself. That search for an aim had been simply a search for God,
and suddenly in his captivity he had learned not by words or
reasoning but by direct feeling what his nurse had told him long
ago: that God is here and everywhere. In his captivity he had
learned that in Karataev God was greater, more infinite and
unfathomable than in the Architect of the Universe recognized by
the Freemasons. He felt like a man who after straining his eyes
to see into the far distance finds what he sought at his very
feet. All his life he had looked over the heads of the men around
him, when he should have merely looked in front of him without
straining his eyes.

In the past he had never been able to find that great inscrutable
infinite something. He had only felt that it must exist somewhere
and had looked for it. In everything near and comprehensible he
had only what was limited, petty, commonplace, and senseless. He
had equipped himself with a mental telescope and looked into
remote space, where petty worldliness hiding itself in misty
distance had seemed to him great and infinite merely because it
was not clearly seen. And such had European life, politics,
Freemasonry, philosophy, and philanthropy seemed to him. But even
then, at moments of weakness as he had accounted them, his mind
had penetrated to those distances and he had there seen the same
pettiness, worldliness, and senselessness. Now, however, he had
learned to see the great, eternal, and infinite in everything,
and therefore---to see it and enjoy its contemplation---he
naturally threw away the telescope through which he had till now
gazed over men's heads, and gladly regarded the ever-changing,
eternally great, unfathomable, and infinite life around him. And
the closer he looked the more tranquil and happy he became. That
dreadful question, ``What for?'' which had formerly destroyed all
his mental edifices, no longer existed for him.  To that
question, ``What for?'' a simple answer was now always ready in
his soul: ``Because there is a God, that God without whose will
not one hair falls from a man's head.''

% % % % % % % % % % % % % % % % % % % % % % % % % % % % % % % % %
% % % % % % % % % % % % % % % % % % % % % % % % % % % % % % % % %
% % % % % % % % % % % % % % % % % % % % % % % % % % % % % % % % %
% % % % % % % % % % % % % % % % % % % % % % % % % % % % % % % % %
% % % % % % % % % % % % % % % % % % % % % % % % % % % % % % % % %
% % % % % % % % % % % % % % % % % % % % % % % % % % % % % % % % %
% % % % % % % % % % % % % % % % % % % % % % % % % % % % % % % % %
% % % % % % % % % % % % % % % % % % % % % % % % % % % % % % % % %
% % % % % % % % % % % % % % % % % % % % % % % % % % % % % % % % %
% % % % % % % % % % % % % % % % % % % % % % % % % % % % % % % % %
% % % % % % % % % % % % % % % % % % % % % % % % % % % % % % % % %
% % % % % % % % % % % % % % % % % % % % % % % % % % % % % %

\chapter*{Chapter XIII}
\ifaudio 
\marginpar{
\href{http://ia800504.us.archive.org/9/items/war_and_peace_15_1105_librivox/war_and_peace_15_13_tolstoy_64kb.mp3}{Audio}}
\fi

\initial{I}{n} external ways Pierre had hardly changed at all. In appearance
he was just what he used to be. As before he was absent-minded
and seemed occupied not with what was before his eyes but with
something special of his own. The difference between his former
and present self was that formerly when he did not grasp what lay
before him or was said to him, he had puckered his forehead
painfully as if vainly seeking to distinguish something at a
distance. At present he still forgot what was said to him and
still did not see what was before his eyes, but he now looked
with a scarcely perceptible and seemingly ironic smile at what
was before him and listened to what was said, though evidently
seeing and hearing something quite different. Formerly he had
appeared to be a kindhearted but unhappy man, and so people had
been inclined to avoid him. Now a smile at the joy of life always
played round his lips, and sympathy for others, shone in his eyes
with a questioning look as to whether they were as contented as
he was, and people felt pleased by his presence.

Previously he had talked a great deal, grew excited when he
talked, and seldom listened; now he was seldom carried away in
conversation and knew how to listen so that people readily told
him their most intimate secrets.

The princess, who had never liked Pierre and had been
particularly hostile to him since she had felt herself under
obligations to him after the old count's death, now after staying
a short time in Orel---where she had come intending to show
Pierre that in spite of his ingratitude she considered it her
duty to nurse him---felt to her surprise and vexation that she
had become fond of him. Pierre did not in any way seek her
approval, he merely studied her with interest. Formerly she had
felt that he regarded her with indifference and irony, and so had
shrunk into herself as she did with others and had shown him only
the combative side of her nature; but now he seemed to be trying
to understand the most intimate places of her heart, and,
mistrustfully at first but afterwards gratefully, she let him see
the hidden, kindly sides of her character.

The most cunning man could not have crept into her confidence
more successfully, evoking memories of the best times of her
youth and showing sympathy with them. Yet Pierre's cunning
consisted simply in finding pleasure in drawing out the human
qualities of the embittered, hard, and (in her own way) proud
princess.

``Yes, he is a very, very kind man when he is not under the
influence of bad people but of people such as myself,'' thought
she.

His servants too---Terenty and Vaska---in their own way noticed
the change that had taken place in Pierre. They considered that
he had become much ``simpler.'' Terenty, when he had helped him
undress and wished him good night, often lingered with his
master's boots in his hands and clothes over his arm, to see
whether he would not start a talk. And Pierre, noticing that
Terenty wanted a chat, generally kept him there.

``Well, tell me... now, how did you get food?'' he would ask.

And Terenty would begin talking of the destruction of Moscow, and
of the old count, and would stand for a long time holding the
clothes and talking, or sometimes listening to Pierre's stories,
and then would go out into the hall with a pleasant sense of
intimacy with his master and affection for him.

The doctor who attended Pierre and visited him every day, though
he considered it his duty as a doctor to pose as a man whose
every moment was of value to suffering humanity, would sit for
hours with Pierre telling him his favorite anecdotes and his
observations on the characters of his patients in general, and
especially of the ladies.

``It's a pleasure to talk to a man like that; he is not like our
provincials,'' he would say.

There were several prisoners from the French army in Orel, and
the doctor brought one of them, a young Italian, to see Pierre.

This officer began visiting Pierre, and the princess used to make
fun of the tenderness the Italian expressed for him.

The Italian seemed happy only when he could come to see Pierre,
talk with him, tell him about his past, his life at home, and his
love, and pour out to him his indignation against the French and
especially against Napoleon.

``If all Russians are in the least like you, it is sacrilege to
fight such a nation,'' he said to Pierre. ``You, who have
suffered so from the French, do not even feel animosity toward
them.''

Pierre had evoked the passionate affection of the Italian merely
by evoking the best side of his nature and taking a pleasure in
so doing.

During the last days of Pierre's stay in Orel his old masonic
acquaintance Count Willarski, who had introduced him to the lodge
in 1807, came to see him. Willarski was married to a Russian
heiress who had a large estate in Orel province, and he occupied
a temporary post in the commissariat department in that town.

Hearing that Bezukhov was in Orel, Willarski, though they had
never been intimate, came to him with the professions of
friendship and intimacy that people who meet in a desert
generally express for one another.  Willarski felt dull in Orel
and was pleased to meet a man of his own circle and, as he
supposed, of similar interests.

But to his surprise Willarski soon noticed that Pierre had lagged
much behind the times, and had sunk, as he expressed it to
himself, into apathy and egotism.

``You are letting yourself go, my dear fellow,'' he said.

But for all that Willarski found it pleasanter now than it had
been formerly to be with Pierre, and came to see him every
day. To Pierre as he looked at and listened to Willarski, it
seemed strange to think that he had been like that himself but a
short time before.

Willarski was a married man with a family, busy with his family
affairs, his wife's affairs, and his official duties. He regarded
all these occupations as hindrances to life, and considered that
they were all contemptible because their aim was the welfare of
himself and his family. Military, administrative, political, and
masonic interests continually absorbed his attention. And Pierre,
without trying to change the other's views and without condemning
him, but with the quiet, joyful, and amused smile now habitual to
him, was interested in this strange though very familiar
phenomenon.

There was a new feature in Pierre's relations with Willarski,
with the princess, with the doctor, and with all the people he
now met, which gained for him the general good will. This was his
acknowledgment of the impossibility of changing a man's
convictions by words, and his recognition of the possibility of
everyone thinking, feeling, and seeing things each from his own
point of view. This legitimate peculiarity of each individual
which used to excite and irritate Pierre now became a basis of
the sympathy he felt for, and the interest he took in, other
people. The difference, and sometimes complete contradiction,
between men's opinions and their lives, and between one man and
another, pleased him and drew from him an amused and gentle
smile.

In practical matters Pierre unexpectedly felt within himself a
center of gravity he had previously lacked. Formerly all
pecuniary questions, especially requests for money to which, as
an extremely wealthy man, he was very exposed, produced in him a
state of hopeless agitation and perplexity. ``To give or not to
give?'' he had asked himself. ``I have it and he needs it. But
someone else needs it still more. Who needs it most? And perhaps
they are both impostors?'' In the old days he had been unable to
find a way out of all these surmises and had given to all who
asked as long as he had anything to give. Formerly he had been in
a similar state of perplexity with regard to every question
concerning his property, when one person advised one thing and
another something else.

Now to his surprise he found that he no longer felt either doubt
or perplexity about these questions. There was now within him a
judge who by some rule unknown to him decided what should or
should not be done.

He was as indifferent as heretofore to money matters, but now he
felt certain of what ought and what ought not to be done. The
first time he had recourse to his new judge was when a French
prisoner, a colonel, came to him and, after talking a great deal
about his exploits, concluded by making what amounted to a demand
that Pierre should give him four thousand francs to send to his
wife and children. Pierre refused without the least difficulty or
effort, and was afterwards surprised how simple and easy had been
what used to appear so insurmountably difficult. At the same time
that he refused the colonel's demand he made up his mind that he
must have recourse to artifice when leaving Orel, to induce the
Italian officer to accept some money of which he was evidently in
need. A further proof to Pierre of his own more settled outlook
on practical matters was furnished by his decision with regard to
his wife's debts and to the rebuilding of his houses in and near
Moscow.

His head steward came to him at Orel and Pierre reckoned up with
him his diminished income. The burning of Moscow had cost him,
according to the head steward's calculation, about two million
rubles.

To console Pierre for these losses the head steward gave him an
estimate showing that despite these losses his income would not
be diminished but would even be increased if he refused to pay
his wife's debts which he was under no obligation to meet, and
did not rebuild his Moscow house and the country house on his
Moscow estate, which had cost him eighty thousand rubles a year
and brought in nothing.

``Yes, of course that's true,'' said Pierre with a cheerful
smile. ``I don't need all that at all. By being ruined I have
become much richer.''

But in January Savelich came from Moscow and gave him an account
of the state of things there, and spoke of the estimate an
architect had made of the cost of rebuilding the town and country
houses, speaking of this as of a settled matter. About the same
time he received letters from Prince Vasili and other Petersburg
acquaintances speaking of his wife's debts. And Pierre decided
that the steward's proposals which had so pleased him were wrong
and that he must go to Petersburg and settle his wife's affairs
and must rebuild in Moscow. Why this was necessary he did not
know, but he knew for certain that it was necessary. His income
would be reduced by three fourths, but he felt it must be done.

Willarski was going to Moscow and they agreed to travel together.

During the whole time of his convalescence in Orel Pierre had
experienced a feeling of joy, freedom, and life; but when during
his journey he found himself in the open world and saw hundreds
of new faces, that feeling was intensified. Throughout his
journey he felt like a schoolboy on holiday. Everyone---the
stagecoach driver, the post-house overseers, the peasants on the
roads and in the villages---had a new significance for him. The
presence and remarks of Willarski who continually deplored the
ignorance and poverty of Russia and its backwardness compared
with Europe only heightened Pierre's pleasure.  Where Willarski
saw deadness Pierre saw an extraordinary strength and
vitality---the strength which in that vast space amid the snows
maintained the life of this original, peculiar, and unique
people. He did not contradict Willarski and even seemed to agree
with him---an apparent agreement being the simplest way to avoid
discussions that could lead to nothing---and he smiled joyfully
as he listened to him.

% % % % % % % % % % % % % % % % % % % % % % % % % % % % % % % % %
% % % % % % % % % % % % % % % % % % % % % % % % % % % % % % % % %
% % % % % % % % % % % % % % % % % % % % % % % % % % % % % % % % %
% % % % % % % % % % % % % % % % % % % % % % % % % % % % % % % % %
% % % % % % % % % % % % % % % % % % % % % % % % % % % % % % % % %
% % % % % % % % % % % % % % % % % % % % % % % % % % % % % % % % %
% % % % % % % % % % % % % % % % % % % % % % % % % % % % % % % % %
% % % % % % % % % % % % % % % % % % % % % % % % % % % % % % % % %
% % % % % % % % % % % % % % % % % % % % % % % % % % % % % % % % %
% % % % % % % % % % % % % % % % % % % % % % % % % % % % % % % % %
% % % % % % % % % % % % % % % % % % % % % % % % % % % % % % % % %
% % % % % % % % % % % % % % % % % % % % % % % % % % % % % %

\chapter*{Chapter XIV}
\ifaudio 
\marginpar{
\href{http://ia800504.us.archive.org/9/items/war_and_peace_15_1105_librivox/war_and_peace_15_14_tolstoy_64kb.mp3}{Audio}}
\fi

\initial{I}{t} would be difficult to explain why and whither ants whose heap
has been destroyed are hurrying: some from the heap dragging bits
of rubbish, larvae, and corpses, others back to the heap, or why
they jostle, overtake one another, and fight, and it would be
equally difficult to explain what caused the Russians after the
departure of the French to throng to the place that had formerly
been Moscow. But when we watch the ants round their ruined heap,
the tenacity, energy, and immense number of the delving insects
prove that despite the destruction of the heap, something
indestructible, which though intangible is the real strength of
the colony, still exists; and similarly, though in Moscow in the
month of October there was no government and no churches,
shrines, riches, or houses---it was still the Moscow it had been
in August. All was destroyed, except something intangible yet
powerful and indestructible.

The motives of those who thronged from all sides to Moscow after
it had been cleared of the enemy were most diverse and personal,
and at first for the most part savage and brutal. One motive only
they all had in common: a desire to get to the place that had
been called Moscow, to apply their activities there.

Within a week Moscow already had fifteen thousand inhabitants, in
a fortnight twenty-five thousand, and so on. By the autumn of
1813 the number, ever increasing and increasing, exceeded what it
had been in 1812.

The first Russians to enter Moscow were the Cossacks of
Wintzingerode's detachment, peasants from the adjacent villages,
and residents who had fled from Moscow and had been hiding in its
vicinity. The Russians who entered Moscow, finding it plundered,
plundered it in their turn. They continued what the French had
begun. Trains of peasant carts came to Moscow to carry off to the
villages what had been abandoned in the ruined houses and the
streets. The Cossacks carried off what they could to their camps,
and the householders seized all they could find in other houses
and moved it to their own, pretending that it was their property.

But the first plunderers were followed by a second and a third
contingent, and with increasing numbers plundering became more
and more difficult and assumed more definite forms.

The French found Moscow abandoned but with all the organizations
of regular life, with diverse branches of commerce and
craftsmanship, with luxury, and governmental and religious
institutions. These forms were lifeless but still existed. There
were bazaars, shops, warehouses, market stalls, granaries---for
the most part still stocked with goods---and there were factories
and workshops, palaces and wealthy houses filled with luxuries,
hospitals, prisons, government offices, churches, and
cathedrals. The longer the French remained the more these forms
of town life perished, until finally all was merged into one
confused, lifeless scene of plunder.

The more the plundering by the French continued, the more both
the wealth of Moscow and the strength of its plunderers was
destroyed. But plundering by the Russians, with which the
reoccupation of the city began, had an opposite effect: the
longer it continued and the greater the number of people taking
part in it the more rapidly was the wealth of the city and its
regular life restored.

Besides the plunderers, very various people, some drawn by
curiosity, some by official duties, some by self-interest---house
owners, clergy, officials of all kinds, tradesmen, artisans, and
peasants---streamed into Moscow as blood flows to the heart.

Within a week the peasants who came with empty carts to carry off
plunder were stopped by the authorities and made to cart the
corpses out of the town. Other peasants, having heard of their
comrades' discomfiture, came to town bringing rye, oats, and hay,
and beat down one another's prices to below what they had been in
former days. Gangs of carpenters hoping for high pay arrived in
Moscow every day, and on all sides logs were being hewn, new
houses built, and old, charred ones repaired. Tradesmen began
trading in booths. Cookshops and taverns were opened in partially
burned houses. The clergy resumed the services in many churches
that had not been burned. Donors contributed Church property that
had been stolen. Government clerks set up their baize-covered
tables and their pigeonholes of documents in small rooms. The
higher authorities and the police organized the distribution of
goods left behind by the French. The owners of houses in which
much property had been left, brought there from other houses,
complained of the injustice of taking everything to the Faceted
Palace in the Kremlin; others insisted that as the French had
gathered things from different houses into this or that house, it
would be unfair to allow its owner to keep all that was found
there. They abused the police and bribed them, made out estimates
at ten times their value for government stores that had perished
in the fire, and demanded relief. And Count Rostopchin wrote
proclamations.

% % % % % % % % % % % % % % % % % % % % % % % % % % % % % % % % %
% % % % % % % % % % % % % % % % % % % % % % % % % % % % % % % % %
% % % % % % % % % % % % % % % % % % % % % % % % % % % % % % % % %
% % % % % % % % % % % % % % % % % % % % % % % % % % % % % % % % %
% % % % % % % % % % % % % % % % % % % % % % % % % % % % % % % % %
% % % % % % % % % % % % % % % % % % % % % % % % % % % % % % % % %
% % % % % % % % % % % % % % % % % % % % % % % % % % % % % % % % %
% % % % % % % % % % % % % % % % % % % % % % % % % % % % % % % % %
% % % % % % % % % % % % % % % % % % % % % % % % % % % % % % % % %
% % % % % % % % % % % % % % % % % % % % % % % % % % % % % % % % %
% % % % % % % % % % % % % % % % % % % % % % % % % % % % % % % % %
% % % % % % % % % % % % % % % % % % % % % % % % % % % % % %

\chapter*{Chapter XV}
\ifaudio 
\marginpar{
\href{http://ia800504.us.archive.org/9/items/war_and_peace_15_1105_librivox/war_and_peace_15_15_tolstoy_64kb.mp3}{Audio}}
\fi

\initial{A}{t} the end of January Pierre went to Moscow and stayed in an
annex of his house which had not been burned. He called on Count
Rostopchin and on some acquaintances who were back in Moscow, and
he intended to leave for Petersburg two days later. Everybody was
celebrating the victory, everything was bubbling with life in the
ruined but reviving city.  Everyone was pleased to see Pierre,
everyone wished to meet him, and everyone questioned him about
what he had seen. Pierre felt particularly well disposed toward
them all, but was now instinctively on his guard for fear of
binding himself in any way. To all questions put to him---whether
important or quite trifling---such as: Where would he live? Was
he going to rebuild? When was he going to Petersburg and would he
mind taking a parcel for someone?---he replied: ``Yes, perhaps,''
or, ``I think so,'' and so on.

He had heard that the Rostovs were at Kostroma but the thought of
Natasha seldom occurred to him. If it did it was only as a
pleasant memory of the distant past. He felt himself not only
free from social obligations but also from that feeling which, it
seemed to him, he had aroused in himself.

On the third day after his arrival he heard from the Drubetskoys
that Princess Mary was in Moscow. The death, sufferings, and last
days of Prince Andrew had often occupied Pierre's thoughts and
now recurred to him with fresh vividness. Having heard at dinner
that Princess Mary was in Moscow and living in her house---which
had not been burned---in Vozdvizhenka Street, he drove that same
evening to see her.

On his way to the house Pierre kept thinking of Prince Andrew, of
their friendship, of his various meetings with him, and
especially of the last one at Borodino.

``Is it possible that he died in the bitter frame of mind he was
then in?  Is it possible that the meaning of life was not
disclosed to him before he died?'' thought Pierre. He recalled
Karataev and his death and involuntarily began to compare these
two men, so different, and yet so similar in that they had both
lived and both died and in the love he felt for both of them.

Pierre drove up to the house of the old prince in a most serious
mood.  The house had escaped the fire; it showed signs of damage
but its general aspect was unchanged. The old footman, who met
Pierre with a stern face as if wishing to make the visitor feel
that the absence of the old prince had not disturbed the order of
things in the house, informed him that the princess had gone to
her own apartments, and that she received on Sundays.

``Announce me. Perhaps she will see me,'' said Pierre.

``Yes, sir,'' said the man. ``Please step into the portrait
gallery.''

A few minutes later the footman returned with Dessalles, who
brought word from the princess that she would be very glad to see
Pierre if he would excuse her want of ceremony and come upstairs
to her apartment.

In a rather low room lit by one candle sat the princess and with
her another person dressed in black. Pierre remembered that the
princess always had lady companions, but who they were and what
they were like he never knew or remembered. ``This must be one of
her companions,'' he thought, glancing at the lady in the black
dress.

The princess rose quickly to meet him and held out her hand.

``Yes,'' she said, looking at his altered face after he had
kissed her hand, ``so this is how we meet again. He spoke of you
even at the very last,'' she went on, turning her eyes from
Pierre to her companion with a shyness that surprised him for an
instant.

``I was so glad to hear of your safety. It was the first piece of
good news we had received for a long time.''

Again the princess glanced round at her companion with even more
uneasiness in her manner and was about to add something, but
Pierre interrupted her.

``Just imagine---I knew nothing about him!'' said he. ``I thought
he had been killed. All I know I heard at second hand from
others. I only know that he fell in with the Rostovs... What a
strange coincidence!''

Pierre spoke rapidly and with animation. He glanced once at the
companion's face, saw her attentive and kindly gaze fixed on him,
and, as often happens when one is talking, felt somehow that this
companion in the black dress was a good, kind, excellent creature
who would not hinder his conversing freely with Princess Mary.

But when he mentioned the Rostovs, Princess Mary's face expressed
still greater embarrassment. She again glanced rapidly from
Pierre's face to that of the lady in the black dress and said:

``Do you really not recognize her?''

Pierre looked again at the companion's pale, delicate face with
its black eyes and peculiar mouth, and something near to him,
long forgotten and more than sweet, looked at him from those
attentive eyes.

``But no, it can't be!'' he thought. ``This stern, thin, pale
face that looks so much older! It cannot be she. It merely
reminds me of her.'' But at that moment Princess Mary said,
``Natasha!'' And with difficulty, effort, and stress, like the
opening of a door grown rusty on its hinges, a smile appeared on
the face with the attentive eyes, and from that opening door came
a breath of fragrance which suffused Pierre with a happiness he
had long forgotten and of which he had not even been
thinking---especially at that moment. It suffused him, seized
him, and enveloped him completely. When she smiled doubt was no
longer possible, it was Natasha and he loved her.

At that moment Pierre involuntarily betrayed to her, to Princess
Mary, and above all to himself, a secret of which he himself had
been unaware.  He flushed joyfully yet with painful distress. He
tried to hide his agitation. But the more he tried to hide it the
more clearly---clearer than any words could have done---did he
betray to himself, to her, and to Princess Mary that he loved
her.

``No, it's only the unexpectedness of it,'' thought Pierre. But
as soon as he tried to continue the conversation he had begun
with Princess Mary he again glanced at Natasha, and a
still-deeper flush suffused his face and a still-stronger
agitation of mingled joy and fear seized his soul. He became
confused in his speech and stopped in the middle of what he was
saying.

Pierre had failed to notice Natasha because he did not at all
expect to see her there, but he had failed to recognize her
because the change in her since he last saw her was immense. She
had grown thin and pale, but that was not what made her
unrecognizable; she was unrecognizable at the moment he entered
because on that face whose eyes had always shone with a
suppressed smile of the joy of life, now when he first entered
and glanced at her there was not the least shadow of a smile:
only her eyes were kindly attentive and sadly interrogative.

Pierre's confusion was not reflected by any confusion on
Natasha's part, but only by the pleasure that just perceptibly
lit up her whole face.

% % % % % % % % % % % % % % % % % % % % % % % % % % % % % % % % %
% % % % % % % % % % % % % % % % % % % % % % % % % % % % % % % % %
% % % % % % % % % % % % % % % % % % % % % % % % % % % % % % % % %
% % % % % % % % % % % % % % % % % % % % % % % % % % % % % % % % %
% % % % % % % % % % % % % % % % % % % % % % % % % % % % % % % % %
% % % % % % % % % % % % % % % % % % % % % % % % % % % % % % % % %
% % % % % % % % % % % % % % % % % % % % % % % % % % % % % % % % %
% % % % % % % % % % % % % % % % % % % % % % % % % % % % % % % % %
% % % % % % % % % % % % % % % % % % % % % % % % % % % % % % % % %
% % % % % % % % % % % % % % % % % % % % % % % % % % % % % % % % %
% % % % % % % % % % % % % % % % % % % % % % % % % % % % % % % % %
% % % % % % % % % % % % % % % % % % % % % % % % % % % % % %

\chapter*{Chapter XVI}
\ifaudio 
\marginpar{
\href{http://ia800504.us.archive.org/9/items/war_and_peace_15_1105_librivox/war_and_peace_15_16_tolstoy_64kb.mp3}{Audio}}
\fi

\initial*{S}{he} has come to stay with me,'' said Princess Mary. ``The count
and countess will be here in a few days. The countess is in a
dreadful state; but it was necessary for Natasha herself to see a
doctor. They insisted on her coming with me.''

``Yes, is there a family free from sorrow now?'' said Pierre,
addressing Natasha. ``You know it happened the very day we were
rescued. I saw him.  What a delightful boy he was!''

Natasha looked at him, and by way of answer to his words her eyes
widened and lit up.

``What can one say or think of as a consolation?'' said
Pierre. ``Nothing!  Why had such a splendid boy, so full of life,
to die?''

``Yes, in these days it would be hard to live without faith...''
remarked Princess Mary.

``Yes, yes, that is really true,'' Pierre hastily interrupted
her.

``Why is it true?'' Natasha asked, looking attentively into
Pierre's eyes.

``How can you ask why?'' said Princess Mary. ``The thought alone
of what awaits...''

Natasha without waiting for Princess Mary to finish again looked
inquiringly at Pierre.

``And because,'' Pierre continued, ``only one who believes that
there is a God ruling us can bear a loss such as hers
and... yours.''

Natasha had already opened her mouth to speak but suddenly
stopped.  Pierre hurriedly turned away from her and again
addressed Princess Mary, asking about his friend's last days.

Pierre's confusion had now almost vanished, but at the same time
he felt that his freedom had also completely gone. He felt that
there was now a judge of his every word and action whose judgment
mattered more to him than that of all the rest of the world. As
he spoke now he was considering what impression his words would
make on Natasha. He did not purposely say things to please her,
but whatever he was saying he regarded from her standpoint.

Princess Mary---reluctantly as is usual in such cases---began
telling of the condition in which she had found Prince
Andrew. But Pierre's face quivering with emotion, his questions
and his eager restless expression, gradually compelled her to go
into details which she feared to recall for her own sake.

``Yes, yes, and so...?'' Pierre kept saying as he leaned toward
her with his whole body and eagerly listened to her story. ``Yes,
yes... so he grew tranquil and softened? With all his soul he had
always sought one thing---to be perfectly good---so he could not
be afraid of death. The faults he had---if he had any---were not
of his making. So he did soften?... What a happy thing that he
saw you again,'' he added, suddenly turning to Natasha and
looking at her with eyes full of tears.

Natasha's face twitched. She frowned and lowered her eyes for a
moment.  She hesitated for an instant whether to speak or not.

``Yes, that was happiness,'' she then said in her quiet voice
with its deep chest notes. ``For me it certainly was happiness.''
She paused. ``And he... he... he said he was wishing for it at
the very moment I entered the room...''

Natasha's voice broke. She blushed, pressed her clasped hands on
her knees, and then controlling herself with an evident effort
lifted her head and began to speak rapidly.

``We knew nothing of it when we started from Moscow. I did not
dare to ask about him. Then suddenly Sonya told me he was
traveling with us. I had no idea and could not imagine what state
he was in, all I wanted was to see him and be with him,'' she
said, trembling, and breathing quickly.

And not letting them interrupt her she went on to tell what she
had never yet mentioned to anyone---all she had lived through
during those three weeks of their journey and life at Yaroslavl.

Pierre listened to her with lips parted and eyes fixed upon her
full of tears. As he listened he did not think of Prince Andrew,
nor of death, nor of what she was telling. He listened to her and
felt only pity for her, for what she was suffering now while she
was speaking.

Princess Mary, frowning in her effort to hold back her tears, sat
beside Natasha, and heard for the first time the story of those
last days of her brother's and Natasha's love.

Evidently Natasha needed to tell that painful yet joyful tale.

She spoke, mingling most trifling details with the intimate
secrets of her soul, and it seemed as if she could never
finish. Several times she repeated the same thing twice.

Dessalles' voice was heard outside the door asking whether little
Nicholas might come in to say good night.

``Well, that's all---everything,'' said Natasha.

She got up quickly just as Nicholas entered, almost ran to the
door which was hidden by curtains, struck her head against it,
and rushed from the room with a moan either of pain or sorrow.

Pierre gazed at the door through which she had disappeared and
did not understand why he suddenly felt all alone in the world.

Princess Mary roused him from his abstraction by drawing his
attention to her nephew who had entered the room.

At that moment of emotional tenderness young Nicholas' face,
which resembled his father's, affected Pierre so much that when
he had kissed the boy he got up quickly, took out his
handkerchief, and went to the window. He wished to take leave of
Princess Mary, but she would not let him go.

``No, Natasha and I sometimes don't go to sleep till after two,
so please don't go. I will order supper. Go downstairs, we will
come immediately.''

Before Pierre left the room Princess Mary told him: ``This is the
first time she has talked of him like that.''

% % % % % % % % % % % % % % % % % % % % % % % % % % % % % % % % %
% % % % % % % % % % % % % % % % % % % % % % % % % % % % % % % % %
% % % % % % % % % % % % % % % % % % % % % % % % % % % % % % % % %
% % % % % % % % % % % % % % % % % % % % % % % % % % % % % % % % %
% % % % % % % % % % % % % % % % % % % % % % % % % % % % % % % % %
% % % % % % % % % % % % % % % % % % % % % % % % % % % % % % % % %
% % % % % % % % % % % % % % % % % % % % % % % % % % % % % % % % %
% % % % % % % % % % % % % % % % % % % % % % % % % % % % % % % % %
% % % % % % % % % % % % % % % % % % % % % % % % % % % % % % % % %
% % % % % % % % % % % % % % % % % % % % % % % % % % % % % % % % %
% % % % % % % % % % % % % % % % % % % % % % % % % % % % % % % % %
% % % % % % % % % % % % % % % % % % % % % % % % % % % % % %

\chapter*{Chapter XVII}
\ifaudio 
\marginpar{
\href{http://ia800504.us.archive.org/9/items/war_and_peace_15_1105_librivox/war_and_peace_15_17_tolstoy_64kb.mp3}{Audio}}
\fi

\initial{P}{ierre} was shown into the large, brightly lit dining room; a few
minutes later he heard footsteps and Princess Mary entered with
Natasha. Natasha was calm, though a severe and grave expression
had again settled on her face. They all three of them now
experienced that feeling of awkwardness which usually follows
after a serious and heartfelt talk. It is impossible to go back
to the same conversation, to talk of trifles is awkward, and yet
the desire to speak is there and silence seems like
affectation. They went silently to table. The footmen drew back
the chairs and pushed them up again. Pierre unfolded his cold
table napkin and, resolving to break the silence, looked at
Natasha and at Princess Mary. They had evidently both formed the
same resolution; the eyes of both shone with satisfaction and a
confession that besides sorrow life also has joy.

``Do you take vodka, Count?'' asked Princess Mary, and those
words suddenly banished the shadows of the past. ``Now tell us
about yourself,'' said she. ``One hears such improbable wonders
about you.''

``Yes,'' replied Pierre with the smile of mild irony now habitual
to him.  ``They even tell me wonders I myself never dreamed of!
Mary Abramovna invited me to her house and kept telling me what
had happened, or ought to have happened, to me. Stepan Stepanych
also instructed me how I ought to tell of my experiences. In
general I have noticed that it is very easy to be an interesting
man (I am an interesting man now); people invite me out and tell
me all about myself.''

Natasha smiled and was on the point of speaking.

``We have been told,'' Princess Mary interrupted her, ``that you
lost two millions in Moscow. Is that true?''

``But I am three times as rich as before,'' returned Pierre.

Though the position was now altered by his decision to pay his
wife's debts and to rebuild his houses, Pierre still maintained
that he had become three times as rich as before.

``What I have certainly gained is freedom,'' he began seriously,
but did not continue, noticing that this theme was too egotistic.

``And are you building?''

``Yes. Savelich says I must!''

``Tell me, you did not know of the countess' death when you
decided to remain in Moscow?'' asked Princess Mary and
immediately blushed, noticing that her question, following his
mention of freedom, ascribed to his words a meaning he had
perhaps not intended.

``No,'' answered Pierre, evidently not considering awkward the
meaning Princess Mary had given to his words. ``I heard of it in
Orel and you cannot imagine how it shocked me. We were not an
exemplary couple,'' he added quickly, glancing at Natasha and
noticing on her face curiosity as to how he would speak of his
wife, ``but her death shocked me terribly.  When two people
quarrel they are always both in fault, and one's own guilt
suddenly becomes terribly serious when the other is no longer
alive. And then such a death... without friends and without
consolation!  I am very, very sorry for her,'' he concluded, and
was pleased to notice a look of glad approval on Natasha's face.

``Yes, and so you are once more an eligible bachelor,'' said
Princess Mary.

Pierre suddenly flushed crimson and for a long time tried not to
look at Natasha. When he ventured to glance her way again her
face was cold, stern, and he fancied even contemptuous.

``And did you really see and speak to Napoleon, as we have been
told?''  said Princess Mary.

Pierre laughed.

``No, not once! Everybody seems to imagine that being taken
prisoner means being Napoleon's guest. Not only did I never see
him but I heard nothing about him---I was in much lower
company!''

Supper was over, and Pierre who at first declined to speak about
his captivity was gradually led on to do so.

``But it's true that you remained in Moscow to kill Napoleon?''
Natasha asked with a slight smile. ``I guessed it then when we
met at the Sukharev tower, do you remember?''

Pierre admitted that it was true, and from that was gradually led
by Princess Mary's questions and especially by Natasha's into
giving a detailed account of his adventures.

At first he spoke with the amused and mild irony now customary
with him toward everybody and especially toward himself, but when
he came to describe the horrors and sufferings he had witnessed
he was unconsciously carried away and began speaking with the
suppressed emotion of a man re-experiencing in recollection
strong impressions he has lived through.

Princess Mary with a gentle smile looked now at Pierre and now at
Natasha. In the whole narrative she saw only Pierre and his
goodness.  Natasha, leaning on her elbow, the expression of her
face constantly changing with the narrative, watched Pierre with
an attention that never wandered---evidently herself experiencing
all that he described. Not only her look, but her exclamations
and the brief questions she put, showed Pierre that she
understood just what he wished to convey. It was clear that she
understood not only what he said but also what he wished to, but
could not, express in words. The account Pierre gave of the
incident with the child and the woman for protecting whom he was
arrested was this: ``It was an awful sight---children abandoned,
some in the flames...  One was snatched out before my eyes... and
there were women who had their things snatched off and their
earrings torn out...'' he flushed and grew confused. ``Then a
patrol arrived and all the men---all those who were not looting,
that is---were arrested, and I among them.''

``I am sure you're not telling us everything; I am sure you did
something...'' said Natasha and pausing added, ``something
fine?''

Pierre continued. When he spoke of the execution he wanted to
pass over the horrible details, but Natasha insisted that he
should not omit anything.

Pierre began to tell about Karataev, but paused. By this time he
had risen from the table and was pacing the room, Natasha
following him with her eyes. Then he added:

``No, you can't understand what I learned from that illiterate
man---that simple fellow.''

``Yes, yes, go on!'' said Natasha. ``Where is he?''

``They killed him almost before my eyes.''

And Pierre, his voice trembling continually, went on to tell of
the last days of their retreat, of Karataev's illness and his
death.

He told of his adventures as he had never yet recalled them. He
now, as it were, saw a new meaning in all he had gone
through. Now that he was telling it all to Natasha he experienced
that pleasure which a man has when women listen to him---not
clever women who when listening either try to remember what they
hear to enrich their minds and when opportunity offers to retell
it, or who wish to adopt it to some thought of their own and
promptly contribute their own clever comments prepared in their
little mental workshop---but the pleasure given by real women
gifted with a capacity to select and absorb the very best a man
shows of himself.  Natasha without knowing it was all attention:
she did not lose a word, no single quiver in Pierre's voice, no
look, no twitch of a muscle in his face, nor a single
gesture. She caught the unfinished word in its flight and took it
straight into her open heart, divining the secret meaning of all
Pierre's mental travail.

Princess Mary understood his story and sympathized with him, but
she now saw something else that absorbed all her attention. She
saw the possibility of love and happiness between Natasha and
Pierre, and the first thought of this filled her heart with
gladness.

It was three o'clock in the morning. The footmen came in with sad
and stern faces to change the candles, but no one noticed them.

Pierre finished his story. Natasha continued to look at him
intently with bright, attentive, and animated eyes, as if trying
to understand something more which he had perhaps left
untold. Pierre in shamefaced and happy confusion glanced
occasionally at her, and tried to think what to say next to
introduce a fresh subject. Princess Mary was silent. It occurred
to none of them that it was three o'clock and time to go to bed.

``People speak of misfortunes and sufferings,'' remarked Pierre,
``but if at this moment I were asked: 'Would you rather be what
you were before you were taken prisoner, or go through all this
again?' then for heaven's sake let me again have captivity and
horseflesh! We imagine that when we are thrown out of our usual
ruts all is lost, but it is only then that what is new and good
begins. While there is life there is happiness. There is much,
much before us. I say this to you,'' he added, turning to
Natasha.

``Yes, yes,'' she said, answering something quite different. ``I
too should wish nothing but to relive it all from the
beginning.''

Pierre looked intently at her.

``Yes, and nothing more,'' said Natasha.

``It's not true, not true!'' cried Pierre. ``I am not to blame
for being alive and wishing to live---nor you either.''

Suddenly Natasha bent her head, covered her face with her hands,
and began to cry.

``What is it, Natasha?'' said Princess Mary.

``Nothing, nothing.'' She smiled at Pierre through her
tears. ``Good night!  It is time for bed.''

Pierre rose and took his leave.

Princess Mary and Natasha met as usual in the bedroom. They
talked of what Pierre had told them. Princess Mary did not
express her opinion of Pierre nor did Natasha speak of him.

``Well, good night, Mary!'' said Natasha. ``Do you know, I am
often afraid that by not speaking of him'' (she meant Prince
Andrew) ``for fear of not doing justice to our feelings, we
forget him.''

Princess Mary sighed deeply and thereby acknowledged the justice
of Natasha's remark, but she did not express agreement in words.

``Is it possible to forget?'' said she.

``It did me so much good to tell all about it today. It was hard
and painful, but good, very good!'' said Natasha. ``I am sure he
really loved him. That is why I told him... Was it all right?''
she added, suddenly blushing.

``To tell Pierre? Oh, yes. What a splendid man he is!'' said
Princess Mary.

``Do you know, Mary...'' Natasha suddenly said with a mischievous
smile such as Princess Mary had not seen on her face for a long
time, ``he has somehow grown so clean, smooth, and fresh---as if
he had just come out of a Russian bath; do you understand? Out of
a moral bath. Isn't it true?''

``Yes,'' replied Princess Mary. ``He has greatly improved.''

``With a short coat and his hair cropped; just as if, well, just
as if he had come straight from the bath... Papa used to...''

``I understand why he'' (Prince Andrew) ``liked no one so much as
him,'' said Princess Mary.

``Yes, and yet he is quite different. They say men are friends
when they are quite different. That must be true. Really he is
quite unlike him---in everything.''

``Yes, but he's wonderful.''

``Well, good night,'' said Natasha.

And the same mischievous smile lingered for a long time on her
face as if it had been forgotten there.

% % % % % % % % % % % % % % % % % % % % % % % % % % % % % % % % %
% % % % % % % % % % % % % % % % % % % % % % % % % % % % % % % % %
% % % % % % % % % % % % % % % % % % % % % % % % % % % % % % % % %
% % % % % % % % % % % % % % % % % % % % % % % % % % % % % % % % %
% % % % % % % % % % % % % % % % % % % % % % % % % % % % % % % % %
% % % % % % % % % % % % % % % % % % % % % % % % % % % % % % % % %
% % % % % % % % % % % % % % % % % % % % % % % % % % % % % % % % %
% % % % % % % % % % % % % % % % % % % % % % % % % % % % % % % % %
% % % % % % % % % % % % % % % % % % % % % % % % % % % % % % % % %
% % % % % % % % % % % % % % % % % % % % % % % % % % % % % % % % %
% % % % % % % % % % % % % % % % % % % % % % % % % % % % % % % % %
% % % % % % % % % % % % % % % % % % % % % % % % % % % % % %

\chapter*{Chapter XVIII}
\ifaudio 
\marginpar{
\href{http://ia800504.us.archive.org/9/items/war_and_peace_15_1105_librivox/war_and_peace_15_18_tolstoy_64kb.mp3}{Audio}}
\fi

\initial{I}{t} was a long time before Pierre could fall asleep that night. He
paced up and down his room, now turning his thoughts on a
difficult problem and frowning, now suddenly shrugging his
shoulders and wincing, and now smiling happily.

He was thinking of Prince Andrew, of Natasha, and of their love,
at one moment jealous of her past, then reproaching himself for
that feeling.  It was already six in the morning and he still
paced up and down the room.

``Well, what's to be done if it cannot be avoided? What's to be
done?  Evidently it has to be so,'' said he to himself, and
hastily undressing he got into bed, happy and agitated but free
from hesitation or indecision.

``Strange and impossible as such happiness seems, I must do
everything that she and I may be man and wife,'' he told himself.

A few days previously Pierre had decided to go to Petersburg on
the Friday. When he awoke on the Thursday, Savelich came to ask
him about packing for the journey.

``What, to Petersburg? What is Petersburg? Who is there in
Petersburg?''  he asked involuntarily, though only to
himself. ``Oh, yes, long ago before this happened I did for some
reason mean to go to Petersburg,'' he reflected. ``Why? But
perhaps I shall go. What a good fellow he is and how attentive,
and how he remembers everything,'' he thought, looking at
Savelich's old face, ``and what a pleasant smile he has!''

``Well, Savelich, do you still not wish to accept your freedom?''
Pierre asked him.

``What's the good of freedom to me, your excellency? We lived
under the late count---the kingdom of heaven be his!---and we
have lived under you too, without ever being wronged.''

``And your children?''

``The children will live just the same. With such masters one can
live.''

``But what about my heirs?'' said Pierre. ``Supposing I suddenly
marry...  it might happen,'' he added with an involuntary smile.

``If I may take the liberty, your excellency, it would be a good
thing.''

``How easy he thinks it,'' thought Pierre. ``He doesn't know how
terrible it is and how dangerous. Too soon or too late... it is
terrible!''

``So what are your orders? Are you starting tomorrow?'' asked
Savelich.

``No, I'll put it off for a bit. I'll tell you later. You must
forgive the trouble I have put you to,'' said Pierre, and seeing
Savelich smile, he thought: ``But how strange it is that he
should not know that now there is no Petersburg for me, and that
that must be settled first of all! But probably he knows it well
enough and is only pretending. Shall I have a talk with him and
see what he thinks?'' Pierre reflected. ``No, another time.''

At breakfast Pierre told the princess, his cousin, that he had
been to see Princess Mary the day before and had there
met---``Whom do you think?  Natasha Rostova!''

The princess seemed to see nothing more extraordinary in that
than if he had seen Anna Semenovna.

``Do you know her?'' asked Pierre.

``I have seen the princess,'' she replied. ``I heard that they
were arranging a match for her with young Rostov. It would be a
very good thing for the Rostovs, they are said to be utterly
ruined.''

``No; I mean do you know Natasha Rostova?''

``I heard about that affair of hers at the time. It was a great
pity.''

``No, she either doesn't understand or is pretending,'' thought
Pierre.  ``Better not say anything to her either.''

The princess too had prepared provisions for Pierre's journey.

``How kind they all are,'' thought Pierre. ``What is surprising
is that they should trouble about these things now when it can no
longer be of interest to them. And all for me!''

On the same day the Chief of Police came to Pierre, inviting him
to send a representative to the Faceted Palace to recover things
that were to be returned to their owners that day.

``And this man too,'' thought Pierre, looking into the face of
the Chief of Police. ``What a fine, good-looking officer and how
kind. Fancy bothering about such trifles now! And they actually
say he is not honest and takes bribes. What nonsense! Besides,
why shouldn't he take bribes?  That's the way he was brought up,
and everybody does it. But what a kind, pleasant face and how he
smiles as he looks at me.''

Pierre went to Princess Mary's to dinner.

As he drove through the streets past the houses that had been
burned down, he was surprised by the beauty of those ruins. The
picturesqueness of the chimney stacks and tumble-down walls of
the burned-out quarters of the town, stretching out and
concealing one another, reminded him of the Rhine and the
Colosseum. The cabmen he met and their passengers, the carpenters
cutting the timber for new houses with axes, the women hawkers,
and the shopkeepers, all looked at him with cheerful beaming eyes
that seemed to say: ``Ah, there he is! Let's see what will come
of it!''

At the entrance to Princess Mary's house Pierre felt doubtful
whether he had really been there the night before and really seen
Natasha and talked to her. ``Perhaps I imagined it; perhaps I
shall go in and find no one there.'' But he had hardly entered
the room before he felt her presence with his whole being by the
loss of his sense of freedom. She was in the same black dress
with soft folds and her hair was done the same way as the day
before, yet she was quite different. Had she been like this when
he entered the day before he could not for a moment have failed
to recognize her.

She was as he had known her almost as a child and later on as
Prince Andrew's fiancee. A bright questioning light shone in her
eyes, and on her face was a friendly and strangely roguish
expression.

Pierre dined with them and would have spent the whole evening
there, but Princess Mary was going to vespers and Pierre left the
house with her.

Next day he came early, dined, and stayed the whole
evening. Though Princess Mary and Natasha were evidently glad to
see their visitor and though all Pierre's interest was now
centered in that house, by the evening they had talked over
everything and the conversation passed from one trivial topic to
another and repeatedly broke off. He stayed so long that Princess
Mary and Natasha exchanged glances, evidently wondering when he
would go. Pierre noticed this but could not go. He felt uneasy
and embarrassed, but sat on because he simply could not get up
and take his leave.

Princess Mary, foreseeing no end to this, rose first, and
complaining of a headache began to say good night.

``So you are going to Petersburg tomorrow?'' she asked.

``No, I am not going,'' Pierre replied hastily, in a surprised
tone and as though offended. ``Yes... no... to Petersburg?
Tomorrow---but I won't say good-by yet. I will call round in case
you have any commissions for me,'' said he, standing before
Princess Mary and turning red, but not taking his departure.

Natasha gave him her hand and went out. Princess Mary on the
other hand instead of going away sank into an armchair, and
looked sternly and intently at him with her deep, radiant
eyes. The weariness she had plainly shown before had now quite
passed off. With a deep and long-drawn sigh she seemed to be
prepared for a lengthy talk.

When Natasha left the room Pierre's confusion and awkwardness
immediately vanished and were replaced by eager excitement. He
quickly moved an armchair toward Princess Mary.

``Yes, I wanted to tell you,'' said he, answering her look as if
she had spoken. ``Princess, help me! What am I to do? Can I hope?
Princess, my dear friend, listen! I know it all. I know I am not
worthy of her, I know it's impossible to speak of it now. But I
want to be a brother to her. No, not that, I don't, I can't...''

He paused and rubbed his face and eyes with his hands.

``Well,'' he went on with an evident effort at self-control and
coherence.  ``I don't know when I began to love her, but I have
loved her and her alone all my life, and I love her so that I
cannot imagine life without her. I cannot propose to her at
present, but the thought that perhaps she might someday be my
wife and that I may be missing that possibility... that
possibility... is terrible. Tell me, can I hope?  Tell me what I
am to do, dear princess!'' he added after a pause, and touched
her hand as she did not reply.

``I am thinking of what you have told me,'' answered Princess
Mary. ``This is what I will say. You are right that to speak to
her of love at present...''

Princess Mary stopped. She was going to say that to speak of love
was impossible, but she stopped because she had seen by the
sudden change in Natasha two days before that she would not only
not be hurt if Pierre spoke of his love, but that it was the very
thing she wished for.

``To speak to her now wouldn't do,'' said the princess all the
same.

``But what am I to do?''

``Leave it to me,'' said Princess Mary. ``I know...''

Pierre was looking into Princess Mary's eyes.

``Well?... Well?...'' he said.

``I know that she loves... will love you,'' Princess Mary
corrected herself.

Before her words were out, Pierre had sprung up and with a
frightened expression seized Princess Mary's hand.

``What makes you think so? You think I may hope? You think...?''

``Yes, I think so,'' said Princess Mary with a smile. ``Write to
her parents, and leave it to me. I will tell her when I can. I
wish it to happen and my heart tells me it will.''

``No, it cannot be! How happy I am! But it can't be... How happy
I am!  No, it can't be!'' Pierre kept saying as he kissed
Princess Mary's hands.

``Go to Petersburg, that will be best. And I will write to you,''
she said.

``To Petersburg? Go there? Very well, I'll go. But I may come
again tomorrow?''

Next day Pierre came to say good-by. Natasha was less animated
than she had been the day before; but that day as he looked at
her Pierre sometimes felt as if he was vanishing and that neither
he nor she existed any longer, that nothing existed but
happiness. ``Is it possible?  No, it can't be,'' he told himself
at every look, gesture, and word that filled his soul with joy.

When on saying good-by he took her thin, slender hand, he could
not help holding it a little longer in his own.

``Is it possible that this hand, that face, those eyes, all this
treasure of feminine charm so strange to me now, is it possible
that it will one day be mine forever, as familiar to me as I am
to myself?... No, that's impossible!...''

``Good-bye, Count,'' she said aloud. ``I shall look forward very
much to your return,'' she added in a whisper.

And these simple words, her look, and the expression on her face
which accompanied them, formed for two months the subject of
inexhaustible memories, interpretations, and happy meditations
for Pierre. ``'I shall look forward very much to your return...'
Yes, yes, how did she say it?  Yes, 'I shall look forward very
much to your return.' Oh, how happy I am! What is happening to
me? How happy I am!'' said Pierre to himself.

% % % % % % % % % % % % % % % % % % % % % % % % % % % % % % % % %
% % % % % % % % % % % % % % % % % % % % % % % % % % % % % % % % %
% % % % % % % % % % % % % % % % % % % % % % % % % % % % % % % % %
% % % % % % % % % % % % % % % % % % % % % % % % % % % % % % % % %
% % % % % % % % % % % % % % % % % % % % % % % % % % % % % % % % %
% % % % % % % % % % % % % % % % % % % % % % % % % % % % % % % % %
% % % % % % % % % % % % % % % % % % % % % % % % % % % % % % % % %
% % % % % % % % % % % % % % % % % % % % % % % % % % % % % % % % %
% % % % % % % % % % % % % % % % % % % % % % % % % % % % % % % % %
% % % % % % % % % % % % % % % % % % % % % % % % % % % % % % % % %
% % % % % % % % % % % % % % % % % % % % % % % % % % % % % % % % %
% % % % % % % % % % % % % % % % % % % % % % % % % % % % % %

\chapter*{Chapter XIX}
\ifaudio 
\marginpar{
\href{http://ia800504.us.archive.org/9/items/war_and_peace_15_1105_librivox/war_and_peace_15_19_tolstoy_64kb.mp3}{Audio}}
\fi

\initial{T}{here} was nothing in Pierre's soul now at all like what had
troubled it during his courtship of Helene.

He did not repeat to himself with a sickening feeling of shame
the words he had spoken, or say: ``Oh, why did I not say that?''
and, ``Whatever made me say 'Je vous aime'?'' On the contrary, he
now repeated in imagination every word that he or Natasha had
spoken and pictured every detail of her face and smile, and did
not wish to diminish or add anything, but only to repeat it again
and again. There was now not a shadow of doubt in his mind as to
whether what he had undertaken was right or wrong.  Only one
terrible doubt sometimes crossed his mind: ``Wasn't it all a
dream? Isn't Princess Mary mistaken? Am I not too conceited and
self-confident? I believe all this---and suddenly Princess Mary
will tell her, and she will be sure to smile and say: 'How
strange! He must be deluding himself. Doesn't he know that he is
a man, just a man, while I...? I am something altogether
different and higher.'\ ''

That was the only doubt often troubling Pierre. He did not now
make any plans. The happiness before him appeared so
inconceivable that if only he could attain it, it would be the
end of all things. Everything ended with that.

A joyful, unexpected frenzy, of which he had thought himself
incapable, possessed him. The whole meaning of life---not for him
alone but for the whole world---seemed to him centered in his
love and the possibility of being loved by her. At times
everybody seemed to him to be occupied with one thing only---his
future happiness. Sometimes it seemed to him that other people
were all as pleased as he was himself and merely tried to hide
that pleasure by pretending to be busy with other interests. In
every word and gesture he saw allusions to his happiness. He
often surprised those he met by his significantly happy looks and
smiles which seemed to express a secret understanding between him
and them. And when he realized that people might not be aware of
his happiness, he pitied them with his whole heart and felt a
desire somehow to explain to them that all that occupied them was
a mere frivolous trifle unworthy of attention.

When it was suggested to him that he should enter the civil
service, or when the war or any general political affairs were
discussed on the assumption that everybody's welfare depended on
this or that issue of events, he would listen with a mild and
pitying smile and surprise people by his strange comments. But at
this time he saw everybody---both those who, as he imagined,
understood the real meaning of life (that is, what he was
feeling) and those unfortunates who evidently did not understand
it---in the bright light of the emotion that shone within
himself, and at once without any effort saw in everyone he met
everything that was good and worthy of being loved.

When dealing with the affairs and papers of his dead wife, her
memory aroused in him no feeling but pity that she had not known
the bliss he now knew. Prince Vasili, who having obtained a new
post and some fresh decorations was particularly proud at this
time, seemed to him a pathetic, kindly old man much to be pitied.

Often in afterlife Pierre recalled this period of blissful
insanity. All the views he formed of men and circumstances at
this time remained true for him always. He not only did not
renounce them subsequently, but when he was in doubt or inwardly
at variance, he referred to the views he had held at this time of
his madness and they always proved correct.

``I may have appeared strange and queer then,'' he thought, ``but
I was not so mad as I seemed. On the contrary I was then wiser
and had more insight than at any other time, and understood all
that is worth understanding in life, because... because I was
happy.''

Pierre's insanity consisted in not waiting, as he used to do, to
discover personal attributes which he termed ``good qualities''
in people before loving them; his heart was now overflowing with
love, and by loving people without cause he discovered
indubitable causes for loving them.

% % % % % % % % % % % % % % % % % % % % % % % % % % % % % % % % %
% % % % % % % % % % % % % % % % % % % % % % % % % % % % % % % % %
% % % % % % % % % % % % % % % % % % % % % % % % % % % % % % % % %
% % % % % % % % % % % % % % % % % % % % % % % % % % % % % % % % %
% % % % % % % % % % % % % % % % % % % % % % % % % % % % % % % % %
% % % % % % % % % % % % % % % % % % % % % % % % % % % % % % % % %
% % % % % % % % % % % % % % % % % % % % % % % % % % % % % % % % %
% % % % % % % % % % % % % % % % % % % % % % % % % % % % % % % % %
% % % % % % % % % % % % % % % % % % % % % % % % % % % % % % % % %
% % % % % % % % % % % % % % % % % % % % % % % % % % % % % % % % %
% % % % % % % % % % % % % % % % % % % % % % % % % % % % % % % % %
% % % % % % % % % % % % % % % % % % % % % % % % % % % % % %

\chapter*{Chapter XX}
\ifaudio 
\marginpar{
\href{http://ia800504.us.archive.org/9/items/war_and_peace_15_1105_librivox/war_and_peace_15_20_tolstoy_64kb.mp3}{Audio}}
\fi

\initial{A}{fter} Pierre's departure that first evening, when Natasha had
said to Princess Mary with a gaily mocking smile: ``He looks
just, yes, just as if he had come out of a Russian bath---in a
short coat and with his hair cropped,'' something hidden and
unknown to herself, but irrepressible, awoke in Natasha's soul.

Everything: her face, walk, look, and voice, was suddenly
altered. To her own surprise a power of life and hope of
happiness rose to the surface and demanded satisfaction. From
that evening she seemed to have forgotten all that had happened
to her. She no longer complained of her position, did not say a
word about the past, and no longer feared to make happy plans for
the future. She spoke little of Pierre, but when Princess Mary
mentioned him a long-extinguished light once more kindled in her
eyes and her lips curved with a strange smile.

The change that took place in Natasha at first surprised Princess
Mary; but when she understood its meaning it grieved her. ``Can
she have loved my brother so little as to be able to forget him
so soon?'' she thought when she reflected on the change. But when
she was with Natasha she was not vexed with her and did not
reproach her. The reawakened power of life that had seized
Natasha was so evidently irrepressible and unexpected by her that
in her presence Princess Mary felt that she had no right to
reproach her even in her heart.

Natasha gave herself up so fully and frankly to this new feeling
that she did not try to hide the fact that she was no longer sad,
but bright and cheerful.

When Princess Mary returned to her room after her nocturnal talk
with Pierre, Natasha met her on the threshold.

``He has spoken? Yes? He has spoken?'' she repeated.

And a joyful yet pathetic expression which seemed to beg
forgiveness for her joy settled on Natasha's face.

``I wanted to listen at the door, but I knew you would tell me.''

Understandable and touching as the look with which Natasha gazed
at her seemed to Princess Mary, and sorry as she was to see her
agitation, these words pained her for a moment. She remembered
her brother and his love.

``But what's to be done? She can't help it,'' thought the
princess.

And with a sad and rather stern look she told Natasha all that
Pierre had said. On hearing that he was going to Petersburg
Natasha was astounded.

``To Petersburg!'' she repeated as if unable to understand.

But noticing the grieved expression on Princess Mary's face she
guessed the reason of that sadness and suddenly began to cry.

``Mary,'' said she, ``tell me what I should do! I am afraid of
being bad.  Whatever you tell me, I will do. Tell me...''

``You love him?''

``Yes,'' whispered Natasha.

``Then why are you crying? I am happy for your sake,'' said
Princess Mary, who because of those tears quite forgave Natasha's
joy.

``It won't be just yet---someday. Think what fun it will be when
I am his wife and you marry Nicholas!''

``Natasha, I have asked you not to speak of that. Let us talk
about you.''

They were silent awhile.

``But why go to Petersburg?'' Natasha suddenly asked, and hastily
replied to her own question. ``But no, no, he must... Yes, Mary,
He must...''
