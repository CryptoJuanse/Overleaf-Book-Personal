\part*{Book Two: 1805}

% % % % % % % % % % % % % % % % % % % % % % % % % % % % % % % % %
% % % % % % % % % % % % % % % % % % % % % % % % % % % % % % % % %
% % % % % % % % % % % % % % % % % % % % % % % % % % % % % % % % %
% % % % % % % % % % % % % % % % % % % % % % % % % % % % % % % % %
% % % % % % % % % % % % % % % % % % % % % % % % % % % % % % % % %
% % % % % % % % % % % % % % % % % % % % % % % % % % % % % % % % %
% % % % % % % % % % % % % % % % % % % % % % % % % % % % % % % % %
% % % % % % % % % % % % % % % % % % % % % % % % % % % % % % % % %
% % % % % % % % % % % % % % % % % % % % % % % % % % % % % % % % %
% % % % % % % % % % % % % % % % % % % % % % % % % % % % % % % % %
% % % % % % % % % % % % % % % % % % % % % % % % % % % % % % % % %
% % % % % % % % % % % % % % % % % % % % % % % % % % % % % %

\chapter*{Chapter I}
\ifaudio     \marginpar{
\href{http://ia902608.us.archive.org/23/items/war_and_peace_02_0801_librivox/war_and_peace_02_01_tolstoy_64kb.mp3}{Audio}
} \fi

\lettrine[lines=2, loversize=0.3, lraise=0]{\initfamily I}{n}
October, 1805, a Russian army was occupying the villages and
towns of the Archduchy of Austria, and yet other regiments
freshly arriving from Russia were settling near the fortress of
Braunau and burdening the inhabitants on whom they were
quartered. Braunau was the headquarters of the
commander-in-chief, Kutuzov.

On October 11, 1805, one of the infantry regiments that had just
reached Braunau had halted half a mile from the town, waiting to
be inspected by the commander-in-chief. Despite the un-Russian
appearance of the locality and sur\-roundings---fruit gardens,
stone fences, tiled roofs, and hills in the distance---and
despite the fact that the inhabitants (who gazed with curiosity
at the soldiers) were not Russians, the regiment had just the
appearance of any Russian regiment preparing for an inspection
anywhere in the heart of Russia.

On the evening of the last day's march an order had been received
that the commander-in-chief would inspect the regiment on the
march. Though the words of the order were not clear to the
regimental commander, and the question arose whether the troops
were to be in marching order or not, it was decided at a
consultation between the battalion commanders to present the
regiment in parade order, on the principle that it is always
better to \emph{bow too low than not bow low enough}. So the
soldiers, after a twenty-mile march, were kept mending and
cleaning all night long without closing their eyes, while the
adjutants and company commanders calculated and reckoned, and by
morning the regiment---instead of the straggling, disorderly
crowd it had been on its last march the day before---presented a
well-ordered array of two thousand men each of whom knew his
place and his duty, had every button and every strap in place,
and shone with cleanliness. And not only externally was all in
order, but had it pleased the commander-in-chief to look under
the uniforms he would have found on every man a clean shirt, and
in every knapsack the appointed number of articles, \emph{awl,
soap, and all}, as the soldiers say.  There was only one
circumstance concerning which no one could be at ease. It was the
state of the soldiers' boots. More than half the men's boots were
in holes. But this defect was not due to any fault of the
regimental commander, for in spite of repeated demands boots had
not been issued by the Austrian commissariat, and the regiment
had marched some seven hundred miles.

The commander of the regiment was an elderly, choleric, stout,
and thick-set general with grizzled eyebrows and whiskers, and
wider from chest to back than across the shoulders. He had on a
brand-new uniform showing the creases where it had been folded
and thick gold epaulettes which seemed to stand rather than lie
down on his massive shoulders. He had the air of a man happily
performing one of the most solemn duties of his life. He walked
about in front of the line and at every step pulled himself up,
slightly arching his back. It was plain that the commander
admired his regiment, rejoiced in it, and that his whole mind was
engrossed by it, yet his strut seemed to indicate that, besides
military matters, social interests and the fair sex occupied no
small part of his thoughts.

``Well, Michael Mitrich, sir?'' he said, addressing one of the
battalion commanders who smilingly pressed forward (it was plain
that they both felt happy). ``We had our hands full last
night. However, I think the regiment is not a bad one, eh?''

The battalion commander perceived the jovial irony and laughed.

``It would not be turned off the field even on the Tsaritsin
Meadow.''

``What?'' asked the commander.

At that moment, on the road from the town on which signalers had
been posted, two men appeared on horse back. They were an
aide-de-camp followed by a Cossack.

The aide-de-camp was sent to confirm the order which had not been
clearly worded the day before, namely, that the
commander-in-chief wished to see the regiment just in the state
in which it had been on the march: in their greatcoats, and
packs, and without any preparation whatever.

A member of the Hofkriegsrath from Vienna had come to Kutuzov the
day before with proposals and demands for him to join up with the
army of the Archduke Ferdinand and Mack, and Kutuzov, not
considering this junction advisable, meant, among other arguments
in support of his view, to show the Austrian general the wretched
state in which the troops arrived from Russia. With this object
he intended to meet the regiment; so the worse the condition it
was in, the better pleased the commander-in-chief would
be. Though the aide-de-camp did not know these circumstances, he
nevertheless delivered the definite order that the men should be
in their greatcoats and in marching order, and that the
commander-in-chief would otherwise be dissatisfied. On hearing
this the regimental commander hung his head, silently shrugged
his shoulders, and spread out his arms with a choleric gesture.

``A fine mess we've made of it!'' he remarked.

``There now! Didn't I tell you, Michael Mitrich, that if it was
said 'on the march' it meant in greatcoats?'' said he
reproachfully to the battalion commander. ``Oh, my God!'' he
added, stepping resolutely forward. ``Company commanders!'' he
shouted in a voice accustomed to command. ``Sergeants
major!... How soon will he be here?'' he asked the aide-de-camp
with a respectful politeness evidently relating to the personage
he was referring to.

``In an hour's time, I should say.''

``Shall we have time to change clothes?''

``I don't know, General...''

The regimental commander, going up to the line himself, ordered
the soldiers to change into their greatcoats. The company
commanders ran off to their companies, the sergeants major began
bustling (the greatcoats were not in very good condition), and
instantly the squares that had up to then been in regular order
and silent began to sway and stretch and hum with voices. On all
sides soldiers were running to and fro, throwing up their
knapsacks with a jerk of their shoulders and pulling the straps
over their heads, unstrapping their overcoats and drawing the
sleeves on with upraised arms.

In half an hour all was again in order, only the squares had
become gray instead of black. The regimental commander walked
with his jerky steps to the front of the regiment and examined it
from a distance.

``Whatever is this? This!'' he shouted and stood
still. ``Commander of the third company!''

``Commander of the third company wanted by the
general!... commander to the general... third company to the
commander.'' The words passed along the lines and an adjutant ran
to look for the missing officer.

When the eager but misrepeated words had reached their
destination in a cry of: ``The general to the third company,''
the missing officer appeared from behind his company and, though
he was a middle-aged man and not in the habit of running, trotted
awkwardly stumbling on his toes toward the general. The captain's
face showed the uneasiness of a schoolboy who is told to repeat a
lesson he has not learned. Spots appeared on his nose, the
redness of which was evidently due to intemperance, and his mouth
twitched nervously. The general looked the captain up and down as
he came up panting, slackening his pace as he approached.

``You will soon be dressing your men in petticoats! What is
this?''  shouted the regimental commander, thrusting forward his
jaw and pointing at a soldier in the ranks of the third company
in a greatcoat of bluish cloth, which contrasted with the
others. ``What have you been after? The commander in chief is
expected and you leave your place? Eh? I'll teach you to dress
the men in fancy coats for a parade... Eh...?''

The commander of the company, with his eyes fixed on his
superior, pressed two fingers more and more rigidly to his cap,
as if in this pressure lay his only hope of salvation.

``Well, why don't you speak? Whom have you got there dressed up
as a Hungarian?'' said the commander with an austere gibe.

``Your excellency...''

``Well, your excellency, what? Your excellency! But what about
your excellency?... nobody knows.''

``Your excellency, it's the officer Dolokhov, who has been
reduced to the ranks,'' said the captain softly.

``Well? Has he been degraded into a field marshal, or into a
soldier? If a soldier, he should be dressed in regulation uniform
like the others.''

``Your excellency, you gave him leave yourself, on the march.''

``Gave him leave? Leave? That's just like you young men,'' said
the regimental commander cooling down a little. ``Leave
indeed... One says a word to you and you... What?'' he added with
renewed irritation, ``I beg you to dress your men decently.''

And the commander, turning to look at the adjutant, directed his
jerky steps down the line. He was evidently pleased at his own
display of anger and walking up to the regiment wished to find a
further excuse for wrath. Having snapped at an officer for an
unpolished badge, at another because his line was not straight,
he reached the third company.

``H-o-o-w are you standing? Where's your leg? Your leg?'' shouted
the commander with a tone of suffering in his voice, while there
were still five men between him and Dolokhov with his bluish-gray
uniform.

Dolokhov slowly straightened his bent knee, looking straight with
his clear, insolent eyes in the general's face.

``Why a blue coat? Off with it... Sergeant major! Change his
coat... the ras...'' he did not finish.

``General, I must obey orders, but I am not bound to endure...''
Dolokhov hurriedly interrupted.

``No talking in the ranks!... No talking, no talking!''

``Not bound to endure insults,'' Dolokhov concluded in loud,
ringing tones.

The eyes of the general and the soldier met. The general became
silent, angrily pulling down his tight scarf.

``I request you to have the goodness to change your coat,'' he
said as he turned away.

% % % % % % % % % % % % % % % % % % % % % % % % % % % % % % % % %
% % % % % % % % % % % % % % % % % % % % % % % % % % % % % % % % %
% % % % % % % % % % % % % % % % % % % % % % % % % % % % % % % % %
% % % % % % % % % % % % % % % % % % % % % % % % % % % % % % % % %
% % % % % % % % % % % % % % % % % % % % % % % % % % % % % % % % %
% % % % % % % % % % % % % % % % % % % % % % % % % % % % % % % % %
% % % % % % % % % % % % % % % % % % % % % % % % % % % % % % % % %
% % % % % % % % % % % % % % % % % % % % % % % % % % % % % % % % %
% % % % % % % % % % % % % % % % % % % % % % % % % % % % % % % % %
% % % % % % % % % % % % % % % % % % % % % % % % % % % % % % % % %
% % % % % % % % % % % % % % % % % % % % % % % % % % % % % % % % %
% % % % % % % % % % % % % % % % % % % % % % % % % % % % % %

\chapter*{Chapter II}
\ifaudio     \marginpar{
\href{http://ia902608.us.archive.org/23/items/war_and_peace_02_0801_librivox/war_and_peace_02_02_tolstoy_64kb.mp3}{Audio}
} \fi

\lettrine[lines=1, loversize=0.3, lraise=0]{``\initfamily H}{e}'s coming!'' shouted the signaler at that moment.

The regimental commander, flushing, ran to his horse, seized the
stirrup with trembling hands, threw his body across the saddle,
righted himself, drew his saber, and with a happy and resolute
countenance, opening his mouth awry, prepared to shout. The
regiment fluttered like a bird preening its plumage and became
motionless.

``Att-ention!'' shouted the regimental commander in a
soul-shaking voice which expressed joy for himself, severity for
the regiment, and welcome for the approaching chief.

Along the broad country road, edged on both sides by trees, came
a high, light blue Viennese caleche, slightly creaking on its
springs and drawn by six horses at a smart trot. Behind the
caleche galloped the suite and a convoy of Croats. Beside Kutuzov
sat an Austrian general, in a white uniform that looked strange
among the Russian black ones. The caleche stopped in front of the
regiment. Kutuzov and the Austrian general were talking in low
voices and Kutuzov smiled slightly as treading heavily he stepped
down from the carriage just as if those two thousand men
breathlessly gazing at him and the regimental commander did not
exist.

The word of command rang out, and again the regiment quivered, as
with a jingling sound it presented arms. Then amidst a dead
silence the feeble voice of the commander-in-chief was heard. The
regiment roared, ``Health to your ex... len... len... lency!''
and again all became silent. At first Kutuzov stood still while
the regiment moved; then he and the general in white, accompanied
by the suite, walked between the ranks.

From the way the regimental commander saluted the
commander-in-chief and devoured him with his eyes, drawing
himself up obsequiously, and from the way he walked through the
ranks behind the generals, bending forward and hardly able to
restrain his jerky movements, and from the way he darted forward
at every word or gesture of the commander-in-chief, it was
evident that he performed his duty as a subordinate with even
greater zeal than his duty as a commander. Thanks to the
strictness and assiduity of its commander the regiment, in
comparison with others that had reached Braunau at the same time,
was in splendid condition. There were only 217 sick and
stragglers. Everything was in good order except the boots.

Kutuzov walked through the ranks, sometimes stopping to say a few
friendly words to officers he had known in the Turkish war,
sometimes also to the soldiers. Looking at their boots he several
times shook his head sadly, pointing them out to the Austrian
general with an expression which seemed to say that he was not
blaming anyone, but could not help noticing what a bad state of
things it was. The regimental commander ran forward on each such
occasion, fearing to miss a single word of the
commander-in-chief's regarding the regiment. Behind Kutuzov, at a
distance that allowed every softly spoken word to be heard,
followed some twenty men of his suite. These gentlemen talked
among themselves and sometimes laughed. Nearest of all to the
commander-in-chief walked a handsome adjutant. This was Prince
Bolkonski. Beside him was his comrade Nesvitski, a tall staff
officer, extremely stout, with a kindly, smiling, handsome face
and moist eyes. Nesvitski could hardly keep from laughter
provoked by a swarthy hussar officer who walked beside him.  This
hussar, with a grave face and without a smile or a change in the
expression of his fixed eyes, watched the regimental commander's
back and mimicked his every movement. Each time the commander
started and bent forward, the hussar started and bent forward in
exactly the same manner. Nesvitski laughed and nudged the others
to make them look at the wag.

Kutuzov walked slowly and languidly past thousands of eyes which
were starting from their sockets to watch their chief. On
reaching the third company he suddenly stopped. His suite, not
having expected this, involuntarily came closer to him.

``Ah, Timokhin!'' said he, recognizing the red-nosed captain who
had been reprimanded on account of the blue greatcoat.

One would have thought it impossible for a man to stretch himself
more than Timokhin had done when he was reprimanded by the
regimental commander, but now that the commander-in-chief
addressed him he drew himself up to such an extent that it seemed
he could not have sustained it had the commander-in-chief
continued to look at him, and so Kutuzov, who evidently
understood his case and wished him nothing but good, quickly
turned away, a scarcely perceptible smile flitting over his
scarred and puffy face.

``Another Ismail comrade,'' said he. ``A brave officer! Are you
satisfied with him?'' he asked the regimental commander.

And the latter---unconscious that he was being reflected in the
hussar officer as in a looking glass---started, moved forward,
and answered: ``Highly satisfied, your excellency!''

``We all have our weaknesses,'' said Kutuzov smiling and walking
away from him. ``He used to have a predilection for Bacchus.''

The regimental commander was afraid he might be blamed for this
and did not answer. The hussar at that moment noticed the face of
the red-nosed captain and his drawn-in stomach, and mimicked his
expression and pose with such exactitude that Nesvitski could not
help laughing. Kutuzov turned round. The officer evidently had
complete control of his face, and while Kutuzov was turning
managed to make a grimace and then assume a most serious,
deferential, and innocent expression.

The third company was the last, and Kutuzov pondered, apparently
trying to recollect something. Prince Andrew stepped forward from
among the suite and said in French:

``You told me to remind you of the officer Dolokhov, reduced to
the ranks in this regiment.''

``Where is Dolokhov?'' asked Kutuzov.

Dolokhov, who had already changed into a soldier's gray
greatcoat, did not wait to be called. The shapely figure of the
fair-haired soldier, with his clear blue eyes, stepped forward
from the ranks, went up to the commander in chief, and presented
arms.

``Have you a complaint to make?'' Kutuzov asked with a slight
frown.

``This is Dolokhov,'' said Prince Andrew.

``Ah!'' said Kutuzov. ``I hope this will be a lesson to you. Do
your duty.  The Emperor is gracious, and I shan't forget you if
you deserve well.''

The clear blue eyes looked at the commander-in-chief just as
boldly as they had looked at the regimental commander, seeming by
their expression to tear open the veil of convention that
separates a commander-in-chief so widely from a private.

``One thing I ask of your excellency,'' Dolokhov said in his
firm, ringing, deliberate voice. ``I ask an opportunity to atone
for my fault and prove my devotion to His Majesty the Emperor and
to Russia!''

Kutuzov turned away. The same smile of the eyes with which he had
turned from Captain Timokhin again flitted over his face. He
turned away with a grimace as if to say that everything Dolokhov
had said to him and everything he could say had long been known
to him, that he was weary of it and it was not at all what he
wanted. He turned away and went to the carriage.

The regiment broke up into companies, which went to their
appointed quarters near Braunau, where they hoped to receive
boots and clothes and to rest after their hard marches.

``You won't bear me a grudge, Prokhor Ignatych?'' said the
regimental commander, overtaking the third company on its way to
its quarters and riding up to Captain Timokhin who was walking in
front. (The regimental commander's face now that the inspection
was happily over beamed with irrepressible delight.) ``It's in
the Emperor's service... it can't be helped... one is sometimes a
bit hasty on parade... I am the first to apologize, you know
me!... He was very pleased!'' And he held out his hand to the
captain.

``Don't mention it, General, as if I'd be so bold!'' replied the
captain, his nose growing redder as he gave a smile which showed
where two front teeth were missing that had been knocked out by
the butt end of a gun at Ismail.

``And tell Mr. Dolokhov that I won't forget him---he may be quite
easy.  And tell me, please---I've been meaning to ask---how is he
behaving himself, and in general...''

``As far as the service goes he is quite punctilious, your
excellency; but his character...'' said Timokhin.

``And what about his character?'' asked the regimental commander.

``It's different on different days,'' answered the captain. ``One
day he is sensible, well educated, and good-natured, and the next
he's a wild beast... In Poland, if you please, he nearly killed a
Jew.''

``Oh, well, well!'' remarked the regimental commander. ``Still,
one must have pity on a young man in misfortune. You know he has
important connections... Well, then, you just...''

``I will, your excellency,'' said Timokhin, showing by his smile
that he understood his commander's wish.

``Well, of course, of course!''

The regimental commander sought out Dolokhov in the ranks and,
reining in his horse, said to him:

``After the next affair... epaulettes.''

Dolokhov looked round but did not say anything, nor did the
mocking smile on his lips change.

``Well, that's all right,'' continued the regimental
commander. ``A cup of vodka for the men from me,'' he added so
that the soldiers could hear. ``I thank you all! God be
praised!'' and he rode past that company and overtook the next
one.

``Well, he's really a good fellow, one can serve under him,''
said Timokhin to the subaltern beside him.

``In a word, a hearty one...'' said the subaltern, laughing (the
regimental commander was nicknamed King of Hearts).

The cheerful mood of their officers after the inspection infected
the soldiers. The company marched on gaily. The soldiers' voices
could be heard on every side.

``And they said Kutuzov was blind of one eye?''

``And so he is! Quite blind!''

``No, friend, he is sharper-eyed than you are. Boots and leg
bands... he noticed everything...''

``When he looked at my feet, friend... well, thinks I...''

``And that other one with him, the Austrian, looked as if he were
smeared with chalk---as white as flour! I suppose they polish him
up as they do the guns.''

``I say, Fedeshon!... Did he say when the battles are to begin?
You were near him. Everybody said that Buonaparte himself was at
Braunau.''

``Buonaparte himself!... Just listen to the fool, what he doesn't
know!  The Prussians are up in arms now. The Austrians, you see,
are putting them down. When they've been put down, the war with
Buonaparte will begin. And he says Buonaparte is in Braunau!
Shows you're a fool. You'd better listen more carefully!''

``What devils these quartermasters are! See, the fifth company is
turning into the village already... they will have their
buckwheat cooked before we reach our quarters.''

``Give me a biscuit, you devil!''

``And did you give me tobacco yesterday? That's just it, friend!
Ah, well, never mind, here you are.''

``They might call a halt here or we'll have to do another four
miles without eating.''

``Wasn't it fine when those Germans gave us lifts! You just sit
still and are drawn along.''

``And here, friend, the people are quite beggarly. There they all
seemed to be Poles---all under the Russian crown---but here
they're all regular Germans.''

``Singers to the front'' came the captain's order.

And from the different ranks some twenty men ran to the front. A
drummer, their leader, turned round facing the singers, and
flourishing his arm, began a long-drawn-out soldiers' song,
commencing with the words: ``Morning dawned, the sun was
rising,'' and concluding: ``On then, brothers, on to glory, led
by Father Kamenski.'' This song had been composed in the Turkish
campaign and now being sung in Austria, the only change being
that the words ``Father Kamenski'' were replaced by ``Father
Kutuzov.''

Having jerked out these last words as soldiers do and waved his
arms as if flinging something to the ground, the drummer---a
lean, handsome soldier of forty---looked sternly at the singers
and screwed up his eyes.  Then having satisfied himself that all
eyes were fixed on him, he raised both arms as if carefully
lifting some invisible but precious object above his head and,
holding it there for some seconds, suddenly flung it down and
began:

``Oh, my bower, oh, my bower...!''

``Oh, my bower new...!'' chimed in twenty voices, and the
castanet player, in spite of the burden of his equipment, rushed
out to the front and, walking backwards before the company,
jerked his shoulders and flourished his castanets as if
threatening someone. The soldiers, swinging their arms and
keeping time spontaneously, marched with long steps. Behind the
company the sound of wheels, the creaking of springs, and the
tramp of horses' hoofs were heard. Kutuzov and his suite were
returning to the town. The commander-in-chief made a sign that
the men should continue to march at ease, and he and all his
suite showed pleasure at the sound of the singing and the sight
of the dancing soldier and the gay and smartly marching men. In
the second file from the right flank, beside which the carriage
passed the company, a blue-eyed soldier involuntarily attracted
notice. It was Dolokhov marching with particular grace and
boldness in time to the song and looking at those driving past as
if he pitied all who were not at that moment marching with the
company. The hussar cornet of Kutuzov's suite who had mimicked
the regimental commander, fell back from the carriage and rode up
to Dolokhov.

Hussar cornet Zherkov had at one time, in Petersburg, belonged to
the wild set led by Dolokhov. Zherkov had met Dolokhov abroad as
a private and had not seen fit to recognize him. But now that
Kutuzov had spoken to the gentleman ranker, he addressed him with
the cordiality of an old friend.

``My dear fellow, how are you?'' said he through the singing,
making his horse keep pace with the company.

``How am I?'' Dolokhov answered coldly. ``I am as you see.''

The lively song gave a special flavor to the tone of free and
easy gaiety with which Zherkov spoke, and to the intentional
coldness of Dolokhov's reply.

``And how do you get on with the officers?'' inquired Zherkov.

``All right. They are good fellows. And how have you wriggled
onto the staff?''

``I was attached; I'm on duty.''

Both were silent.

``She let the hawk fly upward from her wide right sleeve,'' went
the song, arousing an involuntary sensation of courage and
cheerfulness. Their conversation would probably have been
different but for the effect of that song.

``Is it true that Austrians have been beaten?'' asked Dolokhov.

``The devil only knows! They say so.''

``I'm glad,'' answered Dolokhov briefly and clearly, as the song
demanded.

``I say, come round some evening and we'll have a game of faro!''
said Zherkov.

``Why, have you too much money?''

``Do come.''

``I can't. I've sworn not to. I won't drink and won't play till I
get reinstated.''

``Well, that's only till the first engagement.''

``We shall see.''

They were again silent.

``Come if you need anything. One can at least be of use on the
staff...''

Dolokhov smiled. ``Don't trouble. If I want anything, I won't
beg---I'll take it!''

``Well, never mind; I only...''

``And I only...''

``Good-bye.''

``Good health...''

``It's a long, long way. To my native land...''

Zherkov touched his horse with the spurs; it pranced excitedly
from foot to foot uncertain with which to start, then settled
down, galloped past the company, and overtook the carriage, still
keeping time to the song.

% % % % % % % % % % % % % % % % % % % % % % % % % % % % % % % % %
% % % % % % % % % % % % % % % % % % % % % % % % % % % % % % % % %
% % % % % % % % % % % % % % % % % % % % % % % % % % % % % % % % %
% % % % % % % % % % % % % % % % % % % % % % % % % % % % % % % % %
% % % % % % % % % % % % % % % % % % % % % % % % % % % % % % % % %
% % % % % % % % % % % % % % % % % % % % % % % % % % % % % % % % %
% % % % % % % % % % % % % % % % % % % % % % % % % % % % % % % % %
% % % % % % % % % % % % % % % % % % % % % % % % % % % % % % % % %
% % % % % % % % % % % % % % % % % % % % % % % % % % % % % % % % %
% % % % % % % % % % % % % % % % % % % % % % % % % % % % % % % % %
% % % % % % % % % % % % % % % % % % % % % % % % % % % % % % % % %
% % % % % % % % % % % % % % % % % % % % % % % % % % % % % %

\chapter*{Chapter III}

\ifaudio     \marginpar{
\href{http://ia902608.us.archive.org/23/items/war_and_peace_02_0801_librivox/war_and_peace_02_03_tolstoy_64kb.mp3}{Audio}
} \fi

\lettrine[lines=2, loversize=0.3, lraise=0]{\initfamily O}{n}
returning from the review, Kutuzov took the Austrian general
into his private room and, calling his adjutant, asked for some
papers relating to the condition of the troops on their arrival,
and the letters that had come from the Archduke Ferdinand, who
was in command of the advanced army. Prince Andrew Bolkonski came
into the room with the required papers. Kutuzov and the Austrian
member of the Hofkriegsrath were sitting at the table on which a
plan was spread out.

``Ah!...'' said Kutuzov glancing at Bolkonski as if by this
exclamation he was asking the adjutant to wait, and he went on
with the conversation in French.

``All I can say, General,'' said he with a pleasant elegance of
expression and intonation that obliged one to listen to each
deliberately spoken word. It was evident that Kutuzov himself
listened with pleasure to his own voice. ``All I can say,
General, is that if the matter depended on my personal wishes,
the will of His Majesty the Emperor Francis would have been
fulfilled long ago. I should long ago have joined the
archduke. And believe me on my honour that to me personally it
would be a pleasure to hand over the supreme command of the army
into the hands of a better informed and more skillful
general---of whom Austria has so many---and to lay down all this
heavy responsibility. But circumstances are sometimes too strong
for us, General.''

And Kutuzov smiled in a way that seemed to say, ``You are quite
at liberty not to believe me and I don't even care whether you do
or not, but you have no grounds for telling me so. And that is
the whole point.''

The Austrian general looked dissatisfied, but had no option but
to reply in the same tone.

``On the contrary,'' he said, in a querulous and angry tone that
contrasted with his flattering words, ``on the contrary, your
excellency's participation in the common action is highly valued
by His Majesty; but we think the present delay is depriving the
splendid Russian troops and their commander of the laurels they
have been accustomed to win in their battles,'' he concluded his
evidently prearranged sentence.

Kutuzov bowed with the same smile.

``But that is my conviction, and judging by the last letter with
which His Highness the Archduke Ferdinand has honored me, I
imagine that the Austrian troops, under the direction of so
skillful a leader as General Mack, have by now already gained a
decisive victory and no longer need our aid,'' said Kutuzov.

The general frowned. Though there was no definite news of an
Austrian defeat, there were many circumstances confirming the
unfavorable rumors that were afloat, and so Kutuzov's suggestion
of an Austrian victory sounded much like irony. But Kutuzov went
on blandly smiling with the same expression, which seemed to say
that he had a right to suppose so.  And, in fact, the last letter
he had received from Mack's army informed him of a victory and
stated strategically the position of the army was very favorable.

``Give me that letter,'' said Kutuzov turning to Prince
Andrew. ``Please have a look at it''---and Kutuzov with an
ironical smile about the corners of his mouth read to the
Austrian general the following passage, in German, from the
Archduke Ferdinand's letter:

We have fully concentrated forces of nearly seventy thousand men
with which to attack and defeat the enemy should he cross the
Lech. Also, as we are masters of Ulm, we cannot be deprived of
the advantage of commanding both sides of the Danube, so that
should the enemy not cross the Lech, we can cross the Danube,
throw ourselves on his line of communications, recross the river
lower down, and frustrate his intention should he try to direct
his whole force against our faithful ally. We shall therefore
confidently await the moment when the Imperial Russian army will
be fully equipped, and shall then, in conjunction with it, easily
find a way to prepare for the enemy the fate he deserves.

Kutuzov sighed deeply on finishing this paragraph and looked at
the member of the Hofkriegsrath mildly and attentively.

``But you know the wise maxim your excellency, advising one to
expect the worst,'' said the Austrian general, evidently wishing
to have done with jests and to come to business. He involuntarily
looked round at the aide-de-camp.

``Excuse me, General,'' interrupted Kutuzov, also turning to
Prince Andrew. ``Look here, my dear fellow, get from Kozlovski
all the reports from our scouts. Here are two letters from Count
Nostitz and here is one from His Highness the Archduke Ferdinand
and here are these,'' he said, handing him several papers, ``make
a neat memorandum in French out of all this, showing all the news
we have had of the movements of the Austrian army, and then give
it to his excellency.''

Prince Andrew bowed his head in token of having understood from
the first not only what had been said but also what Kutuzov would
have liked to tell him. He gathered up the papers and with a bow
to both, stepped softly over the carpet and went out into the
waiting room.

Though not much time had passed since Prince Andrew had left
Russia, he had changed greatly during that period. In the
expression of his face, in his movements, in his walk, scarcely a
trace was left of his former affected languor and indolence. He
now looked like a man who has time to think of the impression he
makes on others, but is occupied with agreeable and interesting
work. His face expressed more satisfaction with himself and those
around him, his smile and glance were brighter and more
attractive.

Kutuzov, whom he had overtaken in Poland, had received him very
kindly, promised not to forget him, distinguished him above the
other adjutants, and had taken him to Vienna and given him the
more serious commissions.  From Vienna Kutuzov wrote to his old
comrade, Prince Andrew's father.

Your son bids fair to become an officer distinguished by his
industry, firmness, and expedition. I consider myself fortunate
to have such a subordinate by me.

On Kutuzov's staff, among his fellow officers and in the army
generally, Prince Andrew had, as he had had in Petersburg
society, two quite opposite reputations. Some, a minority,
acknowledged him to be different from themselves and from
everyone else, expected great things of him, listened to him,
admired, and imitated him, and with them Prince Andrew was
natural and pleasant. Others, the majority, disliked him and
considered him conceited, cold, and disagreeable. But among these
people Prince Andrew knew how to take his stand so that they
respected and even feared him.

Coming out of Kutuzov's room into the waiting room with the
papers in his hand Prince Andrew came up to his comrade, the
aide-de-camp on duty, Kozlovski, who was sitting at the window
with a book.

``Well, Prince?'' asked Kozlovski.

``I am ordered to write a memorandum explaining why we are not
advancing.''

``And why is it?''

Prince Andrew shrugged his shoulders.

``Any news from Mack?''

``No.''

``If it were true that he has been beaten, news would have
come.''

``Probably,'' said Prince Andrew moving toward the outer door.

But at that instant a tall Austrian general in a greatcoat, with
the order of Maria Theresa on his neck and a black bandage round
his head, who had evidently just arrived, entered quickly,
slamming the door.  Prince Andrew stopped short.

``Commander in Chief Kutuzov?'' said the newly arrived general
speaking quickly with a harsh German accent, looking to both
sides and advancing straight toward the inner door.

``The commander-in-chief is engaged,'' said Kozlovski, going
hurriedly up to the unknown general and blocking his way to the
door. ``Whom shall I announce?''

The unknown general looked disdainfully down at Kozlovski, who
was rather short, as if surprised that anyone should not know
him.

``The commander-in-chief is engaged,'' repeated Kozlovski calmly.

The general's face clouded, his lips quivered and trembled. He
took out a notebook, hurriedly scribbled something in pencil,
tore out the leaf, gave it to Kozlovski, stepped quickly to the
window, and threw himself into a chair, gazing at those in the
room as if asking, ``Why do they look at me?'' Then he lifted his
head, stretched his neck as if he intended to say something, but
immediately, with affected indifference, began to hum to himself,
producing a queer sound which immediately broke off. The door of
the private room opened and Kutuzov appeared in the doorway. The
general with the bandaged head bent forward as though running
away from some danger, and, making long, quick strides with his
thin legs, went up to Kutuzov.

``Vous voyez le malheureux Mack,'' he uttered in a broken voice.

Kutuzov's face as he stood in the open doorway remained perfectly
immobile for a few moments. Then wrinkles ran over his face like
a wave and his forehead became smooth again, he bowed his head
respectfully, closed his eyes, silently let Mack enter his room
before him, and closed the door himself behind him.

The report which had been circulated that the Austrians had been
beaten and that the whole army had surrendered at Ulm proved to
be correct.  Within half an hour adjutants had been sent in
various directions with orders which showed that the Russian
troops, who had hitherto been inactive, would also soon have to
meet the enemy.

Prince Andrew was one of those rare staff officers whose chief
interest lay in the general progress of the war. When he saw Mack
and heard the details of his disaster he understood that half the
campaign was lost, understood all the difficulties of the Russian
army's position, and vividly imagined what awaited it and the
part he would have to play.  Involuntarily he felt a joyful
agitation at the thought of the humiliation of arrogant Austria
and that in a week's time he might, perhaps, see and take part in
the first Russian encounter with the French since Suvorov met
them. He feared that Bonaparte's genius might outweigh all the
courage of the Russian troops, and at the same time could not
admit the idea of his hero being disgraced.

Excited and irritated by these thoughts Prince Andrew went toward
his room to write to his father, to whom he wrote every day. In
the corridor he met Nesvitski, with whom he shared a room, and
the wag Zherkov; they were as usual laughing.

``Why are you so glum?'' asked Nesvitski noticing Prince Andrew's
pale face and glittering eyes.

``There's nothing to be gay about,'' answered Bolkonski.

Just as Prince Andrew met Nesvitski and Zherkov, there came
toward them from the other end of the corridor, Strauch, an
Austrian general who on Kutuzov's staff in charge of the
provisioning of the Russian army, and the member of the
Hofkriegsrath who had arrived the previous evening.  There was
room enough in the wide corridor for the generals to pass the
three officers quite easily, but Zherkov, pushing Nesvitski aside
with his arm, said in a breathless voice,

``They're coming!... they're coming!... Stand aside, make way,
please make way!''

The generals were passing by, looking as if they wished to avoid
embarrassing attentions. On the face of the wag Zherkov there
suddenly appeared a stupid smile of glee which he seemed unable
to suppress.

``Your excellency,'' said he in German, stepping forward and
addressing the Austrian general, ``I have the honor to
congratulate you.''

He bowed his head and scraped first with one foot and then with
the other, awkwardly, like a child at a dancing lesson.

The member of the Hofkriegsrath looked at him severely but,
seeing the seriousness of his stupid smile, could not but give
him a moment's attention. He screwed up his eyes showing that he
was listening.

``I have the honor to congratulate you. General Mack has arrived,
quite well, only a little bruised just here,'' he added, pointing
with a beaming smile to his head.

The general frowned, turned away, and went on.

``Gott, wie naiv!''\footnote{``Good God, what simplicity!''} said
he angrily, after he had gone a few steps.

Nesvitski with a laugh threw his arms round Prince Andrew, but
Bolkonski, turning still paler, pushed him away with an angry
look and turned to Zherkov. The nervous irritation aroused by the
appearance of Mack, the news of his defeat, and the thought of
what lay before the Russian army found vent in anger at Zherkov's
untimely jest.

``If you, sir, choose to make a buffoon of yourself,'' he said
sharply, with a slight trembling of the lower jaw, ``I can't
prevent your doing so; but I warn you that if you dare to play
the fool in my presence, I will teach you to behave yourself.''

Nesvitski and Zherkov were so surprised by this outburst that
they gazed at Bolkonski silently with wide-open eyes.

``What's the matter? I only congratulated them,'' said Zherkov.

``I am not jesting with you; please be silent!'' cried Bolkonski,
and taking Nesvitski's arm he left Zherkov, who did not know what
to say.

``Come, what's the matter, old fellow?'' said Nesvitski trying to
soothe him.

``What's the matter?'' exclaimed Prince Andrew standing still in
his excitement. ``Don't you understand that either we are
officers serving our Tsar and our country, rejoicing in the
successes and grieving at the misfortunes of our common cause, or
we are merely lackeys who care nothing for their master's
business. Quarante mille hommes massacres et l'armee de nos
allies detruite, et vous trouvez la le mot pour
rire,''\footnote{``Forty thousand men massacred and the army of
our allies destroyed, and you find that a cause for jesting!''}
he said, as if strengthening his views by this French
sentence. ``C'est bien pour un garcon de rien comme cet individu
dont vous avez fait un ami, mais pas pour vous, pas pour
vous.\footnote{``It is all very well for that good-for-nothing
fellow of whom you have made a friend, but not for you, not for
you.''} Only a hobbledehoy could amuse himself in this way,'' he
added in Russian---but pronouncing the word with a French
accent---having noticed that Zherkov could still hear him.

He waited a moment to see whether the cornet would answer, but he
turned and went out of the corridor.

% % % % % % % % % % % % % % % % % % % % % % % % % % % % % % % % %
% % % % % % % % % % % % % % % % % % % % % % % % % % % % % % % % %
% % % % % % % % % % % % % % % % % % % % % % % % % % % % % % % % %
% % % % % % % % % % % % % % % % % % % % % % % % % % % % % % % % %
% % % % % % % % % % % % % % % % % % % % % % % % % % % % % % % % %
% % % % % % % % % % % % % % % % % % % % % % % % % % % % % % % % %
% % % % % % % % % % % % % % % % % % % % % % % % % % % % % % % % %
% % % % % % % % % % % % % % % % % % % % % % % % % % % % % % % % %
% % % % % % % % % % % % % % % % % % % % % % % % % % % % % % % % %
% % % % % % % % % % % % % % % % % % % % % % % % % % % % % % % % %
% % % % % % % % % % % % % % % % % % % % % % % % % % % % % % % % %
% % % % % % % % % % % % % % % % % % % % % % % % % % % % % %

\chapter*{Chapter IV}
\ifaudio     \marginpar{
\href{http://ia902608.us.archive.org/23/items/war_and_peace_02_0801_librivox/war_and_peace_02_03_tolstoy_64kb.mp3}{Audio}
} \fi

\lettrine[lines=2, loversize=0.3, lraise=0]{\initfamily T}{he}
Pavlograd Hussars were stationed two miles from Braunau. The
squadron in which Nicholas Rostov served as a cadet was quartered
in the German village of Salzeneck. The best quarters in the
village were assigned to cavalry-captain Denisov, the squadron
commander, known throughout the whole cavalry division as Vaska
Denisov. Cadet Rostov, ever since he had overtaken the regiment
in Poland, had lived with the squadron commander.

On October 11, the day when all was astir at headquarters over
the news of Mack's defeat, the camp life of the officers of this
squadron was proceeding as usual. Denisov, who had been losing at
cards all night, had not yet come home when Rostov rode back
early in the morning from a foraging expedition. Rostov in his
cadet uniform, with a jerk to his horse, rode up to the porch,
swung his leg over the saddle with a supple youthful movement,
stood for a moment in the stirrup as if loathe to part from his
horse, and at last sprang down and called to his orderly.

``Ah, Bondarenko, dear friend!'' said he to the hussar who rushed
up headlong to the horse. ``Walk him up and down, my dear
fellow,'' he continued, with that gay brotherly cordiality which
goodhearted young people show to everyone when they are happy.

``Yes, your excellency,'' answered the Ukrainian gaily, tossing
his head.

``Mind, walk him up and down well!''

Another hussar also rushed toward the horse, but Bondarenko had
already thrown the reins of the snaffle bridle over the horse's
head. It was evident that the cadet was liberal with his tips and
that it paid to serve him. Rostov patted the horse's neck and
then his flank, and lingered for a moment.

``Splendid! What a horse he will be!'' he thought with a smile,
and holding up his saber, his spurs jingling, he ran up the steps
of the porch. His landlord, who in a waistcoat and a pointed cap,
pitchfork in hand, was clearing manure from the cowhouse, looked
out, and his face immediately brightened on seeing
Rostov. ``Schon gut Morgen! Schon gut Morgen!''\footnote{``A very
good morning! A very good morning!''} he said winking with a
merry smile, evidently pleased to greet the young man.

``Schon fleissig?''\footnote{``Busy already?''} said Rostov with
the same gay brotherly smile which did not leave his eager
face. ``Hoch Oestreicher! Hoch Russen! Kaiser Alexander
hoch!''\footnote{``Hurrah for the Austrians! Hurrah for the
Russians! Hurrah for Emperor Alexander!''} said he, quoting words
often repeated by the German landlord.

The German laughed, came out of the cowshed, pulled off his cap,
and waving it above his head cried:

``Und die ganze Welt hoch!''\footnote{``And hurrah for the whole
world!''}

Rostov waved his cap above his head like the German and cried
laughing, ``Und vivat die ganze Welt!'' Though neither the German
cleaning his cowshed nor Rostov back with his platoon from
foraging for hay had any reason for rejoicing, they looked at
each other with joyful delight and brotherly love, wagged their
heads in token of their mutual affection, and parted smiling, the
German returning to his cowshed and Rostov going to the cottage
he occupied with Denisov.

``What about your master?'' he asked Lavrushka, Denisov's
orderly, whom all the regiment knew for a rogue.

``Hasn't been in since the evening. Must have been losing,''
answered Lavrushka. ``I know by now, if he wins he comes back
early to brag about it, but if he stays out till morning it means
he's lost and will come back in a rage. Will you have coffee?''

``Yes, bring some.''

Ten minutes later Lavrushka brought the coffee. ``He's coming!''
said he.  ``Now for trouble!'' Rostov looked out of the window
and saw Denisov coming home. Denisov was a small man with a red
face, sparkling black eyes, and black tousled mustache and
hair. He wore an unfastened cloak, wide breeches hanging down in
creases, and a crumpled shako on the back of his head. He came up
to the porch gloomily, hanging his head.

``Lavwuska!'' he shouted loudly and angrily, ``take it off,
blockhead!''

``Well, I am taking it off,'' replied Lavrushka's voice.

``Ah, you're up already,'' said Denisov, entering the room.

``Long ago,'' answered Rostov, ``I have already been for the hay,
and have seen Fraulein Mathilde.''

``Weally! And I've been losing, bwother. I lost yesterday like a
damned fool!'' cried Denisov, not pronouncing his r's. ``Such ill
luck! Such ill luck. As soon as you left, it began and went
on. Hullo there! Tea!''

Puckering up his face though smiling, and showing his short
strong teeth, he began with stubby fingers of both hands to
ruffle up his thick tangled black hair.

``And what devil made me go to that wat?'' (an officer nicknamed
``the rat'') he said, rubbing his forehead and whole face with
both hands.  ``Just fancy, he didn't let me win a single cahd,
not one cahd.''

He took the lighted pipe that was offered to him, gripped it in
his fist, and tapped it on the floor, making the sparks fly,
while he continued to shout.

``He lets one win the singles and collahs it as soon as one
doubles it; gives the singles and snatches the doubles!''

He scattered the burning tobacco, smashed the pipe, and threw it
away.  Then he remained silent for a while, and all at once
looked cheerfully with his glittering, black eyes at Rostov.

``If at least we had some women here; but there's nothing foh one
to do but dwink. If we could only get to fighting soon. Hullo,
who's there?''  he said, turning to the door as he heard a tread
of heavy boots and the clinking of spurs that came to a stop, and
a respectful cough.

``The squadron quartermaster!'' said Lavrushka.

Denisov's face puckered still more.

``Wetched!'' he muttered, throwing down a purse with some gold in
it.  ``Wostov, deah fellow, just see how much there is left and
shove the purse undah the pillow,'' he said, and went out to the
quartermaster.

Rostov took the money and, mechanically arranging the old and new
coins in separate piles, began counting them.

``Ah! Telyanin! How d'ye do? They plucked me last night,'' came
Denisov's voice from the next room.

``Where? At Bykov's, at the rat's... I knew it,'' replied a
piping voice, and Lieutenant Telyanin, a small officer of the
same squadron, entered the room.

Rostov thrust the purse under the pillow and shook the damp
little hand which was offered him. Telyanin for some reason had
been transferred from the Guards just before this campaign. He
behaved very well in the regiment but was not liked; Rostov
especially detested him and was unable to overcome or conceal his
groundless antipathy to the man.

``Well, young cavalryman, how is my Rook behaving?'' he
asked. (Rook was a young horse Telyanin had sold to Rostov.)

The lieutenant never looked the man he was speaking to straight
in the face; his eyes continually wandered from one object to
another.

``I saw you riding this morning...'' he added.

``Oh, he's all right, a good horse,'' answered Rostov, though the
horse for which he had paid seven hundred rubles was not worth
half that sum.  ``He's begun to go a little lame on the left
foreleg,'' he added.

``The hoof's cracked! That's nothing. I'll teach you what to do
and show you what kind of rivet to use.''

``Yes, please do,'' said Rostov.

``I'll show you, I'll show you! It's not a secret. And it's a
horse you'll thank me for.''

``Then I'll have it brought round,'' said Rostov wishing to avoid
Telyanin, and he went out to give the order.

In the passage Denisov, with a pipe, was squatting on the
threshold facing the quartermaster who was reporting to him. On
seeing Rostov, Denisov screwed up his face and pointing over his
shoulder with his thumb to the room where Telyanin was sitting,
he frowned and gave a shudder of disgust.

``Ugh! I don't like that fellow,'' he said, regardless of the
quartermaster's presence.

Rostov shrugged his shoulders as much as to say: ``Nor do I, but
what's one to do?'' and, having given his order, he returned to
Telyanin.

Telyanin was sitting in the same indolent pose in which Rostov
had left him, rubbing his small white hands.

``Well there certainly are disgusting people,'' thought Rostov as
he entered.

``Have you told them to bring the horse?'' asked Telyanin,
getting up and looking carelessly about him.

``I have.''

``Let us go ourselves. I only came round to ask Denisov about
yesterday's order. Have you got it, Denisov?''

``Not yet. But where are you off to?''

``I want to teach this young man how to shoe a horse,'' said
Telyanin.

They went through the porch and into the stable. The lieutenant
explained how to rivet the hoof and went away to his own
quarters.

When Rostov went back there was a bottle of vodka and a sausage
on the table. Denisov was sitting there scratching with his pen
on a sheet of paper. He looked gloomily in Rostov's face and
said: ``I am witing to her.''

He leaned his elbows on the table with his pen in his hand and,
evidently glad of a chance to say quicker in words what he wanted
to write, told Rostov the contents of his letter.

``You see, my fwiend,'' he said, ``we sleep when we don't
love. We are childwen of the dust... but one falls in love and
one is a God, one is pua' as on the first day of
cweation... Who's that now? Send him to the devil, I'm busy!'' he
shouted to Lavrushka, who went up to him not in the least
abashed.

``Who should it be? You yourself told him to come. It's the
quartermaster for the money.''

Denisov frowned and was about to shout some reply but stopped.

``Wetched business,'' he muttered to himself. ``How much is left
in the puhse?'' he asked, turning to Rostov.

``Seven new and three old imperials.''

``Oh, it's wetched! Well, what are you standing there for, you
sca'cwow?  Call the quahtehmasteh,'' he shouted to Lavrushka.

``Please, Denisov, let me lend you some: I have some, you know,''
said Rostov, blushing.

``Don't like bowwowing from my own fellows, I don't,'' growled
Denisov.

``But if you won't accept money from me like a comrade, you will
offend me. Really I have some,'' Rostov repeated.

``No, I tell you.''

And Denisov went to the bed to get the purse from under the
pillow.

``Where have you put it, Wostov?''

``Under the lower pillow.''

``It's not there.''

Denisov threw both pillows on the floor. The purse was not there.

``That's a miwacle.''

``Wait, haven't you dropped it?'' said Rostov, picking up the
pillows one at a time and shaking them.

He pulled off the quilt and shook it. The purse was not there.

``Dear me, can I have forgotten? No, I remember thinking that you
kept it under your head like a treasure,'' said Rostov. ``I put
it just here.  Where is it?'' he asked, turning to Lavrushka.

``I haven't been in the room. It must be where you put it.''

``But it isn't?...''

``You're always like that; you thwow a thing down anywhere and
forget it.  Feel in your pockets.''

``No, if I hadn't thought of it being a treasure,'' said Rostov,
``but I remember putting it there.''

Lavrushka turned all the bedding over, looked under the bed and
under the table, searched everywhere, and stood still in the
middle of the room. Denisov silently watched Lavrushka's
movements, and when the latter threw up his arms in surprise
saying it was nowhere to be found Denisov glanced at Rostov.

``Wostov, you've not been playing schoolboy twicks...''

Rostov felt Denisov's gaze fixed on him, raised his eyes, and
instantly dropped them again. All the blood which had seemed
congested somewhere below his throat rushed to his face and
eyes. He could not draw breath.

``And there hasn't been anyone in the room except the lieutenant
and yourselves. It must be here somewhere,'' said Lavrushka.

``Now then, you devil's puppet, look alive and hunt for it!''
shouted Denisov, suddenly, turning purple and rushing at the man
with a threatening gesture. ``If the purse isn't found I'll flog
you, I'll flog you all.''

Rostov, his eyes avoiding Denisov, began buttoning his coat,
buckled on his saber, and put on his cap.

``I must have that purse, I tell you,'' shouted Denisov, shaking
his orderly by the shoulders and knocking him against the wall.

``Denisov, let him alone, I know who has taken it,'' said Rostov,
going toward the door without raising his eyes. Denisov paused,
thought a moment, and, evidently understanding what Rostov hinted
at, seized his arm.

``Nonsense!'' he cried, and the veins on his forehead and neck
stood out like cords. ``You are mad, I tell you. I won't allow
it. The purse is here! I'll flay this scoundwel alive, and it
will be found.''

``I know who has taken it,'' repeated Rostov in an unsteady
voice, and went to the door.

``And I tell you, don't you dahe to do it!'' shouted Denisov,
rushing at the cadet to restrain him.

But Rostov pulled away his arm and, with as much anger as though
Denisov were his worst enemy, firmly fixed his eyes directly on
his face.

``Do you understand what you're saying?'' he said in a trembling
voice.  ``There was no one else in the room except myself. So
that if it is not so, then...''

He could not finish, and ran out of the room.

``Ah, may the devil take you and evewybody,'' were the last words
Rostov heard.

Rostov went to Telyanin's quarters.

``The master is not in, he's gone to headquarters,'' said
Telyanin's orderly. ``Has something happened?'' he added,
surprised at the cadet's troubled face.

``No, nothing.''

``You've only just missed him,'' said the orderly.

The headquarters were situated two miles away from Salzeneck, and
Rostov, without returning home, took a horse and rode
there. There was an inn in the village which the officers
frequented. Rostov rode up to it and saw Telyanin's horse at the
porch.

In the second room of the inn the lieutenant was sitting over a
dish of sausages and a bottle of wine.

``Ah, you've come here too, young man!'' he said, smiling and
raising his eyebrows.

``Yes,'' said Rostov as if it cost him a great deal to utter the
word; and he sat down at the nearest table.

Both were silent. There were two Germans and a Russian officer in
the room. No one spoke and the only sounds heard were the clatter
of knives and the munching of the lieutenant.

When Telyanin had finished his lunch he took out of his pocket a
double purse and, drawing its rings aside with his small, white,
turned-up fingers, drew out a gold imperial, and lifting his
eyebrows gave it to the waiter.

``Please be quick,'' he said.

The coin was a new one. Rostov rose and went up to Telyanin.

``Allow me to look at your purse,'' he said in a low, almost
inaudible, voice.

With shifting eyes but eyebrows still raised, Telyanin handed him
the purse.

``Yes, it's a nice purse. Yes, yes,'' he said, growing suddenly
pale, and added, ``Look at it, young man.''

Rostov took the purse in his hand, examined it and the money in
it, and looked at Telyanin. The lieutenant was looking about in
his usual way and suddenly seemed to grow very merry.

``If we get to Vienna I'll get rid of it there but in these
wretched little towns there's nowhere to spend it,'' said
he. ``Well, let me have it, young man, I'm going.''

Rostov did not speak.

``And you? Are you going to have lunch too? They feed you quite
decently here,'' continued Telyanin. ``Now then, let me have
it.''

He stretched out his hand to take hold of the purse. Rostov let
go of it. Telyanin took the purse and began carelessly slipping
it into the pocket of his riding breeches, with his eyebrows
lifted and his mouth slightly open, as if to say, ``Yes, yes, I
am putting my purse in my pocket and that's quite simple and is
no one else's business.''

``Well, young man?'' he said with a sigh, and from under his
lifted brows he glanced into Rostov's eyes.

Some flash as of an electric spark shot from Telyanin's eyes to
Rostov's and back, and back again and again in an instant.

``Come here,'' said Rostov, catching hold of Telyanin's arm and
almost dragging him to the window. ``That money is Denisov's; you
took it...'' he whispered just above Telyanin's ear.

``What? What? How dare you? What?'' said Telyanin.

But these words came like a piteous, despairing cry and an
entreaty for pardon. As soon as Rostov heard them, an enormous
load of doubt fell from him. He was glad, and at the same instant
began to pity the miserable man who stood before him, but the
task he had begun had to be completed.

``Heaven only knows what the people here may imagine,'' muttered
Telyanin, taking up his cap and moving toward a small empty
room. ``We must have an explanation...''

``I know it and shall prove it,'' said Rostov.

``I...''

Every muscle of Telyanin's pale, terrified face began to quiver,
his eyes still shifted from side to side but with a downward look
not rising to Rostov's face, and his sobs were audible.

``Count!... Don't ruin a young fellow... here is this wretched
money, take it...'' He threw it on the table. ``I have an old
father and mother!...''

Rostov took the money, avoiding Telyanin's eyes, and went out of
the room without a word. But at the door he stopped and then
retraced his steps. ``O God,'' he said with tears in his eyes,
``how could you do it?''

``Count...'' said Telyanin drawing nearer to him.

``Don't touch me,'' said Rostov, drawing back. ``If you need it,
take the money,'' and he threw the purse to him and ran out of
the inn.

% % % % % % % % % % % % % % % % % % % % % % % % % % % % % % % % %
% % % % % % % % % % % % % % % % % % % % % % % % % % % % % % % % %
% % % % % % % % % % % % % % % % % % % % % % % % % % % % % % % % %
% % % % % % % % % % % % % % % % % % % % % % % % % % % % % % % % %
% % % % % % % % % % % % % % % % % % % % % % % % % % % % % % % % %
% % % % % % % % % % % % % % % % % % % % % % % % % % % % % % % % %
% % % % % % % % % % % % % % % % % % % % % % % % % % % % % % % % %
% % % % % % % % % % % % % % % % % % % % % % % % % % % % % % % % %
% % % % % % % % % % % % % % % % % % % % % % % % % % % % % % % % %
% % % % % % % % % % % % % % % % % % % % % % % % % % % % % % % % %
% % % % % % % % % % % % % % % % % % % % % % % % % % % % % % % % %
% % % % % % % % % % % % % % % % % % % % % % % % % % % % % %

\chapter*{Chapter V}
\ifaudio     \marginpar{
\href{http://ia902608.us.archive.org/23/items/war_and_peace_02_0801_librivox/war_and_peace_02_05_tolstoy_64kb.mp3}{Audio}
} \fi

\lettrine[lines=2, loversize=0.3, lraise=0]{\initfamily T}{hat}
same evening there was an animated discussion among the
squadron's officers in Denisov's quarters.

``And I tell you, Rostov, that you must apologize to the
colonel!'' said a tall, grizzly-haired staff captain, with
enormous mustaches and many wrinkles on his large features, to
Rostov who was crimson with excitement.

The staff captain, Kirsten, had twice been reduced to the ranks
for affairs of honor and had twice regained his commission.

``I will allow no one to call me a liar!'' cried Rostov. ``He
told me I lied, and I told him he lied. And there it rests. He
may keep me on duty every day, or may place me under arrest, but
no one can make me apologize, because if he, as commander of this
regiment, thinks it beneath his dignity to give me satisfaction,
then...''

``You just wait a moment, my dear fellow, and listen,''
interrupted the staff captain in his deep bass, calmly stroking
his long mustache. ``You tell the colonel in the presence of
other officers that an officer has stolen...''

``I'm not to blame that the conversation began in the presence of
other officers. Perhaps I ought not to have spoken before them,
but I am not a diplomatist. That's why I joined the hussars,
thinking that here one would not need finesse; and he tells me
that I am lying---so let him give me satisfaction...''

``That's all right. No one thinks you a coward, but that's not
the point.  Ask Denisov whether it is not out of the question for
a cadet to demand satisfaction of his regimental commander?''

Denisov sat gloomily biting his mustache and listening to the
conversation, evidently with no wish to take part in it. He
answered the staff captain's question by a disapproving shake of
his head.

``You speak to the colonel about this nasty business before other
officers,'' continued the staff captain, ``and Bogdanich'' (the
colonel was called Bogdanich) ``shuts you up.''

``He did not shut me up, he said I was telling an untruth.''

``Well, have it so, and you talked a lot of nonsense to him and
must apologize.''

``Not on any account!'' exclaimed Rostov.

``I did not expect this of you,'' said the staff captain
seriously and severely. ``You don't wish to apologize, but, man,
it's not only to him but to the whole regiment---all of
us---you're to blame all round. The case is this: you ought to
have thought the matter over and taken advice; but no, you go and
blurt it all straight out before the officers. Now what was the
colonel to do? Have the officer tried and disgrace the whole
regiment? Disgrace the whole regiment because of one scoundrel?
Is that how you look at it? We don't see it like that. And
Bogdanich was a brick: he told you you were saying what was not
true.  It's not pleasant, but what's to be done, my dear fellow?
You landed yourself in it. And now, when one wants to smooth the
thing over, some conceit prevents your apologizing, and you wish
to make the whole affair public. You are offended at being put on
duty a bit, but why not apologize to an old and honorable
officer? Whatever Bogdanich may be, anyway he is an honorable and
brave old colonel! You're quick at taking offense, but you don't
mind disgracing the whole regiment!'' The staff captain's voice
began to tremble. ``You have been in the regiment next to no
time, my lad, you're here today and tomorrow you'll be appointed
adjutant somewhere and can snap your fingers when it is said
'There are thieves among the Pavlograd officers!' But it's not
all the same to us!  Am I not right, Denisov? It's not the
same!''

Denisov remained silent and did not move, but occasionally looked
with his glittering black eyes at Rostov.

``You value your own pride and don't wish to apologize,''
continued the staff captain, ``but we old fellows, who have grown
up in and, God willing, are going to die in the regiment, we
prize the honor of the regiment, and Bogdanich knows it. Oh, we
do prize it, old fellow! And all this is not right, it's not
right! You may take offense or not but I always stick to mother
truth. It's not right!''

And the staff captain rose and turned away from Rostov.

``That's twue, devil take it!'' shouted Denisov, jumping
up. ``Now then, Wostov, now then!''

Rostov, growing red and pale alternately, looked first at one
officer and then at the other.

``No, gentlemen, no... you mustn't think... I quite
understand. You're wrong to think that of me... I... for
me... for the honor of the regiment I'd... Ah well, I'll show
that in action, and for me the honor of the flag... Well, never
mind, it's true I'm to blame, to blame all round. Well, what else
do you want?...''

``Come, that's right, Count!'' cried the staff captain, turning
round and clapping Rostov on the shoulder with his big hand.

``I tell you,'' shouted Denisov, ``he's a fine fellow.''

``That's better, Count,'' said the staff captain, beginning to
address Rostov by his title, as if in recognition of his
confession. ``Go and apologize, your excellency. Yes, go!''

``Gentlemen, I'll do anything. No one shall hear a word from
me,'' said Rostov in an imploring voice, ``but I can't apologize,
by God I can't, do what you will! How can I go and apologize like
a little boy asking forgiveness?''

Denisov began to laugh.

``It'll be worse for you. Bogdanich is vindictive and you'll pay
for your obstinacy,'' said Kirsten.

``No, on my word it's not obstinacy! I can't describe the
feeling. I can't...''

``Well, it's as you like,'' said the staff captain. ``And what
has become of that scoundrel?'' he asked Denisov.

``He has weported himself sick, he's to be stwuck off the list
tomowwow,'' muttered Denisov.

``It is an illness, there's no other way of explaining it,'' said
the staff captain.

``Illness or not, he'd better not cwoss my path. I'd kill him!''
shouted Denisov in a bloodthirsty tone.

Just then Zherkov entered the room.

``What brings you here?'' cried the officers turning to the
newcomer.

``We're to go into action, gentlemen! Mack has surrendered with
his whole army.''

``It's not true!''

``I've seen him myself!''

``What? Saw the real Mack? With hands and feet?''

``Into action! Into action! Bring him a bottle for such news! But
how did you come here?''

``I've been sent back to the regiment all on account of that
devil, Mack.  An Austrian general complained of me. I
congratulated him on Mack's arrival... What's the matter, Rostov?
You look as if you'd just come out of a hot bath.''

``Oh, my dear fellow, we're in such a stew here these last two
days.''

The regimental adjutant came in and confirmed the news brought by
Zherkov. They were under orders to advance next day.

``We're going into action, gentlemen!''

``Well, thank God! We've been sitting here too long!''

% % % % % % % % % % % % % % % % % % % % % % % % % % % % % % % % %
% % % % % % % % % % % % % % % % % % % % % % % % % % % % % % % % %
% % % % % % % % % % % % % % % % % % % % % % % % % % % % % % % % %
% % % % % % % % % % % % % % % % % % % % % % % % % % % % % % % % %
% % % % % % % % % % % % % % % % % % % % % % % % % % % % % % % % %
% % % % % % % % % % % % % % % % % % % % % % % % % % % % % % % % %
% % % % % % % % % % % % % % % % % % % % % % % % % % % % % % % % %
% % % % % % % % % % % % % % % % % % % % % % % % % % % % % % % % %
% % % % % % % % % % % % % % % % % % % % % % % % % % % % % % % % %
% % % % % % % % % % % % % % % % % % % % % % % % % % % % % % % % %
% % % % % % % % % % % % % % % % % % % % % % % % % % % % % % % % %
% % % % % % % % % % % % % % % % % % % % % % % % % % % % % %

\chapter*{Chapter VI}

\lettrine[lines=2, loversize=0.3, lraise=0]{\initfamily K}{utuzov}
fell back toward Vienna, destroying behind him the
bridges over the rivers Inn (at Braunau) and Traun (near
Linz). On October 23 the Russian troops were crossing the river
Enns. At midday the Russian baggage train, the artillery, and
columns of troops were defiling through the town of Enns on both
sides of the bridge.

It was a warm, rainy, autumnal day. The wide expanse that opened
out before the heights on which the Russian batteries stood
guarding the bridge was at times veiled by a diaphanous curtain
of slanting rain, and then, suddenly spread out in the sunlight,
far-distant objects could be clearly seen glittering as though
freshly varnished. Down below, the little town could be seen with
its white, red-roofed houses, its cathedral, and its bridge, on
both sides of which streamed jostling masses of Russian
troops. At the bend of the Danube, vessels, an island, and a
castle with a park surrounded by the waters of the confluence of
the Enns and the Danube became visible, and the rocky left bank
of the Danube covered with pine forests, with a mystic background
of green treetops and bluish gorges. The turrets of a convent
stood out beyond a wild virgin pine forest, and far away on the
other side of the Enns the enemy's horse patrols could be
discerned.

Among the field guns on the brow of the hill the general in
command of the rearguard stood with a staff officer, scanning the
country through his fieldglass. A little behind them Nesvitski,
who had been sent to the rearguard by the commander-in-chief, was
sitting on the trail of a gun carriage. A Cossack who accompanied
him had handed him a knapsack and a flask, and Nesvitski was
treating some officers to pies and real doppelkummel. The
officers gladly gathered round him, some on their knees, some
squatting Turkish fashion on the wet grass.

``Yes, the Austrian prince who built that castle was no
fool. It's a fine place! Why are you not eating anything,
gentlemen?'' Nesvitski was saying.

``Thank you very much, Prince,'' answered one of the officers,
pleased to be talking to a staff officer of such
importance. ``It's a lovely place!  We passed close to the park
and saw two deer... and what a splendid house!''

``Look, Prince,'' said another, who would have dearly liked to
take another pie but felt shy, and therefore pretended to be
examining the countryside---``See, our infantrymen have already
got there. Look there in the meadow behind the village, three of
them are dragging something.  They'll ransack that castle,'' he
remarked with evident approval.

``So they will,'' said Nesvitski. ``No, but what I should like,''
added he, munching a pie in his moist-lipped handsome mouth,
``would be to slip in over there.''

He pointed with a smile to a turreted nunnery, and his eyes
narrowed and gleamed.

``That would be fine, gentlemen!''

The officers laughed.

``Just to flutter the nuns a bit. They say there are Italian
girls among them. On my word I'd give five years of my life for
it!''

``They must be feeling dull, too,'' said one of the bolder
officers, laughing.

Meanwhile the staff officer standing in front pointed out
something to the general, who looked through his field glass.

``Yes, so it is, so it is,'' said the general angrily, lowering
the field glass and shrugging his shoulders, ``so it is! They'll
be fired on at the crossing. And why are they dawdling there?''

On the opposite side the enemy could be seen by the naked eye,
and from their battery a milk-white cloud arose. Then came the
distant report of a shot, and our troops could be seen hurrying
to the crossing.

Nesvitski rose, puffing, and went up to the general, smiling.

``Would not your excellency like a little refreshment?'' he said.

``It's a bad business,'' said the general without answering him,
``our men have been wasting time.''

``Hadn't I better ride over, your excellency?'' asked Nesvitski.

``Yes, please do,'' answered the general, and he repeated the
order that had already once been given in detail: ``and tell the
hussars that they are to cross last and to fire the bridge as I
ordered; and the inflammable material on the bridge must be
reinspected.''

``Very good,'' answered Nesvitski.

He called the Cossack with his horse, told him to put away the
knapsack and flask, and swung his heavy person easily into the
saddle.

``I'll really call in on the nuns,'' he said to the officers who
watched him smilingly, and he rode off by the winding path down
the hill.

``Now then, let's see how far it will carry, Captain. Just try!''
said the general, turning to an artillery officer. ``Have a
little fun to pass the time.''

``Crew, to your guns!'' commanded the officer.

In a moment the men came running gaily from their campfires and
began loading.

``One!'' came the command.

Number one jumped briskly aside. The gun rang out with a
deafening metallic roar, and a whistling grenade flew above the
heads of our troops below the hill and fell far short of the
enemy, a little smoke showing the spot where it burst.

The faces of officers and men brightened up at the
sound. Everyone got up and began watching the movements of our
troops below, as plainly visible as if but a stone's throw away,
and the movements of the approaching enemy farther off. At the
same instant the sun came fully out from behind the clouds, and
the clear sound of the solitary shot and the brilliance of the
bright sunshine merged in a single joyous and spirited
impression.

% % % % % % % % % % % % % % % % % % % % % % % % % % % % % % % % %
% % % % % % % % % % % % % % % % % % % % % % % % % % % % % % % % %
% % % % % % % % % % % % % % % % % % % % % % % % % % % % % % % % %
% % % % % % % % % % % % % % % % % % % % % % % % % % % % % % % % %
% % % % % % % % % % % % % % % % % % % % % % % % % % % % % % % % %
% % % % % % % % % % % % % % % % % % % % % % % % % % % % % % % % %
% % % % % % % % % % % % % % % % % % % % % % % % % % % % % % % % %
% % % % % % % % % % % % % % % % % % % % % % % % % % % % % % % % %
% % % % % % % % % % % % % % % % % % % % % % % % % % % % % % % % %
% % % % % % % % % % % % % % % % % % % % % % % % % % % % % % % % %
% % % % % % % % % % % % % % % % % % % % % % % % % % % % % % % % %
% % % % % % % % % % % % % % % % % % % % % % % % % % % % % %

\chapter*{Chapter VII}
\ifaudio     \marginpar{
\href{http://ia902608.us.archive.org/23/items/war_and_peace_02_0801_librivox/war_and_peace_02_07_tolstoy_64kb.mp3}{Audio}
} \fi

\lettrine[lines=2, loversize=0.3, lraise=0]{\initfamily T}{wo}
of the enemy's shots had already flown across the bridge,
where there was a crush. Halfway across stood Prince Nesvitski,
who had alighted from his horse and whose big body was jammed
against the railings. He looked back laughing to the Cossack who
stood a few steps behind him holding two horses by their
bridles. Each time Prince Nesvitski tried to move on, soldiers
and carts pushed him back again and pressed him against the
railings, and all he could do was to smile.

``What a fine fellow you are, friend!'' said the Cossack to a
convoy soldier with a wagon, who was pressing onto the
infantrymen who were crowded together close to his wheels and his
horses. ``What a fellow! You can't wait a moment! Don't you see
the general wants to pass?''

But the convoyman took no notice of the word ``general'' and
shouted at the soldiers who were blocking his way. ``Hi there,
boys! Keep to the left! Wait a bit.'' But the soldiers, crowded
together shoulder to shoulder, their bayonets interlocking, moved
over the bridge in a dense mass. Looking down over the rails
Prince Nesvitski saw the rapid, noisy little waves of the Enns,
which rippling and eddying round the piles of the bridge chased
each other along. Looking on the bridge he saw equally uniform
living waves of soldiers, shoulder straps, covered shakos,
knapsacks, bayonets, long muskets, and, under the shakos, faces
with broad cheekbones, sunken cheeks, and listless tired
expressions, and feet that moved through the sticky mud that
covered the planks of the bridge. Sometimes through the
monotonous waves of men, like a fleck of white foam on the waves
of the Enns, an officer, in a cloak and with a type of face
different from that of the men, squeezed his way along; sometimes
like a chip of wood whirling in the river, an hussar on foot, an
orderly, or a townsman was carried through the waves of infantry;
and sometimes like a log floating down the river, an officers' or
company's baggage wagon, piled high, leather covered, and hemmed
in on all sides, moved across the bridge.

``It's as if a dam had burst,'' said the Cossack
hopelessly. ``Are there many more of you to come?''

``A million all but one!'' replied a waggish soldier in a torn
coat, with a wink, and passed on followed by another, an old man.

``If he'' (he meant the enemy) ``begins popping at the bridge
now,'' said the old soldier dismally to a comrade, ``you'll
forget to scratch yourself.''

That soldier passed on, and after him came another sitting on a
cart.

``Where the devil have the leg bands been shoved to?'' said an
orderly, running behind the cart and fumbling in the back of it.

And he also passed on with the wagon. Then came some merry
soldiers who had evidently been drinking.

``And then, old fellow, he gives him one in the teeth with the
butt end of his gun...'' a soldier whose greatcoat was well
tucked up said gaily, with a wide swing of his arm.

``Yes, the ham was just delicious...'' answered another with a
loud laugh.  And they, too, passed on, so that Nesvitski did not
learn who had been struck on the teeth, or what the ham had to do
with it.

``Bah! How they scurry. He just sends a ball and they think
they'll all be killed,'' a sergeant was saying angrily and
reproachfully.

``As it flies past me, Daddy, the ball I mean,'' said a young
soldier with an enormous mouth, hardly refraining from laughing,
``I felt like dying of fright. I did, 'pon my word, I got that
frightened!'' said he, as if bragging of having been frightened.

That one also passed. Then followed a cart unlike any that had
gone before. It was a German cart with a pair of horses led by a
German, and seemed loaded with a whole houseful of effects. A
fine brindled cow with a large udder was attached to the cart
behind. A woman with an unweaned baby, an old woman, and a
healthy German girl with bright red cheeks were sitting on some
feather beds. Evidently these fugitives were allowed to pass by
special permission. The eyes of all the soldiers turned toward
the women, and while the vehicle was passing at foot pace all the
soldiers' remarks related to the two young ones. Every face bore
almost the same smile, expressing unseemly thoughts about the
women.

``Just see, the German sausage is making tracks, too!''

``Sell me the missis,'' said another soldier, addressing the
German, who, angry and frightened, strode energetically along
with downcast eyes.

``See how smart she's made herself! Oh, the devils!''

``There, Fedotov, you should be quartered on them!''

``I have seen as much before now, mate!''

``Where are you going?'' asked an infantry officer who was eating
an apple, also half smiling as he looked at the handsome girl.

The German closed his eyes, signifying that he did not
understand.

``Take it if you like,'' said the officer, giving the girl an
apple.

The girl smiled and took it. Nesvitski like the rest of the men
on the bridge did not take his eyes off the women till they had
passed. When they had gone by, the same stream of soldiers
followed, with the same kind of talk, and at last all stopped. As
often happens, the horses of a convoy wagon became restive at the
end of the bridge, and the whole crowd had to wait.

``And why are they stopping? There's no proper order!'' said the
soldiers.  ``Where are you shoving to? Devil take you! Can't you
wait? It'll be worse if he fires the bridge. See, here's an
officer jammed in too''---different voices were saying in the
crowd, as the men looked at one another, and all pressed toward
the exit from the bridge.

Looking down at the waters of the Enns under the bridge,
Nesvitski suddenly heard a sound new to him, of something swiftly
approaching...  something big, that splashed into the water.

``Just see where it carries to!'' a soldier near by said sternly,
looking round at the sound.

``Encouraging us to get along quicker,'' said another uneasily.

The crowd moved on again. Nesvitski realized that it was a cannon
ball.

``Hey, Cossack, my horse!'' he said. ``Now, then, you there! get
out of the way! Make way!''

With great difficulty he managed to get to his horse, and
shouting continually he moved on. The soldiers squeezed
themselves to make way for him, but again pressed on him so that
they jammed his leg, and those nearest him were not to blame for
they were themselves pressed still harder from behind.

``Nesvitski, Nesvitski! you numskull!'' came a hoarse voice from
behind him.

Nesvitski looked round and saw, some fifteen paces away but
separated by the living mass of moving infantry, Vaska Denisov,
red and shaggy, with his cap on the back of his black head and a
cloak hanging jauntily over his shoulder.

``Tell these devils, these fiends, to let me pass!'' shouted
Denisov evidently in a fit of rage, his coal-black eyes with
their bloodshot whites glittering and rolling as he waved his
sheathed saber in a small bare hand as red as his face.

``Ah, Vaska!'' joyfully replied Nesvitski. ``What's up with
you?''

``The squadwon can't pass,'' shouted Vaska Denisov, showing his
white teeth fiercely and spurring his black thoroughbred Arab,
which twitched its ears as the bayonets touched it, and snorted,
spurting white foam from his bit, tramping the planks of the
bridge with his hoofs, and apparently ready to jump over the
railings had his rider let him. ``What is this? They're like
sheep! Just like sheep! Out of the way!... Let us pass!... Stop
there, you devil with the cart! I'll hack you with my saber!'' he
shouted, actually drawing his saber from its scabbard and
flourishing it.

The soldiers crowded against one another with terrified faces,
and Denisov joined Nesvitski.

``How's it you're not drunk today?'' said Nesvitski when the
other had ridden up to him.

``They don't even give one time to dwink!'' answered Vaska
Denisov. ``They keep dwagging the wegiment to and fwo all day. If
they mean to fight, let's fight. But the devil knows what this
is.''

``What a dandy you are today!'' said Nesvitski, looking at
Denisov's new cloak and saddlecloth.

Denisov smiled, took out of his sabretache a handkerchief that
diffused a smell of perfume, and put it to Nesvitski's nose.

``Of course. I'm going into action! I've shaved, bwushed my
teeth, and scented myself.''

The imposing figure of Nesvitski followed by his Cossack, and the
determination of Denisov who flourished his sword and shouted
frantically, had such an effect that they managed to squeeze
through to the farther side of the bridge and stopped the
infantry. Beside the bridge Nesvitski found the colonel to whom
he had to deliver the order, and having done this he rode back.

Having cleared the way Denisov stopped at the end of the bridge.
Carelessly holding in his stallion that was neighing and pawing
the ground, eager to rejoin its fellows, he watched his squadron
draw nearer. Then the clang of hoofs, as of several horses
galloping, resounded on the planks of the bridge, and the
squadron, officers in front and men four abreast, spread across
the bridge and began to emerge on his side of it.

The infantry who had been stopped crowded near the bridge in the
trampled mud and gazed with that particular feeling of ill-will,
estrangement, and ridicule with which troops of different arms
usually encounter one another at the clean, smart hussars who
moved past them in regular order.

``Smart lads! Only fit for a fair!'' said one.

``What good are they? They're led about just for show!'' remarked
another.

``Don't kick up the dust, you infantry!'' jested an hussar whose
prancing horse had splashed mud over some foot soldiers.

``I'd like to put you on a two days' march with a knapsack! Your
fine cords would soon get a bit rubbed,'' said an infantryman,
wiping the mud off his face with his sleeve. ``Perched up there,
you're more like a bird than a man.''

``There now, Zikin, they ought to put you on a horse. You'd look
fine,'' said a corporal, chaffing a thin little soldier who bent
under the weight of his knapsack.

``Take a stick between your legs, that'll suit you for a horse!''
the hussar shouted back.

% % % % % % % % % % % % % % % % % % % % % % % % % % % % % % % % %
% % % % % % % % % % % % % % % % % % % % % % % % % % % % % % % % %
% % % % % % % % % % % % % % % % % % % % % % % % % % % % % % % % %
% % % % % % % % % % % % % % % % % % % % % % % % % % % % % % % % %
% % % % % % % % % % % % % % % % % % % % % % % % % % % % % % % % %
% % % % % % % % % % % % % % % % % % % % % % % % % % % % % % % % %
% % % % % % % % % % % % % % % % % % % % % % % % % % % % % % % % %
% % % % % % % % % % % % % % % % % % % % % % % % % % % % % % % % %
% % % % % % % % % % % % % % % % % % % % % % % % % % % % % % % % %
% % % % % % % % % % % % % % % % % % % % % % % % % % % % % % % % %
% % % % % % % % % % % % % % % % % % % % % % % % % % % % % % % % %
% % % % % % % % % % % % % % % % % % % % % % % % % % % % % %

\chapter*{Chapter VIII}
\ifaudio     \marginpar{
\href{http://ia902608.us.archive.org/23/items/war_and_peace_02_0801_librivox/war_and_peace_02_08_tolstoy_64kb.mp3}{Audio}
} \fi

\lettrine[lines=2, loversize=0.3, lraise=0]{\initfamily T}{he}
last of the infantry hurriedly crossed the bridge, squeezing
together as they approached it as if passing through a funnel. At
last the baggage wagons had all crossed, the crush was less, and
the last battalion came onto the bridge. Only Denisov's squadron
of hussars remained on the farther side of the bridge facing the
enemy, who could be seen from the hill on the opposite bank but
was not yet visible from the bridge, for the horizon as seen from
the valley through which the river flowed was formed by the
rising ground only half a mile away. At the foot of the hill lay
wasteland over which a few groups of our Cossack scouts were
moving. Suddenly on the road at the top of the high ground,
artillery and troops in blue uniform were seen. These were the
French. A group of Cossack scouts retired down the hill at a
trot. All the officers and men of Denisov's squadron, though they
tried to talk of other things and to look in other directions,
thought only of what was there on the hilltop, and kept
constantly looking at the patches appearing on the skyline, which
they knew to be the enemy's troops. The weather had cleared again
since noon and the sun was descending brightly upon the Danube
and the dark hills around it. It was calm, and at intervals the
bugle calls and the shouts of the enemy could be heard from the
hill. There was no one now between the squadron and the enemy
except a few scattered skirmishers. An empty space of some seven
hundred yards was all that separated them. The enemy ceased
firing, and that stern, threatening, inaccessible, and intangible
line which separates two hostile armies was all the more clearly
felt.

``One step beyond that boundary line which resembles the line
dividing the living from the dead lies uncertainty, suffering,
and death. And what is there? Who is there?---there beyond that
field, that tree, that roof lit up by the sun? No one knows, but
one wants to know. You fear and yet long to cross that line, and
know that sooner or later it must be crossed and you will have to
find out what is there, just as you will inevitably have to learn
what lies the other side of death. But you are strong, healthy,
cheerful, and excited, and are surrounded by other such excitedly
animated and healthy men.'' So thinks, or at any rate feels,
anyone who comes in sight of the enemy, and that feeling gives a
particular glamour and glad keenness of impression to everything
that takes place at such moments.

On the high ground where the enemy was, the smoke of a cannon
rose, and a ball flew whistling over the heads of the hussar
squadron. The officers who had been standing together rode off to
their places. The hussars began carefully aligning their
horses. Silence fell on the whole squadron. All were looking at
the enemy in front and at the squadron commander, awaiting the
word of command. A second and a third cannon ball flew
past. Evidently they were firing at the hussars, but the balls
with rapid rhythmic whistle flew over the heads of the horsemen
and fell somewhere beyond them. The hussars did not look round,
but at the sound of each shot, as at the word of command, the
whole squadron with its rows of faces so alike yet so different,
holding its breath while the ball flew past, rose in the stirrups
and sank back again. The soldiers without turning their heads
glanced at one another, curious to see their comrades'
impression. Every face, from Denisov's to that of the bugler,
showed one common expression of conflict, irritation, and
excitement, around chin and mouth. The quartermaster frowned,
looking at the soldiers as if threatening to punish them. Cadet
Mironov ducked every time a ball flew past. Rostov on the left
flank, mounted on his Rook---a handsome horse despite its game
leg---had the happy air of a schoolboy called up before a large
audience for an examination in which he feels sure he will
distinguish himself. He was glancing at everyone with a clear,
bright expression, as if asking them to notice how calmly he sat
under fire. But despite himself, on his face too that same
indication of something new and stern showed round the mouth.

``Who's that curtseying there? Cadet Miwonov! That's not wight!
Look at me,'' cried Denisov who, unable to keep still on one
spot, kept turning his horse in front of the squadron.

The black, hairy, snub-nosed face of Vaska Denisov, and his whole
short sturdy figure with the sinewy hairy hand and stumpy fingers
in which he held the hilt of his naked saber, looked just as it
usually did, especially toward evening when he had emptied his
second bottle; he was only redder than usual. With his shaggy
head thrown back like birds when they drink, pressing his spurs
mercilessly into the sides of his good horse, Bedouin, and
sitting as though falling backwards in the saddle, he galloped to
the other flank of the squadron and shouted in a hoarse voice to
the men to look to their pistols. He rode up to Kirsten. The
staff captain on his broad-backed, steady mare came at a walk to
meet him. His face with its long mustache was serious as always,
only his eyes were brighter than usual.

``Well, what about it?'' said he to Denisov. ``It won't come to a
fight.  You'll see---we shall retire.''

``The devil only knows what they're about!'' muttered
Denisov. ``Ah, Wostov,'' he cried noticing the cadet's bright
face, ``you've got it at last.''

And he smiled approvingly, evidently pleased with the
cadet. Rostov felt perfectly happy. Just then the commander
appeared on the bridge. Denisov galloped up to him.

``Your excellency! Let us attack them! I'll dwive them off.''

``Attack indeed!'' said the colonel in a bored voice, puckering
up his face as if driving off a troublesome fly. ``And why are
you stopping here? Don't you see the skirmishers are retreating?
Lead the squadron back.''

The squadron crossed the bridge and drew out of range of fire
without having lost a single man. The second squadron that had
been in the front line followed them across and the last Cossacks
quitted the farther side of the river.

The two Pavlograd squadrons, having crossed the bridge, retired
up the hill one after the other. Their colonel, Karl Bogdanich
Schubert, came up to Denisov's squadron and rode at a footpace
not far from Rostov, without taking any notice of him although
they were now meeting for the first time since their encounter
concerning Telyanin. Rostov, feeling that he was at the front and
in the power of a man toward whom he now admitted that he had
been to blame, did not lift his eyes from the colonel's athletic
back, his nape covered with light hair, and his red neck. It
seemed to Rostov that Bogdanich was only pretending not to notice
him, and that his whole aim now was to test the cadet's courage,
so he drew himself up and looked around him merrily; then it
seemed to him that Bogdanich rode so near in order to show him
his courage. Next he thought that his enemy would send the
squadron on a desperate attack just to punish him---Rostov. Then
he imagined how, after the attack, Bogdanich would come up to him
as he lay wounded and would magnanimously extend the hand of
reconciliation.

The high-shouldered figure of Zherkov, familiar to the Pavlograds
as he had but recently left their regiment, rode up to the
colonel. After his dismissal from headquarters Zherkov had not
remained in the regiment, saying he was not such a fool as to
slave at the front when he could get more rewards by doing
nothing on the staff, and had succeeded in attaching himself as
an orderly officer to Prince Bagration. He now came to his former
chief with an order from the commander of the rear guard.

``Colonel,'' he said, addressing Rostov's enemy with an air of
gloomy gravity and glancing round at his comrades, ``there is an
order to stop and fire the bridge.''

``An order to who?'' asked the colonel morosely.

``I don't myself know 'to who,'{}'' replied the cornet in a
serious tone, ``but the prince told me to 'go and tell the
colonel that the hussars must return quickly and fire the
bridge.'{}''

Zherkov was followed by an officer of the suite who rode up to
the colonel of hussars with the same order. After him the stout
Nesvitski came galloping up on a Cossack horse that could
scarcely carry his weight.

``How's this, Colonel?'' he shouted as he approached. ``I told
you to fire the bridge, and now someone has gone and blundered;
they are all beside themselves over there and one can't make
anything out.''

The colonel deliberately stopped the regiment and turned to
Nesvitski.

``You spoke to me of inflammable material,'' said he, ``but you
said nothing about firing it.''

``But, my dear sir,'' said Nesvitski as he drew up, taking off
his cap and smoothing his hair wet with perspiration with his
plump hand, ``wasn't I telling you to fire the bridge, when
inflammable material had been put in position?''

``I am not your 'dear sir,' Mr. Staff Officer, and you did not
tell me to burn the bridge! I know the service, and it is my
habit orders strictly to obey. You said the bridge would be
burned, but who would it burn, I could not know by the holy
spirit!''

``Ah, that's always the way!'' said Nesvitski with a wave of the
hand.  ``How did you get here?'' said he, turning to Zherkov.

``On the same business. But you are damp! Let me wring you out!''

``You were saying, Mr. Staff Officer...'' continued the colonel
in an offended tone.

``Colonel,'' interrupted the officer of the suite, ``You must be
quick or the enemy will bring up his guns to use grapeshot.''

The colonel looked silently at the officer of the suite, at the
stout staff officer, and at Zherkov, and he frowned.

``I will the bridge fire,'' he said in a solemn tone as if to
announce that in spite of all the unpleasantness he had to endure
he would still do the right thing.

Striking his horse with his long muscular legs as if it were to
blame for everything, the colonel moved forward and ordered the
second squadron, that in which Rostov was serving under Denisov,
to return to the bridge.

``There, it's just as I thought,'' said Rostov to himself. ``He
wishes to test me!'' His heart contracted and the blood rushed to
his face. ``Let him see whether I am a coward!'' he thought.

Again on all the bright faces of the squadron the serious
expression appeared that they had worn when under fire. Rostov
watched his enemy, the colonel, closely---to find in his face
confirmation of his own conjecture, but the colonel did not once
glance at Rostov, and looked as he always did when at the front,
solemn and stern. Then came the word of command.

``Look sharp! Look sharp!'' several voices repeated around him.

Their sabers catching in the bridles and their spurs jingling,
the hussars hastily dismounted, not knowing what they were to
do. The men were crossing themselves. Rostov no longer looked at
the colonel, he had no time. He was afraid of falling behind the
hussars, so much afraid that his heart stood still. His hand
trembled as he gave his horse into an orderly's charge, and he
felt the blood rush to his heart with a thud. Denisov rode past
him, leaning back and shouting something. Rostov saw nothing but
the hussars running all around him, their spurs catching and
their sabers clattering.

``Stretchers!'' shouted someone behind him.

Rostov did not think what this call for stretchers meant; he ran
on, trying only to be ahead of the others; but just at the
bridge, not looking at the ground, he came on some sticky,
trodden mud, stumbled, and fell on his hands. The others
outstripped him.

``At boss zides, Captain,'' he heard the voice of the colonel,
who, having ridden ahead, had pulled up his horse near the
bridge, with a triumphant, cheerful face.

Rostov wiping his muddy hands on his breeches looked at his enemy
and was about to run on, thinking that the farther he went to the
front the better. But Bogdanich, without looking at or
recognizing Rostov, shouted to him:

``Who's that running on the middle of the bridge? To the right!
Come back, Cadet!'' he cried angrily; and turning to Denisov,
who, showing off his courage, had ridden on to the planks of the
bridge:

``Why run risks, Captain? You should dismount,'' he said.

``Oh, every bullet has its billet,'' answered Vaska Denisov,
turning in his saddle.

Meanwhile Nesvitski, Zherkov, and the officer of the suite were
standing together out of range of the shots, watching, now the
small group of men with yellow shakos, dark-green jackets braided
with cord, and blue riding breeches, who were swarming near the
bridge, and then at what was approaching in the distance from the
opposite side---the blue uniforms and groups with horses, easily
recognizable as artillery.

``Will they burn the bridge or not? Who'll get there first? Will
they get there and fire the bridge or will the French get within
grapeshot range and wipe them out?'' These were the questions
each man of the troops on the high ground above the bridge
involuntarily asked himself with a sinking heart---watching the
bridge and the hussars in the bright evening light and the blue
tunics advancing from the other side with their bayonets and
guns.

``Ugh. The hussars will get it hot!'' said Nesvitski; ``they are
within grapeshot range now.''

``He shouldn't have taken so many men,'' said the officer of the
suite.

``True enough,'' answered Nesvitski; ``two smart fellows could
have done the job just as well.''

``Ah, your excellency,'' put in Zherkov, his eyes fixed on the
hussars, but still with that naive air that made it impossible to
know whether he was speaking in jest or in earnest. ``Ah, your
excellency! How you look at things! Send two men? And who then
would give us the Vladimir medal and ribbon? But now, even if
they do get peppered, the squadron may be recommended for honors
and he may get a ribbon. Our Bogdanich knows how things are
done.''

``There now!'' said the officer of the suite, ``that's
grapeshot.''

He pointed to the French guns, the limbers of which were being
detached and hurriedly removed.

On the French side, amid the groups with cannon, a cloud of smoke
appeared, then a second and a third almost simultaneously, and at
the moment when the first report was heard a fourth was
seen. Then two reports one after another, and a third.

``Oh! Oh!'' groaned Nesvitski as if in fierce pain, seizing the
officer of the suite by the arm. ``Look! A man has fallen!
Fallen, fallen!''

``Two, I think.''

``If I were Tsar I would never go to war,'' said Nesvitski,
turning away.

The French guns were hastily reloaded. The infantry in their blue
uniforms advanced toward the bridge at a run. Smoke appeared
again but at irregular intervals, and grapeshot cracked and
rattled onto the bridge. But this time Nesvitski could not see
what was happening there, as a dense cloud of smoke arose from
it. The hussars had succeeded in setting it on fire and the
French batteries were now firing at them, no longer to hinder
them but because the guns were trained and there was someone to
fire at.

The French had time to fire three rounds of grapeshot before the
hussars got back to their horses. Two were misdirected and the
shot went too high, but the last round fell in the midst of a
group of hussars and knocked three of them over.

Rostov, absorbed by his relations with Bogdanich, had paused on
the bridge not knowing what to do. There was no one to hew down
(as he had always imagined battles to himself), nor could he help
to fire the bridge because he had not brought any burning straw
with him like the other soldiers. He stood looking about him,
when suddenly he heard a rattle on the bridge as if nuts were
being spilt, and the hussar nearest to him fell against the rails
with a groan. Rostov ran up to him with the others. Again someone
shouted, ``Stretchers!'' Four men seized the hussar and began
lifting him.

``Oooh! For Christ's sake let me alone!'' cried the wounded man,
but still he was lifted and laid on the stretcher.

Nicholas Rostov turned away and, as if searching for something,
gazed into the distance, at the waters of the Danube, at the sky,
and at the sun. How beautiful the sky looked; how blue, how calm,
and how deep! How bright and glorious was the setting sun! With
what soft glitter the waters of the distant Danube shone. And
fairer still were the faraway blue mountains beyond the river,
the nunnery, the mysterious gorges, and the pine forests veiled
in the mist of their summits... There was peace and
happiness... ``I should wish for nothing else, nothing, if only I
were there,'' thought Rostov. ``In myself alone and in that
sunshine there is so much happiness; but here... groans,
suffering, fear, and this uncertainty and hurry... There---they
are shouting again, and again are all running back somewhere, and
I shall run with them, and it, death, is here above me and
around... Another instant and I shall never again see the sun,
this water, that gorge!...''

At that instant the sun began to hide behind the clouds, and
other stretchers came into view before Rostov. And the fear of
death and of the stretchers, and love of the sun and of life, all
merged into one feeling of sickening agitation.

``O Lord God! Thou who art in that heaven, save, forgive, and
protect me!'' Rostov whispered.

The hussars ran back to the men who held their horses; their
voices sounded louder and calmer, the stretchers disappeared from
sight.

``Well, fwiend? So you've smelt powdah!'' shouted Vaska Denisov
just above his ear.

``It's all over; but I am a coward---yes, a coward!'' thought
Rostov, and sighing deeply he took Rook, his horse, which stood
resting one foot, from the orderly and began to mount.

``Was that grapeshot?'' he asked Denisov.

``Yes and no mistake!'' cried Denisov. ``You worked like wegular
bwicks and it's nasty work! An attack's pleasant work! Hacking
away at the dogs!  But this sort of thing is the very devil, with
them shooting at you like a target.''

And Denisov rode up to a group that had stopped near Rostov,
composed of the colonel, Nesvitski, Zherkov, and the officer from
the suite.

``Well, it seems that no one has noticed,'' thought Rostov. And
this was true. No one had taken any notice, for everyone knew the
sensation which the cadet under fire for the first time had
experienced.

``Here's something for you to report,'' said Zherkov. ``See if I
don't get promoted to a sublieutenancy.''

``Inform the prince that I the bridge fired!'' said the colonel
triumphantly and gaily.

``And if he asks about the losses?''

``A trifle,'' said the colonel in his bass voice: ``two hussars
wounded, and one knocked out,'' he added, unable to restrain a
happy smile, and pronouncing the phrase ``knocked out'' with
ringing distinctness.

% % % % % % % % % % % % % % % % % % % % % % % % % % % % % % % % %
% % % % % % % % % % % % % % % % % % % % % % % % % % % % % % % % %
% % % % % % % % % % % % % % % % % % % % % % % % % % % % % % % % %
% % % % % % % % % % % % % % % % % % % % % % % % % % % % % % % % %
% % % % % % % % % % % % % % % % % % % % % % % % % % % % % % % % %
% % % % % % % % % % % % % % % % % % % % % % % % % % % % % % % % %
% % % % % % % % % % % % % % % % % % % % % % % % % % % % % % % % %
% % % % % % % % % % % % % % % % % % % % % % % % % % % % % % % % %
% % % % % % % % % % % % % % % % % % % % % % % % % % % % % % % % %
% % % % % % % % % % % % % % % % % % % % % % % % % % % % % % % % %
% % % % % % % % % % % % % % % % % % % % % % % % % % % % % % % % %
% % % % % % % % % % % % % % % % % % % % % % % % % % % % % %

\chapter*{Chapter IX}
\ifaudio     \marginpar{
\href{http://ia902608.us.archive.org/23/items/war_and_peace_02_0801_librivox/war_and_peace_02_09_tolstoy_64kb.mp3}{Audio}
} \fi

\lettrine[lines=2, loversize=0.3, lraise=0]{\initfamily P}{ursued}
by the French army of a hundred thousand men under the
command of Bonaparte, encountering a population that was
unfriendly to it, losing confidence in its allies, suffering from
shortness of supplies, and compelled to act under conditions of
war unlike anything that had been foreseen, the Russian army of
thirty-five thousand men commanded by Kutuzov was hurriedly
retreating along the Danube, stopping where overtaken by the
enemy and fighting rearguard actions only as far as necessary to
enable it to retreat without losing its heavy equipment.  There
had been actions at Lambach, Amstetten, and Melk; but despite the
courage and endurance---acknowledged even by the enemy---with
which the Russians fought, the only consequence of these actions
was a yet more rapid retreat. Austrian troops that had escaped
capture at Ulm and had joined Kutuzov at Braunau now separated
from the Russian army, and Kutuzov was left with only his own
weak and exhausted forces. The defense of Vienna was no longer to
be thought of. Instead of an offensive, the plan of which,
carefully prepared in accord with the modern science of
strategics, had been handed to Kutuzov when he was in Vienna by
the Austrian Hofkriegsrath, the sole and almost unattainable aim
remaining for him was to effect a junction with the forces that
were advancing from Russia, without losing his army as Mack had
done at Ulm.

On the twenty-eighth of October Kutuzov with his army crossed to
the left bank of the Danube and took up a position for the first
time with the river between himself and the main body of the
French. On the thirtieth he attacked Mortier's division, which
was on the left bank, and broke it up. In this action for the
first time trophies were taken: banners, cannon, and two enemy
generals. For the first time, after a fortnight's retreat, the
Russian troops had halted and after a fight had not only held the
field but had repulsed the French. Though the troops were
ill-clad, exhausted, and had lost a third of their number in
killed, wounded, sick, and stragglers; though a number of sick
and wounded had been abandoned on the other side of the Danube
with a letter in which Kutuzov entrusted them to the humanity of
the enemy; and though the big hospitals and the houses in Krems
converted into military hospitals could no longer accommodate all
the sick and wounded, yet the stand made at Krems and the victory
over Mortier raised the spirits of the army
considerably. Throughout the whole army and at headquarters most
joyful though erroneous rumors were rife of the imaginary
approach of columns from Russia, of some victory gained by the
Austrians, and of the retreat of the frightened Bonaparte.

Prince Andrew during the battle had been in attendance on the
Austrian General Schmidt, who was killed in the action. His horse
had been wounded under him and his own arm slightly grazed by a
bullet. As a mark of the commander-in-chief's special favor he
was sent with the news of this victory to the Austrian court, now
no longer at Vienna (which was threatened by the French) but at
Brunn. Despite his apparently delicate build Prince Andrew could
endure physical fatigue far better than many very muscular men,
and on the night of the battle, having arrived at Krems excited
but not weary, with dispatches from Dokhturov to Kutuzov, he was
sent immediately with a special dispatch to Brunn. To be so sent
meant not only a reward but an important step toward promotion.

The night was dark but starry, the road showed black in the snow
that had fallen the previous day---the day of the
battle. Reviewing his impressions of the recent battle, picturing
pleasantly to himself the impression his news of a victory would
create, or recalling the send-off given him by the
commander-in-chief and his fellow officers, Prince Andrew was
galloping along in a post chaise enjoying the feelings of a man
who has at length begun to attain a long-desired happiness. As
soon as he closed his eyes his ears seemed filled with the rattle
of the wheels and the sensation of victory. Then he began to
imagine that the Russians were running away and that he himself
was killed, but he quickly roused himself with a feeling of joy,
as if learning afresh that this was not so but that on the
contrary the French had run away. He again recalled all the
details of the victory and his own calm courage during the
battle, and feeling reassured he dozed off... The dark starry
night was followed by a bright cheerful morning. The snow was
thawing in the sunshine, the horses galloped quickly, and on both
sides of the road were forests of different kinds, fields, and
villages.

At one of the post stations he overtook a convoy of Russian
wounded. The Russian officer in charge of the transport lolled
back in the front cart, shouting and scolding a soldier with
coarse abuse. In each of the long German carts six or more pale,
dirty, bandaged men were being jolted over the stony road. Some
of them were talking (he heard Russian words), others were eating
bread; the more severely wounded looked silently, with the
languid interest of sick children, at the envoy hurrying past
them.

Prince Andrew told his driver to stop, and asked a soldier in
what action they had been wounded. ``Day before yesterday, on the
Danube,'' answered the soldier. Prince Andrew took out his purse
and gave the soldier three gold pieces.

``That's for them all,'' he said to the officer who came up.

``Get well soon, lads!'' he continued, turning to the
soldiers. ``There's plenty to do still.''

``What news, sir?'' asked the officer, evidently anxious to start
a conversation.

``Good news!... Go on!'' he shouted to the driver, and they
galloped on.

It was already quite dark when Prince Andrew rattled over the
paved streets of Brunn and found himself surrounded by high
buildings, the lights of shops, houses, and street lamps, fine
carriages, and all that atmosphere of a large and active town
which is always so attractive to a soldier after camp
life. Despite his rapid journey and sleepless night, Prince
Andrew when he drove up to the palace felt even more vigorous and
alert than he had done the day before. Only his eyes gleamed
feverishly and his thoughts followed one another with
extraordinary clearness and rapidity. He again vividly recalled
the details of the battle, no longer dim, but definite and in the
concise form in which he imagined himself stating them to the
Emperor Francis. He vividly imagined the casual questions that
might be put to him and the answers he would give. He expected to
be at once presented to the Emperor. At the chief entrance to the
palace, however, an official came running out to meet him, and
learning that he was a special messenger led him to another
entrance.

``To the right from the corridor, Euer Hochgeboren! There you
will find the adjutant on duty,'' said the official. ``He will
conduct you to the Minister of War.''

The adjutant on duty, meeting Prince Andrew, asked him to wait,
and went in to the Minister of War. Five minutes later he
returned and bowing with particular courtesy ushered Prince
Andrew before him along a corridor to the cabinet where the
Minister of War was at work. The adjutant by his elaborate
courtesy appeared to wish to ward off any attempt at familiarity
on the part of the Russian messenger.

Prince Andrew's joyous feeling was considerably weakened as he
approached the door of the minister's room. He felt offended, and
without his noticing it the feeling of offense immediately turned
into one of disdain which was quite uncalled for. His fertile
mind instantly suggested to him a point of view which gave him a
right to despise the adjutant and the minister. ``Away from the
smell of powder, they probably think it easy to gain victories!''
he thought. His eyes narrowed disdainfully, he entered the room
of the Minister of War with peculiarly deliberate steps. This
feeling of disdain was heightened when he saw the minister seated
at a large table reading some papers and making pencil notes on
them, and for the first two or three minutes taking no notice of
his arrival. A wax candle stood at each side of the minister's
bent bald head with its gray temples. He went on reading to the
end, without raising his eyes at the opening of the door and the
sound of footsteps.

``Take this and deliver it,'' said he to his adjutant, handing
him the papers and still taking no notice of the special
messenger.

Prince Andrew felt that either the actions of Kutuzov's army
interested the Minister of War less than any of the other matters
he was concerned with, or he wanted to give the Russian special
messenger that impression. ``But that is a matter of perfect
indifference to me,'' he thought. The minister drew the remaining
papers together, arranged them evenly, and then raised his
head. He had an intellectual and distinctive head, but the
instant he turned to Prince Andrew the firm, intelligent
expression on his face changed in a way evidently deliberate and
habitual to him. His face took on the stupid artificial smile
(which does not even attempt to hide its artificiality) of a man
who is continually receiving many petitioners one after another.

``From General Field Marshal Kutuzov?'' he asked. ``I hope it is
good news?  There has been an encounter with Mortier? A victory?
It was high time!''

He took the dispatch which was addressed to him and began to read
it with a mournful expression.

``Oh, my God! My God! Schmidt!'' he exclaimed in German. ``What a
calamity!  What a calamity!''

Having glanced through the dispatch he laid it on the table and
looked at Prince Andrew, evidently considering something.

``Ah what a calamity! You say the affair was decisive? But
Mortier is not captured.'' Again he pondered. ``I am very glad
you have brought good news, though Schmidt's death is a heavy
price to pay for the victory.  His Majesty will no doubt wish to
see you, but not today. I thank you!  You must have a rest. Be at
the levee tomorrow after the parade.  However, I will let you
know.''

The stupid smile, which had left his face while he was speaking,
reappeared.

``Au revoir! Thank you very much. His Majesty will probably
desire to see you,'' he added, bowing his head.

When Prince Andrew left the palace he felt that all the interest
and happiness the victory had afforded him had been now left in
the indifferent hands of the Minister of War and the polite
adjutant. The whole tenor of his thoughts instantaneously
changed; the battle seemed the memory of a remote event long
past.

% % % % % % % % % % % % % % % % % % % % % % % % % % % % % % % % %
% % % % % % % % % % % % % % % % % % % % % % % % % % % % % % % % %
% % % % % % % % % % % % % % % % % % % % % % % % % % % % % % % % %
% % % % % % % % % % % % % % % % % % % % % % % % % % % % % % % % %
% % % % % % % % % % % % % % % % % % % % % % % % % % % % % % % % %
% % % % % % % % % % % % % % % % % % % % % % % % % % % % % % % % %
% % % % % % % % % % % % % % % % % % % % % % % % % % % % % % % % %
% % % % % % % % % % % % % % % % % % % % % % % % % % % % % % % % %
% % % % % % % % % % % % % % % % % % % % % % % % % % % % % % % % %
% % % % % % % % % % % % % % % % % % % % % % % % % % % % % % % % %
% % % % % % % % % % % % % % % % % % % % % % % % % % % % % % % % %
% % % % % % % % % % % % % % % % % % % % % % % % % % % % % %

\chapter*{Chapter X}
\ifaudio     \marginpar{
\href{http://ia902608.us.archive.org/23/items/war_and_peace_02_0801_librivox/war_and_peace_02_10_tolstoy_64kb.mp3}{Audio}
} \fi

\lettrine[lines=2, loversize=0.3, lraise=0]{\initfamily P}{rince}
Andrew stayed at Brunn with Bilibin, a Russian
acquaintance of his in the diplomatic service.

``Ah, my dear prince! I could not have a more welcome visitor,''
said Bilibin as he came out to meet Prince Andrew. ``Franz, put
the prince's things in my bedroom,'' said he to the servant who
was ushering Bolkonski in. ``So you're a messenger of victory,
eh? Splendid! And I am sitting here ill, as you see.''

After washing and dressing, Prince Andrew came into the
diplomat's luxurious study and sat down to the dinner prepared
for him. Bilibin settled down comfortably beside the fire.

After his journey and the campaign during which he had been
deprived of all the comforts of cleanliness and all the
refinements of life, Prince Andrew felt a pleasant sense of
repose among luxurious surroundings such as he had been
accustomed to from childhood. Besides it was pleasant, after his
reception by the Austrians, to speak if not in Russian (for they
were speaking French) at least with a Russian who would, he
supposed, share the general Russian antipathy to the Austrians
which was then particularly strong.

Bilibin was a man of thirty-five, a bachelor, and of the same
circle as Prince Andrew. They had known each other previously in
Petersburg, but had become more intimate when Prince Andrew was
in Vienna with Kutuzov.  Just as Prince Andrew was a young man
who gave promise of rising high in the military profession, so to
an even greater extent Bilibin gave promise of rising in his
diplomatic career. He still a young man but no longer a young
diplomat, as he had entered the service at the age of sixteen,
had been in Paris and Copenhagen, and now held a rather important
post in Vienna. Both the foreign minister and our ambassador in
Vienna knew him and valued him. He was not one of those many
diplomats who are esteemed because they have certain negative
qualities, avoid doing certain things, and speak French. He was
one of those, who, liking work, knew how to do it, and despite
his indolence would sometimes spend a whole night at his writing
table. He worked well whatever the import of his work. It was not
the question \emph{What for?} but the question \emph{How?} that
interested him. What the diplomatic matter might be he did not
care, but it gave him great pleasure to prepare a circular,
memorandum, or report, skillfully, pointedly, and elegantly.
Bilibin's services were valued not only for what he wrote, but
also for his skill in dealing and conversing with those in the
highest spheres.

Bilibin liked conversation as he liked work, only when it could
be made elegantly witty. In society he always awaited an
opportunity to say something striking and took part in a
conversation only when that was possible. His conversation was
always sprinkled with wittily original, finished phrases of
general interest. These sayings were prepared in the inner
laboratory of his mind in a portable form as if intentionally, so
that insignificant society people might carry them from drawing
room to drawing room. And, in fact, Bilibin's witticisms were
hawked about in the Viennese drawing rooms and often had an
influence on matters considered important.

His thin, worn, sallow face was covered with deep wrinkles, which
always looked as clean and well washed as the tips of one's
fingers after a Russian bath. The movement of these wrinkles
formed the principal play of expression on his face. Now his
forehead would pucker into deep folds and his eyebrows were
lifted, then his eyebrows would descend and deep wrinkles would
crease his cheeks. His small, deep-set eyes always twinkled and
looked out straight.

``Well, now tell me about your exploits,'' said he.

Bolkonski, very modestly without once mentioning himself,
described the engagement and his reception by the Minister of
War.

``They received me and my news as one receives a dog in a game of
skittles,'' said he in conclusion.

Bilibin smiled and the wrinkles on his face disappeared.

``Cependant, mon cher,'' he remarked, examining his nails from a
distance and puckering the skin above his left eye, ``malgre la
haute estime que je professe pour the Orthodox Russian army,
j'avoue que votre victoire n'est pas des plus victorieuses.''
\footnote{``But my dear fellow, with all my respect for the
Orthodox Russian army, I must say that your victory was not
particularly victorious.''}

He went on talking in this way in French, uttering only those
words in Russian on which he wished to put a contemptuous
emphasis.

``Come now! You with all your forces fall on the unfortunate
Mortier and his one division, and even then Mortier slips through
your fingers!  Where's the victory?''

``But seriously,'' said Prince Andrew, ``we can at any rate say
without boasting that it was a little better than at Ulm...''

``Why didn't you capture one, just one, marshal for us?''

``Because not everything happens as one expects or with the
smoothness of a parade. We had expected, as I told you, to get at
their rear by seven in the morning but had not reached it by five
in the afternoon.''

``And why didn't you do it at seven in the morning? You ought to
have been there at seven in the morning,'' returned Bilibin with
a smile. ``You ought to have been there at seven in the
morning.''

``Why did you not succeed in impressing on Bonaparte by
diplomatic methods that he had better leave Genoa alone?''
retorted Prince Andrew in the same tone.

``I know,'' interrupted Bilibin, ``you're thinking it's very easy
to take marshals, sitting on a sofa by the fire! That is true,
but still why didn't you capture him? So don't be surprised if
not only the Minister of War but also his Most August Majesty the
Emperor and King Francis is not much delighted by your
victory. Even I, a poor secretary of the Russian Embassy, do not
feel any need in token of my joy to give my Franz a thaler, or
let him go with his Liebchen to the Prater... True, we have no
Prater here...''

He looked straight at Prince Andrew and suddenly unwrinkled his
forehead.

``It is now my turn to ask you 'why?' mon cher,'' said
Bolkonski. ``I confess I do not understand: perhaps there are
diplomatic subtleties here beyond my feeble intelligence, but I
can't make it out. Mack loses a whole army, the Archduke
Ferdinand and the Archduke Karl give no signs of life and make
blunder after blunder. Kutuzov alone at last gains a real
victory, destroying the spell of the invincibility of the French,
and the Minister of War does not even care to hear the details.''

``That's just it, my dear fellow. You see it's hurrah for the
Tsar, for Russia, for the Orthodox Greek faith! All that is
beautiful, but what do we, I mean the Austrian court, care for
your victories? Bring us nice news of a victory by the Archduke
Karl or Ferdinand (one archduke's as good as another, as you
know) and even if it is only over a fire brigade of Bonaparte's,
that will be another story and we'll fire off some cannon! But
this sort of thing seems done on purpose to vex us. The Archduke
Karl does nothing, the Archduke Ferdinand disgraces himself.  You
abandon Vienna, give up its defense---as much as to say: 'Heaven
is with us, but heaven help you and your capital!' The one
general whom we all loved, Schmidt, you expose to a bullet, and
then you congratulate us on the victory! Admit that more
irritating news than yours could not have been conceived. It's as
if it had been done on purpose, on purpose.  Besides, suppose you
did gain a brilliant victory, if even the Archduke Karl gained a
victory, what effect would that have on the general course of
events? It's too late now when Vienna is occupied by the French
army!''

``What? Occupied? Vienna occupied?''

``Not only occupied, but Bonaparte is at Schonbrunn, and the
count, our dear Count Vrbna, goes to him for orders.''

After the fatigues and impressions of the journey, his reception,
and especially after having dined, Bolkonski felt that he could
not take in the full significance of the words he heard.

``Count Lichtenfels was here this morning,'' Bilibin continued,
``and showed me a letter in which the parade of the French in
Vienna was fully described: Prince Murat et tout le
tremblement... You see that your victory is not a matter for
great rejoicing and that you can't be received as a savior.''

``Really I don't care about that, I don't care at all,'' said
Prince Andrew, beginning to understand that his news of the
battle before Krems was really of small importance in view of
such events as the fall of Austria's capital. ``How is it Vienna
was taken? What of the bridge and its celebrated bridgehead and
Prince Auersperg? We heard reports that Prince Auersperg was
defending Vienna?'' he said.

``Prince Auersperg is on this, on our side of the river, and is
defending us---doing it very badly, I think, but still he is
defending us. But Vienna is on the other side. No, the bridge has
not yet been taken and I hope it will not be, for it is mined and
orders have been given to blow it up. Otherwise we should long
ago have been in the mountains of Bohemia, and you and your army
would have spent a bad quarter of an hour between two fires.''

``But still this does not mean that the campaign is over,'' said
Prince Andrew.

``Well, I think it is. The bigwigs here think so too, but they
daren't say so. It will be as I said at the beginning of the
campaign, it won't be your skirmishing at Durrenstein, or
gunpowder at all, that will decide the matter, but those who
devised it,'' said Bilibin quoting one of his own mots, releasing
the wrinkles on his forehead, and pausing.  ``The only question
is what will come of the meeting between the Emperor Alexander
and the King of Prussia in Berlin? If Prussia joins the Allies,
Austria's hand will be forced and there will be war. If not it is
merely a question of settling where the preliminaries of the new
Campo Formio are to be drawn up.''

``What an extraordinary genius!'' Prince Andrew suddenly
exclaimed, clenching his small hand and striking the table with
it, ``and what luck the man has!''

``Buonaparte?'' said Bilibin inquiringly, puckering up his
forehead to indicate that he was about to say something
witty. ``Buonaparte?'' he repeated, accentuating the u: ``I
think, however, now that he lays down laws for Austria at
Schonbrunn, il faut lui faire grace de l'u!\footnote{``We must
let him off the u!''} I shall certainly adopt an innovation and
call him simply Bonaparte!''

``But joking apart,'' said Prince Andrew, ``do you really think
the campaign is over?''

``This is what I think. Austria has been made a fool of, and she
is not used to it. She will retaliate. And she has been fooled in
the first place because her provinces have been pillaged---they
say the Holy Russian army loots terribly---her army is destroyed,
her capital taken, and all this for the beaux yeux\footnote{Fine
eyes.} of His Sardinian Majesty. And therefore---this is between
ourselves---I instinctively feel that we are being deceived, my
instinct tells me of negotiations with France and projects for
peace, a secret peace concluded separately.''

``Impossible!'' cried Prince Andrew. ``That would be too base.''

``If we live we shall see,'' replied Bilibin, his face again
becoming smooth as a sign that the conversation was at an end.

When Prince Andrew reached the room prepared for him and lay down
in a clean shirt on the feather bed with its warmed and fragrant
pillows, he felt that the battle of which he had brought tidings
was far, far away from him. The alliance with Prussia, Austria's
treachery, Bonaparte's new triumph, tomorrow's levee and parade,
and the audience with the Emperor Francis occupied his thoughts.

He closed his eyes, and immediately a sound of cannonading, of
musketry and the rattling of carriage wheels seemed to fill his
ears, and now again drawn out in a thin line the musketeers were
descending the hill, the French were firing, and he felt his
heart palpitating as he rode forward beside Schmidt with the
bullets merrily whistling all around, and he experienced tenfold
the joy of living, as he had not done since childhood.

He woke up...

``Yes, that all happened!'' he said, and, smiling happily to
himself like a child, he fell into a deep, youthful slumber.

% % % % % % % % % % % % % % % % % % % % % % % % % % % % % % % % %
% % % % % % % % % % % % % % % % % % % % % % % % % % % % % % % % %
% % % % % % % % % % % % % % % % % % % % % % % % % % % % % % % % %
% % % % % % % % % % % % % % % % % % % % % % % % % % % % % % % % %
% % % % % % % % % % % % % % % % % % % % % % % % % % % % % % % % %
% % % % % % % % % % % % % % % % % % % % % % % % % % % % % % % % %
% % % % % % % % % % % % % % % % % % % % % % % % % % % % % % % % %
% % % % % % % % % % % % % % % % % % % % % % % % % % % % % % % % %
% % % % % % % % % % % % % % % % % % % % % % % % % % % % % % % % %
% % % % % % % % % % % % % % % % % % % % % % % % % % % % % % % % %
% % % % % % % % % % % % % % % % % % % % % % % % % % % % % % % % %
% % % % % % % % % % % % % % % % % % % % % % % % % % % % % %

\chapter*{Chapter XI}
\ifaudio     \marginpar{
\href{http://ia902608.us.archive.org/23/items/war_and_peace_02_0801_librivox/war_and_peace_02_11_tolstoy_64kb.mp3}{Audio}
} \fi

\lettrine[lines=2, loversize=0.3, lraise=0]{\initfamily N}{ext}
day he woke late. Recalling his recent impressions, the
first thought that came into his mind was that today he had to be
presented to the Emperor Francis; he remembered the Minister of
War, the polite Austrian adjutant, Bilibin, and last night's
conversation. Having dressed for his attendance at court in full
parade uniform, which he had not worn for a long time, he went
into Bilibin's study fresh, animated, and handsome, with his hand
bandaged. In the study were four gentlemen of the diplomatic
corps. With Prince Hippolyte Kuragin, who was a secretary to the
embassy, Bolkonski was already acquainted. Bilibin introduced him
to the others.

The gentlemen assembled at Bilibin's were young, wealthy, gay
society men, who here, as in Vienna, formed a special set which
Bilibin, their leader, called les notres.\footnote{Ours.} This
set, consisting almost exclusively of diplomats, evidently had
its own interests which had nothing to do with war or politics
but related to high society, to certain women, and to the
official side of the service. These gentlemen received Prince
Andrew as one of themselves, an honor they did not extend to
many. From politeness and to start conversation, they asked him a
few questions about the army and the battle, and then the talk
went off into merry jests and gossip.

``But the best of it was,'' said one, telling of the misfortune
of a fellow diplomat, ``that the Chancellor told him flatly that
his appointment to London was a promotion and that he was so to
regard it.  Can you fancy the figure he cut?...''

``But the worst of it, gentlemen---I am giving Kuragin away to
you---is that that man suffers, and this Don Juan, wicked fellow,
is taking advantage of it!''

Prince Hippolyte was lolling in a lounge chair with his legs over
its arm. He began to laugh.

``Tell me about that!'' he said.

``Oh, you Don Juan! You serpent!'' cried several voices.

``You, Bolkonski, don't know,'' said Bilibin turning to Prince
Andrew, ``that all the atrocities of the French army (I nearly
said of the Russian army) are nothing compared to what this man
has been doing among the women!''

``La femme est la compagne de l'homme,''\footnote{``Woman is
man's companion.''} announced Prince Hippolyte, and began looking
through a lorgnette at his elevated legs.

Bilibin and the rest of ``ours'' burst out laughing in
Hippolyte's face, and Prince Andrew saw that Hippolyte, of
whom---he had to admit---he had almost been jealous on his wife's
account, was the butt of this set.

``Oh, I must give you a treat,'' Bilibin whispered to
Bolkonski. ``Kuragin is exquisite when he discusses
politics---you should see his gravity!''

He sat down beside Hippolyte and wrinkling his forehead began
talking to him about politics. Prince Andrew and the others
gathered round these two.

``The Berlin cabinet cannot express a feeling of alliance,''
began Hippolyte gazing round with importance at the others,
``without expressing... as in its last note... you
understand... Besides, unless His Majesty the Emperor derogates
from the principle of our alliance...''

``Wait, I have not finished...'' he said to Prince Andrew,
seizing him by the arm, ``I believe that intervention will be
stronger than nonintervention. And...'' he paused. ``Finally one
cannot impute the nonreceipt of our dispatch of November 18. That
is how it will end.'' And he released Bolkonski's arm to indicate
that he had now quite finished.

``Demosthenes, I know thee by the pebble thou secretest in thy
golden mouth!'' said Bilibin, and the mop of hair on his head
moved with satisfaction.

Everybody laughed, and Hippolyte louder than anyone. He was
evidently distressed, and breathed painfully, but could not
restrain the wild laughter that convulsed his usually impassive
features.

``Well now, gentlemen,'' said Bilibin, ``Bolkonski is my guest in
this house and in Brunn itself. I want to entertain him as far as
I can, with all the pleasures of life here. If we were in Vienna
it would be easy, but here, in this wretched Moravian hole, it is
more difficult, and I beg you all to help me. Brunn's attractions
must be shown him. You can undertake the theater, I society, and
you, Hippolyte, of course the women.''

``We must let him see Amelie, she's exquisite!'' said one of
``ours,'' kissing his finger tips.

``In general we must turn this bloodthirsty soldier to more
humane interests,'' said Bilibin.

``I shall scarcely be able to avail myself of your hospitality,
gentlemen, it is already time for me to go,'' replied Prince
Andrew looking at his watch.

``Where to?''

``To the Emperor.''

``Oh! Oh! Oh! Well, au revoir, Bolkonski! Au revoir, Prince! Come
back early to dinner,'' cried several voices. ``We'll take you in
hand.''

``When speaking to the Emperor, try as far as you can to praise
the way that provisions are supplied and the routes indicated,''
said Bilibin, accompanying him to the hall.

``I should like to speak well of them, but as far as I know the
facts, I can't,'' replied Bolkonski, smiling.

``Well, talk as much as you can, anyway. He has a passion for
giving audiences, but he does not like talking himself and can't
do it, as you will see.''

% % % % % % % % % % % % % % % % % % % % % % % % % % % % % % % % %
% % % % % % % % % % % % % % % % % % % % % % % % % % % % % % % % %
% % % % % % % % % % % % % % % % % % % % % % % % % % % % % % % % %
% % % % % % % % % % % % % % % % % % % % % % % % % % % % % % % % %
% % % % % % % % % % % % % % % % % % % % % % % % % % % % % % % % %
% % % % % % % % % % % % % % % % % % % % % % % % % % % % % % % % %
% % % % % % % % % % % % % % % % % % % % % % % % % % % % % % % % %
% % % % % % % % % % % % % % % % % % % % % % % % % % % % % % % % %
% % % % % % % % % % % % % % % % % % % % % % % % % % % % % % % % %
% % % % % % % % % % % % % % % % % % % % % % % % % % % % % % % % %
% % % % % % % % % % % % % % % % % % % % % % % % % % % % % % % % %
% % % % % % % % % % % % % % % % % % % % % % % % % % % % % %

\chapter*{Chapter XII}
\ifaudio     \marginpar{
\href{http://ia902608.us.archive.org/23/items/war_and_peace_02_0801_librivox/war_and_peace_02_12_tolstoy_64kb.mp3}{Audio}
} \fi

\lettrine[lines=2, loversize=0.3, lraise=0]{\initfamily A}{t}
the levee Prince Andrew stood among the Austrian officers as
he had been told to, and the Emperor Francis merely looked
fixedly into his face and just nodded to him with his long
head. But after it was over, the adjutant he had seen the
previous day ceremoniously informed Bolkonski that the Emperor
desired to give him an audience. The Emperor Francis received him
standing in the middle of the room. Before the conversation began
Prince Andrew was struck by the fact that the Emperor seemed
confused and blushed as if not knowing what to say.

``Tell me, when did the battle begin?'' he asked hurriedly.

Prince Andrew replied. Then followed other questions just as
simple: ``Was Kutuzov well? When had he left Krems?'' and so
on. The Emperor spoke as if his sole aim were to put a given
number of questions---the answers to these questions, as was only
too evident, did not interest him.

``At what o'clock did the battle begin?'' asked the Emperor.

``I cannot inform Your Majesty at what o'clock the battle began
at the front, but at Durrenstein, where I was, our attack began
after five in the afternoon,'' replied Bolkonski growing more
animated and expecting that he would have a chance to give a
reliable account, which he had ready in his mind, of all he knew
and had seen. But the Emperor smiled and interrupted him.

``How many miles?''

``From where to where, Your Majesty?''

``From Durrenstein to Krems.''

``Three and a half miles, Your Majesty.''

``The French have abandoned the left bank?''

``According to the scouts the last of them crossed on rafts
during the night.''

``Is there sufficient forage in Krems?''

``Forage has not been supplied to the extent...''

The Emperor interrupted him.

``At what o'clock was General Schmidt killed?''

``At seven o'clock, I believe.''

``At seven o'clock? It's very sad, very sad!''

The Emperor thanked Prince Andrew and bowed. Prince Andrew
withdrew and was immediately surrounded by courtiers on all
sides. Everywhere he saw friendly looks and heard friendly
words. Yesterday's adjutant reproached him for not having stayed
at the palace, and offered him his own house.  The Minister of
War came up and congratulated him on the Maria Theresa Order of
the third grade, which the Emperor was conferring on him. The
Empress' chamberlain invited him to see Her Majesty. The
archduchess also wished to see him. He did not know whom to
answer, and for a few seconds collected his thoughts. Then the
Russian ambassador took him by the shoulder, led him to the
window, and began to talk to him.

Contrary to Bilibin's forecast the news he had brought was
joyfully received. A thanksgiving service was arranged, Kutuzov
was awarded the Grand Cross of Maria Theresa, and the whole army
received rewards.  Bolkonski was invited everywhere, and had to
spend the whole morning calling on the principal Austrian
dignitaries. Between four and five in the afternoon, having made
all his calls, he was returning to Bilibin's house thinking out a
letter to his father about the battle and his visit to Brunn. At
the door he found a vehicle half full of luggage. Franz,
Bilibin's man, was dragging a portmanteau with some difficulty
out of the front door.

Before returning to Bilibin's Prince Andrew had gone to a
bookshop to provide himself with some books for the campaign, and
had spent some time in the shop.

``What is it?'' he asked.

``Oh, your excellency!'' said Franz, with difficulty rolling the
portmanteau into the vehicle, ``we are to move on still
farther. The scoundrel is again at our heels!''

``Eh? What?'' asked Prince Andrew.

Bilibin came out to meet him. His usually calm face showed
excitement.

``There now! Confess that this is delightful,'' said he. ``This
affair of the Thabor Bridge, at Vienna... They have crossed
without striking a blow!''

Prince Andrew could not understand.

``But where do you come from not to know what every coachman in
the town knows?''

``I come from the archduchess'. I heard nothing there.''

``And you didn't see that everybody is packing up?''

``I did not... What is it all about?'' inquired Prince Andrew
impatiently.

``What's it all about? Why, the French have crossed the bridge
that Auersperg was defending, and the bridge was not blown up: so
Murat is now rushing along the road to Brunn and will be here in
a day or two.''

``What? Here? But why did they not blow up the bridge, if it was
mined?''

``That is what I ask you. No one, not even Bonaparte, knows
why.''

Bolkonski shrugged his shoulders.

``But if the bridge is crossed it means that the army too is
lost? It will be cut off,'' said he.

``That's just it,'' answered Bilibin. ``Listen! The French
entered Vienna as I told you. Very well. Next day, which was
yesterday, those gentlemen, messieurs les marechaux,\footnote{The
marshalls.}  Murat, Lannes, and Belliard, mount and ride to the
bridge. (Observe that all three are Gascons.)  'Gentlemen,' says
one of them, 'you know the Thabor Bridge is mined and doubly
mined and that there are menacing fortifications at its head and
an army of fifteen thousand men has been ordered to blow up the
bridge and not let us cross? But it will please our sovereign the
Emperor Napoleon if we take this bridge, so let us three go and
take it!' 'Yes, let's!' say the others. And off they go and take
the bridge, cross it, and now with their whole army are on this
side of the Danube, marching on us, you, and your lines of
communication.''

``Stop jesting,'' said Prince Andrew sadly and seriously. This
news grieved him and yet he was pleased.

As soon as he learned that the Russian army was in such a
hopeless situation it occurred to him that it was he who was
destined to lead it out of this position; that here was the
Toulon that would lift him from the ranks of obscure officers and
offer him the first step to fame!  Listening to Bilibin he was
already imagining how on reaching the army he would give an
opinion at the war council which would be the only one that could
save the army, and how he alone would be entrusted with the
executing of the plan.

``Stop this jesting,'' he said.

``I am not jesting,'' Bilibin went on. ``Nothing is truer or
sadder. These gentlemen ride onto the bridge alone and wave white
handkerchiefs; they assure the officer on duty that they, the
marshals, are on their way to negotiate with Prince Auersperg. He
lets them enter the t??te-de-pont.\footnote{Bridgehead.}  They
spin him a thousand gasconades, saying that the war is over, that
the Emperor Francis is arranging a meeting with Bonaparte, that
they desire to see Prince Auersperg, and so on. The officer sends
for Auersperg; these gentlemen embrace the officers, crack jokes,
sit on the cannon, and meanwhile a French battalion gets to the
bridge unobserved, flings the bags of incendiary material into
the water, and approaches the t??te-de-pont. At length appears the
lieutenant general, our dear Prince Auersperg von Mautern
himself. 'Dearest foe! Flower of the Austrian army, hero of the
Turkish wars Hostilities are ended, we can shake one another's
hand... The Emperor Napoleon burns with impatience to make Prince
Auersperg's acquaintance.' In a word, those gentlemen, Gascons
indeed, so bewildered him with fine words, and he is so flattered
by his rapidly established intimacy with the French marshals, and
so dazzled by the sight of Murat's mantle and ostrich plumes,
qu'il n'y voit que du feu, et oublie celui qu'il devait faire
faire sur l'ennemi!''\footnote{That their fire gets into his eyes
and he forgets that he ought to be firing at the enemy.} In spite
of the animation of his speech, Bilibin did not forget to pause
after this mot to give time for its due appreciation.  ``The
French battalion rushes to the bridgehead, spikes the guns, and
the bridge is taken! But what is best of all,'' he went on, his
excitement subsiding under the delightful interest of his own
story, ``is that the sergeant in charge of the cannon which was
to give the signal to fire the mines and blow up the bridge, this
sergeant, seeing that the French troops were running onto the
bridge, was about to fire, but Lannes stayed his hand. The
sergeant, who was evidently wiser than his general, goes up to
Auersperg and says: 'Prince, you are being deceived, here are the
French!' Murat, seeing that all is lost if the sergeant is
allowed to speak, turns to Auersperg with feigned astonishment
(he is a true Gascon) and says: 'I don't recognize the
world-famous Austrian discipline, if you allow a subordinate to
address you like that!' It was a stroke of genius. Prince
Auersperg feels his dignity at stake and orders the sergeant to
be arrested. Come, you must own that this affair of the Thabor
Bridge is delightful! It is not exactly stupidity, nor
rascality...''

``It may be treachery,'' said Prince Andrew, vividly imagining
the gray overcoats, wounds, the smoke of gunpowder, the sounds of
firing, and the glory that awaited him.

``Not that either. That puts the court in too bad a light,''
replied Bilibin. ``It's not treachery nor rascality nor
stupidity: it is just as at Ulm... it is...''---he seemed to be
trying to find the right expression. ``C'est... c'est du
Mack. Nous sommes mackes (It is... it is a bit of Mack. We are
Macked),'' he concluded, feeling that he had produced a good
epigram, a fresh one that would be repeated. His hitherto
puckered brow became smooth as a sign of pleasure, and with a
slight smile he began to examine his nails.

``Where are you off to?'' he said suddenly to Prince Andrew who
had risen and was going toward his room.

``I am going away.''

``Where to?''

``To the army.''

``But you meant to stay another two days?''

``But now I am off at once.''

And Prince Andrew after giving directions about his departure
went to his room.

``Do you know, mon cher,'' said Bilibin following him, ``I have
been thinking about you. Why are you going?''

And in proof of the conclusiveness of his opinion all the
wrinkles vanished from his face.

Prince Andrew looked inquiringly at him and gave no reply.

``Why are you going? I know you think it your duty to gallop back
to the army now that it is in danger. I understand that. Mon
cher, it is heroism!''

``Not at all,'' said Prince Andrew.

``But as you are a philosopher, be a consistent one, look at the
other side of the question and you will see that your duty, on
the contrary, is to take care of yourself. Leave it to those who
are no longer fit for anything else... You have not been ordered
to return and have not been dismissed from here; therefore, you
can stay and go with us wherever our ill luck takes us. They say
we are going to Olmutz, and Olmutz is a very decent town. You and
I will travel comfortably in my caleche.''

``Do stop joking, Bilibin,'' cried Bolkonski.

``I am speaking sincerely as a friend! Consider! Where and why
are you going, when you might remain here? You are faced by one
of two things,'' and the skin over his left temple puckered,
``either you will not reach your regiment before peace is
concluded, or you will share defeat and disgrace with Kutuzov's
whole army.''

And Bilibin unwrinkled his temple, feeling that the dilemma was
insoluble.

``I cannot argue about it,'' replied Prince Andrew coldly, but he
thought: ``I am going to save the army.''

``My dear fellow, you are a hero!'' said Bilibin.

% % % % % % % % % % % % % % % % % % % % % % % % % % % % % % % % %
% % % % % % % % % % % % % % % % % % % % % % % % % % % % % % % % %
% % % % % % % % % % % % % % % % % % % % % % % % % % % % % % % % %
% % % % % % % % % % % % % % % % % % % % % % % % % % % % % % % % %
% % % % % % % % % % % % % % % % % % % % % % % % % % % % % % % % %
% % % % % % % % % % % % % % % % % % % % % % % % % % % % % % % % %
% % % % % % % % % % % % % % % % % % % % % % % % % % % % % % % % %
% % % % % % % % % % % % % % % % % % % % % % % % % % % % % % % % %
% % % % % % % % % % % % % % % % % % % % % % % % % % % % % % % % %
% % % % % % % % % % % % % % % % % % % % % % % % % % % % % % % % %
% % % % % % % % % % % % % % % % % % % % % % % % % % % % % % % % %
% % % % % % % % % % % % % % % % % % % % % % % % % % % % % %

\chapter*{Chapter XIII}
\ifaudio     \marginpar{
\href{http://ia902608.us.archive.org/23/items/war_and_peace_02_0801_librivox/war_and_peace_02_13_tolstoy_64kb.mp3}{Audio}
} \fi


\lettrine[lines=2, loversize=0.3, lraise=0]{\initfamily T}{hat}
same night, having taken leave of the Minister of War,
Bolkonski set off to rejoin the army, not knowing where he would
find it and fearing to be captured by the French on the way to
Krems.

In Brunn everybody attached to the court was packing up, and the
heavy baggage was already being dispatched to Olmutz. Near
Hetzelsdorf Prince Andrew struck the high road along which the
Russian army was moving with great haste and in the greatest
disorder. The road was so obstructed with carts that it was
impossible to get by in a carriage. Prince Andrew took a horse
and a Cossack from a Cossack commander, and hungry and weary,
making his way past the baggage wagons, rode in search of the
commander-in-chief and of his own luggage. Very sinister reports
of the position of the army reached him as he went along, and the
appearance of the troops in their disorderly flight confirmed
these rumors.

``Cette armee russe que l'or de l'Angleterre a transportee des
extremites de l'univers, nous allons lui faire eprouver le meme
sort---(le sort de l'armee d'Ulm).''\footnote{ ``That Russian
army which has been brought from the ends of the earth by English
gold, we shall cause to share the same fate---(the fate of the
army at Ulm).''} He remembered these words in Bonaparte's address
to his army at the beginning of the campaign, and they awoke in
him astonishment at the genius of his hero, a feeling of wounded
pride, and a hope of glory. ``And should there be nothing left
but to die?'' he thought. ``Well, if need be, I shall do it no
worse than others.''

He looked with disdain at the endless confused mass of
detachments, carts, guns, artillery, and again baggage wagons and
vehicles of all kinds overtaking one another and blocking the
muddy road, three and sometimes four abreast. From all sides,
behind and before, as far as ear could reach, there were the
rattle of wheels, the creaking of carts and gun carriages, the
tramp of horses, the crack of whips, shouts, the urging of
horses, and the swearing of soldiers, orderlies, and officers.
All along the sides of the road fallen horses were to be seen,
some flayed, some not, and broken-down carts beside which
solitary soldiers sat waiting for something, and again soldiers
straggling from their companies, crowds of whom set off to the
neighboring villages, or returned from them dragging sheep,
fowls, hay, and bulging sacks. At each ascent or descent of the
road the crowds were yet denser and the din of shouting more
incessant. Soldiers floundering knee-deep in mud pushed the guns
and wagons themselves. Whips cracked, hoofs slipped, traces
broke, and lungs were strained with shouting. The officers
directing the march rode backward and forward between the
carts. Their voices were but feebly heard amid the uproar and one
saw by their faces that they despaired of the possibility of
checking this disorder.

``Here is our dear Orthodox Russian army,'' thought Bolkonski,
recalling Bilibin's words.

Wishing to find out where the commander-in-chief was, he rode up
to a convoy. Directly opposite to him came a strange one-horse
vehicle, evidently rigged up by soldiers out of any available
materials and looking like something between a cart, a cabriolet,
and a caleche. A soldier was driving, and a woman enveloped in
shawls sat behind the apron under the leather hood of the
vehicle. Prince Andrew rode up and was just putting his question
to a soldier when his attention was diverted by the desperate
shrieks of the woman in the vehicle. An officer in charge of
transport was beating the soldier who was driving the woman's
vehicle for trying to get ahead of others, and the strokes of his
whip fell on the apron of the equipage. The woman screamed
piercingly. Seeing Prince Andrew she leaned out from behind the
apron and, waving her thin arms from under the woolen shawl,
cried:

``Mr. Aide-de-camp! Mr. Aide-de-camp!... For heaven's
sake... Protect me!  What will become of us? I am the wife of the
doctor of the Seventh Chasseurs... They won't let us pass, we are
left behind and have lost our people...''

``I'll flatten you into a pancake!'' shouted the angry officer to
the soldier. ``Turn back with your slut!''

``Mr. Aide-de-camp! Help me!... What does it all mean?'' screamed
the doctor's wife.

``Kindly let this cart pass. Don't you see it's a woman?'' said
Prince Andrew riding up to the officer.

The officer glanced at him, and without replying turned again to
the soldier. ``I'll teach you to push on!... Back!''

``Let them pass, I tell you!'' repeated Prince Andrew,
compressing his lips.

``And who are you?'' cried the officer, turning on him with tipsy
rage, ``who are you? Are you in command here? Eh? I am commander
here, not you!  Go back or I'll flatten you into a pancake,''
repeated he. This expression evidently pleased him.

``That was a nice snub for the little aide-de-camp,'' came a
voice from behind.

Prince Andrew saw that the officer was in that state of
senseless, tipsy rage when a man does not know what he is
saying. He saw that his championship of the doctor's wife in her
queer trap might expose him to what he dreaded more than anything
in the world---to ridicule; but his instinct urged him on. Before
the officer finished his sentence Prince Andrew, his face
distorted with fury, rode up to him and raised his riding whip.

``Kind...ly let---them---pass!''

The officer flourished his arm and hastily rode away.

``It's all the fault of these fellows on the staff that there's
this disorder,'' he muttered. ``Do as you like.''

Prince Andrew without lifting his eyes rode hastily away from the
doctor's wife, who was calling him her deliverer, and recalling
with a sense of disgust the minutest details of this humiliating
scene he galloped on to the village where he was told that the
commander-in-chief was.

On reaching the village he dismounted and went to the nearest
house, intending to rest if but for a moment, eat something, and
try to sort out the stinging and tormenting thoughts that
confused his mind. ``This is a mob of scoundrels and not an
army,'' he was thinking as he went up to the window of the first
house, when a familiar voice called him by name.

He turned round. Nesvitski's handsome face looked out of the
little window. Nesvitski, moving his moist lips as he chewed
something, and flourishing his arm, called him to enter.

``Bolkonski! Bolkonski!... Don't you hear? Eh? Come quick...'' he
shouted.

Entering the house, Prince Andrew saw Nesvitski and another
adjutant having something to eat. They hastily turned round to
him asking if he had any news. On their familiar faces he read
agitation and alarm. This was particularly noticeable on
Nesvitski's usually laughing countenance.

``Where is the commander-in-chief?'' asked Bolkonski.

``Here, in that house,'' answered the adjutant.

``Well, is it true that it's peace and capitulation?'' asked
Nesvitski.

``I was going to ask you. I know nothing except that it was all I
could do to get here.''

``And we, my dear boy! It's terrible! I was wrong to laugh at
Mack, we're getting it still worse,'' said Nesvitski. ``But sit
down and have something to eat.''

``You won't be able to find either your baggage or anything else
now, Prince. And God only knows where your man Peter is,'' said
the other adjutant.

``Where are headquarters?''

``We are to spend the night in Znaim.''

``Well, I have got all I need into packs for two horses,'' said
Nesvitski.  ``They've made up splendid packs for me---fit to
cross the Bohemian mountains with. It's a bad lookout, old
fellow! But what's the matter with you? You must be ill to shiver
like that,'' he added, noticing that Prince Andrew winced as at
an electric shock.

``It's nothing,'' replied Prince Andrew.

He had just remembered his recent encounter with the doctor's
wife and the convoy officer.

``What is the commander-in-chief doing here?'' he asked.

``I can't make out at all,'' said Nesvitski.

``Well, all I can make out is that everything is abominable,
abominable, quite abominable!'' said Prince Andrew, and he went
off to the house where the commander-in-chief was.

Passing by Kutuzov's carriage and the exhausted saddle horses of
his suite, with their Cossacks who were talking loudly together,
Prince Andrew entered the passage. Kutuzov himself, he was told,
was in the house with Prince Bagration and Weyrother. Weyrother
was the Austrian general who had succeeded Schmidt. In the
passage little Kozlovski was squatting on his heels in front of a
clerk. The clerk, with cuffs turned up, was hastily writing at a
tub turned bottom upwards. Kozlovski's face looked worn---he too
had evidently not slept all night. He glanced at Prince Andrew
and did not even nod to him.

``Second line... have you written it?'' he continued dictating to
the clerk. ``The Kiev Grenadiers, Podolian...''

``One can't write so fast, your honor,'' said the clerk, glancing
angrily and disrespectfully at Kozlovski.

Through the door came the sounds of Kutuzov's voice, excited and
dissatisfied, interrupted by another, an unfamiliar voice. From
the sound of these voices, the inattentive way Kozlovski looked
at him, the disrespectful manner of the exhausted clerk, the fact
that the clerk and Kozlovski were squatting on the floor by a tub
so near to the commander in chief, and from the noisy laughter of
the Cossacks holding the horses near the window, Prince Andrew
felt that something important and disastrous was about to happen.

He turned to Kozlovski with urgent questions.

``Immediately, Prince,'' said Kozlovski. ``Dispositions for
Bagration.''

``What about capitulation?''

``Nothing of the sort. Orders are issued for a battle.''

Prince Andrew moved toward the door from whence voices were
heard. Just as he was going to open it the sounds ceased, the
door opened, and Kutuzov with his eagle nose and puffy face
appeared in the doorway.  Prince Andrew stood right in front of
Kutuzov but the expression of the commander in chief's one sound
eye showed him to be so preoccupied with thoughts and anxieties
as to be oblivious of his presence. He looked straight at his
adjutant's face without recognizing him.

``Well, have you finished?'' said he to Kozlovski.

``One moment, your excellency.''

Bagration, a gaunt middle-aged man of medium height with a firm,
impassive face of Oriental type, came out after the
commander-in-chief.

``I have the honor to present myself,'' repeated Prince Andrew
rather loudly, handing Kutuzov an envelope.

``Ah, from Vienna? Very good. Later, later!''

Kutuzov went out into the porch with Bagration.

``Well, good-by, Prince,'' said he to Bagration. ``My blessing,
and may Christ be with you in your great endeavor!''

His face suddenly softened and tears came into his eyes. With his
left hand he drew Bagration toward him, and with his right, on
which he wore a ring, he made the sign of the cross over him with
a gesture evidently habitual, offering his puffy cheek, but
Bagration kissed him on the neck instead.

``Christ be with you!'' Kutuzov repeated and went toward his
carriage.  ``Get in with me,'' said he to Bolkonski.

``Your excellency, I should like to be of use here. Allow me to
remain with Prince Bagration's detachment.''

``Get in,'' said Kutuzov, and noticing that Bolkonski still
delayed, he added: ``I need good officers myself, need them
myself!''

They got into the carriage and drove for a few minutes in
silence.

``There is still much, much before us,'' he said, as if with an
old man's penetration he understood all that was passing in
Bolkonski's mind. ``If a tenth part of his detachment returns I
shall thank God,'' he added as if speaking to himself.

Prince Andrew glanced at Kutuzov's face only a foot distant from
him and involuntarily noticed the carefully washed seams of the
scar near his temple, where an Ismail bullet had pierced his
skull, and the empty eye socket. ``Yes, he has a right to speak
so calmly of those men's death,'' thought Bolkonski.

``That is why I beg to be sent to that detachment,'' he said.

Kutuzov did not reply. He seemed to have forgotten what he had
been saying, and sat plunged in thought. Five minutes later,
gently swaying on the soft springs of the carriage, he turned to
Prince Andrew. There was not a trace of agitation on his
face. With delicate irony he questioned Prince Andrew about the
details of his interview with the Emperor, about the remarks he
had heard at court concerning the Krems affair, and about some
ladies they both knew.

% % % % % % % % % % % % % % % % % % % % % % % % % % % % % % % % %
% % % % % % % % % % % % % % % % % % % % % % % % % % % % % % % % %
% % % % % % % % % % % % % % % % % % % % % % % % % % % % % % % % %
% % % % % % % % % % % % % % % % % % % % % % % % % % % % % % % % %
% % % % % % % % % % % % % % % % % % % % % % % % % % % % % % % % %
% % % % % % % % % % % % % % % % % % % % % % % % % % % % % % % % %
% % % % % % % % % % % % % % % % % % % % % % % % % % % % % % % % %
% % % % % % % % % % % % % % % % % % % % % % % % % % % % % % % % %
% % % % % % % % % % % % % % % % % % % % % % % % % % % % % % % % %
% % % % % % % % % % % % % % % % % % % % % % % % % % % % % % % % %
% % % % % % % % % % % % % % % % % % % % % % % % % % % % % % % % %
% % % % % % % % % % % % % % % % % % % % % % % % % % % % % %

\chapter*{Chapter XIV}
\ifaudio     \marginpar{
\href{http://ia902608.us.archive.org/23/items/war_and_peace_02_0801_librivox/war_and_peace_02_14_tolstoy_64kb.mp3}{Audio}
} \fi

\lettrine[lines=2, loversize=0.3, lraise=0]{\initfamily O}{n}
November 1 Kutuzov had received, through a spy, news that the
army he commanded was in an almost hopeless position. The spy
reported that the French, after crossing the bridge at Vienna,
were advancing in immense force upon Kutuzov's line of
communication with the troops that were arriving from Russia. If
Kutuzov decided to remain at Krems, Napoleon's army of one
hundred and fifty thousand men would cut him off completely and
surround his exhausted army of forty thousand, and he would find
himself in the position of Mack at Ulm. If Kutuzov decided to
abandon the road connecting him with the troops arriving from
Russia, he would have to march with no road into unknown parts of
the Bohemian mountains, defending himself against superior forces
of the enemy and abandoning all hope of a junction with
Buxhowden. If Kutuzov decided to retreat along the road from
Krems to Olmutz, to unite with the troops arriving from Russia,
he risked being forestalled on that road by the French who had
crossed the Vienna bridge, and encumbered by his baggage and
transport, having to accept battle on the march against an enemy
three times as strong, who would hem him in from two sides.

Kutuzov chose this latter course.

The French, the spy reported, having crossed the Vienna bridge,
were advancing by forced marches toward Znaim, which lay
sixty-six miles off on the line of Kutuzov's retreat. If he
reached Znaim before the French, there would be great hope of
saving the army; to let the French forestall him at Znaim meant
the exposure of his whole army to a disgrace such as that of Ulm,
or to utter destruction. But to forestall the French with his
whole army was impossible. The road for the French from Vienna to
Znaim was shorter and better than the road for the Russians from
Krems to Znaim.

The night he received the news, Kutuzov sent Bagration's
vanguard, four thousand strong, to the right across the hills
from the Krems-Znaim to the Vienna-Znaim road. Bagration was to
make this march without resting, and to halt facing Vienna with
Znaim to his rear, and if he succeeded in forestalling the French
he was to delay them as long as possible.  Kutuzov himself with
all his transport took the road to Znaim.

Marching thirty miles that stormy night across roadless hills,
with his hungry, ill-shod soldiers, and losing a third of his men
as stragglers by the way, Bagration came out on the Vienna-Znaim
road at Hollabrunn a few hours ahead of the French who were
approaching Hollabrunn from Vienna. Kutuzov with his transport
had still to march for some days before he could reach
Znaim. Hence Bagration with his four thousand hungry, exhausted
men would have to detain for days the whole enemy army that came
upon him at Hollabrunn, which was clearly impossible. But a freak
of fate made the impossible possible. The success of the trick
that had placed the Vienna bridge in the hands of the French
without a fight led Murat to try to deceive Kutuzov in a similar
way. Meeting Bagration's weak detachment on the Znaim road he
supposed it to be Kutuzov's whole army. To be able to crush it
absolutely he awaited the arrival of the rest of the troops who
were on their way from Vienna, and with this object offered a
three days' truce on condition that both armies should remain in
position without moving. Murat declared that negotiations for
peace were already proceeding, and that he therefore offered this
truce to avoid unnecessary bloodshed. Count Nostitz, the Austrian
general occupying the advanced posts, believed Murat's emissary
and retired, leaving Bagration's division exposed. Another
emissary rode to the Russian line to announce the peace
negotiations and to offer the Russian army the three days'
truce. Bagration replied that he was not authorized either to
accept or refuse a truce and sent his adjutant to Kutuzov to
report the offer he had received.

A truce was Kutuzov's sole chance of gaining time, giving
Bagration's exhausted troops some rest, and letting the transport
and heavy convoys (whose movements were concealed from the
French) advance if but one stage nearer Znaim. The offer of a
truce gave the only, and a quite unexpected, chance of saving the
army. On receiving the news he immediately dispatched Adjutant
General Wintzingerode, who was in attendance on him, to the enemy
camp. Wintzingerode was not merely to agree to the truce but also
to offer terms of capitulation, and meanwhile Kutuzov sent his
adjutants back to hasten to the utmost the movements of the
baggage trains of the entire army along the Krems-Znaim
road. Bagration's exhausted and hungry detachment, which alone
covered this movement of the transport and of the whole army, had
to remain stationary in face of an enemy eight times as strong as
itself.

Kutuzov's expectations that the proposals of capitulation (which
were in no way binding) might give time for part of the transport
to pass, and also that Murat's mistake would very soon be
discovered, proved correct.  As soon as Bonaparte (who was at
Schonbrunn, sixteen miles from Hollabrunn) received Murat's
dispatch with the proposal of a truce and a capitulation, he
detected a ruse and wrote the following letter to Murat:

\begin{quote} \calli

Schonbrunn, 25th Brumaire, 1805,

at eight o'clock in the morning

To \textsc{Prince Murat},

I cannot find words to express to you my displeasure. You command
only my advance guard, and have no right to arrange an armistice
without my order. You are causing me to lose the fruits of a
campaign. Break the armistice immediately and march on the
enemy. Inform him that the general who signed that capitulation
had no right to do so, and that no one but the Emperor of Russia
has that right.

If, however, the Emperor of Russia ratifies that convention, I
will ratify it; but it is only a trick. March on, destroy the
Russian army... You are in a position to seize its baggage and
artillery.

The Russian Emperor's aide-de-camp is an impostor. Officers are
nothing when they have no powers; this one had none... The
Austrians let themselves be tricked at the crossing of the Vienna
bridge, you are letting yourself be tricked by an aide-de-camp of
the Emperor.

\textsc{Napoleon}
\end{quote}

Bonaparte's adjutant rode full gallop with this menacing letter
to Murat. Bonaparte himself, not trusting to his generals, moved
with all the Guards to the field of battle, afraid of letting a
ready victim escape, and Bagration's four thousand men merrily
lighted campfires, dried and warmed themselves, cooked their
porridge for the first time for three days, and not one of them
knew or imagined what was in store for him.

% % % % % % % % % % % % % % % % % % % % % % % % % % % % % % % % %
% % % % % % % % % % % % % % % % % % % % % % % % % % % % % % % % %
% % % % % % % % % % % % % % % % % % % % % % % % % % % % % % % % %
% % % % % % % % % % % % % % % % % % % % % % % % % % % % % % % % %
% % % % % % % % % % % % % % % % % % % % % % % % % % % % % % % % %
% % % % % % % % % % % % % % % % % % % % % % % % % % % % % % % % %
% % % % % % % % % % % % % % % % % % % % % % % % % % % % % % % % %
% % % % % % % % % % % % % % % % % % % % % % % % % % % % % % % % %
% % % % % % % % % % % % % % % % % % % % % % % % % % % % % % % % %
% % % % % % % % % % % % % % % % % % % % % % % % % % % % % % % % %
% % % % % % % % % % % % % % % % % % % % % % % % % % % % % % % % %
% % % % % % % % % % % % % % % % % % % % % % % % % % % % % %

\chapter*{Chapter XV}
\ifaudio     \marginpar{
\href{http://ia902608.us.archive.org/23/items/war_and_peace_02_0801_librivox/war_and_peace_02_15_tolstoy_64kb.mp3}{Audio}
} \fi

\lettrine[lines=2, loversize=0.3, lraise=0]{\initfamily B}{etween}
three and four o'clock in the afternoon Prince Andrew,
who had persisted in his request to Kutuzov, arrived at Grunth
and reported himself to Bagration. Bonaparte's adjutant had not
yet reached Murat's detachment and the battle had not yet
begun. In Bagration's detachment no one knew anything of the
general position of affairs. They talked of peace but did not
believe in its possibility; others talked of a battle but also
disbelieved in the nearness of an engagement. Bagration, knowing
Bolkonski to be a favorite and trusted adjutant, received him
with distinction and special marks of favor, explaining to him
that there would probably be an engagement that day or the next,
and giving him full liberty to remain with him during the battle
or to join the rearguard and have an eye on the order of retreat,
\emph{which is also very important}.

``However, there will hardly be an engagement today,'' said
Bagration as if to reassure Prince Andrew.

``If he is one of the ordinary little staff dandies sent to earn
a medal he can get his reward just as well in the rearguard, but
if he wishes to stay with me, let him... he'll be of use here if
he's a brave officer,'' thought Bagration. Prince Andrew, without
replying, asked the prince's permission to ride round the
position to see the disposition of the forces, so as to know his
bearings should he be sent to execute an order. The officer on
duty, a handsome, elegantly dressed man with a diamond ring on
his forefinger, who was fond of speaking French though he spoke
it badly, offered to conduct Prince Andrew.

On all sides they saw rain-soaked officers with dejected faces
who seemed to be seeking something, and soldiers dragging doors,
benches, and fencing from the village.

``There now, Prince! We can't stop those fellows,'' said the
staff officer pointing to the soldiers. ``The officers don't keep
them in hand. And there,'' he pointed to a sutler's tent, ``they
crowd in and sit. This morning I turned them all out and now
look, it's full again. I must go there, Prince, and scare them a
bit. It won't take a moment.''

``Yes, let's go in and I will get myself a roll and some
cheese,'' said Prince Andrew who had not yet had time to eat
anything.

``Why didn't you mention it, Prince? I would have offered you
something.''

They dismounted and entered the tent. Several officers, with
flushed and weary faces, were sitting at the table eating and
drinking.

``Now what does this mean, gentlemen?'' said the staff officer,
in the reproachful tone of a man who has repeated the same thing
more than once. ``You know it won't do to leave your posts like
this. The prince gave orders that no one should leave his
post. Now you, Captain,'' and he turned to a thin, dirty little
artillery officer who without his boots (he had given them to the
canteen keeper to dry), in only his stockings, rose when they
entered, smiling not altogether comfortably.

``Well, aren't you ashamed of yourself, Captain Tushin?'' he
continued.  ``One would think that as an artillery officer you
would set a good example, yet here you are without your boots!
The alarm will be sounded and you'll be in a pretty position
without your boots!'' (The staff officer smiled.) ``Kindly return
to your posts, gentlemen, all of you, all!'' he added in a tone
of command.

Prince Andrew smiled involuntarily as he looked at the artillery
officer Tushin, who silent and smiling, shifting from one
stockinged foot to the other, glanced inquiringly with his large,
intelligent, kindly eyes from Prince Andrew to the staff officer.

``The soldiers say it feels easier without boots,'' said Captain
Tushin smiling shyly in his uncomfortable position, evidently
wishing to adopt a jocular tone. But before he had finished he
felt that his jest was unacceptable and had not come off. He grew
confused.

``Kindly return to your posts,'' said the staff officer trying to
preserve his gravity.

Prince Andrew glanced again at the artillery officer's small
figure.  There was something peculiar about it, quite
unsoldierly, rather comic, but extremely attractive.

The staff officer and Prince Andrew mounted their horses and rode
on.

Having ridden beyond the village, continually meeting and
overtaking soldiers and officers of various regiments, they saw
on their left some entrenchments being thrown up, the freshly dug
clay of which showed up red. Several battalions of soldiers, in
their shirt sleeves despite the cold wind, swarmed in these
earthworks like a host of white ants; spadefuls of red clay were
continually being thrown up from behind the bank by unseen
hands. Prince Andrew and the officer rode up, looked at the
entrenchment, and went on again. Just behind it they came upon
some dozens of soldiers, continually replaced by others, who ran
from the entrenchment. They had to hold their noses and put their
horses to a trot to escape from the poisoned atmosphere of these
latrines.

``Voila l'agrement des camps, monsieur le
Prince,''\footnote{``This is a pleasure one gets in camp,
Prince.''} said the staff officer.

They rode up the opposite hill. From there the French could
already be seen. Prince Andrew stopped and began examining the
position.

``That's our battery,'' said the staff officer indicating the
highest point. ``It's in charge of the queer fellow we saw
without his boots. You can see everything from there; let's go
there, Prince.''

``Thank you very much, I will go on alone,'' said Prince Andrew,
wishing to rid himself of this staff officer's company, ``please
don't trouble yourself further.''

The staff officer remained behind and Prince Andrew rode on
alone.

The farther forward and nearer the enemy he went, the more
orderly and cheerful were the troops. The greatest disorder and
depression had been in the baggage train he had passed that
morning on the Znaim road seven miles away from the French. At
Grunth also some apprehension and alarm could be felt, but the
nearer Prince Andrew came to the French lines the more confident
was the appearance of our troops. The soldiers in their
greatcoats were ranged in lines, the sergeants major and company
officers were counting the men, poking the last man in each
section in the ribs and telling him to hold his hand up. Soldiers
scattered over the whole place were dragging logs and brushwood
and were building shelters with merry chatter and laughter;
around the fires sat others, dressed and undressed, drying their
shirts and leg bands or mending boots or overcoats and crowding
round the boilers and porridge cookers.  In one company dinner
was ready, and the soldiers were gazing eagerly at the steaming
boiler, waiting till the sample, which a quartermaster sergeant
was carrying in a wooden bowl to an officer who sat on a log
before his shelter, had been tasted.

Another company, a lucky one for not all the companies had vodka,
crowded round a pockmarked, broad-shouldered sergeant major who,
tilting a keg, filled one after another the canteen lids held out
to him. The soldiers lifted the canteen lids to their lips with
reverential faces, emptied them, rolling the vodka in their
mouths, and walked away from the sergeant major with brightened
expressions, licking their lips and wiping them on the sleeves of
their greatcoats. All their faces were as serene as if all this
were happening at home awaiting peaceful encampment, and not
within sight of the enemy before an action in which at least half
of them would be left on the field. After passing a chasseur
regiment and in the lines of the Kiev grenadiers---fine fellows
busy with similar peaceful affairs---near the shelter of the
regimental commander, higher than and different from the others,
Prince Andrew came out in front of a platoon of grenadiers before
whom lay a naked man. Two soldiers held him while two others were
flourishing their switches and striking him regularly on his bare
back. The man shrieked unnaturally. A stout major was pacing up
and down the line, and regardless of the screams kept repeating:

``It's a shame for a soldier to steal; a soldier must be honest,
honorable, and brave, but if he robs his fellows there is no
honor in him, he's a scoundrel. Go on! Go on!''

So the swishing sound of the strokes, and the desperate but
unnatural screams, continued.

``Go on, go on!'' said the major.

A young officer with a bewildered and pained expression on his
face stepped away from the man and looked round inquiringly at
the adjutant as he rode by.

Prince Andrew, having reached the front line, rode along it. Our
front line and that of the enemy were far apart on the right and
left flanks, but in the center where the men with a flag of truce
had passed that morning, the lines were so near together that the
men could see one another's faces and speak to one
another. Besides the soldiers who formed the picket line on
either side, there were many curious onlookers who, jesting and
laughing, stared at their strange foreign enemies.

Since early morning---despite an injunction not to approach the
picket line---the officers had been unable to keep sight-seers
away. The soldiers forming the picket line, like showmen
exhibiting a curiosity, no longer looked at the French but paid
attention to the sight-seers and grew weary waiting to be
relieved. Prince Andrew halted to have a look at the French.

``Look! Look there!'' one soldier was saying to another, pointing
to a Russian musketeer who had gone up to the picket line with an
officer and was rapidly and excitedly talking to a French
grenadier. ``Hark to him jabbering! Fine, isn't it? It's all the
Frenchy can do to keep up with him. There now, Sidorov!''

``Wait a bit and listen. It's fine!'' answered Sidorov, who was
considered an adept at French.

The soldier to whom the laughers referred was Dolokhov. Prince
Andrew recognized him and stopped to listen to what he was
saying. Dolokhov had come from the left flank where their
regiment was stationed, with his captain.

``Now then, go on, go on!'' incited the officer, bending forward
and trying not to lose a word of the speech which was
incomprehensible to him. ``More, please: more! What's he
saying?''

Dolokhov did not answer the captain; he had been drawn into a hot
dispute with the French grenadier. They were naturally talking
about the campaign. The Frenchman, confusing the Austrians with
the Russians, was trying to prove that the Russians had
surrendered and had fled all the way from Ulm, while Dolokhov
maintained that the Russians had not surrendered but had beaten
the French.

``We have orders to drive you off here, and we shall drive you
off,'' said Dolokhov.

``Only take care you and your Cossacks are not all captured!''
said the French grenadier.

The French onlookers and listeners laughed.

``We'll make you dance as we did under
Suvorov...,''\footnote{``On vous fera danser.''} said Dolokhov.

``Qu' est-ce qu'il chante?''\footnote{ ``What's he singing
about?''} asked a Frenchman.

``It's ancient history,'' said another, guessing that it referred
to a former war. ``The Emperor will teach your Suvara as he has
taught the others...''

``Bonaparte...'' began Dolokhov, but the Frenchman interrupted
him.

``Not Bonaparte. He is the Emperor! Sacre nom...!'' cried he
angrily.

``The devil skin your Emperor.''

And Dolokhov swore at him in coarse soldier's Russian and
shouldering his musket walked away.

``Let us go, Ivan Lukich,'' he said to the captain.

``Ah, that's the way to talk French,'' said the picket
soldiers. ``Now, Sidorov, you have a try!''

Sidorov, turning to the French, winked, and began to jabber
meaningless sounds very fast: ``Kari, mala, tafa, safi, muter,
Kaska,'' he said, trying to give an expressive intonation to his
voice.

``Ho! ho! ho! Ha! ha! ha! ha! Ouh! ouh!'' came peals of such
healthy and good-humored laughter from the soldiers that it
infected the French involuntarily, so much so that the only thing
left to do seemed to be to unload the muskets, explode the
ammunition, and all return home as quickly as possible.

But the guns remained loaded, the loopholes in blockhouses and
entrenchments looked out just as menacingly, and the unlimbered
cannon confronted one another as before.

% % % % % % % % % % % % % % % % % % % % % % % % % % % % % % % % %
% % % % % % % % % % % % % % % % % % % % % % % % % % % % % % % % %
% % % % % % % % % % % % % % % % % % % % % % % % % % % % % % % % %
% % % % % % % % % % % % % % % % % % % % % % % % % % % % % % % % %
% % % % % % % % % % % % % % % % % % % % % % % % % % % % % % % % %
% % % % % % % % % % % % % % % % % % % % % % % % % % % % % % % % %
% % % % % % % % % % % % % % % % % % % % % % % % % % % % % % % % %
% % % % % % % % % % % % % % % % % % % % % % % % % % % % % % % % %
% % % % % % % % % % % % % % % % % % % % % % % % % % % % % % % % %
% % % % % % % % % % % % % % % % % % % % % % % % % % % % % % % % %
% % % % % % % % % % % % % % % % % % % % % % % % % % % % % % % % %
% % % % % % % % % % % % % % % % % % % % % % % % % % % % % %

\chapter*{Chapter XVI}
\ifaudio     \marginpar{
\href{http://ia902608.us.archive.org/23/items/war_and_peace_02_0801_librivox/war_and_peace_02_16_tolstoy_64kb.mp3}{Audio}
} \fi

\lettrine[lines=2, loversize=0.3, lraise=0]{\initfamily H}{aving}
ridden round the whole line from right flank to left,
Prince Andrew made his way up to the battery from which the staff
officer had told him the whole field could be seen. Here he
dismounted, and stopped beside the farthest of the four
unlimbered cannon. Before the guns an artillery sentry was pacing
up and down; he stood at attention when the officer arrived, but
at a sign resumed his measured, monotonous pacing.  Behind the
guns were their limbers and still farther back picket ropes and
artillerymen's bonfires. To the left, not far from the farthest
cannon, was a small, newly constructed wattle shed from which
came the sound of officers' voices in eager conversation.

It was true that a view over nearly the whole Russian position
and the greater part of the enemy's opened out from this
battery. Just facing it, on the crest of the opposite hill, the
village of Schon Grabern could be seen, and in three places to
left and right the French troops amid the smoke of their
campfires, the greater part of whom were evidently in the village
itself and behind the hill. To the left from that village, amid
the smoke, was something resembling a battery, but it was
impossible to see it clearly with the naked eye. Our right flank
was posted on a rather steep incline which dominated the French
position.  Our infantry were stationed there, and at the farthest
point the dragoons. In the center, where Tushin's battery stood
and from which Prince Andrew was surveying the position, was the
easiest and most direct descent and ascent to the brook
separating us from Schon Grabern.  On the left our troops were
close to a copse, in which smoked the bonfires of our infantry
who were felling wood. The French line was wider than ours, and
it was plain that they could easily outflank us on both
sides. Behind our position was a steep and deep dip, making it
difficult for artillery and cavalry to retire. Prince Andrew took
out his notebook and, leaning on the cannon, sketched a plan of
the position. He made some notes on two points, intending to
mention them to Bagration. His idea was, first, to concentrate
all the artillery in the center, and secondly, to withdraw the
cavalry to the other side of the dip. Prince Andrew, being always
near the commander in chief, closely following the mass movements
and general orders, and constantly studying historical accounts
of battles, involuntarily pictured to himself the course of
events in the forthcoming action in broad outline. He imagined
only important possibilities: ``If the enemy attacks the right
flank,'' he said to himself, ``the Kiev grenadiers and the
Podolsk chasseurs must hold their position till reserves from the
center come up. In that case the dragoons could successfully make
a flank counterattack. If they attack our center we, having the
center battery on this high ground, shall withdraw the left flank
under its cover, and retreat to the dip by echelons.'' So he
reasoned... All the time he had been beside the gun, he had heard
the voices of the officers distinctly, but as often happens had
not understood a word of what they were saying. Suddenly,
however, he was struck by a voice coming from the shed, and its
tone was so sincere that he could not but listen.

``No, friend,'' said a pleasant and, as it seemed to Prince
Andrew, a familiar voice, ``what I say is that if it were
possible to know what is beyond death, none of us would be afraid
of it. That's so, friend.''

Another, a younger voice, interrupted him: ``Afraid or not, you
can't escape it anyhow.''

``All the same, one is afraid! Oh, you clever people,'' said a
third manly voice interrupting them both. ``Of course you
artillery men are very wise, because you can take everything
along with you---vodka and snacks.''

And the owner of the manly voice, evidently an infantry officer,
laughed.

``Yes, one is afraid,'' continued the first speaker, he of the
familiar voice. ``One is afraid of the unknown, that's what it
is. Whatever we may say about the soul going to the sky... we
know there is no sky but only an atmosphere.''

The manly voice again interrupted the artillery officer.

``Well, stand us some of your herb vodka, Tushin,'' it said.

``Why,'' thought Prince Andrew, ``that's the captain who stood up
in the sutler's hut without his boots.'' He recognized the
agreeable, philosophizing voice with pleasure.

``Some herb vodka? Certainly!'' said Tushin. ``But still, to
conceive a future life...''

He did not finish. Just then there was a whistle in the air;
nearer and nearer, faster and louder, louder and faster, a cannon
ball, as if it had not finished saying what was necessary,
thudded into the ground near the shed with super human force,
throwing up a mass of earth. The ground seemed to groan at the
terrible impact.

And immediately Tushin, with a short pipe in the corner of his
mouth and his kind, intelligent face rather pale, rushed out of
the shed followed by the owner of the manly voice, a dashing
infantry officer who hurried off to his company, buttoning up his
coat as he ran.

% % % % % % % % % % % % % % % % % % % % % % % % % % % % % % % % %
% % % % % % % % % % % % % % % % % % % % % % % % % % % % % % % % %
% % % % % % % % % % % % % % % % % % % % % % % % % % % % % % % % %
% % % % % % % % % % % % % % % % % % % % % % % % % % % % % % % % %
% % % % % % % % % % % % % % % % % % % % % % % % % % % % % % % % %
% % % % % % % % % % % % % % % % % % % % % % % % % % % % % % % % %
% % % % % % % % % % % % % % % % % % % % % % % % % % % % % % % % %
% % % % % % % % % % % % % % % % % % % % % % % % % % % % % % % % %
% % % % % % % % % % % % % % % % % % % % % % % % % % % % % % % % %
% % % % % % % % % % % % % % % % % % % % % % % % % % % % % % % % %
% % % % % % % % % % % % % % % % % % % % % % % % % % % % % % % % %
% % % % % % % % % % % % % % % % % % % % % % % % % % % % % %

\chapter*{Chapter XVII}
\ifaudio     \marginpar{
\href{http://ia902608.us.archive.org/23/items/war_and_peace_02_0801_librivox/war_and_peace_02_17_tolstoy_64kb.mp3}{Audio}
} \fi

\lettrine[lines=2, loversize=0.3, lraise=0]{\initfamily M}{ounting}
his horse again Prince Andrew lingered with the battery,
looking at the puff from the gun that had sent the ball. His eyes
ran rapidly over the wide space, but he only saw that the
hitherto motionless masses of the French now swayed and that
there really was a battery to their left. The smoke above it had
not yet dispersed. Two mounted Frenchmen, probably adjutants,
were galloping up the hill. A small but distinctly visible enemy
column was moving down the hill, probably to strengthen the front
line. The smoke of the first shot had not yet dispersed before
another puff appeared, followed by a report.  The battle had
begun! Prince Andrew turned his horse and galloped back to Grunth
to find Prince Bagration. He heard the cannonade behind him
growing louder and more frequent. Evidently our guns had begun to
reply.  From the bottom of the slope, where the parleys had taken
place, came the report of musketry.

Lemarrois had just arrived at a gallop with Bonaparte's stern
letter, and Murat, humiliated and anxious to expiate his fault,
had at once moved his forces to attack the center and outflank
both the Russian wings, hoping before evening and before the
arrival of the Emperor to crush the contemptible detachment that
stood before him.

``It has begun. Here it is!'' thought Prince Andrew, feeling the
blood rush to his heart. ``But where and how will my Toulon
present itself?''

Passing between the companies that had been eating porridge and
drinking vodka a quarter of an hour before, he saw everywhere the
same rapid movement of soldiers forming ranks and getting their
muskets ready, and on all their faces he recognized the same
eagerness that filled his heart. ``It has begun! Here it is,
dreadful but enjoyable!'' was what the face of each soldier and
each officer seemed to say.

Before he had reached the embankments that were being thrown up,
he saw, in the light of the dull autumn evening, mounted men
coming toward him.  The foremost, wearing a Cossack cloak and
lambskin cap and riding a white horse, was Prince
Bagration. Prince Andrew stopped, waiting for him to come up;
Prince Bagration reined in his horse and recognizing Prince
Andrew nodded to him. He still looked ahead while Prince Andrew
told him what he had seen.

The feeling, ``It has begun! Here it is!'' was seen even on
Prince Bagration's hard brown face with its half-closed, dull,
sleepy eyes.  Prince Andrew gazed with anxious curiosity at that
impassive face and wished he could tell what, if anything, this
man was thinking and feeling at that moment. ``Is there anything
at all behind that impassive face?'' Prince Andrew asked himself
as he looked. Prince Bagration bent his head in sign of agreement
with what Prince Andrew told him, and said, ``Very good!'' in a
tone that seemed to imply that everything that took place and was
reported to him was exactly what he had foreseen.  Prince Andrew,
out of breath with his rapid ride, spoke quickly. Prince
Bagration, uttering his words with an Oriental accent, spoke
particularly slowly, as if to impress the fact that there was no
need to hurry. However, he put his horse to a trot in the
direction of Tushin's battery. Prince Andrew followed with the
suite. Behind Prince Bagration rode an officer of the suite, the
prince's personal adjutant, Zherkov, an orderly officer, the
staff officer on duty, riding a fine bobtailed horse, and a
civilian---an accountant who had asked permission to be present
at the battle out of curiosity. The accountant, a stout,
full-faced man, looked around him with a naive smile of
satisfaction and presented a strange appearance among the
hussars, Cossacks, and adjutants, in his camlet coat, as he
jolted on his horse with a convoy officer's saddle.

``He wants to see a battle,'' said Zherkov to Bolkonski, pointing
to the accountant, ``but he feels a pain in the pit of his
stomach already.''

``Oh, leave off!'' said the accountant with a beaming but rather
cunning smile, as if flattered at being made the subject of
Zherkov's joke, and purposely trying to appear stupider than he
really was.

``It is very strange, mon Monsieur Prince,'' said the staff
officer. (He remembered that in French there is some peculiar way
of addressing a prince, but could not get it quite right.)

By this time they were all approaching Tushin's battery, and a
ball struck the ground in front of them.

``What's that that has fallen?'' asked the accountant with a
naive smile.

``A French pancake,'' answered Zherkov.

``So that's what they hit with?'' asked the accountant. ``How
awful!''

He seemed to swell with satisfaction. He had hardly finished
speaking when they again heard an unexpectedly violent whistling
which suddenly ended with a thud into something soft... f-f-flop!
and a Cossack, riding a little to their right and behind the
accountant, crashed to earth with his horse. Zherkov and the
staff officer bent over their saddles and turned their horses
away. The accountant stopped, facing the Cossack, and examined
him with attentive curiosity. The Cossack was dead, but the horse
still struggled.

Prince Bagration screwed up his eyes, looked round, and, seeing
the cause of the confusion, turned away with indifference, as if
to say, ``Is it worth while noticing trifles?'' He reined in his
horse with the care of a skillful rider and, slightly bending
over, disengaged his saber which had caught in his cloak. It was
an old-fashioned saber of a kind no longer in general use. Prince
Andrew remembered the story of Suvorov giving his saber to
Bagration in Italy, and the recollection was particularly
pleasant at that moment. They had reached the battery at which
Prince Andrew had been when he examined the battlefield.

``Whose company?'' asked Prince Bagration of an artilleryman
standing by the ammunition wagon.

He asked, ``Whose company?'' but he really meant, ``Are you
frightened here?'' and the artilleryman understood him.

``Captain Tushin's, your excellency!'' shouted the red-haired,
freckled gunner in a merry voice, standing to attention.

``Yes, yes,'' muttered Bagration as if considering something, and
he rode past the limbers to the farthest cannon.

As he approached, a ringing shot issued from it deafening him and
his suite, and in the smoke that suddenly surrounded the gun they
could see the gunners who had seized it straining to roll it
quickly back to its former position. A huge, broad-shouldered
gunner, Number One, holding a mop, his legs far apart, sprang to
the wheel; while Number Two with a trembling hand placed a charge
in the cannon's mouth. The short, round-shouldered Captain
Tushin, stumbling over the tail of the gun carriage, moved
forward and, not noticing the general, looked out shading his
eyes with his small hand.

``Lift it two lines more and it will be just right,'' cried he in
a feeble voice to which he tried to impart a dashing note,
ill-suited to his weak figure. ``Number Two!'' he
squeaked. ``Fire, Medvedev!''

Bagration called to him, and Tushin, raising three fingers to his
cap with a bashful and awkward gesture not at all like a military
salute but like a priest's benediction, approached the
general. Though Tushin's guns had been intended to cannonade the
valley, he was firing incendiary balls at the village of Schon
Grabern visible just opposite, in front of which large masses of
French were advancing.

No one had given Tushin orders where and at what to fire, but
after consulting his sergeant major, Zakharchenko, for whom he
had great respect, he had decided that it would be a good thing
to set fire to the village. ``Very good!'' said Bagration in
reply to the officer's report, and began deliberately to examine
the whole battlefield extended before him. The French had
advanced nearest on our right. Below the height on which the Kiev
regiment was stationed, in the hollow where the rivulet flowed,
the soul-stirring rolling and crackling of musketry was heard,
and much farther to the right beyond the dragoons, the officer of
the suite pointed out to Bagration a French column that was
outflanking us.  To the left the horizon bounded by the adjacent
wood. Prince Bagration ordered two battalions from the center to
be sent to reinforce the right flank. The officer of the suite
ventured to remark to the prince that if these battalions went
away, the guns would remain without support.  Prince Bagration
turned to the officer and with his dull eyes looked at him in
silence. It seemed to Prince Andrew that the officer's remark was
just and that really no answer could be made to it. But at that
moment an adjutant galloped up with a message from the commander
of the regiment in the hollow and news that immense masses of the
French were coming down upon them and that his regiment was in
disorder and was retreating upon the Kiev grenadiers. Prince
Bagration bowed his head in sign of assent and approval. He rode
off at a walk to the right and sent an adjutant to the dragoons
with orders to attack the French. But this adjutant returned half
an hour later with the news that the commander of the dragoons
had already retreated beyond the dip in the ground, as a heavy
fire had been opened on him and he was losing men uselessly, and
so had hastened to throw some sharpshooters into the wood.

``Very good!'' said Bagration.

As he was leaving the battery, firing was heard on the left also,
and as it was too far to the left flank for him to have time to
go there himself, Prince Bagration sent Zherkov to tell the
general in command (the one who had paraded his regiment before
Kutuzov at Braunau) that he must retreat as quickly as possible
behind the hollow in the rear, as the right flank would probably
not be able to withstand the enemy's attack very long. About
Tushin and the battalion that had been in support of his battery
all was forgotten. Prince Andrew listened attentively to
Bagration's colloquies with the commanding officers and the
orders he gave them and, to his surprise, found that no orders
were really given, but that Prince Bagration tried to make it
appear that everything done by necessity, by accident, or by the
will of subordinate commanders was done, if not by his direct
command, at least in accord with his intentions. Prince Andrew
noticed, however, that though what happened was due to chance and
was independent of the commander's will, owing to the tact
Bagration showed, his presence was very valuable.  Officers who
approached him with disturbed countenances became calm; soldiers
and officers greeted him gaily, grew more cheerful in his
presence, and were evidently anxious to display their courage
before him.

% % % % % % % % % % % % % % % % % % % % % % % % % % % % % % % % %
% % % % % % % % % % % % % % % % % % % % % % % % % % % % % % % % %
% % % % % % % % % % % % % % % % % % % % % % % % % % % % % % % % %
% % % % % % % % % % % % % % % % % % % % % % % % % % % % % % % % %
% % % % % % % % % % % % % % % % % % % % % % % % % % % % % % % % %
% % % % % % % % % % % % % % % % % % % % % % % % % % % % % % % % %
% % % % % % % % % % % % % % % % % % % % % % % % % % % % % % % % %
% % % % % % % % % % % % % % % % % % % % % % % % % % % % % % % % %
% % % % % % % % % % % % % % % % % % % % % % % % % % % % % % % % %
% % % % % % % % % % % % % % % % % % % % % % % % % % % % % % % % %
% % % % % % % % % % % % % % % % % % % % % % % % % % % % % % % % %
% % % % % % % % % % % % % % % % % % % % % % % % % % % % % %

\chapter*{Chapter XVIII}
\ifaudio     \marginpar{
\href{http://ia902608.us.archive.org/23/items/war_and_peace_02_0801_librivox/war_and_peace_02_18_tolstoy_64kb.mp3}{Audio}
} \fi

\lettrine[lines=2, loversize=0.3, lraise=0]{\initfamily P}{rince}
Bagration, having reached the highest point of our right
flank, began riding downhill to where the roll of musketry was
heard but where on account of the smoke nothing could be
seen. The nearer they got to the hollow the less they could see
but the more they felt the nearness of the actual
battlefield. They began to meet wounded men. One with a bleeding
head and no cap was being dragged along by two soldiers who
supported him under the arms. There was a gurgle in his throat
and he was spitting blood. A bullet had evidently hit him in the
throat or mouth. Another was walking sturdily by himself but
without his musket, groaning aloud and swinging his arm which had
just been hurt, while blood from it was streaming over his
greatcoat as from a bottle. He had that moment been wounded and
his face showed fear rather than suffering.  Crossing a road they
descended a steep incline and saw several men lying on the
ground; they also met a crowd of soldiers some of whom were
unwounded. The soldiers were ascending the hill breathing
heavily, and despite the general's presence were talking loudly
and gesticulating. In front of them rows of gray cloaks were
already visible through the smoke, and an officer catching sight
of Bagration rushed shouting after the crowd of retreating
soldiers, ordering them back. Bagration rode up to the ranks
along which shots crackled now here and now there, drowning the
sound of voices and the shouts of command. The whole air reeked
with smoke. The excited faces of the soldiers were blackened with
it. Some were using their ramrods, others putting powder on the
touchpans or taking charges from their pouches, while others were
firing, though who they were firing at could not be seen for the
smoke which there was no wind to carry away. A pleasant humming
and whistling of bullets were often heard. ``What is this?''
thought Prince Andrew approaching the crowd of soldiers. ``It
can't be an attack, for they are not moving; it can't be a
square---for they are not drawn up for that.''

The commander of the regiment, a thin, feeble-looking old man
with a pleasant smile---his eyelids drooping more than half over
his old eyes, giving him a mild expression, rode up to Bagration
and welcomed him as a host welcomes an honored guest. He reported
that his regiment had been attacked by French cavalry and that,
though the attack had been repulsed, he had lost more than half
his men. He said the attack had been repulsed, employing this
military term to describe what had occurred to his regiment, but
in reality he did not himself know what had happened during that
half-hour to the troops entrusted to him, and could not say with
certainty whether the attack had been repulsed or his regiment
had been broken up. All he knew was that at the commencement of
the action balls and shells began flying all over his regiment
and hitting men and that afterwards someone had shouted
``Cavalry!'' and our men had begun firing. They were still
firing, not at the cavalry which had disappeared, but at French
infantry who had come into the hollow and were firing at our
men. Prince Bagration bowed his head as a sign that this was
exactly what he had desired and expected. Turning to his adjutant
he ordered him to bring down the two battalions of the Sixth
Chasseurs whom they had just passed. Prince Andrew was struck by
the changed expression on Prince Bagration's face at this
moment. It expressed the concentrated and happy resolution you
see on the face of a man who on a hot day takes a final run
before plunging into the water.  The dull, sleepy expression was
no longer there, nor the affectation of profound thought. The
round, steady, hawk's eyes looked before him eagerly and rather
disdainfully, not resting on anything although his movements were
still slow and measured.

The commander of the regiment turned to Prince Bagration,
entreating him to go back as it was too dangerous to remain where
they were. ``Please, your excellency, for God's sake!'' he kept
saying, glancing for support at an officer of the suite who
turned away from him. ``There, you see!''  and he drew attention
to the bullets whistling, singing, and hissing continually around
them. He spoke in the tone of entreaty and reproach that a
carpenter uses to a gentleman who has picked up an ax: ``We are
used to it, but you, sir, will blister your hands.'' He spoke as
if those bullets could not kill him, and his half-closed eyes
gave still more persuasiveness to his words. The staff officer
joined in the colonel's appeals, but Bagration did not reply; he
only gave an order to cease firing and re-form, so as to give
room for the two approaching battalions. While he was speaking,
the curtain of smoke that had concealed the hollow, driven by a
rising wind, began to move from right to left as if drawn by an
invisible hand, and the hill opposite, with the French moving
about on it, opened out before them. All eyes fastened
involuntarily on this French column advancing against them and
winding down over the uneven ground. One could already see the
soldiers' shaggy caps, distinguish the officers from the men, and
see the standard flapping against its staff.

``They march splendidly,'' remarked someone in Bagration's suite.

The head of the column had already descended into the hollow. The
clash would take place on this side of it...

The remains of our regiment which had been in action rapidly
formed up and moved to the right; from behind it, dispersing the
laggards, came two battalions of the Sixth Chasseurs in fine
order. Before they had reached Bagration, the weighty tread of
the mass of men marching in step could be heard. On their left
flank, nearest to Bagration, marched a company commander, a fine
round-faced man, with a stupid and happy expression---the same
man who had rushed out of the wattle shed. At that moment he was
clearly thinking of nothing but how dashing a fellow he would
appear as he passed the commander.

With the self-satisfaction of a man on parade, he stepped lightly
with his muscular legs as if sailing along, stretching himself to
his full height without the smallest effort, his ease contrasting
with the heavy tread of the soldiers who were keeping step with
him. He carried close to his leg a narrow unsheathed sword
(small, curved, and not like a real weapon) and looked now at the
superior officers and now back at the men without losing step,
his whole powerful body turning flexibly. It was as if all the
powers of his soul were concentrated on passing the commander in
the best possible manner, and feeling that he was doing it well
he was happy. ``Left... left... left...'' he seemed to repeat to
himself at each alternate step; and in time to this, with stern
but varied faces, the wall of soldiers burdened with knapsacks
and muskets marched in step, and each one of these hundreds of
soldiers seemed to be repeating to himself at each alternate
step, ``Left... left... left...'' A fat major skirted a bush,
puffing and falling out of step; a soldier who had fallen behind,
his face showing alarm at his defection, ran at a trot, panting
to catch up with his company. A cannon ball, cleaving the air,
flew over the heads of Bagration and his suite, and fell into the
column to the measure of ``Left... left!'' ``Close up!'' came the
company commander's voice in jaunty tones. The soldiers passed in
a semicircle round something where the ball had fallen, and an
old trooper on the flank, a noncommissioned officer who had
stopped beside the dead men, ran to catch up his line and,
falling into step with a hop, looked back angrily, and through
the ominous silence and the regular tramp of feet beating the
ground in unison, one seemed to hear left... left... left.

``Well done, lads!'' said Prince Bagration.

``Glad to do our best, your ex'len-lency!'' came a confused shout
from the ranks. A morose soldier marching on the left turned his
eyes on Bagration as he shouted, with an expression that seemed
to say: ``We know that ourselves!'' Another, without looking
round, as though fearing to relax, shouted with his mouth wide
open and passed on.

The order was given to halt and down knapsacks.

Bagration rode round the ranks that had marched past him and
dismounted.  He gave the reins to a Cossack, took off and handed
over his felt coat, stretched his legs, and set his cap
straight. The head of the French column, with its officers
leading, appeared from below the hill.

``Forward, with God!'' said Bagration, in a resolute, sonorous
voice, turning for a moment to the front line, and slightly
swinging his arms, he went forward uneasily over the rough field
with the awkward gait of a cavalryman. Prince Andrew felt that an
invisible power was leading him forward, and experienced great
happiness.

The French were already near. Prince Andrew, walking beside
Bagration, could clearly distinguish their bandoliers, red
epaulets, and even their faces. (He distinctly saw an old French
officer who, with gaitered legs and turned-out toes, climbed the
hill with difficulty.) Prince Bagration gave no further orders
and silently continued to walk on in front of the ranks. Suddenly
one shot after another rang out from the French, smoke appeared
all along their uneven ranks, and musket shots sounded. Several
of our men fell, among them the round-faced officer who had
marched so gaily and complacently. But at the moment the first
report was heard, Bagration looked round and shouted, ``Hurrah!''

``Hurrah---ah!---ah!'' rang a long-drawn shout from our ranks,
and passing Bagration and racing one another they rushed in an
irregular but joyous and eager crowd down the hill at their
disordered foe.

% % % % % % % % % % % % % % % % % % % % % % % % % % % % % % % % %
% % % % % % % % % % % % % % % % % % % % % % % % % % % % % % % % %
% % % % % % % % % % % % % % % % % % % % % % % % % % % % % % % % %
% % % % % % % % % % % % % % % % % % % % % % % % % % % % % % % % %
% % % % % % % % % % % % % % % % % % % % % % % % % % % % % % % % %
% % % % % % % % % % % % % % % % % % % % % % % % % % % % % % % % %
% % % % % % % % % % % % % % % % % % % % % % % % % % % % % % % % %
% % % % % % % % % % % % % % % % % % % % % % % % % % % % % % % % %
% % % % % % % % % % % % % % % % % % % % % % % % % % % % % % % % %
% % % % % % % % % % % % % % % % % % % % % % % % % % % % % % % % %
% % % % % % % % % % % % % % % % % % % % % % % % % % % % % % % % %
% % % % % % % % % % % % % % % % % % % % % % % % % % % % % %

\chapter*{Chapter XIX}
\ifaudio     \marginpar{
\href{http://ia902608.us.archive.org/23/items/war_and_peace_02_0801_librivox/war_and_peace_02_19_tolstoy_64kb.mp3}{Audio}
} \fi

\lettrine[lines=2, loversize=0.3, lraise=0]{\initfamily T}{he}
attack of the Sixth Chasseurs secured the retreat of our
right flank. In the center Tushin's forgotten battery, which had
managed to set fire to the Schon Grabern village, delayed the
French advance. The French were putting out the fire which the
wind was spreading, and thus gave us time to retreat. The
retirement of the center to the other side of the dip in the
ground at the rear was hurried and noisy, but the different
companies did not get mixed. But our left---which consisted of
the Azov and Podolsk infantry and the Pavlograd hussars---was
simultaneously attacked and outflanked by superior French forces
under Lannes and was thrown into confusion. Bagration had sent
Zherkov to the general commanding that left flank with orders to
retreat immediately.

Zherkov, not removing his hand from his cap, turned his horse
about and galloped off. But no sooner had he left Bagration than
his courage failed him. He was seized by panic and could not go
where it was dangerous.

Having reached the left flank, instead of going to the front
where the firing was, he began to look for the general and his
staff where they could not possibly be, and so did not deliver
the order.

The command of the left flank belonged by seniority to the
commander of the regiment Kutuzov had reviewed at Braunau and in
which Dolokhov was serving as a private. But the command of the
extreme left flank had been assigned to the commander of the
Pavlograd regiment in which Rostov was serving, and a
misunderstanding arose. The two commanders were much exasperated
with one another and, long after the action had begun on the
right flank and the French were already advancing, were engaged
in discussion with the sole object of offending one another. But
the regiments, both cavalry and infantry, were by no means ready
for the impending action. From privates to general they were not
expecting a battle and were engaged in peaceful occupations, the
cavalry feeding the horses and the infantry collecting wood.

``He higher iss dan I in rank,'' said the German colonel of the
hussars, flushing and addressing an adjutant who had ridden up,
``so let him do what he vill, but I cannot sacrifice my
hussars... Bugler, sount ze retreat!''

But haste was becoming imperative. Cannon and musketry, mingling
together, thundered on the right and in the center, while the
capotes of Lannes' sharpshooters were already seen crossing the
milldam and forming up within twice the range of a musket
shot. The general in command of the infantry went toward his
horse with jerky steps, and having mounted drew himself up very
straight and tall and rode to the Pavlograd commander. The
commanders met with polite bows but with secret malevolence in
their hearts.

``Once again, Colonel,'' said the general, ``I can't leave half
my men in the wood. I beg of you, I beg of you,'' he repeated,
``to occupy the position and prepare for an attack.''

``I peg of you yourself not to mix in vot is not your business!''
suddenly replied the irate colonel. ``If you vere in the
cavalry...''

``I am not in the cavalry, Colonel, but I am a Russian general
and if you are not aware of the fact...''

``Quite avare, your excellency,'' suddenly shouted the colonel,
touching his horse and turning purple in the face. ``Vill you be
so goot to come to ze front and see dat zis position iss no goot?
I don't vish to destroy my men for your pleasure!''

``You forget yourself, Colonel. I am not considering my own
pleasure and I won't allow it to be said!''

Taking the colonel's outburst as a challenge to his courage, the
general expanded his chest and rode, frowning, beside him to the
front line, as if their differences would be settled there
amongst the bullets. They reached the front, several bullets sped
over them, and they halted in silence. There was nothing fresh to
be seen from the line, for from where they had been before it had
been evident that it was impossible for cavalry to act among the
bushes and broken ground, as well as that the French were
outflanking our left. The general and colonel looked sternly and
significantly at one another like two fighting cocks preparing
for battle, each vainly trying to detect signs of cowardice in
the other. Both passed the examination successfully. As there was
nothing to be said, and neither wished to give occasion for it to
be alleged that he had been the first to leave the range of fire,
they would have remained there for a long time testing each
other's courage had it not been that just then they heard the
rattle of musketry and a muffled shout almost behind them in the
wood. The French had attacked the men collecting wood in the
copse. It was no longer possible for the hussars to retreat with
the infantry. They were cut off from the line of retreat on the
left by the French. However inconvenient the position, it was now
necessary to attack in order to cut a way through for themselves.

The squadron in which Rostov was serving had scarcely time to
mount before it was halted facing the enemy. Again, as at the
Enns bridge, there was nothing between the squadron and the
enemy, and again that terrible dividing line of uncertainty and
fear---resembling the line separating the living from the
dead---lay between them. All were conscious of this unseen line,
and the question whether they would cross it or not, and how they
would cross it, agitated them all.

The colonel rode to the front, angrily gave some reply to
questions put to him by the officers, and, like a man desperately
insisting on having his own way, gave an order. No one said
anything definite, but the rumor of an attack spread through the
squadron. The command to form up rang out and the sabers whizzed
as they were drawn from their scabbards.  Still no one moved. The
troops of the left flank, infantry and hussars alike, felt that
the commander did not himself know what to do, and this
irresolution communicated itself to the men.

``If only they would be quick!'' thought Rostov, feeling that at
last the time had come to experience the joy of an attack of
which he had so often heard from his fellow hussars.

``Fo'ward, with God, lads!'' rang out Denisov's voice. ``At a
twot fo'ward!''

The horses' croups began to sway in the front line. Rook pulled
at the reins and started of his own accord.

Before him, on the right, Rostov saw the front lines of his
hussars and still farther ahead a dark line which he could not
see distinctly but took to be the enemy. Shots could be heard,
but some way off.

``Faster!'' came the word of command, and Rostov felt Rook's
flanks drooping as he broke into a gallop.

Rostov anticipated his horse's movements and became more and more
elated. He had noticed a solitary tree ahead of him. This tree
had been in the middle of the line that had seemed so
terrible---and now he had crossed that line and not only was
there nothing terrible, but everything was becoming more and more
happy and animated. ``Oh, how I will slash at him!'' thought
Rostov, gripping the hilt of his saber.

``Hur-a-a-a-ah!'' came a roar of voices. ``Let anyone come my way
now,'' thought Rostov driving his spurs into Rook and letting him
go at a full gallop so that he outstripped the others. Ahead, the
enemy was already visible. Suddenly something like a birch broom
seemed to sweep over the squadron. Rostov raised his saber, ready
to strike, but at that instant the trooper Nikitenko, who was
galloping ahead, shot away from him, and Rostov felt as in a
dream that he continued to be carried forward with unnatural
speed but yet stayed on the same spot. From behind him
Bondarchuk, an hussar he knew, jolted against him and looked
angrily at him. Bondarchuk's horse swerved and galloped past.

``How is it I am not moving? I have fallen, I am killed!'' Rostov
asked and answered at the same instant. He was alone in the
middle of a field.  Instead of the moving horses and hussars'
backs, he saw nothing before him but the motionless earth and the
stubble around him. There was warm blood under his arm. ``No, I
am wounded and the horse is killed.'' Rook tried to rise on his
forelegs but fell back, pinning his rider's leg.  Blood was
flowing from his head; he struggled but could not rise. Rostov
also tried to rise but fell back, his sabretache having become
entangled in the saddle. Where our men were, and where the
French, he did not know. There was no one near.

Having disentangled his leg, he rose. ``Where, on which side, was
now the line that had so sharply divided the two armies?'' he
asked himself and could not answer. ``Can something bad have
happened to me?'' he wondered as he got up: and at that moment he
felt that something superfluous was hanging on his benumbed left
arm. The wrist felt as if it were not his.  He examined his hand
carefully, vainly trying to find blood on it. ``Ah, here are
people coming,'' he thought joyfully, seeing some men running
toward him. ``They will help me!'' In front came a man wearing a
strange shako and a blue cloak, swarthy, sunburned, and with a
hooked nose. Then came two more, and many more running
behind. One of them said something strange, not in Russian. In
among the hindmost of these men wearing similar shakos was a
Russian hussar. He was being held by the arms and his horse was
being led behind him.

``It must be one of ours, a prisoner. Yes. Can it be that they
will take me too? Who are these men?'' thought Rostov, scarcely
believing his eyes.  ``Can they be French?'' He looked at the
approaching Frenchmen, and though but a moment before he had been
galloping to get at them and hack them to pieces, their proximity
now seemed so awful that he could not believe his eyes. ``Who are
they? Why are they running? Can they be coming at me?  And why?
To kill me? Me whom everyone is so fond of?'' He remembered his
mother's love for him, and his family's, and his friends', and
the enemy's intention to kill him seemed impossible. ``But
perhaps they may do it!'' For more than ten seconds he stood not
moving from the spot or realizing the situation. The foremost
Frenchman, the one with the hooked nose, was already so close
that the expression of his face could be seen. And the excited,
alien face of that man, his bayonet hanging down, holding his
breath, and running so lightly, frightened Rostov. He seized his
pistol and, instead of firing it, flung it at the Frenchman and
ran with all his might toward the bushes. He did not now run with
the feeling of doubt and conflict with which he had trodden the
Enns bridge, but with the feeling of a hare fleeing from the
hounds. One single sentiment, that of fear for his young and
happy life, possessed his whole being. Rapidly leaping the
furrows, he fled across the field with the impetuosity he used to
show at catchplay, now and then turning his good-natured, pale,
young face to look back. A shudder of terror went through him:
``No, better not look,'' he thought, but having reached the
bushes he glanced round once more. The French had fallen behind,
and just as he looked round the first man changed his run to a
walk and, turning, shouted something loudly to a comrade farther
back. Rostov paused. ``No, there's some mistake,'' thought
he. ``They can't have wanted to kill me.'' But at the same time,
his left arm felt as heavy as if a seventy-pound weight were tied
to it. He could run no more. The Frenchman also stopped and took
aim. Rostov closed his eyes and stooped down. One bullet and then
another whistled past him. He mustered his last remaining
strength, took hold of his left hand with his right, and reached
the bushes. Behind these were some Russian sharpshooters.

% % % % % % % % % % % % % % % % % % % % % % % % % % % % % % % % %
% % % % % % % % % % % % % % % % % % % % % % % % % % % % % % % % %
% % % % % % % % % % % % % % % % % % % % % % % % % % % % % % % % %
% % % % % % % % % % % % % % % % % % % % % % % % % % % % % % % % %
% % % % % % % % % % % % % % % % % % % % % % % % % % % % % % % % %
% % % % % % % % % % % % % % % % % % % % % % % % % % % % % % % % %
% % % % % % % % % % % % % % % % % % % % % % % % % % % % % % % % %
% % % % % % % % % % % % % % % % % % % % % % % % % % % % % % % % %
% % % % % % % % % % % % % % % % % % % % % % % % % % % % % % % % %
% % % % % % % % % % % % % % % % % % % % % % % % % % % % % % % % %
% % % % % % % % % % % % % % % % % % % % % % % % % % % % % % % % %
% % % % % % % % % % % % % % % % % % % % % % % % % % % % % %

\chapter*{Chapter XX}
\ifaudio     \marginpar{
\href{http://ia902608.us.archive.org/23/items/war_and_peace_02_0801_librivox/war_and_peace_02_20_tolstoy_64kb.mp3}{Audio}
} \fi

\lettrine[lines=2, loversize=0.3, lraise=0]{\initfamily T}{he}
infantry regiments that had been caught unawares in the
outskirts of the wood ran out of it, the different companies
getting mixed, and retreated as a disorderly crowd. One soldier,
in his fear, uttered the senseless cry, ``Cut off!'' that is so
terrible in battle, and that word infected the whole crowd with a
feeling of panic.

``Surrounded! Cut off? We're lost!'' shouted the fugitives.

The moment he heard the firing and the cry from behind, the
general realized that something dreadful had happened to his
regiment, and the thought that he, an exemplary officer of many
years' service who had never been to blame, might be held
responsible at headquarters for negligence or inefficiency so
staggered him that, forgetting the recalcitrant cavalry colonel,
his own dignity as a general, and above all quite forgetting the
danger and all regard for self-preservation, he clutched the
crupper of his saddle and, spurring his horse, galloped to the
regiment under a hail of bullets which fell around, but
fortunately missed him. His one desire was to know what was
happening and at any cost correct, or remedy, the mistake if he
had made one, so that he, an exemplary officer of twenty-two
years' service, who had never been censured, should not be held
to blame.

Having galloped safely through the French, he reached a field
behind the copse across which our men, regardless of orders, were
running and descending the valley. That moment of moral
hesitation which decides the fate of battles had arrived. Would
this disorderly crowd of soldiers attend to the voice of their
commander, or would they, disregarding him, continue their
flight? Despite his desperate shouts that used to seem so
terrible to the soldiers, despite his furious purple countenance
distorted out of all likeness to his former self, and the
flourishing of his saber, the soldiers all continued to run,
talking, firing into the air, and disobeying orders. The moral
hesitation which decided the fate of battles was evidently
culminating in a panic.

The general had a fit of coughing as a result of shouting and of
the powder smoke and stopped in despair. Everything seemed
lost. But at that moment the French who were attacking, suddenly
and without any apparent reason, ran back and disappeared from
the outskirts, and Russian sharpshooters showed themselves in the
copse. It was Timokhin's company, which alone had maintained its
order in the wood and, having lain in ambush in a ditch, now
attacked the French unexpectedly. Timokhin, armed only with a
sword, had rushed at the enemy with such a desperate cry and such
mad, drunken determination that, taken by surprise, the French
had thrown down their muskets and run. Dolokhov, running beside
Timokhin, killed a Frenchman at close quarters and was the first
to seize the surrendering French officer by his collar. Our
fugitives returned, the battalions re-formed, and the French who
had nearly cut our left flank in half were for the moment
repulsed. Our reserve units were able to join up, and the fight
was at an end. The regimental commander and Major Ekonomov had
stopped beside a bridge, letting the retreating companies pass by
them, when a soldier came up and took hold of the commander's
stirrup, almost leaning against him. The man was wearing a bluish
coat of broadcloth, he had no knapsack or cap, his head was
bandaged, and over his shoulder a French munition pouch was
slung. He had an officer's sword in his hand. The soldier was
pale, his blue eyes looked impudently into the commander's face,
and his lips were smiling. Though the commander was occupied in
giving instructions to Major Ekonomov, he could not help taking
notice of the soldier.

``Your excellency, here are two trophies,'' said Dolokhov,
pointing to the French sword and pouch. ``I have taken an officer
prisoner. I stopped the company.'' Dolokhov breathed heavily from
weariness and spoke in abrupt sentences. ``The whole company can
bear witness. I beg you will remember this, your excellency!''

``All right, all right,'' replied the commander, and turned to
Major Ekonomov.

But Dolokhov did not go away; he untied the handkerchief around
his head, pulled it off, and showed the blood congealed on his
hair.

``A bayonet wound. I remained at the front. Remember, your
excellency!''

Tushin's battery had been forgotten and only at the very end of
the action did Prince Bagration, still hearing the cannonade in
the center, send his orderly staff officer, and later Prince
Andrew also, to order the battery to retire as quickly as
possible. When the supports attached to Tushin's battery had been
moved away in the middle of the action by someone's order, the
battery had continued firing and was only not captured by the
French because the enemy could not surmise that anyone could have
the effrontery to continue firing from four quite undefended
guns. On the contrary, the energetic action of that battery led
the French to suppose that here---in the center---the main
Russian forces were concentrated. Twice they had attempted to
attack this point, but on each occasion had been driven back by
grapeshot from the four isolated guns on the hillock.

Soon after Prince Bagration had left him, Tushin had succeeded in
setting fire to Schon Grabern.

``Look at them scurrying! It's burning! Just see the smoke! Fine!
Grand!  Look at the smoke, the smoke!'' exclaimed the
artillerymen, brightening up.

All the guns, without waiting for orders, were being fired in the
direction of the conflagration. As if urging each other on, the
soldiers cried at each shot: ``Fine! That's good! Look at
it... Grand!'' The fire, fanned by the breeze, was rapidly
spreading. The French columns that had advanced beyond the
village went back; but as though in revenge for this failure, the
enemy placed ten guns to the right of the village and began
firing them at Tushin's battery.

In their childlike glee, aroused by the fire and their luck in
successfully cannonading the French, our artillerymen only
noticed this battery when two balls, and then four more, fell
among our guns, one knocking over two horses and another tearing
off a munition-wagon driver's leg. Their spirits once roused
were, however, not diminished, but only changed character. The
horses were replaced by others from a reserve gun carriage, the
wounded were carried away, and the four guns were turned against
the ten-gun battery. Tushin's companion officer had been killed
at the beginning of the engagement and within an hour seventeen
of the forty men of the guns' crews had been disabled, but the
artillerymen were still as merry and lively as ever. Twice they
noticed the French appearing below them, and then they fired
grapeshot at them.

Little Tushin, moving feebly and awkwardly, kept telling his
orderly to ``refill my pipe for that one!'' and then, scattering
sparks from it, ran forward shading his eyes with his small hand
to look at the French.

``Smack at 'em, lads!'' he kept saying, seizing the guns by the
wheels and working the screws himself.

Amid the smoke, deafened by the incessant reports which always
made him jump, Tushin not taking his pipe from his mouth ran from
gun to gun, now aiming, now counting the charges, now giving
orders about replacing dead or wounded horses and harnessing
fresh ones, and shouting in his feeble voice, so high pitched and
irresolute. His face grew more and more animated. Only when a man
was killed or wounded did he frown and turn away from the sight,
shouting angrily at the men who, as is always the case, hesitated
about lifting the injured or dead. The soldiers, for the most
part handsome fellows and, as is always the case in an artillery
company, a head and shoulders taller and twice as broad as their
officer---all looked at their commander like children in an
embarrassing situation, and the expression on his face was
invariably reflected on theirs.

Owing to the terrible uproar and the necessity for concentration
and activity, Tushin did not experience the slightest unpleasant
sense of fear, and the thought that he might be killed or badly
wounded never occurred to him. On the contrary, he became more
and more elated. It seemed to him that it was a very long time
ago, almost a day, since he had first seen the enemy and fired
the first shot, and that the corner of the field he stood on was
well-known and familiar ground. Though he thought of everything,
considered everything, and did everything the best of officers
could do in his position, he was in a state akin to feverish
delirium or drunkenness.

From the deafening sounds of his own guns around him, the whistle
and thud of the enemy's cannon balls, from the flushed and
perspiring faces of the crew bustling round the guns, from the
sight of the blood of men and horses, from the little puffs of
smoke on the enemy's side (always followed by a ball flying past
and striking the earth, a man, a gun, a horse), from the sight of
all these things a fantastic world of his own had taken
possession of his brain and at that moment afforded him
pleasure. The enemy's guns were in his fancy not guns but pipes
from which occasional puffs were blown by an invisible smoker.

``There... he's puffing again,'' muttered Tushin to himself, as a
small cloud rose from the hill and was borne in a streak to the
left by the wind.

``Now look out for the ball... we'll throw it back.''

``What do you want, your honor?'' asked an artilleryman, standing
close by, who heard him muttering.

``Nothing... only a shell...'' he answered.

``Come along, our Matvevna!'' he said to
himself. ``Matvevna''\footnote{Daughter of Matthew.}  was the
name his fancy gave to the farthest gun of the battery, which was
large and of an old pattern. The French swarming round their guns
seemed to him like ants. In that world, the handsome drunkard
Number One of the second gun's crew was ``uncle''; Tushin looked
at him more often than at anyone else and took delight in his
every movement. The sound of musketry at the foot of the hill,
now diminishing, now increasing, seemed like someone's
breathing. He listened intently to the ebb and flow of these
sounds.

``Ah! Breathing again, breathing!'' he muttered to himself.

He imagined himself as an enormously tall, powerful man who was
throwing cannon balls at the French with both hands.

``Now then, Matvevna, dear old lady, don't let me down!'' he was
saying as he moved from the gun, when a strange, unfamiliar voice
called above his head: ``Captain Tushin! Captain!''

Tushin turned round in dismay. It was the staff officer who had
turned him out of the booth at Grunth. He was shouting in a
gasping voice:

``Are you mad? You have twice been ordered to retreat, and
you...''

``Why are they down on me?'' thought Tushin, looking in alarm at
his superior.

``I... don't...'' he muttered, holding up two fingers to his
cap. ``I...''

But the staff officer did not finish what he wanted to say. A
cannon ball, flying close to him, caused him to duck and bend
over his horse.  He paused, and just as he was about to say
something more, another ball stopped him. He turned his horse and
galloped off.

``Retire! All to retire!'' he shouted from a distance.

The soldiers laughed. A moment later, an adjutant arrived with
the same order.

It was Prince Andrew. The first thing he saw on riding up to the
space where Tushin's guns were stationed was an unharnessed horse
with a broken leg, that lay screaming piteously beside the
harnessed horses.  Blood was gushing from its leg as from a
spring. Among the limbers lay several dead men. One ball after
another passed over as he approached and he felt a nervous
shudder run down his spine. But the mere thought of being afraid
roused him again. ``I cannot be afraid,'' thought he, and
dismounted slowly among the guns. He delivered the order and did
not leave the battery. He decided to have the guns removed from
their positions and withdrawn in his presence. Together with
Tushin, stepping across the bodies and under a terrible fire from
the French, he attended to the removal of the guns.

``A staff officer was here a minute ago, but skipped off,'' said
an artilleryman to Prince Andrew. ``Not like your honor!''

Prince Andrew said nothing to Tushin. They were both so busy as
to seem not to notice one another. When having limbered up the
only two cannon that remained uninjured out of the four, they
began moving down the hill (one shattered gun and one unicorn
were left behind), Prince Andrew rode up to Tushin.

``Well, till we meet again...'' he said, holding out his hand to
Tushin.

``Good-bye, my dear fellow,'' said Tushin. ``Dear soul! Good-bye,
my dear fellow!'' and for some unknown reason tears suddenly
filled his eyes.

% % % % % % % % % % % % % % % % % % % % % % % % % % % % % % % % %
% % % % % % % % % % % % % % % % % % % % % % % % % % % % % % % % %
% % % % % % % % % % % % % % % % % % % % % % % % % % % % % % % % %
% % % % % % % % % % % % % % % % % % % % % % % % % % % % % % % % %
% % % % % % % % % % % % % % % % % % % % % % % % % % % % % % % % %
% % % % % % % % % % % % % % % % % % % % % % % % % % % % % % % % %
% % % % % % % % % % % % % % % % % % % % % % % % % % % % % % % % %
% % % % % % % % % % % % % % % % % % % % % % % % % % % % % % % % %
% % % % % % % % % % % % % % % % % % % % % % % % % % % % % % % % %
% % % % % % % % % % % % % % % % % % % % % % % % % % % % % % % % %
% % % % % % % % % % % % % % % % % % % % % % % % % % % % % % % % %
% % % % % % % % % % % % % % % % % % % % % % % % % % % % % %

\chapter*{Chapter XXI}
\ifaudio     \marginpar{
\href{http://ia902608.us.archive.org/23/items/war_and_peace_02_0801_librivox/war_and_peace_02_21_tolstoy_64kb.mp3}{Audio}
} \fi

\lettrine[lines=2, loversize=0.3, lraise=0]{\initfamily T}{he}
wind had fallen and black clouds, merging with the powder
smoke, hung low over the field of battle on the horizon. It was
growing dark and the glow of two conflagrations was the more
conspicuous. The cannonade was dying down, but the rattle of
musketry behind and on the right sounded oftener and nearer. As
soon as Tushin with his guns, continually driving round or coming
upon wounded men, was out of range of fire and had descended into
the dip, he was met by some of the staff, among them the staff
officer and Zherkov, who had been twice sent to Tushin's battery
but had never reached it. Interrupting one another, they all
gave, and transmitted, orders as to how to proceed, reprimanding
and reproaching him. Tushin gave no orders, and,
silently---fearing to speak because at every word he felt ready
to weep without knowing why---rode behind on his artillery
nag. Though the orders were to abandon the wounded, many of them
dragged themselves after troops and begged for seats on the gun
carriages. The jaunty infantry officer who just before the battle
had rushed out of Tushin's wattle shed was laid, with a bullet in
his stomach, on \emph{Matvevna's} carriage. At the foot of the
hill, a pale hussar cadet, supporting one hand with the other,
came up to Tushin and asked for a seat.

``Captain, for God's sake! I've hurt my arm,'' he said
timidly. ``For God's sake... I can't walk. For God's sake!''

It was plain that this cadet had already repeatedly asked for a
lift and been refused. He asked in a hesitating, piteous voice.

``Tell them to give me a seat, for God's sake!''

``Give him a seat,'' said Tushin. ``Lay a cloak for him to sit
on, lad,'' he said, addressing his favorite soldier. ``And where
is the wounded officer?''

``He has been set down. He died,'' replied someone.

``Help him up. Sit down, dear fellow, sit down! Spread out the
cloak, Antonov.''

The cadet was Rostov. With one hand he supported the other; he
was pale and his jaw trembled, shivering feverishly. He was
placed on ``Matvevna,'' the gun from which they had removed the
dead officer. The cloak they spread under him was wet with blood
which stained his breeches and arm.

``What, are you wounded, my lad?'' said Tushin, approaching the
gun on which Rostov sat.

``No, it's a sprain.''

``Then what is this blood on the gun carriage?'' inquired Tushin.

``It was the officer, your honor, stained it,'' answered the
artilleryman, wiping away the blood with his coat sleeve, as if
apologizing for the state of his gun.

It was all that they could do to get the guns up the rise aided
by the infantry, and having reached the village of Gruntersdorf
they halted. It had grown so dark that one could not distinguish
the uniforms ten paces off, and the firing had begun to
subside. Suddenly, near by on the right, shouting and firing were
again heard. Flashes of shot gleamed in the darkness. This was
the last French attack and was met by soldiers who had sheltered
in the village houses. They all rushed out of the village again,
but Tushin's guns could not move, and the artillerymen, Tushin,
and the cadet exchanged silent glances as they awaited their
fate. The firing died down and soldiers, talking eagerly,
streamed out of a side street.

``Not hurt, Petrov?'' asked one.

``We've given it 'em hot, mate! They won't make another push
now,'' said another.

``You couldn't see a thing. How they shot at their own fellows!
Nothing could be seen. Pitch-dark, brother! Isn't there something
to drink?''

The French had been repulsed for the last time. And again and
again in the complete darkness Tushin's guns moved forward,
surrounded by the humming infantry as by a frame.

In the darkness, it seemed as though a gloomy unseen river was
flowing always in one direction, humming with whispers and talk
and the sound of hoofs and wheels. Amid the general rumble, the
groans and voices of the wounded were more distinctly heard than
any other sound in the darkness of the night. The gloom that
enveloped the army was filled with their groans, which seemed to
melt into one with the darkness of the night.  After a while the
moving mass became agitated, someone rode past on a white horse
followed by his suite, and said something in passing: ``What did
he say? Where to, now? Halt, is it? Did he thank us?'' came eager
questions from all sides. The whole moving mass began pressing
closer together and a report spread that they were ordered to
halt: evidently those in front had halted. All remained where
they were in the middle of the muddy road.

Fires were lighted and the talk became more audible. Captain
Tushin, having given orders to his company, sent a soldier to
find a dressing station or a doctor for the cadet, and sat down
by a bonfire the soldiers had kindled on the road. Rostov, too,
dragged himself to the fire. From pain, cold, and damp, a
feverish shivering shook his whole body. Drowsiness was
irresistibly mastering him, but he kept awake by an excruciating
pain in his arm, for which he could find no satisfactory
position. He kept closing his eyes and then again looking at the
fire, which seemed to him dazzlingly red, and at the feeble,
round-shouldered figure of Tushin who was sitting cross-legged
like a Turk beside him.  Tushin's large, kind, intelligent eyes
were fixed with sympathy and commiseration on Rostov, who saw
that Tushin with his whole heart wished to help him but could
not.

From all sides were heard the footsteps and talk of the infantry,
who were walking, driving past, and settling down all around. The
sound of voices, the tramping feet, the horses' hoofs moving in
mud, the crackling of wood fires near and afar, merged into one
tremulous rumble.

It was no longer, as before, a dark, unseen river flowing through
the gloom, but a dark sea swelling and gradually subsiding after
a storm.  Rostov looked at and listened listlessly to what passed
before and around him. An infantryman came to the fire, squatted
on his heels, held his hands to the blaze, and turned away his
face.

``You don't mind your honor?'' he asked Tushin. ``I've lost my
company, your honor. I don't know where... such bad luck!''

With the soldier, an infantry officer with a bandaged cheek came
up to the bonfire, and addressing Tushin asked him to have the
guns moved a trifle to let a wagon go past. After he had gone,
two soldiers rushed to the campfire. They were quarreling and
fighting desperately, each trying to snatch from the other a boot
they were both holding on to.

``You picked it up?... I dare say! You're very smart!'' one of
them shouted hoarsely.

Then a thin, pale soldier, his neck bandaged with a bloodstained
leg band, came up and in angry tones asked the artillerymen for
water.

``Must one die like a dog?'' said he.

Tushin told them to give the man some water. Then a cheerful
soldier ran up, begging a little fire for the infantry.

``A nice little hot torch for the infantry! Good luck to you,
fellow countrymen. Thanks for the fire---we'll return it with
interest,'' said he, carrying away into the darkness a glowing
stick.

Next came four soldiers, carrying something heavy on a cloak, and
passed by the fire. One of them stumbled.

``Who the devil has put the logs on the road?'' snarled he.

``He's dead---why carry him?'' said another.

``Shut up!''

And they disappeared into the darkness with their load.

``Still aching?'' Tushin asked Rostov in a whisper.

``Yes.''

``Your honor, you're wanted by the general. He is in the hut
here,'' said a gunner, coming up to Tushin.

``Coming, friend.''

Tushin rose and, buttoning his greatcoat and pulling it straight,
walked away from the fire.

Not far from the artillery campfire, in a hut that had been
prepared for him, Prince Bagration sat at dinner, talking with
some commanding officers who had gathered at his quarters. The
little old man with the half-closed eyes was there greedily
gnawing a mutton bone, and the general who had served blamelessly
for twenty-two years, flushed by a glass of vodka and the dinner;
and the staff officer with the signet ring, and Zherkov, uneasily
glancing at them all, and Prince Andrew, pale, with compressed
lips and feverishly glittering eyes.

In a corner of the hut stood a standard captured from the French,
and the accountant with the naive face was feeling its texture,
shaking his head in perplexity---perhaps because the banner
really interested him, perhaps because it was hard for him,
hungry as he was, to look on at a dinner where there was no place
for him. In the next hut there was a French colonel who had been
taken prisoner by our dragoons. Our officers were flocking in to
look at him. Prince Bagration was thanking the individual
commanders and inquiring into details of the action and our
losses. The general whose regiment had been inspected at Braunau
was informing the prince that as soon as the action began he had
withdrawn from the wood, mustered the men who were woodcutting,
and, allowing the French to pass him, had made a bayonet charge
with two battalions and had broken up the French troops.

``When I saw, your excellency, that their first battalion was
disorganized, I stopped in the road and thought: 'I'll let them
come on and will meet them with the fire of the whole
battalion'---and that's what I did.''

The general had so wished to do this and was so sorry he had not
managed to do it that it seemed to him as if it had really
happened. Perhaps it might really have been so? Could one
possibly make out amid all that confusion what did or did not
happen?

``By the way, your excellency, I should inform you,'' he
continued---remembering Dolokhov's conversation with Kutuzov and
his last interview with the gentleman-ranker---``that Private
Dolokhov, who was reduced to the ranks, took a French officer
prisoner in my presence and particularly distinguished himself.''

``I saw the Pavlograd hussars attack there, your excellency,''
chimed in Zherkov, looking uneasily around. He had not seen the
hussars all that day, but had heard about them from an infantry
officer. ``They broke up two squares, your excellency.''

Several of those present smiled at Zherkov's words, expecting one
of his usual jokes, but noticing that what he was saying
redounded to the glory of our arms and of the day's work, they
assumed a serious expression, though many of them knew that what
he was saying was a lie devoid of any foundation. Prince
Bagration turned to the old colonel:

``Gentlemen, I thank you all; all arms have behaved heroically:
infantry, cavalry, and artillery. How was it that two guns were
abandoned in the center?'' he inquired, searching with his eyes
for someone. (Prince Bagration did not ask about the guns on the
left flank; he knew that all the guns there had been abandoned at
the very beginning of the action.)  ``I think I sent you?'' he
added, turning to the staff officer on duty.

``One was damaged,'' answered the staff officer, ``and the other
I can't understand. I was there all the time giving orders and
had only just left... It is true that it was hot there,'' he
added, modestly.

Someone mentioned that Captain Tushin was bivouacking close to
the village and had already been sent for.

``Oh, but you were there?'' said Prince Bagration, addressing
Prince Andrew.

``Of course, we only just missed one another,'' said the staff
officer, with a smile to Bolkonski.

``I had not the pleasure of seeing you,'' said Prince Andrew,
coldly and abruptly.

All were silent. Tushin appeared at the threshold and made his
way timidly from behind the backs of the generals. As he stepped
past the generals in the crowded hut, feeling embarrassed as he
always was by the sight of his superiors, he did not notice the
staff of the banner and stumbled over it. Several of those
present laughed.

``How was it a gun was abandoned?'' asked Bagration, frowning,
not so much at the captain as at those who were laughing, among
whom Zherkov laughed loudest.

Only now, when he was confronted by the stern authorities, did
his guilt and the disgrace of having lost two guns and yet
remaining alive present themselves to Tushin in all their
horror. He had been so excited that he had not thought about it
until that moment. The officers' laughter confused him still
more. He stood before Bagration with his lower jaw trembling and
was hardly able to mutter: ``I don't know... your excellency... I
had no men... your excellency.''

``You might have taken some from the covering troops.''

Tushin did not say that there were no covering troops, though
that was perfectly true. He was afraid of getting some other
officer into trouble, and silently fixed his eyes on Bagration as
a schoolboy who has blundered looks at an examiner.

The silence lasted some time. Prince Bagration, apparently not
wishing to be severe, found nothing to say; the others did not
venture to intervene. Prince Andrew looked at Tushin from under
his brows and his fingers twitched nervously.

``Your excellency!'' Prince Andrew broke the silence with his
abrupt voice, ``you were pleased to send me to Captain Tushin's
battery. I went there and found two thirds of the men and horses
knocked out, two guns smashed, and no supports at all.''

Prince Bagration and Tushin looked with equal intentness at
Bolkonski, who spoke with suppressed agitation.

``And, if your excellency will allow me to express my opinion,''
he continued, ``we owe today's success chiefly to the action of
that battery and the heroic endurance of Captain Tushin and his
company,'' and without awaiting a reply, Prince Andrew rose and
left the table.

Prince Bagration looked at Tushin, evidently reluctant to show
distrust in Bolkonski's emphatic opinion yet not feeling able
fully to credit it, bent his head, and told Tushin that he could
go. Prince Andrew went out with him.

``Thank you; you saved me, my dear fellow!'' said Tushin.

Prince Andrew gave him a look, but said nothing and went away. He
felt sad and depressed. It was all so strange, so unlike what he
had hoped.

``Who are they? Why are they here? What do they want? And when
will all this end?'' thought Rostov, looking at the changing
shadows before him.  The pain in his arm became more and more
intense. Irresistible drowsiness overpowered him, red rings
danced before his eyes, and the impression of those voices and
faces and a sense of loneliness merged with the physical pain. It
was they, these soldiers---wounded and unwounded---it was they
who were crushing, weighing down, and twisting the sinews and
scorching the flesh of his sprained arm and shoulder. To rid
himself of them he closed his eyes.

For a moment he dozed, but in that short interval innumerable
things appeared to him in a dream: his mother and her large white
hand, Sonya's thin little shoulders, Natasha's eyes and laughter,
Denisov with his voice and mustache, and Telyanin and all that
affair with Telyanin and Bogdanich. That affair was the same
thing as this soldier with the harsh voice, and it was that
affair and this soldier that were so agonizingly, incessantly
pulling and pressing his arm and always dragging it in one
direction. He tried to get away from them, but they would not for
an instant let his shoulder move a hair's breadth. It would not
ache---it would be well---if only they did not pull it, but it
was impossible to get rid of them.

He opened his eyes and looked up. The black canopy of night hung
less than a yard above the glow of the charcoal. Flakes of
falling snow were fluttering in that light. Tushin had not
returned, the doctor had not come. He was alone now, except for a
soldier who was sitting naked at the other side of the fire,
warming his thin yellow body.

``Nobody wants me!'' thought Rostov. ``There is no one to help me
or pity me. Yet I was once at home, strong, happy, and loved.''
He sighed and, doing so, groaned involuntarily.

``Eh, is anything hurting you?'' asked the soldier, shaking his
shirt out over the fire, and not waiting for an answer he gave a
grunt and added: ``What a lot of men have been crippled
today---frightful!''

Rostov did not listen to the soldier. He looked at the snowflakes
fluttering above the fire and remembered a Russian winter at his
warm, bright home, his fluffy fur coat, his quickly gliding
sleigh, his healthy body, and all the affection and care of his
family. ``And why did I come here?'' he wondered.

Next day the French army did not renew their attack, and the
remnant of Bagration's detachment was reunited to Kutuzov's army.

