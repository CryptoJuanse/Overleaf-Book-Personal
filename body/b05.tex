\part*{Book Five: 1806 - 07}

% % % % % % % % % % % % % % % % % % % % % % % % % % % % % % % % %
% % % % % % % % % % % % % % % % % % % % % % % % % % % % % % % % %
% % % % % % % % % % % % % % % % % % % % % % % % % % % % % % % % %
% % % % % % % % % % % % % % % % % % % % % % % % % % % % % % % % %
% % % % % % % % % % % % % % % % % % % % % % % % % % % % % % % % %
% % % % % % % % % % % % % % % % % % % % % % % % % % % % % % % % %
% % % % % % % % % % % % % % % % % % % % % % % % % % % % % % % % %
% % % % % % % % % % % % % % % % % % % % % % % % % % % % % % % % %
% % % % % % % % % % % % % % % % % % % % % % % % % % % % % % % % %
% % % % % % % % % % % % % % % % % % % % % % % % % % % % % % % % %
% % % % % % % % % % % % % % % % % % % % % % % % % % % % % % % % %
% % % % % % % % % % % % % % % % % % % % % % % % % % % % % %

\chapter*{Chapter I}
\ifaudio
\marginpar{
\href{http://ia600200.us.archive.org/6/items/war_and_peace_05_0805_librivox/war_and_peace_05_01_tolstoy_64kb.mp3}{Audio}} 
\fi

\lettrine[lines=2, loversize=0.3, lraise=0]{\initfamily A}{fter}
his interview with his wife Pierre left for Petersburg. At
the Torzhok post station, either there were no horses or the
postmaster would not supply them. Pierre was obliged to
wait. Without undressing, he lay down on the leather sofa in
front of a round table, put his big feet in their overboots on
the table, and began to reflect.

``Will you have the portmanteaus brought in? And a bed got ready,
and tea?'' asked his valet.

Pierre gave no answer, for he neither heard nor saw anything. He
had begun to think of the last station and was still pondering on
the same question---one so important that he took no notice of
what went on around him. Not only was he indifferent as to
whether he got to Petersburg earlier or later, or whether he
secured accommodation at this station, but compared to the
thoughts that now occupied him it was a matter of indifference
whether he remained there for a few hours or for the rest of his
life.

The postmaster, his wife, the valet, and a peasant woman selling
Torzhok embroidery came into the room offering their
services. Without changing his careless attitude, Pierre looked
at them over his spectacles unable to understand what they wanted
or how they could go on living without having solved the problems
that so absorbed him. He had been engrossed by the same thoughts
ever since the day he returned from Sokolniki after the duel and
had spent that first agonizing, sleepless night. But now, in the
solitude of the journey, they seized him with special force. No
matter what he thought about, he always returned to these same
questions which he could not solve and yet could not cease to ask
himself. It was as if the thread of the chief screw which held
his life together were stripped, so that the screw could not get
in or out, but went on turning uselessly in the same place.

The postmaster came in and began obsequiously to beg his
excellency to wait only two hours, when, come what might, he
would let his excellency have the courier horses. It was plain
that he was lying and only wanted to get more money from the
traveler.

``Is this good or bad?'' Pierre asked himself. ``It is good for
me, bad for another traveler, and for himself it's unavoidable,
because he needs money for food; the man said an officer had once
given him a thrashing for letting a private traveler have the
courier horses. But the officer thrashed him because he had to
get on as quickly as possible. And I,'' continued Pierre, ``shot
Dolokhov because I considered myself injured, and Louis XVI was
executed because they considered him a criminal, and a year later
they executed those who executed him---also for some reason.
What is bad? What is good? What should one love and what hate?
What does one live for? And what am I? What is life, and what is
death? What power governs all?''

There was no answer to any of these questions, except one, and
that not a logical answer and not at all a reply to them. The
answer was: ``You'll die and all will end. You'll die and know
all, or cease asking.'' But dying was also dreadful.

The Torzhok peddler woman, in a whining voice, went on offering
her wares, especially a pair of goatskin slippers. ``I have
hundreds of rubles I don't know what to do with, and she stands
in her tattered cloak looking timidly at me,'' he thought. ``And
what does she want the money for? As if that money could add a
hair's breadth to happiness or peace of mind. Can anything in the
world make her or me less a prey to evil and death?---death which
ends all and must come today or tomorrow---at any rate, in an
instant as compared with eternity.'' And again he twisted the
screw with the stripped thread, and again it turned uselessly in
the same place.

His servant handed him a half-cut novel, in the form of letters,
by Madame de Souza. He began reading about the sufferings and
virtuous struggles of a certain Emilie de Mansfeld. ``And why did
she resist her seducer when she loved him?'' he thought. ``God
could not have put into her heart an impulse that was against His
will. My wife---as she once was---did not struggle, and perhaps
she was right. Nothing has been found out, nothing discovered,''
Pierre again said to himself. ``All we can know is that we know
nothing. And that's the height of human wisdom.''

Everything within and around him seemed confused, senseless, and
repellent. Yet in this very repugnance to all his circumstances
Pierre found a kind of tantalizing satisfaction.

``I make bold to ask your excellency to move a little for this
gentleman,'' said the postmaster, entering the room followed by
another traveler, also detained for lack of horses.

The newcomer was a short, large-boned, yellow-faced, wrinkled old
man, with gray bushy eyebrows overhanging bright eyes of an
indefinite grayish color.

Pierre took his feet off the table, stood up, and lay down on a
bed that had been got ready for him, glancing now and then at the
newcomer, who, with a gloomy and tired face, was wearily taking
off his wraps with the aid of his servant, and not looking at
Pierre. With a pair of felt boots on his thin bony legs, and
keeping on a worn, nankeen-covered, sheepskin coat, the traveler
sat down on the sofa, leaned back his big head with its broad
temples and close-cropped hair, and looked at Bezukhov. The
stern, shrewd, and penetrating expression of that look struck
Pierre. He felt a wish to speak to the stranger, but by the time
he had made up his mind to ask him a question about the roads,
the traveler had closed his eyes. His shriveled old hands were
folded and on the finger of one of them Pierre noticed a large
cast iron ring with a seal representing a death's head. The
stranger sat without stirring, either resting or, as it seemed to
Pierre, sunk in profound and calm meditation. His servant was
also a yellow, wrinkled old man, without beard or mustache,
evidently not because he was shaven but because they had never
grown.  This active old servant was unpacking the traveler's
canteen and preparing tea. He brought in a boiling samovar. When
everything was ready, the stranger opened his eyes, moved to the
table, filled a tumbler with tea for himself and one for the
beardless old man to whom he passed it. Pierre began to feel a
sense of uneasiness, and the need, even the inevitability, of
entering into conversation with this stranger.

The servant brought back his tumbler turned upside
down,\footnote{To indicate he did not want more tea.} with an
unfinished bit of nibbled sugar, and asked if anything more would
be wanted.

``No. Give me the book,'' said the stranger.

The servant handed him a book which Pierre took to be a
devotional work, and the traveler became absorbed in it. Pierre
looked at him. All at once the stranger closed the book, putting
in a marker, and again, leaning with his arms on the back of the
sofa, sat in his former position with his eyes shut. Pierre
looked at him and had not time to turn away when the old man,
opening his eyes, fixed his steady and severe gaze straight on
Pierre's face.

Pierre felt confused and wished to avoid that look, but the
bright old eyes attracted him irresistibly.

% % % % % % % % % % % % % % % % % % % % % % % % % % % % % % % % %
% % % % % % % % % % % % % % % % % % % % % % % % % % % % % % % % %
% % % % % % % % % % % % % % % % % % % % % % % % % % % % % % % % %
% % % % % % % % % % % % % % % % % % % % % % % % % % % % % % % % %
% % % % % % % % % % % % % % % % % % % % % % % % % % % % % % % % %
% % % % % % % % % % % % % % % % % % % % % % % % % % % % % % % % %
% % % % % % % % % % % % % % % % % % % % % % % % % % % % % % % % %
% % % % % % % % % % % % % % % % % % % % % % % % % % % % % % % % %
% % % % % % % % % % % % % % % % % % % % % % % % % % % % % % % % %
% % % % % % % % % % % % % % % % % % % % % % % % % % % % % % % % %
% % % % % % % % % % % % % % % % % % % % % % % % % % % % % % % % %
% % % % % % % % % % % % % % % % % % % % % % % % % % % % % %

\chapter*{Chapter II}
\ifaudio     
\marginpar{
\href{http://ia600200.us.archive.org/6/items/war_and_peace_05_0805_librivox/war_and_peace_05_02_tolstoy_64kb.mp3}{Audio}} 
\fi

\lettrine[lines=2, loversize=0.3, lraise=0]{ `` \initfamily I}{} have the pleasure of addressing Count Bezukhov, if I am not
mistaken,'' said the stranger in a deliberate and loud voice.

Pierre looked silently and inquiringly at him over his
spectacles.

``I have heard of you, my dear sir,'' continued the stranger,
``and of your misfortune.'' He seemed to emphasize the last word,
as if to say---``Yes, misfortune! Call it what you please, I know
that what happened to you in Moscow was a misfortune.''---``I
regret it very much, my dear sir.''

Pierre flushed and, hurriedly putting his legs down from the bed,
bent forward toward the old man with a forced and timid smile.

``I have not referred to this out of curiosity, my dear sir, but
for greater reasons.''

He paused, his gaze still on Pierre, and moved aside on the sofa
by way of inviting the other to take a seat beside him. Pierre
felt reluctant to enter into conversation with this old man, but,
submitting to him involuntarily, came up and sat down beside him.

``You are unhappy, my dear sir,'' the stranger continued. ``You
are young and I am old. I should like to help you as far as lies
in my power.''

``Oh, yes!'' said Pierre, with a forced smile. ``I am very
grateful to you.  Where are you traveling from?''

The stranger's face was not genial, it was even cold and severe,
but in spite of this, both the face and words of his new
acquaintance were irresistibly attractive to Pierre.

``But if for reason you don't feel inclined to talk to me,'' said
the old man, ``say so, my dear sir.'' And he suddenly smiled, in
an unexpected and tenderly paternal way.

``Oh no, not at all! On the contrary, I am very glad to make your
acquaintance,'' said Pierre. And again, glancing at the
stranger's hands, he looked more closely at the ring, with its
skull---a masonic sign.

``Allow me to ask,'' he said, ``are you a Mason?''

``Yes, I belong to the Brotherhood of the Freemasons,'' said the
stranger, looking deeper and deeper into Pierre's eyes. ``And in
their name and my own I hold out a brotherly hand to you.''

``I am afraid,'' said Pierre, smiling, and wavering between the
confidence the personality of the Freemason inspired in him and
his own habit of ridiculing the masonic beliefs---``I am afraid I
am very far from understanding---how am I to put it?---I am
afraid my way of looking at the world is so opposed to yours that
we shall not understand one another.''

``I know your outlook,'' said the Mason, ``and the view of life
you mention, and which you think is the result of your own mental
efforts, is the one held by the majority of people, and is the
invariable fruit of pride, indolence, and ignorance. Forgive me,
my dear sir, but if I had not known it I should not have
addressed you. Your view of life is a regrettable delusion.''

``Just as I may suppose you to be deluded,'' said Pierre, with a
faint smile.

``I should never dare to say that I know the truth,'' said the
Mason, whose words struck Pierre more and more by their precision
and firmness.  ``No one can attain to truth by himself. Only by
laying stone on stone with the cooperation of all, by the
millions of generations from our forefather Adam to our own
times, is that temple reared which is to be a worthy dwelling
place of the Great God,'' he added, and closed his eyes.

``I ought to tell you that I do not believe... do not believe in
God,'' said Pierre, regretfully and with an effort, feeling it
essential to speak the whole truth.

The Mason looked intently at Pierre and smiled as a rich man with
millions in hand might smile at a poor fellow who told him that
he, poor man, had not the five rubles that would make him happy.

``Yes, you do not know Him, my dear sir,'' said the Mason. ``You
cannot know Him. You do not know Him and that is why you are
unhappy.''

``Yes, yes, I am unhappy,'' assented Pierre. ``But what am I to
do?''

``You know Him not, my dear sir, and so you are very unhappy. You
do not know Him, but He is here, He is in me, He is in my words,
He is in thee, and even in those blasphemous words thou hast just
uttered!'' pronounced the Mason in a stern and tremulous voice.

He paused and sighed, evidently trying to calm himself.

``If He were not,'' he said quietly, ``you and I would not be
speaking of Him, my dear sir. Of what, of whom, are we speaking?
Whom hast thou denied?'' he suddenly asked with exulting
austerity and authority in his voice. ``Who invented Him, if He
did not exist? Whence came thy conception of the existence of
such an incomprehensible Being? didst thou, and why did the whole
world, conceive the idea of the existence of such an
incomprehensible Being, a Being all-powerful, eternal, and
infinite in all His attributes?...''

He stopped and remained silent for a long time.

Pierre could not and did not wish to break this silence.

``He exists, but to understand Him is hard,'' the Mason began
again, looking not at Pierre but straight before him, and turning
the leaves of his book with his old hands which from excitement
he could not keep still. ``If it were a man whose existence thou
didst doubt I could bring him to thee, could take him by the hand
and show him to thee. But how can I, an insignificant mortal,
show His omnipotence, His infinity, and all His mercy to one who
is blind, or who shuts his eyes that he may not see or understand
Him and may not see or understand his own vileness and
sinfulness?'' He paused again. ``Who art thou? Thou dreamest that
thou art wise because thou couldst utter those blasphemous
words,'' he went on, with a somber and scornful smile. ``And thou
art more foolish and unreasonable than a little child, who,
playing with the parts of a skillfully made watch, dares to say
that, as he does not understand its use, he does not believe in
the master who made it. To know Him is hard... For ages, from our
forefather Adam to our own day, we labor to attain that knowledge
and are still infinitely far from our aim; but in our lack of
understanding we see only our weakness and His greatness...''

Pierre listened with swelling heart, gazing into the Mason's face
with shining eyes, not interrupting or questioning him, but
believing with his whole soul what the stranger said. Whether he
accepted the wise reasoning contained in the Mason's words, or
believed as a child believes, in the speaker's tone of conviction
and earnestness, or the tremor of the speaker's voice---which
sometimes almost broke---or those brilliant aged eyes grown old
in this conviction, or the calm firmness and certainty of his
vocation, which radiated from his whole being (and which struck
Pierre especially by contrast with his own dejection and
hopelessness)---at any rate, Pierre longed with his whole soul to
believe and he did believe, and felt a joyful sense of comfort,
regeneration, and return to life.

``He is not to be apprehended by reason, but by life,'' said the
Mason.

``I do not understand,'' said Pierre, feeling with dismay doubts
reawakening. He was afraid of any want of clearness, any
weakness, in the Mason's arguments; he dreaded not to be able to
believe in him. ``I don't understand,'' he said, ``how it is that
the mind of man cannot attain the knowledge of which you speak.''

The Mason smiled with his gentle fatherly smile.

``The highest wisdom and truth are like the purest liquid we may
wish to imbibe,'' he said. ``Can I receive that pure liquid into
an impure vessel and judge of its purity? Only by the inner
purification of myself can I retain in some degree of purity the
liquid I receive.''

``Yes, yes, that is so,'' said Pierre joyfully.

``The highest wisdom is not founded on reason alone, not on those
worldly sciences of physics, history, chemistry, and the like,
into which intellectual knowledge is divided. The highest wisdom
is one. The highest wisdom has but one science---the science of
the whole---the science explaining the whole creation and man's
place in it. To receive that science it is necessary to purify
and renew one's inner self, and so before one can know, it is
necessary to believe and to perfect one's self. And to attain
this end, we have the light called conscience that God has
implanted in our souls.''

``Yes, yes,'' assented Pierre.

``Look then at thy inner self with the eyes of the spirit, and
ask thyself whether thou art content with thyself. What hast thou
attained relying on reason only? What art thou? You are young,
you are rich, you are clever, you are well educated. And what
have you done with all these good gifts? Are you content with
yourself and with your life?''

``No, I hate my life,'' Pierre muttered, wincing.

``Thou hatest it. Then change it, purify thyself; and as thou art
purified, thou wilt gain wisdom. Look at your life, my dear
sir. How have you spent it? In riotous orgies and debauchery,
receiving everything from society and giving nothing in
return. You have become the possessor of wealth. How have you
used it? What have you done for your neighbor? Have you ever
thought of your tens of thousands of slaves? Have you helped them
physically and morally? No! You have profited by their toil to
lead a profligate life. That is what you have done. Have you
chosen a post in which you might be of service to your neighbor?
No! You have spent your life in idleness. Then you married, my
dear sir---took on yourself responsibility for the guidance of a
young woman; and what have you done? You have not helped her to
find the way of truth, my dear sir, but have thrust her into an
abyss of deceit and misery. A man offended you and you shot him,
and you say you do not know God and hate your life. There is
nothing strange in that, my dear sir!''

After these words, the Mason, as if tired by his long discourse,
again leaned his arms on the back of the sofa and closed his
eyes. Pierre looked at that aged, stern, motionless, almost
lifeless face and moved his lips without uttering a sound. He
wished to say, ``Yes, a vile, idle, vicious life!'' but dared not
break the silence.

The Mason cleared his throat huskily, as old men do, and called
his servant.

``How about the horses?'' he asked, without looking at Pierre.

``The exchange horses have just come,'' answered the
servant. ``Will you not rest here?''

``No, tell them to harness.''

``Can he really be going away leaving me alone without having
told me all, and without promising to help me?'' thought Pierre,
rising with downcast head; and he began to pace the room,
glancing occasionally at the Mason. ``Yes, I never thought of it,
but I have led a contemptible and profligate life, though I did
not like it and did not want to,'' thought Pierre. ``But this man
knows the truth and, if he wished to, could disclose it to me.''

Pierre wished to say this to the Mason, but did not dare to. The
traveler, having packed his things with his practiced hands,
began fastening his coat. When he had finished, he turned to
Bezukhov, and said in a tone of indifferent politeness:

``Where are you going to now, my dear sir?''

``I?... I'm going to Petersburg,'' answered Pierre, in a
childlike, hesitating voice. ``I thank you. I agree with all you
have said. But do not suppose me to be so bad. With my whole soul
I wish to be what you would have me be, but I have never had help
from anyone... But it is I, above all, who am to blame for
everything. Help me, teach me, and perhaps I may...''

Pierre could not go on. He gulped and turned away.

The Mason remained silent for a long time, evidently considering.

``Help comes from God alone,'' he said, ``but such measure of
help as our Order can bestow it will render you, my dear sir. You
are going to Petersburg. Hand this to Count Willarski'' (he took
out his notebook and wrote a few words on a large sheet of paper
folded in four). ``Allow me to give you a piece of advice. When
you reach the capital, first of all devote some time to solitude
and self-examination and do not resume your former way of
life. And now I wish you a good journey, my dear sir,'' he added,
seeing that his servant had entered... ``and success.''

The traveler was Joseph Alexeevich Bazdeev, as Pierre saw from
the postmaster's book. Bazdeev had been one of the best-known
Freemasons and Martinists, even in Novikov's time. For a long
while after he had gone, Pierre did not go to bed or order horses
but paced up and down the room, pondering over his vicious past,
and with a rapturous sense of beginning anew pictured to himself
the blissful, irreproachable, virtuous future that seemed to him
so easy. It seemed to him that he had been vicious only because
he had somehow forgotten how good it is to be virtuous. Not a
trace of his former doubts remained in his soul. He firmly
believed in the possibility of the brotherhood of men united in
the aim of supporting one another in the path of virtue, and that
is how Freemasonry presented itself to him.

% % % % % % % % % % % % % % % % % % % % % % % % % % % % % % % % %
% % % % % % % % % % % % % % % % % % % % % % % % % % % % % % % % %
% % % % % % % % % % % % % % % % % % % % % % % % % % % % % % % % %
% % % % % % % % % % % % % % % % % % % % % % % % % % % % % % % % %
% % % % % % % % % % % % % % % % % % % % % % % % % % % % % % % % %
% % % % % % % % % % % % % % % % % % % % % % % % % % % % % % % % %
% % % % % % % % % % % % % % % % % % % % % % % % % % % % % % % % %
% % % % % % % % % % % % % % % % % % % % % % % % % % % % % % % % %
% % % % % % % % % % % % % % % % % % % % % % % % % % % % % % % % %
% % % % % % % % % % % % % % % % % % % % % % % % % % % % % % % % %
% % % % % % % % % % % % % % % % % % % % % % % % % % % % % % % % %
% % % % % % % % % % % % % % % % % % % % % % % % % % % % % %

\chapter*{Chapter III}
\ifaudio     
\marginpar{
\href{http://ia600200.us.archive.org/6/items/war_and_peace_05_0805_librivox/war_and_peace_05_03_tolstoy_64kb.mp3}{Audio}} 
\fi

\lettrine[lines=2, loversize=0.3, lraise=0]{\initfamily O}{n}
reaching Petersburg Pierre did not let anyone know of his
arrival, he went nowhere and spent whole days in reading Thomas a
Kempis, whose book had been sent him by someone unknown. One
thing he continually realized as he read that book: the joy,
hitherto unknown to him, of believing in the possibility of
attaining perfection, and in the possibility of active brotherly
love among men, which Joseph Alexeevich had revealed to him. A
week after his arrival, the young Polish count, Willarski, whom
Pierre had known slightly in Petersburg society, came into his
room one evening in the official and ceremonious manner in which
Dolokhov's second had called on him, and, having closed the door
behind him and satisfied himself that there was nobody else in
the room, addressed Pierre.

``I have come to you with a message and an offer, Count,'' he
said without sitting down. ``A person of very high standing in
our Brotherhood has made application for you to be received into
our Order before the usual term and has proposed to me to be your
sponsor. I consider it a sacred duty to fulfill that person's
wishes. Do you wish to enter the Brotherhood of Freemasons under
my sponsorship?''

The cold, austere tone of this man, whom he had almost always
before met at balls, amiably smiling in the society of the most
brilliant women, surprised Pierre.

``Yes, I do wish it,'' said he.

Willarski bowed his head.

``One more question, Count,'' he said, ``which I beg you to
answer in all sincerity---not as a future Mason but as an honest
man: have you renounced your former convictions---do you believe
in God?''

Pierre considered.

``Yes... yes, I believe in God,'' he said.

``In that case...'' began Willarski, but Pierre interrupted him.

``Yes, I do believe in God,'' he repeated.

``In that case we can go,'' said Willarski. ``My carriage is at
your service.''

Willarski was silent throughout the drive. To Pierre's inquiries
as to what he must do and how he should answer, Willarski only
replied that brothers more worthy than he would test him and that
Pierre had only to tell the truth.

Having entered the courtyard of a large house where the Lodge had
its headquarters, and having ascended a dark staircase, they
entered a small well-lit anteroom where they took off their
cloaks without the aid of a servant. From there they passed into
another room. A man in strange attire appeared at the
door. Willarski, stepping toward him, said something to him in
French in an undertone and then went up to a small wardrobe in
which Pierre noticed garments such as he had never seen
before. Having taken a kerchief from the cupboard, Willarski
bound Pierre's eyes with it and tied it in a knot behind,
catching some hairs painfully in the knot. Then he drew his face
down, kissed him, and taking him by the hand led him forward. The
hairs tied in the knot hurt Pierre and there were lines of pain
on his face and a shamefaced smile.  His huge figure, with arms
hanging down and with a puckered, though smiling face, moved
after Willarski with uncertain, timid steps.

Having led him about ten paces, Willarski stopped.

``Whatever happens to you,'' he said, ``you must bear it all
manfully if you have firmly resolved to join our Brotherhood.''
(Pierre nodded affirmatively.) ``When you hear a knock at the
door, you will uncover your eyes,'' added Willarski. ``I wish you
courage and success,'' and, pressing Pierre's hand, he went out.

Left alone, Pierre went on smiling in the same way. Once or twice
he shrugged his shoulders and raised his hand to the kerchief, as
if wishing to take it off, but let it drop again. The five
minutes spent with his eyes bandaged seemed to him an hour. His
arms felt numb, his legs almost gave way, it seemed to him that
he was tired out. He experienced a variety of most complex
sensations. He felt afraid of what would happen to him and still
more afraid of showing his fear. He felt curious to know what was
going to happen and what would be revealed to him; but most of
all, he felt joyful that the moment had come when he would at
last start on that path of regeneration and on the actively
virtuous life of which he had been dreaming since he met Joseph
Alexeevich. Loud knocks were heard at the door. Pierre took the
bandage off his eyes and glanced around him. The room was in
black darkness, only a small lamp was burning inside something
white. Pierre went nearer and saw that the lamp stood on a black
table on which lay an open book.  The book was the Gospel, and
the white thing with the lamp inside was a human skull with its
cavities and teeth. After reading the first words of the Gospel:
``In the beginning was the Word and the Word was with God,''
Pierre went round the table and saw a large open box filled with
something. It was a coffin with bones inside. He was not at all
surprised by what he saw. Hoping to enter on an entirely new life
quite unlike the old one, he expected everything to be unusual,
even more unusual than what he was seeing. A skull, a coffin, the
Gospel---it seemed to him that he had expected all this and even
more. Trying to stimulate his emotions he looked around. ``God,
death, love, the brotherhood of man,'' he kept saying to himself,
associating these words with vague yet joyful ideas. The door
opened and someone came in.

By the dim light, to which Pierre had already become accustomed,
he saw a rather short man. Having evidently come from the light
into the darkness, the man paused, then moved with cautious steps
toward the table and placed on it his small leather-gloved hands.

This short man had on a white leather apron which covered his
chest and part of his legs; he had on a kind of necklace above
which rose a high white ruffle, outlining his rather long face
which was lit up from below.

``For what have you come hither?'' asked the newcomer, turning in
Pierre's direction at a slight rustle made by the latter. ``Why
have you, who do not believe in the truth of the light and who
have not seen the light, come here? What do you seek from us?
Wisdom, virtue, enlightenment?''

At the moment the door opened and the stranger came in, Pierre
felt a sense of awe and veneration such as he had experienced in
his boyhood at confession; he felt himself in the presence of one
socially a complete stranger, yet nearer to him through the
brotherhood of man. With bated breath and beating heart he moved
toward the Rhetor (by which name the brother who prepared a
seeker for entrance into the Brotherhood was known). Drawing
nearer, he recognized in the Rhetor a man he knew, Smolyaninov,
and it mortified him to think that the newcomer was an
acquaintance---he wished him simply a brother and a virtuous
instructor.  For a long time he could not utter a word, so that
the Rhetor had to repeat his question.

``Yes... I... I... desire regeneration,'' Pierre uttered with
difficulty.

``Very well,'' said Smolyaninov, and went on at once: ``Have you
any idea of the means by which our holy Order will help you to
reach your aim?''  said he quietly and quickly.

``I... hope... for guidance... help... in regeneration,'' said
Pierre, with a trembling voice and some difficulty in utterance
due to his excitement and to being unaccustomed to speak of
abstract matters in Russian.

``What is your conception of Freemasonry?''

``I imagine that Freemasonry is the fraternity and equality of
men who have virtuous aims,'' said Pierre, feeling ashamed of the
inadequacy of his words for the solemnity of the moment, as he
spoke. ``I imagine...''

``Good!'' said the Rhetor quickly, apparently satisfied with this
answer.  ``Have you sought for means of attaining your aim in
religion?''

``No, I considered it erroneous and did not follow it,'' said
Pierre, so softly that the Rhetor did not hear him and asked him
what he was saying. ``I have been an atheist,'' answered Pierre.

``You are seeking for truth in order to follow its laws in your
life, therefore you seek wisdom and virtue. Is that not so?''
said the Rhetor, after a moment's pause.

``Yes, yes,'' assented Pierre.

The Rhetor cleared his throat, crossed his gloved hands on his
breast, and began to speak.

``Now I must disclose to you the chief aim of our Order,'' he
said, ``and if this aim coincides with yours, you may enter our
Brotherhood with profit. The first and chief object of our Order,
the foundation on which it rests and which no human power can
destroy, is the preservation and handing on to posterity of a
certain important mystery... which has come down to us from the
remotest ages, even from the first man---a mystery on which
perhaps the fate of mankind depends. But since this mystery is of
such a nature that nobody can know or use it unless he be
prepared by long and diligent self-purification, not everyone can
hope to attain it quickly. Hence we have a secondary aim, that of
preparing our members as much as possible to reform their hearts,
to purify and enlighten their minds, by means handed on to us by
tradition from those who have striven to attain this mystery, and
thereby to render them capable of receiving it.''

``By purifying and regenerating our members we try, thirdly, to
improve the whole human race, offering it in our members an
example of piety and virtue, and thereby try with all our might
to combat the evil which sways the world. Think this over and I
will come to you again.''

``To combat the evil which sways the world...'' Pierre repeated,
and a mental image of his future activity in this direction rose
in his mind.  He imagined men such as he had himself been a
fortnight ago, and he addressed an edifying exhortation to
them. He imagined to himself vicious and unfortunate people whom
he would assist by word and deed, imagined oppressors whose
victims he would rescue. Of the three objects mentioned by the
Rhetor, this last, that of improving mankind, especially appealed
to Pierre. The important mystery mentioned by the Rhetor, though
it aroused his curiosity, did not seem to him essential, and the
second aim, that of purifying and regenerating himself, did not
much interest him because at that moment he felt with delight
that he was already perfectly cured of his former faults and was
ready for all that was good.

Half an hour later, the Rhetor returned to inform the seeker of
the seven virtues, corresponding to the seven steps of Solomon's
temple, which every Freemason should cultivate in himself. These
virtues were: 1. Discretion, the keeping of the secrets of the
Order. 2. Obedience to those of higher ranks in the
Order. 3. Morality. 4. Love of mankind. 5.
Courage. 6. Generosity. 7. The love of death.

``In the seventh place, try, by the frequent thought of death,''
the Rhetor said, ``to bring yourself to regard it not as a
dreaded foe, but as a friend that frees the soul grown weary in
the labors of virtue from this distressful life, and leads it to
its place of recompense and peace.''

``Yes, that must be so,'' thought Pierre, when after these words
the Rhetor went away, leaving him to solitary meditation. ``It
must be so, but I am still so weak that I love my life, the
meaning of which is only now gradually opening before me.'' But
five of the other virtues which Pierre recalled, counting them on
his fingers, he felt already in his soul: courage, generosity,
morality, love of mankind, and especially obedience---which did
not even seem to him a virtue, but a joy. (He now felt so glad to
be free from his own lawlessness and to submit his will to those
who knew the indubitable truth.) He forgot what the seventh
virtue was and could not recall it.

The third time the Rhetor came back more quickly and asked Pierre
whether he was still firm in his intention and determined to
submit to all that would be required of him.

``I am ready for everything,'' said Pierre.

``I must also inform you,'' said the Rhetor, ``that our Order
delivers its teaching not in words only but also by other means,
which may perhaps have a stronger effect on the sincere seeker
after wisdom and virtue than mere words. This chamber with what
you see therein should already have suggested to your heart, if
it is sincere, more than words could do. You will perhaps also
see in your further initiation a like method of
enlightenment. Our Order imitates the ancient societies that
explained their teaching by hieroglyphics. A hieroglyph,'' said
the Rhetor, ``is an emblem of something not cognizable by the
senses but which possesses qualities resembling those of the
symbol.''

Pierre knew very well what a hieroglyph was, but dared not
speak. He listened to the Rhetor in silence, feeling from all he
said that his ordeal was about to begin.

``If you are resolved, I must begin your initiation,'' said the
Rhetor coming closer to Pierre. ``In token of generosity I ask
you to give me all your valuables.''

``But I have nothing here,'' replied Pierre, supposing that he
was asked to give up all he possessed.

``What you have with you: watch, money, rings...''

Pierre quickly took out his purse and watch, but could not manage
for some time to get the wedding ring off his fat finger. When
that had been done, the Rhetor said:

``In token of obedience, I ask you to undress.''

Pierre took off his coat, waistcoat, and left boot according to
the Rhetor's instructions. The Mason drew the shirt back from
Pierre's left breast, and stooping down pulled up the left leg of
his trousers to above the knee. Pierre hurriedly began taking off
his right boot also and was going to tuck up the other trouser
leg to save this stranger the trouble, but the Mason told him
that was not necessary and gave him a slipper for his left
foot. With a childlike smile of embarrassment, doubt, and
self-derision, which appeared on his face against his will,
Pierre stood with his arms hanging down and legs apart, before
his brother Rhetor, and awaited his further commands.

``And now, in token of candor, I ask you to reveal to me your
chief passion,'' said the latter.

``My passion! I have had so many,'' replied Pierre.

``That passion which more than all others caused you to waver on
the path of virtue,'' said the Mason.

Pierre paused, seeking a reply.

``Wine? Gluttony? Idleness? Laziness? Irritability? Anger?
Women?'' He went over his vices in his mind, not knowing to which
of them to give the pre-eminence.

``Women,'' he said in a low, scarcely audible voice.

The Mason did not move and for a long time said nothing after
this answer. At last he moved up to Pierre and, taking the
kerchief that lay on the table, again bound his eyes.

``For the last time I say to you---turn all your attention upon
yourself, put a bridle on your senses, and seek blessedness, not
in passion but in your own heart. The source of blessedness is
not without us but within...''

Pierre had already long been feeling in himself that refreshing
source of blessedness which now flooded his heart with glad
emotion.

% % % % % % % % % % % % % % % % % % % % % % % % % % % % % % % % %
% % % % % % % % % % % % % % % % % % % % % % % % % % % % % % % % %
% % % % % % % % % % % % % % % % % % % % % % % % % % % % % % % % %
% % % % % % % % % % % % % % % % % % % % % % % % % % % % % % % % %
% % % % % % % % % % % % % % % % % % % % % % % % % % % % % % % % %
% % % % % % % % % % % % % % % % % % % % % % % % % % % % % % % % %
% % % % % % % % % % % % % % % % % % % % % % % % % % % % % % % % %
% % % % % % % % % % % % % % % % % % % % % % % % % % % % % % % % %
% % % % % % % % % % % % % % % % % % % % % % % % % % % % % % % % %
% % % % % % % % % % % % % % % % % % % % % % % % % % % % % % % % %
% % % % % % % % % % % % % % % % % % % % % % % % % % % % % % % % %
% % % % % % % % % % % % % % % % % % % % % % % % % % % % % %

\chapter*{Chapter IV}
\ifaudio     
\marginpar{
\href{http://ia600200.us.archive.org/6/items/war_and_peace_05_0805_librivox/war_and_peace_05_04_tolstoy_64kb.mp3}{Audio}} 
\fi

\lettrine[lines=2, loversize=0.3, lraise=0]{\initfamily S}{oon}
after this there came into the dark chamber to fetch Pierre,
not the Rhetor but Pierre's sponsor, Willarski, whom he
recognized by his voice. To fresh questions as to the firmness of
his resolution Pierre replied: ``Yes, yes, I agree,'' and with a
beaming, childlike smile, his fat chest uncovered, stepping
unevenly and timidly in one slippered and one booted foot, he
advanced, while Willarski held a sword to his bare chest. He was
conducted from that room along passages that turned backwards and
forwards and was at last brought to the doors of the
Lodge. Willarski coughed, he was answered by the masonic knock
with mallets, the doors opened before them. A bass voice (Pierre
was still blindfolded) questioned him as to who he was, when and
where he was born, and so on. Then he was again led somewhere
still blindfolded, and as they went along he was told allegories
of the toils of his pilgrimage, of holy friendship, of the
Eternal Architect of the universe, and of the courage with which
he should endure toils and dangers. During these wanderings,
Pierre noticed that he was spoken of now as the ``Seeker,'' now
as the ``Sufferer,'' and now as the ``Postulant,'' to the
accompaniment of various knockings with mallets and swords. As he
was being led up to some object he noticed a hesitation and
uncertainty among his conductors. He heard those around him
disputing in whispers and one of them insisting that he should be
led along a certain carpet.  After that they took his right hand,
placed it on something, and told him to hold a pair of compasses
to his left breast with the other hand and to repeat after
someone who read aloud an oath of fidelity to the laws of the
Order. The candles were then extinguished and some spirit
lighted, as Pierre knew by the smell, and he was told that he
would now see the lesser light. The bandage was taken off his
eyes and, by the faint light of the burning spirit, Pierre, as in
a dream, saw several men standing before him, wearing aprons like
the Rhetor's and holding swords in their hands pointed at his
breast. Among them stood a man whose white shirt was stained with
blood. On seeing this, Pierre moved forward with his breast
toward the swords, meaning them to pierce it.  But the swords
were drawn back from him and he was at once blindfolded again.

``Now thou hast seen the lesser light,'' uttered a voice. Then
the candles were relit and he was told that he would see the full
light; the bandage was again removed and more than ten voices
said together: ``Sic transit gloria mundi.''

Pierre gradually began to recover himself and looked about at the
room and at the people in it. Round a long table covered with
black sat some twelve men in garments like those he had already
seen. Some of them Pierre had met in Petersburg society. In the
President's chair sat a young man he did not know, with a
peculiar cross hanging from his neck.  On his right sat the
Italian abbe whom Pierre had met at Anna Pavlovna's two years
before. There were also present a very distinguished dignitary
and a Swiss who had formerly been tutor at the Kuragins'. All
maintained a solemn silence, listening to the words of the
President, who held a mallet in his hand. Let into the wall was a
star-shaped light. At one side of the table was a small carpet
with various figures worked upon it, at the other was something
resembling an altar on which lay a Testament and a skull. Round
it stood seven large candlesticks like those used in
churches. Two of the brothers led Pierre up to the altar, placed
his feet at right angles, and bade him lie down, saying that he
must prostrate himself at the Gates of the Temple.

``He must first receive the trowel,'' whispered one of the
brothers.

``Oh, hush, please!'' said another.

Pierre, perplexed, looked round with his shortsighted eyes
without obeying, and suddenly doubts arose in his mind. ``Where
am I? What am I doing? Aren't they laughing at me? Shan't I be
ashamed to remember this?'' But these doubts only lasted a
moment. Pierre glanced at the serious faces of those around,
remembered all he had already gone through, and realized that he
could not stop halfway. He was aghast at his hesitation and,
trying to arouse his former devotional feeling, prostrated
himself before the Gates of the Temple. And really, the feeling
of devotion returned to him even more strongly than before. When
he had lain there some time, he was told to get up, and a white
leather apron, such as the others wore, was put on him: he was
given a trowel and three pairs of gloves, and then the Grand
Master addressed him. He told him that he should try to do
nothing to stain the whiteness of that apron, which symbolized
strength and purity; then of the unexplained trowel, he told him
to toil with it to cleanse his own heart from vice, and
indulgently to smooth with it the heart of his neighbor. As to
the first pair of gloves, a man's, he said that Pierre could not
know their meaning but must keep them. The second pair of man's
gloves he was to wear at the meetings, and finally of the third,
a pair of women's gloves, he said: ``Dear brother, these woman's
gloves are intended for you too. Give them to the woman whom you
shall honor most of all. This gift will be a pledge of your
purity of heart to her whom you select to be your worthy helpmeet
in Masonry.'' And after a pause, he added: ``But beware, dear
brother, that these gloves do not deck hands that are unclean.''
While the Grand Master said these last words it seemed to Pierre
that he grew embarrassed. Pierre himself grew still more
confused, blushed like a child till tears came to his eyes, began
looking about him uneasily, and an awkward pause followed.

This silence was broken by one of the brethren, who led Pierre up
to the rug and began reading to him from a manuscript book an
explanation of all the figures on it: the sun, the moon, a
hammer, a plumb line, a trowel, a rough stone and a squared
stone, a pillar, three windows, and so on. Then a place was
assigned to Pierre, he was shown the signs of the Lodge, told the
password, and at last was permitted to sit down. The Grand Master
began reading the statutes. They were very long, and Pierre, from
joy, agitation, and embarrassment, was not in a state to
understand what was being read. He managed to follow only the
last words of the statutes and these remained in his mind.

``In our temples we recognize no other distinctions,'' read the
Grand Master, ``but those between virtue and vice. Beware of
making any distinctions which may infringe equality. Fly to a
brother's aid whoever he may be, exhort him who goeth astray,
raise him that falleth, never bear malice or enmity toward thy
brother. Be kindly and courteous.  Kindle in all hearts the flame
of virtue. Share thy happiness with thy neighbor, and may envy
never dim the purity of that bliss. Forgive thy enemy, do not
avenge thyself except by doing him good. Thus fulfilling the
highest law thou shalt regain traces of the ancient dignity which
thou hast lost.''

He finished and, getting up, embraced and kissed Pierre, who,
with tears of joy in his eyes, looked round him, not knowing how
to answer the congratulations and greetings from acquaintances
that met him on all sides. He acknowledged no acquaintances but
saw in all these men only brothers, and burned with impatience to
set to work with them.

The Grand Master rapped with his mallet. All the Masons sat down
in their places, and one of them read an exhortation on the
necessity of humility.

The Grand Master proposed that the last duty should be performed,
and the distinguished dignitary who bore the title of ``Collector
of Alms'' went round to all the brothers. Pierre would have liked
to subscribe all he had, but fearing that it might look like
pride subscribed the same amount as the others.

The meeting was at an end, and on reaching home Pierre felt as if
he had returned from a long journey on which he had spent dozens
of years, had become completely changed, and had quite left
behind his former habits and way of life.

% % % % % % % % % % % % % % % % % % % % % % % % % % % % % % % % %
% % % % % % % % % % % % % % % % % % % % % % % % % % % % % % % % %
% % % % % % % % % % % % % % % % % % % % % % % % % % % % % % % % %
% % % % % % % % % % % % % % % % % % % % % % % % % % % % % % % % %
% % % % % % % % % % % % % % % % % % % % % % % % % % % % % % % % %
% % % % % % % % % % % % % % % % % % % % % % % % % % % % % % % % %
% % % % % % % % % % % % % % % % % % % % % % % % % % % % % % % % %
% % % % % % % % % % % % % % % % % % % % % % % % % % % % % % % % %
% % % % % % % % % % % % % % % % % % % % % % % % % % % % % % % % %
% % % % % % % % % % % % % % % % % % % % % % % % % % % % % % % % %
% % % % % % % % % % % % % % % % % % % % % % % % % % % % % % % % %
% % % % % % % % % % % % % % % % % % % % % % % % % % % % % %

\chapter*{Chapter V}
\ifaudio     
\marginpar{
\href{http://ia600200.us.archive.org/6/items/war_and_peace_05_0805_librivox/war_and_peace_05_05_tolstoy_64kb.mp3}{Audio}} 
\fi

\lettrine[lines=2, loversize=0.3, lraise=0]{\initfamily T}{he}
day after he had been received into the Lodge, Pierre was
sitting at home reading a book and trying to fathom the
significance of the Square, one side of which symbolized God,
another moral things, a third physical things, and the fourth a
combination of these. Now and then his attention wandered from
the book and the Square and he formed in imagination a new plan
of life. On the previous evening at the Lodge, he had heard that
a rumor of his duel had reached the Emperor and that it would be
wiser for him to leave Petersburg. Pierre proposed going to his
estates in the south and there attending to the welfare of his
serfs. He was joyfully planning this new life, when Prince Vasili
suddenly entered the room.

``My dear fellow, what have you been up to in Moscow? Why have
you quarreled with Helene, mon cher? You are under a delusion,''
said Prince Vasili, as he entered. ``I know all about it, and I
can tell you positively that Helene is as innocent before you as
Christ was before the Jews.''

Pierre was about to reply, but Prince Vasili interrupted him.

``And why didn't you simply come straight to me as to a friend? I
know all about it and understand it all,'' he said. ``You behaved
as becomes a man who values his honor, perhaps too hastily, but
we won't go into that. But consider the position in which you are
placing her and me in the eyes of society, and even of the
court,'' he added, lowering his voice. ``She is living in Moscow
and you are here. Remember, dear boy,'' and he drew Pierre's arm
downwards, ``it is simply a misunderstanding. I expect you feel
it so yourself. Let us write her a letter at once, and she'll
come here and all will be explained, or else, my dear boy, let me
tell you it's quite likely you'll have to suffer for it.''

Prince Vasili gave Pierre a significant look.

``I know from reliable sources that the Dowager Empress is taking
a keen interest in the whole affair. You know she is very
gracious to Helene.''

Pierre tried several times to speak, but, on one hand, Prince
Vasili did not let him and, on the other, Pierre himself feared
to begin to speak in the tone of decided refusal and disagreement
in which he had firmly resolved to answer his
father-in-law. Moreover, the words of the masonic statutes,
\emph{be kindly and courteous}, recurred to him. He blinked, went
red, got up and sat down again, struggling with himself to do
what was for him the most difficult thing in life---to say an
unpleasant thing to a man's face, to say what the other, whoever
he might be, did not expect. He was so used to submitting to
Prince Vasili's tone of careless self-assurance that he felt he
would be unable to withstand it now, but he also felt that on
what he said now his future depended---whether he would follow
the same old road, or that new path so attractively shown him by
the Masons, on which he firmly believed he would be reborn to a
new life.

``Now, dear boy,'' said Prince Vasili playfully, ``say 'yes,' and
I'll write to her myself, and we will kill the fatted calf.''

But before Prince Vasili had finished his playful speech, Pierre,
without looking at him, and with a kind of fury that made him
like his father, muttered in a whisper:

``Prince, I did not ask you here. Go, please go!'' And he jumped
up and opened the door for him.

``Go!'' he repeated, amazed at himself and glad to see the look
of confusion and fear that showed itself on Prince Vasili's face.

``What's the matter with you? Are you ill?''

``Go!'' the quivering voice repeated. And Prince Vasili had to go
without receiving any explanation.

A week later, Pierre, having taken leave of his new friends, the
Masons, and leaving large sums of money with them for alms, went
away to his estates. His new brethren gave him letters to the
Kiev and Odessa Masons and promised to write to him and guide him
in his new activity.

% % % % % % % % % % % % % % % % % % % % % % % % % % % % % % % % %
% % % % % % % % % % % % % % % % % % % % % % % % % % % % % % % % %
% % % % % % % % % % % % % % % % % % % % % % % % % % % % % % % % %
% % % % % % % % % % % % % % % % % % % % % % % % % % % % % % % % %
% % % % % % % % % % % % % % % % % % % % % % % % % % % % % % % % %
% % % % % % % % % % % % % % % % % % % % % % % % % % % % % % % % %
% % % % % % % % % % % % % % % % % % % % % % % % % % % % % % % % %
% % % % % % % % % % % % % % % % % % % % % % % % % % % % % % % % %
% % % % % % % % % % % % % % % % % % % % % % % % % % % % % % % % %
% % % % % % % % % % % % % % % % % % % % % % % % % % % % % % % % %
% % % % % % % % % % % % % % % % % % % % % % % % % % % % % % % % %
% % % % % % % % % % % % % % % % % % % % % % % % % % % % % %

\chapter*{Chapter VI}
\ifaudio     
\marginpar{
\href{http://ia600200.us.archive.org/6/items/war_and_peace_05_0805_librivox/war_and_peace_05_06_tolstoy_64kb.mp3}{Audio}} 
\fi

\lettrine[lines=2, loversize=0.3, lraise=0]{\initfamily T}{he}
duel between Pierre and Dolokhov was hushed up and, in spite
of the Emperor's severity regarding duels at that time, neither
the principals nor their seconds suffered for it. But the story
of the duel, confirmed by Pierre's rupture with his wife, was the
talk of society. Pierre who had been regarded with patronizing
condescension when he was an illegitimate son, and petted and
extolled when he was the best match in Russia, had sunk greatly
in the esteem of society after his marriage---when the
marriageable daughters and their mothers had nothing to hope from
him---especially as he did not know how, and did not wish, to
court society's favor. Now he alone was blamed for what had
happened, he was said to be insanely jealous and subject like his
father to fits of bloodthirsty rage. And when after Pierre's
departure Helene returned to Petersburg, she was received by all
her acquaintances not only cordially, but even with a shade of
deference due to her misfortune.  When conversation turned on her
husband Helene assumed a dignified expression, which with
characteristic tact she had acquired though she did not
understand its significance. This expression suggested that she
had resolved to endure her troubles uncomplainingly and that her
husband was a cross laid upon her by God. Prince Vasili expressed
his opinion more openly. He shrugged his shoulders when Pierre
was mentioned and, pointing to his forehead, remarked:

``A bit touched---I always said so.''

``I said from the first,'' declared Anna Pavlovna referring to
Pierre, ``I said at the time and before anyone else'' (she
insisted on her priority) ``that that senseless young man was
spoiled by the depraved ideas of these days. I said so even at
the time when everybody was in raptures about him, when he had
just returned from abroad, and when, if you remember, he posed as
a sort of Marat at one of my soirees. And how has it ended? I was
against this marriage even then and foretold all that has
happened.''

Anna Pavlovna continued to give on free evenings the same kind of
soirees as before---such as she alone had the gift of
arranging---at which was to be found ``the cream of really good
society, the bloom of the intellectual essence of Petersburg,''
as she herself put it. Besides this refined selection of society
Anna Pavlovna's receptions were also distinguished by the fact
that she always presented some new and interesting person to the
visitors and that nowhere else was the state of the political
thermometer of legitimate Petersburg court society so dearly and
distinctly indicated.

Toward the end of 1806, when all the sad details of Napoleon's
destruction of the Prussian army at Jena and Auerstadt and the
surrender of most of the Prussian fortresses had been received,
when our troops had already entered Prussia and our second war
with Napoleon was beginning, Anna Pavlovna gave one of her
soirees. The \emph{cream of really good society} consisted of the
fascinating Helene, forsaken by her husband, Mortemart, the
delightful Prince Hippolyte who had just returned from Vienna,
two diplomatists, the old aunt, a young man referred to in that
drawing room as \emph{a man of great merit} (un homme de beaucoup de
merite), a newly appointed maid of honor and her mother, and
several other less noteworthy persons.

The novelty Anna Pavlovna was setting before her guests that
evening was Boris Drubetskoy, who had just arrived as a special
messenger from the Prussian army and was aide-de-camp to a very
important personage.

The temperature shown by the political thermometer to the company
that evening was this:

``Whatever the European sovereigns and commanders may do to
countenance Bonaparte, and to cause me, and us in general,
annoyance and mortification, our opinion of Bonaparte cannot
alter. We shall not cease to express our sincere views on that
subject, and can only say to the King of Prussia and others: 'So
much the worse for you. Tu l'as voulu, George Dandin,' that's all
we have to say about it!''

When Boris, who was to be served up to the guests, entered the
drawing room, almost all the company had assembled, and the
conversation, guided by Anna Pavlovna, was about our diplomatic
relations with Austria and the hope of an alliance with her.

Boris, grown more manly and looking fresh, rosy and
self-possessed, entered the drawing room elegantly dressed in the
uniform of an aide-de-camp and was duly conducted to pay his
respects to the aunt and then brought back to the general circle.

Anna Pavlovna gave him her shriveled hand to kiss and introduced
him to several persons whom he did not know, giving him a
whispered description of each.

``Prince Hippolyte Kuragin, M. Krug, the charge d'affaires from
Co\-pen\-ha\-gen---a profound intellect,'' and simply, ``Mr. Shitov---a
man of great merit''---this of the man usually so described.

Thanks to Anna Mikhaylovna's efforts, his own tastes, and the
peculiarities of his reserved nature, Boris had managed during
his service to place himself very advantageously. He was
aide-de-camp to a very important personage, had been sent on a
very important mission to Prussia, and had just returned from
there as a special messenger. He had become thoroughly conversant
with that unwritten code with which he had been so pleased at
Olmutz and according to which an ensign might rank incomparably
higher than a general, and according to which what was needed for
success in the service was not effort or work, or courage, or
perseverance, but only the knowledge of how to get on with those
who can grant rewards, and he was himself often surprised at the
rapidity of his success and at the inability of others to
understand these things. In consequence of this discovery his
whole manner of life, all his relations with old friends, all his
plans for his future, were completely altered. He was not rich,
but would spend his last groat to be better dressed than others,
and would rather deprive himself of many pleasures than allow
himself to be seen in a shabby equipage or appear in the streets
of Petersburg in an old uniform. He made friends with and sought
the acquaintance of only those above him in position and who
could therefore be of use to him. He liked Petersburg and
despised Moscow. The remembrance of the Rostovs' house and of his
childish love for Natasha was unpleasant to him and he had not
once been to see the Rostovs since the day of his departure for
the army. To be in Anna Pavlovna's drawing room he considered an
important step up in the service, and he at once understood his
role, letting his hostess make use of whatever interest he had to
offer. He himself carefully scanned each face, appraising the
possibilities of establishing intimacy with each of those
present, and the advantages that might accrue. He took the seat
indicated to him beside the fair Helene and listened to the
general conversation.

``Vienna considers the bases of the proposed treaty so
unattainable that not even a continuity of most brilliant
successes would secure them, and she doubts the means we have of
gaining them. That is the actual phrase used by the Vienna
cabinet,'' said the Danish charge d'affaires.

``The doubt is flattering,'' said ``the man of profound
intellect,'' with a subtle smile.

``We must distinguish between the Vienna cabinet and the Emperor
of Austria,'' said Mortemart. ``The Emperor of Austria can never
have thought of such a thing, it is only the cabinet that says
it.''

``Ah, my dear vicomte,'' put in Anna Pavlovna, ``L'Urope'' (for
some reason she called it Urope as if that were a specially
refined French pronunciation which she could allow herself when
conversing with a Frenchman), ``L'Urope ne sera jamais notre
alliee sincere.''\footnote{``Europe will never be our sincere
ally.''}

After that Anna Pavlovna led up to the courage and firmness of
the King of Prussia, in order to draw Boris into the
conversation.

Boris listened attentively to each of the speakers, awaiting his
turn, but managed meanwhile to look round repeatedly at his
neighbor, the beautiful Helene, whose eyes several times met
those of the handsome young aide-de-camp with a smile.

Speaking of the position of Prussia, Anna Pavlovna very naturally
asked Boris to tell them about his journey to Glogau and in what
state he found the Prussian army. Boris, speaking with
deliberation, told them in pure, correct French many interesting
details about the armies and the court, carefully abstaining from
expressing an opinion of his own about the facts he was
recounting. For some time he engrossed the general attention, and
Anna Pavlovna felt that the novelty she had served up was
received with pleasure by all her visitors. The greatest
attention of all to Boris' narrative was shown by Helene. She
asked him several questions about his journey and seemed greatly
interested in the state of the Prussian army. As soon as he had
finished she turned to him with her usual smile.

``You absolutely must come and see me,'' she said in a tone that
implied that, for certain considerations he could not know of,
this was absolutely necessary.

``On Tuesday between eight and nine. It will give me great
pleasure.''

Boris promised to fulfill her wish and was about to begin a
conversation with her, when Anna Pavlovna called him away on the
pretext that her aunt wished to hear him.

``You know her husband, of course?'' said Anna Pavlovna, closing
her eyes and indicating Helene with a sorrowful gesture. ``Ah,
she is such an unfortunate and charming woman! Don't mention him
before her---please don't! It is too painful for her!''

% % % % % % % % % % % % % % % % % % % % % % % % % % % % % % % % %
% % % % % % % % % % % % % % % % % % % % % % % % % % % % % % % % %
% % % % % % % % % % % % % % % % % % % % % % % % % % % % % % % % %
% % % % % % % % % % % % % % % % % % % % % % % % % % % % % % % % %
% % % % % % % % % % % % % % % % % % % % % % % % % % % % % % % % %
% % % % % % % % % % % % % % % % % % % % % % % % % % % % % % % % %
% % % % % % % % % % % % % % % % % % % % % % % % % % % % % % % % %
% % % % % % % % % % % % % % % % % % % % % % % % % % % % % % % % %
% % % % % % % % % % % % % % % % % % % % % % % % % % % % % % % % %
% % % % % % % % % % % % % % % % % % % % % % % % % % % % % % % % %
% % % % % % % % % % % % % % % % % % % % % % % % % % % % % % % % %
% % % % % % % % % % % % % % % % % % % % % % % % % % % % % %

\chapter*{Chapter VII}
\ifaudio     
\marginpar{
\href{http://ia600200.us.archive.org/6/items/war_and_peace_05_0805_librivox/war_and_peace_05_07_tolstoy_64kb.mp3}{Audio}} 
\fi

\lettrine[lines=2, loversize=0.3, lraise=0]{\initfamily W}{hen}
Boris and Anna Pavlovna returned to the others Prince
Hippolyte had the ear of the company.

Bending forward in his armchair he said: ``Le Roi de Prusse!''
and having said this laughed. Everyone turned toward him.

``Le Roi de Prusse?'' Hippolyte said interrogatively, again
laughing, and then calmly and seriously sat back in his
chair. Anna Pavlovna waited for him to go on, but as he seemed
quite decided to say no more she began to tell of how at Potsdam
the impious Bonaparte had stolen the sword of Frederick the
Great.

``It is the sword of Frederick the Great which I...'' she began,
but Hippolyte interrupted her with the words: ``Le Roi de
Prusse...'' and again, as soon as all turned toward him, excused
himself and said no more.

Anna Pavlovna frowned. Mortemart, Hippolyte's friend, addressed
him firmly.

``Come now, what about your Roi de Prusse?''

Hippolyte laughed as if ashamed of laughing.

``Oh, it's nothing. I only wished to say...'' (he wanted to
repeat a joke he had heard in Vienna and which he had been trying
all that evening to get in) ``I only wished to say that we are
wrong to fight pour le Roi de Prusse!''

Boris smiled circumspectly, so that it might be taken as ironical
or appreciative according to the way the joke was
received. Everybody laughed.

``Your joke is too bad, it's witty but unjust,'' said Anna
Pavlovna, shaking her little shriveled finger at him.

``We are not fighting pour le Roi de Prusse, but for right
principles.  Oh, that wicked Prince Hippolyte!'' she said.

The conversation did not flag all evening and turned chiefly on
the political news. It became particularly animated toward the
end of the evening when the rewards bestowed by the Emperor were
mentioned.

``You know N--- N--- received a snuffbox with the portrait last
year?'' said ``the man of profound intellect.'' ``Why shouldn't
S--- S--- get the same distinction?''

``Pardon me! A snuffbox with the Emperor's portrait is a reward
but not a distinction,'' said the diplomatist---``a gift,
rather.''

``There are precedents, I may mention Schwarzenberg.''

``It's impossible,'' replied another.

``Will you bet? The ribbon of the order is a different
matter...''

When everybody rose to go, Helene who had spoken very little all
the evening again turned to Boris, asking him in a tone of
caressing significant command to come to her on Tuesday.

``It is of great importance to me,'' she said, turning with a
smile toward Anna Pavlovna, and Anna Pavlovna, with the same sad
smile with which she spoke of her exalted patroness, supported
Helene's wish.

It seemed as if from some words Boris had spoken that evening
about the Prussian army, Helene had suddenly found it necessary
to see him. She seemed to promise to explain that necessity to
him when he came on Tuesday.

But on Tuesday evening, having come to Helene's splendid salon,
Boris received no clear explanation of why it had been necessary
for him to come. There were other guests and the countess talked
little to him, and only as he kissed her hand on taking leave
said unexpectedly and in a whisper, with a strangely unsmiling
face: ``Come to dinner tomorrow... in the evening. You must
come... Come!''

During that stay in Petersburg, Boris became an intimate in the
countess' house.

% % % % % % % % % % % % % % % % % % % % % % % % % % % % % % % % %
% % % % % % % % % % % % % % % % % % % % % % % % % % % % % % % % %
% % % % % % % % % % % % % % % % % % % % % % % % % % % % % % % % %
% % % % % % % % % % % % % % % % % % % % % % % % % % % % % % % % %
% % % % % % % % % % % % % % % % % % % % % % % % % % % % % % % % %
% % % % % % % % % % % % % % % % % % % % % % % % % % % % % % % % %
% % % % % % % % % % % % % % % % % % % % % % % % % % % % % % % % %
% % % % % % % % % % % % % % % % % % % % % % % % % % % % % % % % %
% % % % % % % % % % % % % % % % % % % % % % % % % % % % % % % % %
% % % % % % % % % % % % % % % % % % % % % % % % % % % % % % % % %
% % % % % % % % % % % % % % % % % % % % % % % % % % % % % % % % %
% % % % % % % % % % % % % % % % % % % % % % % % % % % % % %

\chapter*{Chapter VIII}

\lettrine[lines=2, loversize=0.3, lraise=0]{\initfamily T}{he}
war was flaming up and nearing the Russian
frontier. Everywhere one heard curses on Bonaparte, \emph{the enemy
of mankind}. Militiamen and recruits were being enrolled in the
villages, and from the seat of war came contradictory news, false
as usual and therefore variously interpreted. The life of old
Prince Bolkonski, Prince Andrew, and Princess Mary had greatly
changed since 1805.

In 1806 the old prince was made one of the eight commanders in
chief then appointed to supervise the enrollment decreed
throughout Russia.  Despite the weakness of age, which had become
particularly noticeable since the time when he thought his son
had been killed, he did not think it right to refuse a duty to
which he had been appointed by the Emperor himself, and this
fresh opportunity for action gave him new energy and strength. He
was continually traveling through the three provinces entrusted
to him, was pedantic in the fulfillment of his duties, severe to
cruel with his subordinates, and went into everything down to the
minutest details himself. Princess Mary had ceased taking lessons
in mathematics from her father, and when the old prince was at
home went to his study with the wet nurse and little Prince
Nicholas (as his grandfather called him). The baby Prince
Nicholas lived with his wet nurse and nurse Savishna in the late
princess' rooms and Princess Mary spent most of the day in the
nursery, taking a mother's place to her little nephew as best she
could. Mademoiselle Bourienne, too, seemed passionately fond of
the boy, and Princess Mary often deprived herself to give her
friend the pleasure of dandling the little angel---as she called
her nephew---and playing with him.

Near the altar of the church at Bald Hills there was a chapel
over the tomb of the little princess, and in this chapel was a
marble monument brought from Italy, representing an angel with
outspread wings ready to fly upwards. The angel's upper lip was
slightly raised as though about to smile, and once on coming out
of the chapel Prince Andrew and Princess Mary admitted to one
another that the angel's face reminded them strangely of the
little princess. But what was still stranger, though of this
Prince Andrew said nothing to his sister, was that in the
expression the sculptor had happened to give the angel's face,
Prince Andrew read the same mild reproach he had read on the face
of his dead wife: ``Ah, why have you done this to me?''

Soon after Prince Andrew's return the old prince made over to him
a large estate, Bogucharovo, about twenty-five miles from Bald
Hills.  Partly because of the depressing memories associated with
Bald Hills, partly because Prince Andrew did not always feel
equal to bearing with his father's peculiarities, and partly
because he needed solitude, Prince Andrew made use of
Bogucharovo, began building and spent most of his time there.

After the Austerlitz campaign Prince Andrew had firmly resolved
not to continue his military service, and when the war
recommenced and everybody had to serve, he took a post under his
father in the recruitment so as to avoid active service. The old
prince and his son seemed to have changed roles since the
campaign of 1805. The old man, roused by activity, expected the
best results from the new campaign, while Prince Andrew on the
contrary, taking no part in the war and secretly regretting this,
saw only the dark side.

On February 26, 1807, the old prince set off on one of his
circuits.  Prince Andrew remained at Bald Hills as usual during
his father's absence. Little Nicholas had been unwell for four
days. The coachman who had driven the old prince to town returned
bringing papers and letters for Prince Andrew.

Not finding the young prince in his study the valet went with the
letters to Princess Mary's apartments, but did not find him
there. He was told that the prince had gone to the nursery.

``If you please, your excellency, Petrusha has brought some
papers,'' said one of the nursemaids to Prince Andrew who was
sitting on a child's little chair while, frowning and with
trembling hands, he poured drops from a medicine bottle into a
wineglass half full of water.

``What is it?'' he said crossly, and, his hand shaking
unintentionally, he poured too many drops into the glass. He
threw the mixture onto the floor and asked for some more
water. The maid brought it.

There were in the room a child's cot, two boxes, two armchairs, a
table, a child's table, and the little chair on which Prince
Andrew was sitting. The curtains were drawn, and a single candle
was burning on the table, screened by a bound music book so that
the light did not fall on the cot.

``My dear,'' said Princess Mary, addressing her brother from
beside the cot where she was standing, ``better wait a
bit... later...''

``Oh, leave off, you always talk nonsense and keep putting things
off---and this is what comes of it!'' said Prince Andrew in an
exasperated whisper, evidently meaning to wound his sister.

``My dear, really... it's better not to wake him... he's
asleep,'' said the princess in a tone of entreaty.

Prince Andrew got up and went on tiptoe up to the little bed,
wineglass in hand.

``Perhaps we'd really better not wake him,'' he said hesitating.

``As you please... really... I think so... but as you please,''
said Princess Mary, evidently intimidated and confused that her
opinion had prevailed. She drew her brother's attention to the
maid who was calling him in a whisper.

It was the second night that neither of them had slept, watching
the boy who was in a high fever. These last days, mistrusting
their household doctor and expecting another for whom they had
sent to town, they had been trying first one remedy and then
another. Worn out by sleeplessness and anxiety they threw their
burden of sorrow on one another and reproached and disputed with
each other.

``Petrusha has come with papers from your father,'' whispered the
maid.

Prince Andrew went out.

``Devil take them!'' he muttered, and after listening to the
verbal instructions his father had sent and taking the
correspondence and his father's letter, he returned to the
nursery.

``Well?'' he asked.

``Still the same. Wait, for heaven's sake. Karl Ivanich always
says that sleep is more important than anything,'' whispered
Princess Mary with a sigh.

Prince Andrew went up to the child and felt him. He was burning
hot.

``Confound you and your Karl Ivanich!'' He took the glass with
the drops and again went up to the cot.

``Andrew, don't!'' said Princess Mary.

But he scowled at her angrily though also with suffering in his
eyes, and stooped glass in hand over the infant.

``But I wish it,'' he said. ``I beg you---give it him!''

Princess Mary shrugged her shoulders but took the glass
submissively and calling the nurse began giving the medicine. The
child screamed hoarsely. Prince Andrew winced and, clutching his
head, went out and sat down on a sofa in the next room.

He still had all the letters in his hand. Opening them
mechanically he began reading. The old prince, now and then using
abbreviations, wrote in his large elongated hand on blue paper as
follows:

Have just this moment received by special messenger very joyful
news---if it's not false. Bennigsen seems to have obtained a
complete victory over Buonaparte at Eylau. In Petersburg everyone
is rejoicing, and the rewards sent to the army are
innumerable. Though he is a German---I congratulate him! I can't
make out what the commander at Korchevo---a certain
Khandrikov---is up to; till now the additional men and provisions
have not arrived. Gallop off to him at once and say I'll have his
head off if everything is not here in a week. Have received
another letter about the Preussisch-Eylau battle from
Petenka---he took part in it---and it's all true. When
mischief-makers don't meddle even a German beats Buonaparte. He
is said to be fleeing in great disorder. Mind you gallop off to
Korchevo without delay and carry out instructions!

Prince Andrew sighed and broke the seal of another envelope. It
was a closely written letter of two sheets from Bilibin. He
folded it up without reading it and reread his father's letter,
ending with the words: ``Gallop off to Korchevo and carry out
instructions!''

``No, pardon me, I won't go now till the child is better,''
thought he, going to the door and looking into the nursery.

Princess Mary was still standing by the cot, gently rocking the
baby.

``Ah yes, and what else did he say that's unpleasant?'' thought
Prince Andrew, recalling his father's letter. ``Yes, we have
gained a victory over Bonaparte, just when I'm not serving. Yes,
yes, he's always poking fun at me... Ah, well! Let him!'' And he
began reading Bilibin's letter which was written in French. He
read without understanding half of it, read only to forget, if
but for a moment, what he had too long been thinking of so
painfully to the exclusion of all else.

% % % % % % % % % % % % % % % % % % % % % % % % % % % % % % % % %
% % % % % % % % % % % % % % % % % % % % % % % % % % % % % % % % %
% % % % % % % % % % % % % % % % % % % % % % % % % % % % % % % % %
% % % % % % % % % % % % % % % % % % % % % % % % % % % % % % % % %
% % % % % % % % % % % % % % % % % % % % % % % % % % % % % % % % %
% % % % % % % % % % % % % % % % % % % % % % % % % % % % % % % % %
% % % % % % % % % % % % % % % % % % % % % % % % % % % % % % % % %
% % % % % % % % % % % % % % % % % % % % % % % % % % % % % % % % %
% % % % % % % % % % % % % % % % % % % % % % % % % % % % % % % % %
% % % % % % % % % % % % % % % % % % % % % % % % % % % % % % % % %
% % % % % % % % % % % % % % % % % % % % % % % % % % % % % % % % %
% % % % % % % % % % % % % % % % % % % % % % % % % % % % % %

\chapter*{Chapter IX}
\ifaudio     
\marginpar{
\href{http://ia600200.us.archive.org/6/items/war_and_peace_05_0805_librivox/war_and_peace_05_09_tolstoy_64kb.mp3}{Audio}} 
\fi

\lettrine[lines=2, loversize=0.3, lraise=0]{\initfamily B}{ilibin}
was now at army headquarters in a diplomatic capacity,
and though he wrote in French and used French jests and French
idioms, he described the whole campaign with a fearless
self-censure and self-derision genuinely Russian. Bilibin wrote
that the obligation of diplomatic discretion tormented him, and
he was happy to have in Prince Andrew a reliable correspondent to
whom he could pour out the bile he had accumulated at the sight
of all that was being done in the army. The letter was old,
having been written before the battle at Preussisch-Eylau.

``Since the day of our brilliant success at Austerlitz,'' wrote
Bilibin, ``as you know, my dear prince, I never leave
headquarters. I have certainly acquired a taste for war, and it
is just as well for me; what I have seen during these last three
months is incredible.''

``I begin ab ovo. 'The enemy of the human race,' as you know,
attacks the Prussians. The Prussians are our faithful allies who
have only betrayed us three times in three years. We take up
their cause, but it turns out that 'the enemy of the human race'
pays no heed to our fine speeches and in his rude and savage way
throws himself on the Prussians without giving them time to
finish the parade they had begun, and in two twists of the hand
he breaks them to smithereens and installs himself in the palace
at Potsdam.''

``'I most ardently desire,' writes the King of Prussia to
Bonaparte, 'that Your Majesty should be received and treated in
my palace in a manner agreeable to yourself, and in so far as
circumstances allowed, I have hastened to take all steps to that
end. May I have succeeded!' The Prussian generals pride
themselves on being polite to the French and lay down their arms
at the first demand.''

``The head of the garrison at Glogau, with ten thousand men, asks
the King of Prussia what he is to do if he is summoned to
surrender... All this is absolutely true.''

``In short, hoping to settle matters by taking up a warlike
attitude, it turns out that we have landed ourselves in war, and
what is more, in war on our own frontiers, with and for the King
of Prussia. We have everything in perfect order, only one little
thing is lacking, namely, a commander in chief. As it was
considered that the Austerlitz success might have been more
decisive had the commander-in-chief not been so young, all our
octogenarians were reviewed, and of Prozorovski and Kamenski the
latter was preferred. The general comes to us, Suvorov-like, in a
kibitka, and is received with acclamations of joy and triumph.''

``On the 4th, the first courier arrives from Petersburg. The
mails are taken to the field marshal's room, for he likes to do
everything himself. I am called in to help sort the letters and
take those meant for us. The field marshal looks on and waits for
letters addressed to him. We search, but none are to be
found. The field marshal grows impatient and sets to work himself
and finds letters from the Emperor to Count T., Prince V., and
others. Then he bursts into one of his wild furies and rages at
everyone and everything, seizes the letters, opens them, and
reads those from the Emperor addressed to others. 'Ah! So that's
the way they treat me! No confidence in me! Ah, ordered to keep
an eye on me! Very well then! Get along with you!' So he writes
the famous order of the day to General Bennigsen: 'I am wounded
and cannot ride and consequently cannot command the army.  You
have brought your army corps to Pultusk, routed: here it is
exposed, and without fuel or forage, so something must be done,
and, as you yourself reported to Count Buxhowden yesterday, you
must think of retreating to our frontier---which do today.'{}''

``'From all my riding,' he writes to the Emperor, 'I have got a
saddle sore which, coming after all my previous journeys, quite
prevents my riding and commanding so vast an army, so I have
passed on the command to the general next in seniority, Count
Buxhowden, having sent him my whole staff and all that belongs to
it, advising him if there is a lack of bread, to move farther
into the interior of Prussia, for only one day's ration of bread
remains, and in some regiments none at all, as reported by the
division commanders, Ostermann and Sedmoretzki, and all that the
peasants had has been eaten up. I myself will remain in hospital
at Ostrolenka till I recover. In regard to which I humbly submit
my report, with the information that if the army remains in its
present bivouac another fortnight there will not be a healthy man
left in it by spring.''

``'Grant leave to retire to his country seat to an old man who is
already in any case dishonored by being unable to fulfill the
great and glorious task for which he was chosen. I shall await
your most gracious permission here in hospital, that I may not
have to play the part of a secretary rather than commander in the
army. My removal from the army does not produce the slightest
stir---a blind man has left it. There are thousands such as I in
Russia.'\ ''

``The field marshal is angry with the Emperor and he punishes us
all, isn't it logical?''

``This is the first act. Those that follow are naturally
increasingly interesting and entertaining. After the field
marshal's departure it appears that we are within sight of the
enemy and must give battle.  Buxhowden is commander-in-chief by
seniority, but General Bennigsen does not quite see it; more
particularly as it is he and his corps who are within sight of
the enemy and he wishes to profit by the opportunity to fight a
battle 'on his own hand' as the Germans say. He does so. This is
the battle of Pultusk, which is considered a great victory but in
my opinion was nothing of the kind. We civilians, as you know,
have a very bad way of deciding whether a battle was won or
lost. Those who retreat after a battle have lost it is what we
say; and according to that it is we who lost the battle of
Pultusk. In short, we retreat after the battle but send a courier
to Petersburg with news of a victory, and General Bennigsen,
hoping to receive from Petersburg the post of commander in chief
as a reward for his victory, does not give up the command of the
army to General Buxhowden. During this interregnum we begin a
very original and interesting series of maneuvers. Our aim is no
longer, as it should be, to avoid or attack the enemy, but solely
to avoid General Buxhowden who by right of seniority should be
our chief. So energetically do we pursue this aim that after
crossing an unfordable river we burn the bridges to separate
ourselves from our enemy, who at the moment is not Bonaparte but
Buxhowden. General Buxhowden was all but attacked and captured by
a superior enemy force as a result of one of these maneuvers that
enabled us to escape him. Buxhowden pursues us---we scuttle. He
hardly crosses the river to our side before we recross to the
other. At last our enemy, Buxhowden, catches us and attacks. Both
generals are angry, and the result is a challenge on Buxhowden's
part and an epileptic fit on Bennigsen's. But at the critical
moment the courier who carried the news of our victory at Pultusk
to Petersburg returns bringing our appointment as
commander-in-chief, and our first foe, Buxhowden, is vanquished;
we can now turn our thoughts to the second, Bonaparte. But as it
turns out, just at that moment a third enemy rises before
us---namely the Orthodox Russian soldiers, loudly demanding
bread, meat, biscuits, fodder, and whatnot! The stores are empty,
the roads impassable. The Orthodox begin looting, and in a way of
which our last campaign can give you no idea. Half the regiments
form bands and scour the countryside and put everything to fire
and sword.  The inhabitants are totally ruined, the hospitals
overflow with sick, and famine is everywhere. Twice the marauders
even attack our headquarters, and the commander-in-chief has to
ask for a battalion to disperse them. During one of these attacks
they carried off my empty portmanteau and my dressing gown. The
Emperor proposes to give all commanders of divisions the right to
shoot marauders, but I much fear this will oblige one half the
army to shoot the other.''

At first Prince Andrew read with his eyes only, but after a
while, in spite of himself (although he knew how far it was safe
to trust Bilibin), what he had read began to interest him more
and more. When he had read thus far, he crumpled the letter up
and threw it away. It was not what he had read that vexed him,
but the fact that the life out there in which he had now no part
could perturb him. He shut his eyes, rubbed his forehead as if to
rid himself of all interest in what he had read, and listened to
what was passing in the nursery. Suddenly he thought he heard a
strange noise through the door. He was seized with alarm lest
something should have happened to the child while he was reading
the letter. He went on tiptoe to the nursery door and opened it.

Just as he went in he saw that the nurse was hiding something
from him with a scared look and that Princess Mary was no longer
by the cot.

``My dear,'' he heard what seemed to him her despairing whisper
behind him.

As often happens after long sleeplessness and long anxiety, he
was seized by an unreasoning panic---it occurred to him that the
child was dead. All that he saw and heard seemed to confirm this
terror.

``All is over,'' he thought, and a cold sweat broke out on his
forehead.  He went to the cot in confusion, sure that he would
find it empty and that the nurse had been hiding the dead
baby. He drew the curtain aside and for some time his frightened,
restless eyes could not find the baby.  At last he saw him: the
rosy boy had tossed about till he lay across the bed with his
head lower than the pillow, and was smacking his lips in his
sleep and breathing evenly.

Prince Andrew was as glad to find the boy like that, as if he had
already lost him. He bent over him and, as his sister had taught
him, tried with his lips whether the child was still
feverish. The soft forehead was moist. Prince Andrew touched the
head with his hand; even the hair was wet, so profusely had the
child perspired. He was not dead, but evidently the crisis was
over and he was convalescent. Prince Andrew longed to snatch up,
to squeeze, to hold to his heart, this helpless little creature,
but dared not do so. He stood over him, gazing at his head and at
the little arms and legs which showed under the blanket. He heard
a rustle behind him and a shadow appeared under the curtain of
the cot. He did not look round, but still gazing at the infant's
face listened to his regular breathing. The dark shadow was
Princess Mary, who had come up to the cot with noiseless steps,
lifted the curtain, and dropped it again behind her. Prince
Andrew recognized her without looking and held out his hand to
her. She pressed it.

``He has perspired,'' said Prince Andrew.

``I was coming to tell you so.''

The child moved slightly in his sleep, smiled, and rubbed his
forehead against the pillow.

Prince Andrew looked at his sister. In the dim shadow of the
curtain her luminous eyes shone more brightly than usual from the
tears of joy that were in them. She leaned over to her brother
and kissed him, slightly catching the curtain of the cot. Each
made the other a warning gesture and stood still in the dim light
beneath the curtain as if not wishing to leave that seclusion
where they three were shut off from all the world. Prince Andrew
was the first to move away, ruffling his hair against the muslin
of the curtain.

``Yes, this is the one thing left me now,'' he said with a sigh.

% % % % % % % % % % % % % % % % % % % % % % % % % % % % % % % % %
% % % % % % % % % % % % % % % % % % % % % % % % % % % % % % % % %
% % % % % % % % % % % % % % % % % % % % % % % % % % % % % % % % %
% % % % % % % % % % % % % % % % % % % % % % % % % % % % % % % % %
% % % % % % % % % % % % % % % % % % % % % % % % % % % % % % % % %
% % % % % % % % % % % % % % % % % % % % % % % % % % % % % % % % %
% % % % % % % % % % % % % % % % % % % % % % % % % % % % % % % % %
% % % % % % % % % % % % % % % % % % % % % % % % % % % % % % % % %
% % % % % % % % % % % % % % % % % % % % % % % % % % % % % % % % %
% % % % % % % % % % % % % % % % % % % % % % % % % % % % % % % % %
% % % % % % % % % % % % % % % % % % % % % % % % % % % % % % % % %
% % % % % % % % % % % % % % % % % % % % % % % % % % % % % %

\chapter*{Chapter X}
\ifaudio     
\marginpar{
\href{http://ia600200.us.archive.org/6/items/war_and_peace_05_0805_librivox/war_and_peace_05_10_tolstoy_64kb.mp3}{Audio}} 
\fi

\lettrine[lines=2, loversize=0.3, lraise=0]{\initfamily S}{oon}
after his admission to the masonic Brotherhood, Pierre went
to the Kiev province, where he had the greatest number of serfs,
taking with him full directions which he had written down for his
own guidance as to what he should do on his estates.

When he reached Kiev he sent for all his stewards to the head
office and explained to them his intentions and wishes. He told
them that steps would be taken immediately to free his
serfs---and that till then they were not to be overburdened with
labor, women while nursing their babies were not to be sent to
work, assistance was to be given to the serfs, punishments were
to be admonitory and not corporal, and hospitals, asylums, and
schools were to be established on all the estates. Some of the
stewards (there were semiliterate foremen among them) listened
with alarm, supposing these words to mean that the young count
was displeased with their management and embezzlement of money,
some after their first fright were amused by Pierre's lisp and
the new words they had not heard before, others simply enjoyed
hearing how the master talked, while the cleverest among them,
including the chief steward, understood from this speech how they
could best handle the master for their own ends.

The chief steward expressed great sympathy with Pierre's
intentions, but remarked that besides these changes it would be
necessary to go into the general state of affairs which was far
from satisfactory.

Despite Count Bezukhov's enormous wealth, since he had come into
an income which was said to amount to five hundred thousand
rubles a year, Pierre felt himself far poorer than when his
father had made him an allowance of ten thousand rubles. He had a
dim perception of the following budget:

About 80,000 went in payments on all the estates to the Land
Bank, about 30,000 went for the upkeep of the estate near Moscow,
the town house, and the allowance to the three princesses; about
15,000 was given in pensions and the same amount for asylums;
150,000 alimony was sent to the countess; about 70,000 went for
interest on debts. The building of a new church, previously
begun, had cost about 10,000 in each of the last two years, and
he did not know how the rest, about 100,000 rubles, was spent,
and almost every year he was obliged to borrow. Besides this the
chief steward wrote every year telling him of fires and bad
harvests, or of the necessity of rebuilding factories and
workshops. So the first task Pierre had to face was one for which
he had very little aptitude or inclination---practical business.

He discussed estate affairs every day with his chief steward. But
he felt that this did not forward matters at all. He felt that
these consultations were detached from real affairs and did not
link up with them or make them move. On the one hand, the chief
steward put the state of things to him in the very worst light,
pointing out the necessity of paying off the debts and
undertaking new activities with serf labor, to which Pierre did
not agree. On the other hand, Pierre demanded that steps should
be taken to liberate the serfs, which the steward met by showing
the necessity of first paying off the loans from the Land Bank,
and the consequent impossibility of a speedy emancipation.

The steward did not say it was quite impossible, but suggested
selling the forests in the province of Kostroma, the land lower
down the river, and the Crimean estate, in order to make it
possible: all of which operations according to him were connected
with such complicated measures---the removal of injunctions,
petitions, permits, and so on---that Pierre became quite
bewildered and only replied:

``Yes, yes, do so.''

Pierre had none of the practical persistence that would have
enabled him to attend to the business himself and so he disliked
it and only tried to pretend to the steward that he was attending
to it. The steward for his part tried to pretend to the count
that he considered these consultations very valuable for the
proprietor and troublesome to himself.

In Kiev Pierre found some people he knew, and strangers hastened
to make his acquaintance and joyfully welcomed the rich newcomer,
the largest landowner of the province. Temptations to Pierre's
greatest weakness---the one to which he had confessed when
admitted to the Lodge---were so strong that he could not resist
them. Again whole days, weeks, and months of his life passed in
as great a rush and were as much occupied with evening parties,
dinners, lunches, and balls, giving him no time for reflection,
as in Petersburg. Instead of the new life he had hoped to lead he
still lived the old life, only in new surroundings.

Of the three precepts of Freemasonry Pierre realized that he did
not fulfill the one which enjoined every Mason to set an example
of moral life, and that of the seven virtues he lacked
two---morality and the love of death. He consoled himself with
the thought that he fulfilled another of the precepts---that of
reforming the human race---and had other virtues---love of his
neighbor, and especially generosity.

In the spring of 1807 he decided to return to Petersburg. On the
way he intended to visit all his estates and see for himself how
far his orders had been carried out and in what state were the
serfs whom God had entrusted to his care and whom he intended to
benefit.

The chief steward, who considered the young count's attempts
almost insane---unprofitable to himself, to the count, and to the
serfs---made some concessions. Continuing to represent the
liberation of the serfs as impracticable, he arranged for the
erection of large buildings---schools, hospitals, and
asylums---on all the estates before the master arrived.
Everywhere preparations were made not for ceremonious welcomes
(which he knew Pierre would not like), but for just such
gratefully religious ones, with offerings of icons and the bread
and salt of hospitality, as, according to his understanding of
his master, would touch and delude him.

The southern spring, the comfortable rapid traveling in a Vienna
carriage, and the solitude of the road, all had a gladdening
effect on Pierre. The estates he had not before visited were each
more picturesque than the other; the serfs everywhere seemed
thriving and touchingly grateful for the benefits conferred on
them. Everywhere were receptions, which though they embarrassed
Pierre awakened a joyful feeling in the depth of his heart. In
one place the peasants presented him with bread and salt and an
icon of Saint Peter and Saint Paul, asking permission, as a mark
of their gratitude for the benefits he had conferred on them, to
build a new chantry to the church at their own expense in honor
of Peter and Paul, his patron saints. In another place the women
with infants in arms met him to thank him for releasing them from
hard work.  On a third estate the priest, bearing a cross, came
to meet him surrounded by children whom, by the count's
generosity, he was instructing in reading, writing, and
religion. On all his estates Pierre saw with his own eyes brick
buildings erected or in course of erection, all on one plan, for
hospitals, schools, and almshouses, which were soon to be
opened. Everywhere he saw the stewards' accounts, according to
which the serfs' manorial labor had been diminished, and heard
the touching thanks of deputations of serfs in their full-skirted
blue coats.

What Pierre did not know was that the place where they presented
him with bread and salt and wished to build a chantry in honor of
Peter and Paul was a market village where a fair was held on
St. Peter's day, and that the richest peasants (who formed the
deputation) had begun the chantry long before, but that nine
tenths of the peasants in that villages were in a state of the
greatest poverty. He did not know that since the nursing mothers
were no longer sent to work on his land, they did still harder
work on their own land. He did not know that the priest who met
him with the cross oppressed the peasants by his exactions, and
that the pupils' parents wept at having to let him take their
children and secured their release by heavy payments. He did not
know that the brick buildings, built to plan, were being built by
serfs whose manorial labor was thus increased, though lessened on
paper. He did not know that where the steward had shown him in
the accounts that the serfs' payments had been diminished by a
third, their obligatory manorial work had been increased by a
half. And so Pierre was delighted with his visit to his estates
and quite recovered the philanthropic mood in which he had left
Petersburg, and wrote enthusiastic letters to his
\emph{brother-instructor} as he called the Grand Master.

``How easy it is, how little effort it needs, to do so much
good,'' thought Pierre, ``and how little attention we pay to
it!''

He was pleased at the gratitude he received, but felt abashed at
receiving it. This gratitude reminded him of how much more he
might do for these simple, kindly people.

The chief steward, a very stupid but cunning man who saw
perfectly through the naive and intelligent count and played with
him as with a toy, seeing the effect these prearranged receptions
had on Pierre, pressed him still harder with proofs of the
impossibility and above all the uselessness of freeing the serfs,
who were quite happy as it was.

Pierre in his secret soul agreed with the steward that it would
be difficult to imagine happier people, and that God only knew
what would happen to them when they were free, but he insisted,
though reluctantly, on what he thought right. The steward
promised to do all in his power to carry out the count's wishes,
seeing clearly that not only would the count never be able to
find out whether all measures had been taken for the sale of the
land and forests and to release them from the Land Bank, but
would probably never even inquire and would never know that the
newly erected buildings were standing empty and that the serfs
continued to give in money and work all that other people's serfs
gave---that is to say, all that could be got out of them.

% % % % % % % % % % % % % % % % % % % % % % % % % % % % % % % % %
% % % % % % % % % % % % % % % % % % % % % % % % % % % % % % % % %
% % % % % % % % % % % % % % % % % % % % % % % % % % % % % % % % %
% % % % % % % % % % % % % % % % % % % % % % % % % % % % % % % % %
% % % % % % % % % % % % % % % % % % % % % % % % % % % % % % % % %
% % % % % % % % % % % % % % % % % % % % % % % % % % % % % % % % %
% % % % % % % % % % % % % % % % % % % % % % % % % % % % % % % % %
% % % % % % % % % % % % % % % % % % % % % % % % % % % % % % % % %
% % % % % % % % % % % % % % % % % % % % % % % % % % % % % % % % %
% % % % % % % % % % % % % % % % % % % % % % % % % % % % % % % % %
% % % % % % % % % % % % % % % % % % % % % % % % % % % % % % % % %
% % % % % % % % % % % % % % % % % % % % % % % % % % % % % %

\chapter*{Chapter XI}
\ifaudio     
\marginpar{
\href{http://ia600200.us.archive.org/6/items/war_and_peace_05_0805_librivox/war_and_peace_05_11_tolstoy_64kb.mp3}{Audio}} 
\fi

\lettrine[lines=2, loversize=0.3, lraise=0]{\initfamily R}{eturning}
from his journey through South Russia in the happiest
state of mind, Pierre carried out an intention he had long had of
visiting his friend Bolkonski, whom he had not seen for two
years.

Bogucharovo lay in a flat uninteresting part of the country among
fields and forests of fir and birch, which were partly cut
down. The house lay behind a newly dug pond filled with water to
the brink and with banks still bare of grass. It was at the end
of a village that stretched along the highroad in the midst of a
young copse in which were a few fir trees.

The homestead consisted of a threshing floor, outhouses, stables,
a bathhouse, a lodge, and a large brick house with semicircular
facade still in course of construction. Round the house was a
garden newly laid out. The fences and gates were new and solid;
two fire pumps and a water cart, painted green, stood in a shed;
the paths were straight, the bridges were strong and had
handrails. Everything bore an impress of tidiness and good
management. Some domestic serfs Pierre met, in reply to inquiries
as to where the prince lived, pointed out a small newly built
lodge close to the pond. Anton, a man who had looked after Prince
Andrew in his boyhood, helped Pierre out of his carriage, said
that the prince was at home, and showed him into a clean little
anteroom.

Pierre was struck by the modesty of the small though clean house
after the brilliant surroundings in which he had last met his
friend in Petersburg.

He quickly entered the small reception room with its
still-unplastered wooden walls redolent of pine, and would have
gone farther, but Anton ran ahead on tiptoe and knocked at a
door.

``Well, what is it?'' came a sharp, unpleasant voice.

``A visitor,'' answered Anton.

``Ask him to wait,'' and the sound was heard of a chair being
pushed back.

Pierre went with rapid steps to the door and suddenly came face
to face with Prince Andrew, who came out frowning and looking
old. Pierre embraced him and lifting his spectacles kissed his
friend on the cheek and looked at him closely.

``Well, I did not expect you, I am very glad,'' said Prince
Andrew.

Pierre said nothing; he looked fixedly at his friend with
surprise. He was struck by the change in him. His words were
kindly and there was a smile on his lips and face, but his eyes
were dull and lifeless and in spite of his evident wish to do so
he could not give them a joyous and glad sparkle. Prince Andrew
had grown thinner, paler, and more manly-looking, but what amazed
and estranged Pierre till he got used to it were his inertia and
a wrinkle on his brow indicating prolonged concentration on some
one thought.

As is usually the case with people meeting after a prolonged
separation, it was long before their conversation could settle on
anything. They put questions and gave brief replies about things
they knew ought to be talked over at length. At last the
conversation gradually settled on some of the topics at first
lightly touched on: their past life, plans for the future,
Pierre's journeys and occupations, the war, and so on.  The
preoccupation and despondency which Pierre had noticed in his
friend's look was now still more clearly expressed in the smile
with which he listened to Pierre, especially when he spoke with
joyful animation of the past or the future. It was as if Prince
Andrew would have liked to sympathize with what Pierre was
saying, but could not. The latter began to feel that it was in
bad taste to speak of his enthusiasms, dreams, and hopes of
happiness or goodness, in Prince Andrew's presence. He was
ashamed to express his new masonic views, which had been
particularly revived and strengthened by his late tour.  He
checked himself, fearing to seem naive, yet he felt an
irresistible desire to show his friend as soon as possible that
he was now a quite different, and better, Pierre than he had been
in Petersburg.

``I can't tell you how much I have lived through since then. I
hardly know myself again.''

``Yes, we have altered much, very much, since then,'' said Prince
Andrew.

``Well, and you? What are your plans?''

``Plans!'' repeated Prince Andrew ironically. ``My plans?'' he
said, as if astonished at the word. ``Well, you see, I'm
building. I mean to settle here altogether next year...''

Pierre looked silently and searchingly into Prince Andrew's face,
which had grown much older.

``No, I meant to ask...'' Pierre began, but Prince Andrew
interrupted him.

``But why talk of me?... Talk to me, yes, tell me about your
travels and all you have been doing on your estates.''

Pierre began describing what he had done on his estates, trying
as far as possible to conceal his own part in the improvements
that had been made. Prince Andrew several times prompted Pierre's
story of what he had been doing, as though it were all an
old-time story, and he listened not only without interest but
even as if ashamed of what Pierre was telling him.

Pierre felt uncomfortable and even depressed in his friend's
company and at last became silent.

``I'll tell you what, my dear fellow,'' said Prince Andrew, who
evidently also felt depressed and constrained with his visitor,
``I am only bivouacking here and have just come to look round. I
am going back to my sister today. I will introduce you to
her. But of course you know her already,'' he said, evidently
trying to entertain a visitor with whom he now found nothing in
common. ``We will go after dinner. And would you now like to look
round my place?''

They went out and walked about till dinnertime, talking of the
political news and common acquaintances like people who do not
know each other intimately. Prince Andrew spoke with some
animation and interest only of the new homestead he was
constructing and its buildings, but even here, while on the
scaffolding, in the midst of a talk explaining the future
arrangements of the house, he interrupted himself:

``However, this is not at all interesting. Let us have dinner,
and then we'll set off.''

At dinner, conversation turned on Pierre's marriage.

``I was very much surprised when I heard of it,'' said Prince
Andrew.

Pierre blushed, as he always did when it was mentioned, and said
hurriedly: ``I will tell you some time how it all happened. But
you know it is all over, and forever.''

``Forever?'' said Prince Andrew. ``Nothing's forever.''

``But you know how it all ended, don't you? You heard of the
duel?''

``And so you had to go through that too!''

``One thing I thank God for is that I did not kill that man,''
said Pierre.

``Why so?'' asked Prince Andrew. ``To kill a vicious dog is a
very good thing really.''

``No, to kill a man is bad---wrong.''

``Why is it wrong?'' urged Prince Andrew. ``It is not given to
man to know what is right and what is wrong. Men always did and
always will err, and in nothing more than in what they consider
right and wrong.''

``What does harm to another is wrong,'' said Pierre, feeling with
pleasure that for the first time since his arrival Prince Andrew
was roused, had begun to talk, and wanted to express what had
brought him to his present state.

``And who has told you what is bad for another man?'' he asked.

``Bad! Bad!'' exclaimed Pierre. ``We all know what is bad for
ourselves.''

``Yes, we know that, but the harm I am conscious of in myself is
something I cannot inflict on others,'' said Prince Andrew,
growing more and more animated and evidently wishing to express
his new outlook to Pierre. He spoke in French. ``I only know two
very real evils in life: remorse and illness. The only good is
the absence of those evils. To live for myself avoiding those two
evils is my whole philosophy now.''

``And love of one's neighbor, and self-sacrifice?'' began
Pierre. ``No, I can't agree with you! To live only so as not to
do evil and not to have to repent is not enough. I lived like
that, I lived for myself and ruined my life. And only now when I
am living, or at least trying'' (Pierre's modesty made him
correct himself) ``to live for others, only now have I understood
all the happiness of life. No, I shall not agree with you, and
you do not really believe what you are saying.'' Prince Andrew
looked silently at Pierre with an ironic smile.

``When you see my sister, Princess Mary, you'll get on with
her,'' he said. ``Perhaps you are right for yourself,'' he added
after a short pause, ``but everyone lives in his own way. You
lived for yourself and say you nearly ruined your life and only
found happiness when you began living for others. I experienced
just the reverse. I lived for glory.---And after all what is
glory? The same love of others, a desire to do something for
them, a desire for their approval.---So I lived for others, and
not almost, but quite, ruined my life. And I have become calmer
since I began to live only for myself.''

``But what do you mean by living only for yourself?'' asked
Pierre, growing excited. ``What about your son, your sister, and
your father?''

``But that's just the same as myself---they are not others,''
explained Prince Andrew. ``The others, one's neighbors, le
prochain, as you and Princess Mary call it, are the chief source
of all error and evil. Le prochain---your Kiev peasants to whom
you want to do good.''

And he looked at Pierre with a mocking, challenging
expression. He evidently wished to draw him on.

``You are joking,'' replied Pierre, growing more and more
excited. ``What error or evil can there be in my wishing to do
good, and even doing a little---though I did very little and did
it very badly? What evil can there be in it if unfortunate
people, our serfs, people like ourselves, were growing up and
dying with no idea of God and truth beyond ceremonies and
meaningless prayers and are now instructed in a comforting belief
in future life, retribution, recompense, and consolation? What
evil and error are there in it, if people were dying of disease
without help while material assistance could so easily be
rendered, and I supplied them with a doctor, a hospital, and an
asylum for the aged? And is it not a palpable, unquestionable
good if a peasant, or a woman with a baby, has no rest day or
night and I give them rest and leisure?'' said Pierre, hurrying
and lisping. ``And I have done that though badly and to a small
extent; but I have done something toward it and you cannot
persuade me that it was not a good action, and more than that,
you can't make me believe that you do not think so yourself. And
the main thing is,'' he continued, ``that I know, and know for
certain, that the enjoyment of doing this good is the only sure
happiness in life.''

``Yes, if you put it like that it's quite a different matter,''
said Prince Andrew. ``I build a house and lay out a garden, and
you build hospitals. The one and the other may serve as a
pastime. But what's right and what's good must be judged by one
who knows all, but not by us. Well, you want an argument,'' he
added, ``come on then.''

They rose from the table and sat down in the entrance porch which
served as a veranda.

``Come, let's argue then,'' said Prince Andrew, ``You talk of
schools,'' he went on, crooking a finger, ``education and so
forth; that is, you want to raise him'' (pointing to a peasant
who passed by them taking off his cap) ``from his animal
condition and awaken in him spiritual needs, while it seems to me
that animal happiness is the only happiness possible, and that is
just what you want to deprive him of. I envy him, but you want to
make him what I am, without giving him my means. Then you say,
'lighten his toil.' But as I see it, physical labor is as
essential to him, as much a condition of his existence, as mental
activity is to you or me. You can't help thinking. I go to bed
after two in the morning, thoughts come and I can't sleep but
toss about till dawn, because I think and can't help thinking,
just as he can't help plowing and mowing; if he didn't, he would
go to the drink shop or fall ill. Just as I could not stand his
terrible physical labor but should die of it in a week, so he
could not stand my physical idleness, but would grow fat and
die. The third thing---what else was it you talked about?'' and
Prince Andrew crooked a third finger. ``Ah, yes, hospitals,
medicine. He has a fit, he is dying, and you come and bleed him
and patch him up. He will drag about as a cripple, a burden to
everybody, for another ten years. It would be far easier and
simpler for him to die. Others are being born and there are
plenty of them as it is. It would be different if you grudged
losing a laborer---that's how I regard him---but you want to cure
him from love of him. And he does not want that. And besides,
what a notion that medicine ever cured anyone! Killed them,
yes!'' said he, frowning angrily and turning away from Pierre.

Prince Andrew expressed his ideas so clearly and distinctly that
it was evident he had reflected on this subject more than once,
and he spoke readily and rapidly like a man who has not talked
for a long time. His glance became more animated as his
conclusions became more hopeless.

``Oh, that is dreadful, dreadful!'' said Pierre. ``I don't
understand how one can live with such ideas. I had such moments
myself not long ago, in Moscow and when traveling, but at such
times I collapsed so that I don't live at all---everything seems
hateful to me... myself most of all. Then I don't eat, don't
wash... and how is it with you?...''

``Why not wash? That is not cleanly,'' said Prince Andrew; ``on
the contrary one must try to make one's life as pleasant as
possible. I'm alive, that is not my fault, so I must live out my
life as best I can without hurting others.''

``But with such ideas what motive have you for living? One would
sit without moving, undertaking nothing...''

``Life as it is leaves one no peace. I should be thankful to do
nothing, but here on the one hand the local nobility have done me
the honor to choose me to be their marshal; it was all I could do
to get out of it.  They could not understand that I have not the
necessary qualifications for it---the kind of good-natured, fussy
shallowness necessary for the position. Then there's this house,
which must be built in order to have a nook of one's own in which
to be quiet. And now there's this recruiting.''

``Why aren't you serving in the army?''

``After Austerlitz!'' said Prince Andrew gloomily. ``No, thank
you very much! I have promised myself not to serve again in the
active Russian army. And I won't---not even if Bonaparte were
here at Smolensk threatening Bald Hills---even then I wouldn't
serve in the Russian army!  Well, as I was saying,'' he
continued, recovering his composure, ``now there's this
recruiting. My father is chief in command of the Third District,
and my only way of avoiding active service is to serve under
him.''

``Then you are serving?''

``I am.''

He paused a little while.

``And why do you serve?''

``Why, for this reason! My father is one of the most remarkable
men of his time. But he is growing old, and though not exactly
cruel he has too energetic a character. He is so accustomed to
unlimited power that he is terrible, and now he has this
authority of a commander-in-chief of the recruiting, granted by
the Emperor. If I had been two hours late a fortnight ago he
would have had a paymaster's clerk at Yukhnovna hanged,'' said
Prince Andrew with a smile. ``So I am serving because I alone
have any influence with my father, and now and then can save him
from actions which would torment him afterwards.''

``Well, there you see!''

``Yes, but it is not as you imagine,'' Prince Andrew
continued. ``I did not, and do not, in the least care about that
scoundrel of a clerk who had stolen some boots from the recruits;
I should even have been very glad to see him hanged, but I was
sorry for my father---that again is for myself.''

Prince Andrew grew more and more animated. His eyes glittered
feverishly while he tried to prove to Pierre that in his actions
there was no desire to do good to his neighbor.

``There now, you wish to liberate your serfs,'' he continued;
``that is a very good thing, but not for you---I don't suppose
you ever had anyone flogged or sent to Siberia---and still less
for your serfs. If they are beaten, flogged, or sent to Siberia,
I don't suppose they are any the worse off. In Siberia they lead
the same animal life, and the stripes on their bodies heal, and
they are happy as before. But it is a good thing for proprietors
who perish morally, bring remorse upon themselves, stifle this
remorse and grow callous, as a result of being able to inflict
punishments justly and unjustly. It is those people I pity, and
for their sake I should like to liberate the serfs. You may not
have seen, but I have seen, how good men brought up in those
traditions of unlimited power, in time when they grow more
irritable, become cruel and harsh, are conscious of it, but
cannot restrain themselves and grow more and more miserable.''

Prince Andrew spoke so earnestly that Pierre could not help
thinking that these thoughts had been suggested to Prince Andrew
by his father's case.

He did not reply.

``So that's what I'm sorry for---human dignity, peace of mind,
purity, and not the serfs' backs and foreheads, which, beat and
shave as you may, always remain the same backs and foreheads.''

``No, no! A thousand times no! I shall never agree with you,''
said Pierre.

% % % % % % % % % % % % % % % % % % % % % % % % % % % % % % % % %
% % % % % % % % % % % % % % % % % % % % % % % % % % % % % % % % %
% % % % % % % % % % % % % % % % % % % % % % % % % % % % % % % % %
% % % % % % % % % % % % % % % % % % % % % % % % % % % % % % % % %
% % % % % % % % % % % % % % % % % % % % % % % % % % % % % % % % %
% % % % % % % % % % % % % % % % % % % % % % % % % % % % % % % % %
% % % % % % % % % % % % % % % % % % % % % % % % % % % % % % % % %
% % % % % % % % % % % % % % % % % % % % % % % % % % % % % % % % %
% % % % % % % % % % % % % % % % % % % % % % % % % % % % % % % % %
% % % % % % % % % % % % % % % % % % % % % % % % % % % % % % % % %
% % % % % % % % % % % % % % % % % % % % % % % % % % % % % % % % %
% % % % % % % % % % % % % % % % % % % % % % % % % % % % % %

\chapter*{Chapter XII}
\ifaudio     
\marginpar{
\href{http://ia600200.us.archive.org/6/items/war_and_peace_05_0805_librivox/war_and_peace_05_12_tolstoy_64kb.mp3}{Audio}} 
\fi

\lettrine[lines=2, loversize=0.3, lraise=0]{\initfamily I}{n} the evening Andrew and Pierre got into the open carriage and
drove to Bald Hills. Prince Andrew, glancing at Pierre, broke the
silence now and then with remarks which showed that he was in a
good temper.

Pointing to the fields, he spoke of the improvements he was
making in his husbandry.

Pierre remained gloomily silent, answering in monosyllables and
apparently immersed in his own thoughts.

He was thinking that Prince Andrew was unhappy, had gone astray,
did not see the true light, and that he, Pierre, ought to aid,
enlighten, and raise him. But as soon as he thought of what he
should say, he felt that Prince Andrew with one word, one
argument, would upset all his teaching, and he shrank from
beginning, afraid of exposing to possible ridicule what to him
was precious and sacred.

``No, but why do you think so?'' Pierre suddenly began, lowering
his head and looking like a bull about to charge, ``why do you
think so? You should not think so.''

``Think? What about?'' asked Prince Andrew with surprise.

``About life, about man's destiny. It can't be so. I myself
thought like that, and do you know what saved me? Freemasonry!
No, don't smile.  Freemasonry is not a religious ceremonial sect,
as I thought it was: Freemasonry is the best expression of the
best, the eternal, aspects of humanity.''

And he began to explain Freemasonry as he understood it to Prince
Andrew. He said that Freemasonry is the teaching of Christianity
freed from the bonds of State and Church, a teaching of equality,
brotherhood, and love.

``Only our holy brotherhood has the real meaning of life, all the
rest is a dream,'' said Pierre. ``Understand, my dear fellow,
that outside this union all is filled with deceit and falsehood
and I agree with you that nothing is left for an intelligent and
good man but to live out his life, like you, merely trying not to
harm others. But make our fundamental convictions your own, join
our brotherhood, give yourself up to us, let yourself be guided,
and you will at once feel yourself, as I have felt myself, a part
of that vast invisible chain the beginning of which is hidden in
heaven,'' said Pierre.

Prince Andrew, looking straight in front of him, listened in
silence to Pierre's words. More than once, when the noise of the
wheels prevented his catching what Pierre said, he asked him to
repeat it, and by the peculiar glow that came into Prince
Andrew's eyes and by his silence, Pierre saw that his words were
not in vain and that Prince Andrew would not interrupt him or
laugh at what he said.

They reached a river that had overflowed its banks and which they
had to cross by ferry. While the carriage and horses were being
placed on it, they also stepped on the raft.

Prince Andrew, leaning his arms on the raft railing, gazed
silently at the flooding waters glittering in the setting sun.

``Well, what do you think about it?'' Pierre asked. ``Why are you
silent?''

``What do I think about it? I am listening to you. It's all very
well...  You say: join our brotherhood and we will show you the
aim of life, the destiny of man, and the laws which govern the
world. But who are we?  Men. How is it you know everything? Why
do I alone not see what you see?  You see a reign of goodness and
truth on earth, but I don't see it.''

Pierre interrupted him.

``Do you believe in a future life?'' he asked.

``A future life?'' Prince Andrew repeated, but Pierre, giving him
no time to reply, took the repetition for a denial, the more
readily as he knew Prince Andrew's former atheistic convictions.

``You say you can't see a reign of goodness and truth on
earth. Nor could I, and it cannot be seen if one looks on our
life here as the end of everything. On earth, here on this
earth'' (Pierre pointed to the fields), ``there is no truth, all
is false and evil; but in the universe, in the whole universe
there is a kingdom of truth, and we who are now the children of
earth are---eternally---children of the whole universe.  Don't I
feel in my soul that I am part of this vast harmonious whole?
Don't I feel that I form one link, one step, between the lower
and higher beings, in this vast harmonious multitude of beings in
whom the Deity---the Supreme Power if you prefer the term---is
manifest? If I see, clearly see, that ladder leading from plant
to man, why should I suppose it breaks off at me and does not go
farther and farther? I feel that I cannot vanish, since nothing
vanishes in this world, but that I shall always exist and always
have existed. I feel that beyond me and above me there are
spirits, and that in this world there is truth.''

``Yes, that is Herder's theory,'' said Prince Andrew, ``but it is
not that which can convince me, dear friend---life and death are
what convince.  What convinces is when one sees a being dear to
one, bound up with one's own life, before whom one was to blame
and had hoped to make it right'' (Prince Andrew's voice trembled
and he turned away), ``and suddenly that being is seized with
pain, suffers, and ceases to exist... Why? It cannot be that
there is no answer. And I believe there is... That's what
convinces, that is what has convinced me,'' said Prince Andrew.

``Yes, yes, of course,'' said Pierre, ``isn't that what I'm
saying?''

``No. All I say is that it is not argument that convinces me of
the necessity of a future life, but this: when you go hand in
hand with someone and all at once that person vanishes there,
into nowhere, and you yourself are left facing that abyss, and
look in. And I have looked in...''

``Well, that's it then! You know that there is a there and there
is a Someone? There is the future life. The Someone is---God.''

Prince Andrew did not reply. The carriage and horses had long
since been taken off, onto the farther bank, and reharnessed. The
sun had sunk half below the horizon and an evening frost was
starring the puddles near the ferry, but Pierre and Andrew, to
the astonishment of the footmen, coachmen, and ferrymen, still
stood on the raft and talked.

``If there is a God and future life, there is truth and good, and
man's highest happiness consists in striving to attain them. We
must live, we must love, and we must believe that we live not
only today on this scrap of earth, but have lived and shall live
forever, there, in the Whole,'' said Pierre, and he pointed to
the sky.

Prince Andrew stood leaning on the railing of the raft listening
to Pierre, and he gazed with his eyes fixed on the red reflection
of the sun gleaming on the blue waters. There was perfect
stillness. Pierre became silent. The raft had long since stopped
and only the waves of the current beat softly against it
below. Prince Andrew felt as if the sound of the waves kept up a
refrain to Pierre's words, whispering:

``It is true, believe it.''

He sighed, and glanced with a radiant, childlike, tender look at
Pierre's face, flushed and rapturous, but yet shy before his
superior friend.

``Yes, if it only were so!'' said Prince Andrew. ``However, it is
time to get on,'' he added, and, stepping off the raft, he looked
up at the sky to which Pierre had pointed, and for the first time
since Austerlitz saw that high, everlasting sky he had seen while
lying on that battlefield; and something that had long been
slumbering, something that was best within him, suddenly awoke,
joyful and youthful, in his soul. It vanished as soon as he
returned to the customary conditions of his life, but he knew
that this feeling which he did not know how to develop existed
within him. His meeting with Pierre formed an epoch in Prince
Andrew's life. Though outwardly he continued to live in the same
old way, inwardly he began a new life.

% % % % % % % % % % % % % % % % % % % % % % % % % % % % % % % % %
% % % % % % % % % % % % % % % % % % % % % % % % % % % % % % % % %
% % % % % % % % % % % % % % % % % % % % % % % % % % % % % % % % %
% % % % % % % % % % % % % % % % % % % % % % % % % % % % % % % % %
% % % % % % % % % % % % % % % % % % % % % % % % % % % % % % % % %
% % % % % % % % % % % % % % % % % % % % % % % % % % % % % % % % %
% % % % % % % % % % % % % % % % % % % % % % % % % % % % % % % % %
% % % % % % % % % % % % % % % % % % % % % % % % % % % % % % % % %
% % % % % % % % % % % % % % % % % % % % % % % % % % % % % % % % %
% % % % % % % % % % % % % % % % % % % % % % % % % % % % % % % % %
% % % % % % % % % % % % % % % % % % % % % % % % % % % % % % % % %
% % % % % % % % % % % % % % % % % % % % % % % % % % % % % %

\chapter*{Chapter XIII}
\ifaudio     
\marginpar{
\href{http://ia600200.us.archive.org/6/items/war_and_peace_05_0805_librivox/war_and_peace_05_13_tolstoy_64kb.mp3}{Audio}} 
\fi

\lettrine[lines=2, loversize=0.3, lraise=0]{\initfamily I}{t}
was getting dusk when Prince Andrew and Pierre drove up to the
front entrance of the house at Bald Hills. As they approached the
house, Prince Andrew with a smile drew Pierre's attention to a
commotion going on at the back porch. A woman, bent with age,
with a wallet on her back, and a short, long-haired, young man in
a black garment had rushed back to the gate on seeing the
carriage driving up. Two women ran out after them, and all four,
looking round at the carriage, ran in dismay up the steps of the
back porch.

``Those are Mary's 'God's folk,'{}'' said Prince Andrew. ``They
have mistaken us for my father. This is the one matter in which
she disobeys him. He orders these pilgrims to be driven away, but
she receives them.''

``But what are 'God's folk'?'' asked Pierre.

Prince Andrew had no time to answer. The servants came out to
meet them, and he asked where the old prince was and whether he
was expected back soon.

The old prince had gone to the town and was expected back any
minute.

Prince Andrew led Pierre to his own apartments, which were always
kept in perfect order and readiness for him in his father's
house; he himself went to the nursery.

``Let us go and see my sister,'' he said to Pierre when he
returned. ``I have not found her yet, she is hiding now, sitting
with her 'God's folk.' It will serve her right, she will be
confused, but you will see her 'God's folk.' It's really very
curious.''

``What are 'God's folk'?'' asked Pierre.

``Come, and you'll see for yourself.''

Princess Mary really was disconcerted and red patches came on her
face when they went in. In her snug room, with lamps burning
before the icon stand, a young lad with a long nose and long
hair, wearing a monk's cassock, sat on the sofa beside her,
behind a samovar. Near them, in an armchair, sat a thin,
shriveled, old woman, with a meek expression on her childlike
face.

``Andrew, why didn't you warn me?'' said the princess, with mild
reproach, as she stood before her pilgrims like a hen before her
chickens.

``Charmee de vous voir. Je suis tres contente de vous
voir,''\footnote{``Delighted to see you. I am very glad to see
you.''} she said to Pierre as he kissed her hand. She had known
him as a child, and now his friendship with Andrew, his
misfortune with his wife, and above all his kindly, simple face
disposed her favorably toward him. She looked at him with her
beautiful radiant eyes and seemed to say, ``I like you very much,
but please don't laugh at my people.'' After exchanging the first
greetings, they sat down.

``Ah, and Ivanushka is here too!'' said Prince Andrew, glancing
with a smile at the young pilgrim.

``Andrew!'' said Princess Mary, imploringly. ``Il faut que vous
sachiez que c'est une femme,''\footnote{``You must know that this
is a woman.''} said Prince Andrew to Pierre.``Andrew, au nom de
Dieu!''\footnote{``For heaven's sake.''} Princess Mary repeated.

It was evident that Prince Andrew's ironical tone toward the
pilgrims and Princess Mary's helpless attempts to protect them
were their customary long-established relations on the matter.

``Mais, ma bonne amie,'' said Prince Andrew, ``vous devriez au
contraire m'??tre reconnaissante de ce que j'explique a Pierre
votre intimit?? avec ce jeune homme.''\footnote{``But, my dear,
you ought on the contrary to be grateful to me for explaining to
Pierre your intimacy with this young man.''}

``Really?'' said Pierre, gazing over his spectacles with
curiosity and seriousness (for which Princess Mary was specially
grateful to him) into Ivanushka's face, who, seeing that she was
being spoken about, looked round at them all with crafty eyes.

Princess Mary's embarrassment on her people's account was quite
unnecessary. They were not in the least abashed. The old woman,
lowering her eyes but casting side glances at the newcomers, had
turned her cup upside down and placed a nibbled bit of sugar
beside it, and sat quietly in her armchair, though hoping to be
offered another cup of tea.  Ivanushka, sipping out of her
saucer, looked with sly womanish eyes from under her brows at the
young men.

``Where have you been? To Kiev?'' Prince Andrew asked the old
woman.

``I have, good sir,'' she answered garrulously. ``Just at
Christmastime I was deemed worthy to partake of the holy and
heavenly sacrament at the shrine of the saint. And now I'm from
Kolyazin, master, where a great and wonderful blessing has been
revealed.''

``And was Ivanushka with you?''

``I go by myself, benefactor,'' said Ivanushka, trying to speak
in a bass voice. ``I only came across Pelageya in Yukhnovo...''

Pelageya interrupted her companion; she evidently wished to tell
what she had seen.

``In Kolyazin, master, a wonderful blessing has been revealed.''

``What is it? Some new relics?'' asked Prince Andrew.

``Andrew, do leave off,'' said Princess Mary. ``Don't tell him,
Pelageya.''

``No... why not, my dear, why shouldn't I? I like him. He is
kind, he is one of God's chosen, he's a benefactor, he once gave
me ten rubles, I remember. When I was in Kiev, Crazy Cyril says
to me (he's one of God's own and goes barefoot summer and
winter), he says, 'Why are you not going to the right place? Go
to Kolyazin where a wonder-working icon of the Holy Mother of God
has been revealed.' On hearing those words I said good-by to the
holy folk and went.''

All were silent, only the pilgrim woman went on in measured
tones, drawing in her breath.

``So I come, master, and the people say to me: 'A great blessing
has been revealed, holy oil trickles from the cheeks of our
blessed Mother, the Holy Virgin Mother of God'...''

``All right, all right, you can tell us afterwards,'' said
Princess Mary, flushing.

``Let me ask her,'' said Pierre. ``Did you see it yourselves?''
he inquired.

``Oh, yes, master, I was found worthy. Such a brightness on the
face like the light of heaven, and from the blessed Mother's
cheek it drops and drops...''

``But, dear me, that must be a fraud!'' said Pierre, naively, who
had listened attentively to the pilgrim.

``Oh, master, what are you saying?'' exclaimed the horrified
Pelageya, turning to Princess Mary for support.

``They impose on the people,'' he repeated.

``Lord Jesus Christ!'' exclaimed the pilgrim woman, crossing
herself. ``Oh, don't speak so, master! There was a general who
did not believe, and said, 'The monks cheat,' and as soon as he'd
said it he went blind. And he dreamed that the Holy Virgin Mother
of the Kiev catacombs came to him and said, 'Believe in me and I
will make you whole.' So he begged: 'Take me to her, take me to
her.' It's the real truth I'm telling you, I saw it myself. So he
was brought, quite blind, straight to her, and he goes up to her
and falls down and says, 'Make me whole,' says he, 'and I'll give
thee what the Tsar bestowed on me.' I saw it myself, master, the
star is fixed into the icon. Well, and what do you think? He
received his sight! It's a sin to speak so. God will punish
you,'' she said admonishingly, turning to Pierre.

``How did the star get into the icon?'' Pierre asked.

``And was the Holy Mother promoted to the rank of general?'' said
Prince Andrew, with a smile.

Pelageya suddenly grew quite pale and clasped her hands.

``Oh, master, master, what a sin! And you who have a son!'' she
began, her pallor suddenly turning to a vivid red. ``Master, what
have you said? God forgive you!'' And she crossed herself. ``Lord
forgive him! My dear, what does it mean?...'' she asked, turning
to Princess Mary. She got up and, almost crying, began to arrange
her wallet. She evidently felt frightened and ashamed to have
accepted charity in a house where such things could be said, and
was at the same time sorry to have now to forgo the charity of
this house.

``Now, why need you do it?'' said Princess Mary. ``Why did you
come to me?...''

``Come, Pelageya, I was joking,'' said Pierre. ``Princesse, ma
parole, je n'ai pas voulu l'offenser.\footnote{''Princess, on my
word, I did not wish to offend her.``} I did not mean anything, I
was only joking,'' he said, smiling shyly and trying to efface
his offense. ``It was all my fault, and Andrew was only joking.''

Pelageya stopped doubtfully, but in Pierre's face there was such
a look of sincere penitence, and Prince Andrew glanced so meekly
now at her and now at Pierre, that she was gradually reassured.

% % % % % % % % % % % % % % % % % % % % % % % % % % % % % % % % %
% % % % % % % % % % % % % % % % % % % % % % % % % % % % % % % % %
% % % % % % % % % % % % % % % % % % % % % % % % % % % % % % % % %
% % % % % % % % % % % % % % % % % % % % % % % % % % % % % % % % %
% % % % % % % % % % % % % % % % % % % % % % % % % % % % % % % % %
% % % % % % % % % % % % % % % % % % % % % % % % % % % % % % % % %
% % % % % % % % % % % % % % % % % % % % % % % % % % % % % % % % %
% % % % % % % % % % % % % % % % % % % % % % % % % % % % % % % % %
% % % % % % % % % % % % % % % % % % % % % % % % % % % % % % % % %
% % % % % % % % % % % % % % % % % % % % % % % % % % % % % % % % %
% % % % % % % % % % % % % % % % % % % % % % % % % % % % % % % % %
% % % % % % % % % % % % % % % % % % % % % % % % % % % % % %

\chapter*{Chapter XIV}
\ifaudio     
\marginpar{
\href{http://ia600200.us.archive.org/6/items/war_and_peace_05_0805_librivox/war_and_peace_05_14_tolstoy_64kb.mp3}{Audio}} 
\fi

\lettrine[lines=2, loversize=0.3, lraise=0]{\initfamily T}{he}
pilgrim woman was appeased and, being encouraged to talk,
gave a long account of Father Amphilochus, who led so holy a life
that his hands smelled of incense, and how on her last visit to
Kiev some monks she knew let her have the keys of the catacombs,
and how she, taking some dried bread with her, had spent two days
in the catacombs with the saints. ``I'd pray awhile to one,
ponder awhile, then go on to another.  I'd sleep a bit and then
again go and kiss the relics, and there was such peace all
around, such blessedness, that one don't want to come out, even
into the light of heaven again.''

Pierre listened to her attentively and seriously. Prince Andrew
went out of the room, and then, leaving \emph{God's folk} to finish
their tea, Princess Mary took Pierre into the drawing room.

``You are very kind,'' she said to him.

``Oh, I really did not mean to hurt her feelings. I understand
them so well and have the greatest respect for them.''

Princess Mary looked at him silently and smiled affectionately.

``I have known you a long time, you see, and am as fond of you as
of a brother,'' she said. ``How do you find Andrew?'' she added
hurriedly, not giving him time to reply to her affectionate
words. ``I am very anxious about him. His health was better in
the winter, but last spring his wound reopened and the doctor
said he ought to go away for a cure. And I am also very much
afraid for him spiritually. He has not a character like us women
who, when we suffer, can weep away our sorrows. He keeps it all
within him. Today he is cheerful and in good spirits, but that is
the effect of your visit---he is not often like that. If you
could persuade him to go abroad. He needs activity, and this
quiet regular life is very bad for him. Others don't notice it,
but I see it.''

Toward ten o'clock the men servants rushed to the front door,
hearing the bells of the old prince's carriage
approaching. Prince Andrew and Pierre also went out into the
porch.

``Who's that?'' asked the old prince, noticing Pierre as he got
out of the carriage.

``Ah! Very glad! Kiss me,'' he said, having learned who the young
stranger was.

The old prince was in a good temper and very gracious to Pierre.

Before supper, Prince Andrew, coming back to his father's study,
found him disputing hotly with his visitor. Pierre was
maintaining that a time would come when there would be no more
wars. The old prince disputed it chaffingly, but without getting
angry.

``Drain the blood from men's veins and put in water instead, then
there will be no more war! Old women's nonsense---old women's
nonsense!'' he repeated, but still he patted Pierre
affectionately on the shoulder, and then went up to the table
where Prince Andrew, evidently not wishing to join in the
conversation, was looking over the papers his father had brought
from town. The old prince went up to him and began to talk
business.

``The marshal, a Count Rostov, hasn't sent half his
contingent. He came to town and wanted to invite me to dinner---I
gave him a pretty dinner!... And there, look at this... Well, my
boy,'' the old prince went on, addressing his son and patting
Pierre on the shoulder. ``A fine fellow---your friend---I like
him! He stirs me up. Another says clever things and one doesn't
care to listen, but this one talks rubbish yet stirs an old
fellow up. Well, go! Get along! Perhaps I'll come and sit with
you at supper. We'll have another dispute. Make friends with my
little fool, Princess Mary,'' he shouted after Pierre, through
the door.

Only now, on his visit to Bald Hills, did Pierre fully realize
the strength and charm of his friendship with Prince Andrew. That
charm was not expressed so much in his relations with him as with
all his family and with the household. With the stern old prince
and the gentle, timid Princess Mary, though he had scarcely known
them, Pierre at once felt like an old friend. They were all fond
of him already. Not only Princess Mary, who had been won by his
gentleness with the pilgrims, gave him her most radiant looks,
but even the one-year-old ``Prince Nicholas'' (as his grandfather
called him) smiled at Pierre and let himself be taken in his
arms, and Michael Ivanovich and Mademoiselle Bourienne looked at
him with pleasant smiles when he talked to the old prince.

The old prince came in to supper; this was evidently on Pierre's
account. And during the two days of the young man's visit he was
extremely kind to him and told him to visit them again.

When Pierre had gone and the members of the household met
together, they began to express their opinions of him as people
always do after a new acquaintance has left, but as seldom
happens, no one said anything but what was good of him.

% % % % % % % % % % % % % % % % % % % % % % % % % % % % % % % % %
% % % % % % % % % % % % % % % % % % % % % % % % % % % % % % % % %
% % % % % % % % % % % % % % % % % % % % % % % % % % % % % % % % %
% % % % % % % % % % % % % % % % % % % % % % % % % % % % % % % % %
% % % % % % % % % % % % % % % % % % % % % % % % % % % % % % % % %
% % % % % % % % % % % % % % % % % % % % % % % % % % % % % % % % %
% % % % % % % % % % % % % % % % % % % % % % % % % % % % % % % % %
% % % % % % % % % % % % % % % % % % % % % % % % % % % % % % % % %
% % % % % % % % % % % % % % % % % % % % % % % % % % % % % % % % %
% % % % % % % % % % % % % % % % % % % % % % % % % % % % % % % % %
% % % % % % % % % % % % % % % % % % % % % % % % % % % % % % % % %
% % % % % % % % % % % % % % % % % % % % % % % % % % % % % %

\chapter*{Chapter XV}
\ifaudio     
\marginpar{
\href{http://ia600200.us.archive.org/6/items/war_and_peace_05_0805_librivox/war_and_peace_05_15_tolstoy_64kb.mp3}{Audio}} 
\fi

\lettrine[lines=2, loversize=0.3, lraise=0]{\initfamily W}{hen}
returning from his leave, Rostov felt, for the first time,
how close was the bond that united him to Denisov and the whole
regiment.

On approaching it, Rostov felt as he had done when approaching
his home in Moscow. When he saw the first hussar with the
unbuttoned uniform of his regiment, when he recognized red-haired
Dementyev and saw the picket ropes of the roan horses, when
Lavrushka gleefully shouted to his master, ``The count has
come!'' and Denisov, who had been asleep on his bed, ran all
disheveled out of the mud hut to embrace him, and the officers
collected round to greet the new arrival, Rostov experienced the
same feeling as when his mother, his father, and his sister had
embraced him, and tears of joy choked him so that he could not
speak.  The regiment was also a home, and as unalterably dear and
precious as his parents' house.

When he had reported himself to the commander of the regiment and
had been reassigned to his former squadron, had been on duty and
had gone out foraging, when he had again entered into all the
little interests of the regiment and felt himself deprived of
liberty and bound in one narrow, unchanging frame, he experienced
the same sense of peace, of moral support, and the same sense of
being at home here in his own place, as he had felt under the
parental roof. But here was none of all that turmoil of the world
at large, where he did not know his right place and took mistaken
decisions; here was no Sonya with whom he ought, or ought not, to
have an explanation; here was no possibility of going there or
not going there; here there were not twenty-four hours in the day
which could be spent in such a variety of ways; there was not
that innumerable crowd of people of whom not one was nearer to
him or farther from him than another; there were none of those
uncertain and undefined money relations with his father, and
nothing to recall that terrible loss to Dolokhov. Here, in the
regiment, all was clear and simple. The whole world was divided
into two unequal parts: one, our Pavlograd regiment; the other,
all the rest. And the rest was no concern of his.  In the
regiment, everything was definite: who was lieutenant, who
captain, who was a good fellow, who a bad one, and most of all,
who was a comrade. The canteenkeeper gave one credit, one's pay
came every four months, there was nothing to think out or decide,
you had only to do nothing that was considered bad in the
Pavlograd regiment and, when given an order, to do what was
clearly, distinctly, and definitely ordered---and all would be
well.

Having once more entered into the definite conditions of this
regimental life, Rostov felt the joy and relief a tired man feels
on lying down to rest. Life in the regiment, during this
campaign, was all the pleasanter for him, because, after his loss
to Dolokhov (for which, in spite of all his family's efforts to
console him, he could not forgive himself), he had made up his
mind to atone for his fault by serving, not as he had done
before, but really well, and by being a perfectly first-rate
comrade and officer---in a word, a splendid man altogether, a
thing which seemed so difficult out in the world, but so possible
in the regiment.

After his losses, he had determined to pay back his debt to his
parents in five years. He received ten thousand rubles a year,
but now resolved to take only two thousand and leave the rest to
repay the debt to his parents.

Our army, after repeated retreats and advances and battles at
Pultusk and Preussisch-Eylau, was concentrated near
Bartenstein. It was awaiting the Emperor's arrival and the
beginning of a new campaign.

The Pavlograd regiment, belonging to that part of the army which
had served in the 1805 campaign, had been recruiting up to
strength in Russia, and arrived too late to take part in the
first actions of the campaign. It had been neither at Pultusk nor
at Preussisch-Eylau and, when it joined the army in the field in
the second half of the campaign, was attached to Platov's
division.

Platov's division was acting independently of the main
army. Several times parts of the Pavlograd regiment had exchanged
shots with the enemy, had taken prisoners, and once had even
captured Marshal Oudinot's carriages. In April the Pavlograds
were stationed immovably for some weeks near a totally ruined and
deserted German village.

A thaw had set in, it was muddy and cold, the ice on the river
broke, and the roads became impassable. For days neither
provisions for the men nor fodder for the horses had been
issued. As no transports could arrive, the men dispersed about
the abandoned and deserted villages, searching for potatoes, but
found few even of these.

Everything had been eaten up and the inhabitants had all
fled---if any remained, they were worse than beggars and nothing
more could be taken from them; even the soldiers, usually
pitiless enough, instead of taking anything from them, often gave
them the last of their rations.

The Pavlograd regiment had had only two men wounded in action,
but had lost nearly half its men from hunger and sickness. In the
hospitals, death was so certain that soldiers suffering from
fever, or the swelling that came from bad food, preferred to
remain on duty, and hardly able to drag their legs went to the
front rather than to the hospitals. When spring came on, the
soldiers found a plant just showing out of the ground that looked
like asparagus, which, for some reason, they called
\emph{Mashka's sweet root}. It was very bitter, but they wandered
about the fields seeking it and dug it out with their sabers and
ate it, though they were ordered not to do so, as it was a
noxious plant. That spring a new disease broke out among the
soldiers, a swelling of the arms, legs, and face, which the
doctors attributed to eating this root. But in spite of all this,
the soldiers of Denisov's squadron fed chiefly on \emph{Mashka's
  sweet root}, because it was the second week that the last of
the biscuits were being doled out at the rate of half a pound a
man and the last potatoes received had sprouted and frozen.

The horses also had been fed for a fortnight on straw from the
thatched roofs and had become terribly thin, though still covered
with tufts of felty winter hair.

Despite this destitution, the soldiers and officers went on
living just as usual. Despite their pale swollen faces and
tattered uniforms, the hussars formed line for roll call, kept
things in order, groomed their horses, polished their arms,
brought in straw from the thatched roofs in place of fodder, and
sat down to dine round the caldrons from which they rose up
hungry, joking about their nasty food and their hunger. As usual,
in their spare time, they lit bonfires, steamed themselves before
them naked; smoked, picked out and baked sprouting rotten
potatoes, told and listened to stories of Potemkin's and
Suvorov's campaigns, or to legends of Alesha the Sly, or the
priest's laborer Mikolka.

The officers, as usual, lived in twos and threes in the roofless,
half-ruined houses. The seniors tried to collect straw and
potatoes and, in general, food for the men. The younger ones
occupied themselves as before, some playing cards (there was
plenty of money, though there was no food), some with more
innocent games, such as quoits and skittles.  The general trend
of the campaign was rarely spoken of, partly because nothing
certain was known about it, partly because there was a vague
feeling that in the main it was going badly.

Rostov lived, as before, with Denisov, and since their furlough
they had become more friendly than ever. Denisov never spoke of
Rostov's family, but by the tender friendship his commander
showed him, Rostov felt that the elder hussar's luckless love for
Natasha played a part in strengthening their friendship. Denisov
evidently tried to expose Rostov to danger as seldom as possible,
and after an action greeted his safe return with evident joy. On
one of his foraging expeditions, in a deserted and ruined village
to which he had come in search of provisions, Rostov found a
family consisting of an old Pole and his daughter with an infant
in arms. They were half clad, hungry, too weak to get away on
foot and had no means of obtaining a conveyance. Rostov brought
them to his quarters, placed them in his own lodging, and kept
them for some weeks while the old man was recovering. One of his
comrades, talking of women, began chaffing Rostov, saying that he
was more wily than any of them and that it would not be a bad
thing if he introduced to them the pretty Polish girl he had
saved. Rostov took the joke as an insult, flared up, and said
such unpleasant things to the officer that it was all Denisov
could do to prevent a duel. When the officer had gone away,
Denisov, who did not himself know what Rostov's relations with
the Polish girl might be, began to upbraid him for his quickness
of temper, and Rostov replied:

``Say what you like... She is like a sister to me, and I can't
tell you how it offended me... because... well, for that
reason...''

Denisov patted him on the shoulder and began rapidly pacing the
room without looking at Rostov, as was his way at moments of deep
feeling.

``Ah, what a mad bweed you Wostovs are!'' he muttered, and Rostov
noticed tears in his eyes.

% % % % % % % % % % % % % % % % % % % % % % % % % % % % % % % % %
% % % % % % % % % % % % % % % % % % % % % % % % % % % % % % % % %
% % % % % % % % % % % % % % % % % % % % % % % % % % % % % % % % %
% % % % % % % % % % % % % % % % % % % % % % % % % % % % % % % % %
% % % % % % % % % % % % % % % % % % % % % % % % % % % % % % % % %
% % % % % % % % % % % % % % % % % % % % % % % % % % % % % % % % %
% % % % % % % % % % % % % % % % % % % % % % % % % % % % % % % % %
% % % % % % % % % % % % % % % % % % % % % % % % % % % % % % % % %
% % % % % % % % % % % % % % % % % % % % % % % % % % % % % % % % %
% % % % % % % % % % % % % % % % % % % % % % % % % % % % % % % % %
% % % % % % % % % % % % % % % % % % % % % % % % % % % % % % % % %
% % % % % % % % % % % % % % % % % % % % % % % % % % % % % %

\chapter*{Chapter XVI}
\ifaudio     
\marginpar{
\href{http://ia600200.us.archive.org/6/items/war_and_peace_05_0805_librivox/war_and_peace_05_16_tolstoy_64kb.mp3}{Audio}} 
\fi

\lettrine[lines=2, loversize=0.3, lraise=0]{\initfamily I}{n}
April the troops were enlivened by news of the Emperor's
arrival, but Rostov had no chance of being present at the review
he held at Bartenstein, as the Pavlograds were at the outposts
far beyond that place.

They were bivouacking. Denisov and Rostov were living in an earth
hut, dug out for them by the soldiers and roofed with branches
and turf. The hut was made in the following manner, which had
then come into vogue. A trench was dug three and a half feet
wide, four feet eight inches deep, and eight feet long. At one
end of the trench, steps were cut out and these formed the
entrance and vestibule. The trench itself was the room, in which
the lucky ones, such as the squadron commander, had a board,
lying on piles at the end opposite the entrance, to serve as a
table. On each side of the trench, the earth was cut out to a
breadth of about two and a half feet, and this did duty for
bedsteads and couches. The roof was so constructed that one could
stand up in the middle of the trench and could even sit up on the
beds if one drew close to the table.  Denisov, who was living
luxuriously because the soldiers of his squadron liked him, had
also a board in the roof at the farther end, with a piece of
(broken but mended) glass in it for a window. When it was very
cold, embers from the soldiers' campfire were placed on a bent
sheet of iron on the steps in the \emph{reception room}---as Denisov
called that part of the hut---and it was then so warm that the
officers, of whom there were always some with Denisov and Rostov,
sat in their shirt sleeves.

In April, Rostov was on orderly duty. One morning, between seven
and eight, returning after a sleepless night, he sent for embers,
changed his rain-soaked underclothes, said his prayers, drank
tea, got warm, then tidied up the things on the table and in his
own corner, and, his face glowing from exposure to the wind and
with nothing on but his shirt, lay down on his back, putting his
arms under his head. He was pleasantly considering the
probability of being promoted in a few days for his last
reconnoitering expedition, and was awaiting Denisov, who had gone
out somewhere and with whom he wanted a talk.

Suddenly he heard Denisov shouting in a vibrating voice behind
the hut, evidently much excited. Rostov moved to the window to
see whom he was speaking to, and saw the quartermaster,
Topcheenko.

``I ordered you not to let them eat that Mashka woot stuff!''
Denisov was shouting. ``And I saw with my own eyes how Lazarchuk
bwought some fwom the fields.''

``I have given the order again and again, your honor, but they
don't obey,'' answered the quartermaster.

Rostov lay down again on his bed and thought complacently: ``Let
him fuss and bustle now, my job's done and I'm lying
down---capitally!'' He could hear that Lavrushka---that sly, bold
orderly of Denisov's---was talking, as well as the
quartermaster. Lavrushka was saying something about loaded
wagons, biscuits, and oxen he had seen when he had gone out for
provisions.

Then Denisov's voice was heard shouting farther and farther away.
``Saddle! Second platoon!''

``Where are they off to now?'' thought Rostov.

Five minutes later, Denisov came into the hut, climbed with muddy
boots on the bed, lit his pipe, furiously scattered his things
about, took his leaded whip, buckled on his saber, and went out
again. In answer to Rostov's inquiry where he was going, he
answered vaguely and crossly that he had some business.

``Let God and our gweat monarch judge me afterwards!'' said
Denisov going out, and Rostov heard the hoofs of several horses
splashing through the mud. He did not even trouble to find out
where Denisov had gone. Having got warm in his corner, he fell
asleep and did not leave the hut till toward evening. Denisov had
not yet returned. The weather had cleared up, and near the next
hut two officers and a cadet were playing svayka, laughing as
they threw their missiles which buried themselves in the soft
mud. Rostov joined them. In the middle of the game, the officers
saw some wagons approaching with fifteen hussars on their skinny
horses behind them. The wagons escorted by the hussars drew up to
the picket ropes and a crowd of hussars surrounded them.

``There now, Denisov has been worrying,'' said Rostov, ``and here
are the provisions.''

``So they are!'' said the officers. ``Won't the soldiers be
glad!''

A little behind the hussars came Denisov, accompanied by two
infantry officers with whom he was talking.

Rostov went to meet them.

``I warn you, Captain,'' one of the officers, a short thin man,
evidently very angry, was saying.

``Haven't I told you I won't give them up?'' replied Denisov.

``You will answer for it, Captain. It is mutiny---seizing the
transport of one's own army. Our men have had nothing to eat for
two days.''

``And mine have had nothing for two weeks,'' said Denisov.

``It is robbery! You'll answer for it, sir!'' said the infantry
officer, raising his voice.

``Now, what are you pestewing me for?'' cried Denisov, suddenly
losing his temper. ``I shall answer for it and not you, and you'd
better not buzz about here till you get hurt. Be off! Go!'' he
shouted at the officers.

``Very well, then!'' shouted the little officer, undaunted and
not riding away. ``If you are determined to rob, I'll...''

``Go to the devil! quick ma'ch, while you're safe and sound!''
and Denisov turned his horse on the officer.

``Very well, very well!'' muttered the officer, threateningly,
and turning his horse he trotted away, jolting in his saddle.

``A dog astwide a fence! A weal dog astwide a fence!'' shouted
Denisov after him (the most insulting expression a cavalryman can
address to a mounted infantryman) and riding up to Rostov, he
burst out laughing.

``I've taken twansports from the infantwy by force!'' he
said. ``After all, can't let our men starve.''

The wagons that had reached the hussars had been consigned to an
infantry regiment, but learning from Lavrushka that the transport
was unescorted, Denisov with his hussars had seized it by
force. The soldiers had biscuits dealt out to them freely, and
they even shared them with the other squadrons.

The next day the regimental commander sent for Denisov, and
holding his fingers spread out before his eyes said:

``This is how I look at this affair: I know nothing about it and
won't begin proceedings, but I advise you to ride over to the
staff and settle the business there in the commissariat
department and if possible sign a receipt for such and such
stores received. If not, as the demand was booked against an
infantry regiment, there will be a row and the affair may end
badly.''

From the regimental commander's, Denisov rode straight to the
staff with a sincere desire to act on this advice. In the evening
he came back to his dugout in a state such as Rostov had never
yet seen him in. Denisov could not speak and gasped for
breath. When Rostov asked what was the matter, he only uttered
some incoherent oaths and threats in a hoarse, feeble voice.

Alarmed at Denisov's condition, Rostov suggested that he should
undress, drink some water, and send for the doctor.

``Twy me for wobbewy... oh! Some more water... Let them twy me,
but I'll always thwash scoundwels... and I'll tell the
Empewo'... Ice...'' he muttered.

The regimental doctor, when he came, said it was absolutely
necessary to bleed Denisov. A deep saucer of black blood was
taken from his hairy arm and only then was he able to relate what
had happened to him.

``I get there,'' began Denisov. ``'Now then, where's your chief's
quarters?' They were pointed out. 'Please to wait.' 'I've widden
twenty miles and have duties to attend to and no time to
wait. Announce me.'  Vewy well, so out comes their head
chief---also took it into his head to lecture me: 'It's
wobbewy!'---'Wobbewy,' I say, 'is not done by man who seizes
pwovisions to feed his soldiers, but by him who takes them to
fill his own pockets!' 'Will you please be silent?' 'Vewy good!'
Then he says: 'Go and give a weceipt to the commissioner, but
your affair will be passed on to headquarters.' I go to the
commissioner. I enter, and at the table... who do you think? No,
but wait a bit!... Who is it that's starving us?'' shouted
Denisov, hitting the table with the fist of his newly bled arm so
violently that the table nearly broke down and the tumblers on it
jumped about. ``Telyanin! 'What? So it's you who's starving us to
death! Is it? Take this and this!' and I hit him so pat, stwaight
on his snout... 'Ah, what a... what a...!' and I sta'ted fwashing
him... Well, I've had a bit of fun I can tell you!'' cried
Denisov, gleeful and yet angry, his white teeth showing under his
black mustache. ``I'd have killed him if they hadn't taken him
away!''

``But what are you shouting for? Calm yourself,'' said
Rostov. ``You've set your arm bleeding afresh. Wait, we must tie
it up again.''

Denisov was bandaged up again and put to bed. Next day he woke
calm and cheerful.

But at noon the adjutant of the regiment came into Rostov's and
Denisov's dugout with a grave and serious face and regretfully
showed them a paper addressed to Major Denisov from the
regimental commander in which inquiries were made about
yesterday's occurrence. The adjutant told them that the affair
was likely to take a very bad turn: that a court-martial had been
appointed, and that in view of the severity with which marauding
and insubordination were now regarded, degradation to the ranks
would be the best that could be hoped for.

The case, as represented by the offended parties, was that, after
seizing the transports, Major Denisov, being drunk, went to the
chief quartermaster and without any provocation called him a
thief, threatened to strike him, and on being led out had rushed
into the office and given two officials a thrashing, and
dislocated the arm of one of them.

In answer to Rostov's renewed questions, Denisov said, laughing,
that he thought he remembered that some other fellow had got
mixed up in it, but that it was all nonsense and rubbish, and he
did not in the least fear any kind of trial, and that if those
scoundrels dared attack him he would give them an answer that
they would not easily forget.

Denisov spoke contemptuously of the whole matter, but Rostov knew
him too well not to detect that (while hiding it from others) at
heart he feared a court-martial and was worried over the affair,
which was evidently taking a bad turn. Every day, letters of
inquiry and notices from the court arrived, and on the first of
May, Denisov was ordered to hand the squadron over to the next in
seniority and appear before the staff of his division to explain
his violence at the commissariat office. On the previous day
Platov reconnoitered with two Cossack regiments and two squadrons
of hussars. Denisov, as was his wont, rode out in front of the
outposts, parading his courage. A bullet fired by a French
sharpshooter hit him in the fleshy part of his leg. Perhaps at
another time Denisov would not have left the regiment for so
slight a wound, but now he took advantage of it to excuse himself
from appearing at the staff and went into hospital.

% % % % % % % % % % % % % % % % % % % % % % % % % % % % % % % % %
% % % % % % % % % % % % % % % % % % % % % % % % % % % % % % % % %
% % % % % % % % % % % % % % % % % % % % % % % % % % % % % % % % %
% % % % % % % % % % % % % % % % % % % % % % % % % % % % % % % % %
% % % % % % % % % % % % % % % % % % % % % % % % % % % % % % % % %
% % % % % % % % % % % % % % % % % % % % % % % % % % % % % % % % %
% % % % % % % % % % % % % % % % % % % % % % % % % % % % % % % % %
% % % % % % % % % % % % % % % % % % % % % % % % % % % % % % % % %
% % % % % % % % % % % % % % % % % % % % % % % % % % % % % % % % %
% % % % % % % % % % % % % % % % % % % % % % % % % % % % % % % % %
% % % % % % % % % % % % % % % % % % % % % % % % % % % % % % % % %
% % % % % % % % % % % % % % % % % % % % % % % % % % % % % %

\chapter*{Chapter XVII}
\ifaudio     
\marginpar{
\href{http://ia600200.us.archive.org/6/items/war_and_peace_05_0805_librivox/war_and_peace_05_17_tolstoy_64kb.mp3}{Audio}} 
\fi

\lettrine[lines=2, loversize=0.3, lraise=0]{\initfamily I}{n}
June the battle of Friedland was fought, in which the
Pavlograds did not take part, and after that an armistice was
proclaimed. Rostov, who felt his friend's absence very much,
having no news of him since he left and feeling very anxious
about his wound and the progress of his affairs, took advantage
of the armistice to get leave to visit Denisov in hospital.

The hospital was in a small Prussian town that had been twice
devastated by Russian and French troops. Because it was summer,
when it is so beautiful out in the fields, the little town
presented a particularly dismal appearance with its broken roofs
and fences, its foul streets, tattered inhabitants, and the sick
and drunken soldiers wandering about.

The hospital was in a brick building with some of the window
frames and panes broken and a courtyard surrounded by the remains
of a wooden fence that had been pulled to pieces. Several
bandaged soldiers, with pale swollen faces, were sitting or
walking about in the sunshine in the yard.

Directly Rostov entered the door he was enveloped by a smell of
putrefaction and hospital air. On the stairs he met a Russian
army doctor smoking a cigar. The doctor was followed by a Russian
assistant.

``I can't tear myself to pieces,'' the doctor was saying. ``Come
to Makar Alexeevich in the evening. I shall be there.''

The assistant asked some further questions.

``Oh, do the best you can! Isn't it all the same?'' The doctor
noticed Rostov coming upstairs.

``What do you want, sir?'' said the doctor. ``What do you want?
The bullets having spared you, do you want to try typhus? This is
a pesthouse, sir.''

``How so?'' asked Rostov.

``Typhus, sir. It's death to go in. Only we two, Makeev and I''
(he pointed to the assistant), ``keep on here. Some five of us
doctors have died in this place... When a new one comes he is
done for in a week,'' said the doctor with evident
satisfaction. ``Prussian doctors have been invited here, but our
allies don't like it at all.''

Rostov explained that he wanted to see Major Denisov of the
hussars, who was wounded.

``I don't know. I can't tell you, sir. Only think! I am alone in
charge of three hospitals with more than four hundred patients!
It's well that the charitable Prussian ladies send us two pounds
of coffee and some lint each month or we should be lost!'' he
laughed. ``Four hundred, sir, and they're always sending me fresh
ones. There are four hundred? Eh?''  he asked, turning to the
assistant.

The assistant looked fagged out. He was evidently vexed and
impatient for the talkative doctor to go.

``Major Denisov,'' Rostov said again. ``He was wounded at
Molliten.''

``Dead, I fancy. Eh, Makeev?'' queried the doctor, in a tone of
indifference.

The assistant, however, did not confirm the doctor's words.

``Is he tall and with reddish hair?'' asked the doctor.

Rostov described Denisov's appearance.

``There was one like that,'' said the doctor, as if
pleased. ``That one is dead, I fancy. However, I'll look up our
list. We had a list. Have you got it, Makeev?''

``Makar Alexeevich has the list,'' answered the assistant. ``But
if you'll step into the officers' wards you'll see for
yourself,'' he added, turning to Rostov.

``Ah, you'd better not go, sir,'' said the doctor, ``or you may
have to stay here yourself.''

But Rostov bowed himself away from the doctor and asked the
assistant to show him the way.

``Only don't blame me!'' the doctor shouted up after him.

Rostov and the assistant went into the dark corridor. The smell
was so strong there that Rostov held his nose and had to pause
and collect his strength before he could go on. A door opened to
the right, and an emaciated sallow man on crutches, barefoot and
in underclothing, limped out and, leaning against the doorpost,
looked with glittering envious eyes at those who were
passing. Glancing in at the door, Rostov saw that the sick and
wounded were lying on the floor on straw and overcoats.

``May I go in and look?''

``What is there to see?'' said the assistant.

But, just because the assistant evidently did not want him to go
in, Rostov entered the soldiers' ward. The foul air, to which he
had already begun to get used in the corridor, was still stronger
here. It was a little different, more pungent, and one felt that
this was where it originated.

In the long room, brightly lit up by the sun through the large
windows, the sick and wounded lay in two rows with their heads to
the walls, and leaving a passage in the middle. Most of them were
unconscious and paid no attention to the newcomers. Those who
were conscious raised themselves or lifted their thin yellow
faces, and all looked intently at Rostov with the same expression
of hope, of relief, reproach, and envy of another's
health. Rostov went to the middle of the room and looking through
the open doors into the two adjoining rooms saw the same thing
there. He stood still, looking silently around. He had not at all
expected such a sight. Just before him, almost across the middle
of the passage on the bare floor, lay a sick man, probably a
Cossack to judge by the cut of his hair. The man lay on his back,
his huge arms and legs outstretched. His face was purple, his
eyes were rolled back so that only the whites were seen, and on
his bare legs and arms which were still red, the veins stood out
like cords. He was knocking the back of his head against the
floor, hoarsely uttering some word which he kept
repeating. Rostov listened and made out the word. It was ``drink,
drink, a drink!'' Rostov glanced round, looking for someone who
would put this man back in his place and bring him water.

``Who looks after the sick here?'' he asked the assistant.

Just then a commissariat soldier, a hospital orderly, came in
from the next room, marching stiffly, and drew up in front of
Rostov.

``Good day, your honor!'' he shouted, rolling his eyes at Rostov
and evidently mistaking him for one of the hospital authorities.

``Get him to his place and give him some water,'' said Rostov,
pointing to the Cossack.

``Yes, your honor,'' the soldier replied complacently, and
rolling his eyes more than ever he drew himself up still
straighter, but did not move.

``No, it's impossible to do anything here,'' thought Rostov,
lowering his eyes, and he was going out, but became aware of an
intense look fixed on him on his right, and he turned. Close to
the corner, on an overcoat, sat an old, unshaven, gray-bearded
soldier as thin as a skeleton, with a stern sallow face and eyes
intently fixed on Rostov. The man's neighbor on one side
whispered something to him, pointing at Rostov, who noticed that
the old man wanted to speak to him. He drew nearer and saw that
the old man had only one leg bent under him, the other had been
amputated above the knee. His neighbor on the other side, who lay
motionless some distance from him with his head thrown back, was
a young soldier with a snub nose. His pale waxen face was still
freckled and his eyes were rolled back. Rostov looked at the
young soldier and a cold chill ran down his back.

``Why, this one seems...'' he began, turning to the assistant.

``And how we've been begging, your honor,'' said the old soldier,
his jaw quivering. ``He's been dead since morning. After all
we're men, not dogs.''

``I'll send someone at once. He shall be taken away---taken away
at once,'' said the assistant hurriedly. ``Let us go, your
honor.''

``Yes, yes, let us go,'' said Rostov hastily, and lowering his
eyes and shrinking, he tried to pass unnoticed between the rows
of reproachful envious eyes that were fixed upon him, and went
out of the room.

% % % % % % % % % % % % % % % % % % % % % % % % % % % % % % % % %
% % % % % % % % % % % % % % % % % % % % % % % % % % % % % % % % %
% % % % % % % % % % % % % % % % % % % % % % % % % % % % % % % % %
% % % % % % % % % % % % % % % % % % % % % % % % % % % % % % % % %
% % % % % % % % % % % % % % % % % % % % % % % % % % % % % % % % %
% % % % % % % % % % % % % % % % % % % % % % % % % % % % % % % % %
% % % % % % % % % % % % % % % % % % % % % % % % % % % % % % % % %
% % % % % % % % % % % % % % % % % % % % % % % % % % % % % % % % %
% % % % % % % % % % % % % % % % % % % % % % % % % % % % % % % % %
% % % % % % % % % % % % % % % % % % % % % % % % % % % % % % % % %
% % % % % % % % % % % % % % % % % % % % % % % % % % % % % % % % %
% % % % % % % % % % % % % % % % % % % % % % % % % % % % % %

\chapter*{Chapter XVIII}
\ifaudio     
\marginpar{
\href{http://ia600200.us.archive.org/6/items/war_and_peace_05_0805_librivox/war_and_peace_05_18_tolstoy_64kb.mp3}{Audio}} 
\fi

\lettrine[lines=2, loversize=0.3, lraise=0]{\initfamily G}{oing}
along the corridor, the assistant led Rostov to the
officers' wards, consisting of three rooms, the doors of which
stood open. There were beds in these rooms and the sick and
wounded officers were lying or sitting on them. Some were walking
about the rooms in hospital dressing gowns. The first person
Rostov met in the officers' ward was a thin little man with one
arm, who was walking about the first room in a nightcap and
hospital dressing gown, with a pipe between his teeth.  Rostov
looked at him, trying to remember where he had seen him before.

``See where we've met again!'' said the little man. ``Tushin,
Tushin, don't you remember, who gave you a lift at Schon Grabern?
And I've had a bit cut off, you see...'' he went on with a smile,
pointing to the empty sleeve of his dressing gown. ``Looking for
Vasili Dmitrich Denisov? My neighbor,'' he added, when he heard
who Rostov wanted. ``Here, here,'' and Tushin led him into the
next room, from whence came sounds of several laughing voices.

``How can they laugh, or even live at all here?'' thought Rostov,
still aware of that smell of decomposing flesh that had been so
strong in the soldiers' ward, and still seeming to see fixed on
him those envious looks which had followed him out from both
sides, and the face of that young soldier with eyes rolled back.

Denisov lay asleep on his bed with his head under the blanket,
though it was nearly noon.

``Ah, Wostov? How are you, how are you?'' he called out, still in
the same voice as in the regiment, but Rostov noticed sadly that
under this habitual ease and animation some new, sinister, hidden
feeling showed itself in the expression of Denisov's face and the
intonations of his voice.

His wound, though a slight one, had not yet healed even now, six
weeks after he had been hit. His face had the same swollen pallor
as the faces of the other hospital patients, but it was not this
that struck Rostov.  What struck him was that Denisov did not
seem glad to see him, and smiled at him unnaturally. He did not
ask about the regiment, nor about the general state of affairs,
and when Rostov spoke of these matters did not listen.

Rostov even noticed that Denisov did not like to be reminded of
the regiment, or in general of that other free life which was
going on outside the hospital. He seemed to try to forget that
old life and was only interested in the affair with the
commissariat officers. On Rostov's inquiry as to how the matter
stood, he at once produced from under his pillow a paper he had
received from the commission and the rough draft of his answer to
it. He became animated when he began reading his paper and
specially drew Rostov's attention to the stinging rejoinders he
made to his enemies. His hospital companions, who had gathered
round Rostov---a fresh arrival from the world outside---gradually
began to disperse as soon as Denisov began reading his
answer. Rostov noticed by their faces that all those gentlemen
had already heard that story more than once and were tired of
it. Only the man who had the next bed, a stout Uhlan, continued
to sit on his bed, gloomily frowning and smoking a pipe, and
little one-armed Tushin still listened, shaking his head
disapprovingly. In the middle of the reading, the Uhlan
interrupted Denisov.

``But what I say is,'' he said, turning to Rostov, ``it would be
best simply to petition the Emperor for pardon. They say great
rewards will now be distributed, and surely a pardon would be
granted...''

``Me petition the Empewo'!'' exclaimed Denisov, in a voice to
which he tried hard to give the old energy and fire, but which
sounded like an expression of irritable impotence. ``What for? If
I were a wobber I would ask mercy, but I'm being court-martialed
for bwinging wobbers to book.  Let them twy me, I'm not afwaid of
anyone. I've served the Tsar and my countwy honowably and have
not stolen! And am I to be degwaded?...  Listen, I'm w'iting to
them stwaight. This is what I say: 'If I had wobbed the
Tweasuwy...'{}''

``It's certainly well written,'' said Tushin, ``but that's not
the point, Vasili Dmitrich,'' and he also turned to Rostov. ``One
has to submit, and Vasili Dmitrich doesn't want to. You know the
auditor told you it was a bad business.''

``Well, let it be bad,'' said Denisov.

``The auditor wrote out a petition for you,'' continued Tushin,
``and you ought to sign it and ask this gentleman to take it. No
doubt he'' (indicating Rostov) ``has connections on the
staff. You won't find a better opportunity.''

``Haven't I said I'm not going to gwovel?'' Denisov interrupted
him, went on reading his paper.

Rostov had not the courage to persuade Denisov, though he
instinctively felt that the way advised by Tushin and the other
officers was the safest, and though he would have been glad to be
of service to Denisov.  He knew his stubborn will and
straightforward hasty temper.

When the reading of Denisov's virulent reply, which took more
than an hour, was over, Rostov said nothing, and he spent the
rest of the day in a most dejected state of mind amid Denisov's
hospital comrades, who had gathered round him, telling them what
he knew and listening to their stories. Denisov was moodily
silent all the evening.

Late in the evening, when Rostov was about to leave, he asked
Denisov whether he had no commission for him.

``Yes, wait a bit,'' said Denisov, glancing round at the
officers, and taking his papers from under his pillow he went to
the window, where he had an inkpot, and sat down to write.

``It seems it's no use knocking one's head against a wall!'' he
said, coming from the window and giving Rostov a large
envelope. In it was the petition to the Emperor drawn up by the
auditor, in which Denisov, without alluding to the offenses of
the commissariat officials, simply asked for pardon.

``Hand it in. It seems...''

He did not finish, but gave a painfully unnatural smile.

% % % % % % % % % % % % % % % % % % % % % % % % % % % % % % % % %
% % % % % % % % % % % % % % % % % % % % % % % % % % % % % % % % %
% % % % % % % % % % % % % % % % % % % % % % % % % % % % % % % % %
% % % % % % % % % % % % % % % % % % % % % % % % % % % % % % % % %
% % % % % % % % % % % % % % % % % % % % % % % % % % % % % % % % %
% % % % % % % % % % % % % % % % % % % % % % % % % % % % % % % % %
% % % % % % % % % % % % % % % % % % % % % % % % % % % % % % % % %
% % % % % % % % % % % % % % % % % % % % % % % % % % % % % % % % %
% % % % % % % % % % % % % % % % % % % % % % % % % % % % % % % % %
% % % % % % % % % % % % % % % % % % % % % % % % % % % % % % % % %
% % % % % % % % % % % % % % % % % % % % % % % % % % % % % % % % %
% % % % % % % % % % % % % % % % % % % % % % % % % % % % % %

\chapter*{Chapter XIX}
\ifaudio     
\marginpar{
\href{http://ia600200.us.archive.org/6/items/war_and_peace_05_0805_librivox/war_and_peace_05_19_tolstoy_64kb.mp3}{Audio}} 
\fi

\lettrine[lines=2, loversize=0.3, lraise=0]{\initfamily H}{aving}
returned to the regiment and told the commander the state
of Denisov's affairs, Rostov rode to Tilsit with the letter to
the Emperor.

On the thirteenth of June the French and Russian Emperors arrived
in Tilsit. Boris Drubetskoy had asked the important personage on
whom he was in attendance, to include him in the suite appointed
for the stay at Tilsit.

``I should like to see the great man,'' he said, alluding to
Napoleon, whom hitherto he, like everyone else, had always called
Buonaparte.

``You are speaking of Buonaparte?'' asked the general, smiling.

Boris looked at his general inquiringly and immediately saw that
he was being tested.

``I am speaking, Prince, of the Emperor Napoleon,'' he
replied. The general patted him on the shoulder, with a smile.

``You will go far,'' he said, and took him to Tilsit with him.

Boris was among the few present at the Niemen on the day the two
Emperors met. He saw the raft, decorated with monograms, saw
Napoleon pass before the French Guards on the farther bank of the
river, saw the pensive face of the Emperor Alexander as he sat in
silence in a tavern on the bank of the Niemen awaiting Napoleon's
arrival, saw both Emperors get into boats, and saw how
Napoleon---reaching the raft first---stepped quickly forward to
meet Alexander and held out his hand to him, and how they both
retired into the pavilion. Since he had begun to move in the
highest circles Boris had made it his habit to watch attentively
all that went on around him and to note it down. At the time of
the meeting at Tilsit he asked the names of those who had come
with Napoleon and about the uniforms they wore, and listened
attentively to words spoken by important personages. At the
moment the Emperors went into the pavilion he looked at his
watch, and did not forget to look at it again when Alexander came
out. The interview had lasted an hour and fifty-three minutes. He
noted this down that same evening, among other facts he felt to
be of historic importance. As the Emperor's suite was a very
small one, it was a matter of great importance, for a man who
valued his success in the service, to be at Tilsit on the
occasion of this interview between the two Emperors, and having
succeeded in this, Boris felt that henceforth his position was
fully assured. He had not only become known, but people had grown
accustomed to him and accepted him.  Twice he had executed
commissions to the Emperor himself, so that the latter knew his
face, and all those at court, far from cold-shouldering him as at
first when they considered him a newcomer, would now have been
surprised had he been absent.

Boris lodged with another adjutant, the Polish Count Zhilinski.
Zhilinski, a Pole brought up in Paris, was rich, and passionately
fond of the French, and almost every day of the stay at Tilsit,
French officers of the Guard and from French headquarters were
dining and lunching with him and Boris.

On the evening of the twenty-fourth of June, Count Zhilinski
arranged a supper for his French friends. The guest of honor was
an aide-de-camp of Napoleon's, there were also several French
officers of the Guard, and a page of Napoleon's, a young lad of
an old aristocratic French family.  That same day, Rostov,
profiting by the darkness to avoid being recognized in civilian
dress, came to Tilsit and went to the lodging occupied by Boris
and Zhilinski.

Rostov, in common with the whole army from which he came, was far
from having experienced the change of feeling toward Napoleon and
the French---who from being foes had suddenly become
friends---that had taken place at headquarters and in Boris. In
the army, Bonaparte and the French were still regarded with
mingled feelings of anger, contempt, and fear. Only recently,
talking with one of Platov's Cossack officers, Rostov had argued
that if Napoleon were taken prisoner he would be treated not as a
sovereign, but as a criminal. Quite lately, happening to meet a
wounded French colonel on the road, Rostov had maintained with
heat that peace was impossible between a legitimate sovereign and
the criminal Bonaparte. Rostov was therefore unpleasantly struck
by the presence of French officers in Boris' lodging, dressed in
uniforms he had been accustomed to see from quite a different
point of view from the outposts of the flank. As soon as he
noticed a French officer, who thrust his head out of the door,
that warlike feeling of hostility which he always experienced at
the sight of the enemy suddenly seized him. He stopped at the
threshold and asked in Russian whether Drubetskoy lived there.
Boris, hearing a strange voice in the anteroom, came out to meet
him. An expression of annoyance showed itself for a moment on his
face on first recognizing Rostov.

``Ah, it's you? Very glad, very glad to see you,'' he said,
however, coming toward him with a smile. But Rostov had noticed
his first impulse.

``I've come at a bad time I think. I should not have come, but I
have business,'' he said coldly.

``No, I only wonder how you managed to get away from your
regiment. Dans un moment je suis a vous,''\footnote{``In a minute
I shall be at your disposal.''} he said, answering someone who
called him.

``I see I'm intruding,'' Rostov repeated.

The look of annoyance had already disappeared from Boris' face:
having evidently reflected and decided how to act, he very
quietly took both Rostov's hands and led him into the next
room. His eyes, looking serenely and steadily at Rostov, seemed
to be veiled by something, as if screened by blue spectacles of
conventionality. So it seemed to Rostov.

``Oh, come now! As if you could come at a wrong time!'' said
Boris, and he led him into the room where the supper table was
laid and introduced him to his guests, explaining that he was not
a civilian, but an hussar officer, and an old friend of his.

``Count Zhilinski---le Comte N. N.---le Capitaine S. S.,'' said
he, naming his guests. Rostov looked frowningly at the Frenchmen,
bowed reluctantly, and remained silent.

Zhilinski evidently did not receive this new Russian person very
willingly into his circle and did not speak to Rostov. Boris did
not appear to notice the constraint the newcomer produced and,
with the same pleasant composure and the same veiled look in his
eyes with which he had met Rostov, tried to enliven the
conversation. One of the Frenchmen, with the politeness
characteristic of his countrymen, addressed the obstinately
taciturn Rostov, saying that the latter had probably come to
Tilsit to see the Emperor.

``No, I came on business,'' replied Rostov, briefly.

Rostov had been out of humor from the moment he noticed the look
of dissatisfaction on Boris' face, and as always happens to those
in a bad humor, it seemed to him that everyone regarded him with
aversion and that he was in everybody's way. He really was in
their way, for he alone took no part in the conversation which
again became general. The looks the visitors cast on him seemed
to say: ``And what is he sitting here for?'' He rose and went up
to Boris.

``Anyhow, I'm in your way,'' he said in a low tone. ``Come and
talk over my business and I'll go away.''

``Oh, no, not at all,'' said Boris. ``But if you are tired, come
and lie down in my room and have a rest.''

``Yes, really...''

They went into the little room where Boris slept. Rostov, without
sitting down, began at once, irritably (as if Boris were to blame
in some way) telling him about Denisov's affair, asking him
whether, through his general, he could and would intercede with
the Emperor on Denisov's behalf and get Denisov's petition handed
in. When he and Boris were alone, Rostov felt for the first time
that he could not look Boris in the face without a sense of
awkwardness. Boris, with one leg crossed over the other and
stroking his left hand with the slender fingers of his right,
listened to Rostov as a general listens to the report of a
subordinate, now looking aside and now gazing straight into
Rostov's eyes with the same veiled look. Each time this happened
Rostov felt uncomfortable and cast down his eyes.

``I have heard of such cases and know that His Majesty is very
severe in such affairs. I think it would be best not to bring it
before the Emperor, but to apply to the commander of the
corps... But in general, I think...''

``So you don't want to do anything? Well then, say so!'' Rostov
almost shouted, not looking Boris in the face.

Boris smiled.

``On the contrary, I will do what I can. Only I thought...''

At that moment Zhilinski's voice was heard calling Boris.

``Well then, go, go, go...'' said Rostov, and refusing supper and
remaining alone in the little room, he walked up and down for a
long time, hearing the lighthearted French conversation from the
next room.

% % % % % % % % % % % % % % % % % % % % % % % % % % % % % % % % %
% % % % % % % % % % % % % % % % % % % % % % % % % % % % % % % % %
% % % % % % % % % % % % % % % % % % % % % % % % % % % % % % % % %
% % % % % % % % % % % % % % % % % % % % % % % % % % % % % % % % %
% % % % % % % % % % % % % % % % % % % % % % % % % % % % % % % % %
% % % % % % % % % % % % % % % % % % % % % % % % % % % % % % % % %
% % % % % % % % % % % % % % % % % % % % % % % % % % % % % % % % %
% % % % % % % % % % % % % % % % % % % % % % % % % % % % % % % % %
% % % % % % % % % % % % % % % % % % % % % % % % % % % % % % % % %
% % % % % % % % % % % % % % % % % % % % % % % % % % % % % % % % %
% % % % % % % % % % % % % % % % % % % % % % % % % % % % % % % % %
% % % % % % % % % % % % % % % % % % % % % % % % % % % % % %

\chapter*{Chapter XX}
\ifaudio     
\marginpar{
\href{http://ia600200.us.archive.org/6/items/war_and_peace_05_0805_librivox/war_and_peace_05_20_tolstoy_64kb.mp3}{Audio}} 
\fi

\lettrine[lines=2, loversize=0.3, lraise=0]{\initfamily R}{ostov}
had come to Tilsit the day least suitable for a petition
on Denisov's behalf. He could not himself go to the general in
attendance as he was in mufti and had come to Tilsit without
permission to do so, and Boris, even had he wished to, could not
have done so on the following day. On that day, June 27, the
preliminaries of peace were signed. The Emperors exchanged
decorations: Alexander received the Cross of the Legion of Honor
and Napoleon the Order of St. Andrew of the First Degree, and a
dinner had been arranged for the evening, given by a battalion of
the French Guards to the Preobrazhensk battalion. The Emperors
were to be present at that banquet.

Rostov felt so ill at ease and uncomfortable with Boris that,
when the latter looked in after supper, he pretended to be
asleep, and early next morning went away, avoiding Boris. In his
civilian clothes and a round hat, he wandered about the town,
staring at the French and their uniforms and at the streets and
houses where the Russian and French Emperors were staying. In a
square he saw tables being set up and preparations made for the
dinner; he saw the Russian and French colors draped from side to
side of the streets, with huge monograms A and N. In the windows
of the houses also flags and bunting were displayed.

``Boris doesn't want to help me and I don't want to ask
him. That's settled,'' thought Nicholas. ``All is over between
us, but I won't leave here without having done all I can for
Denisov and certainly not without getting his letter to the
Emperor. The Emperor!... He is here!'' thought Rostov, who had
unconsciously returned to the house where Alexander lodged.

Saddled horses were standing before the house and the suite were
assembling, evidently preparing for the Emperor to come out.

``I may see him at any moment,'' thought Rostov. ``If only I were
to hand the letter direct to him and tell him all... could they
really arrest me for my civilian clothes? Surely not! He would
understand on whose side justice lies. He understands everything,
knows everything. Who can be more just, more magnanimous than he?
And even if they did arrest me for being here, what would it
matter?'' thought he, looking at an officer who was entering the
house the Emperor occupied. ``After all, people do go in... It's
all nonsense! I'll go in and hand the letter to the Emperor
myself so much the worse for Drubetskoy who drives me to it!''
And suddenly with a determination he himself did not expect,
Rostov felt for the letter in his pocket and went straight to the
house.

``No, I won't miss my opportunity now, as I did after
Austerlitz,'' he thought, expecting every moment to meet the
monarch, and conscious of the blood that rushed to his heart at
the thought. ``I will fall at his feet and beseech him. He will
lift me up, will listen, and will even thank me. 'I am happy when
I can do good, but to remedy injustice is the greatest
happiness,'{}'' Rostov fancied the sovereign saying. And passing
people who looked after him with curiosity, he entered the porch
of the Emperor's house.

A broad staircase led straight up from the entry, and to the
right he saw a closed door. Below, under the staircase, was a
door leading to the lower floor.

``Whom do you want?'' someone inquired.

``To hand in a letter, a petition, to His Majesty,'' said
Nicholas, with a tremor in his voice.

``A petition? This way, to the officer on duty'' (he was shown
the door leading downstairs), ``only it won't be accepted.''

On hearing this indifferent voice, Rostov grew frightened at what
he was doing; the thought of meeting the Emperor at any moment
was so fascinating and consequently so alarming that he was ready
to run away, but the official who had questioned him opened the
door, and Rostov entered.

A short stout man of about thirty, in white breeches and high
boots and a batiste shirt that he had evidently only just put on,
standing in that room, and his valet was buttoning on to the back
of his breeches a new pair of handsome silk-embroidered braces
that, for some reason, attracted Rostov's attention. This man was
speaking to someone in the adjoining room.

``A good figure and in her first bloom,'' he was saying, but on
seeing Rostov, he stopped short and frowned.

``What is it? A petition?''

``What is it?'' asked the person in the other room.

``Another petitioner,'' answered the man with the braces.

``Tell him to come later. He'll be coming out directly, we must
go.''

``Later... later! Tomorrow. It's too late...''

Rostov turned and was about to go, but the man in the braces
stopped him.

``Whom have you come from? Who are you?''

``I come from Major Denisov,'' answered Rostov.

``Are you an officer?''

``Lieutenant Count Rostov.''

``What audacity! Hand it in through your commander. And go along
with you... go,'' and he continued to put on the uniform the
valet handed him.

Rostov went back into the hall and noticed that in the porch
there were many officers and generals in full parade uniform,
whom he had to pass.

Cursing his temerity, his heart sinking at the thought of finding
himself at any moment face to face with the Emperor and being put
to shame and arrested in his presence, fully alive now to the
impropriety of his conduct and repenting of it, Rostov, with
downcast eyes, was making his way out of the house through the
brilliant suite when a familiar voice called him and a hand
detained him.

``What are you doing here, sir, in civilian dress?'' asked a deep
voice.

It was a cavalry general who had obtained the Emperor's special
favor during this campaign, and who had formerly commanded the
division in which Rostov was serving.

Rostov, in dismay, began justifying himself, but seeing the
kindly, jocular face of the general, he took him aside and in an
excited voice told him the whole affair, asking him to intercede
for Denisov, whom the general knew. Having heard Rostov to the
end, the general shook his head gravely.

``I'm sorry, sorry for that fine fellow. Give me the letter.''

Hardly had Rostov handed him the letter and finished explaining
Denisov's case, when hasty steps and the jingling of spurs were
heard on the stairs, and the general, leaving him, went to the
porch. The gentlemen of the Emperor's suite ran down the stairs
and went to their horses. Hayne, the same groom who had been at
Austerlitz, led up the Emperor's horse, and the faint creak of a
footstep Rostov knew at once was heard on the stairs. Forgetting
the danger of being recognized, Rostov went close to the porch,
together with some inquisitive civilians, and again, after two
years, saw those features he adored: that same face and same look
and step, and the same union of majesty and mildness... And the
feeling of enthusiasm and love for his sovereign rose again in
Rostov's soul in all its old force. In the uniform of the
Preobrazhensk regiment---white chamois-leather breeches and high
boots---and wearing a star Rostov did not know (it was that of
the Legion d'honneur), the monarch came out into the porch,
putting on his gloves and carrying his hat under his arm. He
stopped and looked about him, brightening everything around by
his glance. He spoke a few words to some of the generals, and,
recognizing the former commander of Rostov's division, smiled and
beckoned to him.

All the suite drew back and Rostov saw the general talking for
some time to the Emperor.

The Emperor said a few words to him and took a step toward his
horse.  Again the crowd of members of the suite and street gazers
(among whom was Rostov) moved nearer to the Emperor. Stopping
beside his horse, with his hand on the saddle, the Emperor turned
to the cavalry general and said in a loud voice, evidently
wishing to be heard by all:

``I cannot do it, General. I cannot, because the law is stronger
than I,'' and he raised his foot to the stirrup.

The general bowed his head respectfully, and the monarch mounted
and rode down the street at a gallop. Beside himself with
enthusiasm, Rostov ran after him with the crowd.

% % % % % % % % % % % % % % % % % % % % % % % % % % % % % % % % %
% % % % % % % % % % % % % % % % % % % % % % % % % % % % % % % % %
% % % % % % % % % % % % % % % % % % % % % % % % % % % % % % % % %
% % % % % % % % % % % % % % % % % % % % % % % % % % % % % % % % %
% % % % % % % % % % % % % % % % % % % % % % % % % % % % % % % % %
% % % % % % % % % % % % % % % % % % % % % % % % % % % % % % % % %
% % % % % % % % % % % % % % % % % % % % % % % % % % % % % % % % %
% % % % % % % % % % % % % % % % % % % % % % % % % % % % % % % % %
% % % % % % % % % % % % % % % % % % % % % % % % % % % % % % % % %
% % % % % % % % % % % % % % % % % % % % % % % % % % % % % % % % %
% % % % % % % % % % % % % % % % % % % % % % % % % % % % % % % % %
% % % % % % % % % % % % % % % % % % % % % % % % % % % % % %

\chapter*{Chapter XXI}
\ifaudio     
\marginpar{
\href{http://ia600200.us.archive.org/6/items/war_and_peace_05_0805_librivox/war_and_peace_05_21_tolstoy_64kb.mp3}{Audio}} 
\fi

\lettrine[lines=2, loversize=0.3, lraise=0]{\initfamily T}{he}
Emperor rode to the square where, facing one another, a
battalion of the Preobrazhensk regiment stood on the right and a
battalion of the French Guards in their bearskin caps on the
left.

As the Tsar rode up to one flank of the battalions, which
presented arms, another group of horsemen galloped up to the
opposite flank, and at the head of them Rostov recognized
Napoleon. It could be no one else.  He came at a gallop, wearing
a small hat, a blue uniform open over a white vest, and the
St. Andrew ribbon over his shoulder. He was riding a very fine
thoroughbred gray Arab horse with a crimson gold-embroidered
saddlecloth. On approaching Alexander he raised his hat, and as
he did so, Rostov, with his cavalryman's eye, could not help
noticing that Napoleon did not sit well or firmly in the
saddle. The battalions shouted ``Hurrah!'' and ``Vive
l'Empereur!'' Napoleon said something to Alexander, and both
Emperors dismounted and took each other's hands.  Napoleon's face
wore an unpleasant and artificial smile. Alexander was saying
something affable to him.

In spite of the trampling of the French gendarmes' horses, which
were pushing back the crowd, Rostov kept his eyes on every
movement of Alexander and Bonaparte. It struck him as a surprise
that Alexander treated Bonaparte as an equal and that the latter
was quite at ease with the Tsar, as if such relations with an
Emperor were an everyday matter to him.

Alexander and Napoleon, with the long train of their suites,
approached the right flank of the Preobrazhensk battalion and
came straight up to the crowd standing there. The crowd
unexpectedly found itself so close to the Emperors that Rostov,
standing in the front row, was afraid he might be recognized.

``Sire, I ask your permission to present the Legion of Honor to
the bravest of your soldiers,'' said a sharp, precise voice,
articulating every letter.

This was said by the undersized Napoleon, looking up straight
into Alexander's eyes. Alexander listened attentively to what was
said to him and, bending his head, smiled pleasantly.

``To him who has borne himself most bravely in this last war,''
added Napoleon, accentuating each syllable, as with a composure
and assurance exasperating to Rostov, he ran his eyes over the
Russian ranks drawn up before him, who all presented arms with
their eyes fixed on their Emperor.

``Will Your Majesty allow me to consult the colonel?'' said
Alexander and took a few hasty steps toward Prince Kozlovski, the
commander of the battalion.

Bonaparte meanwhile began taking the glove off his small white
hand, tore it in doing so, and threw it away. An aide-de-camp
behind him rushed forward and picked it up.

``To whom shall it be given?'' the Emperor Alexander asked
Koslovski, in Russian in a low voice.

``To whomever Your Majesty commands.''

The Emperor knit his brows with dissatisfaction and, glancing
back, remarked:

``But we must give him an answer.''

Kozlovski scanned the ranks resolutely and included Rostov in his
scrutiny.

``Can it be me?'' thought Rostov.

``Lazarev!'' the colonel called, with a frown, and Lazarev, the
first soldier in the rank, stepped briskly forward.

``Where are you off to? Stop here!'' voices whispered to Lazarev
who did not know where to go. Lazarev stopped, casting a sidelong
look at his colonel in alarm. His face twitched, as often happens
to soldiers called before the ranks.

Napoleon slightly turned his head, and put his plump little hand
out behind him as if to take something. The members of his suite,
guessing at once what he wanted, moved about and whispered as
they passed something from one to another, and a page---the same
one Rostov had seen the previous evening at Boris'---ran forward
and, bowing respectfully over the outstretched hand and not
keeping it waiting a moment, laid in it an Order on a red
ribbon. Napoleon, without looking, pressed two fingers together
and the badge was between them. Then he approached Lazarev (who
rolled his eyes and persistently gazed at his own monarch),
looked round at the Emperor Alexander to imply that what he was
now doing was done for the sake of his ally, and the small white
hand holding the Order touched one of Lazarev's buttons. It was
as if Napoleon knew that it was only necessary for his hand to
deign to touch that soldier's breast for the soldier to be
forever happy, rewarded, and distinguished from everyone else in
the world. Napoleon merely laid the cross on Lazarev's breast
and, dropping his hand, turned toward Alexander as though sure
that the cross would adhere there. And it really did.

Officious hands, Russian and French, immediately seized the cross
and fastened it to the uniform. Lazarev glanced morosely at the
little man with white hands who was doing something to him and,
still standing motionless presenting arms, looked again straight
into Alexander's eyes, as if asking whether he should stand
there, or go away, or do something else. But receiving no orders,
he remained for some time in that rigid position.

The Emperors remounted and rode away. The Preobrazhensk
battalion, breaking rank, mingled with the French Guards and sat
down at the tables prepared for them.

Lazarev sat in the place of honor. Russian and French officers
embraced him, congratulated him, and pressed his hands. Crowds of
officers and civilians drew near merely to see him. A rumble of
Russian and French voices and laughter filled the air round the
tables in the square. Two officers with flushed faces, looking
cheerful and happy, passed by Rostov.

``What d'you think of the treat? All on silver plate,'' one of
them was saying. ``Have you seen Lazarev?''

``I have.''

``Tomorrow, I hear, the Preobrazhenskis will give them a
dinner.''

``Yes, but what luck for Lazarev! Twelve hundred francs' pension
for life.''

``Here's a cap, lads!'' shouted a Preobrazhensk soldier, donning
a shaggy French cap.

``It's a fine thing! First-rate!''

``Have you heard the password?'' asked one Guards' officer of
another.  ``The day before yesterday it was 'Napoleon, France,
bravoure'; yesterday, 'Alexandre, Russie, grandeur.' One day our
Emperor gives it and next day Napoleon. Tomorrow our Emperor will
send a St. George's Cross to the bravest of the French Guards. It
has to be done. He must respond in kind.''

Boris, too, with his friend Zhilinski, came to see the
Preobrazhensk banquet. On his way back, he noticed Rostov
standing by the corner of a house.

``Rostov! How d'you do? We missed one another,'' he said, and
could not refrain from asking what was the matter, so strangely
dismal and troubled was Rostov's face.

``Nothing, nothing,'' replied Rostov.

``You'll call round?''

``Yes, I will.''

Rostov stood at that corner for a long time, watching the feast
from a distance. In his mind, a painful process was going on
which he could not bring to a conclusion. Terrible doubts rose in
his soul. Now he remembered Denisov with his changed expression,
his submission, and the whole hospital, with arms and legs torn
off and its dirt and disease. So vividly did he recall that
hospital stench of dead flesh that he looked round to see where
the smell came from. Next he thought of that self-satisfied
Bonaparte, with his small white hand, who was now an Emperor,
liked and respected by Alexander. Then why those severed arms and
legs and those dead men?... Then again he thought of Lazarev
rewarded and Denisov punished and unpardoned. He caught himself
harboring such strange thoughts that he was frightened.

The smell of the food the Preobrazhenskis were eating and a sense
of hunger recalled him from these reflections; he had to get
something to eat before going away. He went to a hotel he had
noticed that morning.  There he found so many people, among them
officers who, like himself, had come in civilian clothes, that he
had difficulty in getting a dinner. Two officers of his own
division joined him. The conversation naturally turned on the
peace. The officers, his comrades, like most of the army, were
dissatisfied with the peace concluded after the battle of
Friedland. They said that had we held out a little longer
Napoleon would have been done for, as his troops had neither
provisions nor ammunition.  Nicholas ate and drank (chiefly the
latter) in silence. He finished a couple of bottles of wine by
himself. The process in his mind went on tormenting him without
reaching a conclusion. He feared to give way to his thoughts, yet
could not get rid of them. Suddenly, on one of the officers'
saying that it was humiliating to look at the French, Rostov
began shouting with uncalled-for wrath, and therefore much to the
surprise of the officers:

``How can you judge what's best?'' he cried, the blood suddenly
rushing to his face. ``How can you judge the Emperor's actions?
What right have we to argue? We cannot comprehend either the
Emperor's aims or his actions!''

``But I never said a word about the Emperor!'' said the officer,
justifying himself, and unable to understand Rostov's outburst,
except on the supposition that he was drunk.

But Rostov did not listen to him.

``We are not diplomatic officials, we are soldiers and nothing
more,'' he went on. ``If we are ordered to die, we must die. If
we're punished, it means that we have deserved it, it's not for
us to judge. If the Emperor pleases to recognize Bonaparte as
Emperor and to conclude an alliance with him, it means that that
is the right thing to do. If once we begin judging and arguing
about everything, nothing sacred will be left! That way we shall
be saying there is no God---nothing!'' shouted Nicholas, banging
the table---very little to the point as it seemed to his
listeners, but quite relevantly to the course of his own
thoughts.

``Our business is to do our duty, to fight and not to think!
That's all...'' said he.

``And to drink,'' said one of the officers, not wishing to
quarrel.

``Yes, and to drink,'' assented Nicholas. ``Hullo there! Another
bottle!''  he shouted.

In 1808 the Emperor Alexander went to Erfurt for a fresh
interview with the Emperor Napoleon, and in the upper circles of
Petersburg there was much talk of the grandeur of this important
meeting.

% % % % % % % % % % % % % % % % % % % % % % % % % % % % % % % % %
% % % % % % % % % % % % % % % % % % % % % % % % % % % % % % % % %
% % % % % % % % % % % % % % % % % % % % % % % % % % % % % % % % %
% % % % % % % % % % % % % % % % % % % % % % % % % % % % % % % % %
% % % % % % % % % % % % % % % % % % % % % % % % % % % % % % % % %
% % % % % % % % % % % % % % % % % % % % % % % % % % % % % % % % %
% % % % % % % % % % % % % % % % % % % % % % % % % % % % % % % % %
% % % % % % % % % % % % % % % % % % % % % % % % % % % % % % % % %
% % % % % % % % % % % % % % % % % % % % % % % % % % % % % % % % %
% % % % % % % % % % % % % % % % % % % % % % % % % % % % % % % % %
% % % % % % % % % % % % % % % % % % % % % % % % % % % % % % % % %
% % % % % % % % % % % % % % % % % % % % % % % % % % % % % %

\chapter*{Chapter XXII}
\ifaudio
\marginpar{
\href{http://ia600200.us.archive.org/6/items/war_and_peace_05_0805_librivox/war_and_peace_05_22_tolstoy_64kb.mp3}{Audio}} 
\fi

\lettrine[lines=2, loversize=0.3, lraise=0]{\initfamily I}{n}
1809 the intimacy between \emph{the world's two arbiters}, as
Napoleon and Alexander were called, was such that when Napoleon
declared war on Austria a Russian corps crossed the frontier to
co-operate with our old enemy Bonaparte against our old ally the
Emperor of Austria, and in court circles the possibility of
marriage between Napoleon and one of Alexander's sisters was
spoken of. But besides considerations of foreign policy, the
attention of Russian society was at that time keenly directed on
the internal changes that were being undertaken in all the
departments of government.

Life meanwhile---real life, with its essential interests of
health and sickness, toil and rest, and its intellectual
interests in thought, science, poetry, music, love, friendship,
hatred, and passions---went on as usual, independently of and
apart from political friendship or enmity with Napoleon Bonaparte
and from all the schemes of reconstruction.